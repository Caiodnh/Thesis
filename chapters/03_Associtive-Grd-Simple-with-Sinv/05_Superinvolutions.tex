
\section{Superinvolutions and sesquilinear forms}

Our goal now is to specialize the results of Section \ref{sec:super-anti-auto-and-sesquilinear} to the case where $\vphi$ is a superinvolution.
To this end, let us investigate what super-anti-automorphism of $\D$ and what sesquilinear form on $\U$ determine the super-anti-automorphism $\vphi\inv$.
Again, we suppose $\D$ is a graded division superalgebra, $\U$ is a nonzero right graded module of finite rank over $\D$ and put $R = \End_\D (\U)$.

\begin{defi}\label{def:barB}
	Given a super-anti-automorphism $\vphi_0$ on $\D$ and a $\vphi_0$-sesqui\-li\-near form $B$ on $\U$,  we define $\overline {B}\from \U\times \U \to \D$ by $\overline {B} (u,v) \coloneqq \sign{u}{v} \vphi_0\inv (B(v, u))$ for all $u, v \in \U$.
\end{defi}

% \begin{lemma}
%     Let $B\from \U\times \U\Star \to \D$ be a $\vphi_0$-sesquilinear form. Suppose there is a super-anti-automorphism $\vphi$ on $R = \End_\D(U)$ such that Equation \eqref{eq:superadjunction} holds. If $B$ is nonzero, then it is nondegenerate.
% \end{lemma}

\begin{prop}\label{prop:barB-determines-vphi-inv}
	Under the conditions of Definition \ref{def:barB}, we have that $\overline {B}$ is a $\vphi_0\inv$-sesquilinear form of the same degree and parity as $B$.
	Further, if $B$ is nondegenerate and $\vphi$ is the super-anti-automorphism on $R$ determined by $(\vphi_0, B)$ as in Theorem \ref{thm:vphi-iff-vphi0-and-B}, then $\overline{B}$ is nondegenerate and $\vphi\inv$ is determined by $(\vphi_0\inv, \overline B)$, \ie,
	%
	\begin{equation}\label{eq:barB-superadjunction}
		\forall r\in R\even \cup R\odd ,\,\forall u, v \in \U\even \cup \U\odd,  \quad \overline {B}(ru,v) = \sign{r}{u} \overline {B}(u,\vphi\inv (r)v).
	\end{equation}
	%
\end{prop}

\begin{proof}
	Since $B$ is $\FF$-bilinear, so is $\overline {B}$.
	Also, since $\vphi_0$ preserves degree and parity, $\overline {B}$ is homogeneous of the same degree and parity as $B$.
	Let us check the conditions of Definition \ref{def:sesquilinear-form} and Equation \eqref{eq:barB-superadjunction}.
	\vspace{2mm}
	\begin{align*}
		\intertext{ Condition \eqref{enum:linear-on-the-second}:}
		\overline {B} (u,vd) & = (-1)^{|u|( |v| + |d|)} \vphi_0\inv (B(vd, u))                                                       \\
		                     & = (-1)^{|u|( |v| + |d|)} (-1)^{|d| (|B| + |v|)} \vphi_0\inv \big( \vphi_0(d) B(v, u) \big)            \\
		                     & = (-1)^{|u||v| + |u||d| + |d||B| + |d||v|} (-1)^{|d| (|B| + |v| + |u|) }  \vphi_0\inv \big(B(v, u)) d \\ &= \sign{u}{v} \vphi_0\inv \big(B(v, u)) d = \overline {B}(u, v) d .
		\intertext{Condition \eqref{enum:vphi0-linear-on-the-first}:}
		\overline {B}(ud, v) & = (-1)^{(|u| + |d|) |v|} \vphi_0\inv \big( B(v, ud) \big)                                             \\ &= (-1)^{(|u| + |d|) |v|} \vphi_0\inv \big( B(v, u)d \big) \\ &= (-1)^{(|u| + |d|) |v|} (-1)^{|d| (|B| + |v| + |u|)} \vphi_0\inv (d) \vphi_0\inv\big( B(v, u) \big) \\ &= (-1)^{|u||v| + |d| |B| + |d||u|} \vphi_0\inv (d) \vphi_0\inv\big( B(v, u) \big) \\ &= (-1)^{(|B| + |u|) |d|} \vphi_0\inv (d) \overline {B}(u, v).
		\intertext{For Equation \eqref{eq:barB-superadjunction}, note that replacing $r$ for $\vphi\inv(r)$, Equation \eqref{eq:superadjunction} can be rewritten as}
		B(v, ru)             & = \sign{r}{v} B(\vphi\inv(r)v,u).
		\intertext{Hence, we have that}
		\overline {B}(ru, v) & = (-1)^{(|r| + |u|) |v|} \vphi_0\inv \big( B(v, ru) \big)                                             \\ &= (-1)^{(|r| + |u|) |v|} (-1)^{|r||v|} \vphi_0\inv \big( B(\vphi\inv (r)v, u) \big) \\ &= (-1)^{|u||v|} \vphi_0\inv \big( B(\vphi\inv (r)v, u) \big) \\ &= (-1)^{|u||v|} (-1)^{(|r| + |v|) |u|} \overline {B}(u, \vphi\inv (r)v) \\ &= \sign{r}{u} \overline {B}(u, \vphi\inv (r)v).
	\end{align*}

	Finally, Equation \eqref{eq:barB-superadjunction} together with $B$ being nondegenerate implies that $\overline{B}$ is nondegenerate.
	To see that, let $u$ be a nonzero homogeneous element in $\rad \overline{B}$.
	Then for for every $r\in R\even \cup R\odd$ and $v\in \U\even \cup \U\odd$, we have that $\overline{B}(u, \vphi\inv (r) v) = 0$, hence $\overline{B}(ru, v) = 0$.
	Since $r \in R\even \cup R\odd$ and $v\in \U\even \cup \U\odd$ were arbitrary, this implies $\overline{B} (Ru, \U) = 0$.
	But $\U$ is simple as a graded $R$-supermodule, so we would have $\overline{B}(\U, \U) = 0$ and then, using that $\vphi_0$ is bijective, $B (\U, \U) = 0$, a contradiction.
\end{proof}

\begin{lemma}\label{lemma:bar-dB}
	Under the conditions of Definition \ref{def:barB}, let $d$ be a nonzero $G^\#$-homogeneous element of $\D$ and consider $\vphi_0' \coloneqq \operatorname{sInt}_d\circ\, \vphi_0$ and $B' \coloneqq d B$.
	Then $\overline {B'} = (-1)^{|d|} \vphi_0\inv (d) \overline B$.
\end{lemma}

\begin{proof}
	Note that $(\vphi_0')\inv = \vphi_0\inv \circ \operatorname{sInt}_{d}\inv = \vphi_0\inv \circ \operatorname{sInt}_{d\inv}$.
	Hence, for all $u, v \in \U\even \cup \U\odd$,
	%
	\begin{align*}
		\overline {B'} (u,v) & = \sign{u}{v} (\vphi_0\inv \circ \operatorname{sInt}_{d\inv})  (d B(v, u) )                    \\
		                     & = \sign{u}{v} \vphi_0\inv \big( (-1)^{|d| (|d| + |B| + |u| + |v|)}\,  d\inv d B (v, u) d \big) \\ &= \sign{u}{v} (-1)^{|d|}\, \vphi_0\inv (d) \vphi_0\inv(B(v, u)) =  (-1)^{|d|}\,\vphi_0\inv (d) \overline {B} (u, v).
	\end{align*}
\end{proof}

We are primarily interested in the case $\FF$ is an algebraically closed field and $\D$ is finite dimensional.
In this case, we have that $\D\even_e = \FF 1$, so we are under the hypothesis of the following theorem, which is a graded version of \cite[Theorem 7]{racine}:

\begin{thm}\label{thm:vphi-involution-iff-delta-pm-1}
	% Suppose $\FF$ is an  algebraically closed field. 
	Let $\D$ be a graded division superalgebra such that $\D\even_e = \FF 1$, let $\U$ be a nonzero right graded module of finite rank over $\D$ and let $\vphi$ be a degree-preserving super-anti-automorphism on $R \coloneqq \End_\D (\U)$.
	Consider a super-anti-automorphism $\vphi_0$ on $\D$ and a nondegenerate $\vphi_0$-sesqui\-li\-near form $B$ on $\U$ determining $\vphi$ as in Theorem \ref{thm:vphi-iff-vphi0-and-B}.
	Then $\vphi$ is a superinvolution if, and only if, $\vphi_0$ is a superinvolution and $\overline B = \pm B$.
\end{thm}

\begin{proof}
	Using Proposition \ref{prop:barB-determines-vphi-inv} and Theorem \ref{thm:vphi-iff-vphi0-and-B}, we conclude that $\vphi = \vphi\inv$ if, and only if, there is $\delta \in \D$ such that (i) $\vphi_0\inv = \operatorname{sInt}_\delta \circ \,\vphi_0$ and (ii) $\overline {B} = \delta B$.
	% If $\vphi_0^2 =\id_\D$ and $B = \pm \overline{B}$, we have conditions (i) and (ii) sati
	% To prove the ``if'' part of the statement, we take $\delta \in \{\pm 1\}$, since then (i) becomes $\vphi_0 = \vphi_0\inv$ and (ii), $B = \pm \overline{B}$.

	If $\vphi_0^2 =\id_\D$ and $\overline{B} = \delta B$ with $\delta \in \{\pm 1\}$, conditions (i) and (ii) are satisfied, proving the ``if'' part of the statement.

	For the ``only if'' part, since both parity and degree of $B$ and $\overline {B}$ coincide, (ii) implies $\delta \in \D\even_e$.
	We are assuming $\D\even_e = \FF 1$, so $\operatorname{sInt}_\delta = \id_\D$ and (i) becomes $\vphi_0 = \vphi_0\inv$, \ie, $\vphi_0$ is a superinvolution.
	Also from $\delta \in \FF 1$, $\vphi_0\inv(\delta) = \delta$ and, hence, using Lemma \ref{lemma:bar-dB}, we have that
	\[
		B = \overline {\overline B} = \overline {\delta B} = (-1)^{|\delta|} \vphi_0\inv(\delta) \overline B= \delta \overline B = \delta^2 B.
	\]
	This implies $\delta^2 = 1$.
	Therefore $\delta \in \{ \pm 1 \}$, so $\overline B = \pm B$, concluding the proof.
\end{proof}

% it is clear that $B(u,vd) = B(u,v)d$. By Equation \eqref{eq:sesquilinear-before-B}, the map $B: \U \times \U \to \D$ is a \emph{$\vphi_0$-sesquilinear form}, \ie,
% \[
%     B(ud, v) = (-1)^ {(|B| + |u|)|d|}
% \]

% Consider the map $\vphi_0: \D \to \D$ given by 

% Consider the map $\vphi_0(d) := \sign{\vphi_1}{d}\vphi_1\inv d \vphi_1$ (where juxtaposition denotes composition of maps on the right).
% Then
% \begin{equation}\label{eq:sesquilinear-before-B}
%     (ud)\vphi_1 = (u) (d\vphi_1) =  (u)(\vphi_1\vphi_1\inv d \vphi_1) = \sign{d}{\vphi_1}(u)\vphi_1 \,\vphi_0(d).
% \end{equation}
% It is straightforward to check that $\vphi_0: \D = \End_R(\mc U) \to \End_R(\mc U\Star) = \D\sop$ is an isomorphism of superalgebras.
% We will consider $\vphi_0$ as a super-anti-automorphism on $\D$.

%Recall (Subsection \ref{ssec:superlinear-maps}) that $\vphi_1^\circ$ is an $R$-superlinear map on the left. Using it, the Equation \eqref{eq:sesquilinear-before-B} becomes
% \begin{equation}\label{eq:sesquilinear-with-vphi1-circ}
%     \vphi_1^\circ (ud) = \vphi_1^\circ (u) \vphi_0 (d)
% \end{equation}

%For each $u\in \U$, $\vphi_1(u)$ is an $\D$-linear map on the left. We, then, can define \[B(u,v) = \sign{\vphi_1}{u} ((u)\vphi_1)(v),\]
%or, using the notation introduced in Subsection \ref{ssec:superlinear-maps},
% \[
%     B(u,v) = \vphi_1^\circ (u) (v),
% \]
% for all $u,v \in U$.

%We will consider $\vphi_0$ as a super-anti-automorphism on $\D$.

%Recall (Subsection \ref{ssec:superlinear-maps}) that $\vphi_1^\circ$ is an $R$-superlinear map on the left. Using it, the Equation \eqref{eq:sesquilinear-before-B} becomes

%Since  this new action makes $\mc U\Star$ a simple graded left $R$-supermodule. Also, since we had $\D\sop = \End_{R\sop} (\U\Star)$, we c it follows that $\D\sop$ can be identif
%We, then, have that $\mc U\Star$ is a simple graded left $R$-module and that $\D\sop = \End_R(\mc U\Star)$.

%Since $R$ acts on the left, we will use the convention of writing $R$-superlinear maps on the right. % (see Subsection \ref{ssec:superlinear-maps}).
%By Proposition {\tt ??}, $R$ has only one graded simple supermodule up to isomorphism and shift, \ie, there is an invertible $R$-linear map $\vphi_1: \mc U \to \mc U\Star$ which is homogeneous of some degree $(g_0, \alpha)\in G^\#$.

% Using the notation just introduced, Equation \eqref{eq:sesquilinear-before-B} becomes
% \begin{equation}%\label{eq:sesquilinear-with-vphi1-circ}
%     \vphi_1^\circ (ud) = \vphi_1^\circ (u) \tilde\vphi_0 (d)
% \end{equation}
% or, puting $\vphi_0(d) = (\tilde \vphi_0(d))^\circ$,
% \begin{equation}\label{eq:sesquilinear-with-vphi1-circ}
%     \vphi_1^\circ (ud) = (-1)^{|d|(|\vphi_1|+|u|)}\vphi_0 (d)\vphi_1^\circ (u).
% \end{equation}

% It is important to note that the map $\vphi_1$ is not uniquely determined. For every nonzero homogeneous element $d\in\D$, the map $d\vphi_1: \mc U \to \mc U\Star$ is also an invertible, homogeneous and $R$-superlinear. If we consider $\vphi_1$ fixed, every such map can be obtained this way: if $\vphi_1' : \mc U \to \mc U\Star$ is an invertible homogeneous $R$-superlinear map, then we could take $d := \vphi'\vphi\inv \in \End_R(\mc U) = \D$ and, hence, $\vphi_1' = d \vphi_1$.

% Changing $\vphi_1$ to $\vphi_1'$ also changes the super-anti-automorphism $\vphi_0 (c) = \sign{\vphi_1}{d}\vphi_1\inv d \vphi_1$ to $\vphi_0':= \sign{\delta\vphi_1}{d}\vphi_1\inv\delta\inv d \delta\vphi_1= \sign{\delta}{d} \vphi_0(\delta\inv d \delta)$. In other words, $\vphi_0' = \vphi_0 \circ \operatorname{sInt}_\delta$, where $\operatorname{sInt}_\delta: $

% The choice of the map $\vphi_1$ 

%If $\vphi_1' : \mc U \to \mc U\Star$ is another invertible and homogeneous $R$-superlinear map, then $\vphi'\vphi\inv: \mc U \to \mc U$ is a nonzero homogeneous element of $\D = \End_R(\mc U)$.
%Conversely, for each nonzero homogeneous element $\delta\in\D$, the map $\vphi_1' := \delta\vphi_1: \mc U \to \mc U\Star$ is $R$-superlinear, invertible and homogeneous.
% Changing $\vphi_1$ to $\vphi_1'$ also changes the super-anti-automorphism $\vphi_0(d) = \sign{\vphi_1}{d}\vphi_1\inv d \vphi_1$ to $\vphi_0':= $

% depends on the choice of $\vphi_1$: if we consider $\delta\vphi_1$ instead, then we change $\vphi_0(d) = \vphi_1\inv d \vphi_1$ to $\vphi_0'(d) = \vphi_0(\delta\inv d \delta)$.
