% Intro for A(n,n) odd 
We can also reparametrize the odd graded superalgebras with superinvolution of type $M\times M\sop$ in terms of $G$. 
Let $M^{\mathrm{ex}}(T, \beta, t_p, \kappa, g_0)$ be as in \cref{def:model-grd-MxM-odd-or-QxQ}, and let $K \subseteq T^+$, $t_1 \in T^-$ and $\mc M$ be chosen as in items \eqref{}, \eqref{} and \eqref{} of \cref{def:std-realization-MxM-QxQ}(b). 

% --------

Let $K \subseteq T^+$ be the complement to $\rad \beta^+$ chosen in item \eqref{} of \cref{def:std-realization-MxM-QxQ}(b), and let $t_1 (\rad \beta^+)$ be the coset of 

Let $K \subseteq T^+$ and $t_1 \in T^-$ be chosen as in items \eqref{} and \eqref{} of \cref{def:std-realization-MxM-QxQ}(b). 

% ---------

As done in \cref{ssec:in-terms-of-G}, let $h \in G$ be the projection of $t_1$ onto $G$, \ie, $t_1 = (h, \bar 1)$, and define $\chi \in \widehat{T^+}$ by $\chi(t) \coloneqq \beta(t_1, t)$, for all $t\in T^+$. 
By \cref{eq:beta-from-h-chi}, we can recover the triple $(T, \beta, p)$ from $T^+$, $\beta^+$, $h$ and $\chi$. 

Recall that we have $e \neq f = h^2$, $t_p \neq f$, and $\rad \beta^+ = \langle t_p, f \rangle$, and note that, since $T^+ = K \times (\rad \beta^+)$ and $\beta(t_1, K) = 1$, $\chi$ is determined by 
\[\label{eq:equivalent-def-of-chi}
    \chi(K) = \chi(f) = 1 \,\AND\, \chi(t_p) = -1.
\]
Therefore, we can recover $(T, \tilde\beta, t_p)$ from $T^+$, $\beta^+$, $t_p$, $K$ and $h$.  
The choice of a standard realization of a graded-division algebra $\mc M$ associated to $(K, \beta\restriction_{K \times K})$ (item \eqref{} in Def.) is already in terms of the group $G$, hence we can encode the data necessary to construct a standard realization of a graded-division superalgebra with superinvolution with no reference to $G^\#$. 
In particular, we can fix the map $\eta^+\from T^+ \to \pmone$, 
which, by rmk, is characterized by
\[\label{eq:eta+-unsharpening-MxM}
    \eta^+\restriction_{K} \,\text{is the map associated to transposition on}\, \mc M,\, \eta^+(f) = -1 \,\AND\, \eta^+(t_p) =1. 
\]
and, hence, the condition that $\kappa$ is $g_0$-admissible is well-defined. 
It follows that, and $M^{\mathrm{ex}}(T, \beta, t_p, \kappa, g_0)$ can be constructed from $T^+$, $\beta^+$, $t_p$, $K$, $h$, $\mc M$, $g_0$ and $\kappa$. 

% ----

\begin{remark}
    By \cref{lemma:K-before-t_1}, given the choice of $K$, there are four possible choices for $t_1$. 
    More precisely, we could have chosen another element $t_1' \in t_1 (\rad \beta^+)$ instead of $t_1$. 
    This would imply that we would have an element $h' \in h (\rad \beta^+)$ instead of $h$. 
    Clearly, this would not change the character $\chi$, and we would still recover $(T, \tilde\beta, t_p)$ and the choices i and iii. 
    By \cref{eq:eta+-unsharpening-MxM}, we would get the same $\eta^+$, so the the condition that a map $\kappa$ is $g_0$-admissible is the same. 
    Nevertheless, the map $\eta$ could be different and, as noted in the 3rd rmk, the parametrization of $M^{ex}$ could be different. 
\end{remark}

% ---------

Conversely, let $T^+ \subseteq G$ be a $2$-elementary subgroup, let $e\neq t_p \in T^+$, let $h\in G$ such that $f \coloneqq h^2 \in T^+ \setminus \langle t_p \rangle$, and let $\beta^+\from T^+ \times T^+ \to \pmone$ be an alternating bicharacter such that $\rad \beta^+ = \langle t_p, f \rangle$. 
Choose a complement $K \subseteq T^+$ to $\rad \beta^+$ and let $\chi \in \widehat{T^+}$ be defined by \cref{eq:equivalent-def-of-chi}. 
It is straightforward to see that the pair $(h, \chi) \in \mathbf{O} (T^+, \beta^+)$, see \cref{def:O(T+-beta+)}. 
Set $t_1 \coloneqq (h, \bar 1)$, $T^- \coloneqq t_1 T^+$, $T \coloneqq T^+ \cup T^-$, $\beta\from T\times T\to \pmone$ as in \cref{lemma:existence-beta}. 
Clearly, if we change $h$ by another element in $h (\rad \beta^+)$, we would get the same element $f$ and the same bicharacter $\beta$. 
Therefore, instead of $h$, we will consider as a parameter the coset $h(\rad \beta^+)$
% such that $h^2 \in T^+ \setminus \langle t_p \rangle$ as a parameter
, and fix $h$ to be a representative of this coset. 

It follows that $t_p$ is a parity element and $f\in \rad \beta$, hence, by \cref{cor:radical-with-parity}, that $\rad \beta = \rad \tilde\beta = \langle f \rangle$. 
Therefore, $(T, \tilde\beta, t_p)$ is as in \cref{def:std-realization-MxM-QxQ}(b). 
Further, since $\beta(t_1, K) = \chi(K) = 1$, we can use $K$ and $t_1$ as the choices \eqref{item:choose-t_1-std-realization} and \eqref{item:K-can-be-orthogonal-to-t_1} in the mentioned definition. 
For \eqref{item:choose-mc-M}, we still choose any standard realization $\mc M$ of a matrix algebra with a division grading associated to $(K, \beta\restriction_{K\times K})$. 
We fix $\eta^+\from T^+ \to \pmone$ to be the unique map such that $\mathrm{d}\eta^+ = \beta^+$,  $\eta^+\restriction_{K}$ is the map associated to the transposition on $\mc M$ and $\eta^+\restriction_{\rad \beta^+}$ is the character defined by $\eta^+(t_p) = 1$ and $\eta^+(f) = -1$. 
It is straightforward that, if $(\D, \vphi_0)$ is the standard realization as in \cref{def:std-realization-MxM-QxQ}(b) corresponding to these choices above and $\eta\from T \to \pmone$ is the map determining $\vphi_0$, then $\eta^+ = \eta\restriction_{T^+}$. 



Note that, since $T^+$ is $2$-elementary, the element $f \coloneqq h^2$ depends only on the coset $hT^+$. 
Hence, we can (and will) fix the complement $K$ and the character $\chi$ depending only on $T^+$, $t_p$ and $hT^+$. 

\begin{remark}
    Let $\D$ and $\D'$ be, respectively, odd graded-division superalgebras associated to $(T^+, \beta^+, h,\chi)$ and $(T'^+, \beta'^+, h',\chi')$ as above. 
    By \cref{thm:iso-odd-D-only-G}, if $\D \iso \D'$, then $h'T^+ = hT^+$, but we may have $h' \neq h$. 
    Therefore, it is convenient for us to make $\chi$ depend only on the coset $hT^+$ and not the element $h$. 
\end{remark}

% GIVEN instead of LET:
%
% Conversely, given a $2$-elementary subgroup $T^+ \subseteq G$, nontrivial distinct elements $t_p, f \in T^+$, and an alternating bicharacter $\beta^+\from T^+ \times T^+ \to \pmone$ such that $\rad \beta^+ = \langle t_p, h^2 \rangle$, we fix a complement $K \subseteq T^+$ to $\rad \beta^+$ and a character $\chi \in \widehat{T^+}$ defined by \cref{eq:equivalent-def-of-chi}. 
% Given a element $h \in G$ such that $h^2 = f$, it is straightforward to see that the pair $(h, \chi)$ is $(T^+, \beta^+)$-admissible (\cref{def:O(T+-beta+)}). 


SHOULD THE FOLLOWING BE A LABELED DEFINITION?

\begin{defi}
    Let $T^+$, $\beta^+$, $t_p$ and $h$ be as above, let $g_0 \in G$ and let $\kappa\from G/T^+ \to \ZZ_{\geq}$ be a $g_0$-admissible map. 
    We define $M^{\mathrm{ex}}(T^+, \beta^+, t_p, f, h, \kappa, g_0)$ to be the graded superalgebra with superinvolution $M^{\mathrm{ex}}(T, \beta, t_p, \kappa, g_0)$. 
\end{defi}

\begin{cor}\label{cor:MxMsop-odd-only-G}
    Suppose the superalgebra with superinvolution $M(n,n) \times M(n,n)\sop$ is endowed with a $G$-grading making it graded-simple. 
    Then it is isomorphic, as a graded superalgebra with superinvolution, to $M^{\mathrm{ex}}(T^+, \beta^+, t_p, f, h, \kappa, g_0)$ as above. 
    Moreover, the graded superalgebras with superinvolution $M^{\mathrm{ex}}(T^+, \beta^+, t_p, f, h, \kappa, g_0)$ and $M^{\mathrm{ex}} (T'^+, \beta'^+, t_p',  h', \kappa', g_0')$ are isomorphic if, and only if, $T^+ =T'^+$, $\beta^+ = \beta'^+$, $t_p = t_p'$, $h' \in h \langle t_p, f \rangle$ and there is $g \in G$ such that $\kappa' = g\cdot\kappa$ and $g_0' = g^{-2}g_0$. 
\end{cor}

\begin{proof}
    By \cref{thm:iso-odd-D-only-G,prop:T-beta-determines-iso}, $T =T'$ and $\beta = \beta'$ if, and only if, $(h', \chi')$ is in the $T^+$-orbit of $(h, \chi)$ in $\mathbf {O} (T^+, \beta^+)$ (see \cref{def:T^+-action}). Since we have $\chi = \chi'$ if $h T^+ = h'T^+$, this condition reduces to $h' \in h(\rad \beta^+) = h\langle f \rangle$. 
    Then, the result follows from \cref{thm:MxM-odd}. 
\end{proof}

% Since the the assertion that $\kappa$ is $g_0$-admissible depends only on the fixed map $\eta^+\from T^+ \to \pmone$, we have described all the information necessary to construct $M^{\mathrm{ex}}(T, \beta, t_p, \kappa, g_0)$ with no reference to $G^\#$. 

% ---------------------------
%
% \begin{defi}
%     Let $T^+ \subseteq G$ be a $2$-elementary subgroup, let $e\neq t_p \in T^+$, let $h\in G$ such that $f \coloneqq h^2 \in T^+ \setminus \langle t_p \rangle$, and let $\beta^+\from T^+ \times T^+ \to \pmone$ be an alternating bicharacter such that $\rad \beta^+ = \langle t_p, h^2 \rangle$. 
%     Choose:
%     \begin{enumerate}[(i)]
%         \item a subgroup $K \subseteq T^+$ such that $(\rad \beta^+) \times K = T^+$;
%         \item a standard realization $\mc M$ \cref{def:standard-realization}) of a matrix algebra with a division grading  associated to $(K, \beta\restriction_{K\times K})$. 
%         \label{item:choose-mc-M}
%     \end{enumerate}
% \end{defi}

% % ----------------------
