%%%% If you used '\documentclass[12pt]{pdfathesisi}', then 
%%%% the 'prefatory' enviromment is defined as well as some 
%%%% useful macros to be used whitin the environment.
%%%% Otherwise you are by your own regarding the  formating 
%%%% of your thesis.
%%
%% prefatory pages: front page, abstract, etc.
%
\begin{prefatory}
\hbadness=10000
\frontpage
%%
%%%%%%%%%%%%%%%%%%%%%%%%%%%%%%%%%%%%%%%%%%%%%%%
%%   Abstract is mandatory
%%   It should not exceed 300 words
\abstract% 
This is the abstract. It should not exceed $300$ words.
%%
%%%%%%%%%%%%%%%%%%%%%%%%%%%%%%%%%%%%%%%%%%%%%%
%%   Dedication page is optional
%%   Remove it if you do not want to have it
\dedication%
{\flushright
To whom you what to dedicate this work
}
%%%%%%%%%%%%%%%%%%%%%%%%%%%%%%%%%%%%%%%%%%
%%
%%
\laysummary %
Lay summary is a short account of research targeted at a general
audience, i.e., its goal is to communicate the research results to a
non-specialist readership. It should fit in one page. 

A lay summary provides a brief description of the research in a less
technical language. It should include 1) background of the study, 2)
rationale and methods, and 3) outcomes. 

The following guidelines are recommended. 

a) Write it in every day English, using shot and clear sentences. 

b) The text should be ordered logically and flow naturally, i.e.,
ideas and concepts should introduced as needed. 

c) Everything here should be expressed in active
voice. Preferentially, in first person. 

d) Positive phrasing. 

e) Aims, objectives and results should be clearly stated. 

f) Whenever possible, provide every day examples. 

g) Use the appropriate tone. This summary's goal is to inform not to
entertain.

%%
%%
%%
%%%%%%%%%%%%%%%%%%%%%%%%%%%%%%%%%%%%%%%%%%%%%
%%   Acknowledgements
%%   It can be in singular on in plural
%\acknowledgement%
% or
\acknowledgements%
Among other acknowledgements, the student should declare the
extent to which assistance (paid or unpaid) has been given by faculty
and staff members, fellow students, research assistants, technicians,
or others in the preparation of the thesis (including editorial help).

\noindent
In addition, it is appropriate to recognize the supervision and advice
given by the thesis supervisor(s), supervisory committee members and
other advisors.
%%
%%%%%%%%%%%%%%%%%%%%%%%%%%%%%%%%%%%%%%%%%%%%%%
%%   Statement of contributions
%%   It can be in singular on in plural
%\contributions%
\contribution% 

In this thesis we present a classification of group gradings on the classical Lie superalgebras (series $A, B, C, D, P \AND Q$) over an algebraically closed field $\FF$ of characteristic zero (see \cref{chap:Lie}). 
To this end, we also classify group gradings on the finite dimensional superinvolution-simple associative superalgebras, over an algebraically closed field of characteristic different from $2$ (see \cref{chap:grds-sinv-simple}). 

\noindent
In previous works (together with Helen Samara Dos Santos and Mikhail Kochetov in \cite{paper-MAP}, and together with both and Yuri Bahturin in \cite{paper-Qn}), we already had a complete classification of group gradings on the Lie superalgebras of series $P \AND Q$ and a partial classification of group gradings on Lie superalgebras of series $A$, and in \cite{Helens_thesis} there is a classification of group gradings on the Lie superalgebras of series $B$. 
Nevertheless, the techniques developed in this work allowed us to classify the gradings on Lie superalgebras of all series  $A, B, C, D, P \AND Q$ in a uniform fashion. 
% The correspondence between the different classification is explained. 

\noindent 
For some important results (\cref{thm:vphi-iff-vphi0-and-B,thm:vphi-involution-iff-delta-pm-1,cor:SxSsop-with-dcc}), no assumptions on the base field are needed, and the conditions of finite dimensionality and (superinvolution-)simplicity of the superalgebra is weakened to the descending chain condition on graded one-side superideals and graded-(superinvolution-)simplicity. 

\noindent
Also, some known results are presented in greater generality than found in the literature. 
In \cref{sec:g-hat-action,sec:Omega-algebras}, we develop the correspondence between $G$-gradings and $\widehat G$-actions (see, for example, \cite[Section 1.4]{livromicha}) for universal algebras (assuming the base field is algebraically closed of characteristic zero and $G$ is a finitely generated abelian group). 
In \cref{sec:classification-grd-simple-with-sinv} we present a classification of finite dimensional graded-simple superalgebras associative over an algebraically closed field in terms of the group $G$, without assuming they are simple as algebras as in \cite{paper-MAP}. 



%%%%%%%%%%%%%%%%%%%%%%%%%%%%%%%%%%%%%%%%%%%%%%
\tableofcontents%
% \listoftables    % if apply, otherwise comment
% \listoffigures   % if apply, otherwise comment
%%
%%%%%%%%%%%%%%%%%%%%%%%%%%%%%%%%%%%%%%%%%%%%%%
%%   List of symbos is optional
%%   It is like a two column tabular environment
%
% \begin{symbols}
% $\dnor{\mu}{\sigma^2}$ & %
%     normal distribution with mean $\mu$ and variance $\sigma^2$\\
% $\real{}$ & real numbers\\
% \end{symbols}
%%%%%%%%%%%%%%%%%%%%%%%%%%%%%%%%%%%%%%%%%%%%%%
%%   List of abbreviations is optional
%%   Also, it is like a two column tabular environment
%
% \begin{abbreviations}
% ISO & International Organization for Standardization\\
% PDF & Portable Document Format\\
% PDF/A & ISO-standardized version of PDF specialized for digital archiving.
% \end{abbreviations}
\end{prefatory}
%%%% End of 'prefatory' section
%
%%%%%%%%%%%%%%%%%%%%%%%%%%%%%%%%%%%%%%%%%%%%%
%%
%%  '\doublespacing' should only be used for review/draft
%\doublespacing  %% comment this command for the final submission
%%
%%
%%%%%%%%