\chapter{Super-anti-automorphisms on Graded-simple Associative Superalgebras}\label{chap:super-inv}

% ** Intro of the Chapter

The orthosymplectic and the paraplectic Lie superalgebras are defined using a nondegenerate bilinenar form on a superspace.
More specifically, if $\vphi$ is the superadjunction corresponding to this form, then we consider the subsuperalgebra of the general linear Lie superalgebra consisting of the skew elements with respect to $\vphi$ (see Subsection {\tt ??}).

\vspace{5mm}
{\tt (More intro to come...)}

{\tt Mention that we are developing the approach of Elduque (J. of \\Al\-ge\-bra 2010, see also Chapter 2 in Monograph) in the super setting.}

\begin{itemize}
	\item Cite Bahturin and Tvaladze; maybe Tvaladze and Tvaladze;
	\item Cite Bahturin and Shestakov;
	\item Also Carlos G\'omes Ambrose ``On the Lie Structure of the Skew Elements of a simple Superalgebra with Superinvolution''
	\item And Racine's paper on superinvolution simple superalgebras.
	\item Maybe Elduques's \cite{elduqueSuperinv};
	\item How what we are doing is different?
	\item Analogous to Lie over $\RR$ (Adri\'an); not literally using results, but using ideas.
\end{itemize}

The next result has is a generalization of \cite[Proposition 2.49]{livromicha} and the same proof works.

\begin{prop}\label{prop:grd-simple-vphi-abelian}
    Let $R$ be a associative $G$-graded superalgebra such that the support generates $G$. 
    If $R$ is graded simple and admits a degree-preserving super-anti-automorphism, then $G$ is abelian. \qed
\end{prop}

% For the remainder of the chapter, $G$ is a fixed abelian group.

% In this chapter we are going to describe 

% ** Sections:

\section{The superdual}\label{ssec:superdual}

Let $\D$ be a graded-division superalgebra. 
We start with the definition of the superdual for $G$-graded $\D$-supermodules or, equivalently, the dual for $G^\#$-graded $\D$-modules (see \cite[Definition 2.56]{livromicha}). 

\begin{defi}\label{def:superdual-supermodule}
    Let $\U$ be a graded right $\D$-supermodule of finite rank. 
    The \emph{superdual of $\U$} is defined to be $\U\Star \coloneqq \Hom_\D (\U,\D)$, with its usual $G^\#$-grading (see \cref{prop:Hom_D-is-graded}). 
    We give $\U\Star$ the structure of a \emph{left} graded $\D$-module: if $d \in \D$ and $f \in \U \Star$, then we define $(df)(u) = d\, f(u)$ for all $u\in \mc U$. 
\end{defi}

While the prefix ``super'' is unimportant in \cref{def:superdual-supermodule}, it does make a difference in the definition of the superdual of a $\D$-linear map:

\begin{defi}\label{defi:superdual-map}
    Let $\U$ and $\V$ be graded right $\D$-modules of finite rank. 
    Given a homogeneous $\D$-linear
    $L:\U \rightarrow \V$, we define the \emph{superdual of $L$} to be the $\FF$-linear map $L\Star\from \V\Star \rightarrow \U\Star$ defined by
    \[
        L\Star (f) = (-1)^{|L||f|} f \circ L,
    \] 
    for every homogeneous $f\in \V\Star$. 
    It is easy to see that $L\Star$ is (left) $\D$-linear. 
    We extend the definition of superdual to every map in $\Hom_\D (\U, \V)$ by linearity.
\end{defi}

% Note that, since $\U\Star$ is a left $\D$-module, we have that $\U^{\star\star} \coloneqq \Hom_{\D} (\U^\Star, \D)$ is again a right $\D$-module. 
% As we have with finite dimensional vector spaces, we can identify $\U^{\star\star}$ with $\U$. 

% \begin{lemma}\label{lemma:double-dual}
%     The $\D$-linear map $\epsilon\from \U \to \U^{\star\star}$ determined by $\epsilon(u)(f) = \sign{u}{f} f(u)$
% \end{lemma}

% \begin{defi}
%     Let $\U$ be a graded right $\D$-module of finite rank. 
%     The \emph{double dual} of $\U$ is the graded right $\D$-module $\U^{\star\star} \coloneqq \Hom_{\D\sop} (\U^\Star, \D\sop)$. 
% \end{defi}


We have used right supermodules over a graded-division superalgebra to classify graded-simple superalgebras (see \cref{ssec:D-modules,ssec:supermodules-over-D}). 
Hence, it is convenient to see the superdual of a right supermodule again as right supermodule over a suitable graded-division superalgebra:

\begin{defi}\label{def:superopposite}
    Let $R$ be a $G$-graded superalgebra. 
    We define the \emph{superopposite $G$-graded superalgebra $R\sop$} to be $R$ as graded superspace but with a new product given by $r * s \coloneqq \sign{r}{s} sr$, for every homogeneous elements $r,s \in R$.
\end{defi}

Note that $R\sop$ being a $G$-graded superalgebra depends on the assumption that $G$ is abelian. 

\begin{remark}\label{rmk:sop-super-anti-iso}
    If $\vphi:R \to S$ is a super-anti-isomorphism between the $G$-graded superalgebras $R$ and $S$, then the map $\vphi$ can be viewed as an isomorphism $R\to S\sop$ or $R\sop \to S$.
\end{remark}

It is easy to see that $\D\sop$ is also a graded-division superalgebra. 
For every graded right $\U$ as in \cref{def:superdual-supermodule}, the left $\D$-module $\U\Star$ can be considered as a right $\D\sop$-module by means of the action defined by $f\cdot d := (-1)^{|d||f|} df$, for every homogeneous elements $d\in \D$ and $f\in \U\Star$. 
Also, for every $L\from \U \to \V$ as in \cref{defi:superdual-map}, $L\Star$ is a (right) $\D\sop$-linear map. 

\begin{prop}\label{prop:dual-super-anti-iso}
    Let $\U$ be a nonzero graded right $\D$-supermodule of finite rank. 
    The map $\End_\D (\U) \rightarrow \End_{\D\sop} (\U\Star)$ defined by $L \mapsto L\Star$ is a degree-preserving super-anti-isomorphism.
\end{prop}

\begin{proof}
    Clearly the map $L \mapsto L\Star$ is linear and degree preserving (again, using the assumption that $G$ is abelian). 
    It is easy to verify that $(L \circ M)\Star = \sign{L}{M} M\Star \circ L\Star$. 
    Finally, the inverse of the map $L \mapsto L\Star$ is $M \mapsto \epsilon \circ M\Star \circ \epsilon\inv$ where $\epsilon\from \U \to \U^{\star\star}$ is given by $\epsilon(u)(f) = \sign{u}{f} f(u)$.
\end{proof}

\begin{defi}
    Let $\B = \{u_1, \ldots, u_k\}$ is a graded basis, we can consider its \emph{superdual basis} $\mc B\Star = \{u_1\Star, \ldots, u_k\Star\}$ in $\U\Star$, where $u_i\Star : \U \rightarrow \D$ is defined by $u_i\Star (u_j) = (-1)^{|u_i||u_j|} \delta_{ij}$. 
    Clearly, $\deg u_i\Star = (\deg u_i)\inv$ and $\mc B\Star$ is, indeed, a $\D\sop$-basis for $\U\Star$.
\end{defi}

\begin{remark}
	In the case $\D=\FF$, if we denote by $[L]$ the matrix of $L$ with respect to the graded bases $\mc B$ of $\U$ and $\mc C$ of $\V$, then the supertranspose $[L]\sT$ is the matrix corresponding to $L\Star$ with respect to the superdual bases $\mc C\Star$ and $\mc B\Star$.
\end{remark}

Since $G$ is abelian, $G^\#/T$ is a group and, hence, the map $G^\#/T \to G^\#/T$ given by $x \mapsto x\inv$ is well-defined. 
From the construction of the superdual basis, it is easy to see that $\dim_\D \U_x = \dim_{\D\sop} \U\Star_{x\inv}$, for all $x \in G^\#/T$. 
Hence, if $\kappa\from G^\#/T \to \ZZ_{\geq 0}$ is the map associated to $\U$ as a $G^\#$-graded right $\D$-module (see \cref{ssec:D-modules}), then $\kappa\Star \from G^\#/T \to \ZZ_{\geq 0}$ defined by $\kappa\Star (x) = \kappa (x\inv)$ is the map associated to $\U\Star$ as a $G^\#$-graded right $\D\sop$-module.

It is straightforward to translate this to the maps $G/T \to \ZZ_{\geq 0}$ (even $\D$) and $G/T^+ \to \ZZ_{\geq 0}$ (odd $\D$) associated to the $G$-graded supermodule $\U$, as in \cref{ssec:supermodules-over-D}. 
If $\D$ is even and $\kappa_\bz, \kappa_\bo$ are the maps associated to $\U$, then $\kappa_\bz\Star, \kappa_\bo\Star$ are the maps associated to $\U\Star$. 
If $\D$ is odd and $\kappa$ is the map associated to $\U$, then $\kappa\Star$ is the map associated to $\U\Star$.

Finally, let us assume that $\FF$ is algebraically closed and $\D$ is finite dimensional. 
Then, if $(T, \beta, p)$ is the triple associated to $\D$, then it is clear that $\supp \D\sop = T$ and that the parity map for $\D\sop$ is also $p$. 
Moreover, if $s,t \in T$, $0 \neq X_s \in \D_s$ and $0 \neq X_t \in \D_t$, then, following the notation in \cref{def:superopposite}, we have:
\[
    X_s * X_t = \sign{s}{t} X_tX_s = \sign{s}{t} \beta(t, s) X_sX_t = \beta(t, s) X_t*X_s = \beta(s, t)\inv X_t*X_s.
\]
We conclude that $(T,\beta\inv, p)$ is the triple associated to $\D\sop$.

\section[Super-anti-automorphisms and sesquilinear forms]{Super-anti-automorphisms and \texorpdfstring{\\}{} sesquilinear forms}\label{sec:super-anti-auto-and-sesquilinear}

Let $\D$ be a graded-division superalgebra and let $\U$ be a nonzero graded right $\D$-supermodule of finite rank. 
Consider the graded superalgebra $R := \End_\D(\mc U)$.
Then $\mc U$ is an $(R, \D)$-superbimodule.
By \cref{lemma:converse-density-thm}, 
% $\mc U$ is a simple graded left supermodule over $R$ and 
we have a natural identification between $\D$ and $\End_R(\mc U)$.

% \begin{notation}
%     Let  be . As sets, we have that $S = S\sop$, but given $s\in S$, we will it by $ \bar s$ when regarded as an element of $S\sop$.
% \end{notation}

As we saw on \cref{ssec:superdual}, $\mc U\Star = \Hom_\D (\mc U, \D)$ is a graded right $\D\sop$-supermodule, through $f \bar d = \sign{f}{d} df$.
Also, we can make $\mc U\Star$ a graded right $R$-supermodule by defining $(f r) (u) := f(r u)$ for all $f\in \mc U\Star$, $r\in R$ and $u\in \mc U$ (i.e., $f r = f \circ r$, since $R = \End_\D(\mc U)$).
Hence, we can consider $\mc U\Star$ as a graded left $R\sop$-supermodule via $\bar r f = (-1)^{|r||f|} f r$, % (i.e., $r\cdot f = r\Star (f)$)
and $\mc U\Star$ becomes a $(R\sop, \D\sop)$-superbimodule.

\begin{lemma}\label{lemma:U-star-R-sop}
	The left $R\sop$-supermodule $\mc U\Star$ is graded-simple, and the action of $\D\sop$ gives an isomorphism $\D\sop \to \End_{R\sop}(\mc U\Star)$.
\end{lemma}

\begin{proof}
	By definition of superadjoint operator, the above $R\sop$-action corresponds to the representation $\rho\from R\sop \to \End_{\D\sop} (\mc U\Star)$ given by $ \rho(r) = r\Star$, which is an isomorphism since $\U$ is of finite rank.
	If we use $\rho$ to identify $R\sop$ with $\End_{\D\sop} (\mc U\Star)$, the result follows from \cref{lemma:converse-density-thm}.
\end{proof}

%Note that the $R\sop$-action correspond to the representation $\rho: R\sop \to \End_{\D\sop} (\mc U\Star)$ given by $ \rho(r) = r\Star$, which is an isomorphism. If we use it to identify $R\sop$ with $\End_{\D\sop} (\mc U\Star)$, we can then apply Proposition \ref{lemma:converse-density-thm} again and conclude that $\mc U\Star$ is a graded simple left $R\sop$-module and that we naturally can identify $\D\sop$ with $\End_{R\sop}(\mc U\Star)$.

Now let $\vphi: R\to R$ be a super-anti-automorphism that preserves the $G$-degree (\ie, is homogeneous of degree $e$ with respect to the $G$-grading).
We can see it as an isomorphism $R\to R\sop$, $r \mapsto \overline{ \vphi(r)}$, and use it to identify $R$ with $R\sop$.
In particular, we will make the left $R\sop$-action on $\U\Star$ into a left $R$-action via $r\cdot f = \overline{\vphi(r)}f$, for all $r\in R$ and all $f\in \U\Star$.
In other words,
%
\begin{equation}\label{eq:R-action-back-on-the-right}
	r\cdot f = \sign{r}{f} f \circ \vphi(r) .
\end{equation}

We will now consider $\U\Star$ as an $(R,\D\sop)$-superbimodule.
In particular, the superalgebra $\End_R(\U\Star)$ should be understood as the set of $R$-linear maps with respect to the $R$-action on the left (which is the same set as $\End_{R\sop} (\U\Star)$).
Also, since the action is on the left, we will follow the convention of writing $R$-linear maps on the right.

From Lemma \ref{lemma:U-star-R-sop}, it follows that $\U\Star$ is a simple graded left $R$-supermodule and that we can identify $\D\sop$ with $\End_R (\U\Star) = \End_{R\sop} (\U\Star)$.
But from \cite[Lemma 2.7]{livromicha}, $R$ has only one graded-simple supermodule up to isomorphism and shift, hence there is an invertible $R$-linear map $\vphi_1: \mc U \to \mc U\Star$ which is homogeneous of some degree $(g_0, \alpha)\in G^\#$.
Fix one such $\vphi_1$.
% This map is not unique, but we are going to fix one for now. 

\begin{lemma}\label{lemma:nonuniqueness-of-vphi1}
	A map $\vphi_1' : \U \to \U\Star$ is $R$-linear and homogeneous with respect the $G^\#$-grading if, and only if, there is an element $\bar d\in \D\sop = \End_R (\U\Star)$ homogeneous with the respect the $G^\#$-grading such that $ \vphi_1' = \vphi_1 \bar d$, where juxtaposition represents composition of maps written on the right.
\end{lemma}

\begin{proof}
	This follows from the fact that $ \vphi_1\inv \vphi' \in  \D\sop = \End_R(\U\Star)$ if, and only if, $\vphi' \in \Hom_R(\U, \U\Star)$.
\end{proof}

% \begin{proof}
%     Let $\vphi_1'$ be an homogeneous $R$-linear map. Clearly, $\vphi_1\inv \vphi_1' \in \End_R(\U\Star)$ $ = \D\sop$ (note that we are composing functions written on the right). Then $\vphi_1' = \vphi_1 d$, for some homogeneous $d\in \D\sop$, and hence, by the definitions of the right $\D\sop$-action and the left $\D$-action, $\vphi' (u) = \sign{\vphi_1}{d} d\vphi(u)$ for all $u\in \U$. The converse is a direct computation.
% \end{proof}

We will use
the $R$-linear map
$\vphi_1$ to construct a nondegenerate sesquilinear form $B$ on $\U$.

\begin{defi}\label{def:sesquilinear-form}
	We say that a map $B\colon \U \times \U \to \D$ is a \emph{sesquilinear form on $\U$} if it is $\FF$-bilinear, $G^\#$-homogeneous if considered as a linear map $\U\tensor \U \to \D$, and there is $\vphi_0\from \D \to \D$ a degree-preserving su\-per\--an\-ti\--auto\-mor\-phism such that, for all $u,v \in \U$ and $d\in \D$,
	%
	\begin{enumerate}[(i)]
		\item $B(u,vd) = B(u,v)d$; \label{enum:linear-on-the-second}
		\item $B(ud, v) = (-1)^ {(|B| + |u|)|d|}\vphi_0(d) B(u, v)$. \label{enum:vphi0-linear-on-the-first}
	\end{enumerate}
	If we want to specify the super-anti-automorphism $\vphi_0$, we will say that $B$ is \emph{sesquilinear with respect to $\vphi_0$} or that $B$ is \emph{$\vphi_0$-sesquilinear}.
	The \emph{(left) radical} of $B$ is the set $\rad B \coloneqq \{u\in \U \mid B(u, v) = 0 \text{ for all } v\in \U\}$. 
	We say that the form $B$ is \emph{nondegenerate} if $\rad B = 0$.
\end{defi}

% \begin{remark}\label{rmk:B-determines-vphi_0}
%     Note that if $B$ is nondegenerate and $\U \neq 0$, then $\vphi_0$ is actually determined by $B$.
% \end{remark}

\begin{lemma}\label{lemma:B-determines-vphi_0}
	Let $B \neq 0$ be a sesquilinear form on $\U$.
	Then there is a unique super-anti-automorphism $\vphi_0$ on $\D$ such that $B$ is sesquilinear with respect to $\vphi_0$.
\end{lemma}

\begin{proof}
	Since $B\neq 0$, there are homogeneous elements $u, v\in \U$ such that $B(u,v) \neq 0$ and, hence, $B(u,v)$ is an invertible element of $\D$.
	Suppose $B$ is sesquilinear with respect to super-anti-automorphisms $\vphi_0$ and $\vphi_0'$ on $\D$.
	Then, for all $d\in \D\even \cup \D\odd$, we have
	\[ B(ud,v) = (-1)^ {(|B| + |u|)|d|} \vphi_0(d) B(u,v) = (-1)^ {(|B| + |u|)|d|} \vphi_0'(d) B(u,v) \]
	and, therefore, $\vphi_0(d) = \vphi_0'(d)$.
\end{proof}

Suppose for now that a degree-preserving su\-per\--an\-ti\--auto\-mor\-phism $\vphi_0: \D \to \D$ is given.
We can use it to define a right $\D$-action on $\U\Star$ by interpreting it as an isomorphism from $\D$ to $\D\sop$ and putting
%
\begin{equation}\label{eq:right-D-action}
	f\cdot d \coloneqq f \overline{\vphi_0(d)},
\end{equation}
%
for all $f \in \U\Star$ and $d\in \D$.
Using this action, we have $\End_\D (\U\Star) = \End_{\D\sop} (\U\Star)$. We also have the following:

\begin{prop}\label{prop:sesquilinear-form-iff-D-linear-map}
	Let $\vphi_0\from \D \to \D$ be a a degree-preserving su\-per\--an\-ti\--auto\-mor\-phism and consider the right $\D$-supermodule structure on $\U\Star$ given by Equation \eqref{eq:right-D-action}.
	Then the $\vphi_0$-sesquilinear forms $B:\U\times\U \to \D$ are in a one-to-one correspondence with the homogeneous $\D$-linear maps $\theta\from \U \to \U\Star$ via $B \mapsto \theta$ where $\theta(u) \coloneqq B(u, \cdot)$, for all $u \in \U$ or, inversely, $\theta \mapsto B$ where $B(u,v) \coloneqq \theta(u)(v)$, for all $u, v\in \U$.
	Moreover, $B$ is nondegenerate if, and only if, $\theta$ is an isomorphism.
\end{prop}

\begin{proof}
	Suppose $B$ is given. Condition \eqref{enum:linear-on-the-second} of Definition \ref{def:sesquilinear-form} tells us that $\theta$ defined this way is, indeed, a map from $\U$ to $\U\Star$.
	Also, it is easy to check that $\theta$ is homogeneous of the same parity and degree as $B$.

	Recalling the left $\D$-action on $\U\Star$, condition \eqref{enum:vphi0-linear-on-the-first} tells us that, for all $u\in \U$ and $d\in \D$,
	\[
		\theta (ud) = (-1)^{|d|(|\theta|+|u|)}\vphi_0 (d) \theta (u),
	\]
	which, by the definition of the right $\D\sop$-action on $\U\Star$, is equivalent to $\theta(ud) = \theta(u) \overline{\vphi_0(d)}$, \ie, $\theta$ is, indeed, $\D$-linear considering the left $\D$-action on $\U\Star$ given by Equation \eqref{eq:right-D-action}.

	To show that the correspondence is bijective, note that all the considerations above can be reversed when, given $\theta\from \U \to \U\Star$, we define  $B(u, v) \coloneqq \theta(u)(v)$.

	The ``moreover'' part follows from the fact that $\rad B = \ker \theta$, so the nondegeneracy of $B$ is equivalent to $\theta$ being injective. But $\U$ and $\U\Star$ have the same (finite) rank over $\D$, so $\theta$ is injective if, and only if, it is bijective.
\end{proof}

% \begin{lemma}\label{lemma:sesquilinear-form-iff-D-linear-map}
%     Consider $\U\Star$ as a left $\D$-supermodule by using $\vphi_0$ as above. Then there is a bijection between the $\D$-linear maps $\theta\from \U \to \U\Star$ and the $\vphi_0$-sesquilinear forms $B:\U\times\U \to \D$ via $\theta \mapsto B$ where $B(u,v) \coloneqq \theta(u)(v)$. Moreover, $\theta$ is an isomorphism if, and only if, $B$ is nondegenerate.
% \end{lemma}

% Condition \eqref{enum:linear-on-the-second} on the Definition above allows us to use $B$ to define a map $\U \to \U\Star$ by $u \mapsto B(u, \cdot)$. If this is invertible, we say that the form $B$ is \emph{nondegenerate}.

% Note that a degree-preserving super-anti-automorphism $\vphi_0: \D \to \D$ can be seen as an isomorphism between $\D$ and $\D\sop$ and, hence, one can use it to define a right $\D$-action on $\U\Star$ by $v$

Coming back to our map $\vphi_1\from \U \to \U\Star$ and using the identifications $\D = \End_R(\U)$ and $\D\sop = \End_R(\U\Star)$ introduced above, consider the map $\D \to \D\sop$ sending $d \mapsto \sign{d}{\vphi_1}\vphi_1\inv d \,\vphi_1$, where juxtaposition denotes composition of maps on the right.
It is straightforward to check that this map is an isomorphism and, hence, we can consider it as a super-anti-automorphism $\vphi_0 \from \D \to \D$.
Then, for all $u\in \U$ and $d\in \D$, we have
%
\begin{equation}\label{eq:sesquilinear-before-B}
	(ud)\vphi_1 = u (d\vphi_1) =  u(\vphi_1\vphi_1\inv d \vphi_1) = \sign{d}{\vphi_1}(u\,\vphi_1 )\vphi_0(d).
\end{equation}

% The map $\vphi_0$ depends on the choice o f $\vphi_1$. Following Lemma \ref{lemma:all-possible-vphi1}, if we had started with $\vphi_1' \coloneqq d \vphi_1$ instead, we would have gotten the map $\vphi_0' \coloneqq \vphi_0 \circ \operatorname{sInt}_d$, where $\operatorname{sInt}_d: \D \to \D$ is defined by $\operatorname{sInt}_d(c) = \sign{d}{c} d\inv c d$.

\begin{defi}\label{def:change-map-to-the-left}
	Let $\V$ and $\V'$ be left $R$-supermodules and let $\psi: \V \to \V'$ be a $\ZZ_2$-homogeneous $R$-linear map. 
	We define $\psi^\circ: \V \to  \V'$ to be the following map, written on the left:
	\[
		\forall v\in \V\even \cup \V\odd, \quad \psi^\circ(v) = \sign{\psi}{v} v\psi.
	\]
\end{defi}

For example, using the identification $\D\sop = \End_R (\U\Star)$ as before, the left $\D$-action on $\U\Star$ is given by $df = \bar d^\circ (f)$, for all $d\in \D$ and $f\in \U\Star$.

\begin{lemma}\label{lemma:change-of-side-properties}
	Under the conditions of Definition \ref{def:change-map-to-the-left}, we have that, for all $r\in R\even \cup R\odd$ and $v\in \V$,  $\psi^\circ (rv) = \sign{\psi}{r} r \psi^\circ (v)$.
	Further, given another $\ZZ_2$-homogeneous $R$-linear map $\tau: \V' \to \V''$, we have $(\psi\tau)^\circ = \sign{\psi}{\tau} \tau^\circ\psi^\circ$. \qed
\end{lemma}

% \begin{remark}
%     The map $\psi^\circ$ defined above is not, in general, $R$-linear. Instead, it satisfies $\psi^\circ (rv) = \sign{\psi}{v} r \psi^\circ (v)$, for all $r\in R$ and $v\in V$. 
%     Also, given another $R$-linear map $\theta: \V' \to \V''$, we have that $(\psi\theta)^\circ = \sign{\psi}{\theta} \theta^\circ\psi^\circ$.

% Let $\psi: \V \to \V'$ and $\theta: \V' \to \V''$ be $R$-linear maps between left $R$-supermodules. Then
% \begin{enumerate}[(i)]
%     \item $\psi^\circ (rv) = \sign{\psi}{v} r \psi^\circ (v)$, for all $r\in R$ and $v\in V$;
%     \item $(\psi\theta)^\circ = \sign{\psi}{\theta} \theta^\circ\psi^\circ$.
% \end{enumerate}
% \end{remark}

Using the notation just introduced, we can rewrite Equation \eqref{eq:sesquilinear-before-B} as follows:
%
\begin{equation}\label{eq:vphi1-circ-is-D-linear}
	\begin{split}
		\vphi_1^\circ (ud) &= (-1)^{|\vphi_1|(|u|+|d|)} (ud)\vphi_1 \\
		&=\sign{\vphi_1}{u}(u\,\vphi_1) \overline{\vphi_0(d)} =
		\vphi_1^\circ (u) \overline{\vphi_0(d)},
	\end{split}
\end{equation}
%
which means, considering the right $\D$-action defined via Equation \eqref{eq:right-D-action},
% We can use $\vphi_0\colon \D \to \D\sop$ to define a right $\D$-module structure on $\U\Star$ via $f\cdot d \coloneqq f\vphi_0(d)$, for all $f\in \U\Star$ and $d\in \D$. 
% Considering this action, Equation \eqref{eq:vphi1-circ-is-D-linear} tells us 
that $\vphi_1^\circ$ is $\D$-linear. (Note, however, that Lemma \ref{lemma:change-of-side-properties} shows that $\vphi_1^\circ$ is not $R$-linear, in general.)

% Also, recalling the definition of the right $\D\sop$-action on $\U\Star$, note that Equation \eqref{eq:vphi1-circ-is-D-linear} can be rewritten using the left $\D$-action as
%
% \begin{equation}\label{eq:sesquilinear-with-vphi1-circ}
%     \vphi_1^\circ (ud) = (-1)^{|d|(|\vphi_1|+|u|)}\vphi_0 (d)\vphi_1^\circ (u).
% \end{equation}

% We define $\vphi_1 := \tilde\vphi_1^\circ$ and $\vphi_0: \D \to \D$ by $\vphi_0(d): \sign{\vphi_1}{d} \vphi_1 d^\circ \vphi_1\inv$. 
% Note that the left $\D$-action on $\U\Star$ is connected the right $\D\sop$-action by the formula $f\cdot d = d^\circ \cdot f$, hence $\vphi_0$ is an anti-super-automorphism of $\D$. Also, we have that $\vphi_1(ud) = \sign{u}{d} \vphi_1(d^\circ  $

% \begin{lemma}
%     Let $B: \U\times \U\Star$ be a nondegenerate $\vphi_0$-sesquilinear map. Then, given $r\in R$, there is a unique $s\in R$ such that \[
%         B(ru,v) = \sign{r}{u} B(u,sv).
%     \]
%     Further, the map $r\mapsto s$ is a super-anti-automorphism of $R$.
% \end{lemma}

% \begin{proof}
%     Consider $\theta: \U \to \U\Star$ as in Lemma \ref{lemma:sesquilinear-form-iff-D-linear-map}. 
%     We then have that
%     \begin{align*}
%         \theta(ru)(v) &= \sign{r}{u} \theta(u)(sv)\\
%         \theta (ru)(v) &= \sign{r}{u} (\theta (u)\circ s)(v)\\
%         \theta (ru) &= \sign{r}{u} \theta(u) \circ s\\
%         (\theta \circ r) (u) &= (-1)^{(|\theta| + |u|)|s|} s\Star (
%     \end{align*}
% \end{proof}

Now we define $B: \U\times \U \to \D$ by $B(u, v) = \vphi_1^\circ (u)(v)$.
By Proposition \ref{prop:sesquilinear-form-iff-D-linear-map}, we have that $B$ is a nondegenerate $\vphi_0$-sesquilinear map.
Using Lemma \ref{lemma:change-of-side-properties} and Equation \eqref{eq:R-action-back-on-the-right}, we have
%
\begin{equation*}
	\begin{split}
		B(ru,v) &= \vphi_1^\circ (ru)(v) = \sign{r}{\vphi_1} \big(r \cdot \vphi_1^\circ (u) \big) (v)\\ &= \sign{r}{\vphi_1} (-1)^{|r|(|\vphi_1| + |u|)} \big(\vphi_1^\circ (u) \circ \vphi(r) \big)(v) \\ &= \sign{r}{u} \vphi_1^\circ (u) \big( \vphi(r)v \big)= \sign{r}{u} B(u,\vphi(r)v).
	\end{split}
\end{equation*}
We have proved one direction of Theorem \ref{thm:vphi-iff-vphi0-and-B}, below. 
Recall the superinner automorphism $\operatorname{sInt}_d$ (\cref{def:superinner}). 

\begin{thm}\label{thm:vphi-iff-vphi0-and-B}
	Let $\D$ be a graded division superalgebra and let $\U$ be a nonzero right graded module of finite rank over $\D$.
	If $\vphi$ is degree-preserving super-anti-automorphism on $R \coloneqq \End_\D(\U)$, then there is a pair $(\vphi_0, B)$, where $\vphi_0$ is a degree-preserving super-anti-automorphism on $\D$ and $B\from \U \times \U \to \D$ is a nondegenerate $\vphi_0$-sesquilinear form, such that
	%
	\begin{equation}\label{eq:superadjunction}
		\forall r\in R\even \cup R\odd,\,\forall u, v \in \U\even \cup \U\odd,  \quad B(ru,v) = \sign{r}{u} B(u,\vphi(r)v).
	\end{equation}
	%
	Conversely, given a pair $(\vphi_0, B)$ as above, there is a unique degree-preserving super-anti-automorphism $\vphi$ on $R$ satisfying Equation \eqref{eq:superadjunction}.
	Moreover, another pair $(\vphi_0', B')$ determines the same super-anti-automorphism $\vphi$ if, and only if, there is a nonzero $G^\#$-homogeneous element $d\in \D$ such that $B'(u, v) = dB (u, v)$ for all $u, v \in \U$, and, hence, $\vphi_0' = \mathrm{sInt}_d \circ \vphi_0$.
\end{thm}

\begin{proof}
	The first assertion is already proved. For the converse, let $\vphi_0$ be a degree-preserving super-anti-automorphism on $\D$, let $B\from \U \times \U \to \D$ be a nondegenerate $\vphi_0$-sesquilinear form and consider $\theta$ as in
	Proposition \ref{prop:sesquilinear-form-iff-D-linear-map}. Then Equation \eqref{eq:superadjunction} is equivalent to:
	%
	\begin{alignat*}{2}
		\forall r\in R\even \cup R\odd,\,\forall u, v & \in \U\even \cup \U\odd,            & \theta (ru)(v)          & = \sign{r}{u} \theta(u)(\vphi(r)v)                                                   \\
		                                              &                                     &                         & = \sign{r}{u} \big(\theta (u)\circ \vphi(r)\big)(v)                                  \\
		%
		\intertext{and, hence, equivalent to}
		%
		\forall r\in R\even \cup R\odd,\,             & \forall u \in \U\even \cup  \U\odd, & \theta (ru)             & = \sign{r}{u} \theta(u) \circ \vphi(r).
		%
		\addtocounter{equation}{1}\tag{\theequation}\label{eq:theta-is-almost-R-superlinear}                                                                                                                 \\
		%
		\intertext{Recalling the definition of superadjoint operator, Equation \eqref{eq:theta-is-almost-R-superlinear} becomes}
		%
		\forall r\in R\even \cup R\odd,\,\forall u    & \in \U\even \cup \U\odd,            & (\theta \circ r) (u)    & = \sign{r}{u} (-1)^{(|\theta| + |u|)|r|} \big(\vphi(r)\big)\Star \big(\theta(u)\big) \\
		                                              &                                     &                         & =  \sign{r}{\theta} \big(\big(\vphi(r)\big)\Star \circ \theta \big) (u),
		%
		\intertext{which is the same as}
		%
		\forall r                                     & \in R\even \cup R\odd,              & \theta \circ r          & = \sign{r}{\theta}  \big(\vphi(r)\big)\Star \circ \theta.
		\intertext{In other words, we have}
		\forall r                                     & \in R\even \cup R\odd,              & \big(\vphi(r)\big)\Star & = \sign{r}{\theta}\, \theta \circ r \circ \theta\inv.
		%
		\addtocounter{equation}{1}\tag{\theequation}\label{eq:vphi-r-Star-is-a-superconjugation}
	\end{alignat*}
	%
	% We have shown that
	% %
	% \begin{equation}
	%   \big(\vphi(r)\big)\Star = \sign{r}{\theta}\, \theta \circ r \circ \theta\inv.
	% \end{equation}
	%
	Since $\U$ has finite rank over $\D$, the superadjunction map $\End_\D (\U) \to \End_{\D\sop} (\U\Star)$ is invertible and, hence, $\vphi$ is uniquely determined.
	Also, the properties of superadjunction imply that $\vphi$ is, indeed, a super-anti-automorphism of $R$.

	For the ``moreover'' part, let $d$ be a nonzero $G^\#$-homogeneous element of $\D$ and consider $\vphi_0' = \operatorname{sInt}_d \circ \, \vphi_0$ and $B' = dB$.
	We have that $B'$ is $\vphi_0'$-sesquilinear since, for all $c\in \D\even \cup \D\odd$ and $u,v \in \U\even \cup \U\odd$,
	%
	\begin{align*}
		B' (uc, v) & = dB (uc, v) = (-1)^{(|B| + |u| ) |c|} d\vphi_0(c) B(u,v) \addtocounter{equation}{1}\tag{\theequation}\label{eq:dB-is-sesquilinear} \\
		           & = (-1)^{(|B| + |u| ) |c|} d\vphi_0(c) d\inv d B(u,v)                                                                                \\
		           & =  (-1)^{(|B| + |u| ) |c|} \sign{c}{d} (\operatorname{sInt}_d \circ\, \vphi_0) (c)\, dB(u,v)                                        \\
		           & = (-1)^{(|B'| + |u|) |c|} (\operatorname{sInt}_d \circ\, \vphi_0) (c)\, B'(u,v).
	\end{align*}
	%
	To show $B'$ that is nondegenerate, note that $dB(u,v) = 0$ implies $B(u,v) =0$, hence  $\rad B' \subseteq \rad B$.
	Finally, it is straightforward that Equation \eqref{eq:superadjunction} is still true if we replace $B$ by $B'$.

	%and let $\vphi_0'$ be a super-anti-automorphism on $\D$. 
	% we will first check that $B'$ is, indeed, $\vphi_0'$-sesquilinear.

	To prove the other direction, we consider, again, the left $R$-supermodule structure on $\U\Star$ given by Equation \eqref{eq:R-action-back-on-the-right} and let $\theta\from \U \to \U\Star$ be as above, \ie, $\theta(u) = B(u, \cdot)$. Similarly, let $\theta'\from \U \to \U\Star$ be defined by $\theta' \coloneqq B'(u, \cdot)$.

	Combining Equations \eqref{eq:R-action-back-on-the-right} and  \eqref{eq:theta-is-almost-R-superlinear}, we have that
	%
	\begin{equation}\label{eq:theta-is-R-superlinear}
		\theta(ru) = \sign{\theta}{r} r\cdot \theta(u).
	\end{equation}
	%
	Define the map $\tilde \theta\from \U \to \U\Star$, written on the right, by $u\, \tilde\theta = \sign{u}{\theta} \theta(u)$, for all $u\in \U$ (compare with Definition \ref{def:change-map-to-the-left} and note that $\theta = (\tilde \theta)^\circ$).
	Then Equation \eqref{eq:theta-is-R-superlinear} becomes $(ru)\tilde\theta = r \cdot (u\,\tilde\theta)$, \ie, $\tilde\theta$ is $R$-linear.

	% Now, given a pair $(\vphi_0', B')$, we have the corresponding map $\theta'(u) \coloneqq B'(u, \cdot)$, for all $u\in \U$. 
	All these considerations about $\theta$ are also valid for $\theta'$, so we define $\tilde\theta'\from \U \to \U\Star$ by $u\,\tilde\theta' \coloneqq \sign{u}{\theta'} \theta'(u)$ and we get another $R$-linear map from $\U$ to $\U\Star$.
	By Lemma~\ref{lemma:nonuniqueness-of-vphi1}, there is $\bar d\in \D\sop$ such that $\tilde\theta' = \tilde\theta \bar d$.
	Applying Lemma \ref{lemma:change-of-side-properties}, this implies \[\theta' = \sign{\theta}{\bar d} \bar d^\circ \theta.\]
	But $\bar d^\circ \theta (u) = d\theta(u)$, where in the last term we use the left $\D$-action on $\U\Star$.
	Therefore $B'(u,v) = \theta'(u)(v) = d\theta(u)(v) = \sign{\theta}{d} dB(u, v)$, for all $u,v \in \U$.
	Replacing $d$ by $\sign{\theta}{d} d$, we get $B' = dB$.

	It remains to check that $\vphi_0' = \operatorname{sInt}_d \circ\, \vphi_0$.
	Since $B' = dB$, Equation \eqref{eq:dB-is-sesquilinear} is valid, hence $B'$ is $(\operatorname{sInt}_d \circ \vphi_0)$-sesquilinenar.
	We then have, for all $c\in \D\even \cup \D\odd$ and $u,v \in \U\even \cup \U\odd$,
	\[
		\vphi_0'(c)\, B'(u,v) = (-1)^{(|B'| + |u|) |c|} B'(uc,v) = (\operatorname{sInt_d} \circ\, \vphi_0) (c)\, B'(u,v).
	\]
	The form $B'$ is nondegenerate, so we can choose $G^\#$-homogeneous $u,v\in \U$ with $B'(u,v)\neq 0$. Then $B'(u,v)$ is invertible, hence $\vphi_0' (c) = (\operatorname{sInt}_d \circ\, \vphi_0) (c)$, concluding the proof.
\end{proof}

The ``conversely'' part of Theorem \ref{thm:vphi-iff-vphi0-and-B} motivates the following:

\begin{defi}\label{def:superadjunction}
	Let $\D$ be a graded-division superalgebra, $\U$ a graded right $\D$-module of finite rank and $B$ a nondegenerate sesquilinear form on $\U$.
	The unique super-anti-automorphism $\vphi$ on $\End_\D(\U)$ defined by Equation \eqref{eq:superadjunction} is called the \emph{superadjunction with respect to $B$} and, for every $r\in \End_\D(\U)$, the $\D$-linear map $\vphi(r)$ is called the \emph{superadjoint of $r$}. 
	We will denote by $E(\D, \U, B)$ the graded superalgebra $\End_\D(\U)$ endowed with this super-anti-automorphism $\vphi$. 
\end{defi}

\begin{remark}\label{conv:pick-even-form}
	Under the conditions of Theorem \ref{thm:vphi-iff-vphi0-and-B}, if $\D$ is an odd graded division superalgebra (\ie, $\D\odd \neq 0$), then we can choose the form $B$ to be even.
	This is possible since, by the ``moreover'' part, we can substitute an odd form $B$ by $dB$, for some $d\in \D\odd$.
\end{remark}


\section{Isomorphism of graded-simple superalgebras\texorpdfstring{\\}{} with super-anti-automorphism}\label{sec:iso-vphi-abstract}

In this section, we are going to describe isomorphisms between $G$-graded superalgebras with super-anti-automorphism that are graded-simple and satisfy the \dcc on graded left ideals.
As we have seen, such superalgebras are, up to isomorphism, of the form $\End_\D (\U)$ where the super-anti-automorphism is given by the superadjunction with respect to a nondegenerate homogeneous sesquilinear form $B\from \U \times \U \to \D$.

Even though, as superalgebras, $\End_\D( \U )$ is the same as $\End_\D( \U^{[g]} )$ for every $g \in G^\#$, an extra care should be taken when considering super-anti-automorphisms.
If $g\in G^\#$ is odd, a $\vphi_0$-sesquilinear form $B$ on the $\D$-module $\U$ is not $\vphi_0$-sesquilinear if considered on the $\D$-module $\U^{[g]}$, and Equation \eqref{eq:superadjunction} does not determine the same super-anti-automorphism $\vphi$.
This motivates the following:

\begin{defi}\label{defi:shift-on-B}
	Let $\U$ be a graded right $\D$-module and $\vphi_0\from \D \to \D$ be a super-anti-automorphism.
	Given a homogeneous $\vphi_0$-sesquilinear form $B$ on $\U$ and $g\in G^\#$, we define the $\vphi_0$-sesquilinear form  $B^{[g]}$ on $\U^{[g]}$ by $B^{[g]}(u,v) \coloneqq \sign{u}{g} B(u,v)$, for all $u,v \in \U$.
\end{defi}

\begin{remark}\label{rmk:deg-B^[g]}
	Note that $\deg B^{[g]} = g^{-2} \deg B$ and, in particular, $|B^{[g]}| = |B|$.
\end{remark}

\begin{lemma}\label{lemma:B^[b]-does-the-job}
	% Let $\U$ be a graded $\D$-module, $\vphi_0$ be a super-anti-automorphism on $\D$ and $B$ be a $\vphi_0$-sesquilinear form on $\U$. 
	For every $g\in G^\#$, $B^{[g]}$ is a homogeneous $\vphi_0$-sesquilinear form on $\U^{[g]}$.
	% Further, if $B$ is nondegenerate and the pair $(\vphi_0, B)$ determines the super-anti-automorphism $\vphi$ on $R \coloneqq \End_\D(\U) = \End_\D(\U^{[g]})$, then $B^{[g]}$ is nondegenerate and $(\vphi_0, B^{[g]})$ also determines $\vphi$.
	Further, if $B$ is nondegenerate, then so is $B^{[g]}$ and the superadjunction with respect to both is the same super-anti-automorphism $\vphi$ on $\End_\D(\U) = \End_\D(\U^{[g]})$.
\end{lemma}

\begin{proof}
	Let $u, v \in \U\even \cup \U\odd$.
	To avoid confusion, we will denote $u$ and $v$ by $u^{[g]}$ and $v^{[g]}$, respectively, when regarded as elements of $\U^{[g]}$.
	For all $d\in D\even \cup \D\odd$, we have:
	\begin{align*}
		B^{[g]}(u^{[g]}, v^{[g]}d) & = \sign{g}{u} B(u, vd) = \sign{g}{u} B(u, v)d =  B^{[g]}(u^{[g]}, v^{[g]})d
		\intertext{and}
		B^{[g]}(u^{[g]}d, v^{[g]}) & = \sign{g}{ud} B(ud, v) = (-1)^{|g|( |u| + |d| )} (-1)^{|d|( |B| + |u| )} \vphi_0(d) B(u, v) \\
		                           & = (-1)^{|d| ( |B| + |u| + |g| )} \vphi_0(d) \sign{g}{u} B(u, v)                              \\
		                           & = (-1)^{|d| ( |B| + |u^{[g]}| )} \vphi_0(d) B^{[g]}(u^{[g]}, v^{[g]})d .
		\intertext{Also, for all $r\in R\even \cup R\odd$, we have:}
		B^{[g]}(ru^{[g]}, v^{[g]}) & = \sign{g}{ru} B(ru, v) = (-1)^{|g|( |r| + |u| )} (-1)^{|r||u|} B(u, \vphi(r) v)             \\
		                           & = (-1)^{|r| ( |g| + |u|)} \sign{u}{g} B(u, \vphi(r) v)                                       \\
		                           & = \sign{r}{u^{[g]}} B^{[g]}(u^{[g]}, \vphi(r) v^{[g]}).
	\end{align*}
\end{proof}

Recall the concept of a module induced by a homomorphism of algebras (Definition \ref{def:twist}).

\begin{lemma}\label{lemma:twist-on-(U,B)}
	Let $\D$ and $\D'$ be graded-division superalgebras and let $\psi_0 \from \D \to \D'$ be an isomorphism.
	If $\U'$ is a $\D'$-supermodule and $B'$ is a homogeneous $\vphi_0'$-sesquilinear form on it, then $\psi_0\inv \circ B'$ is a homogeneous $(\psi_0\inv \circ \vphi_0' \circ \psi_0)$-sesquilinear form on the $\D$-supermodule $(\U')^{\psi_0}$ of the same degree as $B'$.
	Further, if $B'$ is nondegenerate, then so is $\psi_0\inv \circ B'$, and the superadjunction with respect to both is the same super-anti-automorphism $\vphi'$ on $\End_\D((\U')^{\psi_0}) = \End_{\D'}(\U')$.
\end{lemma}

\begin{proof}
	To simplify notation, let us put $B'' \coloneqq \psi_0\inv \circ B'$ and $\vphi_0'' \coloneqq \psi_0\inv \circ \vphi_0' \circ \psi_0$.
	It is clear that $\deg B'' = \deg B$ and, hence, $|B''| = |B'|$.
	Let $u,v \in (\U')\even \cup (\U')\odd$.
	To avoid confusion, we will denote $u$ and $v$ by $u^{\psi_0}$ and $v^{\psi_0}$, respectively, when regarded as elements of $(\U')^{\psi_0}$.
	For all $d\in D\even \cup \D\odd$, we have:
	\begin{align*}
		B''(u^{\psi_0}, v^{\psi_0}d)  & = \psi_0\inv \big( B'( u, v \, \psi_0(d)) \big)
		= \psi_0\inv \big( B' ( u, v ) \psi_0(d) \big)                                                                                                   \\
		                              & =  \psi_0\inv \big( B'(u, v) \big) d
		= B''(u^{\psi_0}, v^{\psi_0}) \,d
		% \end{align*}
		\intertext{and}
		% and
		% \begin{align*}
		B'' (u^{\psi_0}d, v^{\psi_0}) & = \psi_0\inv \big( B' ( u \, \psi_0(d), v ) \big)                                                                \\
		                              & = \psi_0\inv \Big( (-1)^{(|B'| + |u|) |\psi_0(d)|} \vphi_0' \big( \psi_0(d) \big) B' ( u, v) \Big)               \\
		                              & = (-1)^{(|B'| + |u|) |d|} \psi_0\inv \big( \vphi_0' \big( \psi_0 (d) \big) \big) \psi_0\inv \big( B'(u, v) \big) \\
		                              & = (-1)^{(|B''| + |u|) |d|} \vphi_0''(d) B''(u^{\psi_0}, v^{\psi_0}).
		% \end{align*}
		\intertext{Also, for all $r\in R\even \cup R\odd$, we have:}
		% Also, for all $r\in R\even \cup R\odd$, we have:
		% \begin{align*}
		B''(ru^{\psi_0}, v^{\psi_0})  & = \psi_0\inv \big( B' ( ru, v ) \big)                                                                            \\
		                              & = \psi_0\inv \big( \sign{r}{u} B' ( u, \vphi'( r ) v ) \big)                                                     \\
		                              & = \sign{r}{u} B'' ( u, \vphi'( r ) v ).
	\end{align*}
\end{proof}

\begin{defi}\label{def:iso-(U,B)}
	Let $\U$ and $\U'$ be graded right $\D$-supermodules, and let $B$ and $B'$ be homogeneous sesquilinear forms on $\U$ and $\U'$, respectively.
	An \emph{isomorphism from $(\U, B)$ to $(\U', B')$} is an isomorphism of graded modules $\theta\from \U \to \U'$ such that $B'( \theta(u), \theta(v) ) = B(u, v)$, for all $u,v \in \U$.
\end{defi}

Note that if $(\U, B)$ and $(\U', B')$ are isomorphic, then $B$ and $B'$ have the same degree in $G^\#$ and are sesquilinear with respect to the same $\vphi_0$.

% \begin{remark}
%     odd iso and $B^{[g]}$
% \end{remark}

\begin{thm}\label{thm:iso-abstract-vphi}
	Let $R \coloneqq \End_\D(\U)$ and $R' \coloneqq \End_{\D'}(\U')$, where $\D$ and $\D'$ are graded division superalgebras, and $\U$ and $\U'$ are nonzero right graded supermodules of finite rank over $\D$ and $\D'$, respectively.
	Let $\vphi$ and $\vphi'$ be degree preserving super-anti-automorphisms on $R$ and $R'$ determined, as in Theorem \ref{thm:vphi-iff-vphi0-and-B}, by pairs $(\vphi_0, B)$ and $(\vphi_0', B')$, respectively.
	If $\psi\from (R, \vphi) \to (R', \vphi')$ is an isomorphism, then there are $g\in G^\#$, a homogeneous element $0\neq d\in \D$, an isomorphism $\psi_0\from \D \to \D'$, and an isomorphism
	\begin{equation}\label{eq:iso-B-implies-vphi}
		\psi_1 \from (\U^{[g]}, dB^{[g]}) \to ( (\U')^{\psi_0}, \psi_0\inv \circ B' )
	\end{equation}
	such that
	\begin{equation}\label{eq:iso-super-anti-auto}
		\forall r\in R, \quad \psi(r) = \psi_1 \circ r \circ \psi_1\inv.
	\end{equation}
	% $\psi(r) = \psi_1 \circ r \circ \psi_1\inv$, for all $r\in R$. 
	Conversely, for any $g$, $d$, $\psi_0$ and $\psi_1$ as above,
	% the formula $\psi(r) \coloneqq \psi_1 \circ r \circ \psi_1\inv$ 
	Equation \eqref{eq:iso-super-anti-auto}
	defines an isomorphism $\psi \from (R, \vphi) \to (R', \vphi')$.
\end{thm}

\begin{proof}
	Given an isomorphism of graded superalgebras $\psi\from R \to R'$, define
	\[
		\tilde \vphi \coloneqq \psi\inv \circ \vphi' \circ \psi.
	\]
	Then $\psi$ is an isomorphism $(R, \vphi) \to (R', \vphi')$ if, and only if, $\vphi = \tilde\vphi$.

	Since $\psi$ is an isomorphism of $G^\#$-graded algebras, we can apply Theorem \ref{thm:iso-abstract} to conclude that there are $g\in G^\#$, an isomorphism of graded superalgebras $\psi_0\from \D \to \D'$, and an isomorphism of graded modules $\psi_1\from \U^{[g]} \to (\U')^{\psi_0}$ such that $\psi(r) = \psi_1 \circ r \circ \psi_1\inv$, for all $r\in R$.

	As in the proof of Lemma \ref{lemma:twist-on-(U,B)}, consider $\vphi_0'' \coloneqq \psi_0\inv \circ \vphi_0' \circ \psi_0$ and $B'' \coloneqq \psi_0\inv \circ B'$.
	Then define $\widetilde B \from \U^{[g]} \times \U^{[g]} \to \D$ by
	\[
		\widetilde B(u, v) \coloneqq B'' \big( \psi_1(u), \psi_1(v) \big)
	\]
	for all $u, v \in \U^{[g]}$.
	We claim that $\widetilde B$ is $\vphi_0''$-sesquilinear.
	Indeed, by Lemma \ref{lemma:twist-on-(U,B)}, we have
	\begin{align*}
		\widetilde B(u, vd) & = B'' \big( \psi_1(u), \psi_1(vd) \big)
		= B''\big( \psi_1(u), \psi_1(v)d \big)                                                                                  \\
		                    & = B''\big( \psi_1(u), \psi_1(v) \big)d = \widetilde B(u, v)d
		%
		\intertext{and}
		%
		\widetilde B(ud, v) & = B'' \big( \psi_1(ud), \psi_1(v) \big) = B'' \big( \psi_1(u) d, \psi_1(v) \big)                  \\
		                    & = (-1)^{(|B''| + |\psi_1(u)|) |d|} \vphi_0'' (d) B'' \big( \psi_1(u), \psi_1(v) \big)             \\
		                    & = (-1)^{(|\widetilde B| + |u|) |d|}  \vphi_0'' (d) \widetilde B(u, v).
		% \end{align*}
		\intertext{Also, $\tilde\vphi$ is the superadjunction with respect to $\widetilde B$:}
		% Also, $\tilde\vphi$ is the superadjunction with respect to $B''$:
		% \begin{align*}
		\widetilde B(ru, v) & = B'' \big( \psi_1(ru), \psi_1(v) \big)
		= B'' \big( (\psi_1 \circ r \circ \psi_1\inv) \psi_1(u), \psi_1(v) \big)                                                \\
		                    & = B'' \big( \psi(r) \psi_1(u), \psi_1(v) \big)                                                    \\
		                    & = \sign{\psi(r)}{\psi_1(u)} B'' \big( \psi_1(u), \vphi'( \psi(r) ) \psi_1(v) \big)                \\
		% &= \sign{r}{u} \psi_0 \inv \Big( B' \big( \psi_1(u), (\vphi'\circ \psi)(r)\psi_1(v) \big)\\
		% &= \sign{r}{u} B'' \big( \psi_1(u), (\psi\circ \tilde\vphi)(r) \psi_1(v) \big)\\
		                    & = \sign{r}{u} B'' \big( \psi_1(u), \psi (\tilde\vphi(r)) \psi_1(v) \big)                          \\
		                    & = \sign{r}{u} B'' \big( \psi_1(u), (\psi_1 \circ \tilde\vphi(r) \circ \psi_1\inv) \psi_1(v) \big) \\
		                    & = \sign{r}{u} B'' \big( \psi_1(u), \psi_1 (\tilde\vphi(r) v) \big)                                \\
		                    & = \sign{r}{u} \widetilde B \big( u, \tilde\vphi(r) v\big),
	\end{align*}
	where we have used Lemma \ref{lemma:twist-on-(U,B)} in the third line and the definition of $\tilde \vphi$ on the fourth line.
	Hence, applying \cref{thm:vphi-iff-vphi0-and-B} for $\U^{[g]}$ and Lemma \ref{lemma:B^[b]-does-the-job}, we conclude that $\vphi = \tilde\vphi$ if, and only if, there is a homogeneous $0 \neq d \in \D$ such that $\widetilde B = dB^{[g]}$.
	The result follows.
\end{proof}

We can interpret Theorem \ref{thm:iso-abstract-vphi} in terms of group actions.
For that, fix a fixed graded-division superalgebra $\D$.
We will define three (left) group actions on the class of pairs $(\U, B)$, where $\U \neq 0$ is a graded $\D$-supermodule and $B$ is a nondegenerate homogeneous sesquilinear form.
Recall that, by \cref{lemma:B-determines-vphi_0}, $B$ is $\vphi_0$-sesquilinear for a unique super-anti-automorphism $\vphi_0$ of $\D$.

Let $\D^\times_{\mathrm{gr}} \coloneqq \big( \bigcup_{g \in G^\#} \D_g \big)\backslash \{ 0 \}$, the group of nonzero homogeneous elements of $\D$.
Given $d\in \D^\times_{\mathrm{gr}}$, we define
\begin{equation}\label{eq:Dx_gr-action}
	d\cdot (\U, B) \coloneqq (\U, dB).
\end{equation}
Note that $dB$ is $(\mathrm{sInt}_d \circ \vphi_0)$-sesquilinear by Theorem \ref{thm:vphi-iff-vphi0-and-B}.

Let $A \coloneqq \Aut (\D)$, the group of automorphisms of $\D$ as a graded superalgebra.
Given $\tau \in A$, we define
\begin{equation}\label{eq:Aut(D)-action}
	\tau \cdot (\U, B) \coloneqq (\U^{\tau\inv}, \tau \circ B).
\end{equation}
Note that $\tau \circ B$ is $(\tau \circ \vphi_0 \circ \tau\inv)$-sesquilinear by Lemma \ref{lemma:twist-on-(U,B)}.
% (Note that we have changed the place of the ``inverse sign'' to get a left action.)

Finally, consider the group $G^\#$.
Given $g \in G^\#$, we define
\begin{equation}\label{eq:G-action}
	g \cdot (\U, B) \coloneqq (\U^{[g]}, B^{[g]}).
\end{equation}
Note that $B^{[g]}$ is $\vphi_0$-sesquilinear by Lemma \ref{lemma:B^[b]-does-the-job}.

\begin{lemma}\label{lemma:action-on-(U,B)}
	The three actions defined above give rise to a $(\D^\times_{\mathrm{gr}} \rtimes A) \times G^\#$-action, where $A$ acts on $\D^\times_{\mathrm{gr}}$ by evaluation.
\end{lemma}

\begin{proof}
	Let $d\in \D^\times_{\mathrm{gr}}$, $\tau \in A$, $g \in G^\#$ and $u, v \in \U$.
	First note that the action of $d$ does not change $\U$, so we only have to consider its effect on $B$.
	Since
	\begin{align*}
		(\tau \circ dB)(u,v) & = \tau \big( d B(u,v) \big) = \tau (d) \tau \big( B(u,v) \big) = \big( \tau(d) (\tau \circ B) \big) (u,v),
	\end{align*}
	the $\D^\times_{\mathrm{gr}}$-action combined with the $A$-action gives us a $(\D^\times_{\mathrm{gr}} \rtimes A)$-action.
	The $G^\#$-action commutes with the $\D^\times_{\mathrm{gr}}$-action since
	\begin{align*}
		(dB)^{[g]} (u, v) = \sign{g}{u} dB(u,v) = d B^{[g]}(u,v).
	\end{align*}
	Finally, the $G^\#$-action also commutes with the $A$-action since $(\U^{\tau\inv})^{[g]} = (\U^{[g]})^{\tau\inv}$ and
	\begin{align*}
		(\tau \circ B^{[g]}) (u,v)
		 & = \tau \big( \sign{g}{u} B(u,v) \big)
		 & = \sign{g}{u} (\tau \circ B) (u,v) = (\tau \circ B)^{[g]} (u,v).
	\end{align*}
\end{proof}

% These 3 actions can be combined into a $(\D^\times_{\mathrm{gr}} \rtimes A) \times G^\#$-action, where $A$ acts on $\D^\times_{\mathrm{gr}}$ by evaluation. 
% To see that, let $d\in \D^\times_{\mathrm{gr}}$, $\psi_0 \in A$ and $g \in G^\#$. 
% The action of $d$ does not change $\U$, hence commutes in the first entry with the other actions. 
% On the other hand, $(\psi_0 \circ dB)(u,v) = \psi_0 \big( (d B(u,v) ) \big) = \psi_0 (d) \psi_0 (B(u,v)) = (\psi_0(d) \circ \psi_0(B) ) (u,v)$, for all $u,v \in \U$.

\begin{cor}\label{cor:iso-with-actions}
	Under the assumptions of Theorem \ref{thm:iso-abstract-vphi}, if $\D \not \iso \D'$, then $(R, \vphi) \not \iso (R', \vphi')$.
	Otherwise, fix an isomorphism $\psi_0\from \D \to \D'$.
	Then $(R, \vphi) \iso (R', \vphi')$ if, and only if, $\big( (\U')^{\psi_0}, \psi_0\inv \circ B' \big)$ is isomorphic to an object in the $(\D^\times_{\mathrm{gr}} \rtimes A) \times G^\#$-orbit of $(\U, B)$.
	% and lie in the same orbit of the $(\D^\times_{\mathrm{gr}} \rtimes A) \times G^\#$-action. \qed
\end{cor}

% ----- 


\section{Matrix representation of a su\-per\--anti\--auto\-mor\-phism}

In this section, we are going to express the super-anti-automorphism (not necessarily involutive) $\vphi$ in terms of matrices with entries in $\D$.
One could do that by following Equation \eqref{eq:vphi-r-Star-is-a-superconjugation}, but we will take a different path.

As before, we suppose $\D$ is a graded division superalgebra, $\U$ is a nonzero right graded module of finite rank over $\D$, $R = \End_\D (\U)$ and $\vphi$ is a degree-preserving super-anti-automorphism on $R$.
Also, let $\vphi_0$ be a super-anti-automorphism on $\D$ and $B$ be a nondegenerate $\vphi_0$-sesquilinear form on $\U$ determinig $\vphi$ as in Theorem \ref{thm:vphi-iff-vphi0-and-B}.

\begin{defi}\label{def:matrix-representing-B}
	Given a graded basis $\{e_1, \ldots, e_k\}$ of $\U$, the \emph{matrix representing the form $B$} is defined to be $\Phi = (\Phi_{ij}) \in M_k(\D)$, where $\Phi_{ij} = B(e_i, e_j)$.
\end{defi}

From now on, let $\mc B = \{e_1, \ldots, e_k\}$ be a fixed homogeneous $\D$-basis of $\U$, following Convention \ref{conv:pick-even-basis} (\ie, if $\D$ is odd, we take $\mc B$ with only even elements).
We will use $\mc B$ to identify $R = \End_\D (\U)$ with $M_k (\D)$.
Also, we will denote $|e_i|$ simply by $|i|$ for all $i \in \{1, \ldots, k\}$.

% We will assume that the even elements precede the odd ones. 

% We will also follow Convention \ref{conv:pick-even-basis}, \ie, if $\D$ is odd, we will choose the $\D$-basis $\mc B$ to be one having only even elements.

% \begin{convention}\label{conv:pick-even-basis}
% 	If $\D$ is odd (\ie, $\D\odd \neq 0$), we will choose the $\D$-basis $\mc B$ to be one having only even elements.
% 	This is possible because, given any homogeneous $\D$-basis, we can multiply its odd elements by a nonzero homogeneous odd element in $\D$.
% \end{convention}

\begin{remark}\label{rmk:M(D)=M(FF)-tensor-D}
	It is well know that we a natural isomorphism between $M_k(\D)$ and $M_k(\FF)\tensor \D$, where we use the usual tensor product of algebras.
	For superalgebras, though, we usually use a different product on the space $M_k(\FF)\tensor \D$, given by \[(r_1 \tensor s_1) (r_2 \tensor s_2) = \sign{r_2}{s_1} (r_1 r_2) \tensor (s_1 s_2).\]
	If we are following Convention \ref{conv:pick-even-basis}, either $M_k(\FF)$ or $\D$ have trivial canonical $\ZZ_2$-grading, hence both tensor products coincide.
\end{remark}

\begin{defi}
	Let $X = (x_{ij})$ be a matrix in $M_k (\D)$.
	We define $\vphi_0 (X)$ to be the matrix obtained by applying $\vphi_0$ in each entry, \ie, $\vphi_0 (X) \coloneqq (\vphi_0(x_{ij}))$.
	We also extend the definition \emph{supertranspose} {(\tt see ??)} to matrices over $\D$ by putting $X\stransp \coloneqq \big((-1)^{(|i| + |j|) |i|} x_{ji} \big)$.
	Note that, with our choice of $\mc B$ in the case of odd $\D$, we have that $X\stransp = X\transp$, the ordinary transpose.
\end{defi}

\begin{prop}\label{prop:matrix-vphi}
	Let $\Phi$ be the matrix representing $B$.
	For every $r\in  R\even \cup R\odd$, let $X \in M_k(\D)$ be the matrix representing $r$ and let $Y \in M_k(\D)$ be the matrix representing $\vphi(r)$.
	Then, following Conventions \ref{conv:pick-even-form} and \ref{conv:pick-even-basis}, we have that
	%
	\begin{align}
		Y & = \Phi\inv\, \vphi_0( X\stransp )\, \Phi. \addtocounter{equation}{1}\tag{\theequation}\label{eq:matrix-vphi-D-even}
		\intertext{and, if $\D$ is odd, we have}
		Y & = \sign{B}{r}\,\Phi\inv\, \vphi_0( X\stransp )\, \Phi.\addtocounter{equation}{1}\tag{\theequation}\label{eq:matrix-vphi-D-odd}
	\end{align}
\end{prop}

\begin{proof}
	First of all, note that Equation \eqref{eq:superadjunction} is equivalent to the following:
	%
	\begin{alignat*}{2}
		\forall e_i, e_j \in \mc B, &  & B(re_i, e_j)                                              & = \sign{r}{i} B(e_i, \vphi(r) e_j),
		\intertext{which, by the definitions of $X$, $Y$ and $\Phi$, becomes}
		\forall e_i, e_j \in \mc B, &  & \quad B\bigg(\sum_{\ell=1}^k e_\ell x_{\ell i}, e_j\bigg) & = \sign{r}{i} B\bigg(e_i, \sum_{\ell=1}^k e_\ell y_{\ell j}\bigg) \addtocounter{equation}{1}\tag{\theequation}\label{eq:superadjunction-matrix}.
	\end{alignat*}
	%

	Fix arbitrary $p,q \in \{1, \ldots, k\}$ and suppose that the $(p,q)$-entry of $X$ is a nonzero $G^\#$-homogeneous element of $\D$ and $x_{ij} = 0$ elsewhere, \ie, $X$ represents the map $r \in \End_\D(\U)$ defined by $r e_i = \delta_{iq} e_p x_{pq}$.
	By the $\FF$-linearity of Equation \eqref{eq:matrix-vphi-D-even} %and \eqref{eq:matrix-vphi-D-odd}
	, it suffices to consider such $X$.
	Note that $|r| = |e_p| + |x_{pq}| - |e_q| = |p| + |q| + |x_{pq}|$.
	Then, on the one hand,
	%
	\begin{align*}
		B\bigg(\sum_{\ell=1}^k e_\ell x_{\ell i}, e_j\bigg) = B (e_p x_{pi}, e_j) & = (-1)^{ (|B| + |p|) |x_{pi}|} \vphi_0(x_{pi}) B(e_p, e_j)
		\\&= (-1)^{ (|B| + |p|) |x_{pi}|} \vphi_0(x_{pi}) \Phi_{pj},
		%\end{align*}
		%
		\intertext{which is only nonzero if $i = q$. On the other hand,}
		%
		%\begin{align*}
		\sign{r}{i} B\bigg(e_i, \sum_{\ell=1}^k e_\ell y_{\ell j}\bigg)
		                                                                          & = \sign{r}{i} \sum_{\ell=1}^k B(e_i, e_\ell) y_{\ell j}                                                                                           \\
		                                                                          & = (-1)^{ (|p| + |q| + |x_{pq}|) |i| } \sum_{\ell=1}^k \Phi_{i \ell} y_{\ell j}.
		%\end{align*}
		%
		\intertext{Therefore, Equation \eqref{eq:superadjunction-matrix} is equivalent to, for all $i,j \in \{1, \ldots, k\}$,}
		%
		%\begin{align*}
		(-1)^{ (|B| + |p|) |x_{pi}|} \vphi_0(x_{pi}) \Phi_{pj} %&= \sign{r}{i} \sum_{\ell=1}^k \Phi_{i \ell} y_{\ell j}\\
		                                                                          & = (-1)^{ (|p| + |q| + |x_{pq}|) |i| } \sum_{\ell=1}^k \Phi_{i \ell} y_{\ell j}. \addtocounter{equation}{1}\tag{\theequation}\label{eq:expression}
	\end{align*}
	%

	If $\D$ is even, then $|x_{pi}| = \bar 0$ and this equation reduces to
	\[              \vphi_0(x_{pi}) \Phi_{pj} = (-1)^{ (|p| + |q|) |i| } \sum_{\ell=1}^k \Phi_{i \ell} y_{\ell j}
	\]
	or, equivalently,
	\[
		\sum_{\ell=1}^k \Phi_{i \ell} y_{\ell j} = (-1)^{ (|p| + |q|) |i| } \vphi_0(x_{pi}) \Phi_{pj}.
	\]
	The left-hand side is the $(i,j)$-entry of $\Phi\, Y$.
	The right-hand side is only nonzero if $i = q$, so it can be rewritten as
	$(-1)^{ (|p| + |i|) |i| } \vphi_0(x_{pi}) \Phi_{pj}$.
	Recalling our choice of $X$, this is equal to $\sum_{\ell =1}^k (-1)^{ (|\ell| + |i|) |i| } \vphi_0(x_{\ell i}) \Phi_{\ell j}$, since $x_{\ell i}$ is only nonzero if $\ell = p$.
	Hence the right-hand side is the $(i,j)$-entry of $\vphi_0 (X\stransp) \Phi$, and Equation \eqref{eq:matrix-vphi-D-even} follows.

	If $\D$ is odd, by our choice of basis, Equation \eqref{eq:expression} reduces to
	\[
		(-1)^{|B| |x_{pi}|} \vphi_0(x_{pi}) \Phi_{pj} =  \sum_{\ell=1}^k \Phi_{i \ell} y_{\ell j},
	\]
	which, by the same reasoning as above, implies $(-1)^{|B||r|} \vphi_0(X\stransp) \Phi = \Phi Y$. 
% 	But we are following Convention \ref{conv:pick-even-form}, hence $|B| = \bar 0$ and, therefore, we have the desired result.
\end{proof}

% \begin{proof}
%     First of all, note that Equation \ref{eq:superadjunction} is equivalent to, for all $e_i, e_j \in \mc B$,
%     %
%     \[
%         B(re_i, e_j) = \sign{r}{i} B(e_i, \vphi(r) e_j).
%     \]
%     By the definitions of $X$, $Y$ and $\Phi$, we have that
%     %
%     \begin{align*}
%         B\bigg(\sum_{\ell=1}^k e_\ell x_{\ell i}, e_j\bigg) &= \sum_{\ell=1}^k (-1)^{( |B| + |\ell|) |x_{\ell i}|} \vphi_0( x_{\ell i} ) B (e_\ell x_{\ell i}, e_j)\\ &= \sum_{\ell=1}^k (-1)^{( |B| + |\ell|) |x_{\ell i}|} \vphi_0( x_{\ell i} ) \Phi_{\ell j}
%         \intertext{and}
%         \sign{r}{i} B\bigg(e_i, \sum_{\ell=1}^k e_\ell y_{\ell j}\bigg) 
%         &= \sign{r}{i} \sum_{\ell=1}^k B(e_i, e_\ell) y_{\ell j}\\
%         &= \sign{r}{i} \sum_{\ell=1}^k \Phi_{i \ell} y_{\ell j}. 
%     \end{align*}
%     %

%     Let $p,q \in \{1, \ldots, k\}$, and suppose $X$ is such that $x_{pq}$ is a nonzero $G^\#$-homogeneous element of $\D$ and $x_{ij} = 0$ elsewhere, \ie, $X$ represents the map $r \in \End_\D(\U)$ defined by $r e_i = \delta_{iq} e_p x_{pq}$. 
%     By the $\FF$-linearity of Equations \eqref{eq:matrix-vphi-D-even} and \eqref{eq:matrix-vphi-D-odd}, it suffices to consider such $X$. 
%     Note that $|r| = |e_q| + |x_{pq}| - |e_p| = |p| + |q| + |x_{pq}|$. 
%     Then
%     %
%     \begin{align*}
%         B\bigg(\sum_{\ell=1}^k e_\ell x_{\ell i}, e_j\bigg) = B (e_p x_{pi}, e_j) &= (-1)^{ (|B| + |p|) |x_{pi}|} \vphi_0(x_{pi}) B(e_p, e_j)
%         \\&= (-1)^{ (|B| + |p|) |x_{pi}|} \vphi_0(x_{pi}) \Phi_{pj},
%     \end{align*}
%     %
%     which is only nonzero if $i = q$. 
%     Combining this with Equation \eqref{eq:superadjunction-matrix}, we get
%     %
%     \begin{align*}
%         (-1)^{ (|B| + |p|) |x_{pi}|} \vphi_0(x_{pi}) \Phi_{pj} &= \sign{r}{i} \sum_{\ell=1}^k \Phi_{i \ell} y_{\ell j}\\
%         &= (-1)^{ (|p| + |q| + |x_{pq}|) |i| } \sum_{\ell=1}^k \Phi_{i \ell} y_{\ell j}. \addtocounter{equation}{1}\tag{\theequation}\label{eq:expression}
%     \end{align*}
%     %
%     If $\D$ is even, then $|x_{pi}| = 0$ and this expression reduces to
%     \[              \vphi_0(x_{pi}) \Phi_{pj} = (-1)^{ (|p| + |q|) |i| } \sum_{\ell=1}^k \Phi_{i \ell} y_{\ell j}
%     \]
%     or, reordering it, 
%     \[
%         \sum_{\ell=1}^k \Phi_{i \ell} y_{\ell j} = (-1)^{ (|p| + |q|) |i| } \vphi_0(x_{pi}) \Phi_{pj}.
%     \]
%     The left-hand side is the entry $ij$ of $\Phi\, Y$. 
%     The right-hand side is only nonzero if $i = q$, so it can be rewritten as 
%     $(-1)^{ (|p| + |i|) |i| } \vphi_0(x_{pi}) \Phi_{pj}$. 
%     Also, recalling our choice of $X$, it is equal to $\sum_{\ell =1}^k (-1)^{ (|\ell| + |i|) |i| } \vphi_0(x_{\ell i}) \Phi_{\ell j}$, since $x_{\ell i}$ is only nonzero if $\ell = p$. Hence the right-hand side is the $ij$ entry of $\vphi_0 (X\stransp) \Phi$, and Equation \eqref{eq:matrix-vphi-D-even} follows. 

%     If $\D$ is odd, by our choice of basis, Equation \eqref{eq:expression} reduces to
%     \[
%         (-1)^{|B| |x_{pi}|} \vphi_0(x_{pi}) \Phi_{pj} =  \sum_{\ell=1}^k \Phi_{i \ell} y_{\ell j},
%     \]
%     which implies $(-1)^{|B||r|} \phi_0(X\stransp) \Phi = \Phi Y$.
% \end{proof}

% \begin{remark}
%     Note that if $\D$ is odd, replacing $B$ by $dB$ with an odd $d$ if necessary, we can guarantee that $B$ is even. 
%     In this case, Equation \eqref{eq:matrix-vphi-D-odd} reduces to \eqref{eq:matrix-vphi-D-even}.
% \end{remark}

% \begin{cor}[of the proof]
%     Let $\vphi_0$ be a super-anti-automorphism on $\D$ and let $B$ be a nonzero $\vphi_0$-sesquilinear form on $\U$. If $\vphi\from R \to R$ be a parity preserving map satisfying Equation \eqref{eq:superadjunction}. 
%     Then $B$ is nondegenerate if, and only if, $\vphi$ is a super-anti-automorphism.
% \end{cor}



\section{Superinvolutions and sesquilinear forms}\label{sec:superinv-sesquilinear-forms}

Our goal now is to specialize the results of Section \ref{sec:super-anti-auto-and-sesquilinear} to the case where $\vphi$ is a superinvolution.
To this end, let us investigate what super-anti-automorphism of $\D$ and what sesquilinear form on $\U$ determine the super-anti-automorphism $\vphi\inv$.
Again, we suppose $\D$ is a graded division superalgebra, $\U$ is a nonzero right graded module of finite rank over $\D$ and put $R = \End_\D (\U)$.

\begin{defi}\label{def:barB}
	Given a super-anti-automorphism $\vphi_0$ on $\D$ and a $\vphi_0$-sesqui\-li\-near form $B$ on $\U$,  we define $\overline {B}\from \U\times \U \to \D$ by $\overline {B} (u,v) \coloneqq \sign{u}{v} \vphi_0\inv (B(v, u))$ for all $u, v \in \U$.
\end{defi}

% \begin{lemma}
%     Let $B\from \U\times \U\Star \to \D$ be a $\vphi_0$-sesquilinear form. Suppose there is a super-anti-automorphism $\vphi$ on $R = \End_\D(U)$ such that Equation \eqref{eq:superadjunction} holds. If $B$ is nonzero, then it is nondegenerate.
% \end{lemma}

\begin{prop}\label{prop:barB-determines-vphi-inv}
	Under the conditions of Definition \ref{def:barB}, we have that $\overline {B}$ is a $\vphi_0\inv$-sesquilinear form of the same degree and parity as $B$.
	Further, if $B$ is nondegenerate and $\vphi$ is the super-anti-automorphism on $R$ determined by $(\vphi_0, B)$ as in Theorem \ref{thm:vphi-iff-vphi0-and-B}, then $\overline{B}$ is nondegenerate and $\vphi\inv$ is determined by $(\vphi_0\inv, \overline B)$, \ie,
	%
	\begin{equation}\label{eq:barB-superadjunction}
		\forall r\in R\even \cup R\odd ,\,\forall u, v \in \U\even \cup \U\odd,  \quad \overline {B}(ru,v) = \sign{r}{u} \overline {B}(u,\vphi\inv (r)v).
	\end{equation}
	%
\end{prop}

\begin{proof}
	Since $B$ is $\FF$-bilinear, so is $\overline {B}$.
	Also, since $\vphi_0$ preserves degree and parity, $\overline {B}$ is homogeneous of the same degree and parity as $B$.
	Let us check the conditions of Definition \ref{def:sesquilinear-form} and Equation \eqref{eq:barB-superadjunction}.
	\vspace{2mm}
	\begin{align*}
		\intertext{ Condition \eqref{enum:linear-on-the-second}:}
		\overline {B} (u,vd) & = (-1)^{|u|( |v| + |d|)} \vphi_0\inv (B(vd, u))                                                       \\
		                     & = (-1)^{|u|( |v| + |d|)} (-1)^{|d| (|B| + |v|)} \vphi_0\inv \big( \vphi_0(d) B(v, u) \big)            \\
		                     & = (-1)^{|u||v| + |u||d| + |d||B| + |d||v|} (-1)^{|d| (|B| + |v| + |u|) }  \vphi_0\inv \big(B(v, u)) d \\ &= \sign{u}{v} \vphi_0\inv \big(B(v, u)) d = \overline {B}(u, v) d .
		\intertext{Condition \eqref{enum:vphi0-linear-on-the-first}:}
		\overline {B}(ud, v) & = (-1)^{(|u| + |d|) |v|} \vphi_0\inv \big( B(v, ud) \big)                                             \\ &= (-1)^{(|u| + |d|) |v|} \vphi_0\inv \big( B(v, u)d \big) \\ &= (-1)^{(|u| + |d|) |v|} (-1)^{|d| (|B| + |v| + |u|)} \vphi_0\inv (d) \vphi_0\inv\big( B(v, u) \big) \\ &= (-1)^{|u||v| + |d| |B| + |d||u|} \vphi_0\inv (d) \vphi_0\inv\big( B(v, u) \big) \\ &= (-1)^{(|B| + |u|) |d|} \vphi_0\inv (d) \overline {B}(u, v).
		\intertext{For Equation \eqref{eq:barB-superadjunction}, note that replacing $r$ for $\vphi\inv(r)$, Equation \eqref{eq:superadjunction} can be rewritten as}
		B(v, ru)             & = \sign{r}{v} B(\vphi\inv(r)v,u).
		\intertext{Hence, we have that}
		\overline {B}(ru, v) & = (-1)^{(|r| + |u|) |v|} \vphi_0\inv \big( B(v, ru) \big)                                             \\ &= (-1)^{(|r| + |u|) |v|} (-1)^{|r||v|} \vphi_0\inv \big( B(\vphi\inv (r)v, u) \big) \\ &= (-1)^{|u||v|} \vphi_0\inv \big( B(\vphi\inv (r)v, u) \big) \\ &= (-1)^{|u||v|} (-1)^{(|r| + |v|) |u|} \overline {B}(u, \vphi\inv (r)v) \\ &= \sign{r}{u} \overline {B}(u, \vphi\inv (r)v).
	\end{align*}

	Finally, Equation \eqref{eq:barB-superadjunction} together with $B$ being nondegenerate implies that $\overline{B}$ is nondegenerate.
	To see that, let $u$ be a nonzero homogeneous element in $\rad \overline{B}$.
	Then for for every $r\in R\even \cup R\odd$ and $v\in \U\even \cup \U\odd$, we have that $\overline{B}(u, \vphi\inv (r) v) = 0$, hence $\overline{B}(ru, v) = 0$.
	Since $r \in R\even \cup R\odd$ and $v\in \U\even \cup \U\odd$ were arbitrary, this implies $\overline{B} (Ru, \U) = 0$.
	But $\U$ is simple as a graded $R$-supermodule, so we would have $\overline{B}(\U, \U) = 0$ and then, using that $\vphi_0$ is bijective, $B (\U, \U) = 0$, a contradiction.
\end{proof}

\begin{lemma}\label{lemma:bar-dB}
	Under the conditions of Definition \ref{def:barB}, let $d$ be a nonzero $G^\#$-homogeneous element of $\D$ and consider $\vphi_0' \coloneqq \operatorname{sInt}_d\circ\, \vphi_0$ and $B' \coloneqq d B$.
	Then $\overline {B'} = (-1)^{|d|} \vphi_0\inv (d) \overline B$.
\end{lemma}

\begin{proof}
	Note that $(\vphi_0')\inv = \vphi_0\inv \circ \operatorname{sInt}_{d}\inv = \vphi_0\inv \circ \operatorname{sInt}_{d\inv}$.
	Hence, for all $u, v \in \U\even \cup \U\odd$,
	%
	\begin{align*}
		\overline {B'} (u,v) & = \sign{u}{v} (\vphi_0\inv \circ \operatorname{sInt}_{d\inv})  (d B(v, u) )                    \\
		                     & = \sign{u}{v} \vphi_0\inv \big( (-1)^{|d| (|d| + |B| + |u| + |v|)}\,  d\inv d B (v, u) d \big) \\ &= \sign{u}{v} (-1)^{|d|}\, \vphi_0\inv (d) \vphi_0\inv(B(v, u)) =  (-1)^{|d|}\,\vphi_0\inv (d) \overline {B} (u, v).
	\end{align*}
\end{proof}

We are primarily interested in the case $\FF$ is an algebraically closed field and $\D$ is finite dimensional.
In this case, we have that $\D\even_e = \FF 1$, so we are under the hypothesis of the following theorem, which is a graded version of \cite[Theorem 7]{racine}:

\begin{thm}\label{thm:vphi-involution-iff-delta-pm-1}
	Let $\D$ be a graded division superalgebra such that $\D_e = \FF 1$, let $\U$ be a nonzero right graded module of finite rank over $\D$ and let $\vphi$ be a degree-preserving super-anti-automorphism on $R \coloneqq \End_\D (\U)$.
	Consider a super-anti-automorphism $\vphi_0$ on $\D$ and a nondegenerate $\vphi_0$-sesqui\-li\-near form $B$ on $\U$ determining $\vphi$ as in Theorem \ref{thm:vphi-iff-vphi0-and-B}.
	Then $\vphi$ is a superinvolution if, and only if, $\overline B = \pm B$. 
	Moreover, if this is the case, then $\vphi_0$ is a superinvolution. 
\end{thm}

\begin{proof}
	Using Proposition \ref{prop:barB-determines-vphi-inv} and Theorem \ref{thm:vphi-iff-vphi0-and-B}, we conclude that $\vphi = \vphi\inv$ if, and only if, there is a $G^\#$-homogeneous element $0 \neq \delta \in \D$ such that $\overline {B} = \delta B$. 
	Hence, it only remains to prove that, in this case, we have $\delta \in \pmone$ and $\vphi_0^2 = \operatorname{id}_\D$. 
	
	Since $B$ and $\overline {B}$ have the same $G^\#$-degree, $\overline {B} = \delta B$ implies that $\delta \in \D_e$. 
	By Lemma \ref{lemma:bar-dB}, we have that
	\[
		B = \overline {\overline B} = \overline {\delta B} = (-1)^{|\delta|} \vphi_0\inv(\delta) \overline B= \vphi_0\inv(\delta) \overline B = \vphi_0\inv(\delta) \delta B,
	\]
	hence $\vphi_0\inv(\delta) \delta = 1$. 
	Since we are assuming $\D_e = \FF 1$, we have that $\vphi_0\inv(\delta) = \delta$, so $\delta^2 = 1$ and, therefore, $\delta \in \{ \pm 1 \}$. 

	To see that $\vphi_0^ 2 = \operatorname{id}_\D$, note that, from Theorem \ref{thm:vphi-iff-vphi0-and-B}, $\vphi_0\inv = \operatorname{sInt}_\delta \circ \,\vphi_0 = \vphi_0$. 
\end{proof}


\section{Classification of graded-simple superalgebras\texorpdfstring{\\}{} with superinvolution}\label{sec:classification-grd-simple-with-sinv}

In this section we introduce parameters that describe the graded finite dimensional associative superalgebras with superinvolution $(R, \vphi)$ where $R$ is graded-simple and, then, we give a classification result in terms of these parameters. 
We follow the ideas used in \cite[Sections 2 and 3]{paper-adrian} to classify graded-simple associative algebras with involution over the field of real numbers. 
Throughout this section, we will assume that $\FF$ is an algebraically closed field with $\Char \FF \neq 2$. 

Recall from \cref{thm:End-over-D,thm:vphi-iff-vphi0-and-B,thm:vphi-involution-iff-delta-pm-1}
% Theorems \ref{thm:End-over-D}, \ref{thm:vphi-iff-vphi0-and-B} and \ref{thm:vphi-involution-iff-delta-pm-1}
that, under the above assumptions, we have that $R \iso \End_\D (\U)$ and that $\vphi$ is determined by a superinvolution $\vphi_0$ on $\D$ and a nondegenerate homogeneous $\vphi_0$-sesquilinear form $B$ on $\U$.

\subsection{Parametrization of \texorpdfstring{$(\D, \vphi_0)$}{(D, phi0)}}\label{ssec:param-D-vphi}

Recall, from \cref{ssec:T-beta-p}, that the isomorphism class of a finite dimensional graded-division superalgebra $\D$ is determined by a triple $(T, \beta, p)$ where $T \coloneqq \supp \D \subseteq G^\#$ is a finite abelian group, $\beta\from T\times T \to \FF^\times$ is an alternating bicharacter and $p\from T \to \ZZ_2$ is a group homomorphism. 
Also recall that we define the skew-symmetric bicharacter $\tilde\beta \from T\times T \to \FF^\times$ by $\tilde\beta (a,b) = (-1)^{p(a) p(b)} \beta(a, b)$, for all $a, b \in T$ (see \cref{eq:tilde-beta-def}). 

Recall that, since each component $\D_t$ of $\D$ is one-dimensional, an invertible degree-preserving map $\vphi_0\from \D \to \D$ is completely determined by a map $\eta\from T \to \FF^\times$.

\begin{prop}\label{prop:superpolarization}
	Let $\vphi_0\from \D \to \D$ be the invertible degree-preserving map determined by a map $\eta\from T \to \FF^\times$ as follows: $\vphi_0(X_t) = \eta(t) X_t$ for all $t\in T$ and $X_t\in \D_t$.
	Then $\vphi_0$ is a super-anti-automorphism if, and only if, 
	%
	\begin{equation}\label{eq:superpolarization}
		\forall a,b\in T, \quad \eta(ab) = \tilde\beta(a,b) \eta(a) \eta(b).
	\end{equation}
	%
	%
% 	\begin{equation}\label{eq:superpolarization}
% 		\forall a,b\in T, \quad (-1)^{p(a) p(b)} \beta(a,b) =  \eta(ab) \eta(a)\inv \eta(b)\inv,
% 	\end{equation}
	% 
	Moreover, $\D$ admits a super-anti-automorphism if, and only if, $\tilde\beta$ (or, equivalently, $\beta$) only takes values $\pm 1$.
\end{prop}

\begin{proof}
	For all $a,b \in T$, let $X_a \in \D_a$ and $X_b\in \D_b$. 
	Then:
	%
	\begin{alignat*}{2}
		     &  & \vphi_0(X_a X_b)             & = (-1)^{p(a) p(b)} \vphi_0(X_b) \vphi_0(X_a)          \\
		\iff &  & \,\, \eta(ab)X_a X_b         & = (-1)^{p(a) p(b)} \eta(a) \eta(b) X_b X_a            \\
		\iff &  & \, \eta(ab)X_a X_b           & = (-1)^{p(a) p(b)} \eta(a) \eta(b) \beta(b,a) X_a X_b \\
		\iff &  & \eta(ab)                     & = (-1)^{p(a) p(b)} \eta(a) \eta(b) \beta(b,a)
		\\
		\iff &  & \eta(ab)                     & = \tilde\beta(b,a) \eta(a) \eta(b)
% 		\\
% 		\iff &  & \eta(ab) \eta(a)\inv \eta(b)\inv & = \tilde\beta(b,a). 
	\end{alignat*}
	%
	If $a$ and $b$ are switched, since $T$ is abelian, we get $\eta(ab) = \tilde\beta(a,b) \eta(a) \eta(b)$, as desired. 
	Also, it follows that $\tilde\beta(b,a) = \tilde\beta(a,b)$. 
	Using that $\tilde\beta$ is skew-symmetric, \ie,  $\tilde\beta(b, a) = \tilde\beta (a, b)\inv$, we have that $\tilde\beta(a,b)^2 = 1$ and, hence, $\tilde\beta$ only takes values $\pm 1$, proving one direction of the ``moreover'' part. 
	The converse follows from the fact that the isomorphism class of $\D\sop$ is determined by $(T, \beta\inv, p)$, so if $\beta$ takes only values in $\{ \pm 1 \}$, there must be an isomorphism from $\D$ to $\D\sop$, which can be seen as a super-anti-automorphism of $\D$.
\end{proof}

\begin{cor}\label{cor:super-anti-auto-squares-in-radical}
    The graded-superalgebra $\D$ admits a super-anti-automorphism if, and only if, 
    $t^2\in \rad\tilde\beta$, for all $t \in T$. \qed
\end{cor}

Next definition is borrowed from the theory of group cohomology, and allows a compact statement of \cref{eq:superpolarization}. 
It will also be used in \cref{sec:model-O}.

% gives  us a more compact way to express Equation \eqref{eq:superpolarization}.  

\begin{defi}\label{def:coboundary}
	Let $H$ and $K$ be groups and let $f\from H \to K$ be any map.
	We define $\mathrm{d} f\from H\times H \to K$ to be the map given by $(\mathrm{d} f)\, (a,b) = f(ab)f(a)\inv f(b)\inv$ for all $a,b \in H$.
	The maps $H\times H \to K$ of this form are called \emph{$2$-coboundaries}.
\end{defi}

% Note that $f\from H \to K$ is a group homomorphism if, and only if, $(\mathrm{d} f)\, (a,b) = e$ for all $a,b \in H$. 

Hence \cref{eq:superpolarization} can be written simply as $\mathrm{d}\eta = \tilde\beta.$

If $\vphi_0$ is a super-anti-automorphism on $\D$ as in \cref{prop:superpolarization}, we say that
% $(T, \beta, p, \eta)$ is the \emph{quadruple associated} to the graded-division superalgebra with super-anti-automorphism $(\D,\vphi_0)$, or that 
$(\D,\vphi_0)$ is a graded-division superalgebra with super-anti-automorphism \emph{associated} to the quadruple $(T, \beta, p, \eta)$. 
It follows from \cref{lemma:colour-tensor-product,prop:superpolarization} that for any finite abelian group $T$, alternating bicharacter $\beta\from T\times T \to \FF^\times$, group homomorphism $p\from T \to \ZZ_2$ and map $\eta\from T \to \FF^\times$ such that $\mathrm{d}\eta = \tilde\beta$, there is a graded-division superalgebra with super-anti-automorphism associated to $(T, \beta, p, \eta)$. 
Thus the quadruples $(T, \beta, p, \eta)$ parametrize the isomorphism classes of finite dimensional graded-division superalgebra with super-anti-automorphism. 
It is clear that the corresponding $\vphi_0$ is a superinvolution if, and only if, $\eta(t) \in \{ \pm 1 \}$ for all $t\in T$. 

\begin{remark}
    By \cref{prop:skew-bicharacter-grd-SA}, it follows that from $(T, \eta)$, where $T$ is a finite abelian group and $\eta\from T \to \FF^\times$ is a map such that $\mathrm{d}\eta$ is a skew-symmetric bicharacter, we can recover both $\beta$ and $p$. 
\end{remark}

% The following is an easy consequence of \cref{prop:superpolarization} and will be used in \cref{chap:grds-sinv-simple}. 

\begin{cor}\label{cor:eta-t-square}
    Suppose $\eta$ determines a superinvolution on $\D$, \ie, $\eta$ takes values in $\pmone$. 
    For every element $t\in T$, $\eta(t^2) = (-1)^{p(t)}$ for all $t\in T$. 
    In particular, every element in $T^-$ has order at least $4$. 
\end{cor}

\begin{proof}
    By \cref{eq:superpolarization}, we have $\eta (t^2) = \tilde\beta(t,t) \eta(t)^2 = (-1)^{p(t)} \beta(t,t) = (-1)^{p(t)}$. 
    
    Now let $t\in T^-$.
	Every odd power of $t$ is also an odd element, so $t$ cannot have an odd order.
	But $\eta (t^2) = -1$, hence $t^2 \neq e$. 
\end{proof}


% \begin{prop}
%     Let $(\D,\vphi_0)$ and $(\D',\vphi_0')$ be finite dimensional graded-division superalgebras with super-anti-automorphism associated to quadruples $(T, \beta, p, \eta)$ and $(T', \beta', p', \eta')$, respectively. 
%     Then $(\D,\vphi_0) \iso (\D,\vphi_0)$ if, and only if, $T =T'$, $\beta = \beta'$, $p = p'$ and $\eta = \eta'$.
% \end{prop}

% \begin{proof}
%     We have that $\D \iso \D'$ if, and only if, $T =T'$, $\beta = \beta'$ and $p = p'$. 
%     Suppose this is the case, fix $0 \neq X_t\in \D_t$ and let $\psi\from \D \to \D'$ be an isomorphism of graded algebras. 
%     We have that $\psi(X_t) = \chi(t) X_t'$
%     If $\eta = \eta'$, it is easy to see that $\psi$ is an isomorphism of graded superalgebras with super-anti-automorphism. 
% \end{proof}

\subsection{Parametrization of \texorpdfstring{$(\U, B)$}{(U, B)}}\label{ssec:parameters-(U-B)}

% Let $R \coloneqq \End_\D (\U)$ and let $\vphi$ be a superinvolution on R. 
% From Theorem \ref{thm:vphi-involution-iff-delta-pm-1}, $\vphi$ is determined by a pair $(\vphi_0, B)$, where $\vphi_0$ is a superinvolution on $\D$ and $B\from \U \times \U \to \D$ is a homogeneous $\vphi_0$-sesquilinear form on $\U$ such that $\overline B = \delta B$ with $\delta \in \{ \pm 1 \}$. 

Let $(\D, \vphi_0)$ be a fixed graded-division superalgebra with super-anti-automorphism associated to $(T, \beta, p, \eta)$.

Recall that a $G$-graded supermodule $\U$ can be regarded as a $G^\#$-graded module 
and that its isomorphism class is determined by the map $\kappa\from G^\#/T \to \ZZ_{\geq 0}$ with finite support (see \cref{ssec:D-modules}).
Explicitly, $\kappa (gT) = \dim_\D \U_{gT}$, where $\U_{gT}$ is the isotypic component associated to the coset $gT$.

% defined by $\kappa (gT) = \dim_\D \U_{gT}$, where $\U_{gT}$ is the isotypic component associated to the coset $gT$ (Subsection \ref{U-in-terms-of-GxZZ2}).

% described by a map $\kappa\from G^\#/T \to \ZZ_{\geq 0}$ with finite support. 
We will now consider a
% nondegenerate 
homogeneous $\vphi_0$-sesquilinear form $B$ on $\U$, $\deg B = g_0 \in G^\#$.
% Its presence gives us further restrictions on $\kappa$. 
Since $B$ has degree $g_0$ and takes values in $\D$, if $B(\U_{g}, \U_{h}) \neq 0$ for some $g, h \in G^\#$, then $g_0 g h \in T$.
In terms of isotypic components, this means that, given $\U_{g T}$, there is at most one isotypic component $\U_{hT}$ such that $B(\U_{gT}, \U_{hT}) \neq 0$, namely, $\U_{g_0\inv g\inv T}$.
% we can only have $B(\U_{g T}, \U_{g' T}) \neq 0$ if $\bar {g'} = \bar g_0\inv \bar g\inv \in G^\#/T$. 
We say that the components $\U_{gT}$ and $\U_{g_0\inv g\inv T}$ are \emph{paired by $B$}.
% Since $B$ is nondegenerate, we have that $\U_{g T}$ and $\U_{g_0\inv g\inv T}$ are in duality via $B$ and, in particular, $\kappa(g T) = \kappa(g_0\inv g \inv T)$. 

% Note that $\U_{g T} = \U_{g_0\inv g\inv T}$ if, and only if, $g_0 g^2 \in T$. 
% In this case, we say that $\U_{gT}$ is a \emph{self-dual component}. 
% Otherwise, we say that $\U_{gT}$ and $\U_{g_0\inv g\inv T}$ form a \emph{pair of dual components}. 

% From now on, we will assume that $\vphi_0$ is a superinvolution, \ie, that $\eta$ takes values in $\pmone$. 
We will now reduce the study of $B$ to the study of $\FF$-bilinear forms.
Fix a set-theoretic section $\xi\from G^\#/T \to G^\#$ of the natural homomorphism $ G^\# \to G^\#/T$, \ie, $\xi (x) \in x$ for all $x \in G^\#/T$, and fix a nonzero element $X_t \in \D_t$ for all $t\in T$.
Note that $\U_{\xi(x)} \tensor \D \iso \U_x$ via the map $u \tensor d \mapsto ud$ and, hence, an $\FF$-basis of $\U_{\xi(x)}$ is a graded $\D$-basis for $\U_x$.
In view of Convention \ref{conv:pick-even-basis}, if $\D$ is odd we choose $\xi$ to take values in $G = G\times \{ \bar 0 \}$.

For a given $x \in G^\#/T$, set $y \coloneqq g_0\inv x\inv \in G^\#/T$ and $t \coloneqq g_0 \xi(x) \xi(y) \in T$.
Also, set $\V_x \coloneqq \U_x + \U_y$ (so $\V_x = \U_x$ if $x=y$ and $\V_x = \U_x \oplus \U_y$ if $x \neq y$) and $V_x \coloneqq \U_{\xi(x)} + \U_{\xi(y)}$ (so $\V_x \iso V_x \tensor \D$).
We will denote the restriction of $B$ to $\V_x$ by $B_x$ and define the bilinear map $\tilde{B}_x\from V_x \times V_x \to \FF$ by
\begin{equation}\label{eq:B_x-tilde}
	\tilde{B}_x (u,v) \coloneqq X_{t}\inv B_x (u,v),
\end{equation}
for all $u,v \in V_x$.
It is clear that $B$ is nondegenerate if, and only if, $B_x$ is nondegenerate for every $x \in G^\#/T$.
If this is the case, $\U_x$ and $\U_y$ are dual to each other and, hence, $\kappa(x) = \kappa(y)$.
% If $x = y$ or, equivalently, $g_0 x^2 = T$, we say that $\U_x$ is a \emph{self-dual component}, otherwise we say that $\U_x$ and $\U_y$ are a \emph{pair of dual components}. 

\begin{lemma}\label{lemma:B_x-nondeg}
	The $\vphi_0$-sesquilinear form $B_x$ is nondegenerate if, and only if, the bilinear form $\tilde{B}_x$ is nondegenerate.
\end{lemma}

\begin{proof}
	Assume $B_x$ is nondegenerate and let $u \in V_x$ be such that $\tilde{B}_x(u,v) = 0$ for all $v \in V_x$.
	Then $B_x(u, v) = X_{t} \tilde{B}_x(u,v) = 0$ for all $v \in V_x$ and, hence, $B_x(u, vd) = B_x(u, v)d = 0$ for all $d \in \D$.
	It follows that $B_x(u,v) = 0$ for all $v\in \V_x = V_x \D$ and, therefore, $u = 0$.

	Now assume $\tilde{B}_x$ is nondegenerate and let $u \in \V_x$ be such that $B_x (u,v) = 0$ for all $v \in \V_x$.
	Let $v\in V_x$ be homogeneous and write $u = \sum_{g\in G^\#} u_g$ where $u_g \in \U_g$.
	Then $0 = B_x(\sum_{g\in G^\#} u_g, v) = \sum_{g\in G^\#} B_x(u_g, v)$ and, since the summands have pairwise distinct degrees, $B_x(u_g, v) = 0$ for all $g\in G^\#$.
	Also, since $\xi(gT)\inv g \in T$, we have that $u_g = \tilde u_g d$ for some homogeneous elements $\tilde u_g \in V_x$ and $0 \neq d \in \D$.
	Then $0 = B_x(u_g, v) = (-1)^{|d|(|B| + |\tilde u_g|)} \vphi_0(d) B_x(\tilde u_g, v)$, and hence $B_x(\tilde u_g, v) = 0$.
	It follows that $\tilde{B}_x(\tilde u_g, v) = 0$, for all $v\in V_x$, which implies $\tilde u_g = 0$.
	Therefore, $u = 0$, concluding the proof.
\end{proof}

We are interested in the case when the superadjunction with respect to $B$ is involutive.
By Theorem \ref{thm:vphi-involution-iff-delta-pm-1}, if this is the case, then $\vphi_0$ is also involutive. 
Hence, from now on, we will assume that $\eta$ takes values in $\pmone$.

\begin{lemma}\label{lemma:B_x-delta}
	Let $\delta \in \pmone$.
	Then $\overline{B_x} = \delta B_x$ if, and only if,
	\[
		\tilde{B}_x (v, u) = (-1)^{|u| |v|} \eta(t) \delta \tilde{B}_x (u, v)
	\]
	for all $u, v \in V_x$, where $t \coloneqq g_0 \xi(x) \xi(g_0\inv x\inv)$.
\end{lemma}

\begin{proof}
	Let $u,v \in \V_x$.
	By definition of $\overline{B_x}$, we have:
	\begin{alignat*}{3}
		\overline{B_x} (u, v) & = \sign{u}{v} \vphi_0\inv( B_x(v, u) )
		                      &                                                  & = \sign{u}{v} \vphi_0\inv( X_t \tilde{B}_x(v, u) ) \\
		                      & = \sign{u}{v} \tilde{B}_x(v, u) \vphi_0\inv(X_t)
		                      &                                                  & = \sign{u}{v} \tilde{B}_x(v, u) \eta(t)\inv X_t,
	\end{alignat*}
	where we have used the fact that $\tilde B_x (v, u) \in \FF$.
	Hence, $\overline{B_x} (u, v) = \delta {B_x} (u, v)$ if, and only if,
	$\sign{u}{v} \tilde{B}_x(v, u) \eta(t)\inv X_t = \delta X_t \tilde{B}_x(u, v)$, and the result follows.
\end{proof}

Recall the identification $M_k (\D) = \M_k(\FF) \tensor \D$ (see Remark \ref{rmk:M(D)=M(FF)-tensor-D}).
In the next two propositions we consider a component paired to itself and two components paired to one another.

% For convenience, we will also fix elements $0 \neq X_t \in \D_t$ for all $t\in T$.

\begin{prop}\label{prop:self-dual-components}
	Let $\delta \in \pmone$.
	Suppose $g_0 x^2 = T$,  and  set $t \coloneqq g_0\xi(x)^2 \in T$.
	Then
	\begin{equation}\label{eq:mu_x}
		\mu_{x} \coloneqq (-1)^{|\xi(x)|} \eta(t) \delta \in \pmone
	\end{equation}
	% $\mu_{gT} = (-1)^{|g|} \eta(g_0 g^2) \delta \in \pmone$
	does not depend on the choice of the section $\xi\from G^\#/T \to G^\#$.
	% Further, there is a $\D$-basis on $\U_{gT}$ consisting only of homogeneous elements of degree $g$ such that  that the restriction of $B$ to $\U_{gT}$ is represented by
	% \begin{enumerate}[(i)]
	%     \item $I_k \tensor X_{g_0g^2}$ if 
	% \end{enumerate}
	Moreover, the restriction $B_x$ of $B$ to $\U_x$ is nondegenerate and satisfies $\overline{B_x} = \delta B_x$ if, and only if, there is a $\D$-basis of $\U_{x}$ consisting only of elements of degree $\xi(x)$ such that the matrix representing $B_x$ is given by
	\begin{enumerate}[(i)]
		\item $I_{\kappa(x)} \tensor X_t$ if $\mu_x = +1$;
		\item $J_{\kappa(x)} \tensor X_t$ if $\mu_{x} = -1$, where $\kappa (x)$ is even and $J_{\kappa(x)} \coloneqq \begin{pmatrix}
				      0                & I_{\kappa(x)/2} \\
				      -I_{\kappa(x)/2} & 0
			      \end{pmatrix}$.
	\end{enumerate}
\end{prop}

\begin{proof}
	Let $g \coloneqq \xi(x)$. If $\xi' \from G^\#/T \to G^\#$ is another section, then there is $s \in T$ such that $\xi' (x) = g s$.
	Hence:
	\begin{align*}
		(-1)^{|\xi' (x)|} \eta(g_0 \xi' (x)^2) & = (-1)^{|gs|} \eta(g_0 g^2 s^2)                                                       \\
		                                       & = (-1)^{|g| + |s|} \sign{g_0 g^2}{s^2} \beta(g_0 g^2, s^2)\eta(g_0 g^2) \eta(s^2)     \\
		                                       & = (-1)^{|g| + |s|} (-1)^{2 |g_0 g^2| |s|} \beta(g_0 g^2, s)^2 \eta(g_0 g^2) \eta(s^2) \\
		                                       & = (-1)^{|g| + |s|} \eta(g_0 g^2) \eta(s^2)                                            \\
		                                       & = (-1)^{|g| + |s|} \eta(g_0 g^2) (-1)^{|s|} \eta (s)^2 = (-1)^{|g|} \eta(g_0 g^2),
	\end{align*}
	where we have used Equation \eqref{eq:superpolarization} twice.

	For the ``moreover'' part, it follows from Lemmas \ref{lemma:B_x-nondeg} and \ref{lemma:B_x-delta} that $B_x$ is nondegenerate and $\overline{B_x} = \delta B_x$ if, and only if, $\tilde B_x$ is nondegenerate and $B_x (u,v) = \mu_x B_x(v, u)$, for all $u,v \in V_x = \U_{\xi(x)}$.
	Then the result follows from the well-known classification of (skew-)symmetric bilinear forms over an algebraically closed field of characteristic different from $2$ and the fact that an $\FF$-basis for $\U_{\xi(x)}$ is a $\D$-basis for $\U_x$.
\end{proof}

\begin{remark}
	Even though $\mu_{x}$ does not depend on $\xi$, the element $t = g_0\xi(x)^2$ may depend on $\xi$.
\end{remark}

\begin{prop}\label{prop:pair-of-dual-components}
	Let $\delta \in \pmone$.
	Suppose $g_0 x y = T$ for $x\neq y$ and set $t \coloneqq g_0\xi(x)\xi(y) \in T$.
	Then the restriction $B_x$ of $B$ to $\U_x \oplus \U_y$ is nondegenerate and satisfies $\overline{B_x} = \delta B_x$ if, and only if, there is a $\D$-basis of $\U_x$ with all elements having degree $\xi(x)$ and a $\D$-basis of $\U_y$ with all elements having degree $\xi(y)$ such that the matrix representing $B_x$ is
	\[
		\begin{pmatrix}
			0                                                  & I_{\kappa(x)} \\
			\sign{\xi(x)}{\xi(y)} \eta(t) \delta I_{\kappa(x)} & 0
		\end{pmatrix} \tensor X_t.
	\]
\end{prop}

\begin{proof}
	Assume that $B_x$ is nondegenerate.
	Then by Lemma \ref{lemma:B_x-nondeg}, the bilinear form $\tilde B_x$ on $V_x = \U_{\xi(x)} \oplus \U_{\xi(y)}^*$ is nondegenerate, and hence, the map $\U_{\xi(x)} \to \U_{\xi(y)}$ given by $u \mapsto \tilde B_x (u, \cdot)$  is an isomorphism of vector spaces.
	Hence, we can fix a basis $\{u_1, \ldots, u_{\kappa(x)} \}$ for $\U_{\xi(x)}$ and take its dual basis $\{v_1, \ldots, v_{\kappa(x)} \}$ for $\U_{\xi(y)}$, \ie, $\tilde B_x (u_i, v_j) = \delta_{ij}$.
	If we also assume $\overline{B_x} = \delta B_x$, then by Lemma \ref{lemma:B_x-delta}, we have $\tilde B_x (v_i, u_j) = \sign{\xi(x)}{\xi(y)} \eta(t) \delta \delta_{ij}$.
	This proves the only if part.
	The converse is clear.
\end{proof}

\begin{defi}\label{def:parameter-of-(U,B)}
	Let $\U\neq 0$ be a graded $\D$-module of finite rank and $B$ be a nondegenerate homogeneous sesquilinear form with respect a superinvolution $\vphi_0$ on $\D$ and such that $\overline{B} = \delta B$ for some $\delta \in \pmone$.
	We say that the quadruple $(\eta, \kappa, g_0, \delta)$ is the \emph{inertia of $(\U, B)$}, where $\eta$ defines $\vphi_0$ by $\vphi_0(X_t) = \eta(t)X_t$ (see Subsection \ref{ssec:param-D-vphi}), $g_0 = \deg B \in G^\#$ and $\kappa(x) = \dim_\D \U_x$ for all $x \in G^\#/T$.
\end{defi}

By the results above, the quadruple $(\eta, \kappa, g_0, \delta)$ satisfies the following:

\begin{defi}\label{defi:X(D)}
	Given $\eta\from T \to \pmone$,
	$\kappa\from G^\#/T \to \ZZ_{\geq 0}$, $g_0 \in G^\#$ and $\delta \in \pmone$, we say that the quadruple $(\eta, \kappa, g_0, \delta)$ is \emph{admissible} if:
	\begin{enumerate}[(i)]
		\item $\mathrm{d}\eta = \tilde\beta$ (see  \cref{eq:tilde-beta-def,def:coboundary}); \label{item:eta-is-eta}
		\item $\kappa$ has finite support; \label{item:kappa-finite-support}
		\item $\kappa(x) = \kappa(g_0\inv x\inv)$ for all $x \in G^\#/T$; \label{item:kappa-duality}
		\item for any $x\in G^\#/T$, if $g_0 x^2 = T$ and $\mu_x = -1$, then $\kappa (x)$ is even (where \\$\mu_x\coloneqq (-1)^{|g|}\eta(g_0 g^2)\delta$ for $g\in x$, see Proposition \ref{prop:self-dual-components}). \label{item:kappa-parity}
	\end{enumerate}
	The set of all admissible quadruples will be denoted by $\mathbf{I}(\D)$ or $\mathbf{I}(T, \beta, p)$.
\end{defi}

Given a quadruple $(\eta, \kappa, g_0, \delta) \in \mathbf{I} (\D)$, we can construct a pair $(\U, B)$ such that $(\eta, \kappa, g_0, \delta)$ is its inertia.
To see that, fix a total order $\leq$ on the set $G^\#/T$.
We define $W_x \coloneqq (\FF^{\kappa(x)})^{[\xi(x)]}$, and $\U = \sum_{x \in G^\#/T} \U_x$ where $\U_x \coloneqq W_x \tensor \D$, for all $x\in G^\#/T$.
For all $x, y \in G^\#/T$ such that $g_0x y = T$ and $x \leq y$, we let $B_x$ be the $\vphi_0$-sesquilinear on $\U_x + \U_y$ represented, relative to the standard basis of $W_x + W_y$, by the matrices in Proposition \ref{prop:self-dual-components}, if $x=y$, or in Proposition \ref{prop:pair-of-dual-components}, if $x\neq y$.

\begin{thm}\label{thm:iso-(U,B)}
	Suppose $\FF$ is an algebraically closed field and $\Char \FF \neq 2$. 
	Let $\D$ be a finite dimensional graded-division superalgebra and let $\vphi_0$ be a degree preserving superinvolution on $\D$. 
	% The map assigning a triple $(\kappa, g_0, \delta) \in \mathbf{I} (\D, \vphi_0)$ for ever
	The assignment of inertia to a pair $(\U, B)$ as in Definition \ref{def:parameter-of-(U,B)} gives a bijection between the isomorphism classes of these pairs and the set $\mathbf{I} (\D)$. 
\end{thm}

\begin{proof}
	Suppose there is an isomorphism $\psi\from (\U, B) \to (\U', B')$.
	Since, in particular, $\psi$ is an isomorphism of graded $\D$-modules, both $\U$ and $\U'$ correspond to the same map $\kappa\from G^\#/T \to \ZZ_{\geq 0}$.
	Also, from the fact that $B'(\psi(u), \psi(v)) = B(u, v)$ for all $u, v\in \U$, it is clear that $\deg B' = \deg B$.
	Moreover, if $\overline{B} = \delta B$, then
	\begin{align*}
		\overline{B'} \big(\psi(u), \psi(v) \big) & = \sign{\psi(u)}{\psi(v)} \vphi_0\inv \Big( B'\big( \psi(v) , \psi(u) \big) \Big) \\
		                                          & = \sign{u}{v} \vphi_0\inv \big( B(v, u) \big)
		= \overline{B} (u, v)                                                                                                         \\
		                                          & = \delta B(u, v) = \delta B' \big( \psi(u), \psi(v) \big),
	\end{align*}
	for all $u, v \in \U$.
	Since $\psi$ is bijective, it follows that $\overline{B'} = \delta B'$.

	Conversely, suppose that $(\U, B)$ and $(\U, B')$ have the same inertia $(\kappa, g_0, \delta)$.
	To show that $(\U, B)$ and $(\U, B')$ are isomorphic, it suffices to find homogeneous $\D$-bases $\{u_1, \ldots, u_k\}$ of $\U$ and $\{u_1', \ldots, u_k'\}$ of $\U'$ such that $\deg u_i = \deg u_i'$, $1 \leq i \leq k$, and $B$ and $B'$ are represented by the same matrix.
	Indeed, in this case the $\D$-linear map $\psi\from \U \to \U'$ defined by $\psi(u_i) = u_i'$, $1 \leq i \leq k$, is degree preserving, and $B(u_i, u_j) = B(u_i', u_j')$, $1 \leq i,j \leq k$, implies $B(u, v) = B( \psi(u), \psi(v) )$ for all $u, v\in \U$.
	The existence of such bases follows from $\dim_\D (\U_x) = \dim_\D (\U_x') = \kappa(x)$, for all $x \in G^\#/T$, and Propositions \ref{prop:self-dual-components} and \ref{prop:pair-of-dual-components}.
\end{proof}

\subsection{Parametrization of \texorpdfstring{$(R, \vphi)$}{(R,phi)}}\label{subsec:param-(R-phi)}

We start with a definition to have a concise description of the graded superalgebras with superinvolution we are working on:

\begin{defi}\label{def:E(D,U,B)}
	Let $\D$ be a finite dimensional graded-division superalgebra over an algebraically closed field $\FF$, $\Char \FF \neq 2$, let $\U\neq 0$ be a graded right $\D$-module of finite rank and let $B$ be a nondegenerate homogeneous sesquilinear form on $\U$ such that $\overline{B} = \pm B$. 
	By \cref{thm:vphi-involution-iff-delta-pm-1},
	$E(\D, \U, B)$ (see \cref{def:superadjunction}) is a (finite dimensional) graded superalgebra with superinvolution. 
	If $\D$ is associated to $(T, \beta, p)$ (see \cref{ssec:grd-div-alg}) and $(\U, B)$ has inertia $(\eta, \kappa, g_0, \delta) \in \mathbf{I}(T, \beta, p)$ (see Definitions \ref{def:parameter-of-(U,B)} and \ref{defi:X(D)}), then we say that $(T, \beta, p, \eta, \kappa, g_0, \delta)$ are the parameters of the triple $(\D, \U, B)$.
\end{defi}

By \cref{thm:End-over-D,thm:vphi-iff-vphi0-and-B,thm:vphi-involution-iff-delta-pm-1}, any finite dimensional graded-simple superalgebra with superinvolution $(R, \vphi)$ is isomorphic to $E(\D, \U, B)$ for some triple $(\D, \U, B)$ as in Definition \ref{def:E(D,U,B)}.
We will now classify these graded superalgebras with superinvolution in terms of the parameters $(T, \beta, p, \eta, \kappa, g_0, \delta)$.
% Let $R \coloneqq \End_\D (\U)$ and $R' \coloneqq \End_{\D'} (\U')$, with superinvolutions $\vphi$ and $\vphi'$, respectively, which are superadjunctions with respect to nondegenerate sesquilinear forms $B$ and $B'$. 
Let $(\D, \U, B)$ and $(\D', \U', B')$ be triples as in Definition \ref{def:E(D,U,B)}, and let $(R, \vphi) \coloneqq E(\D, \U, B)$ and $(R', \vphi') \coloneqq E(\D', \U', B')$.
Consider the corresponding parameters $(T, \beta, p, \eta, \kappa, g_0, \delta)$ and $(T', \beta', p', \eta', g_0', \delta', \kappa')$.
If $(R, \vphi) \iso (R', \vphi')$, then $\D \iso \D'$ and, hence, $T = T'$, $\beta = \beta'$ and $p = p'$.
% For each triple $(T, \beta, p)$, we fix a graded-division superalgebra $\D$ with these parameters, so we may suppose $\D = \D'$.

\begin{lemma}\label{lemma:twist-same-inertia}
	Let $\psi_0\from \D \to \D'$ be a degree preserving isomorphism.
	Then $(\U',B')$ and $((\U')^{\psi_0}, \psi_0\inv \circ B')$ have the same inertia $(\eta', \kappa', g_0', \delta') \in \mathbf{I}(T, \beta, p)$.
\end{lemma}

\begin{proof}
	By Lemma \ref{lemma:twist-on-(U,B)}, $\psi_0\inv \circ B'$ is $(\psi_0\inv \circ \vphi_0' \circ \psi_0)$-sesquilinear.
	Let $X_t \in \D_t$.
	Then $\psi_0 (X_t)\in \D_t'$ and, hence, $\psi_0' \big(\psi_0 (X_t) \big) = \eta' (t) \psi_0 (X_t)$.
	It follows that $(\psi_0\inv \circ \vphi_0' \circ \psi_0) (X_t) = \eta'(t) X_t$, therefore the superinvolution $(\psi_0\inv \circ \vphi_0' \circ \psi_0)$ also corresponds to the map $\eta'\from T \to \FF^\times$.

	Since $\dim_{\D'} \U_x = \dim_\D (\U_x)^{\psi_0\inv}$, for all $x \in G^\#/T$, the graded $\D$-modules $\U$ and $\U^{\psi_0\inv}$ correspond to the same $\kappa$.
	Also, it is clear that $\deg (\psi_0 \circ B) = \deg B$. Finally, using that $\psi_0\inv \circ \vphi_0 \circ \psi_0$ and $\vphi_0$ are involutive, we have that
	\begin{align*}
		\overline{(\psi_0\inv \circ B)} (u,v) & = \sign{u}{v} (\psi_0\inv \circ \vphi_0 \circ \psi_0) \big( (\psi_0\inv \circ B)(v, u) \big)                            \\
		% &= \sign{u}{v} (\psi_0 \circ \vphi_0\inv \circ \psi_0\inv) \big( (\psi_0 \circ B)(v,u) \big) \\
		                                      & = \psi_0\inv \bigg( \sign{u}{v} \vphi_0 \big( B(v,u) \big) \bigg)                                                       \\
		                                      & = \psi_0\inv \big( \overline B (u,v) \big) = \psi_0\inv \big( \delta B (u,v) \big) = \delta (\psi_0\inv \circ B) (u,v),
	\end{align*}
	for all $u, v \in \U\even \cup \U\odd$.
\end{proof}

By Corollary \ref{cor:iso-with-actions}, $(R, \vphi) \iso (R', \vphi')$ if, and only if, $(\eta', g_0', \delta', \kappa')$ is in the orbit of $(\eta, \kappa, g_0, \delta)$ under the action of the group $(\D^\times_{\mathrm{gr}} \rtimes A ) \times G^\#$ on $\mathbf{I}(T, \beta, p)$ determined by its action on the isomorphism classes of $(\U, B)$ (see Lemma \ref{lemma:action-on-(U,B)}).
Let us compute this action explicitly.

First, applying \cref{lemma:twist-same-inertia} in the case $\D' = \D$, we have that the $A$-action on $\mathbf{I}(T, \beta, p)$ is trivial.

Let us now consider the $\D^\times_{\mathrm{gr}}$-action.
Let $t\in T$ and $0\neq d \in \D_t$ and let $(\eta', \kappa', g_0', \delta')$ be the parameters corresponding to $d \cdot (\U, B) = (\U, d B)$ (see Equation \eqref{eq:Dx_gr-action}).
To compute $\eta'$, recall that $dB$ is $(\mathrm{sInt}_d \circ \vphi_0)$-sesquilinear.
Let $s\in T$ and $c \in \D_s$.
Then
\begin{align*}
	(\mathrm{sInt}_{d} \circ \vphi_0) (c) & = \mathrm{sInt}_{d} ( \eta(s) c ) = \sign{t}{s} \eta(s) d c d\inv          \\
	                                      & = \sign{t}{s} \eta(s) \beta(t,s) c d d\inv = \tilde{\beta} (t,s) \eta(s) c
\end{align*}
and, hence, $\eta' (s) = \tilde{\beta} (t,s) \eta(s)$ for all $s\in T$.
Since the action by $d$ does not change $\U$, we have $\kappa' = \kappa$.
Clearly $g_0' = \deg (dB) = t g_0$.
It remains to compute $\delta'$.
Using Lemma \ref{lemma:bar-dB}, we have
\begin{align*}
	\overline{d B} & = (-1)^{|t|} \vphi_0(d) \overline{B} = (-1)^{|t|} \eta(t) d \delta B = (-1)^{|t|} \eta(t) \delta (d B),
\end{align*}
so $\delta' = (-1)^{t} \eta(t) \delta$.
Note that $(\eta', \kappa', g_0', \delta')$ depends only on $t$, so the $\D^\times_\mathrm{gr}$-action on $\mathbf{I}(T, \beta, p)$ factors through the action of $T \iso \D^\times_\mathrm{gr}/\FF^\times$.

Finally, we consider the $G^\#$-action.
Let $g\in G$ and let $(\eta'', \kappa'', g_0'', \delta'')$ be the parameters corresponding to $g \cdot (\U, B) = (\U^{[g]}, B^{[g]})$ (see Equation \eqref{eq:G-action}).
Since $B^{[g]}$ is $\vphi_0$-sesquilinear, $\eta'' = \eta$. By the definition of $\U^{[g]}$, $\kappa'' = g\cdot \kappa$ where $(g\cdot \kappa) (x) = \kappa(g\inv x)$ for all $x\in G^\#/T$. 
As noted in Remark \ref{rmk:deg-B^[g]}, $g_0'' = \deg B^{[g]} = g_0 g^{-2}$.
Also, for all $u, v\ \in \U\even \cup \U\odd$, and writing $u^{[g]}$ and $v^{[g]}$ for $u$ and $v$ when they are considered as elements of $\U^{[g]}$, we compute:
\begin{align*}
	\overline{B^{[g]}}(u^{[g]},v^{[g]}) & = \sign{u^{[g]}}{v^{[g]}} \vphi_0\inv \big( B^{[g]} (v^{[g]}, u^{[g]}) \big)                                              \\
	                                    & = (-1)^{(|g| + |u|) (|g| + |v|)} \vphi_0\inv \big( \sign{g}{v} B(v, u) \big)                                              \\
	                                    & = (-1)^{|g| + |g||u| + |g||v| + |u||v|} \sign{g}{v} \vphi_0\inv \big(B(v, u) \big) = (-1)^{|g| + |g||u|} \overline B(u,v) \\
	                                    & = (-1)^{|g|} \sign{g}{u} \delta B(u,v) = (-1)^{|g|} \delta B^{[g]}(u^{[g]},v^{[g]}),
\end{align*}
We conclude that $\delta'' = (-1)^{|g|} \delta$.

% We summarize this discussion in the following:

\begin{defi}\label{def:TxG-action}
	The group $T\times G^\#$ acts on $\mathbf{I}(T, \beta, p)$ by
	\begin{align}
		t \cdot (\eta, \kappa, g_0, \delta) & \coloneqq (\tilde\beta (t, \cdot) \eta, \kappa, t g_0, (-1)^{|t|} \eta(t)\delta)
		\intertext{and}
		g \cdot (\eta, \kappa, g_0, \delta) & \coloneqq (\eta, g\cdot \kappa, g_0 g^{-2}, (-1)^{|g|} \delta),
	\end{align}
	for all $t\in T$, $g\in G^\#$ and $(\eta, \kappa, g_0, \delta) \in \mathbf{I}(T, \beta, p)$.
\end{defi}

% \begin{equation}
%     t \cdot (\eta, \kappa, g_0, \delta) = (\eta \tilde \beta (t, \cdot), t g_0, (-1)^{|t|} \eta(t)\delta, \kappa).
% \end{equation}

% Finally, the $G^\#$-action is given by
% \begin{equation}
%     g \cdot (\eta, \kappa, g_0, \delta) = (\eta, g_0 g^{-2}, (-1)^{|g|} \delta, g\cdot \kappa),
% \end{equation}
% where $(g \cdot \kappa)(x) \coloneqq \kappa (g\inv x)$ for all $x \in G^\#/T$

In view of these considerations, Corollary \ref{cor:iso-with-actions} can be restated as follows:

\begin{thm}\label{thm:iso-(R-vphi)-with-parameters}
	Let $(\D, \U, B)$ and $(\D', \U', B')$ be triples as in Definition \ref{def:E(D,U,B)}, and let $(T, \beta, p, \eta, \kappa, g_0, \delta)$ and $(T', \beta', p', \eta', \kappa', g_0', \delta')$ be their parameters.
	Then $E(\D, \U, B) \iso E(\D', \U', B')$ if, and only if, $T = T'$, $\beta = \beta'$, $p = p'$, and $(\eta, \kappa, g_0, \delta)$ and $(\eta', \kappa', g_0', \delta')$ lie in the same orbit of the $T\times G^\#$-action in $\mathbf{I}(T, \beta, p)$ given in Definition \ref{def:TxG-action}. \qed
\end{thm}

% From now on, let us consider the triple $(T, \beta, p)$ fixed. 
We will now proceed to simplify our parameter set and the group acting on it, by using \cref{lemma:lemma-on-actions}. 
To this end, consider the following equivalence relation among all possible $\eta$.

\begin{defi}\label{def:equiv-eta}
	Let $\eta, \eta'\from T \to \pmone$ be maps such that $\mathrm{d}\eta = \mathrm{d}\eta' = \tilde\beta$.
	We say that $\eta$ and $\eta'$ are \emph{equivalent}, and write $\eta \sim \eta'$, if there is $t \in T$ such that $\eta' = \tilde\beta(t, \cdot) \eta$. 
\end{defi}

We partition the set $\mathbf{I}(T, \beta, p)$ according to the equivalence class of $\eta$ and refer to $\{ (\eta', \kappa, g_0, \delta) \in \mathbf{I}(T, \beta, p) \mid \eta' \sim \eta \}$ as the $\eta$-block of the partition.
Note that if $(\eta, \kappa, g_0, \delta), (\eta', \kappa', g_0', \delta') \in \mathbf{I}(T, \beta, p)$ are in the same $T\times G^\#$-orbit, then, clearly, $\eta' \sim \eta$.
In other words, the $T\times G^\#$-action on $\mathbf{I}(T, \beta, p)$ restricts to each $\eta$-block of the partition.

We wish to fix $\eta$.
Given $\eta\from T \to \pmone$ such that $\mathrm{d}\eta = \tilde\beta$,
we define
\[
	\mathbf{I}(T, \beta, p)_\eta \coloneqq \{ (\kappa, g_0, \delta) \mid (\eta, \kappa, g_0, \delta) \in \mathbf{I}(T, \beta, p)\}.
\]

It is clear from Definition \ref{def:TxG-action} that the action by $(t,g)\in T\times G^\#$ does not change $\eta$ if, and only if, $t\in \rad \tilde\beta$.
Thus, the $T\times G^\#$-action on $\mathbf{I}(T, \beta, p)$ induces an action of the subgroup $(\rad \tilde\beta) \times G^\#$ on $\mathbf{I}(T, \beta, p)_\eta$ given by
\begin{align}
	t \cdot (\kappa, g_0, \delta) & \coloneqq (\kappa, t g_0, \eta(t)\delta)
	\quad \text{and}                                                                          \\
	%\intertext{and}
	g \cdot (\kappa, g_0, \delta) & \coloneqq (g\cdot \kappa, g_0 g^{-2}, (-1)^{|g|} \delta),
\end{align}
for all $t\in \rad \tilde\beta = (\rad \beta) \cap T^+$ (recall \cref{lemma:rad-tilde-beta}), $g\in G^\#$ and $(\kappa, g_0, \delta) \in \mathbf{I}(T, \beta, p)_\eta$. 

Now we wish to fix $\delta = 1$.
Note that in every orbit we have a triple with $\delta = 1$ since the action by $(e, \bar1) \in G^\# = G \times \ZZ_2$ changes the sign of $\delta$.
We define
\[\label{eq:I-eta-plus}
	\mathbf{I}(T, \beta, p)_\eta^+ \coloneqq \{ (\kappa, g_0) \mid (\eta, \kappa, g_0, 1) \in \mathbf{I}(T, \beta, p) \}.
\]
By the definition of the $(\rad \tilde\beta) \times G^\#$-action on $\mathbf{I}(T, \beta, p)_\eta$, we see that the action of $(t,g)$ does not change $\delta$ if, and only if, $
	\eta(t) = (-1)^{|g|}.
$
Note that $\eta\restriction_{\rad \tilde\beta}$ is a group homomorphism, since $\mathrm{d}\eta = \tilde\beta = 1$ on $\rad \tilde\beta$.
Hence
\[\label{eq:mathcal-G}
	\mathcal G \coloneqq \{ (t,g) \in (\rad \tilde\beta) \times G^\# \mid \eta(t) = (-1)^{|g|} \}
\]
is a subgroup of $(\rad \tilde\beta) \times G^\#$.
Thus, the $(\rad \tilde\beta) \times G^\#$-action on $\mathbf{I}(T, \beta, p)_\eta$ induces a $\mc G$-action on $\mathbf{I}(T, \beta, p)_\eta^+$ given by
\begin{equation}\label{eq:mc-G-action}
\begin{split}
	t \cdot (\kappa, g_0) & \coloneqq (\kappa, t g_0)
	\quad \text{and} \\
	g \cdot (\kappa, g_0) & \coloneqq (g\cdot \kappa, g_0 g^{-2}),
\end{split}
\end{equation}
for all $(t, g)\in \mc G$ and $(\kappa, g_0) \in \mathbf{I}(T, \beta, p)_\eta^+$. 

\Cref{lemma:lemma-on-actions} implies the following:

\begin{prop}\label{prop:after-fixing-delta}
	Fix a map $\eta\from T \to \pmone$ such that $\mathrm{d}\eta = \tilde\beta$.
	Then the map $\iota\from \mathbf{I}(T,\beta,p)_\eta^+ \to \mathbf{I}(T,\beta,p)$ given by $\iota(\kappa, g_0) \coloneqq (\eta,\kappa, g_0, 1)$ induces a bijection between the $\mc G$-orbits in $\mathbf{I}(T,\beta,p)_\eta^+$ and the $T\times G^\#$-orbits in the $\eta$-block of $\mathbf{I}(T,\beta,p)$. \qed
\end{prop}

Therefore, the classification up to isomorphism of $G$-graded superalgebras with superinvolution that are finite dimensional and graded-simple over an algebraically closed field $\FF$, $\Char \FF \neq 2$, reduces to the classification of finite subgroups $T\subseteq G^\#$, alternating bicharacters $\beta\from T\times T \to \pmone$, equivalence classes of $\eta$ (see \cref{def:equiv-eta}) and $\mathcal G$-orbits in $\mathbf{I}(T,\beta, p)^+_\eta$.
This gives a useful simplification when we have one equivalence class of $\eta$, as it will be the case in the next chapter, where we will study gradings on superinvolution-simple associative superalgebras. 

In the case $T^- \neq \emptyset$, another simplification can be made, but it has to be done before fixing $\eta$. 
As noted in \cref{conv:pick-even-form}, we can make the sesquilinear form $B$ even, \ie, we can choose $g_0$ such that $|g_0| = \bar 0$. 
Equivalently, every $T\times G^\#$-orbit has a element $(\eta, \kappa, g_0, \delta)$ with $|g_0| = \bar 0$. 
We let 
\[
    \mathbf{I}(T, \beta, p)^{\bz} \coloneqq \{ (\eta, \kappa, g_0, \delta) \in \mathbf{I}(T, \beta, p) \, \mid \, |g_0| = \bz\}.
\]
If we wish to restrict the action to only these elements, we have to act by $T^+ \times G^\#$. 

\begin{defi}\label{def:equiv-eta-even}
	Let $\eta, \eta'\from T \to \pmone$ be maps such that $\mathrm{d}\eta = \mathrm{d}\eta' = \tilde\beta$.
	We say that $\eta$ and $\eta'$ are \emph{evenly equivalent}, and write $\eta \sim_\bz \eta'$, if there is $t \in T^+$ such that $\eta' = \tilde\beta(t, \cdot) \eta$. 
\end{defi}

We partition the set $\mathbf{I}(T, \beta, p)^{\bz}$ according to this new equivalence relation: the $\eta$-block is now $\{ (\eta', \kappa, g_0, \delta) \in \mathbf{I}(T, \beta, p)^{\bz} \mid \eta' \sim_\bz \eta \}$. 
If $(\eta, \kappa, g_0, \delta), (\eta', \kappa', g_0', \delta') \in \mathbf{I}(T, \beta, p)^{\bz}$ are in the same $T^+ \times G^\#$-orbit, then $\eta' \sim_\bz \eta$.
Hence, the $T^+\times G^\#$-action on $\mathbf{I}(T, \beta, p)^{\bz}$ restricts to each $\eta$-block of the partition. 

\begin{remark}\label{rmk:different-eta-blocks}
    It should be noted that the equivalence class of $\eta$ according to the relation $\sim_\bz$ is, in general, smaller than the equivalence class according to $\sim$. 
    This implies that graded superalgebras corresponding to points in the same $\eta$-block of $\mathbf{I}(T,\beta,p)$ may correspond to different $\eta$-blocks of $\mathbf{I}(T,\beta,p)^{\bz}$.
\end{remark}

As before, we wish to fix $\eta$ and make $\delta = 1$. 
Given $\eta\from T \to \pmone$ such that $\mathrm{d}\eta = \tilde\beta$,
we define
\[
	\mathbf{I}(T, \beta, p)^{\bz, +}_\eta \coloneqq \{ (\kappa, g_0) \mid (\eta, \kappa, g_0, 1) \in \mathbf{I}(T, \beta, p) \AND |g_0| = \bz \}.
\]
Following the same reasoning as above, we get a $\mc G$-action on $\mathbf{I}(T, \beta, p)^{\bz, +}_\eta$, where $\mc G$ is defined by \cref{eq:mathcal-G} and the action by \cref{eq:mc-G-action}. 
Then \cref{lemma:lemma-on-actions} implies:

\begin{prop}\label{prop:now-fixing-parity-of-g0}
	Suppose $T^- \neq \emptyset$ and fix a map $\eta\from T \to \pmone$ such that $\mathrm{d}\eta = \tilde\beta$.
	Then the map $\iota\from \mathbf{I}(T,\beta,p)^{\bz, +}_\eta \to \mathbf{I}(T,\beta,p)^{\bz}$ given by $\iota(\kappa, g_0) \coloneqq (\eta,\kappa, g_0, 1)$ induces a bijection between the $\mc G$-orbits in $\mathbf{I}(T,\beta,p)^{\bz, +}_\eta$ and the $T^+ \times G^\#$-orbits in the $\eta$-block of $\mathbf{I}(T,\beta,p)^{\bz}$. \qed
\end{prop}
