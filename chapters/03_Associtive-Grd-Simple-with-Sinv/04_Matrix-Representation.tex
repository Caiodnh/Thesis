\section{Matrix representation of a su\-per\--anti\--auto\-mor\-phism}

In this section, we are going to express the super-anti-automorphism (not necessarily involutive) $\vphi$ in terms of matrices with entries in $\D$.
One could do that by following Equation \eqref{eq:vphi-r-Star-is-a-superconjugation}, but we will take a different path.

As before, we suppose $\D$ is a graded division superalgebra, $\U$ is a nonzero right graded module of finite rank over $\D$, $R = \End_\D (\U)$ and $\vphi$ is a degree-preserving super-anti-automorphism on $R$.
Also, let $\vphi_0$ be a super-anti-automorphism on $\D$ and $B$ be a nondegenerate $\vphi_0$-sesquilinear form on $\U$ determinig $\vphi$ as in Theorem \ref{thm:vphi-iff-vphi0-and-B}.

\begin{defi}\label{def:matrix-representing-B}
	Given a graded basis $\{u_1, \ldots, u_k\}$ of $\U$, the \emph{matrix representing the form $B$} is defined to be $\Phi = (\Phi_{ij}) \in M_k(\D)$, where $\Phi_{ij} = B(u_i, u_j)$.
\end{defi}

From now on, let $\mc B = \{u_1, \ldots, u_k\}$ be a fixed homogeneous $\D$-basis of $\U$, following Convention \ref{conv:pick-even-basis} (\ie, if $\D$ is odd, we take $\mc B$ with only even elements).
We will use $\mc B$ to identify $R = \End_\D (\U)$ with $M_k (\D)$.
Also, we will denote $|u_i|$ simply by $|i|$ for all $i \in \{1, \ldots, k\}$.

% We will assume that the even elements precede the odd ones. 

% We will also follow Convention \ref{conv:pick-even-basis}, \ie, if $\D$ is odd, we will choose the $\D$-basis $\mc B$ to be one having only even elements.

% \begin{convention}\label{conv:pick-even-basis}
% 	If $\D$ is odd (\ie, $\D\odd \neq 0$), we will choose the $\D$-basis $\mc B$ to be one having only even elements.
% 	This is possible because, given any homogeneous $\D$-basis, we can multiply its odd elements by a nonzero homogeneous odd element in $\D$.
% \end{convention}

\begin{defi}
	Let $X = (x_{ij})$ be a matrix in $M_k (\D)$.
	We define $\vphi_0 (X)$ to be the matrix obtained by applying $\vphi_0$ in each entry, \ie, $\vphi_0 (X) \coloneqq (\vphi_0(x_{ij}))$.
	We also extend the definition \emph{supertranspose}  to matrices over $\D$ by putting $X\stransp \coloneqq \big((-1)^{(|i| + |j|) |i|} x_{ji} \big)$.
	Note that, with our choice of $\mc B$ in the case of odd $\D$, we have that $X\stransp = X\transp$, the ordinary transpose.
\end{defi}

\begin{prop}\label{prop:matrix-vphi}
	Let $\Phi$ be the matrix representing $B$.
	For every $r\in  R\even \cup R\odd$, let $X \in M_k(\D)$ be the matrix representing $r$ and let $Y \in M_k(\D)$ be the matrix representing $\vphi(r)$.
	Then, if $\D$ is even, we have that
	%
	\begin{align}
		Y & = \Phi\inv\, \vphi_0( X\stransp )\, \Phi, \addtocounter{equation}{1}\tag{\theequation}\label{eq:matrix-vphi-D-even} 
		\intertext{and, if $\D$ is odd, following Convention \ref{conv:pick-even-basis}, we have}
		Y & = \sign{B}{r}\,\Phi\inv\, \vphi_0( X\stransp )\, \Phi.\addtocounter{equation}{1}\tag{\theequation}\label{eq:matrix-vphi-D-odd}
	\end{align}
\end{prop}

\begin{proof}
	First of all, note that Equation \eqref{eq:superadjunction} is equivalent to the following:
	%
	\begin{alignat*}{2}
		\forall u_i, u_j \in \mc B, &  & B(ru_i, u_j)                                              & = \sign{r}{i} B(u_i, \vphi(r) u_j),
		\intertext{which, by the definitions of $X$, $Y$ and $\Phi$, becomes}
		\forall u_i, u_j \in \mc B, &  & \quad B\bigg(\sum_{\ell=1}^k u_\ell x_{\ell i}, u_j\bigg) & = \sign{r}{i} B\bigg(u_i, \sum_{\ell=1}^k u_\ell y_{\ell j}\bigg) \addtocounter{equation}{1}\tag{\theequation}\label{eq:superadjunction-matrix}.
	\end{alignat*}
	%

	Fix arbitrary $p,q \in \{1, \ldots, k\}$ and suppose that the $(p,q)$-entry of $X$ is a nonzero $G^\#$-homogeneous element of $\D$ and $x_{ij} = 0$ elsewhere, \ie, $X$ represents the map $r \in \End_\D(\U)$ defined by $r u_i = \delta_{iq} u_p x_{pq}$.
	By the $\FF$-linearity of Equation \eqref{eq:matrix-vphi-D-even} %and \eqref{eq:matrix-vphi-D-odd}
	, it suffices to consider such $X$.
	Note that $|r| = |u_p| + |x_{pq}| - |u_q| = |p| + |q| + |x_{pq}|$.
	Then, on the one hand,
	%
	\begin{align*}
		B\bigg(\sum_{\ell=1}^k u_\ell x_{\ell i}, u_j\bigg) = B (u_p x_{pi}, u_j) & = (-1)^{ (|B| + |p|) |x_{pi}|} \vphi_0(x_{pi}) B(u_p, u_j)
		\\&= (-1)^{ (|B| + |p|) |x_{pi}|} \vphi_0(x_{pi}) \Phi_{pj},
		%\end{align*}
		%
		\intertext{which is only nonzero if $i = q$. On the other hand,}
		%
		%\begin{align*}
		\sign{r}{i} B\bigg(u_i, \sum_{\ell=1}^k u_\ell y_{\ell j}\bigg)
		                                                                          & = \sign{r}{i} \sum_{\ell=1}^k B(u_i, u_\ell) y_{\ell j}                                                                                           \\
		                                                                          & = (-1)^{ (|p| + |q| + |x_{pq}|) |i| } \sum_{\ell=1}^k \Phi_{i \ell} y_{\ell j}.
		%\end{align*}
		%
		\intertext{Therefore, Equation \eqref{eq:superadjunction-matrix} is equivalent to, for all $i,j \in \{1, \ldots, k\}$,}
		%
		%\begin{align*}
		(-1)^{ (|B| + |p|) |x_{pi}|} \vphi_0(x_{pi}) \Phi_{pj} %&= \sign{r}{i} \sum_{\ell=1}^k \Phi_{i \ell} y_{\ell j}\\
		                                                                          & = (-1)^{ (|p| + |q| + |x_{pq}|) |i| } \sum_{\ell=1}^k \Phi_{i \ell} y_{\ell j}. \addtocounter{equation}{1}\tag{\theequation}\label{eq:expression}
	\end{align*}
	%

	If $\D$ is even, then $|x_{pi}| = \bar 0$ and this equation reduces to
	\[              \vphi_0(x_{pi}) \Phi_{pj} = (-1)^{ (|p| + |q|) |i| } \sum_{\ell=1}^k \Phi_{i \ell} y_{\ell j}
	\]
	or, equivalently,
	\[
		\sum_{\ell=1}^k \Phi_{i \ell} y_{\ell j} = (-1)^{ (|p| + |q|) |i| } \vphi_0(x_{pi}) \Phi_{pj}.
	\]
	The left-hand side is the $(i,j)$-entry of $\Phi\, Y$.
	The right-hand side is only nonzero if $i = q$, so it can be rewritten as
	$(-1)^{ (|p| + |i|) |i| } \vphi_0(x_{pi}) \Phi_{pj}$.
	Recalling our choice of $X$, this is equal to $\sum_{\ell =1}^k (-1)^{ (|\ell| + |i|) |i| } \vphi_0(x_{\ell i}) \Phi_{\ell j}$, since $x_{\ell i}$ is only nonzero if $\ell = p$.
	Hence the right-hand side is the $(i,j)$-entry of $\vphi_0 (X\stransp) \Phi$, and Equation \eqref{eq:matrix-vphi-D-even} follows.

	If $\D$ is odd, by our choice of basis, Equation \eqref{eq:expression} reduces to
	\[
		(-1)^{|B| |x_{pi}|} \vphi_0(x_{pi}) \Phi_{pj} =  \sum_{\ell=1}^k \Phi_{i \ell} y_{\ell j},
	\]
	which, by the same reasoning as above, implies $(-1)^{|B||r|} \vphi_0(X\stransp) \Phi = \Phi Y$. 
\end{proof}

\begin{prop}\label{prop:vphi-R-simple-D-simple}
	The superalgebra $R$ is $\vphi$-simple if, and only if, the superalgebra $\D$ is $\vphi_0$-simple. 
\end{prop}

\begin{proof}
	Pick a homogeneous $\D$-basis for $\U$ following Convention \ref{conv:pick-even-basis} and use it to identify $R$ with $M_k(\D) = M_k(\FF) \tensor \D$.
% 	By \cref{conv:pick-even-form}, we may assume that $B$ is even and then, 
	By Proposition \ref{prop:matrix-vphi}, for every $X \in M_k(\D)\even \cup M_k(\D)\odd$, we have
	$\vphi(X) = \sign{B}{X}\Phi\inv \vphi_0(X\stransp) \Phi$, where $\Phi \in M_k(\D)$ is the matrix representing $B$. 

    It was proved in \cref{prop:simple-R-D-super} that the superideals of $M_k(\D) = M_k(\FF) \tensor \D$ are precisely the sets of the form $M_k(I) = M_k(\FF) \tensor I$, where $I$ an superideal of $\D$. 
    We are going to show that $M_k(I)$ is $\vphi$-invariant if, and only if, $I$ is $\vphi_0$-invariant. 
    
	Suppose $I$ is $\vphi_0$-invariant.
	Then if $X \in M_k(\D)\even \cup M_k(\D)\odd$, it is clear that $\vphi_0 (X\stransp)$ is also in $M_k(I)$.
	It follows that $\vphi(X) = \sign{B}{X} \Phi\inv \vphi_0(X\stransp) \Phi \in M_k(I)$ since $M_k(I)$ is an ideal.
	Conversely, suppose $M_k(I)$ is $\vphi$-invariant.
	Take $d \in I\even \cup I\odd$ and define $X \coloneqq E_{11} \tensor d \in M_k(I)\even \cup M_k(I)\odd$.
	Then $E_{11} \tensor \vphi_0(d) = \vphi_0(X\stransp) = \sign{B}{X} \Phi\, \vphi(X)\, \Phi\inv \in M_k(I)$, which shows that $\vphi_0(d) \in I$.
\end{proof}