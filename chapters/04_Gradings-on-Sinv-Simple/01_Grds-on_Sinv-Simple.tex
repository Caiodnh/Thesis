\section{Graded-superinvolution-simple superalgebras}\label{grd-sinv-simple}

In \cite{racine}, finite dimensional superinvolution-simple superalgebras are classified over any field $\FF$ with $\Char \FF \neq 2$. 
In this section we adapt some of the results there to classify the finite dimensional graded-superinvolution-simple superalgebras in the case $\FF$ is algebraically closed. 
Nevertheless, $\FF$ con be any field for the results for some of the results below. 

% The next definition and lemma give us a class of examples of graded-superinvolution-simple superalgebras that are not simple as graded superalgebras.
% In fact, Proposition \ref{prop:only-SxSsop-is-simple} will tell us that these are the only examples. 

\begin{defi}\label{def:SxSsop}
	Let $S$ be a $G$-graded superalgebra and consider the $G$-graded superalgebra $S \times S\sop$ with the homogeneous component of degree $g\in G$ being $S_g \times S\sop_g$. 
	We define the \emph{exchange superinvolution} on $S \times S\sop$ to be the map $\vphi\from S \times S\sop \to S \times S\sop$ given by $\varphi (s_1, \bar s_2) = (s_2, \bar s_1)$ (recall from \cref{ssec:superdual} that $\bar s$ denotes the element $s \in S$ seen as an element of $S\sop$). 
\end{defi}

We will now give two examples where $G$ is trivial, but that will play a major role in \cref{sec:the-nonsimple-case}.

\begin{ex}\label{ex:FxF-iso-FZ2}
	The simplest possible example is to take $S = \FF$, with superalgebra structure. 
	If $\Char \FF \neq 2$, then $S\times S\sop = \FF [\zeta]$ where $\zeta = (1, -1)$ and the exchange superinvolution is given by $\vphi(1) = 1$ and $\vphi(\zeta) = -\zeta$.
	Note that $\FF [\zeta] \iso \FF\ZZ_2$ with the trivial superalgebra structure. 
\end{ex}

\begin{ex}\label{ex:FZ2xFZ2sop-iso-FZ4}
	Consider $S = Q(1)$, so $S\even = \FF 1$ and $S\odd = \FF u$ where $u^2 =1$.
	Note that $S$ is isomorphic to $\FF\ZZ_2$, but this time with the superalgebra structure given by its natural $\ZZ_2$-grading. 
	% we are the group algebra $S = \FF \langle u \rangle$, where $u$ has order $2$, as a superalgebra by declaring  (in other words, $S \iso Q(1)$).
	% We claim that $S\times S\sop$ is isomorphic to $\FF\langle \omega \rangle$, where $\omega$ has order $4$.
	If $\Char \FF \neq 2$, we claim that $R \coloneqq S\times S\sop$ is isomorphic to $F\ZZ_4$.
	Indeed, the element $\omega \coloneqq (u, \bar u) \in S\times S\sop$ has order $4$ and generates $S\times S\sop$: $\omega^2 = (1, - \bar 1)$, $\omega^3 = (u, - \bar u)$ and $\omega^4 = (1, 1)$.
	Hence $R\even = \FF1 \oplus \FF \omega^2$ and $R\odd = \FF \omega \oplus \FF \omega^3$.
	Also, the exchange superinvolution on $R$ is given by $\vphi(1) = 1$, $\vphi(\omega) = \omega$, $\vphi(\omega^2) = -\omega^2$ and $\vphi(\omega^3) = -\omega^3$.
\end{ex}

When those graded superalgebras with superinvolution are graded-superinvolution-simple and, in this case, when they are isomorphic.

\begin{lemma}\label{lemma:SxSsop-simple}
	Let $S$ be a graded superalgebra and consider the exchange superinvolution on $S \times S\sop$. 
	Then $S \times S\sop$ is graded-superinvolution-simple if, and only if, $S$ is graded-simple. 
\end{lemma}

\begin{proof}
    Suppose $S \times S\sop$ is graded-superinvolution-simple and let $I \subseteq S$ be a graded ideal. 
    We have that $I \times I\sop$ is a superinvolution-invariant graded superideal in $S \times S\sop$, and hence either $I \times I\sop = 0$ or $I \times I\sop = S \times S\sop$. 
    In the first case $I = 0$ and in the second $I = S$, hence $S$ is graded-simple. 
    
    Conversely, suppose $S$ is graded-simple. 
    It is clear that $S\sop$ is also graded-simple and, hence, by a standard argument, the graded superideals of $S \times S\sop$ are $0$, $\{ 0 \} \times S\sop$, $S \times \{ 0 \}$ and $S\times S\sop$. 
    Among those, only $0$ and $S\times S\sop$ are superinvolution-invariant, concluding the proof. 
\end{proof}

% \begin{lemma}\label{lemma:ideals-in-SxSsop}
% 	Let $S$ be a graded unital superalgebra and let $\vphi$ be the exchange superinvolution on $R \coloneqq S \times S\sop$. 
% 	The $\vphi$-invariant graded superideals of $(R, \vphi)$ are precisely the subsets of the form $I \times I\sop$ where $I$ is a superideal of $S$. 
% \end{lemma}

% \begin{proof}
% 	Let $I$ be a graded superideal of $S$ and consider $J \coloneqq I \times I\sop$.
% 	Clearly, $J$ is a $\vphi$-invariant graded subsuperspace of $S \times S\sop$.
% 	Suppose $x, y \in I\even \cup I\odd$ and let $r = (s_1, \bar s_2) \in R\even \cup R\odd$.
% 	Then $r\, (x, \bar y) = (s_1 x, \sign{s_2}{y}\, \overline{y s_2}) \in J$ and $(x, \bar y)\, r = (x s_1, \sign{s_2}{y}\, \overline{s_2 y}) \in J$, so $J$ is, indeed, a superideal.

% 	Now let $J$ be any graded $\vphi$-invariant superideal of $S\times S\sop$ and let $I \coloneqq (1,0)\, J$, which we can regard as a subspace $S \iso S \times \{ 0 \}$.
% 	First, note that $I$ is a graded superideal of $S$. 
% 	Since $J$ is a $\vphi$-invariant graded superideal and $I \subseteq J$, we have that $I + \vphi(I) = I \times I\sop \subseteq J$.
% 	If $(x, \bar y) \in J$, then, on the one hand, $x \in I$, and, on the other hand, $\vphi(x, \bar y) = (y, \bar x) \in J$, so $y \in I$.
% 	Therefore $(x, \bar y)\in I \times I\sop$, concluding the proof.
% \end{proof}

% \begin{cor}\label{cor:SxSsop-simple-iff-S-simple}
%     Under the conditions of \cref{lemma:ideals-in-SxSsop}, $S\times S\sop$ is graded-superinvolution-simple if, and only if, $S$ is graded-simple. \qed
% \end{cor}

\begin{lemma}\label{lemma:iso-SxSsop}
    Let $S_1$ and $S_2$ be graded-simple superalgebras. 
    Then $S_1\times S_1\sop \iso S_2\times S_2\sop$ as graded superalgebras with superinvolution if, and only if, $S_1 \iso S_2$ or $S_1 \iso S_2\sop$ as graded superalgebras.
\end{lemma}

\begin{proof}
    Let $\psi\from S_1\times S_1\sop \to S_2\times S_2\sop$ be an isomorphism of graded superalgebras with superinvolution. 
    Since, the only nonzero proper grade superideals of $S_i \times S_i\sop$ are $\{ 0 \} \times S_i\sop$ and $S_i\times S_i\sop$, $i = 1,2$, we have that $\psi(S_1\times \{ 0 \}) = S_1\times \{ 0 \}$ or $\psi(S_2\times \{ 0 \}) = \{ 0 \} \times S_2\sop$.  
    
    For the converse, we can suppose $S_1 \iso S_2$ by relabeling $S_2$ with $S_2\sop$ if necessary.
    If $\zeta \from S_1 \to S_2$ is an isomorphism of graded superalgebras, it is clear that $\psi\from S_1\times S_1\sop \to S_2\times S_2\sop$ given by $\psi (s, \overline{s'}) \coloneqq (\zeta(s), \overline{\zeta(s')})$, for all $s,s' \in S_1$, is an isomorphism of graded superalgebras with superinvolution.
\end{proof}

\begin{prop}\label{prop:only-SxSsop-is-simple}
	Let $(R, \vphi)$ be a 
	graded superalgebra with superinvolution. 
	Then $(R, \vphi)$ is 
	graded-superinvolution-simple if, and only if, either $R$ is a graded-simple or $(R, \vphi)$ is isomorphic to $S\times S\sop$ with the exchange superinvolution, for some graded-simple superalgebra $S$.
\end{prop}

\begin{proof}
	Suppose $(R, \vphi)$ is 
	graded-superinvolution-simple but $R$ is not graded-simple. 
	Let $0 \neq I \subsetneq R$ be a graded superideal.
	Note that $\vphi(I)$ is also a graded superideal, hence $I \cap \vphi(I)$ and $I + \vphi (I)$ are $\vphi$-invariant graded superideals. 
	Since $I \cap \vphi(I) \subseteq I \neq R$, we have $I \cap \vphi(I) = 0$, so we can write $I + \vphi (I) = I \oplus \vphi (I)$. 
	Since $0 \neq I \subseteq I \oplus \vphi (I)$, we conclude that $R = I \oplus \vphi (I)$.
	Clearly, this implies that $(R, \vphi)$ is isomorphic to $I \times I\sop$ with exchange superinvolution. 
	By \cref{lemma:SxSsop-simple}, $I$ must be simple as a graded superalgebra. 
	
	The converse is obvious if $R$ is graded-simple, and follows from \cref{lemma:SxSsop-simple} in the other case.
\end{proof}
 
It is sometimes convenient to replace $S\sop$ by an isomorphic graded superalgebra. 
Explicitly, suppose $\psi\from S \to S'$ is a super-anti-isomorphism of graded superalgebras. 
Then $S\times S\sop$ with the exchange superinvolution is isomorphic to $S\times S'$ endowed with the superinvolution $(s_1, s_2) \mapsto (\psi\inv (s_2), \psi (s_1))$. 

\begin{defi}\label{defi:superdual-exchange}
    Let $\D$ be a graded-division superalgebra and $\U$ be a graded right $\D$-supermodule of finite rank. 
    Recall that $\U\Star \coloneqq \Hom_\D(\U, \D)$ (\cref{def:superdual-supermodule}) is a graded right $\D\sop$-module, and that the map $\End_\D (\U) \to \End_{\D\sop} (\U\Star)$ given by $L \mapsto L\Star$ (\cref{defi:superdual-map}) is a super-anti-isomorphism whose inverse is $L \mapsto {}\Star L$ (\cref{prop:dual-super-anti-iso}). 
    We define $\Eex (\D, \U)$ to be the graded superalgebra $\End_\D (\U) \times \End_{\D\sop} (\U\Star)$ endowed with the superivolution $(L_1, L_2) \mapsto ({}\Star L_2, L_1\Star)$.
\end{defi}

% Let $\D$ be a graded-division superalgebra, let $\U$ be a graded right $\D$-supermodule of finite rank and set $S \coloneqq \End_\D (\U)$. 
%     Recall that $\U\Star \coloneqq \Hom_\D(\U, \D)$ (\cref{def:superdual-supermodule}) is a graded right $\D\sop$-module, and that the map $\psi\from \End_\D (\U) \to \End_{\D\sop} (\U\Star)$ given by $\psi(L) = L\Star$ (\cref{defi:superdual-map}) is a super-anti-isomorphism (\cref{prop:dual-super-anti-iso}). 
%     Hence $S\times S\sop$ with exchange superinvolution is isomorphic to $\End_\D (\U) \times \End_{\D\sop} (\U\Star)$ with superivolution $\vphi$ given by $\vphi(L_1, L_2) = (\psi\inv(L_2), \psi(L_1))$. 

Combining \cref{prop:only-SxSsop-is-simple,thm:End-over-D,thm:vphi-involution-iff-delta-pm-1}, we have:

\begin{cor}\label{cor:SxSsop-with-dcc}
    Let $(R, \vphi)$ be a graded-superalgebra with superinvolution and suppose $R$ satisfies the \dcc on graded left superideals. 
    Then $(R, \vphi)$ is graded-superinvolution-simple if, and only if, there exists a graded-division superalgebra $\D$ and graded right $\D$-supermodule $\U$, of finite rank, such that either
    $(R, \vphi) \iso \Eex (\D, \U)$ or $(R, \vphi) \iso E(\D, \U, B)$, for some nondegenerate sesquilinear form $B$ on $\U$ with $\overline{B} = \pm B$. \qed
\end{cor}

Let us now assume $\FF$ is algebraically closed and $\Char \FF \neq 2$. 
We are in a position to classify up to isomorphism the finite dimensional graded-superinvolution-simple  superalgebras over $\FF$. 
The ones that are graded-simple are classified in Theorem \ref{thm:iso-(R,vphi)-with-parameters}. 
The classification of the ones that are not graded-simple is a consequence of \cref{thm:iso-D-even,thm:iso-D-odd,lemma:iso-SxSsop} and the description of parameters for $\D\sop$ and $\U\Star$ in \cref{ssec:superdual}:

\begin{thm}\label{thm:iso-D-even-ExEsop}
	Let $(\D, \U)$ and $(\D', \U')$ be pairs as in Definition \ref{def:E(D,U)-super}, with both $\D$ and $\D'$ even. 
	Let $(T, \beta, \kappa_\bz, \kappa_\bo)$ and $(T', \beta', \kappa_\bz', \kappa_\bo')$ be the parameters of $(\D, \U)$ and $(\D', \U')$, respectively. 
	Then $\Eex(\D, \U) \iso \Eex(\D', \U')$ only if, $T=T'$ and
	\begin{enumerate}[(i)]
	    \item $\beta'=\beta$ and there is $g\in G$ such that either $g \cdot \kappa_{\bar 0}'=\kappa_{\bar 0}$ and $g \cdot \kappa_{\bar 1}'=\kappa_{\bar 1}$, or $g \cdot \kappa_{\bar 0}'=\kappa_{\bar 1}$ and $g \cdot \kappa_{\bar 1}'=\kappa_{\bar 0}$;
	    
	    or
	    \item $\beta'=\beta\inv$ and there is $g\in G$ such that either $g \cdot \kappa_{\bar 0}'=\kappa_{\bar 0}\Star$ and $g \cdot \kappa_{\bar 1}'=\kappa_{\bar 1}\Star$, or $g \cdot \kappa_{\bar 0}'=\kappa_{\bar 1}\Star$ and $g \cdot \kappa_{\bar 1}'=\kappa_{\bar 0}\Star$. \qed
	\end{enumerate}
\end{thm}

\begin{thm}\label{thm:iso-D-odd-ExEsop}
    Let $(\D, \U)$ and $(\D', \U')$ be pairs as in Definition \ref{def:E(D,U)-super}, with both $\D$ and $\D'$ odd. 
    Let $(T, \beta, p, \kappa)$ and $(T', \beta', p', \kappa')$ be the parameters of $(\D, \U)$ and $(\D', \U')$, respectively. 
	Then $\Eex(\D, \U) \iso \Eex(\D', \U')$ if, and only if, $T=T'$, $p = p'$ and
	\begin{enumerate}[(i)]
	    \item $\beta'=\beta$ and there is $g\in G$ such that $\kappa' = g \cdot \kappa$;
	    
	    or
	    \item $\beta'=\beta\inv$ and there is $g\in G$ such that $\kappa' = g \cdot \kappa\Star$. \qed
	\end{enumerate}
\end{thm}

\section[Finite dimensional superinvolution-simple superalgebras]{Finite dimensional superinvolution-simple superalgebras}

In this section we apply the classification results of \cref{sec:classification-grd-simple-with-sinv,grd-sinv-simple} to the case $G$ is trivial to get a classification of superinvolution-simple superalgebras, which is already known. 

We start with a general result about the center of a superalgebra with superinvolution: 

\begin{lemma}
	Let $(R, \vphi)$ be a superalgebra with super-anti-automorphism.
	Then $Z(R)$ is $\vphi$-invariant.
\end{lemma}

\begin{proof}
	By \cref{lemma:center-is-graded}, with $G = \ZZ_2$, we have $Z(R) = Z(R)\even \oplus Z(R)\odd$, so it is sufficient to show that if $c \in Z(R)\even \cup Z(R)\odd$, then $\vphi(c) \in Z(R)$. 
	Let $r \in R\even \cup R\odd$.
	Since $c\vphi\inv (r) = \vphi\inv (r)c$, we can apply $\vphi$ on both sides and get $\sign{c}{r} r \vphi(c) = \sign{c}{r} \vphi(c) r$ and, hence, $r \vphi(c) = \vphi(c) r$.
\end{proof}

The following result was achieved by similar methods in \cite[Theorem 8.1]{Sh98}, for example. 
See also https://doi.org/10.1016/j.jalgebra.2007.10.044, Theorem 28.

\begin{cor}\label{cor:Q-no-sinv-center}
	If $\Char \FF \neq 2$, the associative superalgebra $Q(n)$ does not admit a superinvolution.
\end{cor}

\begin{proof}
	The center of $Q(n)$ is isomorphic to $\FF1 \oplus \FF u$, where $u$ is an odd element with $u^2 = 1$.
	Let $\vphi$ be a super-anti-automorphism on $Q(n)$.
	Since $u$ is odd and central, $\vphi(u)$ is odd and central.
	Hence there is $\lambda \in \FF$ such that $\vphi(u) = \lambda u$.
	Using that $u^2 = 1$, we have $1 = \vphi(1) = \vphi(u^2) = - \vphi(u)^2 = - \lambda^2$.
	But then $\vphi^2 (u) = \lambda^2 u = -u \neq u$, hence $\vphi^2 \neq \id$.
\end{proof}

\begin{remark}
    \Cref{cor:Q-no-sinv-center} could also be deduced from \cref{lemma:sinv-type-M} assuming $\FF$ is algebraically closed. 
\end{remark}

It follows that, in the case $\FF$ is algebraically closed, we can divide the superinvolution-simple superalgebras in 3 types:

\begin{defi}
    Let $(R, \vphi)$ be a finite dimensional superinvolution-simple superalgebra. 
    \begin{enumerate}[(i)]
        \item If $R$ is of type $M$, we say that $(R, \vphi)$ is of type $M$;
        \item If $(R, \vphi) \iso M(m,n) \times M(m,n)\sop$, for some $m,n \geq 0$, we say that $(R, \vphi)$ is of type $M\times M\sop$;
        \item If $R \iso Q(n) \times Q(n)\sop$, for some $n \geq 0$, we say that $(R, \vphi)$ is of type $Q \times Q\sop$.
    \end{enumerate}
\end{defi}

We can distinguish these types using the center:

\begin{prop}\label{prop:types-of-SA-via-center}
	Let $(R, \vphi)$ be a superalgebra with superinvolution.
	\begin{enumerate}[(i)]
		\item If $(R, \vphi)$ is of type $M$, then $(Z(R), \vphi) \iso (\FF, \id)$;\label{item:F-id}
		\item If $(R, \vphi)$ is of type $M\times M\sop$, then $(Z(R), \vphi)$ is isomorphic to the superalgebra with superinvolution in Example \ref{ex:FxF-iso-FZ2};\label{item:FZ2-exchg}
		\item If $(R, \vphi)$ is of type $Q\times Q\sop$, then $(Z(R), \vphi)$ is isomorphic to the superalgebra with superinvolution in Example \ref{ex:FZ2xFZ2sop-iso-FZ4}.\label{item:FZ4-exchg}
	\end{enumerate}
\end{prop}

\begin{proof}
	Item \eqref{item:F-id} follows from the fact that $Z(M_{m+n}(\FF)) \iso \FF$.
	It is easy to check that $Z(S\times S\sop) = Z(S) \times Z(S)\sop$ for any superalgebra $S$, so items \eqref{item:FZ2-exchg} and \eqref{item:FZ4-exchg} follow from $Z(M_{m+n}(\FF)) \iso \FF$ and $Z(Q(n)) \iso Q(1)$.
\end{proof}

The classification of the superalgebras with superinvolution of types $M\times M\sop$ and $Q\times Q\sop$ is easier. 
The following result is valid over any field.

\begin{prop}
    Let $m,m',n,n' \geq 0$. % such that $m \geq n$ and $m' \geq n'$. 
    Then 
    \begin{enumerate}[(i)]
        \item $M(m,n) \times M(m,n)\sop \iso M(m',n') \times M(m',n')\sop$ if, and only if, either $m = m'$ and $n = n'$, or $m = n'$ and $n = m'$; 
        %
        \item $Q(n) \times Q(n)\sop \iso Q(n') \times Q(n')\sop$ if, and only if, either $m = m'$ and $n = n'$.
    \end{enumerate}
\end{prop}

\begin{proof}
    Let $S_1$ and $S_2$ be simple superalgebras. 
    From \cref{lemma:iso-SxSsop}, $S_1\times S_2\sop \iso S_1\times S_2\sop$ if, and only if, $S_1 \iso S_2$ or $S_1 \iso S_2\sop$. 
    In our case, we claim that  $S_1 \iso S_2$ or $S_1 \iso S_2\sop$ if, and only if, $S_1 \iso S_2$. 
    
    Indeed, if $S_1$ and $S_2$ are of type $M$, then then $S_1 \iso S_2\sop$ imply $S_1 \iso S_2$ since $S_2\sop \iso S_2$ via supertransposition. 
    If $S_1$ and $S_2$ are of type $Q$, then $S_1 \iso S_2\sop$ imply that $S_1 \iso S_2$ by dimension count. 
    
    The isomorphism condition follows from \cref{thm:fd-simple-SA}.
\end{proof}
    % and the fact that $M(m,n) \iso M(m,n)\sop$ and $Q(n) \iso Q(n)\sop$. 
    % To see this last assertion, let $\psi\from M(m,n) \to M(m,n)$ be defined by 
    % \[
    %     \begin{pmatrix}
    %         a & b\\
    %         c & d
    %     \end{pmatrix}
    %     \mapsto
    %     \begin{pmatrix}
    %         a\transp & ic\transp\\
    %         ib\transp & d\transp
    %     \end{pmatrix},
    % \]
    % where $i\in \FF$ is a square root of $-1$. 
    % It is straightforward that $\psi$ is a super-anti-automorphism of $M(m,n)$ and that, if $m=n$, it restricts to a super-anti-automorphism of $Q(n)$. 

It remains to classify the superinvolution-simple superalgebras of type $M$. 
It should be noted that $M(m,n)$ does not admit a superinvolution for every values of $m$ and $n$. 
Also, $M(m,n)$ endowed with different superinvolutions may lead to non-isomorphic superalgebras with superinvolution. 
% making  make $M(m,n)$ a superalgebra with superinvolution two superinvolutions on $M(m,n)$ making it non-isomorphic.

\begin{defi}\label{defi:M(m-n-g_0)}
    Let $m,n \in \ZZ_{\geq 0}$ not both zero and let $g_0 \in \ZZ_2$. 
    If $g_0 = \bar 0$ and $n$ is even, set 
    \[
        \sbox0{$\begin{matrix} 0& I_{n/2}\\ -I_{n/2}& 0\end{matrix}$}
        \Phi \coloneqq \left(\begin{array}{c|c}
                    I_m & 0\\
                    \hline
                    0 & \usebox{0}
                \end{array}\right).
    \]
    If $g_0 = \bar 1$ and $m = n$, let
    \[
        \Phi \coloneqq \left(\begin{array}{c|c}
                    0 & I_n\\
                    \hline 
                    I_n & 0
                \end{array}\right).
    \]
    It is straightforward to see that $\vphi(X) \coloneqq \Phi X\stransp \Phi\inv$ defines a superinvolution on $M(m,n)$. 
    We will denote the superalgebra with superinvolution $(M(m,n), \vphi)$ by $M(m,n, g_0)$. 
\end{defi}

We note that the superalgebras with superinvolution $M(m,n,g_0)$ are the ones used in the Introduction to define orthosymplectic Lie superalgebras (series $B$, $C$ and $D$) and the periplectic superalgebra (series $P$).

\begin{prop}\label{prop:iso-M-with-vphi}
    Every superalgebra with superinvolution of type $M$ is isomorphic to $M(m,n, g_0)$ for some $m,n \geq 0$ and $g_0 \in \ZZ_2$ as in \cref{defi:M(m-n-g_0)}. 
    Moreover, $M(m,n, g_0) \iso M(m',n', g_0')$ if, and only if $m = m'$, $n = n'$ and $g_0 = g_0'$.
\end{prop}

\begin{proof}
    This is precisely the case o graded-simple superalgebras, of \cref{chap:grd-simple-assc}, but with trivial $G$ (and, hence, with $G^\# = \ZZ_1$). 
    By Theorem \ref{thm:iso-(R,vphi)-with-parameters}, those graded-superalgebras are parametrized by $(T, \beta, p, \eta, \kappa, g_0, \delta)$ where $(T, \beta, p)$ are associated to a graded-division superalgebra (see \cref{ssec:param-D-vphi}), and $(\eta, \kappa, g_0, \delta) \in \mathbf{I}(\D)$ (see \cref{defi:X(D)}). 
    
    Superalgebras of type $M$ are of the form $\End_\FF (U)$ for a finite dimensional superspace $U$, so we have $\D = \FF$ and, hence, $T$ is the trivial group and $\beta$, $p$ and $\eta$ are trivial maps. 
    Since we have only one possible $\eta$, we have only one equivalence class of $\eta$ and, therefore, our parametrization reduces to elements of $\mathbf{I}(\FF)_\eta^+$ (see \cref{prop:after-fixing-delta}), \ie, we choose $\delta = 1$ and the only parameters left are $\kappa\from G^\#/T = \ZZ_2 \to \ZZ_{\geq 0}$ and $g_0 \in G^\# = \ZZ_2 = \ZZ_2$. 
    
    The isomorphism classes then become in bijection with orbits by $\mc G$-action (see ??) on $\mathbf{I}(\FF)_\eta^+$. 
    In the present case, it is clear that $\mc G$ is trivial, hence the isomorphism classes are in bijection with points in $\mathbf{I}(\FF)_\eta^+$. 
    Let us describe these points and find a representative for each isomorphism class. 
    
    The map $\kappa\from G^\#/T \iso \ZZ_2 \to \ZZ_{\geq 0}$ is determined by the numbers $m \coloneqq \kappa(\bar 0)$ and $n \coloneqq \kappa(\bar 1)$, so our parametrization reduces to triples $(m,n,g_0)$ satisfying some conditions that come from \cref{defi:X(D)}. 
    The only of those conditions  that are not automatically satisfied are \eqref{item:kappa-duality} and \eqref{item:kappa-parity}. 
    Condition \eqref{item:kappa-duality} is tautological if $g_0 = \bar 0$, and equivalent to $m=n$ if $g_0 = \bar 1$. 
    Condition \eqref{item:kappa-parity} simplifies to if $g_0 = \bar 0$ and $g = \bar 1$, then $n = \kappa (g)$ is even. 
    In other words, conditions \eqref{item:kappa-duality} and \eqref{item:kappa-parity} become equivalent to $n$ is even if $g_0 = \bar 0$, and $m = n$ if $g_0= \bar 1$. 
    
    A representative for each point can be found by using the matrices in \cref{prop:self-dual-components,prop:pair-of-dual-components}. 
\end{proof}

% \begin{prop}
%     Let $(R, \vphi)$ be a superalgebra with superinvolution that is simple as an superalgebra. 
%     Then $R$ is isomorphic to one of the superalgebras of the introduction.
% \end{prop}

% \begin{proof}
%     This is a consequence of Theorem \ref{thm:iso-(R,vphi)-with-parameters}. 
% \end{proof}

% \begin{defi}
%     Let $R$ be a superalgebra and let $\vphi\from R\to R$ be a superinvolution. 
%     We say that $\vphi$ is of the \emph{first kind} if it fixes all the elements of $Z(R)$. 
%     Otherwise, we say that $\vphi$ is of the \emph{second kind}.
% \end{defi}
