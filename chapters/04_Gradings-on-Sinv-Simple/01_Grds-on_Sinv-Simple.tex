
\section{Superinvolution-simple superalgebras}


% In this section we will discuss some general results about superalgebras with superinvolution with no nontrivial ideals that are invariant under the superinvolution. 
% Until Corollary ??, we present some concepts and results from \cite{racine}. %, and gradings other than the canonical one are not considered. 


% \begin{itemize}
%     \item[\done] Need to define superinvolution-simple superalgebra. 
%     Reference to nonexisting chapter about $\widehat G$-action.

%     \item[\done] Define exchange superinvolution on $S\times S\sop$; lemma proving it is $\vphi$-simple iff $S$ is simple.

%     \item[\done] Proposition stating that $\vphi$ simple are either $S$ or $S\times S\sop$ for simple $S$.

%     \item[\done] Example: $\FF\ZZ_2 \times \FF\ZZ_2\sop \iso \FF\ZZ_4$ (true for any field, even when $\FF\ZZ_2\sop \not \iso \FF\ZZ_2$).

%     \item Discuss the center: it is always graded, $\vphi$ always restricts to it, the center of $R$ and $\D$ are the same with same superinvolution.

%     \item If $R$ is super-involution simple and graded simple, then $\D$ is superivolution simple. 
%     Moreover, with the center and its canonical $\ZZ_2$-grading we can say $\D$ has the type ($M$, $M\times M$, $Q$ or $Q\times Q$).

%     \item What are the superinvolutions on the centers of $M$, $M\times M$, $Q$ or $Q\times Q$ it they have a division grading?
% \end{itemize}

% \begin{defi}
%     Let $(R, \vphi)$ be a superalgebra with superinvolution. 
%     A subset $I \subseteq R$ is said to be \emph{$\vphi$-invariant} if $\vphi(I) \subseteq I$. 
%     We say that $(R, \vphi)$ is a \emph{superinvolution-simple superalgebra} if the only $\vphi$-invariant superideals of $R$ are the trivial ones.
% \end{defi}

The next definition and lemma give us a class of examples of superinvolution-simple superalgebras that are not simple as superalgebras.
In fact, Proposition \ref{prop:only-SxSsop-is-simple} will tell us these are the only examples.
% Among all these, Example ?? is of special interest for us will be discussed with details.

\begin{defi}\label{def:SxSsop}
	Let $S$ be a superalgebra and consider the superalgebra $S \times S\sop$.
	We define the \emph{exchange superinvolution} on $S \times S\sop$ to be the map $\vphi\from S \times S\sop \to S \times S\sop$ given by $\varphi (s_1, \bar s_2) = (s_2, \bar s_1)$. Recall that $\bar s$ denotes the element $s \in S$ seen as an element of $S\sop$.
\end{defi}

It should be noted that, if $S$ admits a super-anti-isomorphism $\psi\from S \to S$, then $S\times S\sop$ with the exchange superinvolution is isomorphic to $S\times S$ with the superinvolution given by $(s_1, s_2) \mapsto (\psi\inv (s_2), \psi (s_1))$.

\begin{lemma}\label{lemma:SxSsop-simple-iff-S-simple}
	Let $S$, be a superalgebra and let $\vphi$ be the exchange superinvolution on $R \coloneqq S \times S\sop$.
	Then $(R, \vphi)$ is superinvolution-simple if, and only if, $S$ is simple as a superalgebra.
\end{lemma}

\begin{proof}
	We will prove that the $\vphi$-invariant superideals of $(S \times S\sop, \vphi)$ are precisely the subsets of the form $I \times I\sop$ where $I$ is a superideal of $S$.

	Let $I$ be a superideal of $S$ and consider $J \coloneqq I \times I\sop$.
	Clearly, $J$ is a $\vphi$-invariant subsuperspace of $S \times S\sop$.
	Suppose $i_1, i_2 \in I\even \cup I\odd$ and let $r = (s_1, \bar s_2) \in R\even \cup R\odd$.
	Then $r\, (i_1, \bar i_2) = (s_1 i_1, \sign{s_2}{i_2}\, \overline{i_2 s_2}) \in J$ and $(i_1, \bar i_2)\, r = (i_1 s_1, \sign{s_2}{i_2}\, \overline{s_2 i_2}) \in J$, so $J$ is, indeed, a superideal.

	Now let $J$ be any $\vphi$-invariant superideal of $S\times S\sop$ and let $I \coloneqq (1,0)\, J$, which we can regard as a subspace $S \iso S \times \{ 0 \}$.
	First, note that $I$ is a superideal of $S$.
	Since $J$ is a superideal, $I \subseteq J$, and since $J$ is $\vphi$-invariant, $I + \vphi(I) = I \times I\sop \subseteq J$.
	If $(j_1, \bar j_2) \in J$, then, on the one hand, $j_1 \in I$, and, on the other hand, $\vphi(j_1, \bar j_2) = (j_2, \bar j_1) \in J$, so $j_2 \in I$.
	Therefore $(j_1, \bar j_2)\in I \times I\sop$, concluding the proof.
\end{proof}

\begin{ex}\label{ex:FxF-iso-FZ2}
	The simplest possible example is to take $S = \FF$, with trivial $\ZZ_2$-grading.
	If $\Char \FF \neq 2$, then $S\times S\sop = \FF [\zeta]$ where $\zeta = (1, -1)$ and the exchange superinvolution is given by $\vphi(1) = 1$ and $\vphi(\zeta) = -\zeta$.
	Note that $\FF [\zeta] \iso \FF\ZZ_2$ with the trivial $\ZZ_2$-grading, \ie, we consider all elements of $\FF\ZZ_2$ to be even.
\end{ex}

\begin{ex}\label{ex:FZ2xFZ2sop-iso-FZ4}
	Consider $S = Q(1)$, so $S\even = \FF 1$ and $S\odd = \FF u$ where $u^2 =1$.
	Note that $S$ is isomorphic to $\FF\ZZ_2$, but this time with its natural $\ZZ_2$-grading.
	% we are the group algebra $S = \FF \langle u \rangle$, where $u$ has order $2$, as a superalgebra by declaring  (in other words, $S \iso Q(1)$).
	% We claim that $S\times S\sop$ is isomorphic to $\FF\langle \omega \rangle$, where $\omega$ has order $4$.
	If $\Char \FF \neq 2$, we claim that $R \coloneqq S\times S\sop$ is isomorphic to $F\ZZ_4$.
	Indeed, the element $\omega \coloneqq (u, \bar u) \in S\times S\sop$ has order $4$ and generates $S\times S\sop$: $\omega^2 = (1, - \bar 1)$, $\omega^3 = (u, - \bar u)$ and $\omega^4 = (1, 1)$.
	Hence $R\even = \FF1 \oplus \FF \omega^2$ and $R\odd = \FF \omega \oplus \FF \omega^3$.
	Also, the exchange superinvolution on $R$ is given by $\vphi(1) = 1$, $\vphi(\omega) = \omega$, $\vphi(\omega^2) = -\omega^2$ and $\vphi(\omega^3) = -\omega^3$.
\end{ex}

\begin{prop}\label{prop:only-SxSsop-is-simple}
	Let $(R, \vphi)$ be a superinvolution-simple superalgebra.
	Then either $R$ is a simple superalgebra or there is a simple superalgebra $S$ such that $(R, \vphi)$ is isomorphic to $S\times S\sop$ with the exchange superinvolution.
\end{prop}

\begin{proof}
	Suppose $R$ is not simple and let $0 \neq I \subsetneq R$ be a superideal.
	Note that $\vphi(I)$ is also a superideal, hence $I \cap \vphi(I)$ and $I + \vphi (I)$ are $\vphi$-invariant superideals.
	Since $I \cap \vphi(I) \subseteq I \neq R$, we have $I \cap \vphi(I) = 0$, so we can write $I + \vphi (I) = I \oplus \vphi (I)$.
	Since $0 \neq I \subseteq I \oplus \vphi (I)$, we conclude that $R = I \oplus \vphi (I)$.
	Clearly, this implies that $(R, \vphi)$ is isomorphic to $I \times I\sop$ with exchange superinvolution.
	By Lemma \ref{lemma:SxSsop-simple-iff-S-simple}, $I$ must be simple as a superalgebra, concluding the proof.
\end{proof}

Recall that, over an algebraically closed field, every finite dimensional simple associative superalgebra is either isomorphic to $M(m,n)$ or to $Q(n)$ (prop ??).
Hence, Proposition \ref{prop:only-SxSsop-is-simple} tells us that if $(R, \vphi)$ is a  finite dimensional superinvolution-simple superalgebra, then $R$ is isomorphic to either $M(m,n)$, $Q(n)$, $M(m,n) \times M(m,n)\sop$ or $Q(n) \times Q(n)\sop$.
However, it is well known that the associative superalgebra $Q(n)$ does not admit a superinvolution (see, for example, ?? and ?? or Corollary \ref{cor:Q-no-spuerinv-center}, below),
% (it is a well-known fact, see ?? and ??, and it follows as corollary of the theoryand we have a proof for it later this section, , and it also follows from ??). 
hence, our superalgebra $R$ can only belong to 3 different families of superalgebras.
We will say $(R,\vphi)$ is of \emph{type} $M$, $M\times M\sop$ or $Q\times Q\sop$ according to which family it belongs.
And, as we are going to see, we can use the center of $R$ to distinguish among them.

\begin{defi}
	Let $R$ be an associative (super)algebra.
	The \emph{center} of $R$ is the set
	\[
		Z(R) = \{c\in R \mid cr = rc \text{ for all } r\in R \}.
	\]
\end{defi}

\begin{remark}
	For a superalgebra, there is also the notion of \emph{supercenter} (\ie, the linear span of the set of homogeneous elements $c \in R$ such that $rc = \sign{r}{c} cr$ for every homogeneous element $r \in R$), but this is not what we are considering here.
\end{remark}

% The following is true for any graded associative algebra (hence for superalgebras and graded superalgebras), so we state and prove in this generality. (see ??)

% It is important to note that we are considering here the notion of center as we would do for algebras. 
% In the superalgebra case one could consider the supercentre

\begin{lemma}%\label{lemma:center-is-graded}
	Let $G$ be an abelian group and let $R$ be a $G$-graded associative algebra.
	Then the center $Z(R)$ is a $G$-graded subspace of $R$.
\end{lemma}

\begin{proof}
	Let $c \in Z(R)$ and write $c = \sum_{g\in G} c_g$, where $c_g \in R_g$ for all $g \in G$.
	For every homogeneous $r \in R$, we have that
	%
	\begin{align*}
		\big(\sum_{g\in G} c_g\big)r = r \big(\sum_{g\in G} c_g\big).
	\end{align*}
	%
	Comparing the components of degree $gh = hg$, where $h = \deg r$, we conclude that $rc_g = c_g r$ for all $g \in G$.
	By linearity, $r c_g = c_g r$ for all $r\in R$, hence $c_g \in Z(R)$.
\end{proof}

Applying this lemma to an associative superalgebra $R$, with $G^\#$ playing the role of $G$, we see that $Z(R)$ is a $G$-graded subsuperspace of $R$.

% \begin{proof}
%     Let $c \in Z(S)$ and write $c = c_{\bar 0} + c_{\bar 1}$, where $c_i \in S^i$, $i \in \ZZ_2$.
%     For every $s \in S\even \cup S\odd$, we have that
%     %
%     \begin{align*}
%         (c_{\bar 0} + c_{\bar 1})s = s (c_{\bar 0} + c_{\bar 1}),
%     \intertext{hence,}
%         c_{\bar 0}s + c_{\bar 1}s = s c_{\bar 0} + s c_{\bar 1}.
%     \end{align*}
%     %
%     Since $s$ has parity, we must have $sc_i = c_is$, $i \in \ZZ_2$. 
%     But then it follows, by linearity, $sc_i = c_is$ for all $s\in S$, concluding the proof.
% \end{proof}

\begin{lemma}
	Let $(R, \vphi)$ be a superalgebra with super-anti-automorphism.
	Then $Z(R)$ is $\vphi$-invariant.
\end{lemma}

\begin{proof}
	By Lemma \ref{lemma:center-is-graded}, with $G = \ZZ_2$, we have $Z(R) = Z(R)\even \oplus Z(R)\odd$, so it is sufficient to show that if $c \in Z(R)\even \cup Z(R)\odd$, then $\vphi(c) \in Z(R)$.
	Let $r \in R\even \cup R\odd$.
	Since $c\vphi\inv (r) = \vphi\inv (r)c$, we can apply $\vphi$ on both sides and get $\sign{c}{r} r \vphi(c) = \sign{c}{r} \vphi(c) r$ and, hence, $r \vphi(c) = \vphi(c) r$.
\end{proof}

\begin{cor}\label{cor:Q-no-spuerinv-center}
	If $\Char \FF \neq 2$, the associative superalgebra $Q(n)$ does not admit a superinvolution.
\end{cor}

\begin{proof}
	The center of $Q(n)$ is isomorphic to $\FF1 \oplus \FF u$, where $u$ is an odd element with $u^2 = 1$.
	Let $\vphi$ be a super-anti-automorphism on $Q(n)$.
	Since $u$ is odd and central, $\vphi(u)$ is odd and central.
	Hence there is $\lambda \in \FF$ such that $\vphi(u) = \lambda u$.
	Using that $u^2 = 1$, we have $1 = \vphi(1) = \vphi(u^2) = - \vphi(u)^2 = - \lambda^2$.
	But then $\vphi^2 (u) = \lambda^2 u = -u \neq u$, hence $\vphi^2 \neq \id$.
\end{proof}

% \begin{defi}
%     Let $R$ be a superalgebra and let $\vphi\from R\to R$ be a superinvolution. 
%     We say that $\vphi$ is of the \emph{first kind} if it fixes all the elements of $Z(R)$. 
%     Otherwise, we say that $\vphi$ is of the \emph{second kind}.
% \end{defi}

\begin{prop}\label{prop:types-of-SA-via-center}
	Let $(R, \vphi)$ be a superalgebra with superinvolution.
	\begin{enumerate}[(i)]
		\item If $(R, \vphi)$ is of type $M$, then $(Z(R), \vphi) \iso (\FF, \id)$;\label{item:F-id}
		\item If $(R, \vphi)$ is of type $M\times M\sop$, then $(Z(R), \vphi)$ is isomorphic to the superalgebra with superinvolution in Example \ref{ex:FxF-iso-FZ2};\label{item:FZ2-exchg}
		\item If $(R, \vphi)$ is of type $Q\times Q\sop$, then $(Z(R), \vphi)$ is isomorphic to the superalgebra with superinvolution in Example \ref{ex:FZ2xFZ2sop-iso-FZ4}.\label{item:FZ4-exchg}
	\end{enumerate}
\end{prop}

\begin{proof}
	Item \eqref{item:F-id} follows from the fact that $Z(M_{m+n}(\FF)) \iso \FF$.
	It is easy to check that $Z(S\times S\sop) = Z(S) \times Z(S)\sop$ for any superalgebra $S$, so items \eqref{item:FZ2-exchg} and \eqref{item:FZ4-exchg} follow from $Z(M_{m+n}(\FF)) \iso \FF$ and $Z(Q(n)) \iso Q(1)$.
\end{proof}

For the remainder of the section, we will consider, as before, $R \coloneqq \End_\D(\U)$, where $\D$ is a graded division superalgebra and $\U$ is a nonzero graded right $\D$-module of finite rank.
Also, let $\vphi$ be a super-anti-automorphism on $R$ and let $(\vphi_0, B)$ be pair determining $\vphi$ as in Theorem \ref{thm:vphi-iff-vphi0-and-B} and following Convention \ref{conv:pick-even-form}.

We will now show that we can identify $(Z(\D), \vphi_0)$ with $(Z(R), \vphi)$.
For every $c\in Z(\D)$, consider $r_c\from \U \to \U$ given by $r_c(u) = uc$.
Clearly, $r_c$ is $\D$-linear, so $r_c \in R$.
Actually, we have $r_c\in Z(R)$.
Indeed, for all $r\in R = \End_\D(\U)$ and all $u\in \U$, we have $r (r_c(u)) = r(uc) = r(u) c = r_c(r(u))$.

\begin{prop}%\label{prop:R-and-D-have-the-same-center}
	The map $Z(\D) \to Z(R)$ given by $c \mapsto r_c$ is an isomorphism of $G$-graded superalgebras.
	Moreover, $\vphi (r_c) = r_{\vphi_0(c)}$.
\end{prop}

\begin{proof}
	Given $r\in Z(R)$, we can define $c_r\in \End_R (\U) =\D$ by $uc_r = r(u)$ for all $u\in \U$.
	Computations analogous to the ones above show that $c\in Z(\D)$, and it is clear that the map $r\mapsto c_r$ is the inverse of the map $c \mapsto r_c$.
	The definition of grading on $R = \End_\D (\U)$ implies that these maps are isomorphisms of $G$-graded superalgebras.

	For the ``moreover'' part, fix $c\in Z(\D)\even \cup Z(\D)\odd$ and let $u, v \in \U\even \cup \U\odd$.
	On the one hand,
	\begin{align*}
		B(uc,v) = B(r_c u), v) = (-1)^{|r_c||u|} B(u, \vphi(r_c) v) = (-1)^{|c||u|} B(u, \vphi(r_c) v).
	\end{align*}
	On the other hand,
	\begin{align*}
		B(uc, v) = (-1)^{(|B| + |u|) |c|} \vphi_0(c) B(u, v) & = (-1)^{(|B| + |u|) |c|} B(u, v) \vphi_0(c)     \\
		                                                     & = (-1)^{(|B| + |u|) |c|} B(u, v \vphi_0(c) )    \\
		                                                     & = (-1)^{(|B| + |u|) |c|} B(u, r_{\vphi_0(c)}v).
	\end{align*}
	%
	Since we are following Convention \ref{conv:pick-even-form}, either $\D$ is even, and hence $|c| = \bar 0$, or
	$|B| = \bar 0$.
	In any case, $|B||c| = \bar 0$, so we have that \[(-1)^{|c||u|} B(u, \vphi(r_c) v) = (-1)^{|c||u|} B(u, r_{\vphi_0(c)}v),\] and, hence, $B(u, (\vphi(r_c) - \vphi_0(c)) v) = 0$ for all $u, v \in \U$.
	Since $B$ is nondegenerate, the results follows.
\end{proof}

\begin{prop}\label{prop:vphi-R-simple-D-simple}
	The superalgebra $R$ is $\vphi$-simple if, and only if, the superalgebra $\D$ is $\vphi_0$-simple.
\end{prop}

\begin{proof}
	Pick a homogeneous $\D$-basis for $\U$ following Convention \ref{conv:pick-even-basis} and use it to identify $R$ with $M_k(\D) = M_k(\FF) \tensor \D$.
	By Proposition \ref{prop:matrix-vphi}, there is an invertible matrix $\Phi \in M_k(\D)$ such that, for every $X \in M_k(\D)$,
	$\vphi(X) = \Phi\inv \vphi_0(X\stransp) \Phi$.

	It is well known that the ideals of $M_k(\D)$ are precisely the sets of the form $M_k(I)$ for $I$ an ideal of $\D$.
	We will prove an analog of this, first, for superideals and, then, for $\vphi$-invariant superideals.

	If $I$ is a superideal, $M_k(I) = M_k(\FF) \tensor I$ is also a superideal since it is spanned by a set of $\ZZ_2$-homogeneous elements, namely, the elements of the form $E_{ij}\tensor d$ where $1 \leq i,j \leq k$ and $d \in I\even \cup I\odd$.
	Conversely, if $J = M_k(I)$ is a superideal, then we can write $I = \{ d\in  \D \mid E_{11}\tensor d \in J\}$.
	For every $d\in I$, write $d = d_{\bar 0} + d_{\bar 1}$, where $d_\alpha \in \D^\alpha$, $\alpha \in \ZZ_2$.
	Since the $\ZZ_2$-homogeneous components of $E_{11}\tensor d$ are $E_{11}\tensor d_{\bar 0}$ and $E_{11}\tensor d_{\bar 1}$ and they belong to $J$, we have $d_{\bar 0}, d_{\bar 1} \in I$.

	Now we are going to show that $M_k(I)$ is $\vphi$-invariant if, and only if, $I$ is $\vphi_0$-invariant.
	Suppose $I$ is $\vphi_0$-invariant.
	Then if $X \in M_k(I)$, it is clear that $\vphi_0 (X\stransp)$ is also in $M_k(I)$.
	But then $\vphi(X) = \Phi\inv \vphi_0(X\stransp) \Phi \in M_k(I)$ since $M_k(I)$ is an ideal.
	Conversely, suppose $M_k(I)$ is $\vphi$-invariant.
	Let $d \in I$ and consider $X = E_{11} \tensor d \in M_k(I)$.
	Then $E_{11} \tensor \vphi_0(d) = \vphi_0(X\stransp) = \Phi\, \vphi(X)\, \Phi\inv \in M_k(I)$, which shows that $\vphi_0(d) \in I$.
\end{proof}

\begin{cor}\label{cor:D-has-the-same-type}
	Suppose $\FF$ is an algebraically closed field and $\Char \FF \neq 2$.
	Assume that $\vphi$ is a superinvolution and that $R$ is $\vphi$-simple.
	Then $(R, \vphi)$ is of the same type as $(\D, \vphi_0)$. \qed
\end{cor}

% \begin{cor}\label{cor:D-has-the-same-type}
%     Suppose $\FF$ is an algebraically closed field and $\Char \FF \neq 2$. 
%     Assume that $\vphi$ is a superinvolution and that $R$ is $\vphi$-simple.
%     %
%      \begin{enumerate}[(i)]
%         \item If $R\iso M(m,n)$, then there are $m', n'\geq 0$ such that $\D \iso M(m', n')$;
%         \item If $R\iso M(m,n)\times M(m,n)\sop$ with exchange superinvolution, then there are $m', n'\geq 0$ such that $(\D, \vphi_0)$ is isomorphic to $M(m', n')\times M(m', n')\sop$ with exchange superinvolution;
%         \item If $R\iso Q(n)\times Q(n)\sop$ with exchange superinvolution, then there is $n' \geq 0$ such that $(\D, \vphi_0)$ is isomorphic to $Q(n')\times Q(n')\sop$ with exchange superinvolution.\qed
%     \end{enumerate}
% \end{cor}

We will obtain more precise information about $\D$ in the next section.

% As a consequence of this last Corollary, we can only put a $G$-grading on $M(m,n)\times $Em particular, em MxM precisamos de um elemento par de ordem 2 e em QxQ precisamos de um elemento impar de ordem 4 (comparar com resultado no paper sobre Q).

% \begin{remark}\label{rmk:we-only-need-B'-nonzero}
%     Note that in the proof of the ``moreover'' part of Theorem \ref{thm:vphi-iff-vphi0-and-B}, we do not need to assume $B'$ is nondegenerate, we only use that it is nonzero.
% \end{remark}

% \section{Graded division superalgebras with super-anti-automorphism}
% % -----------------------------------------

% We are now going is to investigate the finite dimensional graded division superalgebras that admit a superinvolution. 
% Throughout this section, we will assume that $\FF$ is an algebraically closed field with $\Char \FF \neq 2$, and we will fix a primitive fourth root of unity $i \in \FF$.
% %In this case, because of Theorem \ref{thm:vphi-involution-iff-delta-pm-1}, we are primarily interested in the case the graded division superalgebra admits a superinvolution.
% Our final goal is to classify the division gradings on finite dimensional superinvolution-simple associative superalgebras.

% Recall that a graded division superalgebra is the same as a graded division algebra if we consider the $G^\#$-grading. 
% In particular, the isomorphism class of a finite dimensional graded division superalgebra $\D$ is determined by a pair $(T, \beta)$ where $T \coloneqq \supp \D \subseteq G^\#$ is a finite abelian group and $\beta\from T\times T \to \FF^\times$ is an alternating bicharacter {\tt (see ??)}. 
% \marginpar{\tt (for the ``see'' part: [EK] $D_4$ and [BK] gradings on classical \\lie algebras)}
% Instead of writing subscripts for the $G$-grading and superscripts for the canonical $\ZZ_2$-grading, it will be convenient to write $\D = \bigoplus_{t\in T} \D_t$ and recover the parity via the map $p\from T \to \ZZ_2$ which is the restriction of the projection $G^\# = G\times \ZZ_2 \to \ZZ_2$.

% Since each component $\D_t$ of $\D$ is one-dimensional, an invertible degree-preserving map $\vphi_0\from \D \to \D$ is completely determined by a map $\eta\from T \to \FF^\times$.

% \begin{prop}\label{prop:superpolarization}
%     Let $\vphi_0\from \D \to \D$ be the invertible degree-preserving map determined by $\eta\from T \to \FF^\times$  such that $\vphi_0(X_t) = \eta(t) X_t$ for all $t\in T$ and $X_t\in \D_t$. 
%     Then $\vphi_0$ is a super-anti-automorphism if, and only if,
%     %
%     \begin{equation}\label{eq:superpolarization}
%         \forall a,b\in T, \quad (-1)^{p(a) p(b)} \beta(a,b) =  \eta(ab) \eta(a)\inv \eta(b)\inv.
%     \end{equation}
%     %
%     Moreover, $\D$ admits a super-anti-automorphism if, and only if, $\beta$ only takes values $\pm 1$.
% \end{prop}

% \begin{proof}
%     For all $a,b \in T$, let $X_a \in \D_a$ and $X_b\in \D_b$. Then:
%     %
%     \begin{alignat*}{2}
%         &&\vphi_0(X_a X_b) &= (-1)^{p(a) p(b)} \vphi_0(X_b) \vphi_0(X_a)\\
%         \iff&&\,\, \eta(ab)X_a X_b &= (-1)^{p(a) p(b)} \eta(a) \eta(b) X_b X_a\\
%         \iff&&\, \eta(ab)X_a X_b &= (-1)^{p(a) p(b)} \eta(a) \eta(b) \beta(b,a) X_a X_b\\
%         \iff&& \eta(ab) &= (-1)^{p(a) p(b)} \eta(a) \eta(b) \beta(b,a)
%         \\
%         \iff&& (-1)^{p(a) p(b)} \beta(b, a) &=  \eta(ab) \eta(a)\inv \eta(b)\inv.
%     \end{alignat*}
%     The right-hand side of this last equation does not change if $a$ and $b$ are switched, hence $\beta(b,a) = \beta(a,b)$. 

%     Since $\beta$ is alternating, $\beta(b, a) = \beta (a, b)\inv$, therefore $\beta(a,b)^2 = 1$, which proves one direction of the ``moreover'' part.
%     %from where we conclude that $\beta(a,b) = \pm 1$. 
%     The converse follows from the fact that the isomorphism class of $\D\sop$ is determined by $(T, \beta\inv)$, so if $\beta$ takes only values in $\{ \pm 1 \}$, there must be an isomorphism from $\D$ to $\D\sop$, which can be seen as a super-anti-isomorphism on $\D$.
% \end{proof}

% With this we can translate our task to the level of abelian groups. 
% Instead of considering the graded superalgebra with super-anti-automorphism $(\D, \vphi_0)$, we can focus, instead, on the data $(T, \beta, p, \eta)$, where $T$ is a finite abelian group, $\beta$ is an alternating bicharacter on $T$, $p\from T \to \ZZ_2$ is a group homomorphism and $\eta\from T \to \FF^\times$ satisfies Equation \eqref{eq:superpolarization}. 
% Also, the condition of $\vphi_0$ being a superinvolution clearly corresponds to $\eta(t) \in \{ \pm 1 \}$ for all $t\in T$.
