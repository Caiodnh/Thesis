\section{Graded-simple associative superalgebras}\label{sec:grd-simple-salg}

We will now adapt the results of the previous section to graded-simple superalgebras. 
For any abelian group $G$, we obtain a classification of $G$-graded graded-simple superalgebra over an algebraically closed field of any characteristic, following the approach used in \cite{paper-MAP} fom $M(m,n)$. 
A different (and more complicated) approach was used in \cite{BS} to obtain a description of $G$-gradings (with some restrictions on characteristic), but the isomorphism problem was not solved there. 

Let $G$ be a group. 
Recall that a $G$-graded superalgebra $R$ can be seen as a $G^\# \coloneqq G\times \ZZ_2$-graded algebra (see \cref{sec:grds-and-sa}) by defining $R_{(g,i)} = R_g \cap R^i$, for all $g\in G$ and $i \in \ZZ_2$. 
We identify $G$ with $G \times \{ \bar 0 \} \subseteq G^\#$. 
Clearly, $R$ is graded-simple as a $G$-graded superalgebra if, and only if, it is graded-simple as a $G^\#$-graded algebra. 
This allows us to easily transfer the results of the previous section to gradings on superalgebras, but at the cost of working in the group $G^\#$ instead of $G$. 

\begin{remark}
	If the canonical $\ZZ_2$-grading is a coarsening of the $G$-grading by means of a homomorphism $p\colon G\rightarrow \ZZ_2$ (referred to as the \emph{parity homomorphism}), then we could work with the pair $(G, p)$ instead of the group $G^\#$. 
	Note that, in this case we would have another isomorphic copy of $G$ in $G^\#$, namely, the image of the embedding $g\mapsto (g, p (g))$, which contains the support of the $G^\#$-grading. 
    % 	In this case, we do not need $G^\#$ and can work with the original $G$-grading.
\end{remark}

\subsection{Graded-division superalgebras and their supermodules}\label{ssec:supermodules-over-D}

\begin{defi}\label{def:grd-div-sa}
    A $G$-graded superalgebra $\D$ is said to be a \emph{graded-division superalgebra} if every nonzero element which is homogeneous with respect to both the $G$-grading and the canonical $\ZZ_2$-grading is invertible. 
    In this case, we may also refer to the $G$-grading on the superalgebra $\D$ as a \emph{division grading}.
\end{defi}

In other words, the graded-division superalgebras with respect to the $G$-grading are precisely the graded-division algebras with respect to the $G^\#$-grading. 
Recall that $T \coloneqq \supp \D$ is a subgroup of $G^\#$, so we can see $\D$ as a $T$-graded algebra and the canonical $\ZZ_2$-grading is the coarsening by the group homomorphism $p\from T \subseteq G^\# \to \ZZ_2$ given by $p (g, i) = i$ for all $(g,i)\in G^\#$. 
We will denote the kernel of $p$ by $T^+$, \ie, $T^+ \coloneqq T\cap (G\times \{ \bar 0 \}) = \supp \D\even$. 
Similarly, we define $T^- \coloneqq T\cap (G\times \{ \bar 1 \}) =  \supp \D\odd$. 

\begin{notation}
    For graded-division superalgebras we will use subscripts to refer to the $T$-grading, \ie, $\D_t$ refers to the homogeneous component $\D_g^i$ where $t = (g, i) \in T \subseteq G^\#$. 
\end{notation}

\begin{ex}\label{ex:Q(1)-as-grd-div-SA}
    Let $\langle u \rangle$ be a cyclic group of order $2$ and let $G = \langle h \rangle$ where $h$ has order at most $2$.  
    The group algebra $\D \coloneqq \FF\langle u \rangle$, with canonical $\ZZ_2$-grading given by $\D\even \coloneqq \FF 1$ and $\D\odd = \FF u$, becomes a $G$-graded graded-division superalgebra if we declare the $G$-degree of $u$ to be $h$. 
    In this case $G^\# = \langle h \rangle \times \ZZ_2$ and $T = \langle (h, \bar 1) \rangle \iso \ZZ_2$ 
    (compare with \cref{ex:group-algebra}). 
    Note that $\D \iso Q(1)$ as a superalgebra.
\end{ex}

\begin{ex}\label{ex:Pauli-2x2-super}
    \Cref{ex:Pauli-2x2} can be seen as a division $\ZZ_2$-grading on $M(1,1)$, $\Char \FF \neq 2$. 
    In this case, $G^\# = \ZZ_2 \times \ZZ_2$.
\end{ex}

Let $\U = \U\even \oplus \U\odd$ be a graded right $\D$-supermodule of finite rank. 
The isomorphism class of the graded right $\D$-supermodule $\U$ is, as in \cref{ssec:D-modules}, determined by the map $\kappa\from G^\#/T \to \ZZ_{\geq 0}$ given by $\kappa (x) = \dim_\D \U_x$ for all $x\in G^\#/T$. 
We also have a description only in terms of the group $G$, but for that we separate the graded-division superalgebras in two classes:

\begin{defi}\label{defi:even-odd-D}
    % Let $\D$ be a graded-division superalgebra. 
    If $\D = \D\even$ (\ie, $\D\odd = 0$) we say that $\D$ is an \emph{even} graded-division superalgebra, and if $\D \neq \D\even$ (\ie, $\D\odd \neq 0$) we say that $\D$ is an \emph{odd} graded-division superalgebra. 
\end{defi}

Note that if $\D$ is odd and finite dimensional, then $\dim_\FF \D\even = \dim_\FF \D\odd$. 

\begin{lemma}\label{lemma:odd-M-m=n}
    If $\D$ is odd and $\D \iso M(m,n)$ as a superalgebra, then $m = n$. 
\end{lemma}

\begin{proof}
    We have that $\dim M(m,n)\even = m^2 + n^2$ and $\dim M(m,n)\odd = 2mn$. 
    Hence $\dim M(m,n)\even = \dim M(m,n)\odd$ if, and only if, $m=n$. 
\end{proof}

Assume that $\D$ is an even graded-division superalgebra. 
Then both $\U\even$ and $\U\odd$ are graded $\D$-submodules, hence we can describe their isomorphism classes, respectively, by maps $\kappa_\bz, \kappa_\bo\from G/T \to \ZZ_{\geq 0}$ given by $\kappa_i (x) = \dim \U^i_x$ for all $i\in \ZZ_2$ and $x\in G/T$. 
Note that, in this case, $G^\#/T = G^\#/T^+$ is the disjoint union of $G/T$ and $(e, \bar 1) \cdot G/T$ and, clearly, $\kappa ((g,i)T) = \kappa_i (gT)$, for all $i\in \ZZ_2$ and $g\in G$. 

Now assume that $\D$ is odd. 
In this case, unless $\U = 0$, the graded subspaces $\U\even$ and $\U\odd$ are not $\D$-submodules. 
But we can follow a different approach.

\begin{defi}
    A graded basis of a graded supermodule $\U$ is said to be an \emph{even basis} if it consists only of even elements.
\end{defi}

Let $\B = \{u_\lambda\}_{\lambda \in \Lambda}$ be any basis for $\U$, and let $0 \neq d_1\in \D\odd$. 
For every $\lambda \in \Lambda$, if $u_\lambda$ is an odd element, let us replace it by $u_\lambda d_1$. 
The resulting set is, clearly, an even basis of $\U$. 

\begin{convention}\label{conv:pick-even-basis}
    If $\D$ is an odd graded-division superalgebra, we choose the graded basis $\B$ to be an even basis.
\end{convention}

It follows that the canonical $\ZZ_2$-grading on $\U$ is determined only by $\D$. 
To see that, take any even basis of $\U$ and let $\tilde \U$ be its $\D_e$-span. 
Clearly, $\tilde \U \subseteq \U\even$ and it is a  $\D_e$-form of $\U$, \ie, we can identify $\U = \tilde\U \tensor_{\D_e} \D$ and, hence, $\U\even = \tilde\U \tensor_{\D_e} \D\even$ and  $\U\odd = \tilde\U \tensor_{\D_e} \D\odd$. 

This can also be seen from the point of view of the map $\kappa$. 
Note that, every coset $x \in G^\#/T$ has an even representative. 
This entails that the map $\iota\from G/T^+ \to G^\#/T$ given by $\iota ( gT^+) = (g, \bar 0) T$ is a bijection, so we can work with $\kappa \circ \iota$, which we will, by abuse of notation, also denote by $\kappa$. 

It should be noted that, even though $\U\even$ and $\U\odd$ are not $\D$-submodules, they are $\D\even$-submodules. 
Clearly, the map $\kappa\from G/T^+ \to \ZZ_{\geq 0}$ above is the map associated to both $\U\even$ and $\U\odd$, and an even graded $\D$-basis for $\U$ is a graded $\D\even$-basis for $\U\even$. 

\begin{remark}\label{rmk:R-even-identificatios}
    Set $R \coloneqq \End_{\D}(\U)$. 
    If $\D$ is even, then we can identify $R\even$ with $\End_{\D}(\U\even) \oplus \End_{\D} (\U\odd)$. 
    If $\D$ is odd, then we can identify $R\even$ with $\End_{\D\even}(\U\even)$. 
\end{remark}

\subsection{Graded Wedderburn theory for superalgebras}\label{ssec:wedderburn-super}

Let $R$ be a $G$-graded superalgebra. 
Note that the graded (left) superideals of $R$ are precisely the graded (left) ideals of $R$ as a $G^\#$-graded superalgebra. 
Hence, \cref{thm:End-over-D,thm:iso-abstract} translate to graded superalgebras by simply attaching the ``super'' prefix wherever it is needed. 

% \begin{thm}\label{thm:super-End-over-D}
% 	Let $G$ be a group and let $R = R\even \oplus R\odd$ be a $G$-graded associative algebra. 
% 	Then $R$ is graded-simple and satisfies the \dcc on graded left superideals if, and only if,	there is a graded-division superalgebra $\D=\D\even \oplus \D\odd$ and a graded right $\D$-supermodule $\U = \U\even \oplus \U\odd$ of finite rank such that $R \simeq \End_{\D} (\mc{U})$ as graded algebras. \qed
% \end{thm}

% \begin{thm}\label{thm:iso-abstract}
% 	Let $R \coloneqq \End_\D(\U)$ and $R' \coloneqq \End_{\D'}(\U')$, where $\D$ and $\D'$ are graded-division superalgebras, and $\U$ and $\U'$ are nonzero right graded supermodules of finite rank over $\D$ and $\D'$, respectively.
% 	Given an isomorphism $\psi\from R \to R'$, there is a triple $(g, \psi_0, \psi_1)$, where $g \in G$, $\psi_0\from {}^{[g\inv]}\D^{[g]} \to \D'$ is an isomorphism of graded superalgebras, $\psi_1\from \U^{[g]} \to (\U')^{\psi_0}$ is an isomorphism of graded right $\D$-supermodules, such that
% 	\begin{equation}\label{eq:def-iso-algebras}
% 		\forall r\in R, \quad \psi(r) = \psi_1 \circ r \circ \psi_1\inv.
% 	\end{equation}
% 	Conversely, given a triple $(g, \psi_0, \psi_1)$ as above, Equation \eqref{eq:def-iso-algebras} defines an isomorphism of graded superalgebras $\psi\from R \to R'$.
% 	Another triple $(g', \psi_0', \psi_1')$ defines the same isomorphism $\psi$ if, and only if, there are $t\in \supp \D'$ and $0 \neq d\in \D'_t$ such that $g'= gt$, $\psi_0' = \mathrm{Int}_{d\inv} \circ \psi_0$ and $\psi_1' (u) = \psi_1 (u) d$ for all $u \in \U$. \qed
% \end{thm}

Let $R$ be a graded-simple superalgebra satisfying the \dcc on graded left superideals and write, as in \cref{thm:End-over-D}, $R\iso \End_\D (\U)$ where $\D$ is a graded-division superalgebra and $\U$ is a graded right $\D$-supermodule of finite rank. 
Since the property of being an even or an odd a graded-division superalgebra is invariant under isomorphism, by \cref{thm:iso-abstract}, we can extend \cref{defi:even-odd-D} to $R$:

\begin{defi}\label{defi:even-odd-R}
    If $\D$ is an even graded-division superalgebra, we say that the grading on $R$ is \emph{even}; if $\D$ is an odd graded-division superalgebra, we say that the grading on $R$ is \emph{odd}. 
\end{defi}

For the remaining part of this subsection, fix a graded-division algebra $\D$ and a nonzero graded right $\D$-module of finite rank $\U$, and set $R \coloneqq \End_{\D} (\U)$. 

By the discussion on \cref{ssec:supermodules-over-D}, if the grading on $R$ is even, all the information related to the canonical $\ZZ_2$-grading on $R$ is encoded in the graded right $\D$-supermodule $\mc U=\mc U\even \oplus \mc U\odd$. 
More explicitly, 
\[\label{eq:almost-ZZ-grading}
    R\even = \End_\D(\U\even) \oplus \End_\D(\U\odd) \AND R\odd = \Hom_\D(\U\even, \U\odd) \oplus \Hom_\D(\U\odd, \U\even).
\]
This can also be seen by fixing a graded basis $\{u_1, \ldots, u_{k_\bz}\}$ of $\U\even$ and a graded basis $\{u_{k_\bz + 1}, \ldots, u_{k_\bz + k_\bo}\}$ of $\U\odd$. 
Then $R$ can be identified with $M_{k_\bz \mid k_\bo}(\D)$ as a graded superalgebra. 
If $G$ is an abelian group, following \cref{rmk:M_k(F)-tensor-D}, we can identify $R$ with the tensor product of $G^\#$-graded algebras $M_{k_\bz \mid k_\bo}(\FF)\tensor \D$. 
So, $R\even = M_{k_\bz \mid k_\bo}(\FF)\even \tensor \D$ and $R\odd = M_{k_\bz \mid k_\bo}(\FF)\odd \tensor \D$. 

\begin{remark}\label{rmk:even-ZZ-grading}
    Note that, if the $G$-grading on $R$ is even, we can refine the $G^\#$-grading to a $G\times \ZZ$-grading by defining $R\inv \coloneqq \Hom_{\D} (\U\odd, \U\even)$, $R^0 \coloneqq R\even = \End_\D(\U\even) \oplus \End_\D(\U\odd)$ and $R^1 \coloneqq \Hom_{\D} (\U\even, \U\odd)$. 
\end{remark}

If the grading on $R$ is odd, then, again by the discussion on \cref{ssec:supermodules-over-D}, all the information about the canonical $\ZZ_2$-grading is encoded in $\D$. 
To see it more explicitly, let us follow \cref{conv:pick-even-basis} and recall the maps $f_{i, j, d} \in R = \End_\D (\U)$ defined in the beginning of \cref{ssec:Grd-Wedderburn-Theory}. 
Those maps generate $R$ and, by \cref{eq:grd-M_k(D)-nonabelian}, their parity depend only on $d \in \D$. 

If $G$ is an abelian group, again, this can also be seen from the identification $R = M_k(\FF)\tensor \D$ in \cref{rmk:M_k(F)-tensor-D}. 
The grading on $M_k(\FF)$ is elementary grading defined by a tuple of elements in $G = G \times \{ \bar 0 \}$ and, hence, $R\even = M_k(\D) \tensor \D\even$ and $R\odd = M_k(\D) \tensor \D\odd$. 

\begin{remark}\label{rmk:M(D)=M(FF)-tensor-D}
	Recall that, for superalgebras, the tensor product is defined with a different multiplication on the superspace $M_k(\FF)\tensor \D$ (\cref{defi:tensorSuperalgebras}). 
	But if we are following Convention \ref{conv:pick-even-basis}, either $M_k(\FF)$ or $\D$ have trivial canonical $\ZZ_2$-grading, hence the tensor product of superalgebras coincides with that of algebras.
\end{remark}

Note that we already have a description of even gradings only in terms of the group $G$. 
For odd gradings, though, we still need to describe $\D$ only in terms of the group $G$, which is a harder task, and we are going to do that only for the case $R$ is a finite dimensional superalgebra over an algebraically closed field (see \cref{ssec:T-beta-p,sec:assc-only-G}).
 
\begin{prop}\label{prop:simple-R-D-super}
	The superalgebra $R = \End_\D (\U)$ is simple if, and only if, the superalgebra $\D$ is simple.
\end{prop}

\begin{proof}
    Pick a graded basis for $\U$ following \cref{conv:pick-even-basis} and use it to identify $R$ with $M_k(\D) = M_k(\FF) \tensor \D$, as in \cref{rmk:M(D)=M(FF)-tensor-D}.

	It is well known that the ideals of $M_k(\D)$ are precisely the sets of the form $M_k(I)$ for $I$ an ideal of $\D$.
	We will prove an analog of this for superideals.

	If $I$ is a superideal, $M_k(I) = M_k(\FF) \tensor I$ is also a superideal since it is spanned by a set of $\ZZ_2$-homogeneous elements, namely, the elements of the form $E_{ij}\tensor d$ where $1 \leq i,j \leq k$ and $d \in I\even \cup I\odd$.
	Conversely, if $J = M_k(I)$ is a superideal, then we can write $I = \{ d\in  \D \mid E_{11}\tensor d \in J\}$.
	For every $d\in I$, write $d = d_{\bar 0} + d_{\bar 1}$, where $d_\alpha \in \D^\alpha$, $\alpha \in \ZZ_2$.
	Since the $\ZZ_2$-homogeneous components of $E_{11}\tensor d$ are $E_{11}\tensor d_{\bar 0}$ and $E_{11}\tensor d_{\bar 1}$ and they belong to $J$, we have $d_{\bar 0}, d_{\bar 1} \in I$.
\end{proof}

We also could easily adapt the correspondence between the centers of $\End_\D (\U)$ and $\D$ (\cref{prop:R-and-D-have-the-same-center}) to a correspondence between supercenters. 
It turns out, though, that the correspondence between centers is more useful even in the case of superalgebras. 
For example, if $\Char \FF \neq 2$, then $sZ(M(m,n)) = sZ(Q(n)) = \FF$ while $Z(M(m,n)) \neq Z(Q(n))$, so we can use the centers to distinguish the types of superalgebra (see \cref{prop:types-of-SA-via-center}). 

% -----------------
\subsection{Finite dimensional graded-division superalgebras over an algebraically closed field}\label{ssec:T-beta-p}

We will now focus on the case where $G$ is abelian and $\FF$ is algebraically closed. 
For many of the results in this section we will also assume that $\Char \FF \neq 2$, but this hypothesis is not necessary for the classification results in \cref{ssec:classification-assc-super}.

Let $\D$ be a finite dimensional graded-division superalgebra and consider the pair $(T, \beta)$ associated to it, where $T \subseteq G^\#$ is a finite subgroup of $G^\#$ and $\beta$ is an alternating bicharacter on $T$. 
As we did in \cref{ssec:supermodules-over-D}, let $p\from T \to \ZZ_2$ be the restriction to $T$ of the projection on the second component of $G^\# = G\times \ZZ_2 \to \ZZ_2$, and set $T^+ = \ker p = T \cap (G \times \{ \bar 0 \})$ and $T^- = T \cap (G \times \{ \bar 1 \})$. 
We will also denote by $\beta^+$ the restriction of $\beta$ to $T^+ \times T^+$. 
As done in \cref{ssec:grd-div-alg}, for every $t\in T$ we fix $0 \neq X_t\in D_t$.

It is useful to consider another bicharacter on $T$. 
We define $\tilde\beta\from T\times T \to \FF^\times$ by
\[\label{eq:tilde-beta-def}
    \forall t, s\in T, \quad \tilde\beta(t,s) \coloneqq (-1)^{p(t)p(s)}\beta(t,s). 
\]
In other words, we have that
\[\label{eq:tilde-beta}
    \forall t, s\in T, \quad X_t X_s = (-1)^{p(t)p(s)} \tilde\beta(t,s) X_s X_t
\]
(compare with \cref{eq:beta}). 
Clearly, the support of $sZ(\D)$, the supercenter of $\D$, is $\rad \tilde\beta$. 

The following result is the first step in the reduction of Lie colour algebras to Lie superalgebras, due to M. Scheunert (see \cite{MR529734}). 

\begin{prop}\label{prop:skew-bicharacter-grd-SA}
    Suppose $\Char \FF \neq 2$. 
    Let $T$ be an abelian group and $\tilde\beta$ be a skew-symmetric bicharacter on $T$. 
    Then there is a alternating bicharacter $\beta$ on $T$ and a group homomorphism $p\from T \to \ZZ_2$ such that $\tilde\beta(t,s) = (-1)^{p(t)p(s)}\beta(t,s)$ for all $t,s\in T$.
\end{prop}

\begin{proof}
    Since $\tilde\beta$ is skew-symmetric, $\tilde\beta(t,t) = \tilde\beta(t,t)\inv$, so $\tilde\beta(t,t)\in \pmone$. 
    Define $h\from T \to \pmone$ by $h(t) \coloneqq \tilde\beta(t,t)$, for all $t\in T$. 
    We have that 
    \[
        h(ts) = \tilde\beta(ts,ts) = \tilde\beta(t,t)\tilde\beta(t,s)\tilde\beta(s,t)\tilde\beta(s,s) = h(t)h(s),
    \] 
    for all $t,s \in T$, so $h$ is a group homomorphism. 
    We define $p\from T \to \ZZ_2$ as the unique group homomorphism such that $h(t) = (-1)^{p(t)}$ for all $t\in T$. 
    
    Finally, we define $\beta\from T\times T \to \FF^\times$ by $\beta(t,s) \coloneqq (-1)^{p(t)p(s)}\tilde\beta(t,s)$ for all $t,s\in T$. 
    Clearly, $\beta$ is a bicharacter and $\tilde\beta(t,s) = (-1)^{p(t)p(s)}\beta(t,s)$. 
    It remains to show that $\beta$ is alternating: 
    $\beta(t,t) = (-1)^{p(t)p(t)}\tilde\beta(t,t) = (-1)^{p(t)}\tilde\beta(t,t) = h(t) \tilde\beta(t,t) = 1$, for all $t\in T$.
\end{proof}

\Cref{prop:skew-bicharacter-grd-SA} tells us that a pair $(T, \tilde\beta)$, where $\tilde\beta$ is a skew-symmetric bicharacter on $T$, carries the same information as a triple $(T, \beta, p)$, where $\beta$ is an alternating bicharacter and $p\from T\to \ZZ_2$ is a group homomorphism. 
Throughout this work we decided to use the triples $(T, \beta, p)$ to parametrize finite dimensional graded-division superalgebras over an algebraically closed field, since we refer to the bicharacter $\beta$ and the homomorphism $p$ frequently and also because this parametrization is valid even in the case $\Char \FF = 2$. 
We we will say that the graded-division superalgebra $\D$ is \emph{associated} to triple $(T, \beta, p)$ if the graded-division algebra $\D$ is associated to $(T, \beta)$ and $p$ is its parity homomorphism.

\begin{remark}
    If $\D$ is considered as a $G^\#$-graded algebra, the parameter $p$ is redundant. 
    Nevertheless, it is convenient to keep it since there are situations where we may want to regard $\D$ as a $T$-graded algebra or as a $G$-graded superalgebra.
\end{remark}

\begin{lemma}\label{lemma:rad-tilde-beta}
	Suppose $\Char \FF \neq 2$. 
	Then $\rad \tilde\beta = (\rad \beta)\cap T^+$ and, therefore, $sZ(\D) = Z(\D)\cap \D\even$.
\end{lemma}

\begin{proof}
	Let $t\in T^+$.
	In this case $\tilde\beta (t, \cdot) = \beta (t, \cdot)$, so
	$\tilde\beta (t, T) = 1$ if, and only if, $\beta (t, T) = 1$.
	Hence $(\rad \tilde\beta)\cap T^+ = (\rad \beta)\cap T^+$.

	Now let $t \in T^-$.
	In this case $\tilde \beta (t,t) = (-1)^{|t|} = -1$, so $t \not\in \rad \tilde\beta$.
	Therefore $\rad \tilde\beta = (\rad \tilde\beta)\cap T^+$, concluding the proof.
\end{proof}

Note that if $\Char \FF = 2$, then $\tilde\beta = \beta$. 
Hence, regardless of the characteristic, we have that if $\beta$ is nondegenerate, then so is $\tilde\beta$. 
The converse, however, is not true. 
In view of \cref{lemma:rad-tilde-beta}, if $\Char \FF \neq 2$, the next result characterizes the case where $\tilde\beta$ is nondegenerate but $\beta$ is degenerate. 

\begin{lemma}\label{lemma:beta-deg-beta-tilde-nondeg}
    Suppose $\beta$ is degenerate. 
    Then $\beta^+$ is nondegenerate if, and only if, $(\rad \beta) \cap T^+ = \{e\}$. 
    If this is the case, then $\rad \beta = \langle t_1 \rangle$ for an element $t_1 \in T^-$ of order $2$. 
\end{lemma}

\begin{proof}
    By definition of radical, it is clear that $(\rad \beta) \cap T^+ \subseteq \rad \beta^+$, hence if $\beta^+$ is nondegenerate, then $(\rad \beta) \cap T^+ = \{e\}$. 
    
    Conversely, suppose $(\rad \beta) \cap T^+ = \{e\}$. 
    Since $\rad\beta \neq \{e\}$, there is $t_1 \in (\rad \beta) \cap T^-$. 
    For any $t_1' \in (\rad \beta) \cap T^-$, we have $t_1 \, t_1' \in (\rad\beta) \cap T^+ = \{e \}$, so $t_1' = t_1\inv$ and, hence, $t_1$ has order $2$ and $\rad \beta = \langle t_1 \rangle$. 
    To show that $\beta^+$ is nondegenerate, let $t_0 \in \rad \beta^+$. 
    Then $\beta(t_0, T^+) = 1$ and, clearly, $\beta(t_0, t_1) = 1$. 
    It follows that $t_0 \in \rad \beta$ and, since $t_0 \in T^+$, $t_0 = \{e\}$, concluding the proof. 
\end{proof}

% \begin{lemma}\label{lemma:beta-deg-beta-tilde-nondeg}
%     Suppose $\tilde\beta$ is nondegenerate but $\beta$ is degenerate. 
%     Then:
%     \begin{enumerate}[(i)]
%         \item $\rad \beta = \langle t_1 \rangle$, where  $t_1 \in T^-$ is a order two element; \label{item:rad-beta=t_1}
%         \item $T = (\rad \beta) \times T^+$; \label{item:T+tensor-t_1}
%         \item the restriction of $\beta$ to $T^+ \times T^+$ is nondegenerate. \label{item:beta+nondeg}
%     \end{enumerate}
% \end{lemma}

In \cref{def:standard-realization} we introduced the standard realizations for finite dimensional graded-division algebras that are simple as algebras. 
We are now going to extend this definition for finite dimensional graded-division superalgebras that are simple as superalgebras, \ie, to include superalgebras of type $Q$. 

If $\D \iso Q(n)$, then $\D\even \iso M(n)$ is a graded-division algebra that is simple as an algebra. 
Clearly, $\D\even$ is associated to the pair $(T^+, \beta^+)$, so by \cref{cor:D-simple-iff-beta-nondeg,lemma:beta-deg-beta-tilde-nondeg}, we have that $\rad \beta = \langle t_1 \rangle$ for an element $t_1\in T^-$ of order $2$, hence $t_1 = (h, \bar 1) \in G^\#$ for an element $h \in G$ of order at most $2$. 

Conversely, given a finite subgroup $T^+ \subseteq G$, a nondegenerate alternating bicharacter $\beta^+\from T^+ \times T^+ \to \FF^\times$ and an element $h\in G$ of order at most $2$, let $\D\even$ be a standard realization (see \cref{def:standard-realization}) of a matrix algebra with a division grading associated to $(T^+, \beta^+)$, set $t_1 \coloneqq (h, \bar 1)$ and $T \coloneqq \langle t_1 \rangle \times T^+$, and define $\beta\from T \times T \to \FF^\times$ by $\beta(s t_1^i, t t_1^j) \coloneqq \beta^+(s,t)$. 
It is clear that $\beta$ is an alternating bicharacter and that $\rad \beta = \langle t_1 \rangle$. 
If we consider $Q(1) = \FF \langle u \rangle$ as in \cref{ex:Q(1)-as-grd-div-SA}, with $h$ the $G$-degree of $u$, then $Q(1)$ is the graded-division superalgebra associated to $(\langle t_1 \rangle, \beta\restriction_{ \langle t_1 \rangle \times \langle t_1 \rangle })$. 
By \cref{lemma:colour-tensor-product}, $\D \coloneqq Q(1)\tensor \D\even$ is the graded $G^\#$-graded graded-division algebra associated to $(T, \beta)$, and, via Kronecker product, it is easy to see that $\D$ is a superalgebra of type $Q$. 

\begin{defi}\label{def:standard-realization-Q}
    The $G$-graded superalgebra $\D \coloneqq Q(1) \tensor \D\even = \D\even \oplus u\D\even$, where we declare the $G$-degree of $u$ to be $h$, will be referred to as a \emph{standard realization} of type $Q$ superalgebra with division grading associated to $(T^+, \beta^+, h)$.
\end{defi}

% \begin{cor}\label{prop:Q-T-beta-t_1}
%     The graded-division superalgebra $\D$ is of type $Q$ if, and only if, 
%     $\D$ is odd and $\D\even$ is simple as an algebra. 
% \end{cor}

% \begin{proof}
%     If $\D \iso Q(n)$, then
%     $Q(n)\odd \neq 0$ and $Q(n)\even \iso M_n(\FF)$. 
    
%     Conversely, if $\D$ is odd and $\D\even$ is simple, then $T^- \neq \emptyset$ and $\beta^+$ is nondegenerate. 
%     Hence, $|T^+|$ is a perfect square. 
%     We then have that $|T| = 2 |T^+|$ is not a perfect square, so $\beta$ must be degenerate. 
%     By definition of radical, it is clear that $(\rad \beta) \cap T^+ \subseteq \rad \beta^+$, hence $(\rad \beta) \cap T^+ = \{e\}$. 
%     It follows that $\rad \beta$ is a complement for $T^+$ in $T$, so $\rad \beta = \langle t_1 \rangle$ where $t_1 \in T^-$ is an element of order $2$. 
%     Since $T = (\rad \beta) \times T^+$,
%     \cref{lemma:colour-tensor-product} tells us that, as a $G^\#$-graded algebra, $\D \iso \mc A \tensor \mc B$, where $\mc A$ is the graded-division algebra associated to $(\langle t_1 \rangle, \beta\restriction_{\langle t_1 \rangle \times \langle t_1 \rangle} )$ and $\mc B$ is the graded-division algebra associated to $(T^+, \beta^+)$. 
%     By \cref{ex:Q(1)-as-grd-div-SA}, we have $\mc A \iso Q(1)$ as a superalgebra. 
%     By \cref{cor:D-simple-iff-beta-nondeg}, $\mc B$ is a $G^\#$-graded matrix algebra, say $M_n(\FF)$. 
%     Hence, as a superalgebra, $\D \iso \mc A \tensor \B \iso Q(1) \tensor M_n(\FF) \iso Q(n)$, where the second isomorphism is given by the Kronecker product. 
% \end{proof}

% Under the conditions of \cref{prop:Q-T-beta-t_1}, we can write $t_1 = (h, \bar 1)$, where $h\in G$ is an element of order at most $2$. 

% Conversely, given a finite subgroup $T^+ \subseteq G$, a nondegenerate alternating bicharacter $\beta^+\from T^+ \times T^+ \to \FF^\times$ and an element $h\in G$ of order at most $2$, let $\D\even$ be a standard realization (see \cref{def:standard-realization}) of a matrix algebra with a division grading associated to $(T^+, \beta^+)$, set $t_1 \coloneqq (h, \bar 1)$ and $T \coloneqq \langle t_1 \rangle \times T^+$, and define $\beta\from T \times T \to \FF^\times$ by $\beta(s t_1^i, t t_1^j) \coloneqq \beta^+(s,t)$. 
% It is clear that $\beta$ is an alternating bicharacter and that $\rad \beta = \langle t_1 \rangle$. 

\begin{cor}\label{cor:tilde-beta-nondeg}
    The graded-division superalgebra $\D$ is simple as a superalgebra if, and only if, $(\rad \beta) \cap T^+ = \{e\}$. 
    More precisely, if this is the case, then
    \begin{enumerate}[(i)]
        \item $\D$ is a superalgebra of type $M$ if, and only if, $\beta$ is nondegenerate;
        \item $\D$ is a superalgebra of type $Q$ if, and only if, if $\beta$ is degenerate.
    \end{enumerate}
\end{cor}

\begin{proof}
    If $\D$ is simple as a superalgebra, then by \cref{thm:fd-simple-SA}, $\D \iso M(m,n)$ or $\D \iso Q(n)$ as superalgebra. 
    In both cases, $sZ(\D) = \FF$, so $\tilde\beta$ is nondegenerate.

    Conversely, suppose $\tilde\beta$ is nondegenerate. 
    If $\beta$ is nondegenerate, then $\D$ is simple as an algebra, by \cref{cor:D-simple-iff-beta-nondeg}.  
    If $\beta$ is degenerate, then combining  \cref{lemma:beta-deg-beta-tilde-nondeg} with \cref{lemma:colour-tensor-product}, we get that, as a $G^\#$-graded algebra, $\D \iso \mc A \tensor \mc B$, where $\mc A$ is the graded-division algebra associated to $(\rad \beta, \beta\restriction_{(\rad \beta)\times (\rad \beta)})$ and $\mc B$ is the graded-division algebra associated to $(T^+, \beta\restriction_{T^+ \times T^+})$. 
    
    By \cref{lemma:beta-deg-beta-tilde-nondeg}, it is easy to see that $\mc A \iso Q(1)$, where we declare the elements of $Q(1)\odd$ to have degree $t_1$. 
    By \cref{lemma:beta-deg-beta-tilde-nondeg} and \cref{cor:D-simple-iff-beta-nondeg}, $\mc B$ is a $G^\#$-graded matrix algebra, say $M_n(\FF)$, with $\supp \mc B = T^+$. 
    Hence, as a superalgebra, $\D \iso \mc A \tensor \B \iso Q(1) \tensor M_n(\FF) \iso Q(n)$, where the second isomorphism is given by the Kronecker product. 
\end{proof}

\begin{remark}\label{rmk:D-simple-iff-tilde-beta-nondeg}
    Clearly, if $\Char \FF \neq 2$, \cref{lemma:rad-tilde-beta,cor:tilde-beta-nondeg} imply that $\D$ is simple as a superalgebra if, and only if, $\tilde\beta $ is nondegenerate (compare with \cref{cor:D-simple-iff-beta-nondeg}). 
\end{remark}

Recall that, any nonzero $G^\#$-homogeneous element $d\in \D$ gives rise to the inner automorphism $\operatorname{Int}_d\from \D \to \D$ defined by $\operatorname{Int}_d (c) \coloneqq dcd\inv$, for all $c\in \D$ (Definition \ref{def:inner-automorphism}). 
We will now generalize this to graded superalgebras:

\begin{defi}\label{def:superinner}
	Let $d\in \D$ be a nonzero $G^\#$-homogeneous element.
	We define the \emph{superinner automorphism} $\operatorname{sInt}_d\from \D \to \D$ by $\operatorname{sInt}_d (c) \coloneqq \sign{c}{d} dcd\inv$, for all $c\in \D$.
\end{defi}

% The next result is a generalization of \cite[??]{livromicha}:

\begin{prop}\label{prop:all-central-automorphisms-of-D-are-superinner}  
    Let $\psi_0\from \D \to \D$ be an automorphism that restricts to the identity on $Z(\D) \cap \D\even$. 
    Then $\psi_0 = \operatorname{sInt}_{X_t}$ for some $t\in T$. 
\end{prop}

\begin{proof}
    By \cref{lemma:Aut(D)-widehat-T}, there is a character $\chi\in \widehat{T}$ such that $\psi_0(X_t) = \chi(t) X_t$, for all $t\in T$. 
    Since $\psi_0$ is the identity on $sZ(\D)$, we have that $\chi(t) = 1$ for all $t \in \rad \tilde\beta = (\rad\beta) \cap T^+ = \rad \tilde\beta$. 
    
    Set $\barr T \coloneqq T/\rad \tilde\beta$ and let $\pi \from T \to \barr T$ denote the quotient homomorphism. 
    It is clear that $b\from \barr T \times \barr T \to \FF^\times$ given by $b( \pi(s), \pi(t) ) \coloneqq \tilde\beta (s,t)$ is well-defined and that $b$ is a nondegenerate skew-symmetric bicharacter on $\barr T$. 
    Therefore the map $\barr T \to \widehat{\barr T}$ given by $t \mapsto b(t, \cdot)$ is a group isomorphism.
    
    Since $\chi$ is trivial on $\rad \tilde\beta$, there is $\barr \chi \in \widehat{\barr T}$ such that $\chi = \barr \chi \circ \pi$ and, since $b$ is nondegenerate, there is an element $t\in T$ such that $b(\pi(t), \cdot ) = \barr \chi$. 
    A straightforward computation shows that $\psi_0 = \operatorname{sInt}_{X_t}$.
\end{proof}

\begin{defi}\label{def:parity-element}
    We say that $t_p \in T$ is \emph{a parity element} if $\tilde\beta(t_p, t) = (-1)^{p(t)}$ for all $t\in T$. 
\end{defi}

Clearly, $t_p$ is a parity element if, and only if, $\operatorname{sInt}_{X_{t_p}} = \nu$, where $\nu\from \D \to \D$ is the parity automorphism $\nu(X_t) = (-1)^{p(t)} X_t$ for every $t\in T$. 
%
For \cref{ex:Q(1)-as-grd-div-SA}, the parity automorphism is given by $\operatorname{sInt}_u$, 
% where
% \[
%     u coloneqq \begin{pmatrix}
%             0 & 1\\
%             1 & 0
%         \end{pmatrix},
% \]
hence $\bar 1 = \deg u \in T^-$ is a parity element. 
For \cref{ex:Pauli-2x2-super}, the parity automorphism is given by $\operatorname{sInt}_d = \operatorname{Int}_d$ where
\[
    d \coloneqq \begin{pmatrix}
            1 & 0\\
            0 & -1
        \end{pmatrix},
\]
hence $\deg d = (\bar 1, \bar 0) \in T^+$ is a parity element. 

\begin{cor}\label{cor:existence-parity-element}
    There always exists a parity element $t_p \in T$, and 
    the set $T_p$ of all parity elements in $T$ is the coset $t_p (\rad \tilde\beta)$. 
    % In particular, the parity element is unique if, and only if, $\tilde\beta$ is nondegenerate. 
\end{cor}

\begin{proof}
    Existence follows from \cref{prop:all-central-automorphisms-of-D-are-superinner}, since the parity automorphism is trivial on $Z(\D) \cap \D\even$. 
    The second assertion is clear from the definition. 
\end{proof}

We can describe the set $T_p = t_p (\rad \tilde\beta)$ differently: 

\begin{lemma}\label{lemma:set-parity-elements}
    Suppose $\Char \FF \neq 2$. 
    If $T^- =\emptyset$, then the set of all parity elements is $\rad \tilde\beta = \rad \beta = \rad \beta^+$. 
    Otherwise, $T_p \cap T^+ = (\rad \beta^+) \smallsetminus (\rad\beta)$ and $T_p \cap T^- = (\rad \beta) \cap T^-$. 
\end{lemma}

\begin{proof}
    If $T^- =\emptyset$, then $(-1)^{p(t)} = 1$ for all $t\in T$, so $T_p = \rad \tilde\beta$ and $\tilde\beta = \beta = \beta^+$. 
    We will now suppose $T^- \neq\emptyset$.
    
    If $t_p$ is an even parity element, then clearly $t_p \in \rad\beta^+$ but $t_p \not\in \rad \beta$. 
    Conversely, if $t_p \in \rad \beta^+ \setminus \rad\beta$, let $t_1\in T^-$. 
    Since $\beta(t_p, T^+) = 1$, we have $\beta(t_p, T^-) = \beta(t_p, t_1 T^+) = \beta(t_p, t_1) \neq 1$. 
    Since $t_1^2 \in T^+$, $\beta(t_p, t_1)^2 = \beta(t_p, t_1^2) = 1$ and, hence, $\beta(t_p, t_1) = - 1$, proving that $t_p$ is a parity element. 
    
    Finally, an odd element $t_p$ is a parity element if, and only if, $\tilde\beta(t_p, t) = (-1)^{p(t)} = (-1)^{p(t_p) p(t)}$ for all $t\in T$ which, by the definition of $\tilde\beta$, is equivalent to $\beta(t_p, t) = 1$ for all $t\in T$. 
    The result follows. 
\end{proof} 

Note that, if $\Char \FF \neq 2$, then, by \cref{lemma:rad-tilde-beta}, all elements in $T_p$ have the same parity, \ie, either $T_p \cap T^+ = \emptyset$ or $T_p \cap T^- = \emptyset$. 

\begin{cor}\label{cor:radical-with-parity}
    Let $t_p\in T$ be a parity element. 
    If $t_p\in T^+$, then $\rad \beta = \rad \tilde\beta$ and $\rad \beta^+ = (\rad \tilde\beta) \cup  t_p(\rad \tilde\beta)$. 
    If $t_p\in T^-$, then $\rad \beta = (\rad \tilde\beta) \cup  t_p(\rad \tilde\beta)$ and $\rad \beta^+ = \rad \tilde\beta$. \qed
\end{cor}



% -----------------
\subsection{Finite dimensional graded-simple superalgebras over an algebraically closed field}\label{ssec:classification-assc-super}

We are now going to specialize the main results of \cref{ssec:param-End_D-U} to the superalgebra case, so we continue assuming that $\FF$ is algebraically closed and that $G$ is abelian. 
Recall from \cref{ssec:supermodules-over-D} that our classification of graded $\D$-modules of finite rank depend on $\D$ being even or odd.

\begin{defi}\label{def:E(D,U)-super}
    Let $\D$ be a finite dimensional graded-division superalgebra over an algebraically closed field $\FF$, associated to the triple $(T, \beta, p)$, and let $\U$ be a graded right $\D$-supermodule of finite rank. 
    We define the \emph{parameters} of the pair $(\D, \U)$ as:
    \begin{enumerate}[(i)]
        \item the quadruple $(T, \beta, \kappa_\bz, \kappa_\bo)$, if $\D$ is even and $\U$ is associated to the maps $\kappa_\bz, \kappa_\bo \from G/T \to \ZZ_{\geq 0}$;
        \item the quadruple $(T, \beta, p, \kappa)$, if $\D$ is odd and $\U$ is associated to the map $\kappa\from G/T^+ \to \ZZ_{\geq 0}$.
    \end{enumerate}
\end{defi}

Recall that for every map $\kappa\from G/T\to \ZZ_{\geq 0}$ with finite support, we define $|\kappa| \coloneqq \sum_{x \in G/T} \kappa(x)$. 
We, then, have that  \cref{thm:iso-End_D-U-with-parameters} translates the following result in the ``super'' case:

\begin{thm}\label{thm:iso-D-even}
	Let $(\D, \U)$ and $(\D', \U')$ be pairs as in Definition \ref{def:E(D,U)-super}, with both $\D$ and $\D'$ even. 
	Let $(T, \beta, \kappa_\bz, \kappa_\bo)$ and $(T', \beta', \kappa_\bz', \kappa_\bo')$ be the parameters of $(\D, \U)$ and $(\D', \U')$, respectively. 
	Then $\End_\D (\U) \iso \End_{\D'} (\U')$ if, and only if, $T=T'$, $\beta=\beta'$, $p = p'$ and there is $g\in G$ such that either $g \cdot \kappa_{\bar 0}=\kappa_{\bar 0}'$ and $g \cdot \kappa_{\bar 1}=\kappa_{\bar 1}'$, or $g \cdot \kappa_{\bar 0}=\kappa_{\bar 1}'$ and $g \cdot \kappa_{\bar 1}=\kappa_{\bar 0}'$. \qed
\end{thm}

\begin{thm}\label{thm:iso-D-odd}
    Let $(\D, \U)$ and $(\D', \U')$ be pairs as in Definition \ref{def:E(D,U)-super}, with both $\D$ and $\D'$ odd. 
    Let $(T, \beta, p, \kappa)$ and $(T', \beta', p', \kappa')$ be the parameters of $(\D, \U)$ and $(\D', \U')$, respectively. 
	Then $\End_\D (\U) \iso \End_{\D'} (\U')$ if, and only if, $T=T'$, $\beta=\beta'$, $p = p'$ and there is $g\in G$ such that $\kappa' = g \cdot \kappa$. \qed
\end{thm}

Gradings on simple associative superalgebras will be needed in \cref{chap:Lie} and are interesting in their own right. 
\Cref{prop:simple-R-D-super,rmk:D-simple-iff-tilde-beta-nondeg} together imply that $\End_\D (\U)$ is simple as a superalgebra if, and only if, $(\rad \beta) \cap T^+ = \{e\}$, which, in the case $\Char \FF \neq 2$, is equivalent to $\tilde\beta$ being nondegenerate (\cref{lemma:rad-tilde-beta}). 
If this is the case,  \cref{thm:iso-D-even,thm:iso-D-odd} give a classification of abelian group gradings on finite dimensional simple superalgebras, as follows. 
We note that, in the same way it as with  \cref{def:Gamma-T-beta-kappa}, there is an abuse of notation in \cref{def:Gamma-T-beta-kappa-even,def:Gamma-T-beta-kappa-odd,def:Gamma-T-beta-kappa-Q}. 

% (compare with \cite{paper-Qn}, \cite{paper-MAP} and \cite{Helens_thesis}). 

% ------------------------------------

We start with even gradings on superalgebras of type $M$:

\begin{defi}\label{def:Gamma-T-beta-kappa-even}
    Let $m, n\in \ZZ_{\geq 0}$, not both zero. 
    Given a finite subgroup $T \subseteq G$, a nondegenerate bicharacter $\beta\from T\times T \to \FF^\times$ and maps $\kappa_\bz, \kappa_\bo \from G/T \to \ZZ_{\geq 0}$ with finite support such that $|\kappa_\bz| \sqrt{|T|} = m$ and $|\kappa_\bo| \sqrt{|T|} = n$, consider:
    \begin{enumerate}[(i)]
        \item a standard realization $\D$ (see \cref{def:standard-realization}) of a matrix algebra with a division grading associated to $(T,\beta)$;
        \item the elementary grading (see \cref{defi:elementary-grd-super}) on $M(k_\bz, k_\bo)$ defined by $(\gamma_\bz, \gamma_\bo)$, where $\gamma_\bz$ and $\gamma_\bo$ are a $k_\bz$-tuple and a $k_\bo$-tuple of elements of $G$ realizing $\kappa_\bz$ and $\kappa_\bo$, respectively, where $k_\bz \coloneqq |\kappa_\bz|$ and $k_\bo \coloneqq |\kappa_\bo|$  (see \cref{defi:gamma-realizes-kappa}).
    \end{enumerate}
    %
    We define $\Gamma_M (T, \beta, \kappa_\bz, \kappa_\bo)$ to be the even grading on $M(m,n)$ given by identifying $M(m,n)$ with the graded superalgebra $M(k_\bz, k_\bo) \tensor \D$ via Kronecker product, \ie,
    \[
        \deg \left( E_{ij} \tensor d \right) = g_ig_j\inv t,
    \] 
    for all $1\leq i, j \leq k_\bz + k_\bo$, $t\in T$ and $0 \neq d \in \D_t$. 
    We will denote the superalgebra $M(m,n)$ endowed with $\Gamma_M (T, \beta, \kappa_\bz, \kappa_\bo)$ by $M(T, \beta, \kappa_\bz, \kappa_\bo)$. 
\end{defi}

\begin{cor}\label{cor:iso-M-even}
    Every even $G$-grading on $M(m,n)$ is isomorphic to $\Gamma_M (T, \beta, \kappa_\bz, \kappa_\bo)$ as in \cref{def:Gamma-T-beta-kappa-even}. 
    Two such gradings $\Gamma_M (T, \beta, \kappa_\bz, \kappa_\bo)$ and $\Gamma_M (T', \beta', \kappa_\bz', \kappa_\bo')$ are isomorphic if, and only if, $T = T'$, $\beta = \beta'$ and there is $g\in G$ such that either $g \cdot \kappa_{\bar 0}=\kappa_{\bar 0}'$ and $g \cdot \kappa_{\bar 1}=\kappa_{\bar 1}'$, or $g \cdot \kappa_{\bar 0}=\kappa_{\bar 1}'$ and $g \cdot \kappa_{\bar 1}=\kappa_{\bar 0}'$.  \qed
\end{cor}

Note that if $m \neq n$, 
then $k_\bz \neq k_\bo$ and, hence, only the case $g \cdot \kappa_{\bar 0}=\kappa_{\bar 0}'$ and $g \cdot \kappa_{\bar 1}=\kappa_{\bar 1}'$ is possible.

% -----

Now let us consider the odd gradings on $M(m,n)$. 
Note that in this case, by \cref{lemma:odd-M-m=n}, we have that $m = n$. 

% Our result will be in terms of the group $G^\#$.

\begin{defi}\label{def:Gamma-T-beta-kappa-odd}
    Let $n > 0$ be a natural number. 
    Given a finite subgroup $T \subseteq G^\#$, $T\not\subseteq G$, a nondegenerate alternating bicharacter $\beta\from T\times T \to \FF^\times$ and a map $\kappa\from G/T^+ \to \ZZ_{\geq 0}$ with finite support such that $|\kappa| \sqrt{|T|} = n$, consider 
    \begin{enumerate}[(i)]
        % \item the group homomorphism $p\from T \to \ZZ_2$ given by the projection on the second component of $G^\# = G\times \ZZ_2$;
        \item a standard realization $\D$ (see \cref{def:standard-realization}) of a matrix algebra with a division grading associated to $(T,\beta)$, viewed as a $G$-graded superalgebra with canonical $\ZZ_2$-grading given by the projection on the second entry $p\from T \subseteq G^\# \to \ZZ_2$;
        \item the elementary grading (see \cref{defi:elementary-grd-matrix}) on $M_{k}(\FF)$ defined by a $k$-tuple $\gamma$ of elements of $G$ realizing $\kappa$, where $k \coloneqq |\kappa|$ (see \cref{defi:gamma-realizes-kappa}). 
    \end{enumerate}
    %
    We define $\Gamma_M (T, \beta, p, \kappa)$ to be the odd grading on $M(n,n)$ given by identifying $M(n,n)$ with the graded superalgebra $M_k(\FF) \tensor \D$ via Kronecker product.  
    We will denote the superalgebra $M(n,n)$ endowed with $\Gamma_M (T, \beta, p, \kappa)$ by $M (T, \beta, p, \kappa)$. 
\end{defi}

\begin{cor}\label{cor:iso-M-odd}
    Every odd $G$-grading on $M(n,n)$ is isomorphic to $\Gamma_M (T, \beta, p, \kappa)$ as in \cref{def:Gamma-T-beta-kappa}. 
    Two such gradings $\Gamma_M (T, \beta, p, \kappa)$ and $\Gamma_M (T', \beta', p, \kappa')$ are isomorphic if, and only if, $T = T'$, $\beta = \beta'$, $p = p'$ and there is a $g\in G$ such that $g\cdot \kappa = \kappa'$. \qed
\end{cor}

% ----------------------------------

Finally, we classify the gradings on the superalgebra $Q(n)$. 
Note that we only have odd gradings in this case. 


\begin{defi}\label{def:Gamma-T-beta-kappa-Q}
    Let $n > 0$ be a natural number. 
    Given a finite subgroup $T^+ \subseteq G$, a nondegenerate bicharacter $\beta\from T^+ \times T^+ \to \FF^\times$, an element $h\in G$ such that $h^2 = 1$ and a map $\kappa\from G/T^+ \to \ZZ_{\geq 0}$ with finite support such that $|\kappa| \sqrt{|T^+|} = n$, consider 
    \begin{enumerate}[(i)]
        \item a standard realization $\D$ (see \cref{def:standard-realization-Q}) of a superalgebra of type $Q$ with a division grading associated to $(T^+, \beta^+, h)$;
        \item the elementary grading (see \cref{defi:elementary-grd-matrix}) on $M_{k}(\FF)$ defined by a $k$-tuple $\gamma$ of elements of $G$  of elements of $G$ realizing $\kappa$, where $k \coloneqq |\kappa|$ (see \cref{defi:gamma-realizes-kappa}).  
    \end{enumerate}
    %
    We define $\Gamma_Q (T^+, \beta^+, h, \kappa)$ to be the grading on $Q(n)$ given by identifying $Q(n)$ with the graded superalgebra $M_k(\FF) \tensor \D$ via Kronecker product. 
    We will denote the superalgebra $Q(n)$ endowed with $\Gamma_Q (T^+, \beta^+, h, \kappa)$ by $Q (T^+, \beta^+, h, \kappa)$. 
\end{defi}

\begin{cor}\label{cor:iso-Q}
    Every $G$-grading on $Q(n)$ is isomorphic to $\Gamma_Q (T^+, \beta^+, h, \kappa)$ as in \cref{def:Gamma-T-beta-kappa}. 
    Two such gradings $\Gamma_Q (T^+, \beta^+, h, \kappa)$ and $\Gamma_Q (T'^+, \beta'^+, h', \kappa')$ are isomorphic if, and only if, $T^+ = T'^+$, $\beta^+ = \beta'^+$, $h = h'$ and there is a $g\in G$ such that $g\cdot \kappa = \kappa'$. \qed
\end{cor}