\section[Graded-division superalgebras of types \texorpdfstring{$M \times M\sop$}{MxMsop} and \texorpdfstring{$Q \times Q\sop$}{QxQsop}]{Graded-division superalgebras of types \\ $M \times M\sop$ and $Q \times Q\sop$}\label{sec:model-O}

In the next section, we will classify all group gradings on the superinvolution-simple superalgebras of types $M \times M\sop$ and $Q \times Q\sop$. 
To this end, we will first construct division gradings on those. 

We will assume $\FF$ is algebraically closed and $\Char \FF \neq 2$. 
Fix $\bi \in \FF$ a primitive fourth-root of unity, \ie, $\bi^2 = -1$. 

Next subsection presents some well-known concepts and results from the theory of group extensions (see, for example, \cite{MR1344215}). 
We put it here for completeness and for fixing notation. 

\subsection{Abelian group extensions}

\begin{defi}
	Let $H, K$ be groups.
	A \emph{group extension of $H$ by $K$} is a group $E$ together with homomorphisms $\iota\from K \to E$ and $\pi\from E \to H$ such that
	%
	\begin{equation}\label{eq:short-exact-sequence}
		0 \rightarrow K \xrightarrow{\iota} E \xrightarrow{\pi} H \rightarrow 0
	\end{equation}
	%
	is a short exact sequence, \ie,
	$\iota$ is injective, $\pi$ is surjective and $\operatorname{im} \iota = \ker \pi$ (and, hence, $H \iso \frac{E}{\iota(K)}$).
	% For brevity, we may refer to the short exact sequence \eqref{eq:short-exact-sequence} as the extension. 
	An \emph{equivalence} between two extensions $K \xrightarrow{\iota} E \xrightarrow{\pi} H$ and $K \xrightarrow{\iota'} E' \xrightarrow{\pi'} H$ is a homomorphism $\alpha\from E \to E'$ such that the diagram
	%     \begin{center}
	% 		\begin{tikzcd}
	% 			&& |[alias=E]|E \arrow[to=E', "\alpha"] \arrow[to = H, "\pi"]&&\\
	% 			K \arrow[to=E, "\iota"] \arrow[to=E', "\iota'"]&&&& |[alias=H]|H\\
	% 			&&|[alias=E']|E' \arrow[to=H, "\pi'"]&&
	% 		\end{tikzcd}
	% 	\end{center}
	%%%%%%%%%%
	\begin{center}
		\begin{tikzcd}
			&& |[alias=E]|E \arrow[to=E', "\alpha"] \arrow[to = H, "\pi"]&&\\
			% 			0 \arrow[to=K]
			& |[alias=K]|K \arrow[to=E, "\iota"] \arrow[to=E', "\iota'"]&& |[alias=H]|H & \\ %\arrow[to=0] & |[alias=0]|0\\
			&&|[alias=E']|E' \arrow[to=H, "\pi'"]&&
		\end{tikzcd}
	\end{center}
	commutes.
	It is easy to see that $\alpha$ is necessarily an isomorphism.
	An equivalence from an extension to itself will be referred as a \emph{self-equivalence}.
\end{defi}

\begin{ex}\label{ex:main-extension}
    Let $T = T^+ \cup T^-$ be the support of a finite graded-division superalgebra.  Consider 
    \[
        E \coloneqq \big(\pmone \times T^+\big) \cup \big(\{\pm \bi\} \times T^-\big) \subseteq \{ \pm 1, \pm \bi \} \times T,
    \] 
    and let $\iota\from \pmone \to E$ and $\pi\from E \to T$ be the homomorphisms given by $\iota(\delta) \coloneqq (\delta, e)$, for all $\delta \in \pmone$,  and $\pi (x, t) \coloneqq t$, for all $x \in \{ \pm 1, \pm \bi \}$ and $t\in T$.
	Then $\pmone \xrightarrow{\iota} E \xrightarrow{\pi} T$ is an extension of $T$ by $\pmone$.
\end{ex}

Let us fix the abelian groups $H$ and $K$ for the remainder of the section.

\begin{defi}\label{defi:cocycle}
	A map $\sigma\from H\times H \to K$ is said to be a \emph{$2$-cocycle} if, for all $a,b,c \in H$, we have $\sigma(a,b) \sigma(ab, c) = \sigma(a, bc) \sigma(b,c)$.
	If, further, we have that $\sigma(a,b) = \sigma(b,a)$, we say that $\sigma$ is \emph{symmetric}, and if we have $\sigma(a, e) = \sigma(e, a) = e$, we say that $\sigma$ is \emph{normalized}.
	In the case $\sigma(h_1, h_2) = e$ for all $h_1, h_2 \in H$, we say that $\sigma$ is \emph{trivial}.
\end{defi}

% In this work, we are only going to deal with normalized symmetric $2$-cocycles, so we may refer to them simply as \emph{cocycles}.  

\begin{ex}\label{ex:coboundary-is-cocycle}
	Let $f\from H \to K$ be any map. 
	Recall (\cref{def:coboundary}) that its $2$-coboundary is the map $\mathrm{d} f\from H\times H \to K$ defined by $\mathrm{d} f\, (a,b) = f(ab)f(a)\inv f(b)\inv$ for all $a,b \in H$. 
	It is easy to see that $\mathrm{d} f$ is a $2$-cocycle, which is symmetric since both $H$ and $K$ are abelian. 
	If $f(e) = e$, $\mathrm{d} f$ is also normalized. 
\end{ex}

\begin{ex}\label{ex:bicharacter-is-cocycle}
    Let $T$ be an abelian group and let $b\from T\times T \to \FF^\times$ be a symmetric bicharacter. 
    It is also easy to see that $b$ is a normalized symmetric $2$-cocycle. 
\end{ex}

\begin{ex}\label{ex:beta-is-cocycle}
    Let $\D$ be a finite dimensional graded-division superalgebra associated to $(T, \beta, p)$. 
    If $\beta$ and $\tilde\beta$ only take values in $\pmone$, then both are symmetric and, hence, normalized symmetric $2$-cocycles. 
    This is not only a particular case of \cref{ex:bicharacter-is-cocycle}, but also of \cref{ex:coboundary-is-cocycle} (by \cref{prop:superpolarization}). 
\end{ex}

% It is easy to check that every symmetric bicharacter is a normalized symmetric $2$-cocycle, and those are the only ones that we will need in this work. 

Given a normalized symmetric $2$-cocycle $\sigma$, we define the abelian group $K \times_\sigma H$ to be the set $K\times H$ with product $*_\sigma$ given by:
\[
	(k_1, h_1)*_\sigma(k_2, h_2) = (\sigma(h_1, h_2) k_1 k_2, h_1 h_2),
\]
for all $k_1, k_2 \in K$ and $h_1, h_2 \in H$.
The condition of $\sigma$ being a symmetric $2$-cocycle is precisely what is needed for the associativity and commutativity of $*_\sigma$.
The identity element is $(e, e)$ and the inverse is given by $(k, h)\inv = (k\inv \sigma(h,h\inv)\inv, h\inv)$.
Note that the usual product on $K\times H$ is recovered as the particular case when $\sigma$ is trivial.

\begin{defi}
	The \emph{(abelian) group extension corresponding to $K \times_\sigma H$} is the group $K \times_\sigma H$ together the group homomorphisms $\iota \from K \to K \times_\sigma H$ and $\pi \from K \times_\sigma H \to H$ defined by $\iota (k) = (k , e)$ and $\pi (k, h) = h$, for all $k\in K$ and $h\in H$.
	If $\sigma$ is trivial, we say that the extension is \emph{trivial}.
\end{defi}

\begin{ex}\label{ex:pmone-x-tilde-beta-T}
    Under the conditions of \cref{ex:beta-is-cocycle}, with $\tilde\beta$ only taking values $\pm 1$, we have a group extension $\pmone \xrightarrow{\iota} \pmone \times_{\tilde\beta} T \xrightarrow{\pi} T$. 
\end{ex}

The next result tells us that, essentially, these are all abelian group extensions:

\begin{prop}\label{prop:all-ext-are-cocycle}
	Every abelian group extension $K \xrightarrow{\iota} E \xrightarrow{\pi} H$ is equivalent to the abelian group extension corresponding to $K\times_\sigma H$ for some normalized symmetric $2$-cocycle $\sigma\from H \to K$.
\end{prop}

\begin{proof}
	Fix $\tau\from H \to E$ a set- theoretic section of $\tau$ such that $\tau(e) = e$.
	For every $x \in E$, we have that $x (\tau(\pi(x)))\inv$ is in $\ker \pi$ and, hence, it is equal to $\iota(k)$ for a unique $k_x\in K$.
	It follows that $x = \iota(k_x) \tau(\pi(x))$.
	Define $\alpha\from E \to K\times H$ by $\alpha(x) = (k_x, \tau(\pi(x)) )$.
	Clearly, $\alpha$ is bijective, so we can make it an isomorphism by using it to define a product on $K\times H$.
	It is easy to check that this product is the one on $K \times_\sigma H$ where $\sigma\from H\times H \to K$ is given by $\sigma(h_1, h_2) = \tau(h_1) \tau(h_2) \tau(h_1 h_2)\inv$ and $\alpha$ is the desired equivalence.
\end{proof}

We will now see when two extensions determined by different normalized symmetric $2$-cocycles are equivalent. 

\begin{prop}\label{prop:coboundary}
	Let $\sigma, \sigma' \from H\times H \to K$ be normalized symmetric $2$-cocycles.
	A map $\alpha\from K \times_{\sigma} H \to K \times_{\sigma'} H$ is an equivalence between the corresponding group extensions if, and only if, there is a map $f\from H \to K$ such that $\alpha(k,h) = (f(h) k, h)$, for all $k\in K$ and $h\in H$, and $\sigma' \sigma\inv = \mathrm{d} f$.
\end{prop}

\begin{proof}
	The result follows from the following claims:
	\newclaims
	\begin{claim}
		Suppose $\alpha$ is an isomorphism.
		Then $\alpha$ is an equivalence if, and only if, there is a map $f\from H \to K$ such that $\alpha(k,h) = (f(h)k, h)$, for all $k\in K$ and $h\in H$.
	\end{claim}

	Suppose $\alpha(k,h) = (f(h)k, h)$.
	First, it is clear that $\pi \alpha = \pi$.
	Also, since $\alpha(e,e) = (e,e)$, it follows that $f(e) = e$ and, hence, $\alpha \iota = \iota$.

	Now suppose $\alpha$ is an equivalence.
	Since for all $h\in H$ $(\pi \alpha) (e, h) = \pi (e, h) = h$, we have that $\alpha (e, h) = (f(h), h)$ for some map $f\from H\to K$.
	Also, since for all $k\in K$ $(\alpha \iota) (k) = \iota(k)$, we have that $\alpha(k, e) = (k,e)$.
	It follows that $\alpha (k,h) = \alpha((k,e) *_\sigma (e, h)) = \alpha(k, e) *_{\sigma'} (e, h) = (f(h)k, h)$.

	\begin{claim}
		Suppose there is a map $f\from H \to K$ such that $\alpha(k,h) = (f(h)k, h)$, for all $k\in K$ and $h\in H$.
		Then $\alpha$ is an isomorphism if, and only if, $\sigma'\sigma\inv = \mathrm{d}f$.
	\end{claim}

	First, it is easy to check that the map $K \times_{\sigma'} H \to K \times_{\sigma} H$ given by $(k,h)\mapsto (f(h)\inv k, h)$ is the inverse of $\alpha$, so $\alpha$ is bijective.

	If $h_1, h_2 \in H$ and $t_1, t_2 \in T$, then, on the one hand,
	\begin{align*}
		\alpha ((h_1, t_1)*_{\sigma}(h_2, t_2) ) & = \alpha (\sigma(t_1, t_2) h_1 h_2, t_1 t_2) =
		( f(t_1 t_2)\sigma(t_1, t_2) h_1 h_2, t_1 t_2)
	\end{align*}
	and, on the other hand,
	\begin{align*}
		\alpha ((h_1, t_1)) *_{\sigma'} \alpha ((h_2, t_2) ) & = ( f(t_1) h_1, t_1) *_{\sigma'} ( f(t_2) h_2, t_2) \\&=
		(\sigma'(t_1, t_2) f(t_1) f(t_2) h_1 h_2, t_1 t_2).
	\end{align*}
	Comparing both, we conclude that $\alpha$ is a homomorphism if, and only if, $\mathrm{d}f = \sigma' \sigma\inv$.
\end{proof}

\begin{cor}\label{cor:self-equivalences-are-homomorphisms}
	The set of self-equivalences of an abelian group extension $K \xrightarrow{\iota} E \xrightarrow{\pi} H$ is in bijection with the set of homomorphisms from $H$ to $K$.
\end{cor}

\begin{proof}
	By Proposition \ref{prop:all-ext-are-cocycle}, we can assume our extension is the one corresponding to $K \times_\sigma H$, for some normalized symmetric $2$-cocycle $\sigma$.
	The result follows from Proposition \ref{prop:coboundary} and the observation that $\mathrm{d}f = 0$ if, and only if, $f$ is a homomorphism.
\end{proof}

\subsection{Quadratic maps and (super-)anti-automorphisms}

Let $\D$ be a finite dimensional graded-division superalgebra associated to $(T, \beta, p)$ and choose elements $0\neq X_t \in \D_t$ for all $t\in T$. 
Recall from \cref{prop:superpolarization} that super-anti-automorphisms of $\D$ are in bijection with maps $\eta\from T \to \FF^\times$ such that $\mathrm{d}\eta = \tilde\beta$. 
The same computation with $p$ trivial shows that anti-automorphisms of $\D$ are in bijection with maps $\eta\from T \to \FF^\times$ such that $\mathrm{d}\eta = \beta$ (see also \cite{livromicha}). 

Even though we are mainly interested in superinvolutions, some su\-per-an\-ti-au\-to\-mor\-phisms which are not involutive (and even some an\-ti-au\-to\-mor\-phisms) will be useful in what follows and also in \cref{chap:Lie}. 
We will need the ones that correspond to quadratic maps.

\begin{defi}
	Let $\mu\from T \to \FF^\times$ be a map.
	We say that $\mu$ is a \emph{(multiplicative) quadratic map} if $\mu(t\inv) = \mu (t)$ for all $t\in T$ and the map $\mathrm{d}\mu\from T\times T \to \FF^\times$ 
	is a bicharacter, which is referred to as the \emph{polarization} of $\mu$.
\end{defi}

\begin{remark}
	In the literature, the additive notation is usually used in the context of quadratic maps.
	In general, for a given commutative ring $\mathbb K$ and $\mathbb K$-modules $A$ and $B$, a quadratic map $q\from A \to B$ is a map satisfying that $q(kx) = k^2 x$, for all $k\in \mathbb K$ and $x \in A$, and that the map $A \times A \to B$ defined by $(x,y) \mapsto q(x+y) - q(x) - q(y)$ is bilinear.
	In the case of abelian groups (\ie, $\mathbb K = \ZZ$), the first condition is equivalent to $q(-x) = q(x)$, justifying our nomenclature. 
\end{remark}

In next example, a quadratic map is used to construct an anti-isomorphism between different graded-division superalgebras:

\begin{ex}\label{ex:quadratic-form-polarization-parity}
	Let $\mu_0\from T \to \FF^\times$ be given by $\mu_0(t) = 1$ for all $t\in T^+$ and $\mu_0(t) = \bi$ for all $t\in T^-$.
	It is clear that $\mu_0(t\inv) = \mu_0(t)$.
	Also, we have that $\mathrm{d}\mu_0 \, (a,b) = \mu_0(ab) \mu_0(a)\inv \mu_0(b)\inv = (-1)^{p(a)p(b)}$, which is a symmetric bicharacter. 
	This quadratic map determines an anti-isomorphism $\D \to \D\sop$, $X_t\mapsto \mu_0(t)\overline{X_t}$. 
	Note that, in particular, $\D\sop$ is isomorphic to $\D^{\mathrm{op}}$.
\end{ex}

\begin{lemma}\label{lemma:quadratic-form-involutions}
	Let $\mu\from T \to \FF^\times$ be a map and let $\psi\from \D \to \D$ be the linear map defined by
	$\psi(X_t) = \mu(t) X_t$ for every $t\in T$ and $X_t \in \D_t$.
	Then:
	\begin{enumerate}[(i)]
		\item If $\mathrm{d}\mu = \tilde\beta$, then: \\
		      $\mu$ is a quadratic map $\iff$ $\mu(t)\in \{ \pm 1 \}$ for all $t \in T^+$ and $\mu(t)\in \{ \pm \bi \}$ for all $t \in T^-$ $\iff$ $\psi$ is a super-anti-automorphism such that $\psi^2$ is the parity automorphism on $\D$; \label{item:super-anti-auto-quadratic}
		\item If $\mathrm{d}\mu = \beta$, then:\\
		      $\mu$ is a quadratic map $\iff$ $\mu(t) \in \{\pm 1\}$ for all $t\in T$ $\iff$ $\psi$ is
		      an involution on $\D$. \label{item:anti-auto-quadratic}
	\end{enumerate}
\end{lemma}

\begin{proof}
	From Equation \eqref{eq:superpolarization}, we have $\tilde\beta (t, t\inv) = \mu (tt\inv) \mu(t)\inv \mu(t\inv)\inv$, which simplifies to $(-1)^{p(t)} = \mu(t)\mu(t\inv)$.
	Therefore $\mu(t\inv) = \mu(t)$ if, and only if, $\mu(t)^2 = (-1)^{p(t)}$, proving \eqref{item:super-anti-auto-quadratic}.

	Item \eqref{item:anti-auto-quadratic} follows from \eqref{item:super-anti-auto-quadratic} once we consider $\D$ as an algebra, \ie, if we consider $p$ to be the trivial homomorphism.
	% First note that, as a particular case of Proposition \ref{prop:superpolarization}, $\psi$ is an antiautomorphism if, and only, if 
	% %
	% \begin{equation}\label{eq:polarization}
	%     \forall a,b\in T, \quad \beta(a,b) = \mu(ab) \mu(a)\inv \mu(b)\inv.
	% \end{equation}
	% %
	% Then $1 = \beta(t, t\inv) = \mu (tt\inv) \mu(t)\inv \mu(t\inv)\inv$, hence $\mu(t\inv) = \mu(t)\inv$. 
	% Therefore we have that $\mu(t\inv) = \mu(t)$ if, and only if, $\mu(t)^2 = 1$, proving (i).
\end{proof}

We will now refine the statement about existence of $\eta$ in \cref{prop:superpolarization}:

\begin{prop}\label{prop:existence-involution}
	If $\beta$ takes values in $\{ \pm 1 \}$, then there are quadratic maps on $T$ whose polarizations are $\beta$ and $\tilde \beta$.
\end{prop}

\begin{proof}
	To show there is a quadratic map whose polarization is $\beta$,
	consider the group $\barr T \coloneqq \frac{T}{\rad \beta}$ and let $t \mapsto \bar t$ be the natural projection on the quotient.
	Also, we denote by $\barr \beta\from \barr T \times \barr T \to \FF^\times$ the bicharacter induced by $\beta$.

	Since $\barr \beta$ is nondegenerate and takes values $\pm 1$, $\barr T$ is an elementary $2$-group. 
	Choose a standard realization $\barr \D$ associated to $(\barr T, \barr \beta)$. 
	By \cref{lemma:transp-std-realization}, the transposition is a degree-preserving involution of $\barr \D$, so it is determined by a map $\bar \mu\from T \to \FF^\times$. 
	By \cref{lemma:quadratic-form-involutions}, $\bar\mu$ is a quadratic map taking values in  $\{ \pm 1 \}$ whose polarization is $\bar\beta$. 
	Define $\mu\from T\to \FF^\times$ by $\mu (t) = \barr \mu (\bar t)$ for all $t\in T$.
	Then $\mu$ takes values in $\{ \pm 1 \}$ and $\beta = \mathrm{d} \mu$. 
	
	To get a quadratic map with polarization $\tilde \beta$, simply multiply $\mu$ by the quadratic map $\mu_0$ of \cref{ex:quadratic-form-polarization-parity}.
\end{proof}

\subsection[Division gradings on \texorpdfstring{$\D \times \D\sop$}{DxDsop}]{Division gradings on $\D \times \D\sop$}

We are now in position to apply these concepts and results to our original problem. 

\begin{thm}\label{thm:refinement-DxDsop}
	Let $\D$ be a graded division superalgebra associated to $(T, \beta, p)$.
	Consider $\mc E \coloneqq \D \times \D\sop$ with its natural $T$-grading $\Gamma\colon \mc E = \bigoplus_{t\in T} \D_t \times \overline{\D_t}$ and let $\vphi$ be the exchange superinvolution on $\mc E$.
	There is a division grading on $(\mc E, \vphi)$ refining $\Gamma$ if, and only if, $\beta$ takes values in $\{ \pm 1 \}$.
	If this is the case, then: 
	\begin{enumerate}[(i)]
		\item For each such refinement $\Delta$, associated to $(T_\Delta, \beta_\Delta, p_\Delta, \eta_\Delta)$, $T_\Delta$ fits into a unique group extension $\{ \pm 1 \} \xrightarrow{\iota} T_\Delta \xrightarrow{\pi} T$ such that ${}^{\pi} \Delta = \Gamma$. 
		Moreover, $\beta_\Delta = \beta \circ (\pi \times \pi)$, $p_\Delta = p \circ \pi$ and $\eta_\Delta \circ \iota = \id_{\pmone}$.
		\label{item:there-is-extension}
		%
		\item If $\tilde \Delta$ is another refinement, associated to $(T_{\tilde\Delta}, \beta_{\tilde\Delta}, p_{\tilde\Delta}, \eta_{\tilde\Delta})$, there is a unique equivalence between the corresponding group extensions $\alpha\from T_\Delta \to T_{\tilde\Delta}$ such that ${}^\alpha \Delta = \tilde \Delta$. 
		Moreover, $\beta_\Delta = \beta_{\tilde\Delta} \circ (\alpha \times \alpha)$, $p_\Delta = p_{\tilde\Delta} \circ \alpha$ and $\eta_\Delta = \eta_{\tilde\Delta} \circ \alpha$.
        \label{item:number-of-extensions}
		%
		\item There is a refinement $\Delta_0$ corresponding to the extension $\{ \pm 1 \} \xrightarrow{\iota} \pmone \times_{\tilde\beta} T \xrightarrow{\pi} T$ (\cref{ex:pmone-x-tilde-beta-T}), for which the map $\eta_{\Delta_0}\from \pmone \times_{\tilde\beta} T \to \pmone$ determining the exchange superinvolution is the projection onto the first component. 
		\label{item:pmone-x-T}
	\end{enumerate}
\end{thm}

\begin{proof}
    We will first prove the main assertion of \eqref{item:there-is-extension}. 
	Recall that, for a division grading, its support is its universal group (Lemma \ref{lemma:div-grd-unvrsl-grp}).
	Since $\Gamma$ is a coarsening of $\Delta$, by the universal property of the universal group, there is a unique homomorphism $\pi\from T_\Delta \to T$ such that ${}^{\pi}\Delta = \Gamma$.
	Also, since $|T_\Delta| = 2 |T|$, we must have $|\ker \pi| = 2$, and therefore there is a unique monomorphism $\iota\from \pmone \to T_\Delta$ with $\iota(\pmone) = \ker \pi$. 

	We claim that all division gradings on $(\mc E, \vphi)$ refining $\Gamma$ must have the same set of subspaces of $\mc E$ as their homogeneous components.
	Indeed, for each $t\in T$, let $\mc E_t$ denote the homogeneous component of degree $t$ of $\Gamma$. 
	Clearly, $\mc E_t$ is $2$-dimensional and $(X_t, \overline{X_t})$ and $(X_t, -\overline{X_t})$ are eigenvectors of $\vphi$, associated to the eigenvalues $1$ and $-1$, respectively.
	Hence, the eigenspaces of $\vphi \restriction_{\mc E_t}$ are $1$-dimensional. 
	By \cref{lemma:eigenvector-homogeneous}, it follows that $(X_t, \overline{X_t})$ and $(X_t, -\overline{X_t})$ are homogeneous in any grading $\Delta$ on $(\mc E, \vphi)$ refining $\Gamma$. 
	If $\Delta$ is a division grading, all the components are $1$-dimensional, so they must be the subspaces $\mc E_{(\delta, t)} \coloneqq \FF (X_t, \delta \overline{X_t})$, for $t\in T$ and $\delta \in \pmone$. 
	
	Now, for all $a, b \in T$ and $\delta_1, \delta_2 \in \{ \pm 1 \}$, we have:
	%
	\begin{align*}
		(X_{a}, \delta_1 \overline{X_{a}})(X_{b}, \delta_2 \overline{X_{b}}) & = (X_{a} X_{b}, \delta_1 \delta_2 \overline{X_{a}} \,\overline{X_{b}})               \\
		                                                                     & = (X_{a} X_{b}, \delta_1 \delta_2 (-1)^{p(a)p(b)} \overline{X_{b}X_{a}})             \\
		                                                                     & =(X_{a} X_{b}, \delta_1 \delta_2 (-1)^{p(a)p(b)} \beta(a, b) \overline{X_{a} X_{b}}) \\
		                                                                     & = (X_{a} X_{b}, \delta_1 \delta_2 \tilde\beta (a,b) \overline{X_{a} X_{b}}).
	\end{align*}
	%
	On the one hand, if $\beta(a, b) \neq \pm 1$ for some $a,b \in T$, the direct sum decomposition $\mc E = \bigoplus \mc E_{(\delta, t)}$ not a grading of $\mc E$ as a superalgebra. 
	On the other hand, if $\beta$ takes values in $\{ \pm 1 \}$, it follows that $\mc E_{(\delta_1, a)} \mc E_{(\delta_2, b)} = \mc E_{(\delta_1 \delta_2 \tilde\beta (a,b), ab)}$. 
	Therefore we have a grading $\Delta_0$ with support $\{ \pm 1\} \times_{\tilde\beta} T$. 
	Clearly, the exchange superinvolution is determined by the projection on the first entry $\{ \pm 1\} \times_{\tilde\beta} T \to \pmone$, $\pi\from \{ \pm 1\} \times_{\tilde\beta} T \to T$ is the projection on the second entry and that $\ker \pi = \pmone \times_{\tilde\beta} \{ e \}$, proving item \eqref{item:pmone-x-T}. 
	
	We will use the universal property of the universal group again to prove item \eqref{item:number-of-extensions}.
	Let $\{ \pm 1 \} \xrightarrow{\iota} T_\Delta \xrightarrow{\pi} T$ and $\{ \pm 1 \} \xrightarrow{\tilde\iota} T_{\tilde \Delta} \xrightarrow{\tilde\pi} T$ be the group extensions corresponding to $\Delta$ and $\tilde\Delta$, respectively. 
	Since they have the same components, $\Delta$ is an (improper) refinement of $\tilde\Delta$, so there is a unique group homomorphism $\alpha\from T_\Delta \to T_{\tilde \Delta}$ such that ${}^\alpha \Delta = \tilde\Delta$. 
	By the uniqueness of $\pi$, we have $\pi = \tilde\pi \alpha$ and, by the uniqueness of $\tilde\iota$, we have $\tilde\iota = \alpha \iota$, hence $\alpha$ is an equivalence between the corresponding extensions. 
	Since the component of $\Delta$ of degree $t$ is the same subspace as the component of $\tilde\Delta$ of degree $\alpha(t)$, we get $\beta_\Delta = \beta_{\tilde\Delta} \circ (\alpha \times \alpha)$, $p_\Delta = p_{\tilde\Delta} \circ \alpha$ and $\eta_\Delta = \eta_{\tilde\Delta} \circ \alpha$. 
	
	Finally, the ``moreover'' part of item \eqref{item:there-is-extension} is trivial for $\Delta_0$, and it follows in general by item \eqref{item:number-of-extensions}.
\end{proof}

\begin{ex}\label{ex:now-FZ2-is-division}
    The (trivially) graded superalgebra with superinvolution $\FF \times \FF\sop = M(1,0) \times M(1,0)\sop$ of \cref{ex:FxF-iso-FZ2} admits a $\ZZ_2$-grading (refining the trivial grading) that makes it isomorphic to $\FF\ZZ_2$: $\deg (1, 1) = \bar 0$ and $\deg (1, -1) = \bar 1$. 
    The exchange superinvolution is determined by $\eta(\bar 0) = 1$ and $\eta(\bar 1) = -1$. 
\end{ex}

\begin{ex}\label{ex:now-FZ4-is-division}
    The graded superalgebra with superinvolution $Q(1) \times Q(1)\sop$ of \cref{ex:FZ2xFZ2sop-iso-FZ4} 
    admits a $\ZZ_4$-grading (refining the trivial grading) that makes it isomorphic to $\FF\ZZ_4$: $\deg (1,1) = \bar 0$, 
    $\deg (u, \bar u) = \bar 1$, 
    $\deg (1,-1) = \bar 2$ and 
    $\deg (u,- \bar u) = \bar 3$. 
    The exchange superinvolution is determined by $\eta(\bar 0) = 1$, $\eta(\bar 1) = 1$, $\eta(\bar 2) = -1$ and $\eta(\bar 3) = -1$. 
    By \cref{thm:refinement-DxDsop}, up to relabeling the components, those are the only possible gradings on these superalgebras with superinvolutions. 
\end{ex}

It is straightforward that, if $\Delta$ is as in item \eqref{item:there-is-extension},  $\{ \pm 1 \} \rightarrow E \rightarrow T$ is another group extension and $\alpha\from T_\Delta \to E$ is an equivalence of extensions, then ${}^\alpha \Delta$ is a refinement corresponding to $\{ \pm 1 \} \rightarrow E \rightarrow T$. 
Hence, we get:

\begin{cor}\label{cor:old-item-iv}
	Under the conditions of Theorem \ref{thm:refinement-DxDsop}, the set of all refinements $\Delta$ such that $\supp \Delta = T_\Delta$ and ${}^\pi \Delta = \Gamma$ is in bijection with the group homomorphisms from $T$ to $\{ \pm 1 \}$.
\end{cor}

\begin{proof}
	It follows from item \eqref{item:number-of-extensions} and Corollary \ref{cor:self-equivalences-are-homomorphisms}.
\end{proof}

Note that all these refinements are nonisomorphic as graded superalgebras with superinvolution (since they have different $\eta$), but they are isomorphic as graded superalgebras (since they have the same $\beta$ and $p$). 

We can use quadratic forms to replace the extension in \cref{ex:pmone-x-tilde-beta-T,thm:refinement-DxDsop} by an equivalent one, which does not depend on $\tilde\beta$. 

\begin{lemma}\label{lemma:tildeT-finally}
	Under the conditions of \cref{thm:refinement-DxDsop}, suppose $\beta$ takes values in $\pmone$ and let $\Delta_0$ be the refinement in item \eqref{item:pmone-x-T}. 
    Then every quadratic map $\mu\from T \to \FF^\times$ such that $\mathrm{d}\mu = \tilde\beta$ gives us an equivalence $\alpha\from \pmone \times_{\tilde\beta} T \to E \coloneqq (\pmone \times T^+) \cup (\{ \pm \mathbf{i}\} \times T^-)$ between the group extensions of Examples \ref{ex:pmone-x-tilde-beta-T} and  \ref{ex:main-extension} given by $\alpha(\delta, t) \coloneqq (\delta \mu(t), t)$. 
    Further, in the grading ${}^\alpha \Delta_0$, the exchange superinvolution is determined by the map $\eta\from E \to \pmone$ given by $\eta(\lambda, t) = \lambda\mu(t)\inv$, for all $(\lambda, t) \in E$. 
\end{lemma}

\begin{proof}
    By \cref{lemma:quadratic-form-involutions}, $\mu(T^+) \subseteq \pmone$ and $\mu(T^-) \subseteq \{ \pm \bi \}$. 
    Hence, by \cref{prop:coboundary}, the map $\alpha'\from \{ \pm 1, \pm \bi \} \times_{\tilde\beta} T \to \{ \pm 1, \pm \bi \} \times T$ given by $\alpha'(\lambda, t) \coloneqq (\lambda\mu(t), t)$, for all $\lambda \in \{ \pm 1, \pm \bi \}$ and $t\in T$, is an equivalence between the corresponding extensions. 
    Restricting $\alpha'$ to $\pmone \times_{\tilde\beta} T$, we get precisely $\alpha$, proving that it is an equivalence. 
    The ``further'' part follows from item \eqref{item:number-of-extensions} in \cref{thm:refinement-DxDsop}. 
\end{proof}

\Cref{lemma:tildeT-finally} can be seen as taking a different model for $\D \times \D\sop$. 
Recall that $\mu$ defines a super-anti-automorphism on $\psi\from \D\to \D$, which can be seen as an isomorphism $\psi\from \D\sop\to \D$. 
Then the map $\D \times \D\sop \to \D \times \D$ given by $(d_1, \bar{d}_2) \mapsto (d_1, \psi(d_2))$, for all $d_1, d_2 \in \D$, is an isomorphism of graded superalgebras with superinvolution, where we consider the superinvolution $(d_1, d_2) \mapsto (\psi\inv (d_2), \psi (d_1))$ on $\D \times \D$. 
Under this isomorphism, the element $(X_t, \delta \overline{X_t})$ goes to $(X_t, \delta\mu(t) X_t)$ (compare with the formula for $\alpha$). 
If one follows the proof of \cref{thm:refinement-DxDsop} with $\D\times \D$ instead of $\D \times \D\sop$, the group $E$ would naturally appear in the place of $\pmone \times_{\tilde\beta} T$. 

We will now construct an example of
an odd graded-division superalgebra of type $M \times M\sop$, which will play an important role in the next section. 
Let $\D$ be the graded-division superalgebra of \cref{ex:Pauli-2x2-super}, \ie,  $\D = M(1,1)$ with the grading by $T \coloneqq \ZZ_2 \times \ZZ_2$ determined by:
\begin{align*}
	\deg \begin{pmatrix}
		\phantom{.}1 & \phantom{-}0\phantom{.} \\
		\phantom{.}0 & \phantom{-}1\phantom{.}
	\end{pmatrix} = (\bar 0, \bar 0),\quad & \deg \begin{pmatrix}
		\phantom{.}0 & \phantom{-}1\phantom{.} \\
		\phantom{.}1 & \phantom{-}0\phantom{.}
	\end{pmatrix} = (\bar 0, \bar 1), \\
	\deg \begin{pmatrix}
		\phantom{.}1 & \phantom{-}0\phantom{.} \\
		\phantom{.}0 & -1\phantom{.}
	\end{pmatrix} = (\bar 1, \bar 0),\quad &
	\deg \begin{pmatrix}
		\phantom{.}0 & -1\phantom{.}           \\
		\phantom{.}1 & \phantom{-}0\phantom{.}
	\end{pmatrix} = (\bar 1, \bar 1).
\end{align*}
%
Following the proof of \cref{thm:refinement-DxDsop}, we get a $\pmone \times_{\tilde\beta} T$-grading refining the natural $T$-grading on $\D \times \D\sop$, given by $\deg (X_t, \delta \overline{X_t}) = (\delta, t)$, for all $\delta \in \pmone$ and $t\in T$. 

To use \cref{lemma:tildeT-finally} and get a simpler model for the group, we need a quadratic map whose polarization is $\tilde\beta$. 
To construct one, we can follow the proof of \cref{prop:existence-involution}. 
First, let $\mu'\from T \to \pmone$ be the quadratic map determining the transposition on $\D = M(1,1)$, \ie, $\mu'(\bar 0, \bar 0) = 1$, $\mu'(\bar 0, \bar 1) = 1$, $\mu'(\bar 1, \bar 0) = 1$ and $\mu'(\bar 1, \bar 1) = -1$. 
Then we define $\mu \coloneqq \mu' \mu_0$, where $\mu_0$ is the quadratic map of \cref{ex:quadratic-form-polarization-parity}. 
In other words, $\mu$ is the map determining the queer supertranspose on $\D$ (see \cref{def:queer-stp}). 
Then, by \cref{lemma:tildeT-finally}, we construct a refinement as in \cref{thm:refinement-DxDsop} with grading group $E = \big(\pmone \times \ZZ_2 \times \{ \bar 0 \} \big) \cup \big(\{\pm \bi \} \times \ZZ_2 \times \{ \bar 1 \} \big)$. 

Finally, we have a group isomorphism $E \to \ZZ_2 \times \ZZ_4$, defined by $(1, \bar 1, \bar 0) \mapsto (\bar 1, \bar 0)$ and $(i, \bar 0, \bar 1) \mapsto (\bar 0, \bar 1)$. 
We then get: 

\begin{ex}\label{ex:superalgebra-O}
	Let $\mc O$ denote the superalgebra $M(1,1) \times M(1,1)\sop$ endowed with the $\ZZ_2\times \ZZ_4$-grading determined by: 
	\begin{align*}
		% -- 1st row --
		\deg \left(\begin{pmatrix}
			\phantom{.}1 & \phantom{-}0\phantom{.} \\
			\phantom{.}0 & \phantom{-}1\phantom{.}
		\end{pmatrix}, \overline{\begin{pmatrix}
				\phantom{.}1 & \phantom{-}0\phantom{.} \\
				\phantom{.}0 & \phantom{-}1\phantom{.}
			\end{pmatrix}}\right) = (\bar 0, \bar 0),\quad &
		\deg \left(\begin{pmatrix}
			\phantom{.}1 & \phantom{-}0\phantom{.} \\
			\phantom{.}0 & \phantom{-}1\phantom{.}
		\end{pmatrix}, -\overline{\begin{pmatrix}
				\phantom{.}1 & \phantom{-}0\phantom{.} \\
				\phantom{.}0 & \phantom{-}1\phantom{.}
			\end{pmatrix}}\right) = (\bar 0, \bar 2),    \\
		% -- 2nd row --
		\deg \left(\begin{pmatrix}
			\phantom{.}1 & \phantom{-}0\phantom{.} \\
			\phantom{.}0 & -1\phantom{.}
		\end{pmatrix}, \overline{\begin{pmatrix}
				\phantom{.}1 & \phantom{-}0\phantom{.} \\
				\phantom{.}0 & -1\phantom{.}
			\end{pmatrix}}\right) = (\bar 1, \bar 0),\quad &
		\deg \left(\begin{pmatrix}
			\phantom{.}1 & \phantom{-}0\phantom{.} \\
			\phantom{.}0 & -1\phantom{.}
		\end{pmatrix}, -\overline{\begin{pmatrix}
				\phantom{.}1 & \phantom{-}0\phantom{.} \\
				\phantom{.}0 & -1\phantom{.}
			\end{pmatrix}}\right) = (\bar 1, \bar 2),    \\
		% -- 3rd row --
		\deg \left(\begin{pmatrix}
			\phantom{.}0 & \phantom{-}1\phantom{.} \\
			\phantom{.}1 & \phantom{-}0\phantom{.}
		\end{pmatrix}, \overline{\begin{pmatrix}
				\phantom{.}0 & \phantom{-}1\phantom{.} \\
				\phantom{.}1 & \phantom{-}0\phantom{.}
			\end{pmatrix}}\right) = (\bar 0, \bar 1),\quad &
		\deg \left(\begin{pmatrix}
			\phantom{.}0 & \phantom{-}1\phantom{.} \\
			\phantom{.}1 & \phantom{-}0\phantom{.}
		\end{pmatrix}, -\overline{\begin{pmatrix}
				\phantom{.}0 & \phantom{-}1\phantom{.} \\
				\phantom{.}1 & \phantom{-}0\phantom{.}
			\end{pmatrix}}\right) = (\bar 0, \bar 3),    \\
		% -- 4th row --
		\deg \left(\begin{pmatrix}
			\phantom{.}0 & -1\phantom{.}           \\
			\phantom{.}1 & \phantom{-}0\phantom{.}
		\end{pmatrix}, \overline{\begin{pmatrix}
				\phantom{.}0 & -1\phantom{.}           \\
				\phantom{.}1 & \phantom{-}0\phantom{.}
			\end{pmatrix}}\right) = (\bar 1, \bar 3),\quad &
		\deg \left(\begin{pmatrix}
			\phantom{.}0 & -1\phantom{.}           \\
			\phantom{.}1 & \phantom{-}0\phantom{.}
		\end{pmatrix}, -\overline{\begin{pmatrix}
				\phantom{.}0 & -1\phantom{.}           \\
				\phantom{.}1 & \phantom{-}0\phantom{.}
			\end{pmatrix}}\right) = (\bar 1, \bar 1).
	\end{align*}
	%
	Then $\mc O$ is an odd graded division superalgebra and the exchange superinvolution on it preserves degrees. 
	Thus $\eta$ takes value $1$ on $(\bar 0, \bar 0)$, $(\bar 1, \bar 0)$, $(\bar 0, \bar 1)$ and $(\bar 1, \bar 3)$, and value $-1$ on $(\bar 0, \bar 2)$, $(\bar 1, \bar 2)$, $(\bar 0, \bar 3)$ and $(\bar 1, \bar 1)$.
\end{ex}

% Of course, the construction of $\mc O$ depend on the choice of the quadratic map $\mu$ on $ T = \ZZ_2 \times \ZZ_2$. 
We note that, by \cref{cor:old-item-iv}, there are $4$ different gradings by $\ZZ_2 \times \ZZ_4$ on the superalgebra with superinvolution $M(1,1)\times M(1,1)\sop$ refining the $\ZZ_2 \times \ZZ_2$-grading. 
The above is one of these $4$ gradings. 

The next result gives us a characterization of graded-division superalgebras with superinvolution that appear in \cref{thm:refinement-DxDsop}. 

\begin{prop}\label{lemma:undoubling-D}
    Let $(\mc E, \vphi_0)$ be a finite dimensional graded-division superalgebra with superinvolution associated to a quadruple $(T_{\mc E}, \beta_{\mc E}, p_{\mc E}, \eta_{\mc E})$. 
    Then there is a graded-division superalgebra $\D$ such that $(\mc E, \vphi_0)$ is isomorphic to $\D \times \D\sop$ with exchange superinvolution and natural grading refined as in \cref{thm:refinement-DxDsop} if, and only if, there is an order $2$ element $f\in \rad \tilde\beta_{\mc E}$ such that $\eta_{\mc E}(f) = -1$. 
\end{prop}

\begin{proof}
    For the ``only if'' direction, let $(T, \beta, p, \eta)$ be the quadruple associated to $\D$. 
    By item \eqref{item:there-is-extension} of \cref{thm:refinement-DxDsop}, the element $f \coloneqq \iota(-1)$ has order $2$ and satisfies $\eta_{\mc E} (f) = -1$. 
    Also, $f \in \ker \pi$, so $f \in \rad \tilde\beta_{\mc E}$.
    
    For the ``if'' direction, define $T \coloneqq T_{\mc E} / \langle f \rangle$, let $\pi\from T_{\mc E} \to T$ be the natural homomorphism, and let $\beta\from T\times T \to \pmone$ and $p\from T \to \ZZ_2$ be the maps induced by $\beta_{\mc E}$ and $p_{\mc E}$, respectively. 
    They are well defined since $f\in \rad \tilde\beta = (\rad \beta) \cap T^+$. 
    Thus, $T_{\mc E}$ fits into an extension $\pmone \xrightarrow{\iota}  T_{\mc E} \xrightarrow{\pi} T$ where $\iota (-1) = f$. 
    For each $t \in T$, there precisely $2$ elements of $T_{\mc E}$, $t'$ and $t''$, with image $t$ under $\pi$, but $\eta(t') = -\eta(t'')$ since $t'' = t'f$. 
    We define a set-theoretic section $\tau\from T \to T_{\mc E}$ of $\pi$ by taking $\tau(t)$ to be the element such that $\eta(\tau(t)) = 1$. 
    Then the map $\alpha\from T_{\mc E} \to \{ \pm 1 \} \times_{\tilde\beta} T$ given by $\alpha(\tau(t)) = (1, t)$ and $\alpha(f\tau(t)) = (-1, t)$, for all $t\in T$, is an equivalence from $T_{\mc E}$ to the extension in item \eqref{item:pmone-x-T} in \cref{thm:refinement-DxDsop} (compare with the proof of \cref{prop:all-ext-are-cocycle}). 
\end{proof}

We conclude with a result that will be useful in \cref{chap:Lie}. 

\begin{prop}\label{prop:lemma-for-undoubling-and-fine-gradings}
    Let $(\mc E, \vphi_0)$ be as in \cref{lemma:undoubling-D}, with $f\in \ker \tilde \beta$ and $\eta(f) = -1$. 
    Let $\V$ be a graded right $\mc E$-supermodule associated to a map $\kappa\from G/T_{\mc E} \to \ZZ_{\geq 0}$ and suppose there is a superinvolution $\vphi$ on $R \coloneqq \End_{\mc E} (\V)$ determined by a nondegenerate $\vphi_0$-sesquilinear form on $\V$. 
    Set $\overline{G} \coloneqq G/\langle f \rangle$, $\barr G^\# \coloneqq \barr G \times \ZZ_2 \iso G^\#/\langle f \rangle$, $\overline{T_{\mc E}} \coloneqq T_{\mc E}/\langle f \rangle$,  and let $\pi\from G \to \overline{G}$ be the natural homomorphism and let $\barr \beta$ and $\bar p$ be the maps induced by $\beta$ and $p$, respectively.
    Then the coarsening $({}^\pi R, \vphi)$ is isomorphic to $S \times S\sop$ with exchange superinvolution, where $S \iso E(\overline{T_{\mc E}}, \barr \beta, \bar p, \kappa)$,  seeing $\kappa$ as a map from $\barr G^\#/\barr T \iso G^\#/T$ to $\ZZ_{\geq 0}$. 
\end{prop}

\begin{proof}  
    Recall that $\rad \tilde\beta = \supp Z(\mc E)\even$. 
    Let $\zeta \in Z(\mc E)\even$ be an element with degree $f$ and, scaling it if necessary, we may assume that $\zeta^2 = 1$. 
    Since $\eta(f) = -1$, $\vphi_0(\zeta) = - \zeta$. 
    By \cref{prop:R-and-D-have-the-same-center-vphi}, we can identify $(Z(\mc E)\even, \vphi_0)$ and $(Z(R)\even, \vphi)$, so $\zeta \in Z(R)\even$ with $\vphi(\zeta) = -\zeta$ (recall that the ground field $\FF$ is algebraically closed). 
    Set $\epsilon \coloneqq \frac{1+\zeta}{2}$ and $\epsilon' \coloneqq \frac{1-\zeta}{2}$. 
    Clearly, $\epsilon$ and $\epsilon'$ are orthogonal central idempotents and $1 = \epsilon + \epsilon'$. 
    It follows that we can write $R = S \oplus S'$, a direct sum of superideals, where $S \coloneqq 
    \epsilon R$ and $S' \coloneqq \epsilon' R$. 
    If we consider on $R$ the $\barr G^\#$-grading, the elements $\epsilon$ and $\epsilon'$ are homogeneous, so $S$ and $S'$ are graded superideals. 
    Since $\vphi(\epsilon) = \epsilon'$, it follows that $\vphi(S) = S'$ and, hence, $S' \iso S\sop$ as $\barr G$-graded superalgebras. 
    It remains to prove that $S$ is graded-simple and describe its parameters. 
    
    Set $\U \coloneqq \V \epsilon$ and $\U' \coloneqq \V \epsilon'$. 
    Clearly, $\U$ and $\U'$ are $\barr G$-graded right $\mc E$-supermodules and $\V = \U \oplus \U'$. 
    For any $\mc E$-linear map $r \in R = \End_{\mc E} (\V)$, it is clear that $\epsilon r(u) = r(u)\epsilon \in \U$ for all $u\in \U$ and, since $\epsilon \epsilon' = 0$, $\epsilon r (u') = 0$ for all $u'\in \U'$. 
    It follows that we can identify $S = \epsilon R$ with $\End_{\mc E}(\U)$. 
    
    Set $\D \coloneqq \epsilon \mc E$. 
    Restricting the action of $\mc E$ on $\U$, we can consider $\U$ as a $\D$-supermodule. 
    We claim that if $f\from \U \to \U$ is $\D$-linear, then it is also $\mc E$-linear. 
    Indeed, $u = u\epsilon$ for every $u \in \U$, so
    $f(ud) = f(u\epsilon d) = f(u) \epsilon d = f(u) d$. 
    We conclude that $\End_{\mc E} (\U) = \End_\D(\U)$. 
    
    We claim that $\D$ is a graded-division superalgebra. 
    Clearly, $\epsilon \in \D$ is the identity element. 
    For every $t \in T_{\mc E}$, choose $0\neq X_t \in \mc E_t$. 
    Note that, for every $t \in T_{\mc E}$, we have $X_{ft} \in \FF \zeta X_t$ and, hence, $\epsilon X_{ft} \in \FF X_t$ since $\epsilon\zeta = \epsilon$. 
    It follows that the $\barr G$-homogeneous component $\D_{\bar t}$ is spanned by $\epsilon X_t$. 
    Since $\epsilon X_t$ is invertible in $\D$, with inverse $\epsilon X_t\inv$, 
    we have that $\D$ is a graded-division superalgebra and that $\supp \D = \overline{T_{\mc E}}$. 
    Also, $\epsilon X_s \epsilon X_t = \epsilon X_s X_t = \beta(s,t) \epsilon X_t X_s = \beta(s,t) \epsilon X_t \epsilon X_s$, for all $s, t\in T$, so $\D$ is associated to $(\overline{T_{\mc E}}, \bar \beta, \bar p)$. 
    
    Finally, to see that the map $\kappa\from G^\#/T_{\mc E} \iso \barr G^\#/ \overline{T_{\mc E}} \to \ZZ_{\geq 0}$ is associated to the graded right $\D$-supermodule $\U$, let $\{v_1, \ldots, v_k\}$ be a $G$-graded $\mc E$-basis for $\V$. 
    We claim that $\mc B = \{v_1 \epsilon , \ldots, v_k \epsilon \}$ is a $\barr G^\#$-graded $\D$-basis for $\U$. 
    Indeed, for every $v \in \V$, write $v = v_1 d_1 + \cdots + v_k d_k$, where $d_1, \ldots, d_k \in \mc E$. 
    Then $v \epsilon = v_1 d_1 \epsilon + \cdots + v_k d_k \epsilon = (v_1 \epsilon) (d_1 \epsilon) + \cdots + (v_k \epsilon) (d_k \epsilon)$. 
    We conclude that $\U$ is the $\D$-span of $\mc B$. 
    For the $\D$-linear independence, we note that if
    $(v_1 \epsilon) (d_1 \epsilon) + \cdots + (v_k \epsilon) (d_k \epsilon) = 0$, then $v_1 d_1 \epsilon + \cdots + v_k d_k \epsilon = 0$, so $d_1 \epsilon = \cdots = d_k \epsilon = 0$. 
\end{proof}

\begin{remark}\label{rmk:S-iso-Ssop-in-this-case}
    The $\overline{G}$-graded superalgebra $S' \iso S\sop$ is also isomorphic to $E(\overline{T_{\mc E}}, \barr \beta, \bar p, \kappa)$, so $S$ admits a super-anti-automorphism. 
    % This would not be true if $\FF = \RR$. 
    Nevertheless, we cannot construct one canonically. 
    We will revisit this point in \cref{ssec:undoubling}.
\end{remark}
