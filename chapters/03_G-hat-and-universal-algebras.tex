\chapter{$G$-Gradings on Universal Algebras and $\widehat G$-actions}

In this chapter we will revisit the correspondence between $G$-gradings and $\widehat G$-actions, which can be found in many places in the literature (see, \eg, ??, ?? and ??). 
Nevertheless, our goal is to generalize it to the context of universal algebras. 
This way we can handle algebras and superalgebras, with or without (super)involutions, in a uniform manner. 

Another advantage of this approach is to have a formal result comparing gradings on univeral algebras with different signature (see Definition \ref{def:universal-algebra} and Theorem ??). 
This will be used in Chapter ?? to transfer gradings on associative superalgebras with superinvolutions to Lie superalgebras. 
The same sort of transfer has been used in other works (see \cite{livromicha} and Paper with Adrian ??), but without having the result stated properly. 

\section{Universal Algebras}

A universal algebra is a space with an arbitrary number (possibly none or infinity) of operations, with may also have different ``arities'' (see Definition \ref{def:universal-algebra} bellow). 
Our approach to universal algebras differs from the one in most books on the subject (see, \eg, ?? and ??) in the sense we will be working on the (monoidal) category of vector spaces instead of the (monoidal) category of sets. 
This is the approach on ?? (book used by Felipe) and it has recently been applied to gradings (and graded identities) on ??.

\begin{notation}
    For a vector space $A$ and nonnegative integer $n$, we will denote the $n^{th}$-tensor power of $A$ by $A^{\tensor n}$, \ie,
    $A^{\tensor n} \coloneqq \underbrace{A\otimes\cdots\otimes A}_{n \text{ times}}$. 
    In the case $n = 0$, we will follow the convention that $A^{\tensor 0} \coloneqq \FF$.
\end{notation}

\begin{defi}\label{def:universal-algebra}
    Let $A$ be a vector space. 
    An \emph{$n$-ary}
\end{defi}


\section{Old - Correspondence between $G$-gradings and\\ $\widehat G$-actions}\label{ssec:G-hat-action}

One of the important tools for dealing with gradings by abelian groups on (super)algebras is the well-known correspondence between  $G$-gradings and $\widehat G$-actions (see e.g. \cite[\S 1.4]{livromicha}), where $\widehat G$ is the algebraic group of characters of $G$, \ie, group homomorphisms $G \rightarrow \FF^{\times}$. The group $\widehat{G}$ acts on any $G$-graded (super)algebra $A = \bigoplus_{g\in G} A_g$ by $\chi \cdot a = \chi(g) a$ for all $a\in A_g$ (extended to arbitrary $a\in A$ by linearity). The map given by the action of a character $\chi \in \widehat{G}$ is an automorphism of $A$. If $\FF$ is algebraically closed and $\Char \FF = 0$, then $A_g = \{ a\in A \mid \chi \cdot a = \chi (g) a\}$, so the grading can be recovered from the action.

For example, if $A=A\even \oplus A\odd$ is a superalgebra, the action of the nontrivial character of $\ZZ_2$ yields the \emph{parity automorphism} $\upsilon$, which acts as the identity on $A\even$ and as the negative identity on $A\odd$. If $A$ is a $\ZZ$-graded algebra, we get a representation $\widehat \ZZ = \FF^\times \rightarrow \Aut (A)$ given by $\lambda \mapsto \upsilon_\lambda$ where $\upsilon_{\lambda} (x) = \lambda^i x$ for all $x\in A^i$, $i\in \ZZ$.

A grading on a (super)algebra over an algebraically closed field of characteristic $0$ is said to be \emph{inner} if it corresponds to an action by inner automorphisms. For example, the inner gradings on $\Sl(n)$ (also known as Type I gradings) are precisely the restrictions of gradings on the associative algebra $M_n(\FF)$.
