\section{Odd gradings on simple associative superalgebras only in terms of \texorpdfstring{$G$}{G}}\label{sec:assc-only-G}

Here $\FF$ is algebraically closed and $\D$ is finite dimensional.

\subsection{Odd graded-simple superalgebras in terms of $G$}\label{ssec:odd-div-G-only}

Let $\D$ be an odd finite dimensional graded-division superalgebra associated to $(T, \beta, p)$. 
Clearly, $\D\even$ is a graded-division algebra associated to $(T^+, \beta^+)$. 
Given any $t_1 \in T^-$, then $t_1 = (h, \bar 1)$ for some $h\in G$ such that $h^2\in T^+$ and the map $\chi \from T^+ \to \FF^\times$ given by $\chi(t) \coloneqq \beta(t_1, t)$, for all $t\in T^+$, is a character in $\widehat{T^+}$ such that $\chi^2 = \beta^+ (h^2, \cdot)$. 

\begin{defi}
    We say that the odd graded-division superalgebra $\D$ is a \emph{associated} to the quadruple $(T^+, \beta^+, h,\chi)$. 
    If $\U$ is a graded $\D$-supermodule of finite rank associated to the map $\kappa\from G/T^+ \to \FF^\times$, we say that $(T^+, \beta^+, h, \chi, \kappa)$ are $G$-parameters of the pair $(\D, \U)$. 
\end{defi}

Clearly, a quadruple $(T^+, \beta^+, h,\chi)$ associated to $\D$ has enough information to recover $T$ and $\beta$, since $T^- = (h, \bar 1) T^+$ and
\[\label{eq:beta-from-h-chi}
    \forall s,t\in T^+,\, \forall i,j \in \ZZ, \quad \beta(s\, t_1^i,t \, t_1^j) = \beta^+(s,t)\, \chi (s)^{-j}\, \chi (t)^i,
\]
and, therefore, determines the isomorphism class of the graded-division superalgebra $\D$ (see \cref{ssec:T-beta-p}). 

\begin{defi}\label{def:odd-parameters}
    Let $T^+$ be any finite subgroup of $G$ and let $\beta^+$ be an alternating bicharacter on $T^+$. 
    A pair $(h,\chi) \in G \times \widehat{T^+}$ is said to be \emph{$(T^+, \beta^+)$-admissible} if 
    $h^2 \in T^+$ and $\chi^2 = \beta^+(h^2, \cdot)$. 
    We will denote the set of all $(T^+, \beta^+)$-admissible pairs by $\mathbf {O} (T^+, \beta^+)$. 
\end{defi}

For each $(T^+, \beta^+)$-admissible pair $(h, \chi)$ we can construct a triple $(T, \beta, p)$ corresponding to a odd graded-division superalgebra. 
First, we set $t_1 \coloneqq (h, \bar 1)$ and define $T \coloneqq T^+ \cup t_1 T^-$. 
Note that $T$ is a subgroup of $G^\#$ since $h^2 \in T^+$. 
We then take $\beta\from T\times T \to \FF^\times$ as in \cref{lemma:existence-beta}, below:

\begin{lemma}\label{lemma:existence-beta}
    There is a unique alternating bicharacter $\beta$ on $T$ such that $\beta\restriction_{T^+ \times T^+} = \beta^+$ and $\beta(t_1, t) = \chi(t)$, for all $t\in T^+$.
\end{lemma}

\begin{proof}
    Clearly, a bicharacter $\beta$ satisfies the conditions above if, and only if, \cref{eq:beta-from-h-chi} is satisfied. 
    It follows that there is at most one such bicharacter. 
    
    % One could also use \cref{eq:beta-from-h-chi} as a definition for $\beta$, but it is not clear if it is well defined. 
    
    Let $\widetilde T$ be the direct product of $T^+$ and the infinite cyclic group generated by a new symbol $\tau$, and let $\phi \from \widetilde T \to T$ be the unique group homomorphism such that $\phi(t) = t$ for all $t\in T^+$ and $\phi(\tau) = t_1$. 
    It is easy to see that $\ker \phi = \langle h^2 \tau^{-2} \rangle$ and that $\phi$ is surjective. 
    
    Let us define $b\from \widetilde T\times \widetilde T \rightarrow \FF^\times$ by $b(s\tau^i,t\tau^j) \coloneqq \beta^+(s,t)\, \chi (s)^{-j}\, \chi (t)^i$, for all $s,t\in T^+$ and $i,j \in \ZZ$. 
    Clearly, $b$ is an alternating bicharacter. 
    We claim that $h^2 \tau^{-2} \in \rad b$. 
    Indeed, for all $t \in T^+$ and $j \in \ZZ$,
    \begin{align}
        b(h^2 \tau^{-2},t\tau^j) = \beta^+(h^2,t)\, \chi (h^2)^{-j}\, \chi (t)^{-2} &= \chi (t)^{2}\, \chi (h^2)^{-j}\, \chi (t)^{-2} = \chi (h^2)^{-j} \\
        &= \chi (h^{-j})^2 = \beta^+(h^2,h^{-j}) = 1.
    \end{align}
    
    It follows that $\ker \phi \subseteq \rad b$ and, hence, $b$ induces an alternating bicharacter $\beta\from T\times T \to \FF$ such that $b = \beta \circ (\phi \times \phi)$. 
    It is clear that $\beta$ satisfies \cref{eq:beta-from-h-chi}, concluding the proof.
\end{proof}

Together with \cref{lemma:colour-tensor-product}, we have:

\begin{prop}
    Let $T^+$ and $\beta^+$ be as in \cref{def:odd-parameters} and let $(h, \chi)$ be a $(T^+, \beta^+)$-admissible pair. 
    Then there exist an odd graded-division superalgebra $\D$ associated to the quadruple $(T^+, \beta^+, h,\chi)$. \qed
\end{prop}

We now have a parametrization of odd graded-division superalgebras only in terms of $G$. 
It remains to see when different quadruples determine isomorphic graded-division superalgebras. 

\begin{defi}\label{def:T^+-action}
    Let $T^+$ and $\beta^+$ be as in \cref{def:odd-parameters}, we define an $T^+$-action on $\mathbf{O} (T^+, \beta^+)$ by
    \[
        t \cdot (h, \chi) = (th, \beta^+(t, \cdot) \chi),
    \]
    for all $t\in T^+$ and $(h, \chi) \in \mathbf{O} (T^+, \beta^+)$.
\end{defi}

If $t\in T^+$ and $(h, \chi) \in \mathbf {O} (T^+, \beta^+)$, then $(th)^2 = t^2h^2\in T^+$ and $\beta(th^2, \cdot) = \beta(t, \cdot) \chi$, hence, indeed, $t \cdot (h, \chi) \in \mathbf {O} (T^+, \beta^+)$. 
It is straightforward that the axioms of action are satisfied. 

\begin{thm}
    Let $\D$ and $\D'$ be finite dimensional odd graded-division superalgebras, associated to quadruples $(T^+, \beta^+, h, \chi)$ and $(T'^+, \beta'^+, h', \chi')$, respectively. 
    Then $\D \iso \D'$ if, and only if, $T^+ = T'^+$, $\beta^+ = \beta'^+$, and $(h', \chi')$ is in the same $T^+$-orbit of $(h, \chi)$ in $\mathbf {O} (T^+, \beta^+)$. 
\end{thm}

\begin{proof}
    Let $(T, \beta, p)$ and $(T', \beta', p')$ be the triples associated to $\D$ and $\D'$, respectively. 
    By other theorem, $\D \iso \D'$ if, and only if, $T= T'$ and $\beta = \beta'$. 
    
    By definition, $T= T'$ if, and only if, $T^+ = T'^+$ and $(h, \bar 1)T^+ = (h', \bar 1)T^+$. 
    Note that the latter condition is equivalent to 
    % Hence $T= T'$ if, and only if, $T^+ = T'^+$ and 
    $h' = th$ for some $t\in T^+$. 
    
    Now suppose $T= T'$ and let $t\in T^+$ be such that $h' = th$. 
    Clearly, $\beta = \beta'$ if, and only if, 
    $\beta^+ = \beta'^+$ and $\beta((h', \bar 1), \cdot) = \beta'((h', \bar 1), \cdot)$. 
    By definition of $\beta'$, the latter condition is equivalent to $\chi' = \beta((h', \bar 1), \cdot)$. 
    Since $\beta((h', \bar 1), \cdot) = \beta ((th, \bar), \cdot) = \beta^+ (t, \cdot) \beta ((h, \bar 1), \cdot) = \chi$, we have the desired result. 
\end{proof}

\begin{cor}
    Iso for any odd grading.
\end{cor}

For specific cases, this parametrization can be simplified using the following lemma about group actions: 

\begin{lemma}\label{lemma:lemma-on-actions}
    Let $H$ be a group, let $X$ and $Y$ be $H$-sets, and let $\pi\from X \to Y$ be an $H$-equivariant map. 
    For every $y \in Y$, the $H$-action on $X$ restricts to a $\operatorname{Stab}_H (y)$-action on $\pi\inv(y)$ and, if the $H$-action on $Y$ is transitive, the inclusion map $\pi\inv(y) \hookrightarrow X$ induces a bijection between $\operatorname{Stab}_H (y)$-orbits in $\pi\inv(y)$ and $H$-orbits in $X$.  
\end{lemma}

\begin{proof}
    The first assertion is straightforward: if $x\in \pi\inv(y)$ and $g \in \operatorname{Stab}_H (y)$, then $g \cdot x \in \pi\inv(y)$. 
    
    Now suppose that the $H$-action on $Y$ is transitive. 
    It is clearly that the $\operatorname{Stab}_H (y)$-orbit of any $x\in \pi\in(y)$ is contained in the $H$-orbit of $x$. Conversely, given $x\in X$, by transitivity, there is $g\in H$ such that $g \cdot \pi(x) = y$ and, hence, $g\cdot x \in \pi\inv (y)$. 
    We claim that $\operatorname{Stab}_H (y)$-orbit of $g\cdot x$ is the only contained in the $H$-orbit of $x$. 
    % We claim the $\operatorname{Stab}_H (y)$-orbit of $g\cdot x$ is the only $\operatorname{Stab}_H (y)$-orbit contained in the $H$-orbit of $x$. 
    Indeed, if for some $g'\in H$ we have $g'\cdot x \in \pi\inv(y)$, then $g'\cdot x = (g'g\inv)\cdot (g\cdot x)$ and, hence, $y = \pi(g'\cdot x) = (g'g\inv)\cdot \pi(g\cdot x) = (g'g\inv) \cdot y$, so $g'g\inv \in \operatorname{Stab}_H (y)$.
\end{proof}

As an application of this lemma, we will revisit the classification of graded-division superalgebras of type $Q$. 

fix $(T^+, \beta^+)$ with $\beta^+$ nondegenerate. 
Consider the action of $T^+$ on $\widehat{T^+}$ given by $t\cdot \chi = \beta^+(t,\cdot) \chi$ and let $\pi\from \mathbf {O} (T^+, \beta^+) \to \widehat{T^+}$ be the projection on the second entry, which clearly a $\widehat{T^+}$-equivariant map. 

Since $\beta^+$ is nondegenerate, the $T^+$-action on $\widehat{T^+}$ is transitive. 
Indeed, for every $\chi \in \widehat{T^+}$, there is $t \in T^+$ such that $\chi = \beta(t, \cdot)$, so $\chi = t \cdot 1$, where $1\in \widehat{T^+}$ is the trivial character. 

The nondegeneracy of $\beta$ also implies that $\operatorname{Stab}_{T^+} (1) = \{e \}$. 
By \cref{lemma:lemma-on-actions}, it follows that the $T^+$-orbits in $\mathbf {O} (T^+, \beta^+)$ are in bijection with points in $\pi\inv (1)$. 
By the definition of $\mathbf {O} (T^+, \beta^+)$, $(h, 1) \in G \times \widehat{T^+}$ if, and only if, $h^2 \in T^+$ and $h^2 \in \ker \beta^+$, hence, if, and only if, $h^2 = e$. 

Our theorem reduces to ... and $h = h'$. 
Note that in this case $(h, \bar 1) \in \ker \beta$, so in view of \cref{lemma:beta-deg-beta-tilde-nondeg}, $\ker \beta = \langle (h, \bar 1) \rangle$ and $\D$ is isomorphic to the standard realization of ... (see \cref{def:standard-realization-Q}).

\subsection{\texorpdfstring{$M(n,n)$}{M(n,n)}}

Let $\D$ be an odd finite dimensional graded-division superalgebra associated to a triple $(T, \beta, p)$. 
By \cref{prop:char-does-not-divide-T}, below, if $\Char \FF = 2$, then $\D$ cannot be of type $M$. 
Note that, since $\D$ is odd, $|T| = |T^+| + |T^-| = 2|T^+|$.

\begin{prop}\label{prop:char-does-not-divide-T}
    Suppose $\Char \FF = p > 0$. 
    If $p \mid |T|$, then $\beta$ is degenerate.
\end{prop}

\begin{proof}
    Let $t\in T$ be an element of order $p$. 
    For every $s\in T$, we have $1 = \beta(e, s) = \beta(t^p, s) = \beta(t, s)^p$, hence, since $\Char \FF = p$, $\beta(t, s) = 1$, \ie, $t\in \rad \beta$. 
\end{proof}
 
For the remaining of this subsection, we will assume that $\Char \FF \neq 2$.

Suppose that $\D$ is of type $M$, \ie, that $\beta$ is nondegenerate. 
Propositions ?? establishes

We will now investigate necessary conditions on the pair $(T^+, \beta^+)$ for $\beta$ be nondegenerate. 
% We need to see what are the characteristics on the parameters of the general case that ensures $M(n,n)$.

% First, necessary conditions.

\begin{prop}\label{prop:parity-element}
    There is an element $t_0 \in T^+$ of order $2$ such that $\rad \beta^+ = \langle t_0 \rangle$.
\end{prop}

\begin{proof}
    Consider $\chi \in \widehat{T}$ given by $\chi(t) = (-1)^{p(t)}$. 
    Since $\beta$ is nondegenerate, there is $t_0\in T$ of order $2$ such that $\chi = \beta(t_0, \cdot)$. 
    Since $\beta(t_0, t_0) = 1 = (-1)^{p(t_0)}$, $t_0$ is even. 
    Hence $t_0 \in \rad \beta^+$. 
    
    If $0 \neq t \in \rad \beta^+$, using again that $\beta$ is nondegenerate, there is $t_1 \in T^-$ such that $\beta(t, t_1) \neq 1$. 
    Since $t_1^2 \in T^+$, $\beta(t_0, t_1)^2 = \beta(t, t_1^2) = 1$. 
    It follows that $\beta(t, t_1) = -1$, and since $T^- = t_1 T^+$, $\beta(t, s) = -1$ for all $s\in T^-$. 
    Hence $\beta(t, s) = (-1)^{p(s)}$ and, therefore, $t = t_0$.
\end{proof}

\begin{remark}
    Note that $t_0$ is the parity element (see ??).
\end{remark}

% \begin{prop}
%     Let $\Gamma$ be the $T$-grading on $\D\iso M(n,n)$, let $H$ be an abelian group and let $\alpha\from T \to H$ be a group homomorphism. 
%     The grading ${}^\alpha\Gamma$ (see def) is even grading if, and only if, $t_0 \in \ker \alpha$. 
% \end{prop}

% \begin{proof}
%     Fix an element $0\neq d_0\in \D$  of degree $t_0$. 
%     Since $t_0 \in \rad \beta$, $d_0 \in Z(\D\even)$, and 
%     since $d_0^2\in \D_e = \FF$, and $\FF$ is algebraically closed, we may rescale $d_0$ so that $d_0^2=1$. 
%     Then $\epsilon := \frac{1}{2}(1+d_0)$ is a central idempotent of $\D\even$. 
%     % = M(n,n)\even \iso M(n) \oplus M(n)$. 
%     Of course, $1-\epsilon$ is another central idempotent of $\D\even$, and $\D\even = \epsilon \D\even \oplus (1-\epsilon)\D\even$ (direct sum of ideals).
    
%     Let $0\neq d_1\in\D\odd$ be a $T$-homogeneous element. 
%     Then $d_1\epsilon d_1\inv = \frac{1}{2}(1-d_0)=1-\epsilon$, which is another central idempotent of $\D\even$ and must have the same rank as $\epsilon$. 
%     Hence, $\D\even \iso \epsilon\D\even \oplus (1-\epsilon)\D\even$ (direct sum of ideals) and, consequently, $E\even \iso \End(\tilde U)\tensor \D\even = \End(\tilde U)\tensor \epsilon\D\even \oplus \End(\tilde U)\tensor (1-\epsilon)\D\even$, where the two summands have the same dimension. Therefore, odd gradings exist only if $m=n$. 
% Also note that we have 
% \begin{equation}\label{eq:D1eps}
% \D\odd \epsilon = (1-\epsilon) \D\odd.
% \end{equation}
% \end{proof}

In vague terms, this is because $(T^+, \beta^+)$ carries some information restricting which elements can be squares of an element $t_1 \in T^-$. 

We are now in position to restrict the general result ?? to the our specific case. 
Let $T^+$ be a finite subgroup of $G$ and $\beta^+$ be an alternating bicharacter on $T^+$ such that $\rad \beta^+ = \langle t_0 \rangle$, where $t_0\in T^+$ is an element of order $2$. 

% As we are going to see (\cref{ex:T+-but-no-T}), the condition that $|\rad \beta^+| = 2$ is not sufficient for us to find a pair $(h, \chi) \in \mathbf{O}(T^+, \beta^+)$ such that $(T^+, \beta^+, h, \chi)$ is associated to a division grading on $M(n, n)$. 

From now on, let $\barr G \coloneqq G/\langle t_0 \rangle$, $\theta\from T^+\rightarrow \barr T=\frac{T^+}{\langle t_0 \rangle}$ be the natural homomorphism, $\barr T^+ \coloneqq \theta(T^+)$ and $\bar\beta^+$ be the bicharacter on $\barr T^+$ induced by $\beta^+$. 
Note that $\bar\beta^+$ is nondegenerate.

\begin{lemma}\label{lemma:motivation-O_M}
    Let $(h, \chi) \in \mathbf{O}(T^+, \beta^+)$ and let $\D$ be a graded-division superalgebra associated to $(T^+, \beta^+, h, \chi)$. 
    Then $\D$ is of type $M$ if, and only if, $\chi(t_0) = -1$.
\end{lemma}

\begin{proof}
    Easy proof.
\end{proof}

\begin{defi}
    We define $\mathbf{O_M}(T^+, \beta^+) \coloneqq \{ (h, \chi) \in \mathbf{O}(T^+, \beta^+) \mid \chi(t_0) = -1 \}$.
\end{defi}

\begin{cor}
    Iso without existence
\end{cor}

We note that the $T^+$-action on $\mathbf{O}(T^+, \beta^+)$ restricts to $\mathbf{O_M}(T^+, \beta^+)$ since $t_0\in \rad \beta^+$. 
In view of \cref{lemma:lemma-on-actions}, we can simplify the description of the orbits of this action. 

Set $Y \coloneqq \{\chi \in \widehat{T^+} \mid \chi (t_0)\}$ with $T^+$-action defined by $t\cdot \chi = \beta^+(t, \cdot) \chi$. 
Again, this action is well-defined since $t_0 \in \rad \beta^+$. 

\begin{lemma}
    We have that $Y \neq \emptyset$ and that the $T^+$-action on $Y$ is transitive.
\end{lemma}

\begin{proof}
    To see that $Y$ is no empty, note that $\beta^+$ is a nondegenerate bicharacter on $\barr T^+$ and that $|T^+| = 2 |\barr T^+|$. 
    By \cref{prop:char-does-not-divide-T}, $\Char \FF$ does not divide $|\barr T^+|$ and, since $\Char \FF \neq 2$, $\Char \FF$ does not divide $|T^+|$. 
    In this case, it is well-known that every character on a subgroup of $|T^+|$ can be extended to a character of $T^+$. 
    Clearly, $\chi(t_0) = -1$ defines a character on $\langle t_0 \rangle$, therefore $Y \neq \emptyset$. 
    
    To see that the action is transitive, let $\chi, \chi' \in Y$. 
    We have $(\chi' \chi\inv) (t_0) = 1$, hence $(\chi' \chi\inv)$ induces a character on $\barr T^+ = T^+/ \langle t_0 \rangle$. 
    By nondegeneracy of $\barr \beta^+$, there is $t\in T^+$ such that $(\chi' \chi\inv) = \beta(\bar t, \cdot)$ as a character of $\barr T^+$. 
    It follows that $(\chi' \chi\inv) = \beta(t, \cdot)$ as a character of $T^+$, so $\chi' = \beta(t, \cdot)\chi$. 
    We conclude that $\chi' = t \cdot \chi$, as desired. 
\end{proof}

Consider $\pi$...
Recall \cref{lemma:lemma-on-actions}. 
Let us see $\pi\inv (\chi)$. 

\begin{lemma}\label{lemma:chi-defines-a}
    Given $\chi \in Y$, there is a unique element $a \in T^+$ such that $\chi^2 = \beta^+(a, \cdot)$ and $\chi(a) = 1$. 
    Further, a element $h\in T^+$ is such that $(h, \chi) \in \mathbf{O_M}(T^+, \beta^+)$ if, and only if, $h^2 = a$. 
\end{lemma}

\begin{proof}
    Note that $\chi^2(t_0) = 1$, hence $\chi^2$ can be seem as a character on $\barr T^+$. 
    It follows that there is $a\in T^+$ such that $\chi^2(\barr t) = \barr\beta(\barr a, \barr t)$ for all $\barr t\in \barr T$. 
    In particular $\chi^2(a) = 1$, so $\chi (a) = \pm 1$. 
    If $\chi(a) = 1$, we put $a_\chi \coloneqq a$, else we put  $a_\chi \coloneqq at_0$. 

    The ``further'' part follows from definition of $\mathbf{O_M}(T^+, \beta^+)$. 
\end{proof}

Recall that in the case $\beta^+$ nondegenerate, in \cref{ssec:odd-div-G-only}, we were able to fix the character to be the trivial one. 
We can also fix the character $\chi$ such that $\chi(t_0) = -1$ in this case, but there is no canonical choice for it. 

\begin{defi}
    Given $\chi \in \widehat{T^+}$ such that $\chi(t_0) = -1$, we define $\mathbf{O_M}(T^+, \beta^+, \chi) \coloneqq \{ h\in G \mid (h, \chi)\in \mathbf{O}(T^+, \beta^+) \mid \chi(t_0) = -1 \}$.
\end{defi}

\begin{cor}\label{cor:iso-odd-M-simplified}
    Apply \cref{lemma:lemma-on-actions} for $\pi\from \mathbf{O}(T^+, \beta^+) \to Y$.
\end{cor}

\Cref{cor:iso-odd-M-simplified} does not gives us the full picture. 
It can be the case (see \cref{ex:T+-but-no-T}) that $\mathbf{O_M}(T^+, \beta^+)$ is empty, \ie, there might be no graded-division superalgebra $\D$ of type $M$ such that $\D\even$ is associated to the pair $(T^+, \beta^+)$.

\begin{defi}
	For every abelian group $H$, we define $H^{[2]} \coloneqq \{h^2 \mid h\in H\}$ and $H_{[2]} \coloneqq \{h\in H \mid h^2 = e \}$.
\end{defi}

If there is $(T, \beta, p)$, then it is clear that $(T^+)^{[2]} \subseteq T^{[2]} \subseteq T^+$, but we can have $(T^+)^{[2]} \neq T^{[2]}$. 

\begin{defi} 
    For any subgroup $A\subseteq \barr{T^+}$, we define its \emph{orthogonal complement} to be
    \[
        A^\perp \coloneqq \{t\in T\mid \bar\beta^+ (t, A) =1\}.
    \]
\end{defi}

\begin{lemma}
    For every subgroup $A \subseteq \barr T^+$, $|A^\perp| = [\barr T^+ : A]$ and $(A^\perp)^\perp = A$.
\end{lemma}

\begin{proof}
    It is easy to see that the map $A^\perp \to \widehat{\left( \frac{\barr T^+}{A}\right)}$ given by $t \mapsto \beta(t, \cdot)$ is an isomorphism of groups. 
    Also, since $\beta$ is nondegenerate, by \cref{prop:char-does-not-divide-T},  $\Char \FF$ does not divide the order of $\frac{\barr T^+}{A}$. 
    In this case, it is well-known that $\widehat{\left( \frac{\barr T^+}{A}\right)} \iso \frac{\barr T^+}{A}$, proving the first assertion.

    Since $\bar\beta^+$ is skew-symmetric, it is clear that $A \subseteq (A^\perp)^\perp$. 
    Using the first assertion, we have that $|A| = |(A^\perp)^\perp|$ and, therefore, $A = (A^\perp)^\perp$.
\end{proof}


\begin{prop}\label{prop:small-perp}
    Consider the subgroups of $\barr T^+$ defined by $A \coloneqq \theta(T^+)_{[2]}$ and $B \coloneqq \theta(T^+)^{[2]}$. 
    Then $A^\perp = B$.
\end{prop}

\begin{proof}
    Let us first prove that $B^\perp = A$:
    %
	\begin{align}
		B^\perp & = \{ \theta(t) \mid t\in T^+ \AND \barr\beta^+ (\theta (t), \theta (s^2)) = 1 \text{ for all } s\in T^+ \}\\ 
		& = \{\theta(t) \mid t\in T^+ \AND \beta^+ (t, s^2) = 1 \text{ for all } s\in T^+\}\\ 
		& = \{ \theta(t) \mid t\in T^+ \AND \beta^+ (t^2, s) = 1 \text{ for all } s\in T^+\}\\ 
		& = \{ \theta(t) \mid t\in T^+ \AND \theta(t)^2=e \} \numberthis\label{4th-line-again} \\ 
		&= A,
	\end{align}
	%
	where we are using that $\beta^+$ is nondegenerate in the fourth line \eqref{4th-line}. 
	By \cref{lemma:perp-perp}, $A^\perp = (B^\perp)^\perp = B$, concluding the proof.
\end{proof}

\begin{prop}\label{prop:square-subgroup}
    Suppose the pair $(T^+, \beta^+)$ extends to a triple $(T, \beta, p)$ with $\beta$ nondegenerate. 
	Consider the subgroups $S \coloneqq \theta(T^{[2]})$ and $R \coloneqq \theta(T^+_{[2]})$ of $\barr T^+$. 
	Then $R^\perp = S$ and, in particular, $\barr R^\perp \subseteq \barr G^{[2]}$.
\end{prop}

\begin{proof}
	Let us first prove that $S^\perp = R$:
	%
	\begin{align}
		S^\perp & = \{ \theta(t) \mid t\in T^+ \AND \barr\beta^+ (\theta (t), \theta (s^2)) = 1 \text{ for all }s\in T\}\\ 
		& = \{\theta(t) \mid t\in T^+ \AND \beta^+ (t, s^2) = 1 \text{ for all }s\in T\}\\ 
		& = \{ \theta(t) \mid t\in T^+ \AND \beta (t^2, s) =1 \text{ for all }s\in T\}\\ 
		& = \{ \theta(t) \mid t\in T^+ \AND t^2=e \} \numberthis\label{4th-line} \\ 
		&= R,
	\end{align}
	%
	where we are using that $\beta$ is nondegenerate in the fourth line \eqref{4th-line}. 
	By \cref{lemma:perp-perp}, $R^\perp = (S^\perp)^\perp = S$, concluding the proof.
\end{proof}


\begin{prop}
    Let $R \coloneqq \frac{T^+_{[2]}}{\langle t_0 \rangle}$. 
    The set $\mathbf{O_M}(T^+, \beta^+)$ is non-empty if, and only if, $R^\perp \subseteq \barr G^{[2]}$.
\end{prop}

\begin{proof}
    The only if part follows from \cref{prop:square-subgroup,lemma:motivation-O_M}. 
    
    For the ``if'' part, let $\chi \in Y$ and $a$ as in \cref{lemma:chi-defines-a}. 
    We claim that $\barr a \in R^\perp$. 
	Indeed, if $b \in T^+_{[2]}$, then $\barr\beta^+( \theta(a) , \theta(b) ) = \chi^2 (b) = \chi (b^2) = \chi (e) = 1$. 
	By our assumption, we conclude that $\theta(a) \in \barr G^{[2]}$. 
	
	We claim that, actually, $a\in G^{[2]}$. 
	Indeed, let $h\in G$ such that $\theta(h)^2 = \theta(a)$. 
	Then, either $a = h^2$ or $a = h^2t_0$ and, in particular, $h^2 \in T^+$. 
% 	If $a=h^2$, the claim is established, so let us suppose $a=h^2t_0$. 
	If $t_0 \in G^{[2]}$, then both $h^2$ and $h^2 t_0$ are in $G^{[2]}$, and the claim is established. 
	Otherwise, $\theta(T^+_{[2]}) = \theta(T^+)_{[2]}$ and, by \cref{prop:small-perp}, .
	
	the restriction of $\theta$ to $G^{[2]}$ is injective. 
	Clearly, $\theta(a) \in $
	Hence $a$
	
	\cref{prop:small-perp}
	
	Hence $R^\perp = (\barr T^+_{[2]})^\perp = \barr T^{[2]} = \theta ((T^+)^{[2]})$. 
	Thus, in this case, we can assume $h\in T^+$. 
	Then $\chi(h^2) = \chi^2(h) = \barr \beta (\barr a, \barr u) = \barr \beta (\barr h^2, \barr h) =1$, hence $h^2 = a$. 
%	
% 	Finally, we set $t_1=(h,\barr 1) \in G^\#$. 
% 	Since $\chi^2(t_0)=1$, we can consider $\chi^2$ as a character of the group $\barr T = \frac{T^+}{\langle t_0 \rangle}$, hence there is $a\in T^+$ such that $\chi^2(\barr t) = \barr\beta(\barr a, \barr t)$ for all $\barr t\in \barr T$. 
\end{proof}






% % ---------
% \section{Odd gradings on simple associative superalgebras only in terms of \texorpdfstring{$G$}{G}}\label{sec:assc-supersimple-only-G}

% In this section we will present a classification, up to isomorphism, only in terms of $G$ of odd gradings on finite dimensional simple superalgebras over an algebraically closed field. 

% Note that there are no odd gradings if $\Char \FF =2$. 
% Indeed, in this case, there is no nondegenerate bicharacter on $T$ because the characteristic of the field divides $|T|=2|T^+|$. 
% From now on, we suppose $\Char \FF \neq 2$.

% As stated in \cref{ssec:wedderburn-super}, the only problem is to describe the odd graded-division superalgebras only in terms of $G$. 
% For what follows, let $\D$ be a finite dimensional odd graded-division superalgebra, associated to the triple $(T, \beta, p)$. 
% We will denote the restriction of $\beta$ to $T^+ \times T^+$. 

% In general terms, our strategy is to describe the isomorphism class of $\D$ in terms of $\D\even$ and and some additional data, \ie, describe $(T, \beta, p)$ in terms of $(T^+, \beta^+)$ and some additional data. 

% \subsection{Odd gradings on \texorpdfstring{$M(n,n)$}{M(n,n)} only in terms of \texorpdfstring{$G$}{G}}\label{ssec:second-odd}

% Assume that we have $(T, \beta, p)$ with $\beta$ nondegenerate. 

% Since $\beta$ is nondegenerate, we have an isomorphism $T\rightarrow \widehat T$ given by $t\mapsto \beta(t,\cdot)$. For a subgroup $A\subseteq T$, we denote by $A^\perp$ its orthogonal complement in $T$ with respect to $\beta$, i.e., $A^\perp = \{t\in T\mid \beta(t, A) =1\}$. In view of the above isomorphism, $|A^\perp| = [T:A]$.

% In particular, we have $(T^+)^\perp = \langle t_0 \rangle$ where $t_0$ is an element of order 2. It follows that $\beta(t_0, t) = 1$ if $t\in T^+$ and $\beta(t_0, t) = -1$ if $t\in T^-$. For this reason, we call $t_0$ the \emph{parity element} of the odd grading $\Gamma$. Note that $\rad \beta^+ = T^+\cap (T^+)^\perp = \langle t_0 \rangle$.

% Fix an element $0\neq d_0\in \D$  of degree $t_0$. By the definition of $\beta$, $d_0$ commutes with all elements of $\D\even$ and anticommutes with all elements of $\D\odd$. Since $d_0^2\in \D_e = \FF$, and $\FF$ is algebraically closed, we may rescale $d_0$ so that $d_0^2=1$. Then $\epsilon := \frac{1}{2}(1+d_0)$ is a central idempotent of $\D\even$. Take a homogeneous element $0\neq d_1\in\D\odd$. Then $d_1\epsilon d_1\inv = \frac{1}{2}(1-d_0)=1-\epsilon$, which is another central idempotent of $\D\even$ and must have the same rank as $\epsilon$. 
% Hence, $\D\even\iso \epsilon\D\even\oplus (1-\epsilon)\D\even$ (direct sum of ideals) and, consequently, $E\even \iso \End(\tilde U)\tensor \D\even = \End(\tilde U)\tensor \epsilon\D\even \oplus \End(\tilde U)\tensor (1-\epsilon)\D\even$, where the two summands have the same dimension. Therefore, odd gradings exist only if $m=n$. 
% Also note that we have 
% \begin{equation}\label{eq:D1eps}
% \D\odd \epsilon = (1-\epsilon) \D\odd.
% \end{equation}

% We are now going construct an even grading by coarsening a given odd grading. The reverse of this construction will be used in Subsection \ref{ssec:second-odd}.

% Let $H$ be a group and suppose we have an even grading $\Gamma'$ on $M(n,n)$ that is the coarsening of $\Gamma$ induced by a group homomorphism $\alpha: G\rightarrow H$. Since $\Gamma'$ is even, then the idempotent $\id_{\tilde U}\tensor\epsilon$ must be homogeneous with respect to $\Gamma'$. This means that $\alpha(t_0)=e$, so $\alpha$ factors through $\barr G := G/\langle t_0 \rangle$. This motivates the following definition:

% \begin{defi}
% 	Let $\Gamma$ be an odd $G$-grading on $M(n,n)$ with parity element $t_0$. 
% 	The \emph{finest even coarsening of $\Gamma$} is the $\barr G$-grading ${}^\theta \Gamma$, where $\barr G := G/\langle t_0 \rangle$ and $\theta: G \to \barr G$ is the natural homomorphism.
% \end{defi}

% % Thm: this is the finest even coarsening
% \begin{thm}
% 	Let $\Gamma = \Gamma(T, \beta, \gamma)$ be an odd grading on $M(n,n)$ with parity element $t_0$. Then its finest even coarsening is isomorphic to $\barr \Gamma = \Gamma(\barr T, \barr \beta, \barr \gamma, \barr u\barr \gamma)$, where $\barr T= \frac{T^+}{\langle t_0 \rangle}$, $\barr\beta$ is the nondegenerate bicharacter on $\barr T$ induced by $\beta^+$, $\barr\gamma$ is the tuple whose entries are the images of the entries of $\gamma$ under $\theta$, and $u \in G$ is 
% 	any element such that $(u, \barr 1) \in T^-$.
% \end{thm}

% \begin{proof}
% 	% D now decomposes as left module
% 	Let us focus our attention on the $G$-graded division algebra $\D$. We now consider it as a $\barr G$-graded algebra, which has a decomposition $\D=\D\epsilon \oplus \D(1-\epsilon)$ as a graded left module over itself.

% 	% Claim 1: D\epsilon is simple
% 	\setcounter{claim}{0}
% 	\begin{claim}
% 		The $\D$-module $\D\epsilon$ is simple as a graded module.
% 	\end{claim}

% 	% Proof of Claim 1
% 	To see this, consider a nontrivial graded submodule $V\subseteq \D\epsilon$ and take a homogeneous element $0\neq v\in V$. Then we can write $v=d\epsilon$ where $d$ is a $\barr G$-homogeneous element of $\D$, so $d = d' + \lambda d' d_0$ where $d'$ is a $G$-homogeneous element and $\lambda\in \FF$. Hence, $v = d'\epsilon + \lambda d'd_0\epsilon = (1+\lambda)d'\epsilon$, where we have used $d_0\epsilon=\epsilon$. Clearly, $(1+\lambda)d'\neq 0$, so it has an inverse in $\D$. We conclude that $\epsilon\in V$, hence $V=\D\epsilon$.\qedclaim 

% 	Let $\barr \D := \epsilon \D \epsilon \iso \End_{\D}(\D\epsilon)$, where we are using the convention of writing endomorphisms of a left module on the right. By Claim 1 and the graded analog of Schur's Lemma (see \eg \cite[Lemma 2.4]{livromicha}), $\barr \D$ is a $\barr G$-graded division algebra.

% 	% Claim 2: parameters of \barr D
% 	\begin{claim}
% 		The support of $\barr \D$ is $\barr T= \frac{T^+}{\langle t_0 \rangle}$ and the bicharacter $\barr \beta: \barr T\times \barr T\rightarrow \FF^\times$ is induced by $\beta^+: T^+\times T^+ \rightarrow \FF^\times$.
% 	\end{claim}

% 	% Proof of Claim 2
% 	We have $\barr \D = \epsilon \D\even \epsilon + \epsilon \D\odd \epsilon$ and $\epsilon \D\odd \epsilon = 0$ by Equation \eqref{eq:D1eps}, so $\supp \barr \D \subseteq \barr T$. On the other hand, for every $0\neq d\in \D\even$ with $G$-degree $t\in T^+$, we have that $\epsilon d\epsilon = d\epsilon = \frac{1}{2}(d+dd_0)\neq 0$, since the component of degree $t$ is different from zero. Hence $\supp \barr \D = \barr T$. Since $\epsilon$ is central in $\D\even$, we obtain $\barr\beta (\barr t,\barr s) = \beta (s, t) = \beta^+ (s, t)$ for all $t, s\in T^+$.\qedclaim 

% 	% a basis for D\epsilon
% 	We now consider $\D\epsilon$ as a graded right $\barr \D$-module. Then we have the decomposition $\D\epsilon = \epsilon \D\epsilon \oplus (1-\epsilon) \D\epsilon$. The set $\{\epsilon\}$ is clearly a basis of $\epsilon \D\epsilon$. To find a basis for $(1-\epsilon)\D\epsilon$, fix any $G$-homogeneous $0\neq d_1\in \D\odd$  with $\deg d_1 = t_1\in T^-$. Then we have $(1-\epsilon)\D\epsilon = (1-\epsilon)\D\even \epsilon + (1-\epsilon)\D\odd \epsilon = (1-\epsilon)\D\odd \epsilon = \D\odd \epsilon$ by Equation \eqref{eq:D1eps}. Since $d_1$ is invertible, $\{d_1\epsilon\}$ is a basis for $(1-\epsilon) \D\epsilon$. We conclude that $\{\epsilon, d_1\epsilon\}$ is a basis for $\D\epsilon$.

% 	% All parameters for D
% 	Using the graded analog of the Density Theorem (see e.g. \cite[Theorem 2.5]{livromicha}), we have $\D\iso \End_{\barr \D}(\D\epsilon)\iso \End(\FF\epsilon\oplus \FF d_1\epsilon)\tensor \barr\D$. Hence,
% 	%
% 	\[
% 	\begin{split}
% 	\End_\D(\mc U)&\iso\End (\tilde U) \tensor \D \iso \End (\tilde U) \tensor \End(\FF\epsilon \oplus \FF d_1\epsilon) \tensor \barr\D \\
% 	&\iso \End(\tilde U\tensor \epsilon \oplus \tilde U\tensor d_1\epsilon) \tensor \barr\D
% 	\end{split}
% 	\]
% 	%
% 	as $\barr G$-graded algebras. The result follows.
% \end{proof}

% In the next section, we will show how to recover $\Gamma$ from $\barr\Gamma$ and some extra data. The following definition and result will be used there.

% \begin{defi}
% 	For every abelian group $A$ we put $A^{[2]} = \{a^2 \mid a\in A\}$ and $A_{[2]} = \{a\in A \mid a^2 = e \}$.
% \end{defi}

% Note that $T^{[2]}\subseteq T^+$, but $T^{[2]}$ can be larger than $(T^+)^{[2]}$ since it also includes the squares of elements of $T^-$. Also, the subgroup $\barr S = \{\barr t \in \barr T \mid t \in T^{[2]}\}$ of $\barr T$ can be larger than $\barr T^{[2]}$, but we will show that, surprisingly, it does not depend on $T^-$.

% \begin{lemma}
% 	Let $\theta: T^+\rightarrow \barr T=\frac{T^+}{\langle t_0 \rangle}$ be the natural homomorphism. 
% 	Consider the subgroups $\barr S = \theta(T^{[2]})$ and $\barr R=\theta(T^+_{[2]})$ of $\barr T$. 
% 	Then $\barr S$ is the orthogonal complement of $\barr R$ with respect to the nondegenerate bicharacter $\barr\beta$.
% \end{lemma}

% \begin{proof}
% 	We claim that $\barr S^\perp = \barr R$. Indeed,
% 	%
% 	\[
% 		\begin{split}
% 			\barr S^\perp & = \{ \theta(t) \mid t\in T^+ \AND \barr\beta(\theta (t), \theta (s^2)) =1 \text{ for all }s\in T\}\\ & = \{\theta(t) \mid t\in T^+ \AND \beta (t, s^2) =1 \text{ for all }s\in T\}\\ & = \{ \theta(t) \mid t\in T^+ \AND \beta (t^2, s) =1 \text{ for all }s\in T\}\\ & = \{ \theta(t) \mid t\in T^+ \AND t^2=e \}\\ & = \barr R\,.
% 		\end{split}
% 	\]
% 	Since $\bar{\beta}$ is nondegenerate, it follows that $\barr S = \barr R^\perp$.
% \end{proof}

% Our second description of an odd grading consists of its finest even coarsening and the data necessary to recover the odd grading from this coarsening. 
% % All parameters will be obtained in terms of $G$ rather than its extension $G^\#=G\times \ZZ_2$.

% Let $t_0\in G$ be an arbitrarily fixed element of order 2 and set $\barr G = \frac{G}{\langle t_0 \rangle}$. Let $\barr T \subseteq \barr G$ be a finite subgroup and let $\barr \beta: \barr T \times \barr T \rightarrow \FF^\times$ be a nondegenerate alternating bicharacter. 
% We define $T^+\subseteq G$ to be the inverse image of $\barr T$ under the natural homomorphism $\theta: G\rightarrow \barr G$. Note that $\barr \beta$ gives rise to a bicharacter $\beta^+$ on $T^+$ whose radical is generated by the element $t_0$. 
% We wish to define $T^-\subseteq G\times \{\barr 1\}$ so that $T=T^+\cup T^-$ is a subgroup of $G^\#$ and $\beta^+$ extends to a nondegenerate alternating bicharacter on $T$.

% From Lemma \ref{lemma:square-subgroup}, we have a necessary condition for the existence of such $T^-$, namely, for $\barr R=\frac{T^+_{[2]}}{\langle t_0 \rangle}$, we need $\barr R' \subseteq \barr G^{[2]}$ (indeed, $\barr S$ is a subgroup of $\overline {G^{[2]}} = \barr G^{[2]}$). 
% We will now prove that this condition is also sufficient.

% \begin{prop}\label{prop:square-subgroup-converse}
% 	If $\left( \frac{T^+_{[2]} }{\langle t_0 \rangle}\right)^\perp \subseteq \barr G^{[2]}$, then there exists an element $t_1\in G\times \{\barr 1\} \subseteq G^\#$ such that $T= T^+ \cup t_1\, T^+$ is a subgroup of $G^\#$ and  $\beta^+$ extends to a nondegenerate alternating bicharacter $\beta:T\times T\rightarrow \FF^\times$.
% \end{prop}

% \begin{proof}
% 	Let $\chi\in \widehat {T^+}$ be such that $\chi(t_0) = -1$. Since $\chi^2(t_0)=1$, we can consider $\chi^2$ as a character of the group $\barr T = \frac{T^+}{\langle t_0 \rangle}$, hence there is $a\in T^+$ such that $\chi^2(\barr t) = \barr\beta(\barr a, \barr t)$ for all $\barr t\in \barr T$. Note that $\chi (a) = \pm 1$ and hence, changing $a$ to $a t_0$ if necessary, we may assume $\chi (a) = 1$.
	
% 	\bigskip 

% 	\textit{(i) Existence of $t_1$}:
	
% 	\medskip 

% 	As before, let $\barr R = \frac{T^+_{[2]}}{\langle t_0 \rangle}$. Then $\barr a \in \barr R^\perp$. Indeed, if $b\in T^+_{[2]}$, then $\barr\beta(\barr a,\barr b) = \chi^2 (\barr b) = \chi (b^2) = \chi (e) =1$. By our assumption, we conclude that $\barr a\in \barr G^{[2]}$. We are going to prove that, actually, $a\in G^{[2]}$. Pick $u\in G$ such that $\barr u^2 = \barr a$. Then, either $a=u^2$ or $a=u^2t_0$. If $t_0 = c^2$ for some $c\in G$, then replacing $u$ by $uc$ if necessary, we can make $u^2 = a$. Otherwise, $t_0$ has no square root in $T^+$, which implies that $\barr R=\barr T_{[2]}$. Hence $\barr R^\perp = (\barr T_{[2]})^\perp = \barr T^{[2]} = \theta ((T^+)^{[2]})$. Thus, in this case, we can assume $u\in T^+$. Then $\chi(u^2) = \chi^2(u) = \barr \beta (\barr a, \barr u) = \barr \beta (\barr u^2, \barr u) =1$, hence $u^2 = a$. Finally, we set $t_1=(u,\barr 1) \in G^\#$.

% 	\bigskip 

% 	\textit{(ii) Existence of $\beta$}:
	
% 	\medskip 

% 	We wish to extend $\beta^+$ to $T=T^+ \cup t_1\, T^+$ by setting $\beta(t_1, t) = \chi (t)$ for all $t\in T^+$. It is clear that there is at most one alternating bicharacter on $T$ with this property that extends $\beta^+$. To show that it exists and is nondegenerate, we will first introduce an auxiliary group $\widetilde T$ and a bicharacter $\tilde\beta$.

% 	Let $\widetilde T$ be the direct product of $T^+$ and the infinite cyclic group generated by a new symbol $\tau$. We define $\tilde\beta:\widetilde T\times \widetilde T \rightarrow \FF^\times$ by $ \tilde\beta(s\tau^i,t\tau^j) = \beta^+(s,t)\, \chi (s)^{-j}\, \chi (t)^i$, where $s,t\in T^+$. It is clear that $\tilde\beta$ is an alternating bicharacter.

% 	\begin{claim*}
% 	$\langle a\tau^{-2} \rangle = \rad \tilde \beta\,$.
% 	\end{claim*}

% 	Let $t\in T^+$ and $\ell\in \ZZ$. Then
% 	\[
% 	\tilde \beta (a\tau^{-2},t\tau^\ell) =
% 	\beta^+(a, t)\,\, \chi(t)^{-2} \, \chi(a)^{-\ell} = 
% 	\barr\beta(\barr a, \barr t)\,\, \chi(t)^{-2} = \chi(t)^2 \, \chi(t)^{-2} = 1,
%     \]
%     hence, $\langle a\tau^{-2} \rangle \subseteq \rad \tilde \beta$.

%     Conversely, if $s\tau^k \in \rad \tilde\beta$, then, $1 = \tilde \beta (s\tau^k, t_0) = \beta^+(s,t_0)\, \chi(t_0)^k = (-1)^k$, hence $k$ is even. From the previous paragraph, we know that $a\tau^{-2} \in \rad \tilde\beta$, hence $a^\frac{k}{2} \tau^{-k} \in \rad \tilde\beta$ and $s a^\frac{k}{2} = (s \tau^k) (a^\frac{k}{2} \tau^{-k}) \in \rad \tilde\beta$. Since $s a^\frac{k}{2} \in T^+$, we get $s a^\frac{k}{2} \in \rad \beta^+ = \{ e, t_0 \}$. But, if $sa^\frac{k}{2} = t_0$, we have $1 = \tilde\beta (sa^\frac{k}{2}, \tau) = \tilde\beta (t_0, \tau) = \chi(t_0)\inv = -1$, a contradiction. It follows that $sa^\frac{k}{2} = e$ and, hence, $s\tau^k = a^{-\frac{k}{2}}\tau^k = (a\tau^{-2})^{\frac{k}{2}}$, concluding the proof of the claim.
%     \qedclaim 

%     We have a homomorphism $\vphi:\widetilde T\rightarrow T$ that is the identity on $T^+$ and sends $\tau$ to $t_1$. Clearly, $\ker \vphi = \langle a\tau^{-2} \rangle$. By the above claim, $\tilde\beta$ induces a nondegenerate alternating bicharacter on $\frac{\widetilde T}{\langle a\tau^{-2} \rangle}$, which can be transferred via $\vphi$ to a nondegenerate alternating bicharacter on $T$ that extends $\beta^+$.
% %
% \end{proof}

% Now fix $\chi\in \widehat {T^+}$ with $\chi(t_0)=-1$ and let $a$ be the unique element of $T^+$ such that $\chi(a)=1$ and $\chi^2(\barr t) = \barr\beta (\barr a, \barr t)$ for all $t\in T^+$. Suppose that the condition of Proposition \ref{prop:square-subgroup-converse} is satisfied. 
% Then part (i) of the proof shows that there exists $u\in G$ such that $u^2=a$. Moreover, part (ii) shows that there exists an extension of $\beta^+$ to 
% a nondegenerate alternating bicharacter $\beta$ on $T=T^+\cup t_1T^+$, where $t_1=(u,\bar 1)$, such that $\beta(t_1,t)=\chi(t)$ for all $t\in T^+$.
% Clearly, such an extension is unique. We will denote it by $\beta_u$ and its domain by $T_u$.

% \begin{prop}\label{prop:roots-of-a}
% For every $T\subseteq G^\#$ such that $T\subsetneq G$ and $T\cap G=T^+$ and for every extension of $\beta^+$ to a nondegenerate alternating bicharacter $\beta$
% on $T$, there exists $u\in G$ such that $u^2=a$, $T=T_u$ and $\beta=\beta_u$.
% Moreover, $\beta_u=\beta_{\tilde{u}}$ if, and only if, 
% $u^{-1} \tilde{u} \in \langle t_0 \rangle$.
% \end{prop}

% \begin{proof}

% We have $T=T^+ \cup T^-$ where $T^-\subseteq G\times \{\barr 1\}$ is a coset of $T^+$.
% We can extend $\chi$ to a character of $T$, which we still denote by $\chi$, and, since $\beta$ is nondegenerate, 
% there is $t_1\in T$ such that $\beta(t_1, t) = \chi(t)$ for all $t\in T$. We have $t_1\in T^-$ since $\beta(t_1,t_0)=\chi(t_0)=-1$, so $t_1=(u,\bar 1)$, 
% for some $u\in G$, and hence $T=T_u$. We claim that $t_1^2 = a = (a, \bar 0)$ and, hence, $u^2 = a$. Indeed, $\chi(t_1^2) = \beta(t_1,t_1^2)=1$ and, for every $t\in T^+$,
% \[
%  	\chi^2(\barr t) = \chi(t)^2 = \beta (t_1, t)^2 = \beta (t_1^2, t) = \barr\beta (\,\overline {(t_1^2)},\, \barr t)\,,
% \]
% so $t_1^2$ satisfies the definition of the element $a$. This completes the proof of the first assertion.

% Now suppose $\beta_u=\beta_{\tilde{u}}$, so in particular $t_1\,T^+=\tilde{t}_1\,T^+$ where $t_1 = (u, \barr 1)$ and $\tilde{t}_1 = (\tilde u, \barr 1)$.
% Then there is $r\in T^+$ such that $\tilde{t}_1 = t_1\,r$. Also, for every $t\in T^+$,
% \[
% \chi(t) = \beta_{\tilde{u}}(\tilde{t}_1,t) = \beta_u (t_1\,r, t)
% 		= \beta_u(t_1, t)\,\beta_u(r,t) = \chi(t) \beta^+(r, t)
% \]
% and, hence, $\beta^+(r, t)=1$ for all $t\in T^+$. This means that $r = u\inv \tilde{u} \in \langle t_0 \rangle$.

% Conversely, if $\tilde u = u r$ for some $r\in \langle t_0 \rangle$, then $t_1\, T^+ = \tilde t_1\, T^+$. Also, for all $t\in T^+$,
% \[
% \beta_u(t_1, t) = \chi(t) = \beta_{\tilde{u}}(\tilde{t}_1, t) = \beta_{\tilde{u}}(t_1r, t) = 
% \beta_{\tilde{u}}(t_1, t)\, \beta^+(r, t) = \beta_{\tilde{u}}(t_1, t).
% \]
% It follows that $\beta_u=\beta_{\tilde{u}}$.
% \end{proof}

% Note that, keeping the character $\chi \in \widehat {T^+}$ with $\chi(t_0) = -1$ fixed, we have a surjective map from the square roots of $a$ to all possible pairs $(T,\beta)$. If we had started with a different character above, we would have obtained a different surjective map. Hence, for parametrization purposes, $\chi$ (and, hence, $a$) will be fixed.

% We are now in a position to give a classification of odd graded-division superalgebras gradings in terms of $G$ only. 
% We already have the following parameters: an element $t_0\in G$ of order $2$ and a pair $(\barr T, \barr\beta)$. 
% For any such $t_0$ and $\barr T$, we fix a character $\chi\in \widehat {T^+}$ satisfying $\chi(t_0) = -1$. 
% The next parameter is an element $u\in G$ such that $u^2 = a$, where $a$ is the unique element of $T^+$ such that $\chi(a)=1$ and $\chi^2(\barr t) = \barr\beta (\barr a, \barr t)$ for all $t\in T^+$. 
% % Finally, let $\gamma = (g_1, \ldots, g_k)$ be a $k$-tuple of elements of $G$. With these data, we construct the grading $\Gamma (t_0, \barr T, \barr \beta, u, \gamma)$ as follows:

% \begin{defi}\label{def:odd-grd-on-Mmn-2}
% 	Let $\D$ be a standard realization of the $G^\#$-graded division algebra with parameters $(T_u,\beta_u)$. 
% 	As a superalgebra, $\D$ is isomorphic to $M(n,n)$. 
% 	We define $\Gamma (t_0, \barr T, \barr \beta, u)$ as the corresponding $G$-grading on $M(n,n)$.
% \end{defi}

% % Theorem \ref{thm:first-odd-iso} together with Proposition \ref{prop:roots-of-a} give the following result:

% \begin{thm}\label{thm:2nd-odd-iso}
% 	Every odd division $G$-grading on the superalgebra $M(n,n)$ is isomorphic to some $\Gamma (t_0, \barr T, \barr \beta, u)$ as in Definition \ref{def:odd-grd-on-Mmn-2}.
% 	Two odd gradings, $\Gamma (t_0, \barr T, \barr \beta, u)$ and $\Gamma (t_0', \barr T', \barr \beta', u')$, 
% 	are isomorphic if, and only if, $t_0=t_0'$, $\barr T = \barr T'$, $\barr\beta = \barr\beta'$ and $u^{-1} u' \in \langle t_0 \rangle$. \qed
% \end{thm}

% \subsection{Gradings on \texorpdfstring{$Q(n)$}{Q(n)} only in terms of \texorpdfstring{$G$}{G}}

% % In this subsection we will not only find a description of gradings on $Q(n)$ only in terms of $G$, but we

% Suppose $\D \iso Q(n)$ as a superalagebra. 
% Recall that, by \cref{prop:tilde-beta-nondeg}, we have that $\tilde\beta$ is nondegenerate but $\beta$ is degenerate, \ie, we are under the conditions of \cref{lemma:beta-deg-beta-tilde-nondeg}. 

% We will denote by $u \coloneqq
%     \begin{pmatrix}
%         0 & I\\
%         I & 0
%     \end{pmatrix}
%     \in Q(n)$. 
% Note that $Z(Q(n))\even = \FF 1$ and $Z(Q(n))\odd = \FF u$. 

% % Let $\Gamma$ be a division on $Q(n)$ and let $(T, \beta, p)$ be the triple associated to the graded-division superalgebra $\D \coloneqq (Q(n), \Gamma)$. 
% % Note that, by \cref{prop:tilde-beta-nondeg}, we are under the conditions of \cref{lemma:beta-deg-beta-tilde-nondeg}. 

% % Clearly, $(T^+, \beta^+)$ defines a division-grading on the algebra $Q(n)\even \iso M_n(\FF)$. 
% By \cref{lemma:beta-deg-beta-tilde-nondeg}\eqref{item:rad-beta=t_1}, we have that $\rad \beta = \langle t_1 \rangle$ for a order two element $t_1 \in T^-$. 
% Then $t_1 = (h, \bar 1)$ for some element $h\in G$ of order at most two. 
% Since $\rad \beta = \supp Z(Q(n))$, we must have that $\deg u = t_1$. 

% \begin{defi}
%     If $\D \iso Q(n)$ as a superalgebra, we define the $G$-parameters of $\D$ to be the triple $(T^+, \beta^+, h)$.
% \end{defi}

% Note that the triple $(T^+, \beta^+, h)$ has the same information as the pair $(\Gamma\even, h)$, where $\Gamma\even$ denote the grading on $\D\even$.

% Conversely, given a division-grading $\Gamma\even$ on the algebra $Q(n)\even$ and an order two element $h \in G$, let $(T^+, \beta^+)$ be the pair corresponding to the graded-division algebra $(Q(n)\even, \Gamma\even)$ and define $t_1 \coloneqq (h, \bar 1)$. 
% We can then define $T \coloneqq T^+ \times \langle t_1 \rangle$, and define $\beta\from T\times T \to \FF^\times$ by $\beta(s t_1^i, t t_1^j) \coloneqq \beta^+(s, t)$. 
% It is easy to see that $\beta$ is a alternating bicharacter on $T$ and $\rad \beta = \langle t_1 \rangle$, and that $(T, \beta, p)$ is the triple associated to the graded-division superalgebra $Q(1)$ where the grading on $Q(n)\even$ is extended to $Q(n)$ by declaring $\deg u \coloneqq t_1$.  

% % We conclude the following:

% % \begin{thm}\label{thm:grd-div-Q-only-G}
    
% % \end{thm}

% \begin{remark}
%     The description of gradings on $Q(n)$ we found here
%     % in \cref{thm:grd-div-Q-only-G} 
%     is essentially the same one as in \cite[Theorem 5.5]{paper-Qn}.
% \end{remark}

% \begin{itemize}
%     \item Add that every $G$-grading on $Q(n)$ is restriction of a $G\times \ZZ_2$-grading on $M(n,n)$.
% \end{itemize}

% % For every associative superalgebra $R = R\even \oplus R\odd$, it is easy to see that $R\odd$ is an $(R\even, R\even)$-bimodule. 
% % In the case of $R \iso Q(n)$, then it is clear that $R\even \iso M(n)$ as an algebra and $R\odd \iso M(n)$ as a $(M(n), M(n))$-bimodule. 


% % \begin{itemize}
% %     \item We have two parametrizations, one here and one in \cite{paper-Qn}.
% %     \item The one here is as follows:
% %     \begin{itemize}
% %         \item Every grading is odd, since it contains $Q(1)$ in the center, and hence in the center of $\D$.
% %         \item Hence we describe it by $(T, \beta, \kappa)$, where $T\subseteq G^\#$ but $T\not\subseteq G$. 
% %         \item Also, $\rad \beta$ is generated by an odd element of order $2$, say $t_0$.
% %         \item $t_0$ is the parity element according to $\tilde\beta$.
% %     \end{itemize}
% %     \item The old parametrization consisted of a grading on the even part and an element of order $2$ (in $G$) to shift it to the odd part.
% %     \item The old parametrization is only in terms of $G$.
% %     \item the connection must be, top down:
% %     \begin{itemize}
% %         \item The grading on the even part is $(T^+, \beta^+, \kappa)$.
% %         \item Note that $\beta^+$ is nondegenerate on $T^+$.
% %         \item If $t_0 = (h, \bar 1)$, then the shift on the odd part is by $h$, which clearly has order $2$.
% %     \end{itemize}
% %     \item the connection must be, bottom up:
% %     \begin{itemize}
% %         \item Define $t_0$ as $(h, \bar 1)$
% %         \item Define $T$ as $T^+ \cup t_0 T^+$
% %         \item Extend $\beta^+$ to $\beta$ by defining $t_0$ to be in the radical
% %     \end{itemize}
% %     \item we can see it not only on the level of parameters but also on the level of $\End_\D (\U)$:
% %     \item top down:
% %     \begin{itemize}
% %         \item Note that $\End_\D (\U) = \End_{\D\even} (\U\even) \oplus d_0 \End_{\D\even} (\U\even)$, where $d_0 \in Z(\D) \cap \D\odd$. 
% %         Why is that?
% %         \item Well, I guess it starts we getting an even basis for $\U$ and its corresponding $\FF$-form $\widetilde U$. 
% %         \item Then $\End_\D (\U) \equiv \End_\FF (\tilde U) \tensor \D$, hence the even part is $\End_\FF (\tilde U) \tensor \D\even$. 
% %         \item Since $\D\odd = d_0 \D\even$, we get that the odd part is $\End_\FF (\tilde U) \tensor \D\odd = d_0 \End_\FF (\tilde U) \tensor \D\even$
% %         \item The apparent problem here is that the choice of the shift on the odd part doesn't matter.
% %         \item Modulo $T^+$ they are clearly the same, but we don't need $T^+$ in the parametrization
% %         \item Oh, it should be the definition of the product in the down model! $(d_0 r) r' = r (d_0 r')$ and its version for only odd elements. So $d_0 \in Z(R)$.
% %         % \item Also, we could see that $\U\even = \tilde U \tensor \D\even$ and $\U\odd = \tilde U \tensor \D\odd$.
% %     \end{itemize}
% %     \item bottom up:
% %     \begin{itemize}
% %         \item The grading is determined be $\D\even$ and $\U\even$
% %         \item Let $R\even = \End_{\D\even} (\U\even)$ 
% %         \item define $t_0 = (h, \bar 1)$ and introduce a symbol $d_0$ of degree $t_0$, such that $d_0^2 = 1$ and it commutes with every element. This defines $R$.
% %         \item one should check the parameters for $R$. What is the new $\D$?
% %         \item By uniqueness of $\D$ up to iso, we may be able to use the top down part to prove it is what is asked. But we then have to show an isomorphism. Luckily, this is obvious by the definition of the product. 
% %         \item Wait, we have to define $\D$ as $\D\even \oplus d_0\D\odd$, and check it is a graded-division algebra (obvious).
% %     \end{itemize}
% % \end{itemize}
