\section{Gradings on simple Lie superalgebras of series \texorpdfstring{$A$ and $Q$}{A and Q}}\label{sec:grds-A-and-Q}

By the discussion in \cref{sec:Aut-Lie-chap}  (\cref{cor:transfer-R-vphi-to-L}), we have a classification of $G$-gradings on the Lie superalgebras of series $A$ and $Q$ in terms of the model $\Skew(R, \vphi)$, where $R = S\times S\sop$ and $\vphi$ is the exchange superinvolution. 
We want to describe these gradings in terms of the original model, $S^{(-)}$, where $S$ is either $M(m,n)$ or $Q(n)$ for some positive integers $m$ and $n$. 
We use an approach similar to \cite[Appendix]{paper-adrian}. 

We will also express the parameters of the grading in terms of the group $G$ rather than $G^\# =G \times \ZZ_2$. 
 
\subsection{Undoubling}\label{ssec:undoubling}
% Reducing from \texorpdfstring{$S\times S\sop$}{SxSsop} to \texorpdfstring{$S$}{S}

Let $L$ be a finite dimensional simple Lie superalgebra in the series $A$ or $Q$, and suppose $L$ is not of type $A(1,1)$. 
By definition, we have that $L = S^{(1)}/Z(S^{(1)})$. 
On the other hand, \cref{cor:transfer-R-vphi-to-L} allows us to classify the gradings on $L$ by considering its isomorphic copy $\tilde L \coloneqq \Skew(R, \vphi)^{(1)}/Z(\Skew(R, \vphi)^{(1)})$, where $R = S\times S\sop$ and $\vphi$ is the exchange superinvolution:
the gradings (and their isomorphism classes) on $\tilde L$ are in bijection with the gradings (and their isomorphism classes) on $(R, \vphi)$. 
The goal of this subsection is to translate the classification of gradings on $L$ from $(R, \vphi)$ to $S$. 
Recall that the isomorphism $\tilde L \to L$ is induced by the projection $R \to S$. 

\begin{defi}\label{defi:types-I-and-II}
    A grading on $L$ is said to be of \emph{Type I} if it is obtained by restriction and reduction modulo the center from a (unique) grading on $S$. 
    Otherwise, the grading on $L$ is said to be of \emph{Type II}. 
\end{defi}

As recalled above, all gradings on $\tilde L \iso L$ come from gradings on $(R, \vphi)$. 
Type I and Type II gradings can be distinguished in this model as follows. 
Recall that gradings on $(R, \vphi)$ were divided in two classes: those that make $R$ graded-simple, classified in \cref{thm:MxM-even,thm:MxM-odd}, and those that do not, classified in \cref{thm:MxM-type-I,thm:QxQ-type-I}. 
We claim that these classes correspond to the Type II and Type I gradings on $\tilde L$, respectively. 
Indeed, consider a Type I grading on $L$ and the corresponding grading on $S$. 
Then $(R, \vphi)$ is naturally graded so that $S\times \{ 0 \}$ and $\{0\} \times S\sop$ are graded superideals, and the projection $R \to S$ is a homomorphism of graded superalgebras. 
This grading on $(R, \vphi)$ induces a grading on $\tilde L$ and the isomorphism of Lie superalgebras $\tilde L \to L$ preserves degrees. 
This proves that the gradings on $\tilde L$ corresponding to Type I gradings on $L$ (under the isomorphism $\tilde L \to L$) come from the gradings on $R$ that do not make it graded-simple. 
By \cref{prop:only-SxSsop-is-simple}, every grading on $(R, \vphi)$ such that $R$ is not graded-simple is of this form, so the converse follows. 

By definition, Type I gradings are described in terms of $S$, so we will now focus on Type II gradings. 

Suppose $(R, \vphi)$ is endowed with a grading $\Gamma$ making $R$ graded-simple. 
By \cref{thm:iso-(R-vphi)-with-parameters}, there is a triple $(\D, \U, B)$ with parameters $(T, \beta, p, \eta, \kappa, g_0, \delta)$ as in Definition \ref{def:E(D,U,B)} such that $(R, \vphi) \iso E(\D, \U, B)$. 
Recall that, in this case, $\rad \tilde\beta = \langle f \rangle$ for an element $f\in T^+$ of order $2$ such that $\eta(f) = -1$ (\cref{prop:types-of-D-via-rad-beta}). 

Set $\barr G = G/\langle f \rangle$, let  $\pi\from G\to \barr G$ be the natural homomorphism and write $\bar g \coloneqq \pi(g)$, for all $g\in G$. 
By \cref{prop:lemma-for-undoubling-and-fine-gradings}, ${}^\pi R$ is the sum of two graded-simple superideals, $S\times \{ 0 \}$ and $\{0\} \times S\sop$, so the restriction of ${}^\pi \Gamma$ to $S \iso S\times \{ 0 \}$ induces a $\barr G$-grading on $L$ of Type I. 

To recover $\Gamma$ from its coarsening ${}^\pi \Gamma$, fix a character $\chi \in \widehat {G^\#}$ such that $\chi (f) = -1$ (which exists since $\FF$ is algebraically closed) and let $\psi\from R \to R$ be the automorphism given by the action of $\chi$, \ie, $\psi(r) \coloneqq \chi(g) r$ for every $r\in R_g$. 
Clearly, $\psi$ restricts to $R_{\bar g} = R_g \oplus R_{gf}$ and acts as the multiplication by $\chi(g)$ on $R_g$ and by $\chi(gf) = - \chi(g)$ on $R_{gf}$. 
Hence:
\[\label{eq:Rg-recovered-with-chi}
    R_g = \{ r \in R_{\bar g} \mid \psi(r) = \chi(g)r \}.
\]
Let $\zeta \coloneqq (1, -1) \in Z(R)\even$. 
We have that $\vphi(\zeta) = -\zeta$ and, by \cref{prop:R-and-D-have-the-same-center-vphi}, $\zeta$ is homogeneous of degree $f$ with respect to $\Gamma$. 
Let $\theta\from R \to R$ be the super-anti-automorphism defined by $\theta \coloneqq \vphi \psi = \psi \vphi$. 
By the definition of $\psi$, $\psi(\zeta) = -\zeta$ and, hence, $\theta(\zeta) = \zeta$. 
It follows that $\theta(1,0) = \theta \left(\frac{1+\zeta}{2}\right) = (1,0)$, so  $\theta(S\times \{0\}) = \theta ((1,0)R) = S\times \{0\}$. 
Hence, $\theta$ restricts to a super-anti-automorphism on $S$. 
Then, it is straightforward to see that
\[\label{eq:psi-in-terms-of-theta}
    \psi(s_1, \overline{s_2}) = (\theta(s_2), \overline{\theta(s_1)}),
\]
for all $s_1, s_2\in S$. 
Since $\Skew(R, \vphi) = \{ (s, -\bar s) \mid s \in S \}$, combining \cref{eq:Rg-recovered-with-chi,eq:psi-in-terms-of-theta}, we get that the Type II grading on $L$ corresponding to $\Gamma$ can be recovered from the restriction of ${}^\pi \Gamma$ to $S$ as follows: we define a $G$-grading $S^{(-)} = \bigoplus_{g\in G} S^{(-)}_g$ by setting
\[\label{eq:refiniment-on-L}
    \forall g\in G, \quad S^{(-)}_g \coloneqq \{ s \in S_{\bar g} \mid \theta(s) = -\chi(g) s\},
\]
and induce a $G$-grading on $L = S^{(1)}/Z(S^{(1)})$. 
Hence, a Type II grading on $L$ can be described in terms of a $\barr G$-grading on $S$ and a super-anti-automorphism on $S$. 

\begin{defi}\label{defi:undoubled-model}
    We will call $L$ endowed with the $G$-grading constructed above the \emph{undoubled model of $\Skew(R, \vphi)/Z(\Skew(R, \vphi))$}, or simply the \emph{undoubled model} if $(R, \vphi)$ is fixed in the context. 
\end{defi}

Our next goal is to describe the undoubled model without reference to $(R,\vphi)$. 
First, we need the parameters of 
$S$ endowed with the restriction of ${}^\pi\Gamma$. 
By \cref{prop:lemma-for-undoubling-and-fine-gradings}, $S \iso E(\barr T, \barr \beta, \barr p, \kappa)$, where $\barr T \coloneqq T/\langle f \rangle$, $\barr \beta$ and $\barr p$ are the maps induced by $\beta$ and $p$ on $\barr T$, and $\kappa\from G^\#/T \to \ZZ_{\geq 0}$ is seen as a map $\kappa\from \barr G^\#/ \barr T \to \ZZ_{\geq 0}$ via the canonical isomorphism $G^\#/T \iso \barr G^\#/ \barr T$. 

It remains to describe the super-anti-automorphism $\theta\from S \to S$. 
Let us write $\theta\from R\to R$ in matrix terms. 
Let $\B = \{ u_1, \ldots, u_k \}$ be a $G$-graded basis of $\U$, following \cref{conv:pick-even-basis}, and use it to identify $\End_\D (\U)$ with $M_k(\D)$. 
We will also assume that $B$ is even if $\D$ is odd (\cref{conv:pick-even-form}). 
Set $g_i \coloneqq \deg u_i$. 
By \cref{prop:matrix-vphi}, we have $\vphi(X) = \Phi\inv \vphi_0(X)\stransp \Phi$, for all $X \in M_k(\D)$, where $\Phi_{ij} = B(u_i, u_j)$. 
Also, given $i, j \in \{ 1, \ldots, k \}$ and $0 \neq d \in \D_t$, $t\in T$, we have $\psi( E_{ij}(d) ) = \chi(g_i t  g_j\inv) E_{ij}(d) = \chi(g_i) \chi(t) \chi(g_j)\inv E_{ij}(d)$. 
Let $\Lambda \in M_k(\D)$ be the diagonal matrix with $\Lambda_{ii} = \chi(g_i)$ for all $i \in \{ 1, \ldots, k \}$, and let $\psi_0$ denote the action of $\chi$ on $\D$. 
Then $\psi(X) = \Lambda \psi_0(X) \Lambda\inv$, for all $X \in M_k(\D)$. 
Define $\theta_0 \coloneqq \psi_0 \vphi_0 = \vphi_0 \psi_0$. 
Then:
\[\label{eq:theta-with-matrix-1}
    \forall X \in M_k(\D), \quad \theta(X) = (\Lambda\Phi)\inv\, \theta_0(X)\stransp \, (\Lambda\Phi).
\] 

Clearly, $\theta_0$ is a super-anti-automorphism on $\D$ associated to the map $\mu \coloneqq \eta \chi$. 
Note that $\mu(f) = 1$ and, hence, we can see $\mu$ as a map $\barr \mu\from \barr T \to \FF^\times$. 
Indeed, since $f\in \ker \tilde\beta$, we have
\begin{equation}
    \forall t\in T, \quad \mu(tf) = \tilde\beta (t,f) \mu(t)\mu(f) = \mu(t)\mu(f) = \mu(t),
\end{equation}
so the map $\barr\mu \from \barr T \to \FF^\times$ given by
\[\label{eq:defi-mu-undoubled}
    \forall \bar t \in \barr T,\quad \barr\mu(\bar t) \coloneqq \eta(t) \chi(t)
\] is well-defined. 
Recall, from \cref{cor:T+-is-elem-2-grp}, that $T^+$ is an elementary $2$-group and for every $t\in T^-$, $t^2 = f$. 
Hence, $\chi$ takes values $\pm 1$ on $T^+$ and $\pm \mathbf{i}$ on $T^-$. 
By \cref{lemma:quadratic-form-involutions}, $\bar\mu$ is a quadratic map on $\barr T$. 

Let $\epsilon \coloneqq (1,0) \in Z(R)\even$ be the identity element of $S$, and consider it also as an element of $Z(\D)\even$ using \cref{prop:R-and-D-have-the-same-center-vphi}. 
Following the proof of \cref{prop:lemma-for-undoubling-and-fine-gradings}, with $\D$ playing the role of $\mc E$ and $\U$ playing the role of $\V$, we have that $S \iso \End_{\barr \D}(\bar \U)$, where $\barr \D \coloneqq \D\epsilon$ and $\barr \U \coloneqq \U\epsilon$, and that $\B\epsilon \coloneqq \{ u_1\epsilon, \ldots, u_k\epsilon\}$ is a $\barr G^\#$-graded $\barr \D$-basis of $\barr \U$. 
Using this basis, we can identify $\End_{\barr \D}(\bar \U)$ with $M_k(\barr \D)$. 

By definition, if $Y \in M_k(\D)$ is the matrix representing an operator
$r\in R$ with respect to the basis $\B$, then $ru_j = \sum_{i = 1}^k u_i Y_{ij}$, for all $1 \leq j \leq k$. 
It follows that 
\begin{align}
    (\epsilon r)(u_j\epsilon) = r(u_j) \epsilon = \sum_i^k u_i Y_{ij} \epsilon = \sum_i^k (u_i\epsilon) Y_{ij} \epsilon, 
\end{align}
\ie, $X \in \M_k (\barr \D)$, given by $X_{ij} \coloneqq Y_{ij} \epsilon$, is the matrix representing $\epsilon r \in S$ with respect the basis $\B\epsilon$. 
From \cref{eq:theta-with-matrix-1}, we now get that
\[\label{eq:theta-with-matrix-2}
    \forall X\in M_k(\barr \D), \quad \theta(X) = \Theta\inv\, \theta_0(X)\stransp\, \Theta,
\]
where 
\[\label{eq:defi-Theta-undoubled-model}
    \Theta \coloneqq \Lambda \Phi \epsilon \in M_k(\barr \D)
\]
or, equivalently, $\Theta_{ij} = \chi(g_i) B(u_i, u_j) \epsilon$, and $\theta_0$ is the super-anti-automorphism on $\barr \D$ associated to $\bar \mu$. 

Finally, we note that the refinement obtained in \cref{eq:refiniment-on-L}, from $\theta$ given by \cref{eq:theta-with-matrix-2}, is determined by the values of $\chi$ on the subgroup $T \subseteq G^\#$. 
% To be more precise, let $\tilde \chi \in \widehat {G^\#}$ be a character such that $\chi\restriction_{T} = \tilde\chi\restriction_{T}$, let $\tilde \psi \from R\to R$ be the action by $\tilde \chi$, and set $\tilde \theta \coloneqq \tilde\psi \vphi = \vphi \tilde\psi$. 
% We claim that, for every $s \in S_{\bar g}$, $\tilde\theta(s) = - \tilde\chi(g) s$ \IFF $\theta(s) = - \chi(g) s$. 

% Indeed, set $\sigma \coloneqq \tilde\chi\chi\inv$, so $\tilde\chi = \sigma \chi$. 
% Then $\sigma$ is a character on $G^\#$ with $\sigma(T) = 1$ and, in particular, can be seen as a character on $G^\#/\langle f \rangle$. 
% Every element in $S_{\bar g}$ is a sum of elements of the form $E_{ij}(d)$, where $1 \leq i, j \leq k$ and $d\in \barr \D_{\bar t}$, such that $\bar g_i \bar t \bar g_j\inv = \bar g$. 
% By \cref{eq:theta-with-matrix-2}, 
% \begin{align}
%     \tilde\theta(E_{ij}(d)) 
%     &= \sigma(g_i) \sigma(g_j\inv) \theta (E_{ij}(d)) 
%     = \sigma(\bar g_i) \sigma(\bar g_j\inv) \theta (E_{ij}(d)) \\ 
%     &= \sigma(\bar t\inv \bar g) \theta(E_{ij}(d)) 
%     = \sigma(\bar g) \theta(E_{ij}(d)).
% \end{align} 
% We conclude that $\tilde\theta(s) = \sigma(\bar g) \theta(s)$ for all $s\in S_{\bar g}$ and the claim follows. 

\begin{lemma}
    Let $\tilde \chi \in \widehat {G^\#}$ be a character such that $\chi\restriction_{T} = \tilde\chi\restriction_{T}$, let $\tilde \psi \from R\to R$ be the action by $\tilde \chi$, and set $\tilde \theta \coloneqq \tilde\psi \vphi = \vphi \tilde\psi$. 
    Then, for every $s \in S_{\bar g}$, $\tilde\theta(s) = - \tilde\chi(g) s$ \IFF $\theta(s) = - \chi(g) s$. 
\end{lemma}

\begin{proof}
    Set $\sigma \coloneqq \tilde\chi\chi\inv$, so $\tilde\chi = \sigma \chi$. 
    Then $\sigma$ is a character on $G^\#$ with $\sigma(T) = 1$ and, in particular, can be seen as a character on $G^\#/\langle f \rangle$. 
    Every element in $S_{\bar g}$ is a sum of elements of the form $E_{ij}(d)$, where $1 \leq i, j \leq k$ and $d\in \barr \D_{\bar t}$, such that $\bar g_i \bar t \bar g_j\inv = \bar g$. 
    By \cref{eq:theta-with-matrix-2}, 
    \begin{align}
        \tilde\theta(E_{ij}(d)) 
        &= \sigma(g_i) \sigma(g_j\inv) \theta (E_{ij}(d)) 
        = \sigma(\bar g_i) \sigma(\bar g_j\inv) \theta (E_{ij}(d)) \\ 
        &= \sigma(\bar t\inv \bar g) \theta(E_{ij}(d)) 
        = \sigma(\bar g) \theta(E_{ij}(d)).
    \end{align} 
    We conclude that $\tilde\theta(s) = \sigma(\bar g) \theta(s)$, for all $s\in S_{\bar g}$. 
\end{proof} 

% Vague summary of the subsec
To summarize, once $\chi \in \widehat{T}$ is fixed, the undoubled model is determined by the $\barr G$-graded superalgebra $M_k(\barr \D)$ with parameters given by \cref{prop:lemma-for-undoubling-and-fine-gradings}, the map $\barr\mu \from \barr T \to \FF^{\times}$ defined by \cref{eq:defi-mu-undoubled}, which is
associated to a super-anti-automorphism $\theta_0$ on $\barr \D$, and the matrix $\Theta \in M_k(\barr \D)$ defined by \cref{eq:defi-Theta-undoubled-model}, which is used to define the super-anti-automorphism $\theta$ on $M_k(\barr \D)$ by \cref{eq:theta-with-matrix-2}. 

% Prepare for following subsecs
In the following subsections, we specialize these considerations to the cases of $A(m,n)$ and $Q(n)$. 
As in \cref{sec:grds-osp-and-p}, we fix a standard realization of a matrix algebra with a division grading associated to each finite abelian group with a nondegenerate alternating bicharacter (see \cref{def:standard-realization}). 

% ---------------------------

\subsection{Gradings on \texorpdfstring{$A(m,n)$}{A(m,n)} for \texorpdfstring{$m \neq n$}{m different from n}}\label{ssec:grds-on-A-m-n}

We now discuss the case where $S = M(m+1, n+1)$ and $m \neq n$,  
so $L = \Sl(m+1 | n+1) = S^{(1)}$ is a simple Lie superalgebra of type $A(m,n)$. 
(As seen before, in this case $Z(S^{(1)}) = \{ 0 \}$.)

We first parametrize the Type I gradings. 
Since $m\neq n$, by \cref{lemma:odd-M-m=n}, every grading on $S$ is even in this case. 
In particular, $\kappa\from G^\#/T \to \ZZ_{\geq 0}$ corresponds to a pair $(\kappa_\bz, \kappa_\bo)$ (see \cref{ssec:supermodules-over-D}). 

In the next definition, we allow the possibility of $m=n$ for future reference (in \cref{ssec:grds-on-Ann}).

\begin{defi}\label{def:A-Type-I}
    Let $m, n\in \ZZ_{\geq 0}$, not both zero, $T \subseteq G$ be a finite subgroup, $\beta\from T\times T \to \FF^\times$ be a nondegenerate alternating bicharacter, and $\kappa_\bz, \kappa_\bo \from G/T \to \ZZ_{\geq 0}$ be maps with finite support such that $m+1 = |\kappa_\bz| \sqrt{|T|/2}$ and $n+1 = |\kappa_\bo| \sqrt{|T|/2} $. 
    We will denote by $\Gamma^{\mathrm{(I)}}_A(T, \beta, \kappa_\bz, \kappa_\bo)$ the restriction to $S^{(1)}$ of the grading $\Gamma_M(T, \beta, \kappa_\bz, \kappa_\bo)$ on $S$ (see \cref{def:Gamma-T-beta-kappa-even}). 
\end{defi}

Parametrizing Type II gradings is more involved. 
As we have seen in \cref{ssec:undoubling}, those correspond to gradings on $(R, \vphi)$ making $R$ graded-simple, where $R \coloneqq S \times S\sop$ and $\vphi$ is the exchange superinvolution. 
Again, since $m\neq n$, these gradings on $R$ are even (see \cref{cor:associative-type-II-odd-m=n}). 
Hence, by \cref{thm:MxM-even}, a Type II grading on $L$ corresponds to a grading on $(R, \vphi)$ making it isomorphic to $M^{\mathrm{ex}} (T, \beta, \kappa_\bz, \kappa_\bo, g_0)$ (\cref{def:model-grd-MxM-even}), where $T \subseteq G$ is a finite $2$-elementary subgroup, $\beta\from T\times T \to \FF^\times$ is an alternating bicharacter with $\rad \beta = \langle f \rangle$ for some $e\neq f \in T$, $g_0$ is an element in $G^\#$, and $\kappa_\bz, \kappa_\bo \from G/T \to \ZZ_{\geq 0}$ are $g_0$-admissible maps (see \cref{inertia-even-and-odd-case}) such that $|\kappa_\bz| \sqrt{|T|/2} = m+1$ and $|\kappa_\bo| \sqrt{|T|/2} = n+1$. 
Since $m\neq n$, the $g_0$-admissibility implies that $g_0 \in G$. 
Recall that the graded algebra $M^{\mathrm{ex}} (T, \beta, \kappa_\bz, \kappa_\bo, g_0)$ corresponds to $(\eta, \kappa, g_0, \delta) \in \mathbf{I}(T, \beta, p)$, where $\eta\from T \to \pmone$ is fixed, $p\from T \to \ZZ_2$ is the trivial homomorphism, and $\delta = 1$. 
The map $\eta$ was determined by fixing a standard realization $\D$ for $(T, \beta, e)$ (see \cref{def:std-realization-MxM-QxQ}), namely, $\eta$ is the map associated to the superinvolution $\vphi_{\mc C} \tensor \vphi_{\mc M}$ on $\D$. 

We will use the parameters $(T, \beta, \kappa_\bz, \kappa_\bo, g_0)$ to construct a representative for the Type II grading in terms of the undoubled model, \ie, we will construct a $\barr G$-grading on $S$, where $\barr G \coloneqq G/\langle f \rangle$, and a super-anti-automorphism $\theta\from S\to S$, as in \cref{ssec:undoubling}. 
As in that subsection, let $\pi\from G \to \barr G$ denote the natural homomorphism, set $\barr T \coloneqq T/\langle f \rangle$, let $\barr \beta$ be the (nondegenerate) bicharacter on $\barr T$ induced by $\beta$, and consider $\kappa_\bz$ and $\kappa_\bo$ as maps defined on $\barr G/\barr T \iso G/T$. 

Let $\barr \D$ be the chosen standard realization associated to $(\barr T, \bar\beta)$, and let $\bar \mu\from \barr T \to \FF^\times$ be the map associated to the transposition map. 

Fix a complement $K$ for $\langle f \rangle$ in $T$, \ie, a subgroup $K \subseteq T$ such that $T = K \times \langle f \rangle$ (it can be done, since $T$ is a elementary $2$-group). 
Let $\chi\from T \to \FF^\times$ be the character defined by $\chi(K) = 1$ and $\chi(f) = -1$, and set $\mu \coloneqq \barr \mu \circ \pi\restriction_{T}$. 
Then set 
\[\label{eq:fix-eta-undouble}
    \forall t\in T, \quad \eta(t) \coloneqq \mu(t) \chi\inv(t).
\]
Clearly, $\mathrm{d} \eta = \beta$. 
Note that this definition agrees with \cref{def:std-realization-MxM-QxQ}(a) (see the proof of \cref{prop:m-not-n-Type-II-correspondence}). 

Extend $\chi$ to $G$. 
It remains to define a graded right $\barr \D$-supermodule $\barr \U$ and the matrix $\Theta$. 
The following is analogous to the construction in \cref{ssec:grds-osp}. 

Let $\xi\from G/ T \to G$ be a set-theoretic section of the natural homomorphism, and let $\leq$ be a total order on the set $G/T \iso \barr G/ \barr T$ with no elements between $x$ and $\bar g_0\inv x\inv$. 
Changing $\xi$ if necessary, we may assume that $\xi(g_0\inv x\inv) = g_0\inv \xi(x)\inv$ if $x < g_0\inv x\inv$. 
For each $i \in \ZZ_2$, set $k_i \coloneqq |\kappa_i|$, let $\gamma_i$ be the $k_i$-tuple of elements in $G$ realizing $\kappa_i$ according to $\xi$ and $\leq$ (see \cref{defi:tuple-governed}), and let $\bar \gamma_i$ be the tuple of elements in $\barr G$ consisting of the images under $\pi\from G\to \barr G$ of the entries of $\gamma_i$ (\ie, $\barr \gamma_i$ is the $k_i$-tuple realizing $\kappa$ according to $\pi \circ \xi$ and $\leq$). 
Consider on $M_{k_\bz | k_\bo}(\FF)$ the elementary grading determined by $(\bar  \gamma_\bz, \bar \gamma_\bo)$ (see \cref{defi:elementary-grd-super}). 
We identify the $\barr G$-graded superalgebra $M_{k_\bz | k_\bo}(\barr \D) = M_{k_\bz | k_\bo}(\FF) \tensor \barr\D$ with $S = M(m+1, n+1)$ via Kronecker product. 

\begin{defi}\label{defi:blocks-of-Theta}
    Let $i\in \ZZ_2$ and $x \in G/T$. 
    If $g_0x^2 = T$, we put $t \coloneqq g_0 \xi(x)^2 \in T$ and let $\bar t \in \barr T$ be its image under the natural homomorphism $T \to \barr T$. 
    We define $\Theta(i, x)$ to be the following $\kappa_i(x) \times \kappa_i(x)$-matrix with entries in $\barr \D$:
    %
    \begin{enumerate}[(i)]
        \item $I_{\kappa_i(x)} \tensor X_{\bar t}$ if $(-1)^i \eta(t) = +1$;
        %
		\item  $J_{\kappa_i(x)} \tensor X_{\bar t}$, where $J_{\kappa_i(x)} \coloneqq \begin{pmatrix}
				      0                & I_{\kappa_i(x)/2} \\
				      -I_{\kappa_i(x)/2} & 0
			      \end{pmatrix}$, if  $(-1)^i \eta(t) = -1$ (recall that, in this case, $\kappa_i(x)$ is even by \cref{inertia-even-and-odd-case}). 
	\end{enumerate}
    %
    If $g_0 x^2 \neq T$, we define $\Theta(i, x)$ to be the following $2\kappa_i(x) \times 2\kappa_i(x)$-matrix:
    %
    \begin{enumerate}[(i)]
        %
        \setcounter{enumi}{2}
        %
		\item $\begin{pmatrix}
			0  &  I_{\kappa_i(x)} \\
			(-1)^{i} \chi(g_0 \xi(x)^2)\inv I_{\kappa_i(x)} & 0
		\end{pmatrix} \tensor 1_{\barr \D}$. 
    \end{enumerate}
\end{defi}

Let $x_1 < \cdots < x_{\ell_\bz}$ be the elements of the set $\{ x \in \supp \kappa_\bz \mid x \leq g_0\inv x\inv \}$ and, similarly, let $y_1 < \ldots < y_{\ell_\bo}$ be the elements of $\{ y \in \supp \kappa_\bo \mid y \leq g_0\inv y\inv \}$. 
Then, we define 
\[\label{eq:puting-the-blocks-of-Phi-together-version-A}
    %
    \sbox0{$\begin{matrix}
        \Theta(\bar 0, x_1)&& \\
        & \ddots &\\
        && \Theta(\bar 0, x_{\ell_\bz})
    \end{matrix}$}
    %
    \sbox1{$\begin{matrix}
        \Theta(\bar 1, y_1)&& \\
        & \ddots &\\
        && \Theta(\bar 1, y_{\ell_\bo})
    \end{matrix}$}
    %
    \Theta \coloneqq
    \left(\begin{array}{c|c}
            \usebox{0} & 0\\
            \hline
            0 & \usebox{1}
        \end{array}\right).
\]
%
We define the super-anti-automorphism $\theta\from S\to S$ by \cref{eq:theta-with-matrix-2}, where $\theta_0\from \barr \D \to \barr \D$ denotes the transposition on $\barr \D$. 
Note that $\theta_0(X)\stransp \in M_{k_\bz \mid k_\bo} (\barr \D)$ becomes $X\stransp \in M(m+1,n+1)$. 
Hence, \cref{eq:theta-with-matrix-2} reduces to 
\[\label{eq:theta-with-matrix-3}
    \forall X\in M(m+1,n+1), \quad \theta(X) = \Theta\inv\, X\stransp\, \Theta.
\]

Finally, we define a $G$-grading on $L = S^{(1)}$ by 
\[\label{eq:def-final-grd-A(m-n)}
    \forall g\in G, \quad L_g \coloneqq \{ s \in S^{(1)}_{\barr g} \mid \theta(s) = - \chi(g) s\}.
\]

In the next definition, we summarize what has been done for future reference. 
We also allow the case $m=n$.

\begin{defi}\label{defi:type-II-A-m-not-n}
    Let $m,n \in \ZZ_{\geq 0}$, not both zero. 
    Let $T \subseteq G$ be a finite $2$-elementary subgroup and let $\beta\from {T\times T} \to \FF^\times$ be an alternating bicharacter with $\rad \beta = \langle f \rangle$, for some $e \neq f\in T$. 
    Set $\barr G \coloneqq G/\langle f \rangle$ and $\barr T \coloneqq T/\langle f \rangle$, and let $\bar \beta$ be the nondegenerate alternating bicharacter on $\barr T$ induced by $\beta$. 
    Consider the chosen standard realization $\barr \D$ of a matrix algebra with division grading associated to $(\barr T, \barr\beta)$ and the chosen subgroup $K \subseteq T$ such that $T = K \times \langle f \rangle$, and define $\eta\from T \to \pmone$ by \cref{eq:fix-eta-undouble}. 
    Then, let $g_0 \in G$ be any element and let $\kappa_\bz, \kappa_\bo \from G/T \to \ZZ_{\geq 0}$ be $g_0$-admissible maps (\cref{inertia-even-and-odd-case}) such that $m+1 = |\kappa_\bz| \sqrt{|T|/2}$ and $n+1 = |\kappa_\bo| \sqrt{|T|/2}$. 
    Choose: 
    \begin{enumerate}[(i)]
        \item a set-theoretic section $\xi\from G/T \to G$ for the natural homomorphism $G \to G/T$; \label{item:A(m-n)-choose-chi}
        \item a total order $\leq$ on $G/T$ such that there are no elements between $x$ and $\bar g_0\inv x\inv$, for all $x\in G/T$; 
        \label{item:A(m-n)-choose-leq}
    \end{enumerate}
    and construct tuples $\bar\gamma_\bz$ and $\bar\gamma_\bo$ realizing $\kappa_\bz$ and $\kappa_\bo$, respectively, according to $\pi \circ \xi$ and $\leq$ (\cref{defi:tuple-governed}), where $\pi\from G\to \barr G$ is the natural homomorphism. 
    Consider the $\barr G$-grading $\Gamma_M(\barr T, \barr \beta, \kappa_\bz, \kappa_\bo)$ on $S \coloneqq M(m+1,n+1)$ constructed using the choices of $\barr \D$, $\barr \gamma_\bz$ and $\barr \gamma_\bo$ above (see \cref{def:Gamma-T-beta-kappa-even}), and consider its restriction to $S^{(1)}$. 
    Define ${\Theta \in S}$ by \cref{eq:puting-the-blocks-of-Phi-together-version-A} and ${\theta\from S \to S}$ by
    \cref{eq:theta-with-matrix-3}. 
    Finally, we define $\Gamma_A^{\mathrm{(II)}}(T, \beta, \kappa_\bz, \kappa_\bo, g_0)$ to be the $G$-grading on $S^{(1)}$ by \cref{eq:def-final-grd-A(m-n)}. 
\end{defi}

The following is an easy result that will also be used in \cref{ssec:grds-on-Q(n),ssec:grds-on-Ann}.

\begin{lemma}\label{lemma:barr-D-to-mc-M}
    Let $T$ be a finite abelian group, $\beta\from T\times T \to \pmone$ be an alternating bicharacter and $K\subseteq T$ be a subgroup such that $T = K \times (\rad \beta)$. 
    Set $\barr T \coloneqq T/\rad \beta$ and let $\barr \beta\from \barr T\times \barr T \to \pmone$ be the (nondegenerate) bicharacter induced by $\beta$. 
    Then the natural homomorphism $\pi\from T \to \barr T$ induces a bijection between standard realizations $\barr \D$ associated to $(\barr T, \barr \beta)$ and standard realizations $\mc M$ associated to $(K, \beta\restriction_{K\times K})$ (see \cref{def:standard-realization}). 
    Further, if $\barr \mu \from \barr T \to \pmone$ is the map associated to the transposition on $\barr \D$, then the restriction of $\mu \coloneqq \barr \mu \circ \pi$ to $K$ is the map associated to the transposition on $\mc M$. 
\end{lemma}

\begin{proof}
    Recall that a standard realization $\barr \D$ associated to $(\barr T, \bar \beta)$ is obtained by choosing subgroups $\barr A$ and $\barr B$ of $\barr T$ such that $\barr T = \barr A \times \barr B$ and $\barr \beta (\barr A, \barr A) = \barr \beta (\barr B, \barr B) = 1$. 
    Since $\pi\restriction_K\from K \to \barr T$ is an isomorphism, and $\beta(s,t) = \barr\beta(\pi(s), \pi(t))$, for all $s,t \in K$, the choice of the subgroups $\barr  A$ and $\barr  B$ as above is equivalent to a choice of subgroups $A, B \subseteq K$ such that $K = A\times B$ and $\beta (A, A) = \beta (B, B) = 1$. 
    The ``further'' part follows from \cref{lemma:transp-std-realization}: since
    $\barr \mu ( \bar a \bar b) = \barr\beta (\bar a, \bar b)$, for all $\bar a\in \barr A$ and $\bar b\in \barr B$, we have that $\mu (a b) = \barr \mu (\barr a \barr b) = \barr\beta (\bar a, \bar b) = \beta(a,b)$, for all $a\in A$ and $b\in B$. 
\end{proof}

\begin{prop}\label{prop:m-not-n-Type-II-correspondence}
    Consider $(R, \vphi) \coloneqq M^{\mathrm{ex}}(T, \beta, \kappa_\bz, \kappa_\bo, g_0)$ (\cref{def:model-grd-MxM-even}). 
    Then $\Skew (R,\vphi)^{(1)}$ is isomorphic to $M(m+1, n+1)^{(1)}$ endowed with $\Gamma_A^{\mathrm{(II)}}(T, \beta, \kappa_\bz, \kappa_\bo, g_0)$. 
\end{prop}

\begin{proof}
    We will show how the choices in \cref{defi:type-II-A-m-not-n} correspond to the choices in
    % \cref{def:model-grd-MxM-even}. 
    Definitions \ref{def:std-realization-MxM-QxQ}(a) and \ref{def:model-grd-MxM-even}.
    
    In view of \cref{lemma:barr-D-to-mc-M}, the choices of $K$ and $\barr \D$  give us the same information as the choices of $K$ in item \eqref{item:K-can-be-orthogonal-to-t_1} and $\mc M$ in item \eqref{item:choose-mc-M} of \cref{def:std-realization-MxM-QxQ}(a), and the map associated to the transposition on $\mc M$ is $\mu\restriction_K$. 
    Let $(\D, \vphi_0)$ denote the standard realization constructed using these choices. 
    Note that the map $\eta\from T\to \pmone$ defined in \cref{eq:fix-eta-undouble} is such that $\mathrm{d}\eta = \beta$, $\eta\restriction_{K} = \mu\restriction_K$ and $\eta(f) = -1$, so, by \cref{rmk:eta-for-MxM-or-QxQ}(a), $\eta$ is the map associated to $\vphi_0$. 
    In particular, the $g_0$-admissibility condition for $\kappa$ is the same in \cref{def:model-grd-MxM-odd-or-QxQ,defi:type-II-A-m-not-n}. 
    
    We can use the choices in items \eqref{item:A(m-n)-choose-chi} and \eqref{item:A(m-n)-choose-leq} in \cref{defi:type-II-A-m-not-n} to choose the pair $(\U, B)$ as in \cref{def:model-grd-MxM-even}, by following the same construction as in \cref{ssec:grds-osp}. 
    With respect to the graded basis $\B = \{ u_1, \ldots, u_k \}$ used there, $B$ is represented by the matrix $\Phi \in M_{k_\bz | k_\bo}(\D)$ as in \cref{eq:puting-the-blocks-of-Phi-together}. 
    
    We will now follow \cref{ssec:undoubling} to undouble $M^{\mathrm{ex}}(T, \beta, \kappa_\bz, \kappa_\bo, g_0)$, constructed with the choices above. % First \D
    Consider the central idempotent $\epsilon \coloneqq \frac{1}{2} (1 + X_f)\in \D$ and the $\barr G$-graded-division superalgebra $\D\epsilon$. 
    By \cref{prop:lemma-for-undoubling-and-fine-gradings}, $\D\epsilon$ has parameters $(\barr T, \barr \beta, \barr p)$, so $\D\epsilon \iso \barr \D$. 
    Specifically, an isomorphism is given by $X_t\epsilon \mapsto X_{\pi (t)}$, for all $t\in T$. 
    From now on, we will identify $\D\epsilon$ with $\barr \D$. 
    Also, by \cref{eq:fix-eta-undouble}, the map $\barr\mu$ as defined in \cref{eq:defi-mu-undoubled} coincides with the map $\barr\mu$ defined above, \ie, the map associated to the transposition on $\barr \D$. 
    
    Let $\Lambda \in M_k(\D)$ be the diagonal matrix with entries $\Lambda_{ii} \coloneqq \chi(\deg u_i)$, for all $i \in \{ 1, \ldots, k \}$, as in \cref{ssec:undoubling}. 
    Consider the
    graded basis $\tilde \B = \{ \tilde u_1, \ldots, \tilde u_k \}$ for $\U$, where
    \[\label{eq:tilde-u_i-from-u_i}
        \tilde u_i \coloneqq 
        \begin{cases}
            \sqrt{\Lambda_{ii}\inv} u_i, & \text{if }(\deg u_i) T = g_0\inv (\deg u_i)\inv T;\\
            \hfill \Lambda_{ii}\inv u_i, & \text{if }(\deg u_i) T < g_0\inv (\deg u_i)\inv T;\\
            \hfill u_i, & \text{if }(\deg u_i) T > g_0\inv (\deg u_i)\inv T.
        \end{cases}
    \]
    Let $\tilde \Phi$ be the matrix representing $B$ with respect to the graded basis $\tilde \B$. 
    Under the identification $\D\epsilon = \barr \D$, the matrix $\Lambda \tilde\Phi\epsilon$ becomes $\Theta$ as in \cref{eq:puting-the-blocks-of-Phi-together-version-A}. 
    Therefore, all the data in the description of the undoubled model of $\Skew(R, \vphi)$ coincide with the data in \cref{defi:type-II-A-m-not-n}, concluding the proof.
\end{proof}

Recall that, for every $\kappa\from G/T \to \ZZ_{\geq 0}$, we defined $\kappa\Star\from G/T \to \ZZ_{\geq 0}$ by $\kappa\Star(x) = \kappa(x)\inv$, for all $x \in G/T$ (see \cref{ssec:superdual}).

\begin{thm}\label{thm:final-m-not-n}
    Let $m,n \in \ZZ_{\geq 0}$, $m \neq n$. 
    Every grading on $\Sl(m+1 | n+1)$ is isomorphic to either $\Gamma_A^{\mathrm{(I)}}(T, \beta, \kappa_\bz, \kappa_\bo)$ or $\Gamma_A^{\mathrm{(II)}}(T,\beta, \kappa_\bz, \kappa_\bo, g_0)$ as in \cref{def:A-Type-I,defi:type-II-A-m-not-n}. 
    Gradings belonging to different types are not isomorphic. 
    Within each type, we have:
    
    \noindent\boxed{\mathrm{Type \,\,I}}
    
    \noindent $\Gamma_A^{\mathrm{(I)}}(T, \beta, \kappa_\bz, \kappa_\bo) \iso \Gamma_A^{\mathrm{(I)}}(T', \beta', \kappa_\bz', \kappa_\bo')$ \IFF  $T' = T$ and one of the following conditions holds:
	\begin{enumerate}[(i)]
	    \item $\beta'=\beta$ and there is $g\in G$ such that $\kappa_{\bar 0}'=g \cdot \kappa_{\bar 0}$ and $\kappa_{\bar 1}'=g \cdot \kappa_{\bar 1}$; 
	    \item $\beta'=\beta\inv$ and there is $g\in G$ such that $\kappa_{\bar 0}'=g \cdot \kappa_{\bar 0}\Star$ and $\kappa_{\bar 1}'=g \cdot \kappa_{\bar 1}\Star$.
	\end{enumerate}

    \noindent\boxed{\mathrm{Type \,\,II}}
    
    \noindent $\Gamma_A^{\mathrm{(II)}}(T,\beta, \kappa_\bz, \kappa_\bo, g_0) \iso \Gamma_A^{\mathrm{(II)}}(T,\beta, \kappa_\bz, \kappa_\bo, g_0)$ \IFF
    $T' =T$, $\beta' = \beta$ and there is $g \in G$ such that $\kappa_\bz' = g\cdot\kappa_\bz$, $\kappa_\bo' = g\cdot\kappa_\bo$ and $g_0' = g^{-2}g_0$.
\end{thm}

\begin{proof}
    The result follows from \cref{thm:MxM-type-I} (Type I gradings) and \cref{thm:MxM-even} (Type II gradings). 
    Indeed, by \cref{cor:transfer-R-vphi-to-L}, the isomorphism classes of gradings on ${\Sl(m+1\, | \, n+1)}$ are in bijection with the isomorphism classes of gradings on ${M(m+1, n+1)} \times M(m+1, n+1)\sop$  endowed with the exchange superinvolution. 
    Type I gradings are already described in terms of $M(m+1, n+1)$, and Type II gradings are described in \cref{prop:m-not-n-Type-II-correspondence}.  
    Note that, since $m \neq n$, the isomorphism conditions in \cref{thm:MxM-type-I,thm:MxM-even} simplify: we cannot have $\kappa_{\bar 1}' = g \cdot \kappa_{\bar 0}$ or $\kappa_{\bar 1}' = g \cdot \kappa_{\bar 0}\Star$. 
\end{proof}

% ----------------------------------

\subsection{Gradings on \texorpdfstring{$Q(n)$}{Q(n)}}\label{ssec:grds-on-Q(n)}

We will now discuss the case where $S$ is the associative superalgebra $Q(n+1)$, so $L = S^{(1)}/Z(S^{(1)})$ is the Lie superalgebra $Q(n)$. 

Let us first parametrize Type I gradings. 
As seen in \cref{ssec:classification-assc-super}, every grading on $S$ is odd. 
Nevertheless, we can use parameters in terms of the group $G$ to parametrize the gradings (see \cref{def:Gamma-T-beta-kappa-Q,cor:iso-Q}).  

\begin{defi}\label{def:Q-Type-I}
    Let $n \in \ZZ_{> 0}$, $T^+ \subseteq G$ be a finite subgroup, $\beta^+ \from T^+ \times T^+ \to \FF^\times$ be a nondegenerate bicharacter, $h\in G$ be an element such that $h^2 = 1$, and $\kappa\from G/T^+ \to \ZZ_{\geq 0}$ be a map with finite support such that $n + 1 = |\kappa| \sqrt{|T^+|}$. 
    We will denote by $\Gamma^{\mathrm{(I)}}_Q(T^+, \beta^+, h, \kappa)$ the grading on $L$ induced from the grading $\Gamma_Q (T^+, \beta^+, h, \kappa)$ (see \cref{def:Gamma-T-beta-kappa-Q}) by reduction modulo the center. 
\end{defi}

For Type II gradings on $L$,
set $R \coloneqq S\times S\sop$ and let $\vphi$ be the exchange superinvolution on it. 
At the end of \cref{sec:MxM-and-QxQ-associative}, just before \cref{cor:QxQ-reduced-to-MxM}, we introduced a parametrization of gradings on $(R, \vphi)$ that make $R$ graded-simple. 
By \cref{cor:QxQ-reduced-to-MxM}, $(R, \vphi)$ endowed with such a grading is isomorphic to $Q^{\mathrm{ex}} (T^+, \beta^+, h, \kappa, g_0)$, where $T^+ \subseteq G$ is a finite $2$-elementary subgroup, $\beta^+\from T^+\times T^+ \to \FF^\times$ is an alternating bicharacter with $\rad \beta^+ = \langle f \rangle$ for some $e\neq f \in T^+$, $h\in G$ is an element in $G$ such that $h^2 = f$, $g_0\in G$, and $\kappa\from G/T^+ \to \ZZ_{\geq 0}$ is a $g_0$-admissible map (see \cref{defi:odd-D-kappa-g_0-admissible}) such that $n+1 = |\kappa| \sqrt{|T^+|/2}$. 

We will use the parameters $(T^+, \beta^+, h, \kappa, g_0)$ to construct a representative of the corresponding isomorphism class of Type II gradings directly on the superalgebra $L$ instead of going through $\Skew(R, \vphi)$. 
Recall that the parametrization above for gradings on $\Skew(R, \vphi)$ was obtained by writing $R = R\even \oplus u R\even$, where $0 \neq u \in Z(R)\odd$, and using that $(R\even, \vphi\restriction_{R\even})$ is of type $M\times M\sop$.  
We will follow an analogous strategy here. 
Recall that $S = Q(n+1) = S\even \oplus u S\even$, where
$ 
    u \coloneqq
    \left(\begin{array}{c|c}
        0 & I_{n+1}\\
        \hline
        I_{n+1} & 0\\
    \end{array}\right)
    \in Z(S)\odd
$,
that $Z(S) = \FF1 \oplus \FF u$ can be identified with the associative superalgebra $Q(1)$, that $S\even$ can be identified with $M_{n+1}(\FF) = M(n+1, 0)$ and that $S$ can be identified with $Q(1)\tensor S\even$ via Kronecker product. 
% With these identifications, we have $S \iso Q(1)\tensor S\even$ via Kronecker product. 

Let $\barr G \coloneqq G/\langle f \rangle$. 
We will, first, construct a $\barr G$-grading and a (super-)anti-automorphism on $S\even$, following the same steps as in \cref{ssec:grds-on-A-m-n}, but with $T^+$ playing the role of $T$ and $\beta^+$ playing the role of $\beta$, and, then, we will extend the grading and the super-anti-automorphism to $S$. 
Let $\pi\from G \to \barr G$ denote the natural homomorphism, set $\barr {T^+} \coloneqq \pi(T^+)$, let $\barr {\beta^+}\from \barr {T^+} \times \barr {T^+} \to \FF^\times$ be the (nondegenerate) bicharacter on $\barr {T^+}$ induced by $\beta^+$, and consider $\kappa$ as a map defined on $\barr G/\barr {T^+} \iso G/{T^+}$.

Let $\barr \D\even$ be the chosen standard realization associated to $(\barr{T^+}, \barr{\beta^+})$, let $\barr {\mu^+}\from \barr{T^+} \to \FF^\times$ be the map associated to the transposition on $\barr \D\even$, and set $\mu^+ \coloneqq \barr {\mu^+} \circ \pi\restriction_{T^+}$. 
Fix a subgroup $K \subseteq T^+$ such that $T^+ = K \times \langle f \rangle$, and let $\chi\from T^+ \to \FF^\times$ be the character defined by $\chi(K) = 1$ and $\chi(f) = -1$. 
Then define $\eta^+\from T^+ \to \pmone$ by
\[\label{eq:fix-eta-plus-undouble-Q}
    \forall t\in T^+, \quad \eta^+(t) \coloneqq \mu^+(t) \chi\inv(t).
\] 
Since $\mathrm{d} \barr{\mu^+} = \barr{\beta^+}$, we get $\mathrm{d} \eta^+ = \beta^+$. 
In the proof of \cref{prop:Q-Type-II-correspondence}, we will show that $\eta^+$ is the map associated to the superinvolution on the even part of a graded-division superalgebra as in \cref{def:std-realization-MxM-QxQ}(c). 

Extend $\chi$ to $G$ and set $k \coloneqq |\kappa|$. 
Following the construction after \cref{eq:fix-eta-undouble} in \cref{ssec:grds-on-A-m-n} with $\kappa_\bz \coloneqq \kappa$ and $\kappa_\bo$ being the zero map, we get an elementary $\barr G$-grading on $M_k(\FF) = M(k , 0)$. 
We identify the $\barr G$-graded superalgebra $M_{k}(\barr \D\even) = M_{k}(\FF) \tensor \barr\D\even$ with $S\even = M_{n+1}(\FF)$ via Kronecker product, and define 
\[\label{eq:Theta_bz-for-Q}
    \Theta_\bz \coloneqq \begin{pmatrix}
        \Theta(\bar 0, x_1)&& \\
        & \ddots &\\
        && \Theta(\bar 0, x_{\ell})
    \end{pmatrix},
\]
where $x_1 < \cdots < x_{\ell}$ are the elements of the set $\{ x \in \supp \kappa \mid x \leq g_0\inv x\inv \}$, and $\Theta(\bar 0, x)$, for $x\in \barr G/\barr {T^+}$, is as in \cref{defi:blocks-of-Theta}. 
We, then, define the (super-)anti-automorphism $\theta\from S\even \to S\even$ by 
\[\label{eq:theta-with-matrix-4}
    \forall X\in M_{n+1}(\FF), \quad \theta(X) \coloneqq \Theta_\bz\inv\, X\transp\, \Theta_\bz.
\]

We extend the $\barr G$-grading on $S\even$ to $S = S\even \oplus u S\even$ by declaring $\deg u = \barr h$, \ie, we define $S\odd_{\barr{h}\bar g} = u 
S\even_{\bar g}$, for all $\barr g \in \barr G$. 
In terms of the identification $S = Q(1) \tensor S\even$, this corresponds to the usual tensor product of graded superalgebras. 

\begin{remark}\label{rmk:change-Q(1)-of-place}
    Note that $S = Q(1) \tensor S\even = Q(1) \tensor M_k(\FF) \tensor \barr \D\even \iso M_k(\FF) \tensor Q(1) \tensor \barr\D\even$. 
    Hence, this $\barr G$-grading is isomorphic to $\Gamma_Q(\barr {T^+}, \barr {\beta^+}, \bar h, \kappa)$ (see \cref{def:Gamma-T-beta-kappa-Q}), by choosing the standard realization of type Q (\cref{def:standard-realization-Q}) to be $\barr \D \coloneqq Q(1)\tensor \barr\D\even = \barr\D\even \oplus u \barr\D\even$.
\end{remark}

We also extend the (super-)anti-automorphism $\theta$ to $S$, by declaring $\theta (u) = \bi u$. 
In terms of the identification $S = Q(1)\tensor S\even$, this corresponds to  
\[\label{eq:theta-with-matrix-5}
    \forall X\in Q(n+1), \quad \theta(X) \coloneqq \Theta\inv\, X\sTq\, \Theta,
\]
where 
\[\label{eq:Theta-for-Q}
    \Theta \coloneqq \left(\begin{array}{c|c}
            \Theta_\bz & 0\\
            \hline
            0 & \Theta_\bz
        \end{array}\right).
\]
Note that, differently from \cref{eq:theta-with-matrix-3}, we are using the queer supertranspose (\cref{def:queer-stp}) here.

Finally, we define a $G$-grading on $S^{(1)}$ by 
\[\label{eq:final-G-grd-on-Q}
    \forall g\in G, \quad S^{(1)}_g \coloneqq \{ s \in S^{(1)}_{\barr g} \mid \theta(s) = - \chi(g) s\},
\]
and reduce it modulo the center to obtain a $G$-grading on $L$. 

The following definition summarizes the above construction: 

\begin{defi}\label{defi:type-II-Q}
    Let $n \in \ZZ_{> 0}$ and let $S$ denote the associative superalgebra $Q(n+1)$. 
    Let $T^+ \subseteq G$ be a finite $2$-elementary subgroup, let $\beta^+\from {T^+\times T^+} \to \FF^\times$ be an alternating bicharacter with $\rad \beta^+ = \langle f \rangle$, for some element $e \neq f\in T$ and let $h \in G$ be an element such that $h^2=f$. 
    Set $\barr G \coloneqq G/\langle f \rangle$ and let $\pi\from G\to \barr G$ be the natural homomorphism. 
    Set $\barr {T^+} \coloneqq \pi(T^+)$, and let $\barr {\beta^+}$ be the (nondegenerate) alternating bicharacter on $\barr {T^+}$ induced by $\beta^+$. 
    Consider the chosen standard realization $\barr \D\even$ of a matrix algebra with division grading associated to $(\barr {T^+}, \barr{\beta^+})$, fix a subgroup $K \subseteq T^+$ such that $T^+ = K \times \langle f \rangle$, and define $\eta^+\from T \to \pmone$ by \cref{eq:fix-eta-plus-undouble-Q}. 
    Then, let $g_0 \in G$ be any element and let $\kappa\from G/{T^+} \to \ZZ_{\geq 0}$ be a $g_0$-admissible map (\cref{defi:odd-D-kappa-g_0-admissible}) such that $n+1 = |\kappa| \sqrt{|T^+|/2}$. 
    Choose:
    \begin{enumerate}[(i)]
        \item a set-theoretic section $\xi\from G/T^+ \to G$ for the natural homomorphism $G \to G/T^+$;
        \label{item:choice-xi-Q}
        %
        \item a total order $\leq$ on $G/T^+$ such that there are no elements between $x$ and $\bar g_0\inv x\inv$, for all $x\in G/T^+$; 
        \label{item:choice-leq-Q}
    \end{enumerate}
    and construct a tuple $\bar\gamma$ realizing $\kappa$ according to $\pi \circ \xi$ and $\leq$ (\cref{defi:tuple-governed}). 
    Consider the $\barr G$-grading $\Gamma_M(\barr {T^+}, \barr {\beta^+}, \kappa)$ on $S\even \iso M_{n+1}(\FF)$ constructed using the choices of $\barr \D\even$ and $\barr \gamma$ above (see \cref{def:Gamma-T-beta-kappa}), and extend it to $S$ by declaring $\deg u \coloneqq \barr h$. 
    Define $\Theta_\bz \in M_{n+1}(\FF)$ by \cref{eq:Theta_bz-for-Q}, $\Theta \in S$ by \cref{eq:Theta-for-Q} and ${\theta\from S \to S}$ by
    \cref{eq:theta-with-matrix-5}
    Finally, we define $\Gamma_Q^{\mathrm{(II)}}(T^+, \beta^+, h, \kappa, g_0)$ to be the $G$-grading on $L = S^{(1)}/Z(S^{(1)})$ induced from the $G$-grading $S^{(1)}$ given by \cref{eq:final-G-grd-on-Q}.
\end{defi}

% -------------------------------

\begin{remark}
    We note that in \cite[Theorem 5.1]{paper-Qn} it is described how to extend a $G$-grading $\Gamma$ on the even part of the Lie superalgebra $Q(n)$ to the whole superalgebra, without using a full description of $\Gamma$. 
    The construction above corresponds to the case where $\Gamma$ is of Type II. 
\end{remark}

% \begin{lemma}
%     The $\barr G$-grading on $S$ constructed above is isomorphic to $\Gamma_Q(\barr {T^+}, \barr {\beta^+}, \bar h, \kappa)$ (see \cref{def:Gamma-T-beta-kappa-Q}). 
% \end{lemma}

% \begin{proof}
%     The $\barr G$-grading on $S$ is given by $S = Q(1) \tensor S\even = Q(1) \tensor $
    
%     Since $u$ is central. 
    
%     by using the choices (i) $\barr \D \coloneqq \barr \D\even \oplus u\barr\D\even$ and (ii) $\barr \gamma$ as above
% \end{proof}

% --------------------

% A BIT OF OVER-EXPLANATION:

% The following is an easy result that will be also used in \cref{ssec:grds-on-Ann}:

% \begin{lemma}
%     Let $\D$ be an odd graded-division superalgebra, $\U$ be graded right $\D$-supermodule and $B\from \U \times \U \to \D$ be a homogeneous nondegenerate sesquilinear form with $\deg B \in G$. 
%     Set $R = \End_\D(\U)$ and, as in \cref{rmk:R-even-identificatios}, identify $R\even$ with $\End_{\D\even} (\U\even)$. 
%     If $\vphi\from R \to R$ is the superadjunction with respect to $B$ (\cref{def:superadjunction}), then $\vphi\restriction_{R\even}\from R\even \to R\even$ is the superadjunction with respect to $B\restriction_{\U\even \times \U\even} \from \U\even \times \U\even \to \D\even$. \qed
% \end{lemma}

\begin{prop}\label{prop:Q-Type-II-correspondence}
    Consider $(R, \vphi) \coloneqq Q^{\mathrm{ex}}(T^+, \beta^+, h, \kappa, g_0)$, as defined before \cref{cor:QxQ-reduced-to-MxM}. 
    Then the graded Lie superalgebra $\Skew (R,\vphi)^{(1)}/Z(\Skew (R,\vphi)^{(1)})$ is isomorphic to the Lie superalgebra $Q(n)$ endowed with $\Gamma_Q^{\mathrm{(II)}}(T^+, \beta^+, h, \kappa, g_0)$. 
\end{prop}

\begin{proof}
    % Recall what Q^ex means, including what $T$ and $t_p$ are, and how to see \kappa with the proper domain. 
    First recall that, by definition, $Q^{\mathrm{ex}}(T^+, \beta^+, h, \kappa, g_0) = Q^{\mathrm{ex}} (T, \beta, t_p, \kappa, g_0)$ (see \cref{def:model-grd-MxM-odd-or-QxQ}), where $t_p \coloneqq (h, \bar 1)$, $T \coloneqq T^+ \cup t_p T^+$, $\beta\from T\times T \to \FF^\times$ is the unique alternating bicharacter extending $\beta^+$ such that $\rad \beta = \langle t_p \rangle$, and $\kappa$ is seen as defined on $G^\#/T \iso G/ T^+$. 
    
    % Say what we are going to do
    We will now show how the choices in \cref{defi:type-II-Q} correspond to the choices in \cref{def:std-realization-MxM-QxQ}(c) and \cref{def:model-grd-MxM-odd-or-QxQ}. 
    % (and, hence, also in \cref{def:std-realization-MxM-QxQ}(c)).
    
    % The choices of K and \barr\D\even given us the correct \eta^+, and fix \D as \M\tensor\C
    By \cref{lemma:barr-D-to-mc-M}, with $(T^+, \beta^+)$ playing the role of $(T, \beta)$, the choices of $K$ and $\barr \D\even$  give us the same information as the choices of $K$ in item \eqref{item:K-can-be-orthogonal-to-t_1} and $\mc M$ in item \eqref{item:choose-mc-M} of \cref{def:std-realization-MxM-QxQ}(c), and the map associated to the transposition on $\mc M$ is $\mu^+\restriction_K$. 
    Let $(\D, \vphi_0)$ denote the standard realization of a graded-division superalgebra with superinvolution constructed using these choices. 
    Note that the map $\eta^+\from T\to \pmone$ defined in \cref{eq:fix-eta-plus-undouble-Q} is such that $\mathrm{d}\eta^+ = \beta^+$, $\eta\restriction_{K} = \mu^+\restriction_K$ and $\eta(f) = -1$, so, by \cref{eq:eta+-unsharpening-QxQ}, $\eta^+$ is the map associated to $\vphi_0\restriction_{\D\even}$. 
    In particular, the $g_0$-admissibility condition for $\kappa$ is the same in \cref{def:model-grd-MxM-odd-or-QxQ,defi:type-II-Q}. 
    
    % Constructs $(\U, B)$
    In \cref{def:model-grd-MxM-odd-or-QxQ}, we have to choose a graded right $\D$-supermodule $\U$ and a $\vphi_0$-sesquilinear form $B\from \U \times \U \to \D$ such that $(\U, B)$ has inertia determined by $\kappa$. 
    We will first construct a graded right $\D\even$-supermodule $\U\even$ and a $(\vphi_0\restriction_{\D\even})$-sesquilinear form $B\even\from \U \times \U \to \D$ such that $(\U\even, B\even)$ has inertia determined by $(\kappa_\bz, \kappa_\bo)$, where we take $\kappa_\bz \coloneqq \kappa$  (seen as a map defined on $G/T^+$) and $\kappa_\bo$ to be the zero map. 
    % (we note that, by \cref{rmk:R-even-identificatios}, the superadjunction with respect to $B\even$ would give us a). 
    To that end, it suffices to follow the same construction as in \cref{ssec:grds-osp}. 
    In other words, define $\D\even \coloneqq (\D\even)^{[g_1]}\oplus \cdots \oplus (\D\even)^{[g_k]}$, where $(g_1, \ldots, g_k)$ is the $k$-tuple of elements in $G$ realizing $\kappa$ according to $\xi$ and $\leq$, and define $B$ by $B(u_i, u_j) \coloneqq \Phi_{ij}$, where $\B \coloneqq \{ u_1, \ldots, u_k \}$ is the canonical graded $\D\even$-basis of $\U\even$ and $\Phi \in M_k(\D\even)$ is the matrix defined by \cref{eq:puting-the-blocks-of-Phi-together}. 
    We then define $\U \coloneqq \U\even \tensor_{\D\even} \D$. 
    Note that $\B$ is an even graded $\D$-basis of $\U$, and let $B\from \U \times \U \to \D$ be the unique $\vphi_0$-sesquilinear extension of $B\even$, which is also represented by $\Phi$, seen as a matrix in $M_k(\D)$.  
    
    % Undoubling what we just constructed
    We will now follow \cref{ssec:undoubling} to undouble $Q^{\mathrm{ex}}(T^+, \beta^+, h, \kappa, g_0)$, constructed with the choices above. 
    % $\D$ is what it should be 
    Consider the central idempotent $\epsilon \coloneqq \frac{1}{2}(1 + X_f)\in \D$ and the $\barr G$-graded-division superalgebra $\D\epsilon$. 
    By \cref{prop:lemma-for-undoubling-and-fine-gradings}, $\D\epsilon$ has parameters $(\barr T, \barr \beta, \barr p)$. 
    Also, $\rad \barr\beta = \langle \barr t_p \rangle$, so $\D\epsilon$ is isomorphic to a standard realization of type Q associated to $(\barr {T^+}, \barr{\beta^+}, \barr h)$ (see discussion preceding \cref{def:standard-realization-Q}), specifically, it is isomorphic to $\barr \D \coloneqq Q(1) \tensor \barr \D\even$ as in \cref{rmk:change-Q(1)-of-place}. 
    From now on, we will identify $\D\epsilon$ with $\barr \D$. 
    
    % Undoubling $\eta$
    We need to extend $\chi$ to $G^\#$. 
    Since $\chi(f) = -1$, we can do it in a way such that $\chi(t_p) = \bi$. 
    Let $\eta\from T \to \pmone$ be the map associated to $\vphi_0$ and let $\barr\mu\from \barr T \to \FF^\times$ be as defined in \cref{eq:defi-mu-undoubled}. 
    % We claim that  is the map associated to the queer supertranspose on $\barr \D$. @1 
    We have already shown that $\eta\restriction_{T^+} = \eta^+$ and, hence, by \cref{eq:fix-eta-plus-undouble-Q}, we have that $\barr\mu\restriction_{\barr{T^+}} = \barr{\mu^+}$.
    By \cref{rmk:eta-for-MxM-or-QxQ}(c), we have $\barr\mu (\barr t_p) = \eta(t_p) \chi(t_p) = \bi$. 
    It follows that $\barr\mu$ is the map associated to the queer supertranspose on $\barr \D$. 
    
    % Undoubling $\Theta$
    As in the proof of \cref{prop:m-not-n-Type-II-correspondence}, let $\Lambda \in M_k(\D)$ be the diagonal matrix with entries $\Lambda_{ii} \coloneqq \chi(\deg u_i)$, consider a different graded $\D$-basis $\tilde \B = \{ \tilde u_1, \ldots, \tilde u_k \}$ of $\U$, where $\tilde u_i$ is defined as in \cref{eq:tilde-u_i-from-u_i}, and let $\tilde \Phi \in M_k(\D)$ be the matrix representing $B$ with respect to $\tilde \B$. 
    Clearly, all the entries of $\tilde\Phi$ are in $\barr \D\even$. 
    We will denote $\tilde\Phi$ by $\tilde\Phi_\bz$ when seen as a matrix in $M_k(\D\even)$. 
    In \cref{ssec:undoubling}, $M_k(\D)\epsilon$ was identified with $M_k(\barr \D)$, so we now identify $M_k(\D\even)\epsilon$ with $M_k(\barr \D\even)$. 
    % In the definition of $\Gamma_Q^{\mathrm{(II)}}(T^+, \beta^+, h, \kappa, g_0)$, we identified $M_k(\barr \D\even) = M_k(\FF) \tensor \barr \D\even$ with $S\even = M_{n+1}(\FF)$ via Kronecker product. 
    Under this identification, the matrix $\Lambda \tilde\Phi_\bz \epsilon$ becomes $\Theta_\bz$ as in \cref{eq:Theta_bz-for-Q}. 
    Also, following the isomorphisms
    \[
        M_k(\barr \D) \iso M_k(\FF) \tensor \barr \D \iso M_k(\FF) \tensor Q(1) \tensor \barr \D\even \iso Q(1) \tensor M_k(\FF) \tensor \barr \D\even \iso Q(1) \tensor S\even,
    \]
    the $\barr G$-grading on $M_k(\barr \D)$ becomes the $\barr G$-grading we defined on $S = Q(1) \tensor S\even$, and $\Lambda\tilde\Phi\epsilon$ is sent to $I_2 \tensor \Theta_\bz = \Theta$. 
    Therefore, all the data in the description of the undoubled model of $\Skew(R, \vphi)$ coincide with the data in \cref{defi:type-II-Q}, concluding the proof.
\end{proof}

% --------------------


\begin{thm}\label{thm:final-Q(n)}
    Let $n \geq 2$. 
    Every grading on the simple Lie superalgebra $Q(n)$ is isomorphic to either $\Gamma_Q^{\mathrm{(I)}}(T^+, \beta^+, h, \kappa)$ or $\Gamma_Q^{\mathrm{(II)}}(T^+, \beta^+, h, \kappa, g_0)$ as in \cref{def:Q-Type-I,defi:type-II-Q}. 
    Gradings belonging to different types are not isomorphic. 
    Within each type, we have:
    
    \noindent\boxed{\mathrm{Type \,\,I}}
    
    \noindent $\Gamma_Q^{\mathrm{(I)}}(T^+, \beta^+, h, \kappa) \iso \Gamma_Q^{\mathrm{(I)}}(T'^+, \beta'^+, h', \kappa')$ \IFF  $T'^+ = T^+$, $h' = h$ and one of the following conditions holds:
	\begin{enumerate}[(i)]
	    \item $\beta'^+ = \beta^+$ and there is $g\in G$ such that $\kappa'=g \cdot\kappa$; 
	    \item $\beta'^+ = (\beta^+)\inv$ and there is $g\in G$ such that $\kappa'= g \cdot \kappa\Star$.
	\end{enumerate}

    \noindent\boxed{\mathrm{Type \,\,II}}
    
    \noindent $\Gamma_A^{\mathrm{(II)}}(T^+, \beta^+, h, \kappa, g_0) \iso \Gamma_A^{\mathrm{(II)}}(T'^+, \beta'^+, h', \kappa', g_0')$ \IFF
    $T'^+ =T^+$, $\beta'^+ = \beta^+$, $h' = h$ and there is $g \in G$ such that $\kappa' = g\cdot\kappa$ and $g_0' = g^{-2}g_0$. 
\end{thm}

\begin{proof}
    The result follows from \cref{thm:QxQ-type-I} (Type I) and
    \cref{cor:QxQ-reduced-to-MxM} (Type II). 
    Indeed, by \cref{cor:transfer-R-vphi-to-L}, the isomorphism classes of gradings on the Lie superalgebra $Q(n)$ are in bijection with the isomorphism classes of gradings on the associative superalgebra $Q(n+1) \times Q(n+1)\sop$  endowed with the exchange superinvolution. 
    Type I gradings are already described in terms of the associative superalgebra $Q(n+1)$, and Type II gradings are described in \cref{prop:Q-Type-II-correspondence}. 
\end{proof}


% ---------------------------

\subsection{Gradings on \texorpdfstring{$A(n,n)$}{A(n,n)}}\label{ssec:grds-on-Ann}

We now go to our final series of Lie superalgebras, $A(n,n)$. 
Let $S = M(n+1, n+1)$,  
so $L = \mathfrak{psl} (n+1, n+1) = S^{(1)}/Z(S^{(1)})$. 

Gradings on $S$ are separated into two classes: even gradings, which are similar to what we saw in \cref{ssec:grds-on-A-m-n}, and odd gradings. 
The gradings on $L$ induced by even gradings on $S$ will be said to be of \emph{Type I\textsubscript{M}}, and the gradings on $L$ induced by odd gradings on $S$ will be said to be \emph{Type I\textsubscript{Q}}. 

\begin{remark}\label{prop:Q-implies-GxZZ2-grading-on-M}
    Our notation is justified by the following fact: 
    every $G$-grading of Type I on the Lie superalgebra $Q(n)$ gives rise to a $G\times \ZZ_2$-grading of Type I\textsubscript{Q} on $A(n,n)$. 
    Indeed, the associative superalgebra $Q(n+1)$ consists of the fixed points of the order $2$ automorphism $\pi$ of $M(n+1, n+1)$, and any automorphism of $Q(n+1)$ extends uniquely to an automorphism of $M(n+1, n+1)$ commuting with $\pi$ (see \cref{sec:Aut-Lie-chap}). 
\end{remark}

\begin{defi}\label{def:Type-I_M}
    Let $n\in \ZZ_{> 0}$, $T \subseteq G$ be a finite subgroup, $\beta\from T\times T \to \FF^\times$ be a nondegenerate alternating bicharacter, and $\kappa_\bz, \kappa_\bo \from G/T \to \ZZ_{\geq 0}$ be maps with finite support such that $|\kappa_\bz| \sqrt{|T|} = |\kappa_\bo| \sqrt{|T|} = n+1$. 
    We define $\Gamma^{\mathrm{(I_M)}}_A(T, \beta, \kappa_\bz, \kappa_\bo)$ to be the grading on $L$ induced from the grading $\Gamma^{\mathrm{(I)}}_A(T, \beta, \kappa_\bz, \kappa_\bo)$ on $M(n+1, n+1)^{(1)}$ (\cref{def:A-Type-I}) by reduction modulo the center.
\end{defi}

To parametrize the gradings of Type I\textsubscript{Q}, we recall the parametrization of odd gradings on $S$ in terms of the group $G$ (\cref{ssec:grds-M(m-n)-only-G}). 
The character $\chi_0 \in \widehat{T^+}$ in the next definition plays the role of $\chi$ there. 

\begin{defi}\label{def:Type-I_Q}
    Let $T^+ \subseteq G$ be a finite subgroup, let $\beta^+\from T^+ \times T^+ \to \FF^\times$ be an alternating bicharacter with $\rad \beta^+ = \langle t_p \rangle$, where $t_p\in T^+$ is an element of order $2$, and let $\kappa \from G/T^+ \to \ZZ_{\geq 0}$ be a map with finite support such that $|\kappa| \sqrt{2|T^+|} = n+1$. 
    Choose a character $\chi_0 \in \widehat{T^+}$ such that $\chi_0 (t_p) = -1$. 
    Then let $a \in T^+$ be the unique element such that $\chi_0^2 = \beta^+(a, \cdot)$ and $\chi_0(a) = 1$ (see \cref{lemma:chi-defines-a}). 
    For each element $h \in G$ such that $h^2 = a$, we set $t_1 \coloneqq (h, \bar 1) \in G^\#$, $T^- \coloneqq t_1 T^+ \subseteq G^\#$ and $T \coloneqq T^+ \cup T^-$. 
    Let $p\from T \to \ZZ_2$ be the homomorphism with kernel $T^+$, and let $\beta\from T \times T \to \FF^\times$ be the unique alternating bicharacter such that $\beta\restriction_{T^+ \times T^+} = \beta^+$ and $\beta(t_1, t) = \chi(t)$, for all $t\in T^+$ (see \cref{lemma:existence-beta}). 
    We will denote by  $\Gamma^{\mathrm{(I_Q)}}_A(T^+, \beta^+, h, \kappa)$ the grading on $L$ induced from the grading $\Gamma_M(T, \beta, p, \kappa)$ on $S$ (see \cref{def:Gamma-T-beta-kappa-odd}) by restriction and reduction modulo the center. 
\end{defi}

Recall that an element $h \in G$ (and, hence, a grading $\Gamma^{\mathrm{(I_Q)}}_A(T^+, \beta^+, h, \kappa)$) exists \IFF $\theta(T^+_{[2]})^\perp \subseteq \barr G^{[2]}$ (see \cref{def:A_[m],defi:perp,prop:O_M-non-empty}).

The Type II gradings on $L$ will also be subdivided into different types. 
As in the case $m\neq n$ (\cref{ssec:grds-on-A-m-n}), these gradings correspond to gradings on $(R, \vphi)$ making $R$ graded-simple, where $R \coloneqq S \times S\sop$ and $\vphi$ is the exchange superinvolution. 

If the grading on $(R,\vphi)$ is even, then, by \cref{thm:MxM-even}, $R$ is isomorphic to $M^{\mathrm{ex}} (T, \beta, \kappa_\bz, \kappa_\bo, g_0)$, where $T$ is a finite elementary $2$-group, $\beta\from {T\times T} \to \FF^\times$ is an alternating bicharacter with $\rad \beta = \langle f \rangle$ for some $e \neq f \in T$, $g_0$ is an element in $G^\#$, and $\kappa_\bz, \kappa_\bo \from G/T \to \ZZ_{\geq 0}$ are $g_0$-admissible maps (see \cref{inertia-even-and-odd-case}) such that $|\kappa_\bz| \sqrt{|T|} = |\kappa_\bo| \sqrt{|T|} = n+1$ (see \cref{def:model-grd-MxM-even}). 
Write $g_0$ as $(h_0, p_0)$, with $h_0 \in G$ and $p_0 \in \ZZ_2$. 
If $p_0 = \bar 0$, we will say that the corresponding grading on $L$ is of Type II\textsubscript{$\mathfrak{osp}$}. 
If $p_0 = \bar 1$, we will say that the corresponding grading on $L$ is of Type II\textsubscript{P}. 

\begin{remark}
    A $G$-grading on the Lie superalgebra $\osp(n+1|n+1)$ (respectively, $P(n)$) gives rise to a $G \times \ZZ_2$-grading on $A(n,n)$ of Type II\textsubscript{$\mathfrak{osp}$} (respectively, II\textsubscript{P}). 
\end{remark}

Recall that, before \cref{eq:fix-eta-undouble}, we fixed a subgroup $K \subseteq T^+$ such that $T^+ = K \times \langle f \rangle$, which was used in \cref{defi:type-II-A-m-not-n}. 

\begin{defi}\label{def:type-II-osp}
    Let $n\in \ZZ_{> 0}$. 
    We define $\Gamma^{\mathrm{(II_{\mathfrak{osp}})}}_A(T, \beta, \kappa_\bz, \kappa_\bo, h_0)$ to be the grading on $L$ induced from the grading $\Gamma^{\mathrm{(II)}}_A(T, \beta, \kappa_\bz, \kappa_\bo, h_0)$ on $M(n+1, n+1)^{(1)}$ (\cref{defi:type-II-A-m-not-n}) by reduction modulo the center.
\end{defi}

To construct a model for gradings of Type II\textsubscript{P}, we follow a similar approach as for Type II\textsubscript{$\mathfrak{osp}$}, but it is simpler since we do not need to choose $\xi$ and $\leq$ to define the matrix $\Theta$ (in the same way, these choices were needed in \cref{ssec:grds-osp} but not in \cref{ssec:grds-P(n)}). 
Also, $\kappa_\bo$ is determined by $\kappa_\bz$ and $h_0$ (recall \cref{inertia-even-and-odd-case}).

\begin{defi}\label{def:type-II-P}
    Let $\barr G \coloneqq G/\langle f \rangle$, $T \coloneqq T/\langle f \rangle$, and let $\bar \beta$ be the nondegenerate alternating bicharacter on $\barr T$ induced by $\beta$. 
    Choose a $k$-tuple $\gamma_\bz = (g_1, \ldots, g_k)$ of elements in $G$ realizing $\kappa_\bz$. 
    Let $\barr \mu\from \barr T \to \FF^\times$ be the map associated to the transposition on $\barr \D$, 
    let $\chi \in \widehat{T}$ be the character such that $\chi(K) = 1$ and $\chi(f) = -1$, and extend $\chi$ to a character of $G$, which will also be denoted by $\chi$. 
    Then define $\mu \coloneqq \bar\mu \circ \pi\restriction_{T}$, where $\pi\from G \to \barr G$ is the natural homomorphism, and fix $\eta\from T \to \pmone$ as in \cref{eq:fix-eta-undouble}. 
    Also, set $\gamma_\bo = (h_0\inv g_1\inv, \ldots, h_0\inv g_k\inv)$.   
    Consider the $\barr G$-grading $\Gamma_M(\barr T, \barr \beta, \kappa_\bz, \kappa_\bo)$ on $S \coloneqq M(n+1,n+1)$ using the choices above (see \cref{def:Gamma-T-beta-kappa-even}), and consider its restriction to $S^{(1)}$. % = \bigoplus_{\bar g \in \barr G} S^{(1)}_{\bar g}$. 
    Consider ${\Theta \in S}$ 
    given by
    \[\label{eq:Theta-for-Type-II-P}
        %
        \sbox0{$\begin{matrix}
            1&& \\
            & \ddots &\\
            && 1
        \end{matrix}$}
        %
        \sbox1{$\begin{matrix}
            \chi(h_0\inv g_1^{-2})&& \\
            & \ddots &\\
            && \chi(h_0\inv g_k^{-2})
        \end{matrix}$}
        %
        \Theta \coloneqq
        \left(\begin{array}{c|c}
            0 & \usebox{0}\\
            \hline
            \usebox{1} & 0
        \end{array}\right) \tensor 1_{\barr \D}.
    \]
    %
    and ${\theta\from S \to S}$ as in
    \cref{eq:theta-with-matrix-3}. 
    We define $\Gamma_A^{\mathrm{(II_P)}}(T, \beta, \kappa_\bz, h_0)$ to be the $G$-grading on $L = S^{(1)}/Z(S^{(1)})$ induced from the grading $S^{(1)} = \bigoplus_{g\in G} S^{(1)}_g$, where
    \[
        S^{(1)}_{g} \coloneqq \{ s\in S^{(1)}_{\bar g} \mid \theta(s) = - \chi(g) s \},
    \]
    for all $g\in G$. 
\end{defi}

% @2

We now proceed to the last case, the odd gradings of Type II, which will be referred to as gradings of Type II\textsubscript{Q}. 

\begin{remark}\label{prop:Q-implies-GxZZ2-grading-on-M-II}
    CHANGE IT!: 
    Our notation is justified by the following fact: 
    every $G$-grading of Type I on the Lie superalgebra $Q(n)$ gives rise to a $G\times \ZZ_2$-grading of Type I\textsubscript{Q} on $A(n,n)$. 
    Indeed, the associative superalgebra $Q(n+1)$ consists of the fixed points of the order $2$ automorphism $\pi$ of $M(n+1, n+1)$, and any automorphism of $Q(n+1)$ extends uniquely to an automorphism of $M(n+1, n+1)$ commuting with $\pi$ (see \cref{sec:Aut-Lie-chap}). 
\end{remark}

At the end of \cref{sec:MxM-and-QxQ-associative}, just before \cref{cor:MxMsop-odd-only-G}, we introduced a parametrization of odd gradings on $(R, \vphi)$ that make $R$ graded-simple. 
By \cref{cor:MxMsop-odd-only-G}, $(R, \vphi)$ endowed with such a grading is isomorphic to $M^{\mathrm{ex}}(T^+, \beta^+, t_p, h, \kappa, g_0)$, where $T^+ \subseteq G$ is a finite $2$-elementary subgroup, $e\neq t_p \in T^+$, $h \in G$ is such that $f \coloneqq h^2 \in T^+ \setminus \langle t_p \rangle$, $\beta^+\from T^+ \times T^+ \to \pmone$ is an alternating bicharacter such that $\rad \beta^+ = \langle t_p, f \rangle$, $g_0\in G$, and $\kappa\from G/T^+ \to \ZZ_{\geq 0}$ is a $g_0$-admissible map (see \cref{defi:odd-D-kappa-g_0-admissible}) such that $n+1 = |\kappa| \sqrt{|T^+|}/2$. 

We will use the parameters $(T^+, \beta^+, t_p, h, \kappa, g_0)$ to construct a representative of the corresponding isomorphism class of Type II\textsubscript{Q} gradings directly on the superalgebra $L$ instead of going through $\Skew(R, \vphi)$. 
Let $\tilde S$ denote the algebra $M_{n+1}(\FF)$. 
Using the Kronecker product, we can identify $S = M(n+1, n+1)$ with $M(1, 1) \tensor \tilde S$.

Let $\barr G \coloneqq G/\langle f \rangle$. 
We will, first, construct a $\barr G$-grading and a (super-)anti-automorphism on $\tilde S$, in a fashion similar to what we did for $S\even$ in \cref{ssec:grds-on-Q(n)}, and, then, we will extend the grading and the super-anti-automorphism to $S$. 
Fix a subgroup $K \subseteq T^+$ such that $T^+ = K \times (\rad \beta^+)$. 
Let $\pi\from G \to \barr G$ denote the natural homomorphism, set $\barr {T^+} \coloneqq \pi(T^+)$ and $\barr {K} \coloneqq \pi(K)$, let $\barr {\beta^+}\from \barr {T^+} \times \barr {T^+} \to \FF^\times$ be the bicharacter on $\barr {T^+}$ induced by $\beta^+$, and consider $\kappa$ as a map defined on $\barr G/\barr {T^+} \iso G/{T^+}$. 
Also, let $\chi,\chi_0 \from T^+ \to \FF^\times$ be the characters defined by $\chi(K) = 1$, $\chi(f) = -1$ and $\chi(t_p) = 1$, and $\chi_0(K) = 1$, $\chi(f)_0 = 1$ and $\chi_0(t_p) = -1$ (we note that $\chi_0$ was denoted by $\chi$ in \cref{eq:equivalent-def-of-chi}). 

Note that $\barr {\beta^+}\restriction_{\barr K \times \barr K}$ is nondegenerate, and let $\widetilde \D$ be the chosen standard realization associated to $(\barr{K}, \barr{\beta^+}\restriction_{\barr K \times \barr K})$. 
Let $ {\widetilde\mu}\from \barr{K} \to \FF^\times$ be the map associated to the transposition on $\widetilde \D$, and let $\barr{\mu^+}\from \barr{T^+} \to \FF^\times$ be the map extending $\widetilde\mu$ defined by $\overline{\mu^+}(\bar t \bar t_p) = \overline{\mu^+}(\bar t) \coloneqq \widetilde\mu (\bar t)$, for all $\bar t\in \barr K$. 
Set $\mu^+ \coloneqq \barr {\mu^+} \circ \pi\restriction_{T^+}$ and define $\eta^+\from T^+ \to \pmone$ by
\[\label{eq:fix-eta-plus-undouble-Ann}
    \forall t\in T^+, \quad \eta^+(t) \coloneqq \mu^+(t) \chi\inv(t).
\] 
Since $\mathrm{d} \widetilde\mu = \barr{\beta^+}\restriction_{\barr K \times \barr K}$, we get $\mathrm{d} \overline{\mu^+} = \barr{\beta^+}$ and, hence, $\mathrm{d} \eta^+ = \beta^+$.  
In the proof of \cref{prop:Ann-Type-II-correspondence}, we will show that $\eta^+$ is the map associated to the superinvolution on the even part of a graded-division superalgebra as in \cref{def:std-realization-MxM-QxQ}(b). 

Set $k \coloneqq |\kappa|$. 
As we did in \cref{ssec:grds-on-Q(n)}, we follow the construction after \cref{eq:fix-eta-undouble} in \cref{ssec:grds-on-A-m-n} with $\kappa_\bz \coloneqq \kappa$ and $\kappa_\bo$ being the zero map. 
We get an elementary $\barr G$-grading on $M_k(\FF) = M(k , 0)$. 
We identify the $\barr G$-graded superalgebra $M_{k}(\widetilde \D) = M_{k}(\FF) \tensor \widetilde \D$ with $\tilde S = M_{n+1}(\FF)$ via Kronecker product. 

Now, consider the (division) $\barr G$-grading on $M(1,1)$ given by:
%
\[\label{eq:bar-G-grd-on-M(1-1)}
\begin{aligned}
	\deg \left(\begin{array}{c|c}
		 \phantom{.}1\phantom{.} & \phantom{-}0 \\
		\hline
		 \phantom{.}0\phantom{.} & \phantom{-}1 
	\end{array}\right) &= \bar e,\quad & \deg \left(\begin{array}{c|c}
		 \phantom{.}0\phantom{.} & \phantom{-}1 \\
		\hline
		 \phantom{.}1\phantom{.} & \phantom{-}0 
	\end{array}\right) &= \bar h, \\
	\deg \left(\begin{array}{c|c}
		 \phantom{.}1\phantom{.} & \phantom{-}0 \\
		\hline
		 \phantom{.}0\phantom{.} & -1 
	\end{array}\right) &= \bar t_p,\quad &
	\deg \left(\begin{array}{c|c}
		 \phantom{.}0\phantom{.} & -1           \\
		\hline
		 \phantom{.}1\phantom{.} & \phantom{-}0 
	\end{array}\right) &= \bar t_p \bar h.
\end{aligned}
\]
(Compare with \cref{ex:Pauli-2x2,ex:Pauli-2x2-super}.) 
Using the tensor product of graded (super)algebras, we get a $\barr G$-grading on $S = M(n+1, n+1) = M(1,1) \tensor \tilde S$. 

\begin{remark}\label{rmk:change-M(1-1)-of-place}
    Note that $S = M(1,1) \tensor \tilde S = M(1,1) \tensor M_k(\FF) \tensor \widetilde \D \iso M_k(\FF) \tensor M(1,1) \tensor \widetilde\D$. 
    Hence, this $\barr G$-grading is isomorphic to $\Gamma_M(\barr {T^+}, \barr {\beta^+}, \bar t_p, \kappa)$ (see \cref{def:Gamma_M-only-G}), where the fixed character is taken to be $\chi_0$. 
\end{remark}

Let $\xi\from \barr G/\barr{T^+} \to \barr G$ be a set-theoretic section of the natural homomorphism $\barr G \to \barr G/\barr{T^+}$. 
We will construct a super-anti-automorphism on $S$. 
The next definition is similar to \cref{defi:blocks-of-Theta}. 
Note that, since $\chi_0(f) = 1$, we can see it as a character on $\barr{T^+}$. 

\begin{defi}\label{defi:blocks-of-Theta-Ann}
    Let $x \in G/T^+$. 
    If $g_0x^2 = T^+$, we put $\bar t \coloneqq \mathrm{pr}_{\barr K} ( \pi(g_0 \xi(x)^2) )$, where $\mathrm{pr}_{\barr K}\from \barr{T^+} \to \barr K$ is the projection on $\barr K$ corresponding to the decomposition ${T^+} = \barr K \times \langle \barr t_p \rangle$. 
    We define $\Theta_K(x)$ to be the following $\kappa(x) \times \kappa(x)$-matrix with entries in $\widetilde \D$:
    %
    \begin{enumerate}[(i)]
        \item $I_{\kappa(x)} \tensor X_{\bar t}$ if $\eta^+(t) = +1$;
        %
		\item  $J_{\kappa(x)} \tensor X_{\bar t}$, where $J_{\kappa(x)} \coloneqq \begin{pmatrix}
				      0                & I_{\kappa(x)/2} \\
				      -I_{\kappa(x)/2} & 0
			      \end{pmatrix}$, if  $\eta^+(t) = -1$ (recall that, in this case, $\kappa(x)$ is even by \cref{inertia-even-and-odd-case}). 
	\end{enumerate}
    %
    If $g_0 x^2 \neq T^+$, we define $\Theta_K(x)$ to be the following $2\kappa(x) \times 2\kappa(x)$-matrix:
    %
    \begin{enumerate}[(i)]
        %
        \setcounter{enumi}{2}
        %
		\item $\begin{pmatrix}
			0  &  I_{\kappa(x)} \\
			\chi(g_0 \xi(x)^2)\inv I_{\kappa(x)} & 0
		\end{pmatrix} \tensor 1_{\widetilde \D}$. 
    \end{enumerate}
\end{defi}

Then, we define 
\[\label{eq:puting-the-blocks-of-Theta-Ann}
    %
    \sbox0{$\begin{matrix}
        \Theta_K(x_1)&& \\
        & \ddots &\\
        && \Theta_K(x_{\ell})
    \end{matrix}$}
    %
    \sbox1{$\begin{matrix}
        \delta_{x_1}\Theta_K(x_1)&& \\
        & \ddots &\\
        && \delta_{x_\ell}\Theta_K(x_{\ell})
    \end{matrix}$}
    %
    \Theta \coloneqq
    \left(\begin{array}{c|c}
            \usebox{0} & 0\\
            \hline
            0 & \usebox{1}
        \end{array}\right).
\]
%

We define $\Theta \in S$ by 
\[\label{eq:Theta_K-for-Ann}
    \widetilde\Theta \coloneqq \begin{pmatrix}
        \Theta(\bar 0, x_1)&& \\
        & \ddots &\\
        && \Theta(\bar 0, x_{\ell})
    \end{pmatrix},
\]
where $x_1 < \ldots < x_{\ell}$ are the elements of the set $\{ x \in \supp \kappa \mid x \leq g_0\inv x\inv \}$, and $\Theta(\bar 0, x)$, for $x\in \barr G/\barr {T^+}$, is as in \cref{defi:blocks-of-Theta}. 
We, then, define the (super-)anti-automorphism $\tilde\theta\from \tilde S \to \tilde S$ by 
\[\label{eq:theta-with-matrix-4}
    \forall X\in M_{n+1}(\FF), \quad \tilde\theta(X) \coloneqq \widetilde\Theta\inv\, X\transp\, \widetilde\Theta.
\]


Let $\theta\from S \to S$ be the tensor product of the queer supertransposition on $M(1,1)$ and $\tilde \theta$ on $\tilde S$. 
In terms of the Kronecker product, this corresponds to  
\[\label{eq:theta-with-matrix-6}
    \forall X\in M(n+1, n+1), \quad \theta(X) \coloneqq \Theta\inv\, X\sTq\, \Theta,
\]
where 
\[\label{eq:Theta-for-Ann}
    \Theta \coloneqq \left(\begin{array}{c|c}
            \widetilde\Theta & 0\\
            \hline
            0 & \widetilde\Theta
        \end{array}\right).
\]

Finally, we define a $G$-grading on $S^{(1)}$ by 
\[\label{eq:final-G-grd-on-Ann}
    \forall g\in G, \quad S^{(1)}_g \coloneqq \{ s \in S^{(1)}_{\barr g} \mid \theta(s) = - \chi(g) s\},
\]
and consider the $G$-grading on $L$ by reducing it modulo the center. 

In summary:

\begin{defi}\label{defi:type-II-Ann}
    Let $n \in \ZZ_{> 0}$, and denote the associative superalgebra $M(n+1, n+1)$ by $S$. 
    Let $T^+ \subseteq G$ be a $2$-elementary subgroup, let $e\neq t_p \in T^+$, let $h \in G$ be such that $f \coloneqq h^2 \in T^+ \setminus \langle t_p \rangle$, and let $\beta^+\from T^+ \times T^+ \to \pmone$ be an alternating bicharacter such that $\rad \beta^+ = \langle t_p, f \rangle$. 
    Let $\pi\from G\to \barr G \coloneqq G/\langle f \rangle$ be the natural homomorphism, fix a subgroup $K \subseteq T^+$ such that $T^+ = K \times \langle f \rangle$, set $\barr {T^+} \coloneqq \pi(T^+)$ and $\barr K \coloneqq \pi(K)$, and let $\barr {\beta^+}$ be the alternating bicharacter on $\barr {T^+}$ induced by $\beta^+$. 
    Consider the chosen standard realization $\widetilde \D$ of a matrix algebra with division grading associated to $(\barr {K}, \barr{\beta^+}\restriction_{\barr K \times \barr K})$, and define $\eta^+\from T \to \pmone$ by \cref{eq:fix-eta-plus-undouble-Ann}. 
    Then, let $g_0 \in G$ be any element and let $\kappa\from G/{T^+} \to \ZZ_{\geq 0}$ be a $g_0$-admissible map (\cref{defi:odd-D-kappa-g_0-admissible}) such that $n+1 = |\kappa| \sqrt{|T^+|}/2$. 
    Choose:
    \begin{enumerate}[(i)]
        \item a set-theoretic section $\xi\from G/T^+ \to G$ for the natural homomorphism $G \to G/T^+$;
        \label{item:choice-xi-Ann}
        %
        \item a total order $\leq$ on $G/T^+$ such that there are no elements between $x$ and $\bar g_0\inv x\inv$, for all $x\in G/T^+$; 
        \label{item:choice-leq-Ann}
    \end{enumerate}
    and construct a tuple $\bar\gamma$ realizing $\kappa$ according to $\pi \circ \xi$ and $\leq$ (\cref{defi:tuple-governed}). 
    Consider the $\barr G$-grading $\Gamma_M(\barr {K}, \barr {\beta^+}\restriction_{\barr K \times \barr K}, \kappa)$ on $M_{n+1}(\FF)$ constructed using the choices of $\widetilde \D$ and $\barr \gamma$ above (see \cref{def:Gamma-T-beta-kappa}),
    and the $\barr G$-grading on $M(1,1)$ given by \cref{eq:bar-G-grd-on-M(1-1)}.  
    We, then, identify $S \coloneqq M(n+1, n+1)$ with the graded superalgebra $M(1,1) \tensor M_{n+1}(\FF)$. 
    Define $\widetilde\Theta \in M_{n+1}(\FF)$ by \cref{eq:Theta_K-for-Ann}, $\Theta \in S$ by \cref{eq:Theta-for-Ann} and ${\theta\from S \to S}$ by
    \cref{eq:theta-with-matrix-6}
    Finally, we define $\Gamma_A^{\mathrm{(II_Q)}}(T^+, \beta^+, t_p, h, \kappa, g_0)$ to be the $G$-grading on $L = S^{(1)}/Z(S^{(1)})$ induced from the $G$-grading $S^{(1)}$ given by \cref{eq:final-G-grd-on-Ann}.
\end{defi}

\begin{prop}\label{prop:Ann-Type-II-correspondence}
    Consider $(R, \vphi) \coloneqq M^{\mathrm{ex}}(T^+, \beta^+, t_p, h, \kappa, g_0)$, as defined before \cref{cor:MxMsop-odd-only-G}. 
    Then the graded Lie superalgebra $\Skew (R,\vphi)^{(1)}/Z(\Skew (R,\vphi)^{(1)})$ is isomorphic to the Lie superalgebra $A(n,n)$ endowed with $\Gamma_A^{\mathrm{(II_Q)}}(T^+, \beta^+, t_p, h, \kappa, g_0)$. 
\end{prop}

THE PROOF IS STILL UNCHANGED

\begin{proof}
    % Recall \chi_0
    Let $\chi_0 \in \widehat{T^+}$ be the character defined by \cref{eq:equivalent-def-of-chi}. 
    % Recall what M^ex means, including what $T$ and $t_p$ are, and how to see \kappa with the proper domain. 
    Recall that $M^{\mathrm{ex}}(T^+, \beta^+, t_p, h, \kappa, g_0) = M^{\mathrm{ex}} (T, \beta, t_p, \kappa, g_0)$ (see \cref{def:model-grd-MxM-odd-or-QxQ}), where $t_1 \coloneqq (h, \bar 1)$, $T \coloneqq T^+ \cup t_1 T^+$, $\beta\from T\times T \to \FF^\times$ is the unique alternating bicharacter extending $\beta^+$ such that $\rad \beta = \langle f \rangle$ and $\beta(t_1, \cdot) = \chi_0$, and $\kappa$ is seen as defined on $G^\#/T \iso G/ T^+$. 
    
    % Say what we are going to do
    We will now show how the choices in \cref{defi:type-II-Ann}  correspond to the choices in \cref{def:std-realization-MxM-QxQ}(b) and \cref{def:model-grd-MxM-odd-or-QxQ}. 
    % (and, hence, also in \cref{def:std-realization-MxM-QxQ}(c)).
    
    % The choices of K and \barr\D\even given us the correct \eta^+, and fix \D as \M\tensor\C
    By \cref{lemma:barr-D-to-mc-M}, with $(T^+, \beta^+)$ playing the role of $(T, \beta)$, the choices of $K$ and $\tilde \D$  give us the same information as the choices of $K$ in item \eqref{item:K-can-be-orthogonal-to-t_1} and $\mc M$ in item \eqref{item:choose-mc-M} of \cref{def:std-realization-MxM-QxQ}(b), and the map associated to the transposition on $\mc M$ is $\mu^+\restriction_K$. 
    Let $(\D, \vphi_0)$ denote the standard realization of a graded-division superalgebra with superinvolution constructed using these choices. 
    Note that the map $\eta^+\from T\to \pmone$ defined in \cref{eq:fix-eta-plus-undouble-Ann} is such that $\mathrm{d}\eta^+ = \beta^+$, $\eta^+\restriction_{K} = \mu^+\restriction_K$, $\eta^+(f) = -1$ and $\eta^+(t_p) = 1$, so, by \cref{eq:eta+-unsharpening-MxM}, $\eta^+$ is the map associated to $\vphi_0\restriction_{\D\even}$. 
    In particular, the $g_0$-admissibility condition for $\kappa$ is the same in \cref{def:model-grd-MxM-odd-or-QxQ,defi:type-II-Ann}. 
    
    % Constructs $(\U, B)$
    In \cref{def:model-grd-MxM-odd-or-QxQ}, we have to choose a graded right $\D$-supermodule $\U$ and a $\vphi_0$-sesquilinear form $B\from \U \times \U \to \D$ such that $(\U, B)$ has inertia determined by $\kappa$. 
    For that, we can follow the same construction as in the proof of \cref{prop:Q-Type-II-correspondence}. 
    
    % Undoubling what we just constructed
    We will now follow \cref{ssec:undoubling} to undouble $M^{\mathrm{ex}}(T^+, \beta^+, t_p, h, \kappa, g_0)$, constructed with the choices above. 
    % Recall \D
    Recall that, by \cref{def:std-realization-MxM-QxQ}(b), $\D = {}^\alpha \mc O \tensor \mc M$, where $\mc O$ is the graded-division superalgebra of \cref{ex:superalgebra-O}, and $\alpha\from \ZZ_2 \times \ZZ_4 \to \langle t_p, t_1 \rangle$ is the group isomorphism given by $\alpha (\bar 1, \bar 0) \coloneqq t_p$ and $\alpha (\bar 0, \bar 1) \coloneqq t_1$. 
    % $\D$ is what it should be 
    Let $\epsilon \coloneqq \frac{1+\zeta}{2}\in \D$ be a central idempotent, where $\deg \zeta = f$, and consider the $\barr G$-graded-division superalgebra $\D\epsilon$. 
    Clearly, $\epsilon \in \mc O$.  
    It is straightforward that $\mc O \epsilon$ is isomorphic to $M(1,1)$ with grading given by \cref{eq:bar-G-grd-on-M(1-1)}, and that $\mc M \epsilon$ is isomorphic to $\widetilde \D$. 
    It follows that $\D\epsilon \iso \barr \D \coloneqq M(1,1) \tensor \widetilde \D$. 
    We will identify $\D\epsilon$ with $\barr \D$ and consider $\barr \D$ as a matrix superalgebra via Kronecker product. 
    
    % Undoubling $\eta$
    We need to extend $\chi$ to $G^\#$. 
    Since $\chi(f) = -1$, we can do it in a way such that $\chi(t_1) = \bi$. 
    Let $\eta\from T \to \pmone$ be the map associated to $\vphi_0$ and let $\barr\mu\from \barr T \to \FF^\times$ be as defined in \cref{eq:defi-mu-undoubled}. 
    % We claim that  is the map associated to the queer supertranspose on $\barr \D$.  
    We have already showed that $\eta\restriction_{T^+} = \eta^+$ and, hence, by \cref{eq:fix-eta-plus-undouble-Ann}, we have that $\barr\mu\restriction_{\barr{T^+}} = \barr{\mu^+}$.
    By \cref{rmk:eta-for-MxM-or-QxQ}(b), we have $\barr\mu (\barr t_1) = \eta(t_1) \chi(t_1) = \bi$. 
    It follows that $\barr\mu$ is the map associated to the queer supertranspose on $\barr \D$. 
    
    % Undoubling $\Theta$
    As in the proof of \cref{prop:m-not-n-Type-II-correspondence}, let $\Lambda \in M_k(\D)$ be the diagonal matrix with entries $\Lambda_{ii} \coloneqq \chi(\deg u_i)$, consider a different graded $\D$-basis $\tilde \B = \{ \tilde u_1, \ldots, \tilde u_k \}$ of $\U$, where $\tilde u_i$ is defined as in \cref{eq:tilde-u_i-from-u_i}, and let $\tilde \Phi \in M_k(\D)$ be the matrix representing $B$ with respect to $\tilde \B$. 
    Clearly, all the entries of $\tilde\Phi$ are in $\barr \D\even$. 
    We will denote $\tilde\Phi$ by $\tilde\Phi_\bz$ when seen as a matrix in $M_k(\D\even)$. 
    In \cref{ssec:undoubling}, $M_k(\D)\epsilon$ was identified with $M_k(\barr \D)$, so we now identify $M_k(\D\even)\epsilon$ with $M_k(\barr \D\even)$. 
    % In the definition of $\Gamma_Q^{\mathrm{(II)}}(T^+, \beta^+, h, \kappa, g_0)$, we identified $M_k(\barr \D\even) = M_k(\FF) \tensor \barr \D\even$ with $S\even = M_{n+1}(\FF)$ via Kronecker product. 
    Under this identification, the matrix $\Lambda \tilde\Phi_\bz \epsilon$ becomes $\Theta_\bz$ as in \cref{eq:Theta_bz-for-Q}. 
    Also, following the isomorphisms
    \[
        M_k(\barr \D) \iso M_k(\FF) \tensor \barr \D \iso M_k(\FF) \tensor Q(1) \tensor \barr \D\even \iso Q(1) \tensor M_k(\FF) \tensor \barr \D\even \iso Q(1) \tensor S\even,
    \]
    the $\barr G$-grading on $M_k(\barr \D)$ becomes the $\barr G$-grading we defined on $S = Q(1) \tensor S\even$, and $\Lambda\tilde\Phi\epsilon$ is sent to $I_2 \tensor \Theta_\bz = \Theta$. 
    Therefore, all the data in the description of the undoubled model of $\Skew(R, \vphi)$ coincide with with the data in \cref{defi:type-II-Q}, concluding the proof.
\end{proof}

% --------------------


\begin{thm}
    Let $n \geq 2$ and 
    let $L$ denote the simple Lie superalgebra $Q(n)$. 
    Every grading on $L$ is isomorphic to either $\Gamma_Q^{\mathrm{(I)}}(T^+, \beta^+, h, \kappa)$ or $\Gamma_Q^{\mathrm{(II)}}(T^+, \beta^+, h, \kappa, g_0)$ as in \cref{def:Q-Type-I,defi:type-II-Q}. 
    Gradings belonging to different types are not isomorphic. 
    Within each type, we have:
    
    \boxed{\mathrm{Type \,\,I}_M}
    
        See \cref{thm:MxM-type-I}.
        
        Moreover, $M(T, \beta, \kappa_\bz, \kappa_\bo) \times M(T, \beta, \kappa_\bz, \kappa_\bo)\sop \iso M(T', \beta', \kappa_\bz', \kappa_\bo') \times M(T', \beta', \kappa_\bz', \kappa_\bo')\sop$ if, and only if, $T = T'$ and one of the following conditions holds:
	\begin{enumerate}[(i)]
	    \item $\beta'=\beta$ and there is $g\in G$ such that such that either $g \cdot \kappa_{\bar 0}'=\kappa_{\bar 0}$ and $g \cdot \kappa_{\bar 1}'=\kappa_{\bar 1}$, or $g \cdot \kappa_{\bar 0}'=\kappa_{\bar 1}$ and $g \cdot \kappa_{\bar 1}'=\kappa_{\bar 0}$; 
	    \item $\beta'=\beta\inv$ and there is $g\in G$ such that either $g \cdot \kappa_{\bar 0}'=\kappa_{\bar 0}\Star$ and $g \cdot \kappa_{\bar 1}'=\kappa_{\bar 1}\Star$, or $g \cdot \kappa_{\bar 0}'=\kappa_{\bar 1}\Star$ and $g \cdot \kappa_{\bar 1}'=\kappa_{\bar 0}\Star$;
	\end{enumerate}
        
    \boxed{\mathrm{Type \,\,I}_Q}
    
        See \cref{thm:MxM-type-I}.
    
        $M (T, \beta, p, \kappa) \times M (T, \beta, p, \kappa)\sop \iso M (T', \beta', p', \kappa') \times M (T', \beta', p', \kappa')\sop$ if, and only if, $T = T'$, $p=p'$ and one of the following conditions holds:
    \begin{enumerate}[(i)]
        \setcounter{enumi}{2}
	    \item $\beta'=\beta$ and there is $g\in G$ such that $\kappa' = g \cdot \kappa$;
	    \item $\beta'=\beta\inv$ and there is $g\in G$ such that $\kappa' = g \cdot \kappa\Star$. 
	\end{enumerate}
        
    \boxed{\mathrm{Type \,\,II}_{osp}}
    
        See \cref{thm:MxM-even}.
    
           Moreover, $M^{\mathrm{ex}} (T, \beta, \kappa_\bz, \kappa_\bo, g_0) \iso M^{\mathrm{ex}} (T', \beta', \kappa_\bz', \kappa_\bo', g_0')$ if, and only if, $T =T'$, $\beta = \beta'$ and there is $g \in G$ such that one of the following conditions holds:
    \begin{enumerate}[(i)]
        \item $\kappa_\bz' = g\cdot\kappa_\bz$, $\kappa_\bo' = g\cdot\kappa_\bo$ and $g_0' = g^{-2}g_0$;
        \item $\kappa_\bz' = g\cdot\kappa_\bo$, $\kappa_\bz' = g\cdot\kappa_\bo$ and $g_0' = fg^{-2}g_0$. 
    \end{enumerate}
        
    \boxed{\mathrm{Type \,\,II}_P}
    
        See \cref{thm:MxM-even}.
    
        Moreover, $M^{\mathrm{ex}} (T, \beta, \kappa_\bz, \kappa_\bo, g_0) \iso M^{\mathrm{ex}} (T', \beta', \kappa_\bz', \kappa_\bo', g_0')$ if, and only if, $T =T'$, $\beta = \beta'$ and there is $g \in G$ such that one of the following conditions holds:
    \begin{enumerate}[(i)]
        \item $\kappa_\bz' = g\cdot\kappa_\bz$, $\kappa_\bo' = g\cdot\kappa_\bo$ and $g_0' = g^{-2}g_0$;
        \item $\kappa_\bz' = g\cdot\kappa_\bo$, $\kappa_\bz' = g\cdot\kappa_\bo$ and $g_0' = fg^{-2}g_0$. 
    \end{enumerate}
        
    \boxed{\mathrm{Type \,\,II}_Q}
    
            See \cref{cor:MxMsop-odd-only-G}.
    
            Moreover, $M^{\mathrm{ex}}(T^+, \beta^+, t_p, h, \kappa, g_0)$ and $M^{\mathrm{ex}} (T'^+, \beta'^+, t_p',  h', \kappa', g_0')$ are isomorphic if, and only if, $T^+ =T'^+$, $\beta^+ = \beta'^+$, $t_p = t_p'$, $h' \in h (\rad \beta^+)$ and there is $g \in G$ such that $\kappa' = g\cdot\kappa$ and
    \begin{enumerate}[(i)]
        \item $g_0' = g^{-2}g_0$ if $h' \in \{ h, f t_p h\}$;
        \item $g_0' = t_p g^{-2}g_0$ if $h' \in \{ f h, t_p h\}$.
    \end{enumerate}
\end{thm}

\begin{proof}
    something
\end{proof}

% As a final remark, we will see what all these different types of gradings mean in therms of the $\widehat{G^\#}$-action. 
% (Another subsection? we need a variable $\chi$ and it does not depend on the undoubled model, it is probably easier in the doubled one.)

