
\section{Basic concepts}\label{sec:graded-bimodules}
% ------------------

In this section, we will expand on concepts discussed in the \hyperref[chap:intro]{Introduction} and present some basic constructions. 

\phantomsection\label{support}
Let $V$ be a vector space. 
The \emph{support of a $G$-grading} $\Gamma: V = \bigoplus_{g\in G} V_g$ is the subset of $G$ given by
\[
    \supp \Gamma \coloneqq \{ g \in G \mid V_g \neq 0 \}.
\]
If $\Gamma$ is fixed, then we may refer to $\supp \Gamma$ as the \emph{support of $V$} and denote it by $\supp V$. 

\begin{defi}\label{defi:elementary-grd-abstract}
    Let $V = \bigoplus_{g\in G} V_g$ and $W = \bigoplus_{g\in G} W_g$ be $G$-graded vector spaces. 
    A linear map $T \from V \to W$ is said to be \emph{homogeneous of degree $g$} if 
    \[
    \forall h \in G, \quad T(V_h) \subseteq W_{gh} .
    \]
    The subspace of $\Hom(V,W)$ consisting of all linear maps of degree $g$ is denoted by $\Hom (V, W)_g$, and we define the graded vector space $\Hom^{\textrm{gr}} (V, W)$ by 
    \[
        \Hom^{\textrm{gr}} (V, W) \coloneqq \bigoplus_{g \in G} \Hom (V, W)_g. 
    \]
    % The grading on $\Hom^{\textrm{gr}} (V, W)$ is called the \emph{elementary grading induced by $V$ and $W$}. 
    In the case $V=W$, we denote $\Hom (V, W)_g$ by $\End (V)_g$ and $\Hom^{\textrm{gr}} (V, W)$ by $\End^{\textrm{gr}} (V)$. 
\end{defi}

If $V$ is finite dimensional, it is easy to see that $\Hom^{\textrm{gr}} (V, W) = \Hom (V, W)$. 
Also, if $U$ is another graded vector space and $S\from U \to V$ and $T\from V \to W$ are homogeneous linear maps of degrees $h$ and $g$, respectively, then $T \circ S$ is a homogeneous map of degree $gh$. 
In particular, $\End^{\textrm{gr}} (V)$ is a $G$-graded algebra (compare with \cref{eq:elementary-1st}). 

We emphasize that by a homomorphism of graded vector spaces we mean a degree preserving linear map, \ie, an element of $Hom (V, W)_e$. 

The following is an easy result that will be used in \cref{chap:grds-sinv-simple}:

\begin{lemma}\label{lemma:eigenvector-homogeneous}
	Let $V$ be a $G$-graded vector space. 
	If $T \from V \to V$ is a degree preserving map, then its eigenspaces are graded subspaces of $V$.
\end{lemma}

\begin{proof}
	Let $v\in V$ be an eigenvector with eigenvalue $\lambda \in \FF$ and write $v = \sum_{g\in G} v_g$, where $v_g \in V_g$. 
	On the one hand, $T(v) = \lambda v = \sum_{g\in G} \lambda v$. 
	On the other hand, $T(v) = \sum_{g\in G} T(v_g)$. 
	Since the sum of $V_g$, $g \in G$, is direct, we must have that $T(v_g) = \lambda v_g$ for all $g\in G$, and the result follows. 
\end{proof}

As we saw above, if $V$ is a finite dimensional vector space, then any $G$-grading on $V$ gives rise to a grading on the associative algebra $\End(V)$; gradings of this form are called \emph{elementary}.
An elementary grading can be described in matrix terms. 
Given a $n$-tuple $\gamma = (g_1, \ldots, g_n)$ of elements in $G$, we can define a $G$-grading on $\FF^n$ by setting $\deg e_i = g_i$, where $\{ e_1, \ldots, e_n\}$ is the canonical basis of $\FF^n$. 
Clearly, any finite dimensional $G$-graded vector space is isomorphic to $\FF^n$ endowed with such a grading. 
The grading on $M_n (\FF)$ induced from $\FF^n$ is the following: 

\begin{defi}\label{defi:elementary-grd-matrix}
    The \emph{elementary grading defined by an $n$-tuple $\gamma = (g_1, \ldots, g_n) \in G^n$} on $M_{n}(\FF)$ is the $G$-grading determined by 
    \[
        E_{ij} = g_i g_j\inv, \quad 1 \leq i, j \leq n,
    \]
    where $E_{ij}$ denotes the matrix with $1$ in the $(i,j)$-entry and $0$ in every other entry. 
\end{defi}

Recall that the superspace $V = \FF^{m|n}$ is defined by setting $V\even = \FF^m$ and $V\odd = \FF^n$ (see \cref{def:grd-superspace-canonical}). 
Given an $m$-tuple $\gamma_\bz$ and an $n$-tuple $\gamma_\bo$ of elements in $G$, we can consider the gradings defined as above on $\FF^m$ and $\FF^n$, and, hence, a grading on $\FF^{m|n}$. 
The corresponding grading on $M(m,n)$ is the following:

\begin{defi}\label{defi:elementary-grd-super}
    Let $\gamma_\bz$ be an $m$-tuple and $\gamma_\bo$ be an $n$-tuple of elements of $G$. 
    The \emph{elementary grading defined by $\gamma_\bz$ and $\gamma_\bo$} on the superalgebra $M(m,n)$ is the elementary grading on its underlying algebra $M_{m+n}(\FF)$ defined by the concatenation of $\gamma_\bz$ and $\gamma_\bo$.  
\end{defi}

It is useful to consider gradings not only on vector (super)spaces and (super)algebras, but also on modules:

\begin{defi}\label{defi:grdModule}
    Let $R = \bigoplus_{g\in G} R_g$ be a graded associative algebra and let $V = \bigoplus_{g\in G} V_g$ be a graded vector space. 
    If $V$ has a structure of left $R$-module, we say that $V$ is a \emph{graded left module} if 
    \[
        \forall g,h \in G, \quad R_g \cdot V_h \subseteq V_{gh},
    \]
    \ie, if the image of the representation $\rho\from R \to \End(V)$ is in $\End^{\textrm{gr}}(V)$ and $\rho$ is degree-preserving. 
    One can define graded right modules and graded bimodules analogously. 
\end{defi}

In the case $R$ is a superalgebra and $V$ is a superspace with their canonical $\ZZ_2$-gradings, we say that $V$ is a \emph{left $R$-supermodule}. 
A \emph{$G$-graded left $R$-supermodule} is a $G^\#$-graded $R$-module (recall the definition of $G^\#$ in \cref{defi:G-sharp}).

\begin{ex}
    Any graded vector space $V$ is a graded left $\End^{\textrm{gr}}(V)$-module, with the natural action. 
    Also $\Hom^{\textrm{gr}}(V, W)$ is a graded $(\End^{\textrm{gr}}(W),\End^{\textrm{gr}}(V))$-bimodule, with the action given by map composition. 
\end{ex}

% \begin{ex}
%     Given a graded Lie superalgebra $L = L\even \oplus L\odd$, we can see $L\odd$ as a module over the graded Lie algebra $L\even$. 
% \end{ex}

%  --- shift


\begin{defi}
    Let $\Gamma: V = \bigoplus_{h\in G} V_h$ be a grading on a vector space $V$. 
    For an element $g\in G$, the \emph{right shift of $\Gamma$ by $g$}, denoted by $\Gamma^{[g]}$, is the grading obtained by replacing every index $h \in G$ by $hg$, \ie, $\Gamma^{[g]} : V = \bigoplus_{h\in G} V_h'$ where $V_{g}' \coloneqq V_{hg\inv}$, for all $h\in G$. 
    Analogously, the \emph{left shift of $\Gamma$ by $g$} is defined to be ${}^{[g]}\Gamma : V = \bigoplus_{h\in G} V_h''$ where $V_{h}'' \coloneqq V_{hg\inv}$, for all $h\in G$. 
    If $\Gamma$ is fixed, we define the \emph{right (respectively, left) shift of $V$} to be $V$ endowed with $\Gamma^{[g]}$ (respectively, ${}^{[g]}\Gamma$) and denote it by $V^{[g]}$ (respectively, ${}^{[g]}V$).
\end{defi}

If $V$ is a graded left module over a graded algebra $R$, then $V^{[g]}$ is a graded left $R$-module. 
Furthermore, ${}^{[g]} R ^{[g\inv]}$ is a graded algebra and ${}^{[g]}V$ is a graded left module over ${}^{[g]} R ^{[g\inv]}$. 
Of course, similar statements hold for graded right modules. 

% THE FOLLOWING IS NOT USED, REMOVE?

% We will make use of the following partial converse (see e.g. \cite[Proposition 3.5]{paper-Qn}):

% \begin{lemma}\label{lemma:simplebimodule}
% 	Let $R$ be a $G$-graded algebra and let $V$ be a finite-dimensional (ungraded) simple $R$-module. 
% 	If $\Gamma$ and $\Gamma'$ are  two $G$-gradings that make $V$ a graded module, then there is $g\in G$ such that $\Gamma' = \Gamma^{[g]}$. \qed
% \end{lemma}

% ---- Tensor

We are now going to define tensor product of graded spaces:

% Def: Grading on tensor product
\begin{defi}\label{defi:tensorProduct}
    Let $V=\bigoplus_{g\in G} V_g$ and $W=\bigoplus_{g\in G} W_g$ be $G$-graded vector spaces. 
    The \emph{tensor product} of $V$ and $W$ is the vector space $V \tensor W$ endowed with the $G$-grading $\Gamma : V \tensor W = \bigoplus_{g\in G} (V \tensor W)_g$ where
    \[
        \forall g\in G, \quad (V \otimes W)_g = \bigoplus_{h\in G} V_{h} \otimes W_{h\inv g}.
    \]
\end{defi}

\phantomsection\label{defi:tensor-algebras}

If $G$ is an abelian group and $R$ and $S$ are $G$-graded algebras, then it follows that $R \tensor S$ is a graded algebra with the usual multiplication: $(r \tensor s) (r' \tensor s') = rr' \tensor ss'$. 
Note that, in this case, the notion of graded bimodule can be reduced to the notion of left module, as in the ungraded case. 

% The external tensor product of is always a graded algebra. 

For superalgebras, though, following the rule of signs mentioned in \cref{sec:Grassmann}, one often considers a different multiplication on the tensor product:

\begin{defi}\label{defi:tensorSuperalgebras}
    Let $R = R\even \oplus R\odd$ and $S = S\even \oplus S\odd$ be superalgebras. 
    We define the superalgebra $R \tensor S$ to be the $\ZZ_2$-graded tensor product $R \tensor S$ endowed with the multiplication determined by
    \[
        \forall r, r' \in R\even \cup R\odd, \, s,s' \in S\even \cup S\odd, \quad 
        (r \tensor s) (r' \tensor s')
        = \sign{s}{r'} \, rr' \tensor ss'.
    \]
\end{defi}

If one of the superalgebras has trivial canonical $\ZZ_2$-grading, \ie, if $R = R\even$ or $S = S\even$, then the tensor product of superalgebras coincides with the tensor product of algebras. 
We will encounter this situation in \cref{chap:grd-simple-assc} (see \cref{rmk:M(D)=M(FF)-tensor-D}). 

% -- Center:

% We finish this section with an important concept for the associative algebra:

\begin{defi}\label{defi:center}
	Let $R$ be an associative superalgebra.
	The \emph{center} of $R$ is the set
	\[
		Z(R) = \{c\in R \mid cr = rc \text{ for all } r\in R \},
	\]
	\ie, the center of $R$ seen as an algebra, and the \emph{supercenter} of $R$ is the set $sZ(R) \coloneqq sZ(R)\even \oplus sZ(R)\odd$, where
	\begin{align*}
		sZ(R)^i = \{c\in R^i \mid cr = (-1)^{i |r|} rc \text{ for all } r\in R\even\cup R\odd \},\quad i \in \ZZ_2.
	\end{align*}
\end{defi}

% \begin{lemma}\label{lemma:center-is-graded}
% 	Suppose $G$ be an abelian group and let $R$ be a $G$-graded algebra.
% 	Then the center $Z(R)$ is $G$-graded subalgebra of $R$.
% \end{lemma}

% \begin{proof}
% 	Let $c \in Z(R)$ and write $c = \sum_{g \in G} c_g$, where $c_g \in R_g$ for all $g \in G$.
% 	For every homogeneous $r \in R$, we have that
% 	%
% 	\begin{align*}
% 		\big(\sum_{g\in G} c_g\big)r = r \big(\sum_{g\in G} c_g\big).
% 	\end{align*}
% 	%
% 	Comparing the components of degree $gh = hg$, where $h = \deg r$, we conclude that $rc_g = c_g r$ for all $g \in G$.
% 	By linearity, $r c_g = c_g r$ for all $r\in R$, hence $c_g \in Z(R)$.
% \end{proof}

% Note that for $G=\ZZ_2$, Lemma \ref{lemma:center-is-graded} implies that the center of a superalgebra is a subsuperalgebra. 

% The proof of \cref{lemma:center-is-graded} can be easily adapted to show that, if $G$ is abelian, then the supercenter of a graded superalgebra is a graded subsuperalgebra. 

\begin{lemma}\label{lemma:center-is-graded}
	Let $G$ be an abelian group and let $R$ be a $G$-graded superalgebra.
	Then the center $Z(R)$ and the supercenter $sZ(R)$ are $G$-graded subsuperalgebras of $R$.
\end{lemma}

\begin{proof}
	We consider $R$ as a $G^\#$-graded algebra.
	Let $c \in Z(R)$ and write $c = \sum_{g \in G^\#} c_g$, where $c_g \in R_g$ for all $g \in G^\#$.
	For every homogeneous $r \in R$, we have
	%
	\begin{align*}
		\big(\sum_{g\in G^\#} c_g\big)r = r \big(\sum_{g\in G^\#} c_g\big).
	\end{align*}
	%
	Comparing the components of degree $gh = hg$, where $h = \deg r$, we conclude that $rc_g = c_g r$ for all $g \in G^\#$.
	By linearity, $r c_g = c_g r$ for all $r\in R$, hence $c_g \in Z(R)$.

	The same argument works to show that $sZ(R)\even$ is graded and, with straightforward modifications, to show that $sZ(R)\odd$ is graded.
\end{proof}

Note that, in particular (for trivial $G$), the center of a superalgebra is a subsuperalgebra. 

% -----