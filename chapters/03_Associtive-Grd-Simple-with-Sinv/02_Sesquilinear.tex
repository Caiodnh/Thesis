\section[Super-anti-automorphisms and sesquilinear forms]{Super-anti-automorphisms and \texorpdfstring{\\}{} sesquilinear forms}\label{sec:super-anti-auto-and-sesquilinear}

Let $\D$ be a graded-division superalgebra and let $\U$ be a nonzero graded right $\D$-supermodule of finite rank. 
Consider the graded superalgebra $R := \End_\D(\mc U)$.
Then $\mc U$ is an $(R, \D)$-superbimodule.
By \cref{lemma:converse-density-thm}, 
% $\mc U$ is a simple graded left supermodule over $R$ and 
we have a natural identification between $\D$ and $\End_R(\mc U)$.

% \begin{notation}
%     Let  be . As sets, we have that $S = S\sop$, but given $s\in S$, we will it by $ \bar s$ when regarded as an element of $S\sop$.
% \end{notation}

As we saw on \cref{ssec:superdual}, $\mc U\Star = \Hom_\D (\mc U, \D)$ is a graded right $\D\sop$-supermodule, through $f \bar d = \sign{f}{d} df$.
Also, we can make $\mc U\Star$ a graded right $R$-supermodule by defining $(f r) (u) := f(r u)$ for all $f\in \mc U\Star$, $r\in R$ and $u\in \mc U$ (i.e., $f r = f \circ r$, since $R = \End_\D(\mc U)$).
Hence, we can consider $\mc U\Star$ as a graded left $R\sop$-supermodule via $\bar r f = (-1)^{|r||f|} f r$, % (i.e., $r\cdot f = r\Star (f)$)
and $\mc U\Star$ becomes a $(R\sop, \D\sop)$-superbimodule.

\begin{lemma}\label{lemma:U-star-R-sop}
	The left $R\sop$-supermodule $\mc U\Star$ is graded-simple, and the action of $\D\sop$ gives an isomorphism $\D\sop \to \End_{R\sop}(\mc U\Star)$.
\end{lemma}

\begin{proof}
	By definition of superadjoint operator, the above $R\sop$-action corresponds to the representation $\rho\from R\sop \to \End_{\D\sop} (\mc U\Star)$ given by $ \rho(r) = r\Star$, which is an isomorphism since $\U$ is of finite rank.
	If we use $\rho$ to identify $R\sop$ with $\End_{\D\sop} (\mc U\Star)$, the result follows from \cref{lemma:converse-density-thm}.
\end{proof}

%Note that the $R\sop$-action correspond to the representation $\rho: R\sop \to \End_{\D\sop} (\mc U\Star)$ given by $ \rho(r) = r\Star$, which is an isomorphism. If we use it to identify $R\sop$ with $\End_{\D\sop} (\mc U\Star)$, we can then apply Proposition \ref{lemma:converse-density-thm} again and conclude that $\mc U\Star$ is a graded simple left $R\sop$-module and that we naturally can identify $\D\sop$ with $\End_{R\sop}(\mc U\Star)$.

Now let $\vphi: R\to R$ be a super-anti-automorphism that preserves the $G$-degree (\ie, is homogeneous of degree $e$ with respect to the $G$-grading).
We can see it as an isomorphism $R\to R\sop$, $r \mapsto \overline{ \vphi(r)}$, and use it to identify $R$ with $R\sop$.
In particular, we will make the left $R\sop$-action on $\U\Star$ into a left $R$-action via $r\cdot f = \overline{\vphi(r)}f$, for all $r\in R$ and all $f\in \U\Star$.
In other words,
%
\begin{equation}\label{eq:R-action-back-on-the-right}
	r\cdot f = \sign{r}{f} f \circ \vphi(r) .
\end{equation}

We will now consider $\U\Star$ as an $(R,\D\sop)$-superbimodule.
In particular, the superalgebra $\End_R(\U\Star)$ should be understood as the set of $R$-linear maps with respect to the $R$-action on the left (which is the same set as $\End_{R\sop} (\U\Star)$).
Also, since the action is on the left, we will follow the convention of writing $R$-linear maps on the right.

From Lemma \ref{lemma:U-star-R-sop}, it follows that $\U\Star$ is a simple graded left $R$-supermodule and that we can identify $\D\sop$ with $\End_R (\U\Star) = \End_{R\sop} (\U\Star)$.
But from \cite[Lemma 2.7]{livromicha}, $R$ has only one graded-simple supermodule up to isomorphism and shift, hence there is an invertible $R$-linear map $\vphi_1: \mc U \to \mc U\Star$ which is homogeneous of some degree $(g_0, \alpha)\in G^\#$.
Fix one such $\vphi_1$.
% This map is not unique, but we are going to fix one for now. 

\begin{lemma}\label{lemma:nonuniqueness-of-vphi1}
	A map $\vphi_1' : \U \to \U\Star$ is $R$-linear and homogeneous with respect to the $G^\#$-grading if, and only if, there is an element $\bar d\in \D\sop = \End_R (\U\Star)$ homogeneous with respect to the $G^\#$-grading such that $ \vphi_1' = \vphi_1 \bar d$, where juxtaposition represents composition of maps written on the right.
\end{lemma}

\begin{proof}
	This follows from the fact that $ \vphi_1\inv \vphi' \in  \D\sop = \End_R(\U\Star)$ if, and only if, $\vphi' \in \Hom_R(\U, \U\Star)$.
\end{proof}

% \begin{proof}
%     Let $\vphi_1'$ be an homogeneous $R$-linear map. Clearly, $\vphi_1\inv \vphi_1' \in \End_R(\U\Star)$ $ = \D\sop$ (note that we are composing functions written on the right). Then $\vphi_1' = \vphi_1 d$, for some homogeneous $d\in \D\sop$, and hence, by the definitions of the right $\D\sop$-action and the left $\D$-action, $\vphi' (u) = \sign{\vphi_1}{d} d\vphi(u)$ for all $u\in \U$. The converse is a direct computation.
% \end{proof}

We will use
the $R$-linear map
$\vphi_1$ to construct a nondegenerate sesquilinear form $B$ on $\U$.

\begin{defi}\label{def:sesquilinear-form}
	We say that a map $B\colon \U \times \U \to \D$ is a \emph{sesquilinear form on $\U$} if it is $\FF$-bilinear, $G^\#$-homogeneous if considered as a linear map $\U\tensor \U \to \D$, and there is $\vphi_0\from \D \to \D$ a degree-preserving su\-per\--an\-ti\--auto\-mor\-phism such that, for all $u,v \in \U$ and $d\in \D$,
	%
	\begin{enumerate}[(i)]
		\item $B(u,vd) = B(u,v)d$; \label{enum:linear-on-the-second}
		\item $B(ud, v) = (-1)^ {(|B| + |u|)|d|}\vphi_0(d) B(u, v)$. \label{enum:vphi0-linear-on-the-first}
	\end{enumerate}
	If we want to specify the super-anti-automorphism $\vphi_0$, we will say that $B$ is \emph{sesquilinear with respect to $\vphi_0$} or that $B$ is \emph{$\vphi_0$-sesquilinear}.
	The \emph{(left) radical} of $B$ is the set $\rad B \coloneqq \{u\in \U \mid B(u, v) = 0 \text{ for all } v\in \U\}$. 
	We say that the form $B$ is \emph{nondegenerate} if $\rad B = 0$.
\end{defi}

% \begin{remark}\label{rmk:B-determines-vphi_0}
%     Note that if $B$ is nondegenerate and $\U \neq 0$, then $\vphi_0$ is actually determined by $B$.
% \end{remark}

\begin{lemma}\label{lemma:B-determines-vphi_0}
	Let $B \neq 0$ be a sesquilinear form on $\U$.
	Then there is a unique super-anti-automorphism $\vphi_0$ on $\D$ such that $B$ is sesquilinear with respect to $\vphi_0$.
\end{lemma}

\begin{proof}
	Since $B\neq 0$, there are homogeneous elements $u, v\in \U$ such that $B(u,v) \neq 0$ and, hence, $B(u,v)$ is an invertible element of $\D$.
	Suppose $B$ is sesquilinear with respect to super-anti-automorphisms $\vphi_0$ and $\vphi_0'$ on $\D$.
	Then, for all $d\in \D\even \cup \D\odd$, we have
	\[ B(ud,v) = (-1)^ {(|B| + |u|)|d|} \vphi_0(d) B(u,v) = (-1)^ {(|B| + |u|)|d|} \vphi_0'(d) B(u,v) \]
	and, therefore, $\vphi_0(d) = \vphi_0'(d)$.
\end{proof}

Suppose for now that a degree-preserving su\-per\--an\-ti\--auto\-mor\-phism $\vphi_0: \D \to \D$ is given.
We can use it to define a right $\D$-action on $\U\Star$ by interpreting it as an isomorphism from $\D$ to $\D\sop$ and putting
%
\begin{equation}\label{eq:right-D-action}
	f\cdot d \coloneqq f \overline{\vphi_0(d)},
\end{equation}
%
for all $f \in \U\Star$ and $d\in \D$.
Using this action, we have $\End_\D (\U\Star) = \End_{\D\sop} (\U\Star)$. We also have the following:

\begin{prop}\label{prop:sesquilinear-form-iff-D-linear-map}
	Let $\vphi_0\from \D \to \D$ be a a degree-preserving su\-per\--an\-ti\--auto\-mor\-phism and consider the right $\D$-supermodule structure on $\U\Star$ given by Equation \eqref{eq:right-D-action}.
	Then the $\vphi_0$-sesquilinear forms $B:\U\times\U \to \D$ are in a one-to-one correspondence with the homogeneous $\D$-linear maps $\theta\from \U \to \U\Star$ via $B \mapsto \theta$ where $\theta(u) \coloneqq B(u, \cdot)$, for all $u \in \U$ or, inversely, $\theta \mapsto B$ where $B(u,v) \coloneqq \theta(u)(v)$, for all $u, v\in \U$.
	Moreover, $B$ is nondegenerate if, and only if, $\theta$ is an isomorphism.
\end{prop}

\begin{proof}
	Suppose $B$ is given. Condition \eqref{enum:linear-on-the-second} of Definition \ref{def:sesquilinear-form} tells us that $\theta$ defined this way is, indeed, a map from $\U$ to $\U\Star$.
	Also, it is easy to check that $\theta$ is homogeneous of the same parity and degree as $B$.

	Recalling the left $\D$-action on $\U\Star$, condition \eqref{enum:vphi0-linear-on-the-first} tells us that, for all $u\in \U$ and $d\in \D$,
	\[
		\theta (ud) = (-1)^{|d|(|\theta|+|u|)}\vphi_0 (d) \theta (u),
	\]
	which, by the definition of the right $\D\sop$-action on $\U\Star$, is equivalent to $\theta(ud) = \theta(u) \overline{\vphi_0(d)}$, \ie, $\theta$ is, indeed, $\D$-linear considering the left $\D$-action on $\U\Star$ given by Equation \eqref{eq:right-D-action}.

	To show that the correspondence is bijective, note that all the considerations above can be reversed when, given $\theta\from \U \to \U\Star$, we define  $B(u, v) \coloneqq \theta(u)(v)$.

	The ``moreover'' part follows from the fact that $\rad B = \ker \theta$, so the nondegeneracy of $B$ is equivalent to $\theta$ being injective. But $\U$ and $\U\Star$ have the same (finite) rank over $\D$, so $\theta$ is injective if, and only if, it is bijective.
\end{proof}

% \begin{lemma}\label{lemma:sesquilinear-form-iff-D-linear-map}
%     Consider $\U\Star$ as a left $\D$-supermodule by using $\vphi_0$ as above. Then there is a bijection between the $\D$-linear maps $\theta\from \U \to \U\Star$ and the $\vphi_0$-sesquilinear forms $B:\U\times\U \to \D$ via $\theta \mapsto B$ where $B(u,v) \coloneqq \theta(u)(v)$. Moreover, $\theta$ is an isomorphism if, and only if, $B$ is nondegenerate.
% \end{lemma}

% Condition \eqref{enum:linear-on-the-second} on the Definition above allows us to use $B$ to define a map $\U \to \U\Star$ by $u \mapsto B(u, \cdot)$. If this is invertible, we say that the form $B$ is \emph{nondegenerate}.

% Note that a degree-preserving super-anti-automorphism $\vphi_0: \D \to \D$ can be seen as an isomorphism between $\D$ and $\D\sop$ and, hence, one can use it to define a right $\D$-action on $\U\Star$ by $v$

Coming back to our map $\vphi_1\from \U \to \U\Star$ and using the identifications $\D = \End_R(\U)$ and $\D\sop = \End_R(\U\Star)$ introduced above, consider the map $\D \to \D\sop$ sending $d \mapsto \sign{d}{\vphi_1}\vphi_1\inv d \,\vphi_1$, where juxtaposition denotes composition of maps on the right.
It is straightforward to check that this map is an isomorphism and, hence, we can consider it as a super-anti-automorphism $\vphi_0 \from \D \to \D$.
Then, for all $u\in \U$ and $d\in \D$, we have
%
\begin{equation}\label{eq:sesquilinear-before-B}
	(ud)\vphi_1 = u (d\vphi_1) =  u(\vphi_1\vphi_1\inv d \vphi_1) = \sign{d}{\vphi_1}(u\,\vphi_1 )\vphi_0(d).
\end{equation}

% The map $\vphi_0$ depends on the choice o f $\vphi_1$. Following Lemma \ref{lemma:all-possible-vphi1}, if we had started with $\vphi_1' \coloneqq d \vphi_1$ instead, we would have gotten the map $\vphi_0' \coloneqq \vphi_0 \circ \operatorname{sInt}_d$, where $\operatorname{sInt}_d: \D \to \D$ is defined by $\operatorname{sInt}_d(c) = \sign{d}{c} d\inv c d$.

\begin{defi}\label{def:change-map-to-the-left}
	Let $\V$ and $\V'$ be left $R$-supermodules and let $\psi: \V \to \V'$ be a $\ZZ_2$-homogeneous $R$-linear map. 
	We define $\psi^\circ: \V \to  \V'$ to be the following map, written on the left:
	\[
		\forall v\in \V\even \cup \V\odd, \quad \psi^\circ(v) = \sign{\psi}{v} v\psi.
	\]
\end{defi}

For example, using the identification $\D\sop = \End_R (\U\Star)$ as before, the left $\D$-action on $\U\Star$ is given by $df = \bar d^\circ (f)$, for all $d\in \D$ and $f\in \U\Star$.

\begin{lemma}\label{lemma:change-of-side-properties}
	Under the conditions of Definition \ref{def:change-map-to-the-left}, we have that, for all $r\in R\even \cup R\odd$ and $v\in \V$,  $\psi^\circ (rv) = \sign{\psi}{r} r \psi^\circ (v)$.
	Further, given another $\ZZ_2$-homogeneous $R$-linear map $\tau: \V' \to \V''$, we have $(\psi\tau)^\circ = \sign{\psi}{\tau} \tau^\circ\psi^\circ$. \qed
\end{lemma}

% \begin{remark}
%     The map $\psi^\circ$ defined above is not, in general, $R$-linear. Instead, it satisfies $\psi^\circ (rv) = \sign{\psi}{v} r \psi^\circ (v)$, for all $r\in R$ and $v\in V$. 
%     Also, given another $R$-linear map $\theta: \V' \to \V''$, we have that $(\psi\theta)^\circ = \sign{\psi}{\theta} \theta^\circ\psi^\circ$.

% Let $\psi: \V \to \V'$ and $\theta: \V' \to \V''$ be $R$-linear maps between left $R$-supermodules. Then
% \begin{enumerate}[(i)]
%     \item $\psi^\circ (rv) = \sign{\psi}{v} r \psi^\circ (v)$, for all $r\in R$ and $v\in V$;
%     \item $(\psi\theta)^\circ = \sign{\psi}{\theta} \theta^\circ\psi^\circ$.
% \end{enumerate}
% \end{remark}

Using the notation just introduced, we can rewrite Equation \eqref{eq:sesquilinear-before-B} as follows:
%
\begin{equation}\label{eq:vphi1-circ-is-D-linear}
	\begin{split}
		\vphi_1^\circ (ud) &= (-1)^{|\vphi_1|(|u|+|d|)} (ud)\vphi_1 \\
		&=\sign{\vphi_1}{u}(u\,\vphi_1) \overline{\vphi_0(d)} =
		\vphi_1^\circ (u) \overline{\vphi_0(d)},
	\end{split}
\end{equation}
%
which means, considering the right $\D$-action defined via Equation \eqref{eq:right-D-action},
% We can use $\vphi_0\colon \D \to \D\sop$ to define a right $\D$-module structure on $\U\Star$ via $f\cdot d \coloneqq f\vphi_0(d)$, for all $f\in \U\Star$ and $d\in \D$. 
% Considering this action, Equation \eqref{eq:vphi1-circ-is-D-linear} tells us 
that $\vphi_1^\circ$ is $\D$-linear. (Note, however, that Lemma \ref{lemma:change-of-side-properties} shows that $\vphi_1^\circ$ is not $R$-linear, in general.)

% Also, recalling the definition of the right $\D\sop$-action on $\U\Star$, note that Equation \eqref{eq:vphi1-circ-is-D-linear} can be rewritten using the left $\D$-action as
%
% \begin{equation}\label{eq:sesquilinear-with-vphi1-circ}
%     \vphi_1^\circ (ud) = (-1)^{|d|(|\vphi_1|+|u|)}\vphi_0 (d)\vphi_1^\circ (u).
% \end{equation}

% We define $\vphi_1 := \tilde\vphi_1^\circ$ and $\vphi_0: \D \to \D$ by $\vphi_0(d): \sign{\vphi_1}{d} \vphi_1 d^\circ \vphi_1\inv$. 
% Note that the left $\D$-action on $\U\Star$ is connected the right $\D\sop$-action by the formula $f\cdot d = d^\circ \cdot f$, hence $\vphi_0$ is an anti-super-automorphism of $\D$. Also, we have that $\vphi_1(ud) = \sign{u}{d} \vphi_1(d^\circ  $

% \begin{lemma}
%     Let $B: \U\times \U\Star$ be a nondegenerate $\vphi_0$-sesquilinear map. Then, given $r\in R$, there is a unique $s\in R$ such that \[
%         B(ru,v) = \sign{r}{u} B(u,sv).
%     \]
%     Further, the map $r\mapsto s$ is a super-anti-automorphism of $R$.
% \end{lemma}

% \begin{proof}
%     Consider $\theta: \U \to \U\Star$ as in Lemma \ref{lemma:sesquilinear-form-iff-D-linear-map}. 
%     We then have that
%     \begin{align*}
%         \theta(ru)(v) &= \sign{r}{u} \theta(u)(sv)\\
%         \theta (ru)(v) &= \sign{r}{u} (\theta (u)\circ s)(v)\\
%         \theta (ru) &= \sign{r}{u} \theta(u) \circ s\\
%         (\theta \circ r) (u) &= (-1)^{(|\theta| + |u|)|s|} s\Star (
%     \end{align*}
% \end{proof}

Now we define $B: \U\times \U \to \D$ by $B(u, v) = \vphi_1^\circ (u)(v)$.
By Proposition \ref{prop:sesquilinear-form-iff-D-linear-map}, we have that $B$ is a nondegenerate $\vphi_0$-sesquilinear map.
Using Lemma \ref{lemma:change-of-side-properties} and Equation \eqref{eq:R-action-back-on-the-right}, we have
%
\begin{equation*}
	\begin{split}
		B(ru,v) &= \vphi_1^\circ (ru)(v) = \sign{r}{\vphi_1} \big(r \cdot \vphi_1^\circ (u) \big) (v)\\ &= \sign{r}{\vphi_1} (-1)^{|r|(|\vphi_1| + |u|)} \big(\vphi_1^\circ (u) \circ \vphi(r) \big)(v) \\ &= \sign{r}{u} \vphi_1^\circ (u) \big( \vphi(r)v \big)= \sign{r}{u} B(u,\vphi(r)v).
	\end{split}
\end{equation*}
We have proved one direction of Theorem \ref{thm:vphi-iff-vphi0-and-B}, below. 
Recall the superinner automorphism $\operatorname{sInt}_d$ (\cref{def:superinner}). 

\begin{thm}\label{thm:vphi-iff-vphi0-and-B}
	Let $\D$ be a graded-division superalgebra and let $\U$ be a nonzero right graded module of finite rank over $\D$.
	If $\vphi$ is degree-preserving super-anti-automorphism on $R \coloneqq \End_\D(\U)$, then there is a pair $(\vphi_0, B)$, where $\vphi_0$ is a degree-preserving super-anti-automorphism on $\D$ and $B\from \U \times \U \to \D$ is a nondegenerate $\vphi_0$-sesquilinear form, such that
	%
	\begin{equation}\label{eq:superadjunction}
		\forall r\in R\even \cup R\odd,\,\forall u, v \in \U\even \cup \U\odd,  \quad B(ru,v) = \sign{r}{u} B(u,\vphi(r)v).
	\end{equation}
	%
	Conversely, given a pair $(\vphi_0, B)$ as above, there is a unique degree-preserving super-anti-automorphism $\vphi$ on $R$ satisfying Equation \eqref{eq:superadjunction}.
	Moreover, another pair $(\vphi_0', B')$ determines the same super-anti-automorphism $\vphi$ if, and only if, there is a nonzero $G^\#$-homogeneous element $d\in \D$ such that $B'(u, v) = dB (u, v)$ for all $u, v \in \U$, and, hence, $\vphi_0' = \mathrm{sInt}_d \circ \vphi_0$.
\end{thm}

\begin{proof}
	The first assertion is already proved. For the converse, let $\vphi_0$ be a degree-preserving super-anti-automorphism on $\D$, let $B\from \U \times \U \to \D$ be a nondegenerate $\vphi_0$-sesquilinear form and consider $\theta$ as in
	Proposition \ref{prop:sesquilinear-form-iff-D-linear-map}. Then Equation \eqref{eq:superadjunction} is equivalent to:
	%
	\begin{alignat*}{2}
		\forall r\in R\even \cup R\odd,\,\forall u, v & \in \U\even \cup \U\odd,            & \theta (ru)(v)          & = \sign{r}{u} \theta(u)(\vphi(r)v)                                                   \\
		                                              &                                     &                         & = \sign{r}{u} \big(\theta (u)\circ \vphi(r)\big)(v)                                  \\
		%
		\intertext{and, hence, equivalent to}
		%
		\forall r\in R\even \cup R\odd,\,             & \forall u \in \U\even \cup  \U\odd, & \theta (ru)             & = \sign{r}{u} \theta(u) \circ \vphi(r).
		%
		\addtocounter{equation}{1}\tag{\theequation}\label{eq:theta-is-almost-R-superlinear}                                                                                                                 \\
		%
		\intertext{Recalling the definition of superadjoint operator, Equation \eqref{eq:theta-is-almost-R-superlinear} becomes}
		%
		\forall r\in R\even \cup R\odd,\,\forall u    & \in \U\even \cup \U\odd,            & (\theta \circ r) (u)    & = \sign{r}{u} (-1)^{(|\theta| + |u|)|r|} \big(\vphi(r)\big)\Star \big(\theta(u)\big) \\
		                                              &                                     &                         & =  \sign{r}{\theta} \big(\big(\vphi(r)\big)\Star \circ \theta \big) (u),
		%
		\intertext{which is the same as}
		%
		\forall r                                     & \in R\even \cup R\odd,              & \theta \circ r          & = \sign{r}{\theta}  \big(\vphi(r)\big)\Star \circ \theta.
		\intertext{In other words, we have}
		\forall r                                     & \in R\even \cup R\odd,              & \big(\vphi(r)\big)\Star & = \sign{r}{\theta}\, \theta \circ r \circ \theta\inv.
		%
		\addtocounter{equation}{1}\tag{\theequation}\label{eq:vphi-r-Star-is-a-superconjugation}
	\end{alignat*}
	%
	% We have shown that
	% %
	% \begin{equation}
	%   \big(\vphi(r)\big)\Star = \sign{r}{\theta}\, \theta \circ r \circ \theta\inv.
	% \end{equation}
	%
	Since $\U$ has finite rank over $\D$, the superadjunction map $\End_\D (\U) \to \End_{\D\sop} (\U\Star)$ is invertible and, hence, $\vphi$ is uniquely determined.
	Also, the properties of superadjunction imply that $\vphi$ is, indeed, a super-anti-automorphism of $R$.

	For the ``moreover'' part, let $d$ be a nonzero $G^\#$-homogeneous element of $\D$ and consider $\vphi_0' = \operatorname{sInt}_d \circ \, \vphi_0$ and $B' = dB$.
	We have that $B'$ is $\vphi_0'$-sesquilinear since, for all $c\in \D\even \cup \D\odd$ and $u,v \in \U\even \cup \U\odd$,
	%
	\begin{align*}
		B' (uc, v) & = dB (uc, v) = (-1)^{(|B| + |u| ) |c|} d\vphi_0(c) B(u,v) \addtocounter{equation}{1}\tag{\theequation}\label{eq:dB-is-sesquilinear} \\
		           & = (-1)^{(|B| + |u| ) |c|} d\vphi_0(c) d\inv d B(u,v)                                                                                \\
		           & =  (-1)^{(|B| + |u| ) |c|} \sign{c}{d} (\operatorname{sInt}_d \circ\, \vphi_0) (c)\, dB(u,v)                                        \\
		           & = (-1)^{(|B'| + |u|) |c|} (\operatorname{sInt}_d \circ\, \vphi_0) (c)\, B'(u,v).
	\end{align*}
	%
	To show that $B'$ is nondegenerate, note that $dB(u,v) = 0$ implies $B(u,v) =0$, hence  $\rad B' \subseteq \rad B$.
	Finally, it is straightforward that Equation \eqref{eq:superadjunction} is still true if we replace $B$ by $B'$.

	%and let $\vphi_0'$ be a super-anti-automorphism on $\D$. 
	% we will first check that $B'$ is, indeed, $\vphi_0'$-sesquilinear.

	To prove the other direction, we consider, again, the left $R$-supermodule structure on $\U\Star$ given by Equation \eqref{eq:R-action-back-on-the-right} and let $\theta\from \U \to \U\Star$ be as above, \ie, $\theta(u) = B(u, \cdot)$. Similarly, let $\theta'\from \U \to \U\Star$ be defined by $\theta' \coloneqq B'(u, \cdot)$.

	Combining Equations \eqref{eq:R-action-back-on-the-right} and  \eqref{eq:theta-is-almost-R-superlinear}, we have that
	%
	\begin{equation}\label{eq:theta-is-R-superlinear}
		\theta(ru) = \sign{\theta}{r} r\cdot \theta(u).
	\end{equation}
	%
	Define the map $\tilde \theta\from \U \to \U\Star$, written on the right, by $u\, \tilde\theta = \sign{u}{\theta} \theta(u)$, for all $u\in \U$ (compare with Definition \ref{def:change-map-to-the-left} and note that $\theta = (\tilde \theta)^\circ$).
	Then Equation \eqref{eq:theta-is-R-superlinear} becomes $(ru)\tilde\theta = r \cdot (u\,\tilde\theta)$, \ie, $\tilde\theta$ is $R$-linear.

	% Now, given a pair $(\vphi_0', B')$, we have the corresponding map $\theta'(u) \coloneqq B'(u, \cdot)$, for all $u\in \U$. 
	All these considerations about $\theta$ are also valid for $\theta'$, so we define $\tilde\theta'\from \U \to \U\Star$ by $u\,\tilde\theta' \coloneqq \sign{u}{\theta'} \theta'(u)$ and we get another $R$-linear map from $\U$ to $\U\Star$.
	By Lemma~\ref{lemma:nonuniqueness-of-vphi1}, there is $\bar d\in \D\sop$ such that $\tilde\theta' = \tilde\theta \bar d$.
	Applying Lemma \ref{lemma:change-of-side-properties}, this implies \[\theta' = \sign{\theta}{\bar d} \bar d^\circ \theta.\]
	But $\bar d^\circ \theta (u) = d\theta(u)$, where in the last term we use the left $\D$-action on $\U\Star$.
	Therefore $B'(u,v) = \theta'(u)(v) = d\theta(u)(v) = \sign{\theta}{d} dB(u, v)$, for all $u,v \in \U$.
	Replacing $d$ by $\sign{\theta}{d} d$, we get $B' = dB$.

	It remains to check that $\vphi_0' = \operatorname{sInt}_d \circ\, \vphi_0$.
	Since $B' = dB$, Equation \eqref{eq:dB-is-sesquilinear} is valid, hence $B'$ is $(\operatorname{sInt}_d \circ \vphi_0)$-sesquilinenar.
	We then have, for all $c\in \D\even \cup \D\odd$ and $u,v \in \U\even \cup \U\odd$,
	\[
		\vphi_0'(c)\, B'(u,v) = (-1)^{(|B'| + |u|) |c|} B'(uc,v) = (\operatorname{sInt_d} \circ\, \vphi_0) (c)\, B'(u,v).
	\]
	The form $B'$ is nondegenerate, so we can choose $G^\#$-homogeneous $u,v\in \U$ with $B'(u,v)\neq 0$. Then $B'(u,v)$ is invertible, hence $\vphi_0' (c) = (\operatorname{sInt}_d \circ\, \vphi_0) (c)$, concluding the proof.
\end{proof}

The ``conversely'' part of Theorem \ref{thm:vphi-iff-vphi0-and-B} motivates the following:

\begin{defi}\label{def:superadjunction}
	Let $\D$ be a graded-division superalgebra, $\U$ a graded right $\D$-module of finite rank and $B$ a nondegenerate sesquilinear form on $\U$.
	The unique super-anti-automorphism $\vphi$ on $\End_\D(\U)$ defined by Equation \eqref{eq:superadjunction} is called the \emph{superadjunction with respect to $B$} and, for every $r\in \End_\D(\U)$, the $\D$-linear map $\vphi(r)$ is called the \emph{superadjoint of $r$}. 
	We will denote by $E(\D, \U, B)$ the graded superalgebra $\End_\D(\U)$ endowed with this super-anti-automorphism $\vphi$. 
\end{defi}

\begin{remark}\label{conv:pick-even-form}
	Under the conditions of Theorem \ref{thm:vphi-iff-vphi0-and-B}, if $\D$ is an odd graded division superalgebra (\ie, $\D\odd \neq 0$), then we can choose the form $B$ to be even. 
	This is possible since, by the ``moreover'' part, we can substitute an odd form $B$ by $dB$, for some $d\in \D\odd$.
\end{remark}

\begin{prop}\label{prop:R-and-D-have-the-same-center-vphi}
	Recall the isomorphism of $G^\#$-graded algebras $\iota\from Z(\D) \to Z(R)$ given by $\iota (d)(u) \coloneqq ud$, for all $d\in Z(\D)$ and $u\in \U$ (see \cref{prop:R-and-D-have-the-same-center}). 
	We have that $\vphi (\iota (d)) = (-1)^{|B||d|} \iota (\vphi_0(d))$. 
\end{prop}

\begin{proof}
	Fix $d\in Z(\D)\even \cup Z(\D)\odd$ and let $u, v \in \U\even \cup \U\odd$.
	On the one hand,
	\begin{align*}
		B(ud,v) = B(\iota (d)(u), v) = (-1)^{|d||u|} B(u, \vphi(\iota (d)) v).
	\end{align*}
	On the other hand,
	\begin{align*}
		B(ud, v) &= (-1)^{(|B| + |u|) |d|} \vphi_0(d) B(u, v) = (-1)^{(|B| + |u|) |d|} B(u, v) \vphi_0(d)     \\
		 & = (-1)^{|B||d| + |u||d|} B(u, v \vphi_0(d) ) 
		 = (-1)^{|B||d|} (-1)^{|u||d|} B(u, \iota (\vphi_0(d))(v)). 
	\end{align*}
	%
	Since $B$ is nondegenerate, we conclude that $\vphi (\iota (d)) = (-1)^{|B||d|} \iota (\vphi_0(d))$, as desired. 
\end{proof}
