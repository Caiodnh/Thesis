
\chapter{``Recycle Bin''}

\section{Isomorphism abstract}


The following result (\cite[Theorem 2.10]{livromicha}) tell us when two such graded algebras are isomorphic.

\begin{defi}
    Let $\D$ and $\D'$ be graded-division algebras and let $\U$ and $\U'$ be graded right modules over $\D$ and $\D'$, respectively. 
    An \emph{isomorphism from $(\D, \U)$ to $(\D, \U')$} is a pair $(\psi_0, \psi_1)$ where $\psi_0\from \D \to \D'$ is an isomorphism of graded algebras and $\psi_1\from \U \to \U'$ is an isomorphism of graded vector spaces such that $\psi_1(ud) = \psi_1(u) \psi_0(d)$, for all $u\in \U$ and $d \in \D$. 
\end{defi}

\begin{thm}
    Let $R \coloneqq \End_\D(\U)$ and $R' \coloneqq \End_{\D'}(\U')$, where $\D, \D'$ are graded division superalgebras and $\U, \U'$ are nonzero right graded module of finite rank over $\D$ and $\D'$, respectively. 
    Given an isomorphism $\psi\from R \to R'$, there is $g \in G$ and and isomorphism $(\psi_0, \psi_1)$ from $(\D, \U^{[g]})$ to $(\D', \U')$ such that
    \begin{equation}\label{eq:abstract-iso}
        \forall r\in R,\,u \in \U, \quad \psi_1(r u) = \psi(r)\psi_1(u).
    \end{equation}
    Conversely, given an isomorphism $(\psi_0, \psi_1)$ from $(\D, \U^{[g]})$ to $(\D', \U')$, there is a unique isomorphism $\psi\from R \to R'$ satisfying Equation \eqref{eq:abstract-iso}. 
    Two isomorphisms $(\psi_0, \psi_1)$ and $(\psi_0', \psi_1')$ determine the same isomorphism $\psi\from R \to R'$ if, and only if, there exists a nonzero homogeneous element $d\in \D'$ such that $\psi_0' (x) = d\inv \psi_0(x) d$ and $\psi_1' (u) = \psi_1 (u) d$ for all $x\in d$ and $u \in \U$.
\end{thm}

\section{Old chapter on $\vphi$-gradings}

The orthosymplectic and the parapletic superalgebras are defined using a nondegenerate bilinenar form. More specifically, if $\vphi$ is the superadjunction related to this form, then we consider the Lie subsuperalgebra of the skew elements with respect to $\vphi$ (see Subsection {\tt ??}). For the remainder of the chapter, $G$ is a fixed abelian group.

We are now going to define the right notion of ``Type I gradings'' for such algebras %(which, as we are going to see, turn out to be all gradings on then).
The problem here is that not all gradings on the associative superalgebra restrict to the Lie superalgebra of skew elements.
%But the restriction actually work if we have a graded superalgebra with superinvolution, \ie, if the the super involtion is homegeneous of degree $e$.

\begin{defi}
    Let $R$ be a superalgebra and let $\vphi: R\to R$ be any map. We call the pair $(R, \vphi)$ a \emph{superalgebra with distinguished map}. If $\vphi$ is a super-anti-automorphism or a superinvolution, we will call it a \emph{superalgebra with super-anti-automorphism} or \emph{superalgebra with superinvolution}, respectively.
\end{defi}

\begin{defi}
    Let $(R, \vphi)$ and $(S, \psi)$ be superalgebras with distinguished map. A \emph{homomorphism} them is a superalgebra homomorphism $f: R\to S$ such that $f\circ \vphi = \psi \circ f$. In particular, the \emph{automorphism group} of $(R, \vphi)$, denoted by $\Aut(R, \vphi)$, consists of all automorphisms of $R$ commuting with $\vphi$.
\end{defi}

We will be more interested in the case $\vphi$ is super-anti-automorphism (or, more specifically, a superinvolution), but the general case is useful to avoid redundancy in the results and to connect regular gradings to $\vphi$-gradings (see Propositions \texttt{?? [bijections between gradings]  and ?? [example of it]}). Note that if the distinguished maps are taken to be the identity, the concepts above reduce to superalgebra and superalgebra homomorphism. 

\begin{defi}
    Let $(R, \vphi)$ be an associative superalgebra with distinguished map. A \emph{$\vphi$-grading} on it is a $G$-grading $\Gamma$ ($G$ is fixed) such that $\vphi$ is homogeneous of degree $e$. In this case, we say that $(R, \vphi)$ equipped with $\Gamma$ is a \emph{graded superalgebra with distinguished map}.
\end{defi}

Again, the concept of $\vphi$-grading reduces to regular gradings when $\vphi = \id$. In the case $\vphi$ is a super-anti-automorphism, the sub-Lie-superalgebra $\Skew (R, \vphi) = \{ r\in R \mid \vphi(r) = -r \}$ is a graded subsuperspace of $R$. This follows from the lemma below:

\begin{lemma}
    Let $V$ be a $G$-graded vector space and let $f: V \to V$ be a homogeneous map of degree $e$. Then the eigenspaces of $f$ are graded subspaces.
\end{lemma}

\begin{proof}
    Let $v\in V$ be an eigenvector with eigenvalue $\lambda \in \FF$ and write $v = \sum_{g\in G} v_g$, where $v_g \in V_g$. On the one hand, $f(v) = \lambda v = \sum_{g\in G} \lambda v$. On the other hand, $f(v) = \sum_{g\in G} f(v_g)$. Since the decomposition on homogeneous elements is unique, we have that $f(v_g) = \lambda v_g$ for all $g\in G$, completing the proof.
\end{proof}

\section{$\vphi$-gradings and $\widehat G$-actions}

It is important to know how to translate this notion to the language of $\widehat G$-actions.

\begin{prop}
    Let $(R, \vphi)$ be a superalgebra with distinguished map and consider the usual correspondence between gradings on $R$ and $\dual G$-actions. Then gradings on $(R, \vphi)$ correspond to $\dual G$-actions by automorphisms of $(R, \vphi)$.
\end{prop}

\begin{proof}
    Let $\Gamma$ be a grading on $R$. The map $\vphi$ is homogeneous of degree $e$ if, and only if, $\vphi$ being $\widehat G$-equivariant (Proposition \ref{prop:homogeneous-deg-e-widehatG}), which means $\chi\cdot \vphi (r) = \vphi (\chi\cdot r)$ for all $\chi \in \widehat G$ and all $r \in R$. But this is equivalent to say that $\eta_\Gamma(\chi)$ commutes with $\vphi$ for every $\chi \in \dual G$ or, in other words, that $\eta_\Gamma(\chi) \in \Aut (R, \vphi)$. 
\end{proof}

\begin{prop}\label{prop:homogeneous-deg-e-widehatG}
    Let $V$ and $W$ be graded vector spaces. A linear map $f: V \to W$ is homogeneous of degree $e$ if, and only if, $f$ is a $\widehat G$-equivariant map.
\end{prop}


\section{Temporary section for references}

Duals of graded supermodules are introduced for the proof of isomorphism theorem of Type I gradings on $A$.

\begin{convention}\label{conv:maps-left-right}
    Whenever it is necessary, we will distinguish maps written on the left (called simply as \emph{maps on the left}) from maps written on the  right (\emph{maps on the right}) as different concepts.
    
    If $\U$ is a left (right) $\mc R$-supermodule, we use the convention of writing the $\mc R$-linear maps on the right (left).
\end{convention}



The following is the converse of the Graded Density Theorem.

\begin{prop}\label{prop:converse-density-thm}
    Let $\D$ be a graded division superalgebra and let $\U$ be a finite dimensional graded right $\D$-supermodule. Consider $\mc R := \End_\D (\U)$. Then $\U$ is simple as a left $\mc R$-supermodule and $\D \iso \End_{\mc R} (\U)$ via the $\D$-action.
    (we will use this isomorphism to identify $\D$ with $\End_{\mc R} (\U)$). 
\end{prop}

\begin{proof}
Let $u \in \U$ be a nonzero vector and complete it to a basis $\{u\} \cup \{ u_\lambda\}_{\lambda \in \Lambda}$, where $\Lambda$ is an indexing set.

Let $v\in \U$ be any element. We can then define $r\in \End_\D (\U)$ by $r(u) = v$ and $f(u_\lambda) = 0$ for all $\lambda \in \Lambda$. Since $u \neq 0$ was arbitrary, this shows $\U$ is a simple $\mc R$-module. 

The statement that the $\mc R$ action is $\D$-linear is, clearly, the same of the $\D$-action being $\mc R$-linear: $r(vd) = (rv)d$ for all $r\in \mc R$ and $d \in \D$. So the $\D$-action give us, indeed, a map $\D \to \End_{\mc R} (\U)$.
Since $\D$ is graded simple and its action is nonzero ($v\cdot 1 = v$ for all $v \in V$), this map is injective.

To show it is surjective, consider $r$ as defined above, but with $v = u$. Let $f\in \End_{\mc R} (U)$. Since $\U$ is a simple $\mc R$-supermodule, $f$ is determined by its value on $u$.
We have that $uf = u d + \sum_{\lambda \in \Lambda} u_\lambda d_\lambda$ for $d, d_\lambda \in \D$ (where all but finitely many $d_\lambda = 0$). 
Them $ uf = (ru)f = r(uf) = r(ud + \sum_{\lambda \in \Lambda} u_\lambda d_\lambda )  = ud$. Hence $f = d$, concluding the proof.
\end{proof}

% % ---------------------------------------------------
% \section{Superdual of a graded module}\label{ssec:superdual}
% % ---------------------------------------------------

% We will need the following concepts. Let $\D$ be an associative superalgebra with a grading by an abelian group $G$, so we may consider $\D$ graded by the group $G^\# = G\times \ZZ_2$. Let $\U$ be a $G^\#$-graded \emph{right} $\D$-module. The parity $|x|$ of a homogeneous element $x \in \D$ or $x\in \U$ is determined by $\deg x \in G^\#$. The \emph{superdual module} of $\U$ is $\U\Star = \Hom_\D (\U,\D)$, with its natural $G^\#$-grading and the $\D$-action defined on the \emph{left}: if $d \in \D$ and $f \in \U \Star$, then $(df)(u) = d\, f(u)$ for all $u\in \mc U$.

% We define the \emph{opposite superalgebra} of $\D$, denoted by $\D\sop$, to be the same graded superspace $\D$, but with a new product $a*b = (-1)^{|a||b|} ba$ for every pair of $\ZZ_2$-homogeneous elements $a,b \in \D$. The left $\D$-module $\U\Star$ can be considered as a right $\D\sop$-module by means of the action defined by $f\cdot d := (-1)^{|d||f|} df$, for every $\ZZ_2$-homogeneous $d\in \D$ and $f\in \U\Star$.

% \begin{lemma}\label{lemma:Dsop}
% 	If $\D$ is a graded division superalgebra associated to the pair $(T,\beta)$, then $\D\sop$ is associated to the pair $(T,\beta\inv)$.\qed
% \end{lemma}

% If $\U$ has a homogeneous $\D$-basis $\mc B = \{e_1, \ldots, e_k\}$, we can consider its \emph{superdual basis} $\mc B\Star = \{e_1\Star, \ldots, e_k\Star\}$ in $\U\Star$, where $e_i\Star : \U \rightarrow \D$ is defined by $e_i\Star (e_j) = (-1)^{|e_i||e_j|} \delta_{ij}$.

% \begin{remark}\label{rmk:gamma-inv}
% 	The superdual basis is a homogeneous basis of $\U\Star$, with $\deg e_i\Star = (\deg e_i)\inv$. So, if $\gamma = (g_1, \ldots, g_k)$ is the $k$-tuple of degrees of $\mc B$, then $\gamma\inv = (g_1\inv, \ldots, g_k\inv)$ is the $k$-tuple of degrees of $\mc B\Star$.
% \end{remark}

% For graded right $\D$-modules $\U$ and $\V$, we consider $\U\Star$ and $\V\Star$ as right $\D\sop$-modules as defined above. If $L:\U \rightarrow \V$ is a $\ZZ_2$-homogeneous $\D$-linear map, then the \emph{superadjoint} of $L$ is the $\D\sop$-linear map $L\Star: \V\Star \rightarrow \U\Star$ defined by $L\Star (f) = (-1)^{|L||f|} f \circ L$. We extend the definition of superadjoint to any map in $\Hom_\D (\U, \V)$ by linearity.

% \begin{remark}
% 	In the case $\D=\FF$, if we denote by $[L]$ the matrix of $L$ with respect to the homogeneous bases $\mc B$ of $\U$ and $\mc C$ of $\V$, then the supertranspose $[L]\sT$ is the matrix corresponding to $L\Star$ with respect to the superdual bases $\mc C\Star$ and $\mc B\Star$.
% \end{remark}

% We denote by $\vphi: \End_\D (\U) \rightarrow \End_{\D\sop} (\U\Star)$ the map $L \mapsto L\Star$. It is clearly a degree-preserving super-anti-isomorphism. It follows that, if we consider the Lie superalgebras $\End_\D (U)^{(-)}$ and $\End_{\D\sop} (U\Star)^{(-)}$, the map $-\vphi$ is an isomorphism.

% We summarize these considerations in the following result:

% \begin{lemma}\label{lemma:iso-inv}
% 	If $\Gamma = \Gamma(T,\beta,\gamma)$ and $\Gamma' = \Gamma(T,\beta\inv,\gamma\inv)$ are $G$-gradings (considered as $G^\#$-gradings) on the associative superalgebra $M(m,n)$, then, as gradings on the Lie superalgebra $M(m,n)^{(-)}$, $\Gamma$ and $\Gamma'$ are isomorphic via an automorphism of $M(m,n)^{(-)}$ that is the negative of a super-anti-automorphism of 	$M(m,n)$.
% \end{lemma}

% \begin{proof}
% 	Let $\D$ be a graded division superalgebra associated to $(T,\beta)$ and let $\U$ be the graded right $\D$-module associated to $\gamma$. The grading $\Gamma$ is obtained by an identification $\psi: M(m, n) \xrightarrow{\sim} \End_\D (\U)$. By Lemma \ref{lemma:Dsop} and Remark \ref{rmk:gamma-inv}, $\Gamma'$ is obtained by an identification $\psi': M(m, n) \xrightarrow{\sim} \End_{\D\sop} (\U\Star)$. Hence we have the diagram:

% 	\begin{center}
% 		\begin{tikzcd}
% 			& \End_\D (\U) \arrow[to=3-2, "-\vphi"]\\
% 			M(m, n) \arrow[ur, "\psi"] \arrow[dr, "\psi'"]\\
% 			& \End_{\D\sop} (\U\Star)
% 		\end{tikzcd}
% 	\end{center}

% 	Thus, the composition $(\psi')\inv \, (-\vphi) \, \psi$ is an automorphism of the Lie superalgebra $M(m,n)^{(-)}$ sending $\Gamma$ to $\Gamma'$.
% \end{proof}
%
%\bibliographystyle{amsalpha}\bibliography{../02_bibliography}
%\biblio
%
% \end{document}


\section{Old version of $\mc O$}

Note that $\widetilde T$ is not, in general, isomorphic to the direct sum $\{ \pm 1 \} \times T$. 
But it can be seen as subgroup of.

We will now show an example of a division grading on $M(1,1)$ with a super-anti-isomorphism preserving degrees. 
This will allow us to construct a grading on $M(1,1) \times M(1,1)\sop$ with the exchange superinvolution.
% Example \ref{ex:FZ2xFZ2sop-iso-FZ4} shows us the situation is different on $Q(n)\times Q(n)\sop$. 
% We will now give two constructions of an odd division grading on $M(2,2)\times M(2,2)\sop$.

\begin{defi}
    Let $R = M(m,n)$. 
    The \emph{$Q$-super-anti-automorphism} on $R$ is the map 
    %We define 
    $\psi\from R \to R$ given by
    \[
        \psi \begin{pmatrix}
        a & b\\
        c & d
        \end{pmatrix} =
        %
        \begin{pmatrix}
        a\transp             & \sqrt{-1}\, c\transp\\
        \sqrt{-1}\, b\transp & d\transp
        \end{pmatrix}.
    \]
    If $m = n$, $\psi$ clearly restricts to a super-anti-automorphism on $Q(n) \subseteq M(n,n)$.
\end{defi}

\begin{remark}
    As superalgebras with super-anti-automorphism, $M(n,n)$ with the $Q$-super-anti-automorphism is isomorphic to $M(n,n)$ with the supertransposition.
\end{remark}

\begin{ex}\label{ex:supertransp-graded-new}
    Consider on $M(1,1)$ the $\ZZ_2$-grading given by declaring 
    $\begin{pmatrix}
       1 & 0 \\
       0 & 1
     \end{pmatrix}$ and
     $\begin{pmatrix}
       0 & 1 \\
       1 & 0
     \end{pmatrix}$ to have degree $\overline 0$ and
     $\begin{pmatrix}
       -1 & 0 \\
       0 & 1
     \end{pmatrix}$ and
     $\begin{pmatrix}
       0 & -1 \\
       1 & 0
     \end{pmatrix}$ to have degree $\overline 1$. In other words, as a $\ZZ_2^\#$-grading on $M_2(\FF)$ we have\\
     %
     \begin{center}
     \begin{tabular}{ l c r }
     $\deg \begin{pmatrix}
      \phantom{-}1 & 0\phantom{..} \\
      \phantom{-}0 & 1\phantom{..}
     \end{pmatrix} = (\bar 0, \bar 0)$, && $\deg \begin{pmatrix}
      \phantom{.}0 & \phantom{-}1\phantom{.} \\
      \phantom{.}1 & \phantom{-}0\phantom{.}
     \end{pmatrix} = (\bar 0, \bar 1)$,\\
     $\deg \begin{pmatrix}
      -1 & 0\phantom{..} \\
       \phantom{-}0 & 1\phantom{..}
     \end{pmatrix} = (\bar 1, \bar 0)$ &
     and
     & $\deg \begin{pmatrix}
      \phantom{.}0 & -1\phantom{.} \\
      \phantom{.}1 & \phantom{-}0\phantom{.}
     \end{pmatrix} = (\bar 1, \bar 1)$,
     \end{tabular}
     \end{center}
     %
     which is clearly a division grading.
     The $Q$-super-anti-automorphism is a super-anti-automorphism on it.
\end{ex}

% Consider $R = M(1,1)\times M(1,1)$ with the superinvolution $\vphi\from R \to R$ given by $\vphi(x, y) = (\psi\inv (y), \psi (x))$. 
% Clearly, $R$ is isomorphic to $M(1,1)\times M(1,1)\sop$ with exchange superinvolution.
% The grading on the example above entails a grading on $M(1,1)\times M(1,1)$, which is not division since not every component has dimension $1$. 
% But we can refine this grading while keeping $\phi$ homogeneous of degree $e$. 


% Corollary \ref{cor:no-odd-M-vphi} tells us we cannot have odd gradings on We will now investigate what are the possible odd division gradings on the superinvolution-simple superalgebras. 


% Among the series of superinvolution-simple superalgebras, we cannot have odd gradings on the ones of type $M$. 
% On the ones of type $Q\times Q\sop$, Example \ref{ex:FZ2xFZ2sop-iso-FZ4} give us an example of a $\ZZ_4$-graded superalgebra.
% It remains to check if there are odd gradings on $M\times M\sop$. 
% We will give two different constructions of the same grading now.

% The map $\vphi_0$ is a super-anti-automorphism if, and only if, for all $a,b \in T$, $X_a \in \D_a$ and $X_b\in \D_b$,
%
% \begin{align*}
%     &\vphi_0(X_a X_b) = (-1)^{|a||b|} \vphi_0(X_b) \vphi_0(X_a)\\ \iff& \eta(ab)X_a X_b = (-1)^{|a||b|} \eta(a) \eta(b) X_b X_a\\ \iff& \eta(ab)X_a X_b = (-1)^{|a||b|} \eta(a) \eta(b) \beta(b,a) X_a X_b\\ \iff& \eta(ab) = (-1)^{|a||b|} \eta(a) \eta(b) \beta(b,a), \numberthis \label{eq:eta-super-anti-auto}
% \end{align*}
%

% \begin{prop}
%     If $\D$ is simple as an algebra, then $T$ is an elementary $2$-group.
% \end{prop}

% \begin{proof}
%     If $\D$ simple as algebra, then $\beta$ nondegenerate, so the map $T \to \widehat T$ given by $t \mapsto \beta(t, \cdot)$ is a group isomorphism.
%     In particular, if $t\in T$ has order $n>2$, then $\beta(t, \cdot)\in \widehat T$ also has order $n$.
%     But, by Proposition \ref{prop:superpolarization}, $\beta$ only takes values on $\{ \pm 1\}$, so  $\beta(t, \cdot )^2 =1$ and, hence, the order of $t$ is at most $2$.
% \end{proof}

% \begin{cor}\label{cor:super-anti-order-4}
%     Suppose $\D$ is simple as an algebra.
%     If $\D\odd = 0$, every super-anti-automorphism is a superinvolution, but if $\D\odd \neq 0$, every super-anti-automorphism has order $4$.
% \end{cor}

% \begin{prop}
%     Suppose $T = \supp \D$ is an elementary $2$-group. If $\D$ is even, then every super-anti-automorphism is a superinvolution, but
%     if $\D$ is odd, then every super-anti-automorphism has order $4$.
% \end{prop}



\begin{thm}
    Every odd division $\vphi$-grading on $R = M(m,n)\times M(m,n)\sop$ is of the form $\mc O \tensor \D$ where $\D$ is even.
\end{thm}

\begin{proof}
    Since $Z(R)$ is $\FF \times \FF$
\end{proof}



We will now focus on the case $\D$ is odd, i.e., $\D\odd \neq 0$.

\begin{ex}\label{ex:supertransp-graded}
    Consider on $M(1,1)$ the $\ZZ_2$-grading given by declaring $\begin{pmatrix}
       1 & 0 \\
       0 & 1
     \end{pmatrix}$ and
     $\begin{pmatrix}
       0 & i \\
       1 & 0
     \end{pmatrix}$ to have degree $\overline 0$ and
     $\begin{pmatrix}
       -1 & 0 \\
       0 & 1
     \end{pmatrix}$ and
     $\begin{pmatrix}
       0 & -i \\
       1 & 0
     \end{pmatrix}$ to have degree $\overline 1$. In other words, as a $\ZZ_2^\#$-grading on $M_2(\FF)$ we have\\
     %
     \begin{center}
     \begin{tabular}{ l c r }
     $\deg \begin{pmatrix}
      \phantom{-}1 & 0\phantom{..} \\
      \phantom{-}0 & 1\phantom{..}
     \end{pmatrix} = (\bar 0, \bar 0)$, && $\deg \begin{pmatrix}
      \phantom{.}0 & \phantom{-}i\phantom{.} \\
      \phantom{.}1 & \phantom{-}0\phantom{.}
     \end{pmatrix} = (\bar 0, \bar 1)$,\\
     $\deg \begin{pmatrix}
      -1 & 0\phantom{..} \\
       \phantom{-}0 & 1\phantom{..}
     \end{pmatrix} = (\bar 1, \bar 0)$ &
     and
     & $\deg \begin{pmatrix}
      \phantom{.}0 & -i\phantom{.} \\
      \phantom{.}1 & \phantom{-}0\phantom{.}
     \end{pmatrix} = (\bar 1, \bar 1)$,
     \end{tabular}
     \end{center}
     %
     which is clearly a division grading. The supertranspose is a super-anti-automorphism on it.
\end{ex}

\begin{ex}\label{ex:FZZ_4-with-superinvoltion}
    Let $T= \langle \omega \rangle$ where $\omega$ has order $4$, $\beta$ be trivial and $| \cdot |: T \to \ZZ_2$ be the only nontrivial homomorphism. In other words, $\D = \FF T$ with $\D\even = \FF 1 \oplus F \omega^2$ and $\D\odd = \FF \omega \oplus \omega^3$. We define a superivolution by $\eta (1) = 1$, $\eta (\omega) = 1$, $\eta (\omega^2) = -1$ and $\eta(\omega^3) = -1$. %We will denote this superinvolution by 
    
    One could verify by hand that $\eta$ satisfies Equation \eqref{eq:superpolarization}, but it also follows from the following argument. 
    In both $\D$ and $\D\sop$, $\omega$ is an element with order $4$ and both are superalgebras with dimension $4$. 
    Hence, both homomorphisms $\theta: \frac{\FF[x]}{\langle x^4 - 1 \rangle} \to \D$ and $\theta': \frac{\FF[x]}{\langle x^4 - 1 \rangle} \to \D\sop$ sending $x$ to $\omega$ are isomorphisms (note that $\theta(x^2) = \omega^2$ and $\theta'(x^2) = (\omega\sop)^2 = - \omega^2$, so they are different maps!). 
    The map $\theta'\circ \theta\inv: \D \to \D\sop$ corresponds to a super-anti-automorphism with the $\eta$ above.
\end{ex}

Example \ref{ex:FZZ_4-with-superinvoltion} is our first example of odd graded division superalgebra with superinvolution. We can use it to transform any even graded division superalgebra into an odd one by ``extending scalars''.

\begin{lemma}
    Let $(R, \vphi)$ and $(S, \psi)$ be graded superalgebras with superinvolution, with grading groups $G$ and $H$, respectively. Then $(R\tensor S, \vphi\tensor \psi)$ is a $G\times H$-graded superalgebra with superinvolution.\qed
\end{lemma}

\begin{lemma}
    Let $\D$ and $\mc E$ be graded division superalgebras, with grading groups $G$ and $H$, respectively. If $\D$ is even, then $\D \tensor \mc E$ is a $G\times H$-graded division superalgebra.
\end{lemma}

\begin{proof}
    It is straightforward that the tensor product of graded division algebras is a graded division algebra, hence the result follows from considering $\D$ as a $G$-graded algebra and $\mc E$ as a $H\times \ZZ_2$-graded algebra.
\end{proof}

WE REMIND THE AUTHOR TO EXPLAIN THE DIFFERENT TYPES OF TENSOR PRODUCT IN THE GENERALITIES.

Using the two Lemmas above and Example \ref{ex:FZZ_4-with-superinvoltion}, we have an infinitely many examples of odd graded division superalgebras. Nevertheless, their centers include a $4$-dimensional superalgebra, so none of them is a Type II odd graded division algebra. 
\begin{ex}
    Consider $M_2(\FF)$ with the division grading
\end{ex}

Neither Example \ref{ex:supertransp-graded} or Example \ref{ex:FZZ_4-with-superinvoltion} appear as graded division superalgebras of a Type II grading on $M(n,n)$. But both can be used to construct such superalgebras. We will now show two constructions, one based on Example \ref{ex:supertransp-graded} and other on Example \ref{ex:FZZ_4-with-superinvoltion}.

% Example \ref{ex:FZZ_4-with-superinvoltion} suggests that to have a graded division superalgebra we need to ``increase'' the canonical $\ZZ_2$-grading to a $\ZZ_4$-grading. We have seen two constructions that do that on Section \ref{sec:loop-and-induced-algebras}.% We will start with

Neither Example \ref{ex:supertransp-graded} or Example \ref{ex:FZZ_4-with-superinvoltion} are examples of odd Type II gradings. Nevertheless, we will now present two constructions of such gradings, one based on Example \ref{ex:supertransp-graded} and other based on Example \ref{ex:FZZ_4-with-superinvoltion}. 

\begin{prop}\label{prop:involution-on-cartesian-product}
    Let $R$ be a $G$-graded superalgebra and $\psi: R \to R$ be a super-anti-automorphism (homogeneous of degree $e$). Then $\vphi: R\times R \to R\times R$ geiven by $\vphi(x, y) = (\vphi\inv (y), \vphi (x))$ is a superinvoltion on the $G$-graded superalgebra $R\times R$. \qed
\end{prop}

Using Proposition \ref{prop:involution-on-cartesian-product} with $R$ being the graded superalgebra of Example \ref{ex:supertransp-graded}, we get a graded superalgebra with involution which is not a graded division superalgebra. To make its grading a division grading, we need to refine it. We could do it explicitly, but to not repeat ourselves we will follow another approach.

\begin{ex}
    Let $(R, \psi)$ be the graded superalgebra with su\-per-an\-ti-auto\-mor\-phism of Example \ref{ex:supertransp-graded}. Interpreting its canonical $\ZZ_2$-grading as a $\frac{\ZZ_4}{\langle \bar 2 \rangle}$-grading, its induced algebra $\D = \operatorname{Ind} x$ is a graded division superalgebra. Indeed, $\D = \chi_1 \tensor R \oplus \chi_i \tensor R$ where $\chi_z: \ZZ_4 \to \FF^\times$ denotes the character that sends $\bar 1$ to $z$. Its grading is $\ldots$ .
    
    Now let $\vphi'$ be the homogeneous of degree $e$ map on induced from $\psi$ and $\theta$ be the automorphism given by the action of $\chi_i$. We define $\vphi := \theta \vphi'$. One can readly check that it a superinvolution on $\D$.
\end{ex}

We will now construct a different model of the same superalgebra. This new construction, though, can be done over any field, even if it does not have a square root of $-1$.

%As we saw in Section \ref{ainda-nao-existe}, in an algebraically closed field of characteristic $0$, the induced algebra is always isomorphic to a corresponding loop algebra.

\begin{prop}
    Let $(S, \psi)$ be the superalgebra with superinvolution of Example \ref{ex:FZZ_4-with-superinvoltion} and let $M_2(\FF)$ be the matrix algebra endowed with the $\ZZ_2 \times \ZZ_2$-division grading given by:...
    
    and let $\theta$ be the transposition on it.
\end{prop}




\section{Lost and found}

We will investigate one last question...

\begin{cor}
    Under the conditions of Theorem \ref{thm:vphi-involution-iff-delta-pm-1}, assume further that $\D$ is not an even 
\end{cor}

We have that $\overline B = \delta B $ where $\delta \in \{\pm 1\}$. 
Changing $B$ for $B' \coloneqq dB$, by Lemma \ref{lemma:bar-dB}, we get $\overline{B'} = (-1)^{|d|} \vphi_0\inv (d) \overline B = (-1)^{|d|} \vphi_0\inv (d) \delta B = (-1)^{|d|} \vphi_0\inv (d) \delta d\inv d B = (-1)^{|d|} \vphi_0\inv (d) \delta d\inv \overline B = (-1)^{|d|} \vphi_0\inv (d) d\inv \delta \overline B$.
So we want to make the scalar $(-1)^{|d|} \vphi_0\inv (d) d\inv$ be $-1$. 
Note that $(-1)^{|d|} \vphi_0\inv (d) d\inv = -1$ iff $\vphi_0\inv (d) = - (-1)^{|d|} d$. 
We cannot find such delta iff $ (-1)^{|d|} \vphi_0 = \id$ (see 1st Prop in next section), which implies $\D$ supercommutative. But not odd graded division superalgebra can be supercommutative (since it implies $x^2 = -x^2$ for odd $x$, hence we cannot have invertible odd elements). So the only problem is when $\D$ is even and commutative. 
    