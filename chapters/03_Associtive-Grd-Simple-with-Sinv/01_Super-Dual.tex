\section{The superdual}\label{ssec:superdual}

Let $\D$ be a graded-division superalgebra. 
We start with the definition of the superdual for $G$-graded $\D$-supermodules or, equivalently, the dual for $G^\#$-graded $\D$-modules (see \cite[Definition 2.56]{livromicha}). 

\begin{defi}\label{def:superdual-supermodule}
    Let $\U$ be a graded right $\D$-supermodule of finite rank. 
    The \emph{superdual of $\U$} is defined to be $\U\Star \coloneqq \Hom_\D (\U,\D)$, with its usual $G^\#$-grading (see \cref{prop:Hom_D-is-graded}). 
    We give $\U\Star$ the structure of a graded \emph{left} $\D$-module: if $d \in \D$ and $f \in \U \Star$, then we define $(df)(u) = d\, f(u)$ for all $u\in \mc U$. 
\end{defi}

We have used right supermodules over a graded-division superalgebra to classify graded-simple superalgebras (see \cref{ssec:D-modules,ssec:supermodules-over-D}). 
Hence, it is convenient to see the superdual of a right supermodule again as a right supermodule over a suitable graded-division superalgebra.

\begin{defi}\label{def:superopposite}
    Let $R$ be a $G$-graded superalgebra. 
    We define the \emph{superopposite $G$-graded superalgebra $R\sop$} to be $R$ as a graded superspace, but with a different product. 
    When an element $r\in R$ is regarded as an element of $R\sop$, we will denote it by $\bar r$. 
    The product on $R\sop$ is defined by $\bar r \, \bar {s} = \sign{r}{s} \, \overline{sr}$ for all $r, s \in R\even \cup R\odd$. 
\end{defi}

Note that $R\sop$ being a $G$-graded superalgebra depends on the assumption that $G$ is abelian. 

\begin{remark}\label{rmk:sop-super-anti-iso}
    If $\vphi\from R \to S$ is a super-anti-isomorphism between the $G$-graded superalgebras $R$ and $S$, then the map $\vphi$ can be viewed as an isomorphism $R\to S\sop$ or $R\sop \to S$.
\end{remark}

It is easy to see that $\D\sop$ is also a graded-division superalgebra. 
If $\V$ is a graded left $\D$-supermodule, then we can regard $\V$ as a graded right $\D\sop$-supermodule by means of the action $v \bar d \coloneqq (-1)^{|d||v|} dv$, for all $d\in \D\even \cup \D\odd$ and $v\in \V\even \cup \V\odd$. 
In particular, if $\U$ is as in \cref{def:superdual-supermodule}, then the left $\D$-supermodule $\U\Star$ can be regarded as a graded right $\D\sop$-supermodule. 

\begin{defi}\label{defi:superdual-map}
    Let $\U$ and $\V$ be graded right $\D$-modules of finite rank. 
    Given a homogeneous $\D$-linear
    $L:\U \rightarrow \V$, we define the \emph{superdual of $L$} to be the $\FF$-linear map $L\Star\from \V\Star \rightarrow \U\Star$ defined by
    \[
        L\Star (f) = (-1)^{|L||f|} f \circ L,
    \] 
    for all $f\in (\V\Star)\even \cup (\V\Star)\odd$. 
    It is easy to see that $L\Star$ is $\D\sop$-linear. 
    We extend the definition of superdual to every map in $\Hom_\D (\U, \V)$ by linearity.
\end{defi}

\begin{defi}\label{defi:two-dual-bases}
    If $\B = \{u_1, \ldots, u_k\}$ is a graded basis, we can consider its two \emph{superdual bases} ${}\Star \mc B = \{{}\Star u_1, \ldots, {}\Star u_k\}$ and $\mc B\Star = \{u_1\Star, \ldots, u_k\Star\}$ in $\U\Star$, where ${}\Star u_i \from \U \rightarrow \D$ is defined by ${}\Star u_i(u_j) = \delta_{ij}$ and $u_i\Star \from \U \rightarrow \D$ is defined by $u_i\Star (u_j) = (-1)^{|u_i||u_j|} \delta_{ij}$. 
    Clearly, $\deg ({}\Star u_i) = \deg  (u_i\Star) = (\deg u_i)\inv$. 
\end{defi}

\begin{remark}
	In the case $\D=\FF$, if we denote by $[L]$ the matrix of $L$ with respect to the graded bases $\mc B$ of $\U$ and $\mc C$ of $\V$, then the supertranspose $[L]\sT$ is the matrix corresponding to $L\Star$ with respect to the superdual bases $\mc C\Star$ and $\mc B\Star$.
\end{remark}

Since $\U\Star$ is a graded right $\D\sop$-supermodule of finite rank, we can define $\U^{\star\star} \coloneqq \Hom_{\D\sop} (\U\Star, \D\sop)$, which is a graded left $\D\sop$-supermodule and, hence, a graded right $\D$-supermodule. 
As with finite dimensional vector spaces, there is a canonical isomorphism $\U \to \U^{\star\star}$: 

\begin{lemma}\label{lemma:double-dual}
    The $\FF$-linear map $\epsilon_\U\from \U \to \U^{\star\star}$ defined by $\epsilon_\U(u)(f) = \sign{u}{f}\, \overline{f(u)}$, for all $u \in \U\even \cup \U\odd$ and $f \in (\U\Star)\even \cup (\U\Star)\odd$, is an isomorphism of graded right $\D$-supermodules. 
\end{lemma}

\begin{proof}
    For brevity, we will write $\epsilon$ for $\epsilon_\U$. 
    First we check that $\epsilon (u)$ belongs to $\U^{\star\star}$, \ie, it is a $\D\sop$-linear map. 
    Let $f\in (\U\Star)\even \cup (\U\Star)\odd$ and $d\in \D\even \cup \D\odd$. 
    Then,
    \begin{align}
        \epsilon(u)(f\bar d) &= \sign{u}{f \bar d}\, \overline{f\bar d(u)} 
        = (-1)^{|u| (|f| + |d|)}\, \sign{f}{d}\, \overline{df(u)} \\
        &= (-1)^{|u||f| + |u||d| + |f||d|} (-1)^{(|f| + |u|) |d|} \, \overline{f(u)} \, \bar d \\
        &= (-1)^{|u||f|} \, \overline{f(u)} \bar d 
        = \epsilon(u)(f) \bar d. 
    \end{align}

    It is easy to see that $\epsilon(u)$ has the same $G^\#$-degree as $u$, for all $u \in \U$. 
    
    To show $\D$-linearity of $\epsilon$, we compute: 
    \begin{align}
        \epsilon(ud)(f) &= \sign{ud}{f}\, \overline{f(ud)} 
        = (-1)^{(|u| + |d|)|f|}\, \overline{f(u)d} \\
        &= (-1)^{|u||f| + |d||f|} (-1)^{(|f| + |u|) |d|} \bar d \,\, \overline{f(u)} \\
        &= (-1)^{|u||f| + |d||u|} \bar d\,\, \overline{f(u)} 
        = \sign{d}{u} \bar d \,\epsilon(u)(f),
        \intertext{for all $f\in (\U\Star)\even \cup (\U\Star)\odd$, so}
        \epsilon(ud) &= \sign{d}{u} \bar d \epsilon(u) = \epsilon(u)d.
    \end{align}

    To see that $\epsilon$ is a isomorphism, let $\{u_1, \ldots, u_k\}$ be a graded basis of $\U$. 
    We have that $\epsilon(u_i)(u_j\Star) = \sign{u_i}{u_j}\, \overline{u_j\Star(u_i)} = \delta_{ij}$, \ie, $\epsilon(u_i) = {}\Star (u_i \Star) = ({}\Star u_i) \Star$. 
    Therefore, $\epsilon$ sends a graded basis to a graded basis, concluding the proof.
\end{proof}

We note in passing that the sign in the definition of $\epsilon_\U$ is essential: without it, $\epsilon_\U$ would not be well-defined in the case of odd $\D$. 

\begin{defi}\label{def:other-Star}
    Let $\U$ and $\V$ be graded right $\D$-supermodules of finite rank.
    For every $\D$-linear map $L\from \V\Star \to \U\Star$, we define ${}\Star L\from \U \to \V$ to be the map $\epsilon_\V\inv \circ L\Star \circ \epsilon_\U$, where $\epsilon_\U$ and $\epsilon_\V$ are as in \cref{lemma:double-dual}.
\end{defi}

The following result is a straightforward computation:

\begin{prop}\label{prop:dual-super-anti-iso}
    Let $\U$ be a nonzero graded right $\D$-supermodule of finite rank. 
    The map $\End_\D (\U) \rightarrow \End_{\D\sop} (\U\Star)$ defined by $L \mapsto L\Star$ is a degree-preserving super-anti-isomorphism and its inverse is given by $L \mapsto {}\Star L$. \qed
\end{prop}

Since $G$ is abelian, $G^\#/T$ is a group and, hence, the map $G^\#/T \to G^\#/T$ given by $x \mapsto x\inv$ is well-defined. 
From the construction of the superdual basis, it is easy to see that $\dim_\D \U_x = \dim_{\D\sop} \U\Star_{x\inv}$, for all $x \in G^\#/T$. 
Hence, if $\kappa\from G^\#/T \to \ZZ_{\geq 0}$ is the map associated to $\U$ as a $G^\#$-graded right $\D$-module (see \cref{ssec:D-modules}), then $\kappa\Star \from G^\#/T \to \ZZ_{\geq 0}$ defined by $\kappa\Star (x) = \kappa (x\inv)$ is the map associated to $\U\Star$ as a $G^\#$-graded right $\D\sop$-module.

It is straightforward to translate this to the maps $G/T \to \ZZ_{\geq 0}$ (even $\D$) and $G/T^+ \to \ZZ_{\geq 0}$ (odd $\D$) associated to the $G$-graded supermodule $\U$, as in \cref{ssec:supermodules-over-D}. 
If $\D$ is even and $\kappa_\bz, \kappa_\bo$ are the maps associated to $\U$, then $\kappa_\bz\Star, \kappa_\bo\Star$ are the maps associated to $\U\Star$. 
If $\D$ is odd and $\kappa$ is the map associated to $\U$, then $\kappa\Star$ is the map associated to $\U\Star$.

Finally, let us assume that $\FF$ is algebraically closed and $\D$ is finite dimensional. 
Then, if $(T, \beta, p)$ is the triple associated to $\D$, then it is clear that $\supp \D\sop = T$ and that the parity map for $\D\sop$ is also $p$. 
Moreover, if $s,t \in T$, $0 \neq X_s \in \D_s$ and $0 \neq X_t \in \D_t$, then, following the notation in \cref{def:superopposite}, we have:
\[
    \overline{X_s} \, \overline{X_t} = \sign{s}{t} \, \overline{X_tX_s} = \sign{s}{t}\,  \beta(t, s) \overline{X_sX_t} = \beta(t, s) \overline{X_t} \, \overline{X_s} = \beta(s, t)\inv \, \overline{X_t} \, \overline{X_s}.
\]
We conclude that $(T,\beta\inv, p)$ is the triple associated to $\D\sop$.