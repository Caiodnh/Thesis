\section{Automorphism groups}\label{sec:Aut-Lie-chap}

In this section, we will present the groups of automorphisms of the finite dimensional simple associative superalgebras (with and without superinvolution), as well as of the classical Lie superalgebras of the $6$ infinite series $A$, $B$, $C$, $D$, $P$ and $Q$. 
The goal is to transfer the results about the classification of gradings from the associative to the Lie case (see \cref{cor:transfer-R-vphi-to-L}). 

The automorphism groups of the simple associative superalgebras, respectively superinvolution-simple associative superalgebras, are easy to compute, and follow from our results in \cref{chap:grd-simple-assc}, respectively \cref{chap:super-inv,chap:grds-sinv-simple}. 
The (outer) automorphism groups of the finite dimensional simple Lie superalgebras were computed in \cite{serganova}. 
It is worth mentioning that the algebraic group schemes of automorphisms of these Lie superalgebras are presented in \cite[Theorem 4.1]{Pianzola}, and the description there is closer to the one we present here. 

\subsection{Associative superalgebras}\label{ssec:aut-ass-sinv}

The automorphisms of the associative superalgebras $M(m,n)$ and $Q(n)$ are well-known and can be computed from \cref{thm:iso-abstract}, with $G = \ZZ_2$. 

For the case of $M(m,n)$, we have $\D = \FF$ and hence, by \cref{thm:iso-abstract}, every automorphism is the conjugation by an invertible element $A \in M(m,n)\even \cup M(m,n)\odd$. 
We define $\mc E (m,n)$ to be the normal subgroup consisting of the conjugations by the invertible elements in $M(m,n)\even$. 
In other words, the elements of $\mc E (m,n)$ are conjugations by matrices of the form
\[\label{eq:mc-E(m-n)-is-conjugation}
\begin{pmatrix}
    x & 0\\
    0 & y
\end{pmatrix},
\]
where $x \in M_{m}(\FF)$ and $y\in M_n(\FF)$ are invertible. 
Clearly, \[
    \mc E(m,n) \iso (\operatorname{GL}_m \times \operatorname{GL}_n) / \FF^\times,
\]
where $\FF^\times$ is identified with scalar matrices.
An odd invertible matrix is necessarily of the form 
\[
\begin{pmatrix}
    0 & x\\
    y & 0
\end{pmatrix},
\]
where $x \in M_{m\times n} (\FF)$ and $y \in M_{n\times m} (\FF)$ are invertible and, hence, $m = n$. 
Therefore, $\Aut( M(m,n)) = \mc E (m,n)$ if $m \neq n$, and we can write $\Aut(M(n,n)) = \mc E (m,n) \rtimes \langle \pi \rangle$, where $\pi$ is the so called \emph{parity transpose}, which is the conjugation by 
\[\label{eq:pi-is-conjugation-by}
    \Pi \coloneqq \begin{pmatrix}
    0 & I_n\\
    I_n & 0
\end{pmatrix}.
\]

To compute the automorphism group of $Q(n)$, let us first consider $n = 1$. 
We know that $Q(1) \iso \FF\ZZ_2$ is a graded-division algebra, so we can apply \cref{lemma:Aut(D)-widehat-T} to conclude that there are only $2$ automorphisms: the identity and the parity automorphism $\nu\from Q(1) \to Q(1)$. 
Now let $n$ be any natural number. 
Recall that, using Kronecker product, we can write $Q(n) \iso M_n(\D)$ with $\D \coloneqq Q(1)$, where $Q(n)^i$ corresponds to $M_n(\D^i)$ (see the proof of \cref{thm:fd-simple-SA}). 
By \cref{thm:iso-abstract,rmk:twist-does-not-change-set}, an automorphism of $M_n(\D)$ is a conjugation by a homogeneous element of $M_n(\D)$, followed by applying an automorphism of $\D$ to each entry of the result. 
In our case, applying the parity automorphism to each entry of a matrix in $M_n(\D)$ is the same as applying the parity automorphism to the matrix. 
Hence, the automorphism group of $Q(n)$ is generated by the conjugations by homogeneous elements of $Q(n)$ and the parity automorphism $\nu\from Q(n) \to Q(n)$. 
Since $\Pi$ is an odd element in the center of $Q(n)$, it is sufficient to take conjugations by even elements only, which are of the form
\[
\begin{pmatrix}
    x & 0\\
    0 & x
\end{pmatrix},
\]
with invertible $x \in M_{n}(\FF)$. 
The parity automorphism commutes with all automorphisms, so we have $\Aut(Q(n)) = \Aut(M_n(\FF)) \times \langle \nu \rangle \iso \operatorname{PGL}_n \times \ZZ_2$. 
Note that every automorphism of $Q(n)$ is of the form $\operatorname{sInt}_A$ for some invertible element $A \in Q(n)\even \cup Q(n)\odd$, since the parity automorphism is $\operatorname{sInt}_\Pi$. 

We will now focus on the finite dimensional superinvolution-simple superalgebras. 
If $R = S\times S\sop$, where $S$ is either $M(m,n)$ or $Q(n)$, and $\vphi\from R \to R$ is the exchange superinvolution, then the automorphisms of $(R, \vphi)$ can be computed following the proof of \cref{lemma:iso-SxSsop}. 
Explicitly, an automorphism of $(R, \vphi)$ is either of the form
\[\label{eq:psi_theta}
    \forall x,y \in S, \quad \psi_\theta (x, \bar {y}) \coloneqq (\theta(x), \overline{\theta(y)}),
\]
where $\theta\from S \to S$ is an automorphism, or of the form
\[\label{eq:psi_zeta}
    \forall x,y \in S, \quad \psi_\theta (x, \bar {y}) \coloneqq (\theta(y), \overline{\theta(x)}),
\]
where $\theta\from S \to S$ is a super-anti-automorphism. 
It is straightforward to check that \cref{eq:psi_theta,eq:psi_zeta} define an isomorphism $\overline{\Aut}(S) \to \Aut(R, \vphi)$, where $\overline{\Aut}(S)$ is the group of automorphisms and super-anti-automorphisms of the superalgebra $S$. 
Recall that, even though there may be no superinvolution on $S \in \{ M(m,n), Q(n) \}$ in general, we always have a super-anti-automorphism (for example, the queer supertranspose of  \cref{def:queer-stp}), so $\Aut (S)$ is a subgroup of index 2 in $\overline{\Aut}(S)$ (excluding the case of $M(1,0)$).


It remains to consider the case $(R, \vphi) = M^*(m,n, p_0)$. 
Let $\langle \, , \rangle \from {\FF^{m|n} \times \FF^{m|n}} \to \FF$ be the nondegenerate symmetric bilinear form with matrix $\Phi$ as in \cref{defi:M(m-n-p_0)}. 
By \cref{thm:iso-abstract-vphi}, we have that every automorphism of $M^*(m,n, p_0)$ is the conjugation by an invertible matrix $A \in M(m,n)\even \cup M(m,n)\odd$ such that 
% there exists $\lambda \in \FF^\times$ with 
\[\label{eq:preserves-the-form-general-case}
    \exists \lambda \in \FF^\times, \, \forall u, v \in (\FF^{m|n})\even \cup (\FF^{m|n})\odd, 
    \quad \langle A u, A v \rangle = \lambda \sign{A}{u} \langle u, v \rangle,
\]
% for all homogeneous elements $u, v \in \FF^{m|n}$ 
(This comes from \cref{eq:iso-B-implies-vphi} together with \cref{defi:shift-on-B}.) 
We claim that $A$ must be even. 
Indeed, let $u,v \in \FF^{m|n}$ with $|u| = \bar 0$ and $|v| = p_0$ such that $\langle u, v \rangle \neq 0$. 
Then, since the form is supersymmetric, we have $\langle u, v \rangle = \langle v, u \rangle$. 
Comparing
\begin{align}
    \langle u, v \rangle 
    &= \lambda\inv \langle A u, A v \rangle
    %
    \intertext{and}
    %
    \langle v, u \rangle 
    &= \lambda\inv (-1)^{|A|\, p_0} \langle A v, A u \rangle \\
    &= \lambda\inv (-1)^{|A|\, p_0} (-1)^{(|A| + p_0) |A|} \langle A u, A v \rangle \\
    &= \lambda\inv (-1)^{|A|} \langle A u, A v \rangle,
\end{align}
we conclude that $|A| = \bar 0$. 
Hence $\Aut(M^*(m,n, p_0)) \subseteq \mc E (m,n)$, \ie, 
$A = \begin{pmatrix}
    x & 0\\
    0 & y
\end{pmatrix}$, 
where $x \in \operatorname{GL}_n$ and $y \in \operatorname{GL}_m$. 

Then \cref{eq:preserves-the-form-general-case} implies that $\langle u, \vphi(A) A v\rangle = \lambda \langle u, v \rangle$, so $\vphi(A) = \lambda A\inv$. % and, hence, $\lambda \in \pmone$, as in the algebra case. 
Since $\FF$ is algebraically closed and we are only interested in the conjugations by $A$, we can multiply $A$ by $1/\sqrt{\lambda}$ and assume $\vphi(A) = A\inv$. 
We conclude that the group of automorphism is
\[
    \Aut(M^*(m,n, p_0)) = \frac{\{ A \in M(m,n)\even \mid A \text{ is invertible and } \vphi(A) = A\inv \}}{\{ \pm I_{m+n}\}}.
\]

For $M^*(m,n, \bar 0)$, 
% \cref{eq:preserves-the-form-general-case} reduces to there is $\lambda \in \FF^\times$ such that $\langle x u_\bz, x v_\bz \rangle = \lambda \langle u_\bz, v_\bz \rangle$ and $\langle y u_\bo, y v_\bo \rangle = \lambda \langle u_\bo, v_\bo \rangle$, for all $u_\bz, v_\bz \in (\FF^{m|n})\even = \FF^m$ and $u_\bo, v_\bo \in (\FF^{m|n})\odd = \FF^n$. 
% Multiplying
% $A = \begin{pmatrix}
%     x & 0\\
%     0 & y
% \end{pmatrix}$
% by $1/\sqrt{\lambda}$ if necessary, we can assume that 
we have that 
$x \in \operatorname{O}_m$ and $y \in \operatorname{Sp}_n$, 
% In this case, $A$ and $A'$ correspond to the same conjugation \IFF $A' = \pm A$. 
% In other words, we have that
so
\[
    \Aut(M^*(m,n, \bar 0)) \iso \frac{\operatorname{O}_m \times \operatorname{Sp}_n}{\{\pm I_{m+n}\}}.
\] 

For $M(n,n, \bar 1)$, note that 
$\vphi %\left( 
\begin{pmatrix}
    x & 0\\
    0 & y
\end{pmatrix}
%\right) 
= 
\begin{pmatrix}
    y\transp & 0\\
    0 & x\transp
\end{pmatrix}$, so $\vphi(A) = A\inv$ \IFF $y = (x\transp)\inv$. 
We conclude that
\[
    \Aut(M^*(m,n, \bar 1)) \iso \frac{\operatorname{GL}_n}{\{ \pm I_n\}}. 
\]

To summarize:

\begin{prop}\label{prop:Aut-associative}
    The automorphism groups of the finite dimensional simple or superinvolution-simple associative superalgebras are: 
    %
    \begin{enumerate}[(i)]
        \item $\Aut(M(m,n)) = \mc E(m,n) \iso (\operatorname{GL}_m \times \operatorname{GL}_n)/\FF^\times$, if $m\neq n$;
        \item $\Aut(M(n,n)) = \mc E(n,n) \rtimes \langle \pi \rangle$;
        \item $\Aut(Q(n)) = \Aut(M_n(\FF)) \times \langle \nu \rangle \iso \operatorname{PGL}_n \times \ZZ_2$;
        \item $\Aut(S\times S\sop, \vphi) \iso \overline{\Aut}(S)$, where $S$ is either $M(m,n)$ or $Q(n)$, and $\vphi$ is the exchange superinvolution;
        \item $\Aut(M^*(m,n, \bar 0)) \iso (\operatorname{O}_m \times \operatorname{Sp}_n)/\{ \pm I_{m+n}\}$;
        \item $\Aut(M^*(n,n, \bar 1)) \iso \operatorname{GL}_n/\{ \pm I_n$\}. \qed
    \end{enumerate}
\end{prop}


\subsection{Lie superalgebras and transfer of gradings}\label{ssec:transfer-result-aut}

We will now compare the automorphism groups of \cref{prop:Aut-associative} with the automorphism groups of classical Lie superalgebras. 

Let $R$ be any associative superalgebra. 
Given an automorphism $\psi\from R \to R$, it is clear that $\psi$ is also an automorphism of the Lie superalgebra $R^{(-)}$. 
Moreover, $\psi$ can be restricted to $R^{(1)} \coloneqq [ R^{(-)}, R^{(-)} ]$ and, also, induces an automorphism on the quotient superalgebra $L \coloneqq R^{(1)}/Z(R^{(1)})$, since an automorphism maps central elements to central elements. 
This gives us an algebraic group homomorphism $\Aut(R) \to \Aut(L)$. 

In the case $R$ is the associative superalgebra $M(m,n)$ (respectively, $Q(n)$), $L$ is simple of type $A(m-1, n-1)$ (resp. $Q(n-1)$). 
Comparing \cite[Theorem 1]{serganova} and \cite[Theorem 4.1]{Pianzola} with \cref{prop:Aut-associative}(i, ii, iii), we see that the homomorphism $\Aut(R) \to \Aut(L)$ is injective but not surjective. 
This means we cannot use \cref{thm:transfer-of-gradings} to reduce the classification of gradings on $L$ to gradings on $R$. 

This problem can be avoided, for almost all cases, if we consider associative superalgebras with superinvolution. 
If $(R, \vphi)$ is one, set $L \coloneqq \Skew(R, \vphi)^{(1)}/Z( \Skew(R, \vphi)^{(1)} )$. 
The considerations above are still valid, and we have an algebraic group homomorphism $\Aut(R, \vphi) \to \Aut(L)$ given by restriction modulo the center. 

If we take $R \coloneqq S\times S\sop$ with exchange superinvolution $\vphi$, where $S$ is the associative superalgebra $M(m,n)$ (respectively, $Q(n)$), then $L$ % \coloneqq \Skew(R, \vphi)^{(1)}/Z( \Skew(R, \vphi)^{(1)} )$  
is simple of type $A(m-1, n-1)$ (resp. $Q(n-1)$). 
Indeed, $\Skew(R, \vphi) = \{ (s, -\bar s) \mid s\in S \}$ and, hence, the projection $R \to S$ restricts to an isomorphism of Lie superalgebras $\Skew(R, \vphi) \to S^{(-)}$. 
Then, by \cref{prop:Aut-associative}(iv), the homomorphism $\Aut(R, \vphi) \to \Aut(L)$ is an isomorphism of algebraic groups, except for the case $L \iso \mathfrak{psl}(2 | 2)$ of type $A(1,1)$.


The same reasoning holds for the orthosymplectic and periplectic Lie superalgebras if we take $R \coloneqq M^*(m, n, p_0)$. 
To summarize, we have the following:

\begin{prop}\label{prop:Aut-R-vphi-and-L-are-the-same}
    Let $(R, \vphi)$ be a finite dimensional superinvolution-simple associative superalgebra with $R\odd \neq 0$, set 
    $L \coloneqq \Skew(R, \vphi)^{(1)}/Z( \Skew(R, \vphi)^{(1)} )$ and assume that $L$ is not of type $A(1,1)$. 
    Then the map $\Aut(R, \vphi) \to \Aut(L)$, given by restriction and reduction modulo the center, is an isomorphism of algebraic groups. \qed
\end{prop}

Together with \cref{thm:transfer-of-gradings}, we get:

\begin{cor}\label{cor:transfer-R-vphi-to-L}
    Let $(R, \vphi)$ and $L$ be as in \cref{prop:Aut-R-vphi-and-L-are-the-same}. 
    Then every grading on $L$ is a restriction and reduction modulo the center of a grading on $(R, \vphi)$, and two gradings on $L$ are isomorphic \IFF they come from isomorphic gradings on $(R, \vphi)$. \qed
\end{cor}

Together with our results from \cref{chap:grd-simple-assc,chap:grds-sinv-simple}, this gives us a classification of gradings on the Lie superalgebras in the series $A$, $B$, $C$, $D$, $P$ and $Q$, except $A(1,1)$. 
In the following sections, we will develop this in detail.