\section{\texorpdfstring{$G$}{G}-gradings and \texorpdfstring{$\widehat{G}$}{G-hat}-actions}\label{sec:g-hat-action}

In this section, $G$ will be assumed to be an \emph{abelian} group. 
% In this section, 
We will introduce an important tool in the theory of gradings: the duality between $G$-gradings and $\widehat G$-actions. 
Here $\widehat G$ denotes the \emph{group of characters} of $G$, \ie, $\widehat G$ is the group whose elements are the group homomorphisms $\chi\from G \to \FF^\times$, with point-wise multiplication of maps. 
We will also assume that $\FF$ is algebraically closed and $\Char \FF = 0$, since this is the only case we need in this work. 
The reader interested in arbritary fields can refer to \cite{livromicha}. 

% the main advantage of this approach is to have formal statement \cref{thm:transfer-of-gradings}, in the following subsection, which is a result that has been used informally in the literature for specific cases (see, \eg,  \cite[Remark 1.40]{livromicha}). 

\begin{defi}
    Given a $G$-grading $\Gamma : V = \bigoplus_{g\in G} V_g$ on a vector space $V$, we define a $\widehat G$-action by
    \[
        \forall \chi \in \widehat G, \, g\in G, \, v_g \in V_g, \quad \chi \cdot v_g \coloneqq \chi(g) v_g. 
    \]
    The corresponding representation map will be denotes by $\eta_\Gamma \from \widehat G \to \GL (V)$. 
\end{defi}

With our assumptions on $G$ and $\FF$, it is well known that $\widehat G$ separate points, \ie, given distinct elements $g, g'\in G$, there is a character $\chi \in \widehat G$ such that $\chi(g) \neq \chi(g')$. 
Hence, we have 
\[
    \forall g\in G, \quad V_g = \{ v\in V \mid \forall\, \chi \in \widehat G, \, \chi \cdot v = \chi(g)v \},
\]
Thus, we can recover the $G$-grading $\Gamma$ from its corresponding $\widehat G$-action $\eta_\Gamma$. 

% As usual, an \emph{automorphism} of $A$ is a bijective homomorphism from $A$ to itself, and the group of automorphisms is denoted by $\Aut(A)$.

\begin{prop}\label{prop:g-hat-Aut-A}
	Let $A$ be an $\Omega$-algebra and $\Gamma$ be a $G$-grading on its underlying vector space.
	Then $\Gamma$ is a $G$-grading on $A$ 
% 	(see \cref{def:grds-on-Omega-algebras}) 
	if, and only if, $\eta_\Gamma(\widehat G) \subseteq \Aut(A)$.
\end{prop}

\begin{proof}
    Let $\omega \in \Omega_n$ and let $a_1, \ldots, a_n \in A$ be homogeneous elements of degrees $g_1, \ldots, g_n \in G$, respectively. 
    Note that $a_1\tensor \cdots \tensor a_n \in A^{\tensor n}$ has degree $g_1 \cdots g_n$. 
    
% 	First, let us assume that $\Gamma$ is a grading on the $\Omega$-algebra $A$. 

    The element $\omega^A (a_1\tensor \cdots \tensor a_n)$ has degree $g_1 \cdots g_n$ \IFF 
    \[\label{eq:aut-g-hat}
        \forall \chi \in \widehat G, \quad
        \chi \cdot \omega^A (a_1\tensor \cdots \tensor a_n) 
		= \chi(g_1 \cdots g_n) \, \omega^A(a_1\tensor \cdots \tensor a_n).
    \]
	Since we have
	\begin{align*}
		\chi(g_1 \cdots g_n) \, \omega^A(a_1\tensor \cdots \tensor a_n) 
		& =\chi(g_1) \cdots \chi(g_n) \, \omega^A(a_1\tensor \cdots \tensor a_n) \\
		& = \omega^A \big(\chi(g_1)a_1\tensor \cdots \tensor \chi(g_n)a_n \big) \\
		& = \omega^A(\chi \cdot a_1 \tensor \cdots \tensor  \chi \cdot a_n),
	\end{align*}
%	
	\cref{eq:aut-g-hat} holds for all homogeneous $a_1, \ldots, a_n \in A$ \IFF $\eta_\Gamma (\chi)$ is an automorphism for every $\chi \in \widehat G$.  
\end{proof}

% ---

The following is straightforward:

\begin{prop}\label{prop:iso-g-hat-action}
    Two $G$-gradings $\Gamma$ and $\Delta$ on an $\Omega$-algebra $A$ are isomorphic \IFF there is an automorphism $\psi \in \Aut(A)$ such that $\eta_\Delta(\chi) = \psi \circ \eta_\Gamma(\chi) \circ \psi\inv$, for all $\chi \in \widehat G$. \qed
\end{prop}

% \begin{proof}
%     Let $\chi \in \widehat G$ be a fixed character and set $\vphi_\Gamma = \eta_\Gamma(\chi)$ and $\vphi_\Delta = \eta_\Delta(\chi)$. 
%     The gradings $\Gamma$ and $\Delta$ are isomorphic \IFF there is $\psi\from (A, \Gamma) \to (A, \Delta)$. 
    
%     $\vphi_\Gamma (v_g) = \chi(g) v_g$
    
%     $\vphi_\Delta ( \psi(v_g) ) = \chi(g) \psi(v_g)$
    
%     Hence, $\vphi_\Delta = \psi \circ \vphi_\Gamma \circ \psi\inv$.
% \end{proof}

% ---

% \input{chapters/01_Generalities/10_excerpt}

If $G$ is a finite abelian group, it is well known (see, \eg, \cite[\S 1.2]{FultonAndHarris}) that every action by $\widehat G$ on a finite dimensional vector space $V$ is \emph{diagonalizable}, \ie, $V$ can be written as a direct sum of subspaces in which each $\chi \in \widehat G$ acts as a nonzero scalar $\lambda_\chi$. 
It follows that the map $\chi \mapsto \lambda_\chi$ is a character of $\widehat G$ and, by duality, there is a unique $g \in G$ such that $\lambda_\chi = \chi(g)$, for every $\chi \in \widehat G$. 
In summary, every $\widehat G$-action corresponds to a $G$-grading.

This can be extended to the case of finitely generated $G$ by considering actions of \emph{algebraic groups} (for a background on algebraic groups, we refer to \cite{MR1064110}, \cite{Arzhantsev-notes} or \cite[Appendix A]{livromicha}). 
Both $\widehat G$ and $\GL (V)$ have natural structures of algebraic groups  (assuming $\dim V < \infty$) and the representation $\eta_\Gamma \from \widehat G \to \GL (V)$ is a homomorphism of algebraic groups. 
The algebraic group $\widehat G$ is a \emph{quasitorus}, \ie, $\widehat G \iso (\FF^\times)^n \times G_f$ where $G_f$ is a finite abelian group. 
Every algebraic representation of a quasitorus $H$ on a finite dimensional vector space $V$ is diagonalizable (see, \eg, \cite[Chapter 3, \S 2, Theorem 3]{MR1064110} or \cite[Theorem 1.6.13]{Arzhantsev-notes}), \ie, $V$ can be written as a direct sum of subspaces in which each $h \in H$ acts as a nonzero scalar $\lambda_h$. 
It can be shown that the map $h \mapsto \lambda_h$ is an \emph{algebraic} character, \ie, a homomorphism of algebraic groups $H \to \FF^\times$. 
The duality can be extended to this case: for every algebraic character $\lambda\from \widehat G \to \FF^\times$, there is a unique element $g\in G$ such that $\lambda_\chi = \chi(g)$. 
We then have the following:

\begin{prop}\label{prop:g-hat-1-to-1}
    Let $V$ be a finite dimensional vector space and assume $G$ is finitely generated. 
    Then the mapping $\Gamma \mapsto \eta_\Gamma$ is a one-to-one correspondence between $G$-gradings on $V$ and homomorphisms of algebraic groups $\widehat G \to \GL(V)$. \qed
\end{prop}

% ---

For an $\Omega$-algebra $A$, it is easy to see that $\Aut (A)$ is a (Zariski) closed subgroup of $\GL (A)$, hence an algebraic group.
Therefore, \cref{prop:g-hat-Aut-A,prop:g-hat-1-to-1} imply that the following is well defined:

\begin{defi}
    Let $A$ and $B$ be finite dimensional universal algebras, not necessarily with the same signature, and assume $G$ is finitely generated.  
    Given a homomorphism of algebraic groups $\theta\from \Aut(A) \to \Aut(B)$ and a $G$-grading $\Gamma$ on $A$, we define $\theta(\Gamma)$ to be the $G$-grading on $B$ corresponding to the homomorphism $\theta \circ \eta_\Gamma\from \widehat G \to \Aut(B)$. 
\end{defi}

Using \cref{prop:iso-g-hat-action}, we get that if $\Gamma$ and $\Delta$ are isomorphic $G$-gradings on $A$, then $\theta(\Gamma)$ and $\theta(\Delta)$ are isomorphic $G$-gradings on $B$. 
Finally, we note that we can drop the hypothesis that $G$ is finitely generated: every $G$-grading $\Gamma$ on $A$ can be seen as a grading by the subgroup generated by $\supp \Gamma$, which can be used to define $\theta (\Gamma)$. 
% because, when working with a finite number of finite dimensional graded spaces, we can replace $G$ by its subgroup generated by the supports. 
We summarize these considerations in the following: 

\begin{thm}\label{thm:transfer-of-gradings}
    Suppose $\FF$ is an algebraically closed field of characteristic $0$. 
    Let $G$ be an abelian group and let $A$ and $B$ be finite dimensional universal algebras, not necessarily with the same signature. 
    If there is a isomorphism of algebraic groups $\Aut(A) \to \Aut(B)$, then there is a bijection between the $G$-gradings on $A$ and the $G$-gradings on $B$ preserving the isomorphism classes. \qed
\end{thm}

We note that this sort of transfer, between algebras with different signatures, has been used in other works, but without having the result stated formally (see, \eg,  \cite[Remark 1.40]{livromicha}).

% ---