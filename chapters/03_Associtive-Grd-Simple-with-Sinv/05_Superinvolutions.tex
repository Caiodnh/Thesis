
\section{Superinvolutions and sesquilinear forms}\label{sec:superinv-sesquilinear-forms}

Our goal now is to specialize the results of Section \ref{sec:super-anti-auto-and-sesquilinear} to the case where $\vphi$ is a superinvolution.
To this end, let us investigate what super-anti-automorphism of $\D$ and what sesquilinear form on $\U$ determine the super-anti-automorphism $\vphi\inv$.
Again, we suppose $\D$ is a graded division superalgebra, $\U$ is a nonzero right graded module of finite rank over $\D$ and put $R = \End_\D (\U)$.

\begin{defi}\label{def:barB}
	Given a super-anti-automorphism $\vphi_0$ on $\D$ and a $\vphi_0$-sesqui\-li\-near form $B$ on $\U$,  we define $\overline {B}\from \U\times \U \to \D$ by $\overline {B} (u,v) \coloneqq \sign{u}{v} \vphi_0\inv (B(v, u))$ for all $u, v \in \U$.
\end{defi}

% \begin{lemma}
%     Let $B\from \U\times \U\Star \to \D$ be a $\vphi_0$-sesquilinear form. Suppose there is a super-anti-automorphism $\vphi$ on $R = \End_\D(U)$ such that Equation \eqref{eq:superadjunction} holds. If $B$ is nonzero, then it is nondegenerate.
% \end{lemma}

\begin{prop}\label{prop:barB-determines-vphi-inv}
	Under the conditions of Definition \ref{def:barB}, we have that $\overline {B}$ is a $\vphi_0\inv$-sesquilinear form of the same degree and parity as $B$.
	Further, if $B$ is nondegenerate and $\vphi$ is the super-anti-automorphism on $R$ determined by $(\vphi_0, B)$ as in Theorem \ref{thm:vphi-iff-vphi0-and-B}, then $\overline{B}$ is nondegenerate and $\vphi\inv$ is determined by $(\vphi_0\inv, \overline B)$, \ie,
	%
	\begin{equation}\label{eq:barB-superadjunction}
		\forall r\in R\even \cup R\odd ,\,\forall u, v \in \U\even \cup \U\odd,  \quad \overline {B}(ru,v) = \sign{r}{u} \overline {B}(u,\vphi\inv (r)v).
	\end{equation}
	%
\end{prop}

\begin{proof}
	Since $B$ is $\FF$-bilinear, so is $\overline {B}$.
	Also, since $\vphi_0$ preserves degree and parity, $\overline {B}$ is homogeneous of the same degree and parity as $B$.
	Let us check the conditions of Definition \ref{def:sesquilinear-form} and Equation \eqref{eq:barB-superadjunction}.
	\vspace{2mm}
	\begin{align*}
		\intertext{ Condition \eqref{enum:linear-on-the-second}:}
		\overline {B} (u,vd) & = (-1)^{|u|( |v| + |d|)} \vphi_0\inv (B(vd, u))                                                       \\
		                     & = (-1)^{|u|( |v| + |d|)} (-1)^{|d| (|B| + |v|)} \vphi_0\inv \big( \vphi_0(d) B(v, u) \big)            \\
		                     & = (-1)^{|u||v| + |u||d| + |d||B| + |d||v|} (-1)^{|d| (|B| + |v| + |u|) }  \vphi_0\inv \big(B(v, u)) d \\ &= \sign{u}{v} \vphi_0\inv \big(B(v, u)) d = \overline {B}(u, v) d .
		\intertext{Condition \eqref{enum:vphi0-linear-on-the-first}:}
		\overline {B}(ud, v) & = (-1)^{(|u| + |d|) |v|} \vphi_0\inv \big( B(v, ud) \big)                                             \\ &= (-1)^{(|u| + |d|) |v|} \vphi_0\inv \big( B(v, u)d \big) \\ &= (-1)^{(|u| + |d|) |v|} (-1)^{|d| (|B| + |v| + |u|)} \vphi_0\inv (d) \vphi_0\inv\big( B(v, u) \big) \\ &= (-1)^{|u||v| + |d| |B| + |d||u|} \vphi_0\inv (d) \vphi_0\inv\big( B(v, u) \big) \\ &= (-1)^{(|B| + |u|) |d|} \vphi_0\inv (d) \overline {B}(u, v).
		\intertext{For Equation \eqref{eq:barB-superadjunction}, note that replacing $r$ for $\vphi\inv(r)$, Equation \eqref{eq:superadjunction} can be rewritten as}
		B(v, ru)             & = \sign{r}{v} B(\vphi\inv(r)v,u).
		\intertext{Hence, we have that}
		\overline {B}(ru, v) & = (-1)^{(|r| + |u|) |v|} \vphi_0\inv \big( B(v, ru) \big)                                             \\ &= (-1)^{(|r| + |u|) |v|} (-1)^{|r||v|} \vphi_0\inv \big( B(\vphi\inv (r)v, u) \big) \\ &= (-1)^{|u||v|} \vphi_0\inv \big( B(\vphi\inv (r)v, u) \big) \\ &= (-1)^{|u||v|} (-1)^{(|r| + |v|) |u|} \overline {B}(u, \vphi\inv (r)v) \\ &= \sign{r}{u} \overline {B}(u, \vphi\inv (r)v).
	\end{align*}

	Finally, Equation \eqref{eq:barB-superadjunction} together with $B$ being nondegenerate implies that $\overline{B}$ is nondegenerate.
	To see that, let $u$ be a nonzero homogeneous element in $\rad \overline{B}$.
	Then for for every $r\in R\even \cup R\odd$ and $v\in \U\even \cup \U\odd$, we have that $\overline{B}(u, \vphi\inv (r) v) = 0$, hence $\overline{B}(ru, v) = 0$.
	Since $r \in R\even \cup R\odd$ and $v\in \U\even \cup \U\odd$ were arbitrary, this implies $\overline{B} (Ru, \U) = 0$.
	But $\U$ is simple as a graded $R$-supermodule, so we would have $\overline{B}(\U, \U) = 0$ and then, using that $\vphi_0$ is bijective, $B (\U, \U) = 0$, a contradiction.
\end{proof}

\begin{lemma}\label{lemma:bar-dB}
	Under the conditions of Definition \ref{def:barB}, let $d$ be a nonzero $G^\#$-homogeneous element of $\D$ and consider $\vphi_0' \coloneqq \operatorname{sInt}_d\circ\, \vphi_0$ and $B' \coloneqq d B$.
	Then $\overline {B'} = (-1)^{|d|} \vphi_0\inv (d) \overline B$.
\end{lemma}

\begin{proof}
	Note that $(\vphi_0')\inv = \vphi_0\inv \circ \operatorname{sInt}_{d}\inv = \vphi_0\inv \circ \operatorname{sInt}_{d\inv}$.
	Hence, for all $u, v \in \U\even \cup \U\odd$,
	%
	\begin{align*}
		\overline {B'} (u,v) & = \sign{u}{v} (\vphi_0\inv \circ \operatorname{sInt}_{d\inv})  (d B(v, u) )                    \\
		                     & = \sign{u}{v} \vphi_0\inv \big( (-1)^{|d| (|d| + |B| + |u| + |v|)}\,  d\inv d B (v, u) d \big) \\ &= \sign{u}{v} (-1)^{|d|}\, \vphi_0\inv (d) \vphi_0\inv(B(v, u)) =  (-1)^{|d|}\,\vphi_0\inv (d) \overline {B} (u, v).
	\end{align*}
\end{proof}

We are primarily interested in the case $\FF$ is an algebraically closed field and $\D$ is finite dimensional.
In this case, we have that $\D\even_e = \FF 1$, so we are under the hypothesis of the following theorem, which is a graded version of \cite[Theorem 7]{racine}:

\begin{thm}\label{thm:vphi-involution-iff-delta-pm-1}
	Let $\D$ be a graded division superalgebra such that $\D_e = \FF 1$, let $\U$ be a nonzero right graded module of finite rank over $\D$ and let $\vphi$ be a degree-preserving super-anti-automorphism on $R \coloneqq \End_\D (\U)$.
	Consider a super-anti-automorphism $\vphi_0$ on $\D$ and a nondegenerate $\vphi_0$-sesqui\-li\-near form $B$ on $\U$ determining $\vphi$ as in Theorem \ref{thm:vphi-iff-vphi0-and-B}.
	Then $\vphi$ is a superinvolution if, and only if, $\overline B = \pm B$. 
	Moreover, if this is the case, then $\vphi_0$ is a superinvolution. 
\end{thm}

\begin{proof}
	Using Proposition \ref{prop:barB-determines-vphi-inv} and Theorem \ref{thm:vphi-iff-vphi0-and-B}, we conclude that $\vphi = \vphi\inv$ if, and only if, there is a $G^\#$-homogeneous element $0 \neq \delta \in \D$ such that $\overline {B} = \delta B$. 
	Hence, it only remains to prove that, in this case, we have $\delta \in \pmone$ and $\vphi_0^2 = \operatorname{id}_\D$. 
	
	Since $B$ and $\overline {B}$ have the same $G^\#$-degree, $\overline {B} = \delta B$ implies that $\delta \in \D_e$. 
	By Lemma \ref{lemma:bar-dB}, we have that
	\[
		B = \overline {\overline B} = \overline {\delta B} = (-1)^{|\delta|} \vphi_0\inv(\delta) \overline B= \vphi_0\inv(\delta) \overline B = \vphi_0\inv(\delta) \delta B,
	\]
	hence $\vphi_0\inv(\delta) \delta = 1$. 
	Since we are assuming $\D_e = \FF 1$, we have that $\vphi_0\inv(\delta) = \delta$, so $\delta^2 = 1$ and, therefore, $\delta \in \{ \pm 1 \}$. 

	To see that $\vphi_0^ 2 = \operatorname{id}_\D$, note that, from Theorem \ref{thm:vphi-iff-vphi0-and-B}, $\vphi_0\inv = \operatorname{sInt}_\delta \circ \,\vphi_0 = \vphi_0$. 
\end{proof}