% Intro for A(n,n) odd 
We can also parametrize $M^{\mathrm{ex}}(T, \beta, t_p, \kappa, g_0)$ with $t_p \not e$ in terms of $G$. 
Let $t_1 \in T^-$ and $K \subseteq T^+$ chosen as in \cref{def:std-realization-MxM-QxQ}(b). 
As done in \cref{ssec:in-terms-of-G}, let $h \in G$ be the element such that $t_1 = (h, \bar 1)$, and define $\chi \in \widehat{T^+}$ by $\chi(t) \coloneqq \beta(t_1, t)$, for all $t\in T^+$. 
The triple $(T, \beta, p)$ can be recovered from the quadruple $(T^+, \beta^+, h, \chi)$ (see \cref{eq:beta-from-h-chi}), so $(T, \beta, t_p)$ can be recovered from $(T^+, \beta^+, h, \chi, t_p)$. 
Recall that we have $e \neq f = h^2$, $t_p \neq f$ and $\rad \beta^+ = \langle t_p, f \rangle$, and note that, since $T^+ = K \times (\rad \beta^+)$, $\chi$ is determined by 
\[\label{eq:equivalent-def-of-chi}
    \chi(K) = \chi(f) = 1 \AND \chi(t_p) = -1.
\]

Conversely, given a $2$-elementary subgroup $T^+ \subseteq G$, an alternating bicharacter $\beta^+\from T^+ \times T^+ \to \pmone$, $e \neq t_p \in T^+$ and $h\in G$ such that $f\coloneqq h^2 \in T^+$, $f \not\in \{e, t_p \}$ and $\rad \beta^+ = \langle t_p, h^2 \rangle$, we can choose a complement $K \subseteq T^+$ to $\rad \beta^+$ and define $\chi \in \widehat{T^+}$ by \cref{eq:equivalent-def-of-chi}. 
Then, it is straightforward that the pair $(h, \chi)$ is $(T^+, \beta^+)$-admissible (\cref{def:O(T+-beta+)}). 
Set $t_1 \coloneqq (h, \bar 1)$, $T^- \coloneqq t_1 T^+$, $T \coloneqq T^+ \cup T^-$, $\beta\from T\times T\to \pmone$ as in \cref{lemma:existence-beta} and $\tilde\beta\from T\times T\to \pmone$ as usual (\cref{}).  
By \cref{}, we get that $\rad \tilde\beta = \langle f \rangle$, and, therefore, $(T, \beta, t_p)$ is as in \cref{}. 

Recall that our definition of $M^{\mathrm{ex}}(T, \beta, t_p, \kappa, g_0)$ depended on a choice of a standard realization $(\D, \vphi)$ associated to $(T, \beta, t_p)$ (\cref{def:std-realization-MxM-QxQ}(b)), which depends on the choices ... (list?). 
Note that $\beta(t_1, K) = 1$ and, hence, we can take $t_1$ and $K$ to be our choices


We define $M^{\mathrm{ex}}(T^+, \beta^+, t_p, h, \kappa, g_0) \coloneqq M^{\mathrm{ex}}(T, \beta, t_p, \kappa, g_0)$. 

Note that $\beta(t_1, K) = 1$ and, hence, we can take $t_1$ and $K$ to be our choices in \cref{def:std-realization-MxM-QxQ}(b). 
Using these, let $\eta \from T \to \pmone$ be the map associated to the superinvolution $\vphi_{\mc M} \tensor \vphi_{\mc C}$, and set $\eta^+ \coloneqq \eta\restriction_{T^+}$. 
It is easy to see how to define $\eta^+$ in terms of our new parameters: we choose 



Note that, if  we can define $\eta^+\from T^+ \to \pmone$




we can

Note that, this definition of $\chi$ is equivalent to defining $\chi$