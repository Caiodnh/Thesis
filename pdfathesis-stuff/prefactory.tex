%%%% If you used '\documentclass[12pt]{pdfathesisi}', then 
%%%% the 'prefatory' enviromment is defined as well as some 
%%%% useful macros to be used whitin the environment.
%%%% Otherwise you are by your own regarding the  formating 
%%%% of your thesis.
%%
%% prefatory pages: front page, abstract, etc.
%
\begin{prefatory}
\hbadness=10000
\frontpage
%%
%%%%%%%%%%%%%%%%%%%%%%%%%%%%%%%%%%%%%%%%%%%%%%%
%%   Abstract is mandatory
%%   It should not exceed 300 words
\abstract%
    Assuming the base field is algebraically closed, we classify, up to isomorphism, gradings by arbitrary groups on non-exceptional classical simple Lie superalgebras, excluding those of type $A(1,1)$, and on finite dimensional superinvolution-simple associative superalgebras. 
    We assume the characteristic to be $0$ in the Lie case, and different from $2$ in the associative case. 
    Our approach is based on a version of Wedderburn Theorem for graded-simple associative superalgebras satisfying a descending chain condition, which allows us to classify superinvolutions using nondegenerate supersymmetric sesquilinear forms on graded modules over a graded-division superalgebra. 
    To transfer the results from the associative case to the Lie case, we use the duality between $G$-gradings and $\widehat G$-actions for finite dimensional universal algebras. 
%%
%%%%%%%%%%%%%%%%%%%%%%%%%%%%%%%%%%%%%%%%%%%%%%
%%   Dedication page is optional
%%   Remove it if you do not want to have it
\dedication%
{\flushright
To Ruth Roy and Dauto Kean-Dos-Santos,\\ 
}
\medskip
\hfill \begin{minipage}[t]{0.46\textwidth}
awesome friends who exemplified the\\ circle of life 
during the production of\\ 
this work.
\end{minipage}


%%%%%%%%%%%%%%%%%%%%%%%%%%%%%%%%%%%%%%%%%%
%%
%%
\laysummary %

One of the most basic concepts in high school mathematics is \emph{polynomials in one variable}. 
Each polynomial is a sum of \emph{monomials}, which are expressions of the form $a x^n$, where $a$ is a real number, $x$ is the variable and $n$ is a nonnegative integer. 
If $a \neq 0$, we say that the number $n$ is the \emph{degree} of the monomial. 
One simple but crucial fact is that a monomial of degree $n$ times a monomial of degree $m$ gives a monomial of degree $n + m$. 

In simple terms, an \emph{algebra}
\footnote{A precise definition can be found in \cite[Chapter 1]{MR3932087}.}
is a set whose elements can be added, multiplied by numbers and multiplied with each other. 
For example, polynomials form an algebra, and so do $k \times k$ matrices. 
A \emph{grading}
\footnote{
    We define it in the \hyperref[chap:intro]{Introduction}, \cref{sec:grds-and-sa}.
} 
on an algebra is a choice of ``building blocks'' in it such that
\begin{enumerate}
    \item Every element of the algebra can be expressed as a sum of these building blocks;
    \item To each building block is associated a \emph{degree}, in a way that the product of building blocks of degrees $m$ and $n$ is either zero or a building block of degree $m+n$.
\end{enumerate}
We will not impose that the degree is an integer. 
It could be, for example, an element of the set $\ZZ_2 = \{ \bar 0, \bar 1 \}$ with addition given by 
% \begin{itemize}
%     \item $\bar 0 + \bar 0 = \bar 0$;
%     \item $\bar 0 + \bar 1 = \bar 1$, 
%     \item $\bar 1 + \bar 0 = \bar 1$; 
%     \item $\bar 1 + \bar 1 = \bar 1$. 
% \end{itemize}
$\bar 0 + \bar 0 = \bar 0$, 
$\bar 0 + \bar 1 = \bar 1$, 
$\bar 1 + \bar 0 = \bar 1$ and 
$\bar 1 + \bar 1 = \bar 1$. 
One way of thinking about $\ZZ_2$ is that $\bar 0$ means ``even'' and $\bar 1$ means ``odd'', so the addition rules are nothing but the usual behaviour of even and odd numbers under addition. 

One example of an algebra with degrees in $\ZZ_2$ is the set of complex numbers. 
A complex number is a sum $a + b\bi$ where $a$ is a real number, $b$ is a real number and $\bi$ is a symbol subject to the rule $\bi^2 = -1$. 
We can take as building blocks the elements of the form $a + 0\bi$ (real numbers), to which we assign degree $\bar 0$, and the elements of the form $0 + b\bi$ (purely imaginary numbers), to which we assign degree $\bar 1$. 
It is easy to see that this satisfies conditions 1 and 2 above. 

More generally, we will consider the degrees to be elements of a \emph{group}\footnote{Again, we refer to \cite[Chapter 1]{MR3932087} for a precise definition.}, which is a set with only one operation. 

Some important algebras arising in Mathematics and Physics have a natural grading with degrees in $\ZZ_2$; they are called \emph{superalgebras}. 
This work is about the \emph{additional} gradings that we can put on certain superalgebras, namely, \emph{classical Lie superalgebras}. 
These superalgebras can be modeled using matrices with the operation of \emph{supercommutator}\footnote{\cref{defi:supercommutator}.} instead of the usual product. 
Note that matrices, unlike polynomials, do not have canonical building blocks and degrees, but there are many natural and interesting choices that we can make. 
In this work we classify all possible choices. 

%%
%%
%%
%%%%%%%%%%%%%%%%%%%%%%%%%%%%%%%%%%%%%%%%%%%%%
%%   Acknowledgements
%%   It can be in singular on in plural
%\acknowledgement%
% or
\acknowledgements%

Given that this thesis is the final requirement for the highest academic degree, I first want to thank all teachers and professors I had during my long educational journey. 
I thank my parents, Marta De Naday and Nelson Hornhardt, for everything they have done for me. 
I am grateful to the tax payers of Brazil and Canada, specially of the state of S\~ao Paulo and the province of Newfoundland and Labrador, for financially supporting not only the universities I studied in, but also myself, through scholarships. 
I want to thank Memorial University of Newfoundland for all the financial, academic and extra-academic support, and also the Natural Sciences and Engineering Research Council of Canada for funding my research. 
Finally, I thank Yuri Bahturin, for suggesting the topic and working with me at the beginning of my program, and Mikhail Kochetov, for his active approach to supervising over all these years. 


%%
%%%%%%%%%%%%%%%%%%%%%%%%%%%%%%%%%%%%%%%%%%%%%%
%%   Statement of contributions
%%   It can be in singular on in plural
%\contributions%
\contribution% 

In this thesis we present a classification of group gradings on the classical Lie superalgebras of series $A, B, C, D, P \AND Q$, except type $A(1,1)$, over an algebraically closed field $\FF$ of characteristic~$0$ (see \cref{thm:last-one-for-P,thm:grds-osp-final,thm:final-m-not-n,thm:final-Q(n),thm:final-A(n-n)}). 
To this end, we also classify group gradings on the finite dimensional superinvolution-simple associative superalgebras over an algebraically closed field of characteristic different from $2$ (see \cref{thm:osp-and-p-associative,thm:MxM-type-I,thm:QxQ-type-I,thm:MxM-even,thm:MxM-odd}). 
%  \cref{chap:grds-sinv-simple}

\noindent
In previous works (together with Helen Samara Dos Santos and Mikhail Kochetov in \cite{paper-MAP}, and with the same coauthors and Yuri Bahturin in \cite{paper-Qn}), we already gave a complete classification of group gradings on the Lie superalgebras of series $P \AND Q$ and a partial classification (Type I gradings) for series $A$. 
Also, the Ph.D. thesis \cite{Helens_thesis} of Helen Samara Dos Santos includes a classification of group gradings on the Lie superalgebras of series $B$. 
Nevertheless, the techniques developed in this work, in collaboration with Mikhail Kochetov, allow us to give a complete classification for all series  $A, B, C, D, P \AND Q$ in a uniform fashion. 
% The correspondence between the different classification is explained. 

\noindent 
For some important results (\cref{thm:vphi-iff-vphi0-and-B,thm:vphi-involution-iff-delta-pm-1,cor:SxSsop-with-dcc}), no assumptions on the base field are needed, and the conditions of finite dimensionality and (superinvolution-)simplicity of the superalgebra is weakened to the descending chain condition on graded one-side superideals and graded-(superinvolution-)simplicity. 

\noindent
Also, we present some known results in greater generality than found in the literature: 
in \cref{sec:g-hat-action}, we develop the correspondence between $G$-gradings and $\widehat G$-actions (see, for example, \cite[Section 1.4]{livromicha}) for universal algebras (assuming the base field is algebraically closed of characteristic $0$ and $G$ is a finitely generated abelian group); 
in \cref{sec:classification-grd-simple-with-sinv}, we give a classification of finite dimensional graded-simple (rather than simple as algebras, as was assumed in \cite{paper-MAP}) associative superalgebras over an algebraically closed field in terms of the group $G$. 



%%%%%%%%%%%%%%%%%%%%%%%%%%%%%%%%%%%%%%%%%%%%%%
\tableofcontents%
% \listoftables    % if apply, otherwise comment
% \listoffigures   % if apply, otherwise comment
%%
%%%%%%%%%%%%%%%%%%%%%%%%%%%%%%%%%%%%%%%%%%%%%%
%%   List of symbos is optional
%%   It is like a two column tabular environment
%
% \begin{symbols}
% $\dnor{\mu}{\sigma^2}$ & %
%     normal distribution with mean $\mu$ and variance $\sigma^2$\\
% $\real{}$ & real numbers\\
% \end{symbols}
%%%%%%%%%%%%%%%%%%%%%%%%%%%%%%%%%%%%%%%%%%%%%%
%%   List of abbreviations is optional
%%   Also, it is like a two column tabular environment
%
% \begin{abbreviations}
% ISO & International Organization for Standardization\\
% PDF & Portable Document Format\\
% PDF/A & ISO-standardized version of PDF specialized for digital archiving.
% \end{abbreviations}
\end{prefatory}
%%%% End of 'prefatory' section
%
%%%%%%%%%%%%%%%%%%%%%%%%%%%%%%%%%%%%%%%%%%%%%
%%
%%  '\doublespacing' should only be used for review/draft
%\doublespacing  %% comment this command for the final submission
%%
%%
%%%%%%%%