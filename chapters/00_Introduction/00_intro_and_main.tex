\chapter*{Introduction}\label{chap:intro}

The use of group gradings in Lie theory can be traced back to 1888 (see \cite{MR1510529}), when W.~Killing introduced the root space decomposition of a complex semisimple Lie algebra $L$, which gives us a $\ZZ^n$-grading, where $n$ is the rank of $L$.  
Later, $\ZZ_2$-gradings appeared in the work of E.~Cartan on real semisimple Lie algebras (see \cite{Cartan-1914}). 
The interest in gradings increased in the 1960's, in connection with the works of J.~Tits, I.L.~Kantor, and M.~Koecher (see \cite{Tit62,Kan64,Koe67}). 
V.~Kac classified gradings by cyclic groups on complex semisimple Lie algebras and used them in the theory of symmetric spaces in differential geometry in \cite{Kac68}, and later for the construction of the so called twisted loop algebras, which are fundamental for the theory of the famous affine Kac--Moody Lie algebras (see, \eg, \cite{Kac90}). 
A systematic classification of group gradings on Lie algebras started with \cite{PZ} and remains an active area in the theory of Lie algebras and their representations.

The first appearances of Lie superalgebras were related to cohomology (see \cite{FN56,Gen63,Gen64,MM65}), but they were independently introduced in physics, in relation to the so called supersymmetries (see \cite{GN64,Miy68,Mic69}). 
They became a mainstream topic in theoretical physics with the development of string theory in the 1970's. 

In the present work, our main goal is to classify, up to isomorphism, the group gradings (see \cref{defi:grd-sa}) on nonexceptional classical Lie superalgebras (see \cref{sec:defi-classical-SA}) over an algebraically closed field of characteristic $0$.  
To achieve this in \cref{chap:Lie}, we first classify group gradings on finite dimensional simple (\cref{chap:grd-simple-assc}) and superinvolution-simple associative superalgebras (\cref{chap:super-inv,chap:grds-sinv-simple}). 
In this Introduction and in \cref{sec:generalities}, we give the basic definitions and tools we will use throughout the thesis. 

% A POSSIBLE APPLICATIONS:

% \begin{itemize}
%     \item Quantum numbers, see \cite{Helens_thesis};
%     \item Another motivation is that the classification problem for Lie colored superalgebras can be reduced to the classification of the gradings on Lie superalgebras, see \cite{MR2497949} and also [Sch79], BM99 and \cite[Theorem 0.10]{livromicha}
%     \item Graded identities, see \cite{MR3611462};
%     \item Graded contractions (compare with \cite{livromicha})
% \end{itemize}


\section{Gradings and superalgebras}\label{sec:grds-and-sa}

The main concepts in this work are group gradings and superalgebras, so let us define these first.
Gradings are usually defined for algebras, but it is useful to consider, more generally, gradings on vector spaces. 
All vector spaces under consideration will be over a fixed field $\FF$. 

\begin{defi}\label{defi:grading}
    Let $G$ be a group. 
	A $G$-\emph{grading on a vector space} $V$ is a direct sum decomposition
	\[\label{eq:grading}
	    \Gamma : V= \bigoplus_{g \in G} V_g,
	\]
	indexed by the elements of $G$.
	When endowed with a fixed $G$-grading $\Gamma$, $V$ will be called a \emph{$G$-graded vector space}. 
\end{defi}

For each $g\in G$, the subspace $V_g$ is called the \emph{homogeneous component of degree $g$}. 
An element $v \in V$ is said to be \emph{homogeneous} if it belongs to a homogeneous component. 
Clearly, a nonzero homogeneous element $v$ belongs to a unique component $V_g$ and, in this case, we say that $g$ is the \emph{degree} of $v$ and write $\deg v = g$. 
Whenever we refer to the degree of an element, we will assume that it is a nonzero homogeneous element. 
\phantomsection
\label{homogeneousMap} Given $G$-graded spaces $V$ and $W$, we say that a linear map $\psi\from V \to W$ is \emph{degree-preserving} or a \emph{homomorphism of graded vector spaces}
if $\psi(V_g) \subseteq W_g$. 
A subspace $U\subseteq V$ is said to be a \emph{graded subspace} if $U = \bigoplus_{g\in G} (U\cap V_g)$, \ie, if every element in $U$ is a sum of homogeneous elements in $U$. 

% defi: grading
\begin{defi}\label{def:grading}
	Let $G$ be a group. 
	A $G$-\emph{grading on an algebra} $A$ is 
% 	a vector space decomposition indexed by the elements of $G$
    a grading 
	$
	    \Gamma : A= \bigoplus_{g \in G} A_g
	$ on the vector space underlying $A$
	such that 
	\[
	    \forall g, h\in G, \quad
	    A_g A_h \subseteq A_{gh}.
	   % \forall a \in A_g, \forall b\in A_h, \quad ab \in A_{gh}.
	\]
% 	$ab \in A_{gh}$, for all $g,h \in G$, $a \in A_{g}$ and $b \in A_{h}$. 
	When endowed with a fixed $G$-grading $\Gamma$, $A$ is called a \emph{$G$-graded algebra}. 
% 	Given a grading $\Gamma$ on $A$, we say that the pair $(A, \Gamma)$ is a \emph{graded algebra}. 
\end{defi}

A \emph{homomorphism of $G$-graded algebras} is a homomorphism of algebras that is also a degree-preserving map. 
If $V$ is a finite dimensional graded vector space, then $\End(V)$ can be considered as a $G$-graded algebra by setting 
\[\label{eq:elementary-1st}
    \End(V)_g \coloneqq \{ T\in \End(V) \mid \forall h\in G,\, T(V_h) \subseteq V_{gh} \}.
\]
(Compare with \cref{defi:elementary-grd-abstract,defi:elementary-grd-matrix}).
Note that $\End(V)$ consists of all linear maps and $\End(V)_e$ consists of the degree-preserving linear maps.

% spaces $V = \bigoplus_{g \in G} V_g$ and $W = \bigoplus_{g \in G} W_g$ is a linear map $\psi\from V \to W$ such that $\psi(V_g) \subseteq W_g$. 

% Elementary grading, abstract

% If $U$ and $V$ are graded vector spaces and $T\from U \to V$ is a linear map, we say that $T$ is \emph{homogeneous of degree $g$}, for some $g \in G$, if 

% (see more in ...)

% defi: graded algebra and superalgebra
\begin{defi}\label{def:superalgebra}
    A \emph{super vector space} or \emph{superspace} is a $\ZZ_2$-graded vector space. 
    A \emph{superalgebra} is a $\ZZ_2$-graded algebra. 
\end{defi}

% superscripts and names even, odd
The $\ZZ_2$-grading in \cref{def:superalgebra} will be called the \emph{canonical $\ZZ_2$-grading}.
We will index the homogeneous components of the canonical $\ZZ_2$-grading by superscripts, \ie, for a superspace $V$, we will write \cref{eq:grading} as $V = V\even \oplus V\odd$. 
The elements of $V\even$ are said to be \emph{even} and the elements of $V\odd$ are said to be \emph{odd}; we may use the word ``\emph{parity}'' instead of ``degree'' and write $|v|$ instead of $\deg v$. 
Also, in this situation, we refer to graded subspaces as \emph{subsuperspaces}. 
The subalgebras of a superalgebra that are also subsuperspaces are called \emph{subsuperalgebras}. 
This special notation and nomenclature will serve to distinguish the canonical $\ZZ_2$-grading from an additional group grading we may consider on $A$:  

\begin{defi}\label{defi:grd-sa}
    Let $G$ be a group. 
    A \emph{$G$-grading on a superalgebra} $A = A\even \oplus A\odd$ is a $G$-grading $\Gamma: A = \bigoplus_{g\in G} A_g$ on the algebra underlying $A$ such that all homogeneous components $A_g$ are subsuperspaces or, equivalently, $A\even$ and $A\odd$ are $G$-graded subspaces. 
    % \[
    %     \forall g\in G, \quad A_g = (A_g \cap A\even) \oplus (A_g \cap A\odd).
    % \]
    When endowed with a fixed grading $\Gamma$, $A$ is called a \emph{$G$-graded superalgebra}. 
    Two $G$-gradings, $\Gamma$ and $\Gamma'$, on a superalgebra $A$ are said to be \emph{isomorphic} if $(A, \Gamma)$ and $(A, \Gamma')$ are isomorphic as graded superalgebras. 
\end{defi}

\phantomsection\label{defi:G-sharp}
% Def of $G^\#$
Given a $G$-graded superalgebra $A$, we can combine the $G$-grading with the canonical $\ZZ_2$-grading:
we set $G^\#\coloneqq G\times \ZZ_2$ and define a $G^\#$-grading on the algebra underlying $A$ by setting $A_{(g,i)} \coloneqq A_g \cap A^i$, for all $g\in G$ and $i\in \ZZ_2$. 
Conversely, it is clear that any $G^\#$-graded algebra can be seen as a $G$-graded superalgebra. 

\section{Varieties of superalgebras}\label{sec:Grassmann}

Now we are going to define the variety of superalgebras we are mainly interested in:

\begin{defi}\label{def:Lie-sa}
	A superalgebra $L=L\even\oplus L\odd$, with product denoted by $[\cdot, \cdot]\from L\times L \to L$, is said to be a \emph{Lie superalgebra} if, for all nonzero homogeneous elements $a, b, c \in L$, we have\footnote{Some authors require additional conditions if $\Char \FF = 2$ or $3$ (\eg, \cite[Subsection 1.2]{MR1192546}).}:
	%
	\begin{enumerate}[(i)]
		\item $[ a, b ] = - \sign{a}{b} [b, a]$ (\emph{super-anti-commutativity});
		\item $[a,[b,c]] = [[a,b],c] + \sign{a}{b} [b, [a,c]]$ (\emph{super Jacoby identity}).
	\end{enumerate}
\end{defi}

Note that if $L\even = L$, then we have the usual definition of a Lie algebra. 

It is easier to define the variety of associative superalgebras:

\begin{defi}\label{def:associative-sa}
    A superalgebra $R$ is said to be an \emph{associative superalgebra} if the algebra underlying $R$ is associative. 
\end{defi}

The reader may be curious why we introduce signs in \cref{def:Lie-sa} but not in \cref{def:associative-sa}. 
This has to do with the so called \emph{rule of signs}. 
Roughly speaking, every time two elements $a$ and $b$ exchange positions in a product in one of the identities that define the variety, we have to introduce the sign $\sign{a}{b}$. 
To make this more precise, we will use the notion of Grassmann envelope.\footnote{Another framework in which the rule of signs can be formulated is the symmetric monoidal category of super vector spaces (see, \eg, \cite[Chapter 3]{MR2069561}).} 

\begin{defi}\label{def:Grassmann-algebra}
	Let $V$ be a vector space. 
	The \emph{exterior} or \emph{Grassmann superalgebra} of $V$ is the algebra
	%
	\[\mc G(V) = \bigoplus_{k=0}^{+\infty} \Exterior^k V,\]
	%
	with product given by $\wedge$ and $\ZZ_2$-grading given by 
	\[
	    \mc G(V)\even = \bigoplus_{i=0}^{+\infty}\Exterior^{2i} V \,\AND\, \mc G(V)\odd = \bigoplus_{i=0}^{+\infty}\Exterior^{2i+1} V.
	\]
\end{defi}

% Given a vector space $V$, the \emph{Grassmann superalgebra} of $V$ is the algebra
% \[
%     \mc G(V) = \sum_{k=0}^{+\infty} \Exterior^k V,
% \]
% with product given by the ``$\wedge$'' operation, endowed with the $\ZZ_2$-grading given by 
% \[
%     \mc G(V)\even = \sum_{i=0}^{+\infty}\Exterior^{2i} V \,\AND\, \mc G(V)\odd = \sum_{i=0}^{+\infty}\Exterior^{2i+1} V.
% \]
Note that, if $a,b \in \mc G(V)$ are homogeneous elements, then $a \wedge b = \sign{a}{b} b \wedge a$. 
Superalgebras with this property are called \emph{(super)commutative}. 

\begin{defi}\label{Grassmann-envelope}
    Let $V$ be a fixed vector space with a countably infinite basis. 
    Given a superalgebra $A = A\even \oplus A\odd$, we define the \emph{Grassmann envelope} of $A$ to be the algebra $(A\even\tensor \mc G(V)\even) \oplus (A\odd\tensor \mc G(V)\odd)$. 
    If $\mathfrak{V}$ is any class of algebras, we say that $A$ is a \emph{$\mathfrak{V}$-superalgebra} if the Grassmann envelope of $A$ belongs to $\mathfrak{V}$. 
\end{defi}

One can easily see that, if $\mathfrak{V}$ is the class of all Lie (respectively associative, commutative) algebras, then the $\mathfrak{V}$-superalgebras are precisely the Lie (respectively associative, commutative) superalgebras. 
This approach can be used to define other varieties of superalgebras (\eg, Jordan). 

Another kind of object that plays a major role in this work is associative superalgebras with superinvolution:

\begin{defi}\label{def:super-anti-automorphism}
    Let $A = A\even \oplus A\odd$ be a superalgebra. 
    We say that a bijective linear map $\vphi\from A \to A$ is a \emph{super-anti-automorphism} if $\vphi(A\even) = A\even$, $\vphi(A\odd) = A\odd$ and
    \[\label{eq:super-anti-automorphism}
        \forall a,b \in A\even \cup A\odd, \quad \vphi(ab) = \sign{a}{b} \vphi(b)\vphi(a).
    \]
    If, further, $\vphi^2 = \id$, we say that $\vphi$ is a \emph{superinvolution}. 
\end{defi}

For example, the identity map $\id\from \mc G(V) \to \mc G(V)$ is a superinvolution. 
For comparison with \cref{Grassmann-envelope}, note that a parity-preserving bijective linear map $\vphi\from A \to A$ satisfies \cref{eq:super-anti-automorphism} \IFF the map $\vphi \tensor \id \from (A\even\tensor \mc G(V)\even) \oplus (A\odd\tensor \mc G(V)\odd) \to (A\even\tensor \mc G(V)\even) \oplus (A\odd\tensor \mc G(V)\odd)$ is an antiautomorphism. 

\begin{defi}\label{defi:grd-superinv}
    Let $(R, \vphi)$ be a superalgebra $R$ endowed with a super\--anti\--auto\-mor\-phism $\vphi$. 
    A $G$-grading on $(R, \vphi)$ is a $G$-grading $\Gamma : R = \bigoplus_{g \in G} R_g$ on the superalgebra $R$ such that $\vphi(R_g) = R_g$, for all $g\in G$. 
\end{defi}

We can always get a Lie superalgebra from an associative one by considering the supercommutator:

\begin{defi}\label{defi:supercommutator}
    Let $R$ be an associative superalgebra. 
    We define the \emph{supercommutator} $[\cdot, \cdot]\from R \to R$ to be the bilinear map such that
    \[
        \forall a,b\in R\even \cup R\odd, \quad [a,b] = ab - \sign{a}{b} ba.
    \]
    The superalgebra $R^{(-)}$ is defined to be the superspace $R$ endowed with the product $[\cdot, \cdot]$. 
    If $\vphi\from R\to R$ is a super-anti-automorphism, then we define
    \[
        \Skew(R, \vphi) \coloneqq \{a \in R^{(-)} \mid \vphi(a) = -a \}.
    \]
\end{defi}

It is straightforward to check that $R^{(-)}$ is a Lie superalgebra and that $\Skew(R, \vphi)$ is a subsuperalgebra of $R^{(-)}$. 
If $G$ is an abelian group, then a $G$-grading on $R$ is also a $G$-grading on $R^{(-)}$. 
Moreover, if $(R, \vphi)$ is $G$-graded, then $\Skew (R, \vphi)$ is a graded subsuperalgebra of $R^{(-)}$. 

% ----
\section{Simple superalgebras}\label{sec:simple-algebras}

\begin{defi}
    Let $A = A\even \oplus A\odd$ be a superalgebra. 
    A \emph{superideal} of $A$ is an ideal $I \subseteq A$ that is also a subsuperspace. 
    We say that $A$ is a  \emph{simple superalgebra} if $A\cdot A \neq 0$ and the only superideals are $0$ and $A$.
\end{defi}


% \begin{defi}
%     Let $A = \bigoplus_{g\in G} A_g$ be a graded algebra. 
%     A \emph{graded ideal} is an ideal $I \subseteq A$ such that $I = \bigoplus_{g\in G} (I \cap A_g)$. 
%     We say that $A$ is \emph{graded-simple} if $A\cdot A \neq 0$ and the only graded ideals in $A$ are $0$ and $A$.
% \end{defi}

% \begin{defi}
%     A superalgebra is said to be \emph{simple} if it is simple as a $\ZZ_2$-graded algebra. 
%     A graded superalgebra is said to be \emph{graded-simple} if it is graded as
% \end{defi}

We are mainly interested in simple Lie superalgebras in this work, but many of them are closely related to simple associative superalgebras, so we start with these. 
If $\FF$ is algebraically closed, the finite dimensional simple associative superalgebras are:
%
\begin{itemize}
    \item the matrix superalgebras $M(m,n)$;
    
    \item the queer superalgebras $Q(n)$.
\end{itemize}

This is a well known result, which we obtain in \cref{chap:grd-simple-assc} as a special case of the classification of graded-simple associative algebras (see \cref{thm:fd-simple-SA}). 
The definitions are given in \cref{subsec:simple-associative}, below. 

If $\FF$ is algebraically closed and $\Char \FF = 0$, 
the simple finite dimensional Lie superalgebras (that are not Lie algebras) were classified by V. G. Kac (see \cite{artigokac} and \cite{livrosuperalgebra}). 
They are divided into two big classes according to the action of $L\even$ on $L\odd$ (note that, for any Lie superalgebra $L=L\even \oplus L\odd$, $L\even$ is a Lie algebra and $L\odd$ is an $L\even$-module). 

\begin{defi}
	Let $L=L\even \oplus L\odd$ be a simple Lie superalgebra.
	\begin{enumerate}[(i)]
		\item We say that $L$ is \emph{classical} if $L\odd$ is a semisimple $L\even$-module.
		\item We say that $L$ is of \emph{Cartan type} if $L\odd$ has a nonzero largest proper submodule, \ie, a proper submodule that contains all proper submodules.
	\end{enumerate}
\end{defi}

Every simple finite dimensional Lie superalgebra (that is not a Lie algebra) is either classical or of Cartan type. 
The classical ones are, in their turn, divided as follows:

\begin{itemize}
	\item 4 series, $A(m,n)$, $B(m,n)$, $C(n)$, $D(m,n)$, which are analogous to the corresponding series of simple Lie algebras;

	\item 2 series, $P(n)$ and $Q(n)$, which are called the \emph{strange Lie superalgebras};

	\item 3 exceptional cases: $F(4)$, $G(3)$ and the family $D(2,1,\alpha)$, $\alpha \in \mathbb{F}\setminus \{0,-1\}$.
\end{itemize}

The Lie superalgebras of Cartan type are analogs of the corresponding simple (infinite dimensional) Lie algebras of Cartan type, as well as the simple restricted Lie algebras of Cartan type.
They are divided in the following series:

\begin{itemize}
	\item the Witt superalgebras $W(n)$;
	\item the special superalgebras $S(n)$ and their deformations $\tilde S(n)$;
	\item the Hamiltonian superalgebras $H(n)$.
\end{itemize}

It is important to mention that there are restrictions on the parameters $m$ and $n$ above. 
We are going to make them explicit when we give the corresponding definitions in \cref{sec:defi-classical-SA,subsec:Cartan}. 
Also note that the symbol $Q(n)$ denotes both an associative superalgebra and a Lie superalgebra; we will explicitly use the words ``associative'' and ``Lie'' if there is a chance of confusion. 

To define some of the simple Lie superalgebras listed above we will need superinvolutions on simple associative superalgebras (which will be described in \cref{subsec:simple-associative}) and also the following concepts:

\begin{defi}
    Let $L$ be a Lie superalgebra. 
    The \emph{(super)center} of $L$ is the superideal $Z(L) \coloneqq \{ x\in L \mid [x, L] = 0\}$. 
    The \emph{derived superalgebra} of $L$ is $L^{(1)} \coloneqq [L, L]$. 
    In the case $L = R^{(-)}$ for an associative superalgebra $R$, we may also denote $L^{(1)}$ by $R^{(1)}$. 
\end{defi}

The notions of simplicity for superalgebras endowed with a super-anti-automorphism and/or a grading will play a major role in this work:

\begin{defi}\label{defi:superinvolution-simple}
    Let $A = A\even \oplus A\odd$ be a superalgebra endowed with a super-anti-automorphism $\vphi\from A \to A$. 
    A superideal $I$ is said to be \emph{$\vphi$-invariant} if $\vphi(I) \subseteq I$. 
    We say that $A$ is \emph{simple as a superalgebra with super-anti-automorphism} if $A\cdot A \neq 0$ and the only $\vphi$-invariant superideals in $A$ are $0$ and $A$. 
    In the case $\vphi$ is a superinvolution, we say that $A$ is \emph{superinvolution-simple}.
\end{defi}

The classification of superinvolution-simple associative superalgebras is well known (see, \eg, \cite{racine}) and will be proved in \cref{sec:sinv-simple} as a special case of the theory we develop in this work. 

\begin{defi}\label{defi:graded-simple}
    Let $A$ be a $G$-graded superalgebra. 
    A \emph{graded superideal} is a superideal that is also a $G$-graded subspace. 
    We say that $A$ is a  \emph{graded-simple superalgebra} if $A\cdot A \neq 0$ and the only graded superideals in $A$ are $0$ and $A$. 
    If $A$ is endowed with a super-anti-automorphism $\vphi\from A \to A$, we say that $A$ is \emph{simple as a graded superalgebra with super-anti-automorphism} if the only $\vphi$-invariant graded superideals in $A$ are $0$ and $A$. 
    In the case $\vphi$ is a superinvolution, we say that $A$ is \emph{graded-superinvolution-simple}.
\end{defi}

\subsection{Simple associative superalgebras}\label{subsec:simple-associative}

\subsubsection{The series \texorpdfstring{$M(m,n)$}{M(m,n)}}

Let $m, n \geq 0$ be integers that are not both zero.
The \emph{matrix superalgebra} $M(m,n)$ is defined to be the matrix algebra $M_{m+n}(\FF)$ endowed with the following $\ZZ_2$-grading:
%
\begin{align}
    M(m,n)\even &\coloneqq \left\{ \left(\begin{array}{c|c}
        A & 0\\
        \hline
        0 & D
    \end{array}\right)
    \mid A\in M_{m}(\FF),\, D\in M_n(\FF) \right\},\\
    %
    M(m,n)\odd &\coloneqq \left\{
    \left(\begin{array}{c|c}
        0 & B\\
        \hline
        C & 0
    \end{array}\right)
    \mid B \in M_{m\times n}(\FF), \, D\in M_{n\times m}(\FF) \right\}.
\end{align}
%
\phantomsection\label{def:grd-superspace-canonical}

Note that this is nothing but the matrix representation of the superalgebra $\End(\FF^{m|n})$, as in \cref{eq:elementary-1st}, where $\FF^{m|n}$ is defined to be the superspace $V = V\even \oplus V\odd$ with $V\even \coloneqq \FF^m$ and $V\odd \coloneqq \FF^n$. 

% As an alternative definition, we first set $\FF^{m|n}$ to be the superspace $V = V\even \oplus V\odd$ where $V\even \coloneqq \FF^m$ and $V\odd \coloneqq \FF^n$; then, we define $M(m,n)$ to be $\End(\FF^{m|n})$ endowed with the $\ZZ_2$-grading given by
% %
% \begin{align}
%     \End(\FF^{m|n})\even &\coloneqq \{ T\in \End(\FF^{m|n}) \mid  T(V\even) \subseteq V\even \,\AND\, T(V\odd) \subseteq V\odd\},\\
%     %
%     \End(\FF^{m|n})\odd &\coloneqq \{ T\in \End(\FF^{m|n}) \mid T(V\even) \subseteq V\odd \,\AND\, T(V\odd) \subseteq V\even\}.
% \end{align}
% 
Clearly, $M(m,n)$ is simple. 
We note that $M(m,n) \iso M(m',n')$ \IFF $m=m'$ and $n=n'$, or $m=n'$ and $n=m'$ (see \cref{thm:fd-simple-SA}). 

There is an important super-anti-automorphism on $M(m,n)$:

\begin{defi}
    We define the \emph{supertranspose} of a matrix in $M(m,n)$ to be
    \[
        \left(\begin{array}{c|c}
            A & B\\
            \hline
            C & D
        \end{array}\right)\stransp \coloneqq
        \left(\begin{array}{c|c}
            A\transp & -C\transp\\
            \hline
            B\transp & \phantom{-}D\transp
        \end{array}\right).
    \]
\end{defi}

Note that, if $\Char \FF \neq 2$, the map $X \mapsto X\stransp$ is not a superinvolution, it has order~$4$. 
We also note that some authors define supertranspose differently, by putting the negative sign in the bottom left block (see, \eg, \cite[Subsection 1.1.2]{MR3012224}). 

If there is an element $\bi\in \FF$ such that $\bi^2 = -1$, there is a variation of supertranspose that will be useful when working with $Q(n)$ and $A(n,n)$:

\begin{defi}\label{def:queer-stp}
    We define the \emph{queer supertranspose} of a matrix in $M(m,n)$ to be
    \[
        \left(\begin{array}{c|c}
            A & B\\
            \hline
            C & D
        \end{array}\right)\sTq \coloneqq
        \left(\begin{array}{c|c}
            \phantom{\bi}A\transp & \bi C\transp\\
            \hline
            \bi B\transp & \phantom{\bi} D\transp
        \end{array}\right).
    \]
\end{defi}

Superinvolutions on $M(m,n)$ only exist for certain values of $m$ and $n$: either $m=n$ or at least one of them is even (see, \eg, \cref{prop:iso-M-with-vphi}). 
They can be described in matrix terms (see \cref{defi:M(m-n-p_0)}), but it is worth describing them more abstractly, in the model $M(m,n) = \End(\FF^{m|n})$. 
To simplify notation, we will write $V$ for $\FF^{m|n}$, as above. 

\begin{defi}
Let $\langle \cdot, \cdot \rangle \from V\times V \to \FF$ be a bilinear form. 
We say that $\langle \cdot, \cdot \rangle$ is  \emph{supersymmetric} if
\[
    \forall u,v \in V\even \cup V\odd, \quad \langle u, v \rangle 
    %
    = \sign{u}{v} \langle v, u \rangle,
\]
and \emph{super-skew-symmetric} if
\[
    \forall u,v \in V\even \cup V\odd, \quad \langle u, v \rangle 
    %
    = -\sign{u}{v} \langle v, u \rangle.
\]
Also, we say that $\langle \cdot, \cdot \rangle$ is \emph{even} if $\langle V\even, V\odd \rangle = \langle V\odd, V\even \rangle = 0$, \emph{odd} if $\langle V\even, V\even \rangle = \langle V\odd, V\odd \rangle = 0$, and \emph{homogeneous} if it is either even or odd. 
If $\langle \cdot, \cdot \rangle$ is homogeneous and nondegenerate, we define the \emph{superadjunction} to be the unique linear map $\vphi\from M(m,n) \to M(m,n)$ such that 
\[
    \forall u,v \in V\even \cup V\odd,\,
    %
    \forall T\in M(m,n)\even \cup M(m,n)\odd, \quad
    %
    \langle T(u), v \rangle = \sign{T}{u} \langle u, \vphi(T)(v) \rangle. 
\]
\end{defi}

The superadjunction is always a super-anti-automorphism, and it is a superinvolution \IFF $\langle \cdot, \cdot \rangle$ is supersymmetric or super-skew-symmetric (see, \eg, \cref{thm:vphi-involution-iff-delta-pm-1}). 
In fact, all super-anti-automorphisms on $M(m,n)$ arise in this way. 
We note that using the isomorphism $M(m,n) \iso M(n,m)$ if necessary, we can avoid super-skew-symmetric bilinear forms altogether. 
This is because if we exchange the degrees of the components $V\even$ and $V\odd$ (\ie, change $\FF^{m|n}$ to $\FF^{n|m}$), a super-skew-symmetric form becomes supersymmetric. 

\subsubsection{The (associative) series \texorpdfstring{$Q(n)$}{Q(n)}}

Let $n > 0$ be an integer. 
The \emph{queer associative superalgebra} $Q(n)$ is the subsuperalgebra of $M(n,n)$ defined by
\[
    Q(n) \coloneqq \left\{ \left(\begin{array}{c|c}
        A & B\\
        \hline
        B & A
    \end{array}\right)
    \mid A,B\in M_{n}(\FF)
    \right\}.
\]

% Differently from $M(m,n)$, 
While $Q(n)$ is not simple as an algebra. 
Also, $Q(n) \iso Q(n')$ \IFF $n=n'$ (see, \eg, \cref{thm:fd-simple-SA}). 

The queer supertranspose on $M(n,n)$ restricts to $Q(n)$, but, if $\Char \FF \neq 2$,
$Q(n)$ does not admit any superinvolution (see, \eg, \cref{cor:Q-no-sinv-center}). 

% \section{Simple Lie superalgebras}

% An \emph{ideal} $I$ of a Lie superalgebra $L=L\even\oplus L\odd$ is an ideal of the underlying algebra that is compatible with the canonical $\Zmod2$-grading in the sense that $I=\left(I\cap A\even \right) \oplus \left(I\cap A\odd \right)$. The superalgebra $L$ is said to be \emph{simple} if it has no proper nonzero ideals and $[L,L]\neq 0$.

\subsection{Classical Lie superalgebras}\label{sec:defi-classical-SA}

\subsubsection{The series $A$} 

Let $m,n > 0$. 
The \emph{general linear Lie superalgebra} $\gl(m|n)$ is defined to be $M(m,n)^{(-)}$. 
The \emph{special linear Lie superalgebra} $\Sl (m|n)$ is the derived superalgebra of $\gl(m|n)$. 
One can check that this is equivalent to
%
\[
    \Sl(m|n) = \left\{
    \left(\begin{array}{c|c}
        A & B\\
        \hline
        C & D
    \end{array}\right)
	\in \gl(m|n)
	\mid \tr A = \tr D
	\right\}.
\]
%
If $m\neq n$ then $\Sl(m|n)$ is a simple Lie superalgebra. 
However, if $m=n$, then
\[
    Z(\Sl(m|n)) = \FF 1 = \left\{
    \left(\begin{array}{c|c}
        \lambda I & 0\\
        \hline
        0 & \lambda I
    \end{array}\right)
	\mid \lambda \in \FF
	\right\}
\]
is a nontrivial superideal. 
The quotient $\Sl(n|n)/ \FF 1$ is a simple superalgebra \IFF $n > 1$. 

For $m,n\geq 0$, the Lie superalgebra $A(m,n)$ is defined to be $\Sl(m+1 \,|\, n+1)$ if $m\neq n$, and $\mathfrak{psl}(n+1 \,|\, n+1) \coloneqq \Sl(n+1 \,|\, n+1)/ \FF 1$ if $m=n$. 

Since $A(m,n)\iso A(n,m)$, one may impose $m\geq n$ to avoid repetition. 

\subsubsection{The series $B$, $C$ and $D$}

The \emph{orthosymplectic Lie superalgebra} $\osp(m|n)$ is defined to be $\Skew(M(m,n),\vphi)$, where $\vphi$ is the superadjunction with respect to an \emph{even} nondegenerate supersymmetric bilinear form. 
Since we are assuming that $\FF$ is algebraically closed and $\Char \FF = 0$, we do not need to specify the form: this is well defined up to isomorphism (see, \eg, \cref{prop:iso-M-with-vphi}). 

Since $\langle V\even, V\odd \rangle = 0$, any even nondegenerate supersymmetric bilinear form $\langle \cdot, \cdot \rangle$ restricts to a nondegenerate symmetric bilinear form on $V\even = \FF^m$ and to a a nondegenerate skew-symmetric bilinear form on $V\odd = \FF^n$. 
Hence the name ``orthosymplectic'': the superinvolution $\vphi$ is a hybrid between orthogonal and symplectic involutions on matrix algebras. 

\phantomsection\label{defi:B-C-D}

If $m,n > 0$, the Lie superalgebra $\osp(m|n)$ is simple. 
We define:
%
\begin{itemize}
	\item $B(m,n) \coloneqq \osp(2m+1 \,|\, 2n)$, for $m \geq 0$ and $n \geq 1$;
	\item $C(n) \coloneqq \osp(2 \,|\,  2n - 2)$, for $n\geq 2$;
	\item $D(m,n) \coloneqq \osp(2m \,|\, 2n)$, for $m\geq 2$ and $n\geq 1$.
\end{itemize}
%
Since $C(2)\iso A(1,0)$, one may impose $n\geq 3$ in the $C(n)$ case to avoid repetition.

\subsubsection{The series $P$}

The \emph{periplectic Lie superalgebra} $\mathfrak{p}(n)$ is defined to be $\Skew(M(n,n), \vphi)$, where $\vphi$ is the superinvolution with respect to an \emph{odd} nondegenerate supersymmetric bilinear form.
% \footnote{We note that there are no nondegenerate odd supersymmetric bilinear forms on $\FF^{m|n}$ if $m\neq n$.}

As in the orthosymplectic case, the isomorphism class of $\mathfrak{p}(n)$ does not depend on the choice of the bilinear form, but in this case it is easier to prove and does not depend on the hypothesis that $\FF$ is algebraically closed. 
Let $\langle \cdot, \cdot \rangle$ be an odd nondegenerate supersymmetric bilinear form on $V = \FF^{m|n}$. 
Since $\langle V\even, V\even \rangle = \langle V\odd, V\odd \rangle = 0$, we must have $V\odd \iso (V\even)^*$ via the map $v \mapsto \langle v, \cdot \rangle$, hence, in particular, $m = n$. 
Using this isomorphism to identify $V\odd$ with $(V\even)^*$, we see that $V$ is isomorphic to $V\even \oplus (V\even)^*$ endowed with the following form:
\[
    \forall u,v\in V\even, \forall u^*,v^*\in (V\even)^*, \quad \langle u+u^*,v + v^* \rangle \coloneqq u^*(v) + v^*(u). 
\]
Using the canonical basis of $V\even = \FF^n$, we can identify $(V\even)^* = (\FF^n)^*$ with $\FF^n$ and obtain:  
\[
    \mathfrak{p}(n) = \left\{
    \left(\begin{array}{c|c}
        A & B\\
        \hline
        \phantom{-}C\phantom{\transp} & -A\transp
    \end{array}\right) \in M(n,n)^{(-)} \mid B=B\transp \AND C=-C\transp\right\}. 
\]

The superalgebra $\mathfrak{p}(n)$ is not simple.
We define $P(n)$ to be the derived superalgebra of $\mathfrak{p}(n+1)$. 
In the model above, we have:
\[
    P(n) = \left\{
    \left(\begin{array}{c|c}
        A & B\\
        \hline
        \phantom{-}C\phantom{\transp} & -A\transp
    \end{array}\right) \in M(n+1,n+1)^{(-)} \mid \tr A =0, \, B=B\transp \AND C=-C\transp\right\}. 
\]

It is known that $P(n)$ is simple \IFF $n\geq 2$. 

\subsubsection{The (Lie) series $Q$}

Let $R$ denote the associative superalgebra $Q(n+1)$. 
Its derived Lie superalgebra is
\[
    R^{(1)} = \left\{
    \left(\begin{array}{c|c}
        A & B\\
        \hline
        B & A
    \end{array}\right) \in R^{(-)} \mid \tr A = \tr B = 0\right\}. 
\]
Since $Z(R^{(1)}) = \FF1$, we define the \emph{queer Lie superalgebra} $Q(n)$ to be $R^{(1)}/\FF1$. 
It is simple \IFF $n\geq 2$. 

\subsubsection{Exceptional Lie Superalgebras} 

Since defining $F(4)$, $G(3)$ and $D(2,1, \alpha)$ here would be a long detour, we refer the reader to \cite{artigokac} or \cite{MR1773773}. 
We do not consider them in this work except $D(2, 1, \alpha)$ for $\alpha \in \{1, -\frac{1}{2}, -2 \}$, which are isomorphic to $D(2,1)$. 

\subsection{Lie superalgebras of Cartan type}\label{subsec:Cartan}

\subsubsection{The series $W$}

Consider the Grassmann superalgebra $\mc G(\FF^n)$, as in \cref{def:Grassmann-algebra}. 
A homogeneous map $D\in \End(\mc G(\FF^n))$ is said to be a \emph{superderivation} if 
\[
    \forall a,b \in \mc{G}(\FF^n)\even \cup \mc{G}(\FF^n)\odd, \quad 
    D (a\wedge b)= D (a)\wedge b + \sign{D}{a} a\wedge D (b).
\]
One can check that, if $D_1$ and $D_2$ are superderivations, then the supercommutator $[D_1, D_2]$ is also a superderivation. 

We define the \emph{Witt superalgebra} $W(n)$ to be the linear span of the superderivations in $\End(\mc G(\FF^n))^{(-)}$. 
It is simple for $n\geq 2$, but $W(2)\iso A(1,0)\iso C(2)$, so we impose $n\geq 3$ in the list of simple Lie superalgebras of Cartan type. 

To see what $W(n)$ is in more concrete terms, let $\{e_1,\ldots,e_n\}$ be the canonical basis of $\FF^n$. 
For $i\in \{1, \ldots ,n\}$, we define $\partial_i \in W(n)$ to be the unique (odd) superderivation such that 
\[
    \forall j\in \{1,\ldots ,n\}, \quad 
    \partial_i (e_j) \coloneqq  \delta_{ij}.
\]
% Note that $|\partial_i| = \bar 1$. 
It is easy to see that every element of $W(n)$ is of the form $\sum_{i=1}^{n} f_i \wedge \partial_i$, where $f_i \in \mc G(\FF^n)$ for all $i\in\{1, \ldots , n\}$.

% The Lie superalgebra $W(n)$ is simple for all $n\geq 2$, but $W(2)\iso A(1,0)\iso C(2)$, hence we impose $n\geq 3$ in our list of simple Lie superalgebras.

\subsubsection{The series $S$}

We define the \emph{special superalgebra} $S(n)$ to be the subsuperalgebra of $W(n)$ given by
\[
	S(n) \coloneqq \left\{ \sum_{i=1}^{n} f_i \wedge \partial_i \in W(n) \mid
	\sum_{i=1}^{n} \partial_i (f_i) =0
	\right\}.
\]
$S(n)$ is simple for $n\geq 3$ but $S(3)\iso P(2)$, so we impose $n\geq 4$. 

\subsubsection{The series $\tilde S$}

Let $n$ be even and fix $\omega = 1 - e_1\wedge e_2\wedge \, ...\, \wedge e_n$. We define $\tilde S(n)$ by
\[
	\tilde S(n) = \left\{ \sum_{i=1}^{n} f_i \wedge \partial_i \in W(n) \mid
	\sum_{i=1}^{n} \partial_i (\omega \wedge f_i) =0
	\right\}.
\]
% It is good to notice that if we change the $\omega$ above by any other element in $W(n)$ the result would be isomorphic to $S(n)$ or $\tilde S(n)$. 
$\tilde S(n)$ is simple for $n\geq 4$.

\subsubsection{The series $H$}

Let us define another multiplication on $\mc G(\FF^n)$, the \emph{Poisson bracket}:
\[
    \forall f,g \in \mc G(\FF^n)\even \cup \mc G(\FF^n)\odd, \quad
	\{f,g\}=(-1)^{|f|} \sum_{i=1}^{n} \partial_i(f)\wedge \partial_i(g).
\]
This bracket turns $\mc G(\FF^n)$ into a Lie superalgebra, which we denote by $\tilde{H}(n)$. 
We have $Z(\tilde{H}(n)) = \FF1$ and, hence, define the \emph{Hamiltonian superalgebra $H(n)$} to be the derived superalgebra of $\tilde{H}(n)/\FF 1$. 

We can see $H(n)$ as a subsuperalgebra of $W(n)$: we send every homogeneous ${f\in \tilde H(n)}$ to $(-1)^{|f|}\sum_{i=1}^n \partial_i(f) \wedge \partial_i$, extend this map linearly and induce an embedding $H(n) \to W(n)$. 

Though the superalgebra $H(n)$ is simple for $n\geq 4$, we have $H(4)\iso A(1,1)$, so we impose $n\geq 5$ in this case.

% ---------------

\section{An overview of known results}

The classification of gradings is best understood over algebraically closed fields, so in this section we assume that $\FF$ is algebraically closed, unless stated otherwise. 

\phantomsection\label{intro-equiv}

In the case $\Char \FF = 0$, there is a bijective correspondence between gradings by a finitely generated abelian group $G$ and actions of the character group $\widehat G$ by automorphisms (see \cref{sec:g-hat-action}). 
Using this, a classification of \emph{fine abelian group gradings} up to \emph{equivalence} (see \cref{defi:fine-grading,defi:equivalence}) on the associative algebra $M_n(\FF)$ was obtained in \cite{HPP}, in terms of \emph{maximal abelian diagonalizible (MAD) subgroups} of $\operatorname{PGL}_n(\FF) \iso \Aut( M_n(\FF) )$. 
% In \cite{Zol02}, a classification of projective representations of finite abelian group $G$ (\ie, homomorphisms $G \to \operatorname{PGL}_n(\FF)$) was given, and this corresponds to a classification of $\widehat G$-gradings up to isomorphism. 
In \cite{BSZ01, BZ02}, $G$-gradings on $M_n(\FF)$ were described intrinsically, and those descriptions were extended to arbitrary characteristic in \cite{BZ03}.
% In \cref{chap:grd-simple-assc}, we review a generalized version of the classification up to isomorphism, following \cite[Chapter 2]{livromicha}. 
Degree-preserving involutions on graded matrix algebras were described in \cite{BZ07, BG08a} and classified up to isomorphism in \cite{BK10}. 

For classical simple Lie algebras, 
% still assuming $\FF$ algebraically closed and
assuming $\Char \FF = 0$, a description of fine gradings was obtained in \cite{HPP}, and an incomplete description of $G$-gradings was obtained in \cite{BSZ05,BZ06}. 
The questions of equivalence and isomorphism of gradings were left open. 
The equivalence problem for fine gradings was solved in \cite{Eld10}. 
The description of $G$-gradings was completed in \cite{BK10} in any characteristic different from $2$; the isomorphism problem was also solved there. 
It is worth mentioning that \cite{BSZ05} introduced the idea of obtaining all gradings on non-exceptional simple Lie and Jordan algebras from associative algebras. 
In this work we follow the same idea for Lie superalgebras. 

The classification of fine group gradings (up to equivalence) on finite dimensional simple Lie algebras has recently been completed in characteristic $0$ by the efforts of many authors: see \cite[Chapters 3–6]{livromicha} and the references therein, \cite{E14} and \cite{YuExc} for types $E_6, E_7 \AND E_8$; an overview can be found in \cite{MR3601079}.    

The classification of all $G$-gradings (up to isomorphism) is complete for the classical simple Lie algebras and also for types $G_2$ and $F_4$, in characteristic different from $2$: see \cite[Chapters 3–6]{livromicha} and the references therein, also \cite{EK_d4} for type $D_4$. 
Note that, in positive characteristic, there are many more finite dimensional simple Lie algebras than in characteristic $0$, including Lie algebras of Cartan type (see \cite{MR2059133,MR2573283,MR3025870}). 
A classification of gradings in the restricted case for the Witt and special series was obtained in \cite{MR2878812} (see also \cite[Chapter 7]{livromicha}). 

% -- Check $Char \FF = 0$ in what follows: 

Gradings have been classified for some non-simple algebras. 
For example, the classification of $G$-gradings on semisimple algebras reduces to the classification of graded-simple algebras, and this latter, if $G$ is abelian, can be obtained using the generalization of loop algebra construction introduced in \cite{MR2418198} if we know the gradings by quotients of $G$ on simple algebras (see, \eg, \cite{MR3857543}). 
Other examples include the upper triangular matrices, considered as an associative \cite{MR2322777}, Lie \cite{MR3614154} or Jordan algebra \cite{MR3697052}, and certain nilpotent Lie algebras \cite{MR3413670}. 

The situation is more complicated if the base field is not algebraically closed. 
For real closed fields (for example, the field of real numbers), all group gradings on classical central simple and type $G_2$ Lie algebras were classified in \cite{paper-adrian} and \cite{MR3841529}. 

Abelian group gradings on finite dimensional simple associative superalgebras were described in \cite{BS}, and those on superinvolution-simple but not simple associative superalgebras were described in \cite{BTT}. 
Both papers imposed some restrictions on characteristic and did not consider the isomorphism problem. 

The $\mathbb{Z}$-gradings on classical Lie superalgebras where classified in \cite{kacZ}. 
In \cite{serganova}, gradings by finite cyclic groups were considered and the corresponding twisted loop superalgebras were classified. 
Fine gradings on the exceptional Lie superalgebras were classified up to equivalence in \cite{artigoelduque}.

For a given graded associative or Lie algebra, there is a natural concept of graded module (see \cref{defi:grdModule}). 
A classification of graded-simple modules over semisimple graded Lie algebras was obtained in \cite{EK15,EK_d4,MR3707906} and further studied in \cite{grdd-modules-loop}. 
Gradings on a Lie superalgebra $L = L\even \oplus L\odd$ can be approached by considering $L
\odd$ as a graded module over the graded algebra $L\even$. 
This was used in \cite{paper-Qn} for the series $Q$ and in \cite{paper-MAP} for the series $P$ and $A$ (only Type I gradings for the latter, see \cref{defi:types-I-and-II}). 
As already mentioned, here we will follow a different approach to all series of classical Lie superalgebras, namely, the reduction of the problem to a suitable associative superalgebra with superinvolution (see \cref{sec:Aut-Lie-chap}). 
The easiest case for this  approach is series $B$, which was treated in \cite{Helens_thesis}. 

% --------

\section{Some applications of gradings}

% -----

Given the role of Lie superalgebras in Physics, gradings on Lie superalgebras have applications in this field. 
In a quantum system, quantum numbers are eigenvalues of operators that commute with the Hamiltonian. 
If the symmetries of the system are described by a Lie (super)algebra $L$, then a grading on $L$ gives rise to \emph{additive quantum numbers} (see \cite{MR974693,MR1025215,MR1940451}). 

% -----

Physicists are also interested in \emph{contractions} of Lie (super)algebras, to compare phenomena in system with different symmetries (see \cite{MR55352}). 
Many interesting contractions arise from gradings (see \cite{MR1098200}): if $L = \bigoplus_{g \in G} L_g$ is a $G$-graded (super)algebra, with product denoted by $[\cdot, \cdot]$, and $\sigma\from G\times G \to \FF$ is any map, we define $L^\sigma$ to be $G$-graded (super)algebra with product determined by
\[
    \forall g,h\in G, \,
    \forall a\in L_g, \,
    \forall b\in L_h, \quad
    [a, b]^\sigma 
    \coloneqq \sigma(g, h)\, [a, b].
\]
If $L$ is a Lie (super)algebra, $G$ is abelian and $\sigma\from G\times G \to \FF^\times$ is a symmetric $2$-cocycle (see \cref{defi:cocycle}), then $L^\sigma$ is again a Lie (super)algebra. 
In this case, the operation is invertible: $(L^\sigma)^{\sigma\inv} = L$; it is known as \emph{cocycle twist}. 

% -----

The twist with a non-symmetric $2$-cocycle can be used to transform color Lie algebras into Lie superalgebras. 
Given an abelian group $G$ and a skew-symmetric bicharacter $\epsilon\from G\times G \to \FF^\times$ (see \cref{def:bicharacter}), a \emph{color Lie algebra} with commutation factor $\epsilon$ is a $G$-graded algebra, with product $[\cdot, \cdot]$, such that, for all $g, h \in G$, $a \in L_g$, $b \in L_h$ and $c \in L$,
\begin{enumerate}[(i)]
    \item $[a, b] 
    = -\epsilon(g, h)\,[b,a]$;
    \item $[a, [b, c]] 
    = [[a,b],c] + \epsilon(g, h)\,[b,[a,c]]$.
\end{enumerate}
%
Note that if $G$ is trivial, we get a Lie algebra, and if $G = \ZZ_2$ and $\epsilon(i,j) = (-1)^{ij}$ for all $i,j \in \ZZ_2$, we get a Lie superalgebra. 
It was proven in \cite{MR529734} (for the case $G$ is finitely generated) and in \cite{MR1600092} (in general) that, for any color Lie algebra $L$, there exists a $2$-cocycle $\sigma\from G\times G \to \FF^\times$ such that $L^\sigma$ is a Lie superalgebra. 
% \cite[Theorem 0.10]{livromicha}
% So gradings on Lie superalgebras can be used as a tool to study Lie color algebras.
Hence, one possible approach to classify simple color Lie algebras is using gradings on Lie superalgebras (see \cite{MR2497949}).

% -----

The identities defining Lie superalgebras and color Lie algebras are examples of the so called \emph{graded polynomial identities}, which are polynomial identities that hold for all elements of specified degrees in a graded algebra. 
Recently, the theory of such identities and their combinatorial characteristics have been extensively studied, especially in the associative case (see, \eg, \cite{MR1935223,MR2176105,MR2680170,MR2639274,MR3068959,MR3030542,MR3152711,MR3271263,MR3773068}). 
% ,MR3920905,MR3886336
Graded identities for certain gradings on $P(n)$ were recently considered in \cite{MR3611462}. 
A classification of all gradings on simple Lie superalgebras opens paths for further research in this area. 

% -----
