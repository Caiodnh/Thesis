


\subsection{Old...}

% % -----------

% Recall that the graded algebra $M^{\mathrm{ex}} (T, \beta, \kappa_\bz, \kappa_\bo, g_0)$ corresponds to $(\eta, \kappa, g_0, \delta) \in \mathbf{I}(T, \beta, p)$, where $\eta\from T \to \pmone$ is fixed, $p\from T \to \ZZ_2$ is the trivial homomorphism, and $\delta = 1$. 
% The map $\eta$ was determined by fixing a standard realization $\D$ for $(T, \beta, e)$ (see \cref{def:std-realization-MxM-QxQ}), 
% namely, $\eta$ is the map associated to the superinvolution $\vphi_{\mc C} \tensor \vphi_{\mc M}$ on $\D$. 

% We will use the parameters $(T, \beta, \kappa_\bz, \kappa_\bo, g_0)$ to construct a representative for the Type II grading in terms of the ``undoubled'' model, \ie, we will construct a $\barr G$-grading on $S$, where $\barr G \coloneqq G/\langle f \rangle$, and a super-anti-automorphism $\theta\from S\to S$, as in \cref{ssec:undoubling}. 
% As in that subsection, let $\pi\from G \to \barr G$ denote the natural homomorphism, set $\barr T \coloneqq T/\langle f \rangle$, let $\barr \beta$ be the (nondegenerate) bicharacter on $\barr T$ induced by $\beta$, and consider $\kappa_\bz$ and $\kappa_\bo$ as maps defined on $\barr G/\barr T \iso G/T$. 

% Let $\barr \D$ be a standard realization associated to $(\barr T, \bar\beta)$ (see \cref{def:standard-realization}), let $\theta_0\from \barr \D \to \barr \D$ be the transposition map, and let $\bar \mu\from \barr T \to \FF^\times$ be the map associated to $\theta_0$. 
% (Recall that this also gives us a choice of elements $0 \neq X_{\bar t} \in \barr\D_{\bar t}$, for all $\bar t \in \barr T$.)

% Choose a complement $K$ for $\langle f \rangle$ in $T$, \ie, a subgroup $K \subseteq T$ such that $T = K \times \langle f \rangle$ (it can be done, since $T$ is a elementary $2$-group). 
% Let $\chi\from T \to \FF^\times$ be the character defined by $\chi(K) = 1$ and $\chi(f) = -1$, and set $\mu \coloneqq \barr \mu \circ \pi$. 
% Then set 
% \[\label{eq:fix-eta-undouble}
%     \forall t\in T, \quad \eta(t) \coloneqq \mu(t) \chi\inv(t).
% \]
% Note that this definition agrees with \cref{def:std-realization-MxM-QxQ} (see the proof of \cref{prop:m-not-n-Type-II-correspondence}). 

% Extend $\chi$ to $G$. 
% It remains to define a graded right $\barr \D$-supermodule $\barr \U$, and the matrix $\Theta$. 
% % Note that, once we fixed $\eta$, we can say that $(\kappa_\bz, \kappa_\bo)$ is $g_0$-balanced. 
% The following is analogous to the construction in \cref{ssec:grds-osp}. 

% Let $\xi\from G/ T \to G$ be a set-theoretic section of the natural homomorphism, and let $\leq$ be a total order on the set $G/T \iso \barr G/ \barr T$ (or just its finite subset $\supp \kappa_\bz \cup \supp \kappa_\bo$) with no elements between $x$ and $\bar g_0\inv x\inv$. 
% Changing $\xi$ if necessary, we may assume that $\xi(g_0\inv x\inv) = g_0\inv \xi(x)\inv$ if $x < g_0\inv x\inv$. 
% Set $k_\bz \coloneqq |\kappa_\bz|$ and let $\gamma_\bz$ be the $k_\bz$-tuple given by putting the elements of $\{ \xi(x) \mid x \in \supp \kappa_\bz\} \subseteq G$ following the order $\leq$ and repeating $\kappa_\bz(x)$ times each element $\xi(x)$. 
% % Clearly, $\gamma_\bz$ realizes $\kappa_\bz$. 
% Similarly, set $k_\bo \coloneqq |\kappa_\bo|$ and construct the $k_\bo$-tuple $\gamma_\bo$.
% % , which realizes $\kappa_\bo$. 
% % We then define $\barr \U\even \coloneqq \barr \D^{[\bar g_1]} \oplus \cdots \oplus \barr \D^{[\bar g_{k_\bz}]}$, $\barr \U\odd \coloneqq \barr \D^{[\bar h_1]} \oplus \cdots \oplus \barr \D^{[\bar h_{k_\bz}]}$ and $\barr \U \coloneqq \barr \U\even \oplus \barr \U\odd$. 
% Let $\bar \gamma_\bz$ and $\bar \gamma_\bo$ to be the tuples of elements of $\barr G$ consisting of the image under $\pi\from G\to \barr G$ of the entries of $\gamma_\bz$ and $\gamma_\bo$, respectively. 
% Clearly, $\barr\gamma_\bz$ and $\barr\gamma_\bo$ realize $\kappa_\bz$ and $\kappa_\bo$ (see \cref{defi:gamma-realizes-kappa}). 
% Consider on $M_{k_\bz | k_\bo}(\FF)$ the elementary grading determined by $(\bar  \gamma_\bz, \bar \gamma_\bo)$ (see \cref{defi:elementary-grd-super}). 
% We identify the $\barr G$-graded superalgebra $M_{k_\bz | k_\bo}(\barr \D) = M_{k_\bz | k_\bo}(\FF) \tensor \barr\D$ with $S = M(m+1, n+1)$ via Kronecker product. 

% % We have that $S \iso \End_{\barr \D} (\barr \U)$ as $\barr G$-graded superalgebra. 
% % Using the $\barr G^\#$-graded canonical basis of $\barr \U$, we can identify $\End_{\barr \D} (\barr \U)$ with $M_{k_\bz | k_\bo}(\barr \D)$. 
% % From now on, we will identify $S$ with $M_{k_\bz | k_\bo}(\barr \D)$. 

% \begin{defi}\label{defi:blocks-of-Theta}
%     Let $i\in \ZZ_2$ and $x \in G/T$. 
%     If $g_0x^2 = T$, we put $t \coloneqq g_0 \xi(x)^2 \in T$ and let $\bar t \in \barr T$ be its image under the canonical homomorphism $T \to \barr T$. 
%     We define $\Theta(i, x)$ to be the following $\kappa(x) \times \kappa(x)$-matrix with entries in $\D$:
%     %
%     \begin{enumerate}[(i)]
%         \item $\chi(\xi(x))\inv I_{\kappa(x)} \tensor X_{\bar t}$ if $(-1)^i \eta(t) = +1$;
%         %
% 		\item  $\chi(\xi(x))\inv J_{\kappa(x)} \tensor X_{\bar t}$, where $J_{\kappa(x)} \coloneqq \begin{pmatrix}
% 				      0                & I_{\kappa(x)/2} \\
% 				      -I_{\kappa(x)/2} & 0
% 			      \end{pmatrix}$, if  $(-1)^i \eta(t) = -1$ (recall that, in this case, $\kappa (x)$ is even by \cref{inertia-even-and-odd-case}). 
% 	\end{enumerate}
%     %
%     If $g_0 x^2 \neq T$, we define $\Phi(i, x)$ to be the following $2\kappa(x) \times 2\kappa(x)$-matrix with entries in $\barr\D$:
%     %
%     \begin{enumerate}[(i)]
%         %
%         \setcounter{enumi}{2}
%         %
% 		\item $\begin{pmatrix}
% 			0                                                  & \chi(\xi(x))\inv I_{\kappa(x)} \\
% 			(-1)^{i} \chi(\xi(x))\inv I_{\kappa(x)} & 0
% 		\end{pmatrix} \tensor 1$, where $1$ is the identity element of $\barr\D$. 
%     \end{enumerate}
% \end{defi}


% % For each $\bar g_i$, $1 \leq i \leq k_\bz$, and for each $\bar h_j$, $1 \leq j \leq k_\bo$, choose representatives $g_i \in G$ and $h_j \in G$, respectively. 
% % Extend $\chi$ to $G$.
% % , and define $\Lambda \in M_{k_\bz | k_\bo}(\FF) \subseteq M_{k_\bz | k_\bo}(\barr \D)$ to be the diagonal matrix 
% % \[\label{eq:Lambda-even-case}
% %     %
% %     \sbox0{$\begin{matrix}
% %         \chi(g_1)\inv && \\
% %         & \ddots &\\
% %         && \chi(g_{k_\bz})\inv
% %     \end{matrix}$}
% %     %
% %     \sbox1{$\begin{matrix}
% %         \chi(h_1)\inv && \\
% %         & \ddots &\\
% %         && \chi(h_{k_\bo})\inv
% %     \end{matrix}$}
% %     %
% %     \Lambda \coloneqq
% %     \left(\begin{array}{c|c}
% %             \usebox{0} & 0\\
% %             \hline
% %             0 & \usebox{1}
% %         \end{array}\right).
% % \]
% % %
% % For each $\bar t \in \barr T$, choose $0\neq X_{\bar t} \in \barr D_{\bar t}$. 

% Let $x_1 < \ldots < x_{\ell_\bz}$ be the elements of $\{ x \in \supp \kappa_\bz \mid x \leq g_0\inv x^2 \}$ and, similarly, let $y_1 < \ldots < y_{\ell_\bo}$ be the elements of $\{ y \in \supp \kappa_\bo \mid y \leq g_0\inv y^2 \}$. 
% Then, we define 
% \[\label{eq:puting-the-blocks-of-Phi-together-version-A}
%     %
%     \sbox0{$\begin{matrix}
%         \Theta(\bar 0, x_1)&& \\
%         & \ddots &\\
%         && \Theta(\bar 0, x_{\ell_\bz})
%     \end{matrix}$}
%     %
%     \sbox1{$\begin{matrix}
%         \Theta(\bar 1, y_1)&& \\
%         & \ddots &\\
%         && \Theta(\bar 1, y_{\ell_\bo})
%     \end{matrix}$}
%     %
%     \Theta \coloneqq
%     \left(\begin{array}{c|c}
%             \usebox{0} & 0\\
%             \hline
%             0 & \usebox{1}
%         \end{array}\right).
% \]
% %
% Finally, we define the super-anti-automorphism $\theta\from S\to S$ by \cref{eq:theta-with-matrix-2}. 
% Since $\theta_0$ is the transposition on $\barr \D$, $\theta_0(X)\stransp \in M_{k_\bz \mid k_\bo} (\barr \D)$ in becomes $X\stransp \in M(m+1,n+1)$. 
% Hence, \cref{eq:theta-with-matrix-2} reduces to 
% \[\label{eq:theta-with-matrix-3}
%     \forall X\in M_{k_\bz \mid k_\bo} (\barr \D), \quad \theta(X) = \Theta\inv\, X\stransp\, \Theta.
% \]

% \begin{defi}\label{defi:type-II-A-m-not-n}
%     Let $m,n \in \ZZ_{\geq 0}$, $m\neq n$. 
%     Let $T \subseteq G$ be a finite $2$-elementary subgroup, let $\beta\from {T\times T} \to \FF^\times$ be an alternating bicharacter with $\rad \beta = \langle f \rangle$, for some $f\in T$, and let $g_0 \in G$. 
%     Set $\barr G \coloneqq G/\langle f \rangle$, $T \coloneqq T/\langle f \rangle$, and let $\bar \beta$ be the nondegenerate alternating bicharacter on $\barr T$ induced by $\beta$. 
%     Choose:
%     \begin{enumerate}[(i)]
%         \item a standard realization $\barr \D$ associated to $(\barr T, \barr \beta)$; 
%         \item a subgroup $K \subseteq T$ such that $T = K \times \langle f \rangle$; 
%         \item a set-theoretic section $\xi\from G/T \to G$ for the quotient homomorphism $G \to G/T$;
%         \item a total order $\leq$ on $G/T$ such that there are no elements between $x$ and $\bar g_0\inv x\inv$, for all $x\in G/T$. 
%     \end{enumerate}
%     Let $\barr \mu\from \barr T \to \FF^\times$ be the map determining the transposition on $\barr \D$ (see \cref{ssec:param-D-vphi}), and 
%     let $\chi \in \widehat{T}$ be the character such that $\chi(K) = 1$ and $\chi(f) = -1$, and extend it to a character on $\widehat{G}$, also denoted by $\chi$. 
%     Then define $\mu \coloneqq \bar\mu \circ \pi$, where $\pi\from G \to \barr G$ is the canonical homomorphism, and fix $\eta\from T \to \pmone$ as in \cref{eq:fix-eta-undouble}. 
%     Let $\kappa_\bz, \kappa_\bo \from G/T \to \ZZ_{\geq 0}$ be $g_0$-admissible maps (\cref{inertia-even-and-odd-case}) such that $m+1 = k_\bz \sqrt{|T|/2}$ and $n+1 = k_\bo \sqrt{|T|/2}$, where $k_i \coloneqq |\kappa_i|$, $i\in \ZZ_2$. 
%     Then construct tuples $\bar\gamma_\bz$ and $\bar\gamma_\bo$ realizing $\kappa_\bz$ and $\kappa_\bo$, respectively, as did before \cref{defi:blocks-of-Theta}. 
%     Consider the $\barr G$-grading $\Gamma_M(\barr T, \barr \beta, \kappa_\bz, \kappa_\bo)$ on $S \coloneqq M(m+1,n+1)$ using the choices above (see \cref{def:Gamma-T-beta-kappa-even}), and consider its restriction $\Gamma : L = \bigoplus_{\bar g \in \barr G} L_{\bar g}$ to $L \coloneqq S^{(1)}$. 
%     Consider ${\Theta \in S}$ as in \cref{eq:puting-the-blocks-of-Phi-together-version-A} and ${\theta\from S \to S}$ as in
%     \cref{eq:theta-with-matrix-3}. 
%     We define $\Gamma_A^{\mathrm{(II)}}(T, \beta, \kappa_\bz, \kappa_\bo, g_0)$ to be the $G$-grading on $L$ given by
%     \[
%         L_{g} \coloneqq \{ s\in L_{\bar g} \mid s = - \chi(g) s \},
%     \]
%     for all $g\in G$. 
% \end{defi}

% \begin{prop}\label{prop:m-not-n-Type-II-correspondence}
%     Consider $(R, \vphi) \coloneqq M^{\mathrm{ex}}(T, \beta, \kappa_\bz, \kappa_\bo, g_0)$ (\cref{def:model-grd-MxM-even}). 
%     Then $\Skew (R,\vphi)$ is isomorphic to $M(m+1, n+1)$ endowed with $\Gamma_A^{\mathrm{(II)}}(T, \beta, \kappa_\bz, \kappa_\bo, g_0)$. 
% \end{prop}

% \begin{proof}
%     We will now show how the choices in \cref{defi:type-II-A-m-not-n} correspond to the choices in Definitions \ref{def:std-realization-MxM-QxQ}(a) and \ref{def:model-grd-MxM-even}. 
    
%     The choices in items (i) and (ii) give us a way to make to the choices in \ref{def:std-realization-MxM-QxQ}(a)
%     % First, we note that the choice of $K$ in \ref{def:std-realization-MxM-QxQ}(a)\eqref{item:K-can-be-orthogonal-to-t_1} can be the same as in \cref{defi:type-II-A-m-not-n}(ii). 
%     Recall that a standard realization $\barr \D$ associated to $(\barr T, \bar \beta)$ (choice \cref{defi:type-II-A-m-not-n}(i)) is obtained by choosing subgroups $\barr A$ and $\barr B$ of $\barr T$ such that $\barr T = \barr A \times \barr B$ and $\beta (\barr A, \barr A) = \beta (\barr B, \barr B) = 1$. 
%     Note that $\pi\from G \to \barr G$ restricts to an isomorphism $K \to \barr T$, and $\beta(s,t) = \barr\beta(\bar s, \bar t)$, for all $s,t \in K$. 
%     Hence, the choice of the subsets $\barr  A$, $\barr  B$ as above is, then, equivalent to a choice of subgroups $A,B \subseteq K$ such that $K = A\times B$ and $\beta (A, A) = \beta (B, B) = 1$. 
%     In other words, our choices of $\barr \D$ and $K$ give us the same information as the choice of $\mc M$ in \cref{def:std-realization-MxM-QxQ}(a)\eqref{item:choose-mc-M}. 
%     Also, by \cref{lemma:transp-std-realization}, $\bar\mu (ab) = \beta(a,b)$ for all $a\in \barr A$ and $b\in \barr B$, so $\mu\restriction_{K}$ is the map determining $\vphi_{\mc M})$. 
%     Further, since $\eta\restriction_{K} = \mu\restriction_{K}$ and $\eta\restriction_{\langle f \rangle} = \chi\restriction_{\langle f \rangle}$, the map $\eta$ defined \cref{eq:fix-eta-undouble} is the map determining to $\vphi_{\mc C} \tensor \vphi_{\mc M}$. 
    
%     We can use choices in items (iii) and (iv) to choose the pair $(\U, B)$ as in \cref{def:model-grd-MxM-even}, by following the exact same construction done in \cref{ssec:grds-osp}. 
    
%     With those choices being made to construct $M^{\mathrm{ex}}(T, \beta, \kappa_\bz, \kappa_\bo, g_0)$, and choosing $\chi \in \widehat{G}$ such that $\chi(K) = 1$ and $\chi(f) = -1$, it is clear that if follow the steps in \cref{ssec:undoubling}, the result follows. 
% \end{proof}

% For the next result, we will we fix choices (i) and (ii), for each pair $(T, \beta)$ as in \cref{defi:type-II-A-m-not-n}, in the ``Type II'' case. 
% Recall that, for every $\kappa\Star\from G/T \to \ZZ_{\geq 0}$, we $\kappa\Star\from G/T \to \ZZ_{\geq 0}$ by $\kappa\Star(x) = \kappa(x)\inv$, for all $x \in G/T$ (see \cref{ssec:superdual}).


\subsection{Gradings on \texorpdfstring{$A(n,n)$}{A(n,n)}}


Here we can have even and odd gradings. 
For the even ones, two kappas and define:

\begin{defi}
    \mbox{}
    
    \boxed{\mathrm{Type \,\,I}_M}
    
        even type I
        
    \boxed{\mathrm{Type \,\,II}_{osp}}
    
        even type II with even $g_0$
        
    \boxed{\mathrm{Type \,\,II}_P}
    
        even type II with odd $g_0$
\end{defi}

The reference for other types of Lie superalgebras come from ...

For odd gradings, we define: 

\begin{defi}
    \mbox{}
    
    \boxed{\mathrm{Type \,\,I}_Q}
    
        odd type I
        
    \boxed{\mathrm{Type \,\,II}_Q}
    
        odd type II
\end{defi}

The reference for other Lie superalgebras come from ....

For even gradings, we do standard realizations and $\Phi$ just like on case $m \neq n$. 
(there is a tiny difference on the isomorphism condition, though)

For odd gradings, 

\begin{defi}
    Type II gradings, standard realization in terms of $G^\#$. 
    Choose $t_p$, $t_1$ and $S$ and use it to define $\chi$. 
\end{defi}

Compare definition with case (b) in previous definition. 

Using $\bar t_1$, we can find another character, $\chi_0$, which is $\beta( t_1, \cdot)$ restricted to $T^+$. 
It is different than $\chi$ by its values on $t_p$ and $f$. 
Then we can reduce $(\bar T, \bar \beta)$ to double bars like in Chapter 1, and have a description which is only in terms of $G$. 
(actually, we first fix $\chi_0$ and then use it to choose a $t_1$ such that $t_1^2 = a$, where $a$ the element such that $\chi^2 = \beta(a, \cdot)$).

\begin{defi}
    Type II gradings, standard realization in terms of $G$. 
\end{defi}

We can then find a suitable $\Phi$ from $\kappa$ fixing stuff.

\begin{defi}
    model for the grading.
\end{defi}
