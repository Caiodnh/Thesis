% \makeatletter
% \def\input@path{{../}}
% \makeatother
% %
% \documentclass[../00_Thesis.tex]{subfiles}
% %
% \overfullrule=2cm
% %
% %
% \begin{document}

\chapter{Generalities on gradings}\label{sec:generalities}
% ==================================================

The purpose of this section is to fix notation and terminology concerning graded algebras and graded modules. We should warn the reader that some terms appearing in Section \ref{ssec:universal_group} are not used consistently in the literature (see discussion in \cite[\S 2.7]{GS}); here we follow \cite{livromicha}.

% --------------------------------------------------
\section{Gradings on vector spaces and (bi)modules}\label{subsec:graded-bimodules}
% --------------------------------------------------

% Def: Graded vector space + Def: Shift by g
Let $G$ be a group. By a \emph{$G$-grading} on a vector space $V$ we mean simply a vector space decomposition $\Gamma:\,V = \bigoplus_{g \in G} V_g$ where the summands are labeled by elements of $G$. If $\Gamma$ is fixed, $V$ is referred to as a {\em $G$-graded vector space}. A subspace $W \subseteq V$ is said to be \emph{graded} if $W = \bigoplus_{g \in G} (W \cap V_g)$. We will refer to $\ZZ_2$-graded vector spaces as \emph{superspaces} and their graded subspaces as \emph{subsuperspaces}.

An element $v$ in a graded vector space $V = \bigoplus_{g \in G} V_g$ is said to be \emph{homogeneous} if $v\in V_g$ for some $g\in G$. 
If $0\ne v\in V_g$, we will say that $g$ is the \emph{degree} of $v$ and write $\deg v = g$. 
In reference to the canonical $\ZZ_2$-grading of a superspace, we will instead speak of the \emph{parity} of $v$ and write $|v| = g$.
Every time we write $\deg v$ or $|v|$, it should be understood that $v$ is a nonzero homogeneous element.

% Def: Grading on tensor product
\begin{defi}
	Given two $G$-graded vector spaces, $V=\bigoplus_{g\in G} V_g$ and $W=\bigoplus_{g\in G} W_g$, we define their tensor product to be the vector space $V\otimes W$
	together with the $G$-grading given by $(V \otimes W)_g = \bigoplus_{ab=g} V_{a} \otimes W_{b}$.
\end{defi}

The concept of grading on a vector space is connected to gradings on algebras by means of the following:

% Def: Homogeneous Map
\begin{defi}
	If $V=\bigoplus_{g\in G} V_{g}$ and $W=\bigoplus_{g\in G} W_{g}$ are two graded vector spaces and $T: V\rightarrow W$ is a linear map, we say that $T$ is \emph{homogeneous of degree $t$}, for some $t\in G$, if $T(V_g)\subseteq W_{tg}$ for all $g\in G$.
\end{defi}

% Def: Homogeneous Transformations
If $S: U\rightarrow V$ and $T: V\rightarrow W$ are homogeneous linear maps of degrees $s$ and $t$, respectively, 
then the composition $T\circ S$ is homogeneous of degree $ts$.
% Def: Space of Homogeneous Transformations
We define the {\em space of graded linear transformations} from $V$ to $W$ to be:
%
\[ \Hom^{\text{gr}} (V,W) = \bigoplus_{g\in G} \Hom (V,W)_{g}\]
%
% Prop: End(V) is a graded algebra
where $\Hom (V,W)_{g}$ denotes the set of all linear maps from $V$ to $W$ that are homogeneous of degree $g$. 
If we assume $V$ to be finite-dimensional then we have $\Hom(V,W)=\Hom^{\gr}(V,W)$ and, in particular, $\End (V) = \bigoplus_{g\in G} \End (V)_g$ is a graded algebra.
% Prop: V is a graded module
We also note that $V$ becomes a graded module over $\End(V)$ in the following sense:

% Def: graded module
\begin{defi}
	Let $A$ be a $G$-graded algebra (associative or Lie) and let $V$ be a (left) module over $A$ that is also a $G$-graded vector space. We say that $V$ is a \emph{graded $A$-module} if $A_g \cdot V_h \subseteq V_{gh}$, for all $g$,$h\in G$. The concept of $G$-\emph{graded bimodule} is defined similarly.
\end{defi}

% L1 is a graded module over L0
If we have a $G$-grading on a Lie superalgebra $L=L\even \oplus L\odd$ then, in particular, we have a grading on the Lie algebra $L\even$ and a grading on the space $L\odd$ that makes it a graded $L\even$-module. If we have a $G$-grading on an associative superalgebra $C=C\even \oplus C\odd$, then $C\odd$ becomes a graded bimodule over $C\even$.

If $ \Gamma$ is a $G$-grading on a vector space $V$ and $g\in G$, we denote by $\Gamma^{[g]} $ the grading given by relabeling the component 
$V_h$ as $V_{hg}$, for all $h \in G$. This is called the \emph{(right) shift of the grading $\Gamma$ by $g$}. 
We denote the graded space $(V, \,  \Gamma^{[g]})$ by $V^{[g]}$.

% Lemma: Shifts on (bi)modules
From now on, we assume that $G$ is abelian.
If $V$ is a graded module over a graded algebra (or a graded bimodule over a pair of graded algebras), then $V^{[g]}$ is also a graded (bi)module. 
We will make use of the following partial converse (see e.g. \cite[Proposition 3.5]{paper-Qn}):

\begin{lemma}\label{lemma:simplebimodule}
	Let $A$ and $B$ be $G$-graded algebras and let $V$ be a finite-dimensional (ungraded) simple $A$-module or $(A,B)$-bimodule.  If $\Gamma$ and $\Gamma'$ are  two $G$-gradings that make $V$ a graded (bi)module, then $\Gamma'$ is a  shift of $\Gamma$.\qed
\end{lemma}

Certain shifts of grading may be applied to graded $\ZZ$- or $\ZZ_2$-superalgebras. In the case of a $\ZZ$-superalgebra $L=L^{-1}\oplus L^{0}\oplus L^{1}$, we have the following:

% Lemma: opposite directions
\begin{lemma}\label{lemma:opposite-directions}
	Let $L=L^{-1}\oplus L^0\oplus L^1$ be a $\ZZ$-superalgebra such that $L^1\, L^{-1}\neq 0$. If we shift the grading on $L^1$ by $g\in G$ and the grading on $L^{-1}$ by $g' \in G$, then we have a grading on $L$ if and only if $g' = g^{-1}$. \qed
\end{lemma}

We will describe this situation as \emph{shift in opposite directions}.

\section{$\Omega$-algebras}

A \emph{general} or \emph{universal algebra} is a set with an arbitrary collection (possibly empty) of operations, which may have different ``arities'' (see Definition \ref{def:universal-algebra} below). 
This is a very general concept (see, \eg, ?? and ??), which includes all classical objects in algebra (groups, rings, etc.), but here we will be interested in the linear case: the objects will be vector spaces over a field $\FF$ and the operations will be assumed multilinear (following ??, book by Razmylov).
%Our approach to universal algebras differs from the one in most books on the subject (see, \eg, ?? and ??) in the sense we will be working on the (monoidal) category of vector spaces instead of the (monoidal) category of sets. 
% This is the approach on ?? (Razmylov: book used by Felipe) and it has recently been applied to gradings (and graded identities) on ??.
This will give us a framework to deal with algebras, superalgebras, superalgebras with superinvolution, etc., in a uniform manner.

\begin{notation}
    For a vector space $A$ and a nonnegative integer $n$, we will denote the $n^{th}$-tensor power of $A$ by $A^{\tensor n}$, \ie,
    $A^{\tensor n} \coloneqq \underbrace{A\otimes\cdots\otimes A}_{n \text{ times}}$. 
    In the case $n = 0$, we will follow the convention that $A^{\tensor 0} \coloneqq \FF$.
\end{notation}

\begin{defi}\label{def:universal-algebra}
    A \emph{signature} $\Omega$ is a set together with a family $\{ \Omega_n \}_{n \geq 0}$ of pairwise disjoint subsets such that $\Omega = \bigcup_{n \geq 0} \Omega_n$. 
    An \emph{$n$-ary operation} on a vector space $A$ is a multilinear map $A^n \to A$ or, equivalently, a linear map $A^{\tensor n} \to A$. 
    An \emph{$\Omega$-algebra} or a \emph{(general) algebra with signature $\Omega$} is a vector space $A$ together with operations $\omega^A$, one for each $\omega \in \Omega$, such that $\omega^A$ is $n$-ary if $\omega \in \Omega_n$.
\end{defi}

We note that $0$-ary operations can be interpreted as constants in the $\Omega$-algebra $A$, since a linear map $\omega^A\from \FF \to A$ is determined by $\omega^A(1)$.

\begin{defi}
    Let $A$ and $B$ be $\Omega$-algebras. 
    A \emph{homomorphism} $\psi\from A \to B$ is a linear map such that for every $\omega \in \Omega_n$ we have
    \[
        \psi( \omega^A (a_1 \tensor \cdots \tensor a_n) ) = \omega^B ( \psi(a_1) \tensor \cdots \tensor \psi(a_n) ),
    \]
    for all $a_1, \ldots, a_n \in A$. 
    As usual, an \emph{automorphism} of $A$ is a bijective homomorphism from $A$ to itself, and the group of automorphisms is denoted by $\Aut(A)$.
\end{defi}

When dealing with a fixed $\Omega$-algebra $A$, we will usually identify the signature $\Omega$ with the corresponding set of operations on $A$.

\begin{ex}\label{ex:omega-vec-space}
    A vector space is an $\Omega$-algebra with $\Omega = \emptyset$.
\end{ex}

\begin{ex}\label{ex:omega-algebra}
    An algebra in the usual sense, with product $*$, is an $\Omega$-algebra with $\Omega = \Omega_2 = \{ * \}$. 
\end{ex}

\begin{ex}
    An algebra $A$ with unity $\mathds{1} \in A$ is an  $\Omega$-algebra with $\Omega = \Omega_0 \cup \Omega_2$ where $\Omega_2 = \{ * \}$, $\Omega_0 = \{ \omega_0 \}$, and $\omega_0\from \FF \to A$ is defined by $\omega_0 (1) = \mathds 1$.
\end{ex}

\begin{ex}\label{ex:omega-alg-SA}
    A superalgebra $A = A\even \oplus A\odd$ is an $\Omega$-algebra with $\Omega = \Omega_1 \cup \Omega_2$, where $\Omega_2 = \{ * \}$, $\Omega_1 = \{ \pi_{\bar 0}, \pi_{\bar 1} \}$, and $\pi_{\bar 0}, \pi_{\bar 1}\from A \to A$ are the projection on the $A\even$ and $A\odd$, respectively. 
    More precisely, an algebra $A$ with such signature is a superalgebra if, and only if,
    \begin{enumerate}[(i)]
        \item $\pi_{\bar 0}^{} + \pi_{\bar 1}^{} = \id$; \label{item:sum-projections}
        \item $\pi_{\bar 0}^{}\pi_{\bar 1}^{} = \pi_{\bar 1}^{}\pi_{\bar 0}^{} = 0$
        %\item $\pi_{\bar 0}^2 = \pi_{\bar 0}^{}$ and $\pi_{\bar 1}^2 = \pi_{\bar 1}^{}$;
        \item For every $x,y \in A$ and $i, j\in \ZZ_2$, we have that $\pi_{i+j}^{}( \pi_i^{} (x)*\pi_j^{} (y) ) = \pi_i^{} (x)*\pi_j^{} (y)$.
    \end{enumerate}
\end{ex}

% \begin{ex}\label{ex:omega-graded-algebra}
%     A $G$-graded algebra $A = \bigoplus_{g\in G}$, with operation $*$, is a $\Omega$-algebra with $\Omega = \Omega_1 \cup \Omega_2$, with $\Omega_2 = \{ * \}$ and $\Omega_1 = \{ \pi_g \mid g\in G\}$, where $\pi_g\from A \to A$ is the projection in the component $A_g$. 
%     As a particular case, superalgebras can be viewed as $\Omega$-algebras.
% \end{ex}

\begin{ex}
    A superalgebra with super-anti-automorphism $(A, \vphi)$ an $\Omega$-algebra with $\Omega = \Omega_1 \cup \Omega_2$ where $\Omega_2 = \{ * \}$ and $\Omega_1 = \{ \vphi, \pi_{\bar 0}, \pi_{\bar 1} \}$. 
    In a similar fashion, algebras with anti-automorphism can be viewed as $\Omega$-algebras.
\end{ex}

It is straightforward to check that, in each of the examples above, the usual notion of homomorphism coincides with the notion of homomorphism as $\Omega$-algebras.

\begin{remark}
    Example \ref{ex:omega-alg-SA} could be generalized to encompass all $G$-graded algebras for a fixed group $G$. 
    Such generalization has been used in the study of graded identities (see ??), but we will not follow this approach.
\end{remark}

Finally, we can define gradings on $\Omega$-algebras:

\begin{defi}\label{def:grds-on-Omega-algebras}
    A \emph{$G$-grading} on an $\Omega$-algebra $A$ is a $G$-grading on its underlying vector space such that, if we consider the usual grading on the tensor powers $A^{\tensor n}$, all the operations $\omega^A$ for $\omega \in \Omega$ are degree preserving.
\end{defi}

It is easy to verify that this notion of grading coincides with to the usual notion for algebras, superalgebras, superalgebras with superinvolution, etc.

% It should be noted that among these examples, only in Examples \ref{ex:omega-vec-space} and \ref{ex:omega-algebra} the correspondence is bijective. 
% In the other cases, the original structures correspond to proper subclasses of the $\Omega$-algebras described, which can be axiomitized. 
% We will present axioms for Example \ref{ex:omega-graded-algebra}, since gradings are not usually describe in terms of the projection maps, we will

% \begin{itemize}
%     \item[\done] paragraph-remark stating only exs 1 and 2 are ``precisely'', the other are proper inclusions.
%     \item discuss, in a remark, the axioms of graded algebras, noting that one of them is not first-order if the group is not finite.
% \end{itemize}

\section{$G$-gradings on  and $\widehat G$-actions}

In this section we will review the correspondence between $G$-gradings and $\widehat G$-actions, which has been used extensively in the study of gradings. 
It can be found in many places in the literature (see, \eg, ??, ?? and ??), but our goal in this section is to set it up in the context of general algebras. 
% The main purpose of this approach is to transfer gradings between general algebras with different signatures (Theorem ??).% This will allow us to handle algebras and superalgebras, with or without (super)involutions, in a uniform manner. 

The main advantage of this approach is to have a formal result comparing gradings on general algebras with different signature (Theorem ??). 
This will be used in Chapter ?? to transfer gradings between Lie superalgebras and associative superalgebras with superinvolutions. 
The same sort of transfer has been used in other works (see \cite{livromicha} and Paper with Adrian ??), but without having the result stated formally.

% If $\FF$ is algebraically closed of characteristic $0$ and $G$ is finitely generated abelian group, there is a well known correspondence between $G$-gradings and algebraic (rational) $\widehat G$-actions on vector spaces. 

The following is well known:

\begin{thm}\label{thm:g-hat-correspondence}
    Suppose $\FF$ is algebraically closed of characteristic $0$ and let $G$ be a finitely generated abelian group. 
    Then there is a bijective correspondence between $G$-gradings and algebraic $\widehat G$-actions on a vector space $V$, definied as follows:
    \begin{enumerate}[(i)]
        \item Given a grading $\Gamma: V = \bigoplus_{g\in G} V_g$, the corresponding action is defined by $\chi\cdot v = \chi(g) v$ for all $v\in V_g$ and all $g\in G$ and extended to all $V$ by linearity;
        \item Given an algebraic action of $\widehat G$ on $V$, the corresponding grading is defined by declaring $v\in V$ homogeneous of degree $g\in G$ if, and only if, $\chi\cdot v = \chi(g) v$ for all $\chi \in \widehat G$. \label{item:action-to-grading} \qed
    \end{enumerate}
\end{thm}

\begin{notation}
    Given a grading $\Gamma$ on $V$, we will denote the corresponding representation of the algebraic group $\widehat G$ on $V$ by $\eta_\Gamma\from \widehat G \to \GL(V)$.
\end{notation}

\begin{remark}
    The correspondence of Theorem \ref{thm:g-hat-correspondence} can be generalized to arbitrary fields and arbitrary abelian groups with the use of group schemes instead of algebraic groups, but this generality is not needed for the current work.
\end{remark}

We will now see how this correspondence applies to $\Omega$-algebras:

% We will now consider gradings on universal algebras with arbitrary signature (which should not be confused with Example \ref{ex:omega-graded-algebra}, which realizes graded algebras as a specific type of universal algebra).

% \begin{defi}
%     A $G$-grading on a $\Omega$-algebra $A$ is a $G$-grading on its vector space underlying such that, if we consider the usual grading on the tensor powers $A^{\tensor n}$, all the operations $\omega^A$ for $\omega \in \Omega$ are degree preserving.
% \end{defi}

% It is straight forward to verify that this notion of grading corresponds to the usual notion of gradings on algebras, superalgebras, algebras with antiautomorphisms and superalgebras with super-anti-automorphisms.

% We will now proceed to generalize the correspondence between $G$-gradings and $\widehat G$-actions (see ?? and ??) to $\Omega$-algebras. 
% We start focusing on graded vector spaces.

% \begin{defi}
%     Let $V$ be a vector space. 
%     Given a $G$-grading $\Gamma\from V= \bigoplus_{g\in G} V_g$ on $V$, we define a $\widehat G$-action in the following way: for $\chi \in \widehat G$ and $a_g \in V_g$, $\chi \cdot v_g = \chi(g)v_g$, and we extend the it by linearity. 
%     We will denote the corresponding representation by $\eta_\Gamma\from \widehat G \to \GL(V)$.
% \end{defi}

% It is easy to see that this really defines an action. 

% For this action to capture more information about the grading, we need some assumptions on the field $\FF$. 

% \begin{lemma}
%     Let $V$ be a $G$-graded vector space. 
%     A element $v\in V$ is homogeneous of degree $g\in G$ if, and only if, $\chi\cdot v = \chi(g)v$ for all $\chi \in \widehat G$.
% \end{lemma}

% \begin{proof}
%     The ``only if'' direction is the definition of the action. 
%     For the other direction, let $v\in V$ and write $v = \sum_{g\in G} v_g$, where $v_g \in V_g$.
% \end{proof}

\begin{prop}\label{prop:g-hat-Aut-A}
    Assume $\FF$ is algebraically closed and $\Char \FF = 0$. 
    Let $A$ be a $\Omega$-algebra and $\Gamma$ be a $G$-grading on its underlying vector space. 
    Then $\Gamma$ is a $G$-grading on $A$ if, and only if, $\eta_\Gamma(\widehat G) \subseteq \Aut(A)$.
\end{prop}

\begin{proof}
    First, assume $\Gamma$ is a grading on the $\Omega$-algebra $A$.
    Let $\chi \in \widehat G$ and let $\psi \coloneqq \eta_\Gamma(\chi)$. 
    We already know that $\psi$ is bijective, it only remains to prove it is a homomorphism. 
    Let $\omega \in \Omega_n$ and let $a_1, \ldots, a_n \in A$ be homogeneous elements of degrees $g_1, \ldots, g_n \in G$, respectively.
    
    Then $a_1\tensor \cdots \tensor a_n \in A^{\tensor n}$ has degree $g_1 \cdots g_n$. Hence
    \begin{align*}
        \psi(\omega^A(a_1\tensor \cdots \tensor a_n)) &= \chi(g_1 \cdots g_n) \omega^A(a_1\tensor \cdots \tensor a_n)\\
        &=\chi(g_1) \cdots \chi(g_n) \omega^A(a_1\tensor \cdots \tensor a_n)\\
        &= \omega^A(\chi(g_1)a_1\tensor \cdots \tensor \chi(g_n)a_n)\\
        &= \omega^A(\psi(a_1)\tensor \cdots \tensor \psi(a_n)),
    \end{align*}
    so $\psi$ is a homomorphism.
    
    Conversely, let, again, $\omega \in \Omega_n$ and $a_1, \ldots, a_n \in A$ be homogeneous elements of degrees $g_1, \ldots, g_n \in G$. 
    Since $\psi$ is an automorphism, we have:
    \begin{align*}
        \psi(\omega^A(a_1\tensor \cdots \tensor a_n)) &= \omega^A(\psi(a_1)\tensor \cdots \tensor \psi(a_n))\\
        &= \omega^A(\chi(g_1)a_1\tensor \cdots \tensor \chi(g_n)a_n)\\
        &=\chi(g_1) \cdots \chi(g_n) \omega^A(a_1\tensor \cdots \tensor a_n)\\
        &=\chi(g_1 \cdots g_n) \omega^A(a_1\tensor \cdots \tensor a_n),
    \end{align*}
    hence, by item \eqref{item:action-to-grading} in Theorem \ref{thm:g-hat-correspondence}, $\omega^A(a_1\tensor \cdots \tensor a_n)$ is homogeneous of degree $g_1 \cdots g_n$, \ie, $\omega^A$ preserves degrees.
\end{proof}

Now let $A$ be an $\Omega$-algebra and $B$ be an $\Omega'$-algebra. 
If there is a group homomorphism $\theta\from \Aut(A) \to \Aut(B)$, then, as consequence of Proposition \ref{prop:g-hat-Aut-A}, we can use it to transfer $G$-gradings on $A$ to $G$-gradings on $B$, even if $\Omega \neq \Omega'$. 
Explicitly, for a $G$-grading $\Gamma$ on $A$, we consider the group homomorphism $\eta_\Gamma\from \widehat G \to \Aut (A)$ and then take the composition $\theta \circ \eta_\Gamma\from \widehat G \to \Aut(B)$, which in turn corresponds to a $G$-grading on $B$. 
We will denote the grading on $B$ by $\theta(\Gamma)$.

% --------------------------------------------------
\section{Universal grading group, equivalence and isomorphism of gradings}\label{ssec:universal_group}
% --------------------------------------------------

% Def: Set gradings and Universal Group
There is a concept of grading not involving groups. A \emph{set grading} on a (super)algebra $A$ is a decomposition $\Gamma:\,A=\bigoplus_{s\in S}A_s$ as a direct sum of sub\-(su\-per)\-spa\-ces indexed by a set $S$ and having the property that, for any $s_1,s_2\in S$ with $A_{s_1}A_{s_2}\ne 0$, there exists $s_3\in S$ such that $A_{s_1}A_{s_2}\subseteq A_{s_3}$. The \emph{support} of $\Gamma$ (or of $A$) is defined to be the set $\supp(\Gamma) := \{s\in S \mid A_s \neq 0\}$.
Similarly, $\supp_\bz(\Gamma) := \{s\in S \mid A_s^\bz \neq 0\}$ and $\supp_\bo(\Gamma) := \{s\in S \mid A_s^\bo \neq 0\}$.

For a set grading $\Gamma:\,A=\bigoplus_{s\in S}A_s$, there may or may not exist a group $G$ containing $\supp(\Gamma)$ that makes $\Gamma$ a $G$-grading. 
If such a group exists, $\Gamma$ is said to be a {\em group grading}. (As already mentioned, we only consider abelian group gradings in this paper.) 
However, $G$ is usually not unique even if we require that it should be generated by $\supp(\Gamma)$. 
The {\em universal (abelian) grading group of $\Gamma$} \cite{PZ} is generated by $\supp(\Gamma)$ and has the defining relations 
$s_1s_2=s_3$ for all $s_1,s_2,s_3\in S$ such that $0\neq A_{s_1}A_{s_2}\subseteq A_{s_3}$. 
This group is universal among all (abelian) groups that realize the grading $\Gamma$ (see e.g. \cite[Chapter 1]{livromicha} for details).

% Def: Equivalence and Isomorphism
Let $\Gamma:\,A=\bigoplus_{g\in G} A_g$ and $\Delta:\,B=\bigoplus_{h\in H} B_h$ be two group gradings on the (super)algebras $A$ and $B$, with supports $S$ and $T$, respectively.
We say that $\Gamma$ and $\Delta$ are {\em equivalent} if there exists an isomorphism of (super)algebras $\vphi: A\to B$ and a bijection $\alpha: S\to T$ such that $\vphi(A_s)=B_{\alpha(s)}$ for all $s\in S$. If $G$ and $H$ are universal grading groups then $\alpha$ extends to an isomorphism $G\to H$. In the case $G=H$, the $G$-gradings $\Gamma$ and $\Delta$ are {\em isomorphic} if $A$ and $B$ are isomorphic as $G$-graded (super)algebras, i.e., if there exists an isomorphism of (super)algebras $\vphi: A\to B$ such that $\vphi(A_g)=B_g$ for all $g\in G$.

% Def: Coarsening and refinement
If $\Gamma:\,A=\bigoplus_{g\in G} A_g$ and $\Gamma':\,A=\bigoplus_{h\in H} A'_h$ are two gradings on the same (super)algebra $A$, with supports $S$ and $T$, respectively, then we will say that $\Gamma'$ is a {\em refinement} of $\Gamma$ (or $\Gamma$ is a {\em coarsening} of $\Gamma'$) if, for any $t\in T$, there exists (unique) $s\in S$ such that $A'_t\subseteq A_s$. If, moreover, $A'_t\ne A_s$ for at least one $t\in T$, then the refinement is said to be {\em proper}. A grading $\Gamma$ is said to be {\em fine} if it does not admit any proper refinements. 
Note that if $A$ is a superalgebra then $A=\bigoplus_{(g,i)\in G\times\mathbb{Z}_2}A_g^i$ is a refinement of $\Gamma$. 
It follows that if $\Gamma$ is fine then the sets $\supp_\bz(\Gamma)$ and $\supp_\bo(\Gamma)$ are disjoint. 
If, moreover, $G$ is the universal group of $\Gamma$, then the superalgebra structure on $A$ is given by the unique homomorphism $p: G \to \ZZ_2$ 
that sends $\supp_\bz(\Gamma)$ to $\bar 0$ and $\supp_\bo(\Gamma)$ to $\bar 1$.
% Def: Coarsening by homomorphism

\begin{defi}
	Let $G$ and $H$ be groups, $\alpha:G\to H$ be a group homomorphism and $\Gamma:\,A=\bigoplus_{g\in G} A_g$ be a $G$-grading. The \emph{coarsening of $\Gamma$ induced by $\alpha$} is the $H$-grading ${}^\alpha \Gamma: A= \bigoplus_{h\in H} B_h$ where
	$ B_h = \bigoplus_{g\in \alpha\inv (h)} A_g$. (This coarsening is not necessarily proper.)
\end{defi}

% The support is the universal group of a division grading
\begin{lemma}\label{lemma:div-grd-unvrsl-grp}
    Let $R$ be a (super)algebra and let $\Gamma$ be a division grading on it, realized over its support $T$. 
    If $\Gamma'$ is a (not necessarily proper) coarsening of $\Gamma$, realized over a group $H$, then there is a unique group homomorphism $\alpha\from T \to H$ such that $\Gamma' = {}^\alpha \Gamma$.
\end{lemma}

The following result appears to be ``folklore''. We include a proof for completeness.

% Lemma
\begin{lemma}\label{lemma:universal-grp}
	Let $\mathcal{F}=\{\Gamma_i\}_{i\in I}$, be a family of pairwise nonequivalent fine (abelian) group gradings on a (super)algebra $A$, where $\Gamma_i$ is a $G_i$-grading and $G_i$ is generated by $\supp(\Gamma_i)$. Suppose that $\mathcal{F}$ has the following property: 
	for any grading $\Gamma$ on $A$ by an (abelian) group $H$, there exists $i\in I$ and a homomorphism $\alpha:G_i\to H$ such that $\Gamma$ 
	is isomorphic to ${}^\alpha\Gamma_i$. Then
	%
	\begin{enumerate}[(i)]
		\item every fine (abelian) group grading on $A$ is equivalent to a unique $\Gamma_i$;
		\item for all $i$, $G_i$ is the universal (abelian) group of $\Gamma_i$.
	\end{enumerate}
\end{lemma}

\begin{proof}
	Let $\Gamma$ be a fine grading on $A$, realized over its universal group $H$. Then there is $i\in I$ and $\alpha: G_i \to H$ such that ${}^\alpha \Gamma_i \iso \Gamma$. Writing $\Gamma_i: A = \bigoplus_{g\in G_i} A_g$ and $\Gamma: A = \bigoplus_{h\in H} B_h$, we then have $\vphi \in \Aut(A)$ such that
	\[
		\vphi\,\big( \bigoplus_{g\in \alpha\inv (h)} A_g \big) = B_h
	\]
	for all $h\in H$. Since $\Gamma$ is fine, we must have $B_h \neq 0$ if, and only if, there is a unique $g\in G_i$ such that $\alpha(g) = h$, $A_g\neq 0$ and $\vphi(A_g) = B_h$. Equivalently, $\alpha$ restricts to a bijection $\supp(\Gamma_i) \to \supp(\Gamma)$ and $\vphi(A_g) = B_{\alpha(g)}$ for all $g \in S_i:= \supp (\Gamma_i)$. This proves assertion $(i)$.

	Let $G$ be the universal group of $\Gamma_i$. It follows that, for all $s_1, s_2, s_3 \in S_i$,
	%
	\begin{equation*} \label{eq:relations-unvrsl-grp}
		\begin{split}
			& s_1s_2 = s_3 \text{ is a defining relation of } G \\
									 \iff & 0 \neq A_{s_1} A_{s_2} \subseteq A_{s_3}\\
									 \iff & 0 \neq B_{\alpha(s_1)} B_{\alpha(s_2)} \subseteq B_{\alpha (s_3)}\\
									 \iff & \alpha(s_1)\alpha(s_2) = \alpha(s_3) \text{ is a defining relation of } H.
		\end{split}
	\end{equation*}
	%
	Therefore, the bijection $\alpha\restriction_{S_i}$ extends uniquely to an isomorphism $\widetilde{\alpha}: G\rightarrow H$.

	By the universal property of $G$, there is a unique homomorphism $\sigma: G\to G_i$ that restricts to the identity on $S_i$. Hence, the following diagram commutes:
	%
	\begin{center}
		\begin{tikzcd}
			G \arrow[to=Gi, "\sigma"] \arrow[to = H, "\widetilde{\alpha}"]&&\\
			&& |[alias=H]|H\\
			|[alias=Gi]|G_i \arrow[to=H, "\alpha"]&&
		\end{tikzcd}
	\end{center}
	%

	Since $\widetilde{\alpha}$ is an isomorphism, $\sigma$ must be injective. But $\sigma$ is also surjective since $S_i$ generates $G_i$. Hence $G_i$ is isomorphic to $G$. Since $\Gamma$ was an arbitrary fine grading, for each given $j\in I$, we can take $\Gamma = \Gamma_j$ (hence, $i=j$ and $H=G$). This concludes the proof of $(ii)$.
\end{proof}

\begin{defi}[\cite{PZ}]
	Let $\Gamma$ be a grading on an algebra $A$. We define $\Aut(\Gamma)$ as the group of all self-equivalences of $\Gamma$, i.e., automorphisms of $A$ that 
	permute the components of $\Gamma$. Let $\operatorname{Stab}(\Gamma)$ be the subgroup of $\Aut(\Gamma)$ consisting of the automorphisms that fix 
	each component of $\Gamma$. Clearly, $\operatorname{Stab}(\Gamma)$ is a normal subgroup of $\Aut(\Gamma)$, so we can define the \emph{Weil group} of 
	$\Gamma$ by $\operatorname W (\Gamma) := \frac{\Aut(\Gamma)}{\operatorname{Stab}(\Gamma)}$. The group $\operatorname W (\Gamma)$ can be seen as a subgroup
	of the permutation group of the support and also as a subgroup of the automorphism group of the universal group of $\Gamma$.
\end{defi}


% \section{Gradings on matrix algebras}
% \label{sec:gradings-on-matrix-algebras}

% In this section we will recall the classification of gradings on matrix algebras \cite{BSZ01, BZ02, BK10}. We will follow the exposition of \cite[Chapter 2]{livromicha} but use slightly different notation, which will be extended to superalgebras in Section~??%\ref{sec:Mmn}.

% The following is the graded version of a classical result (see e.g. \cite[Theorem 2.6]{livromicha}). 
% We recall that a \emph{graded division algebra} is a graded unital associative algebra such that every nonzero homogeneous element is invertible.

% \begin{thm}\label{thm:End-over-D}
% 	Let $G$ be a group and let $R$ be a $G$-graded associative algebra that has no nontrivial graded ideals and satisfies the descending chain condition on 
% 	graded left ideals. Then there is a $G$-graded division algebra $\D$ and a graded (right) $\D$-module $\mc{V}$ such that $R \simeq \End_{\D} (\mc{V})$ as graded algebras.\qed
% \end{thm}

% We apply this result to the algebra $R=M_n(\FF)$ equipped with a grading by an \emph{abelian} group $G$. We will now introduce the parameters that determine 
% $\mc D$ and $\mc V$, and give an explicit isomorphism $\End_{\D} (\mc{V})\simeq M_n(\FF)$ (see Definition \ref{def:explicit-grd-assoc}). It should be mentioned that a classification of $G$-gradings on $M_n(\FF)$ is known for non-abelian $G$ but it is less explicit and involves cohomological data (see \cite[Corollary 2.22]{livromicha} and \cite[Theorem 1.3]{GS}); here we restrict ourselves to the abelian case.

% Let $\D$ be a finite-dimensional $G$-graded division algebra. It is easy to see that $T= \supp \D$ is a finite subgroup of $G$. 
% Also, since we are over an algebraically closed field, each homogeneous component $\D_t$, for $t\in T$, is one-dimensional. We can choose a generator $X_t$ for each $\D_t$. It follows that, for every $u,v\in T$, there is a unique nonzero scalar $\beta (u,v)$ such that $X_u X_v = \beta (u,v) X_v X_u$. Clearly, $\beta (u,v)$ does not depend on the choice of $X_u$ and $X_v$.
% The map $\beta: T\times T \rightarrow \FF^{\times}$ is a \emph{bicharacter}, \ie, both maps $\beta(t,\cdot)$ and $\beta(\cdot,t)$ are characters for every $t \in T$. It is also \emph{alternating} in the sense that $\beta (t,t) = 1$ for all $t\in T$. We define the \emph{radical} of $\beta$ as the set $\rad \beta = \{ t\in T \mid \beta(t, T) = 1 \}$. In the case we are interested in, where $\D$ is simple as an algebra, the bicharacter $\beta$ is \emph{nondegenerate}, \ie, $\rad \beta = \{e\} $. The isomorphism classes of $G$-graded division algebras that are finite-dimensional and simple as algebras are in one-to-one correspondence with the pairs $(T,\beta)$ where $T$ is a finite subgroup of $G$ and $\beta$ is an alternating nondegenerate bicharacter on $T$ (see e.g. \cite[Section 2.2]{livromicha} for a proof).

% Using that the bicharacter $\beta$ is nondegenerate, we can decompose the group $T$ as $A\times B$, where the restrictions of $\beta$ to each of the subgroups $A$ and $B$ is trivial, and hence $A$ and $B$ are in duality by $\beta$. We can choose the elements $X_t\in \D_t$ in a convenient way (see \cite[Remark 2.16]{livromicha} and 
% \cite[Remark 18]{EK15}) such that $X_{ab}=X_aX_b$ for all $a\in A$ and $b\in B$. Using this choice, we can define an action of $\D$ on the vector space underlying the group algebra $\FF B$, by declaring $X_a\cdot e_{b'} = \beta(a, b') e_{b'}$ and $X_b\cdot e_{b'} = e_{bb'}$. 
% This action allows us to identify $\D$ with $\End{(\FF B)}$. Using the basis $\{e_{b}\mid b\in B\}$ in $\FF B$, we can see it as a matrix algebra, where
% %
% \[X_{ab}= \sum_{b'\in B} \beta(a, bb') E_{bb', b'}\]
% %
% and $E_{b'', b'}$ with $b'$, $b'' \in B$, is a matrix unit, namely, the matrix of the operator that sends $e_{b'}$ to $e_{b''}$ and sends all other basis elements to zero.

% \begin{defi}
% 	We will refer to these matrix models of $\mc D$ as its \emph{standard realizations}.
% \end{defi}

% \begin{remark}\label{rmk:2-grp-transp}
% 	The matrix transposition is always an involution of the algebra structure. As to the grading, we have
% 	%
% 	\[
% 	X_{ab}\transp = \sum_{b'\in B} \beta(a, bb') E_{b',bb'}
% 	   = \beta(a,b) \sum_{b''\in B} \beta(a, b^{-1}b'') E_{b^{-1}b'', b''} = \beta(a,b) X_{ab^{-1}}.
% 	\]
% 	%
% 	It follows that if $T$ is an elementary 2-group, then the transposition preserves the degree. 
% 	In this case, we will use it to fix an identification between the graded algebras $\D$ and $\D\op$.
% \end{remark}


% %\subsubsection{Graded modules over $\D$}

% Graded modules over a graded division algebra $\mc D$ behave similarly to vector spaces. The usual proof that every vector space has a basis, with obvious modifications, shows that every graded $\mc D$-module has a \emph{homogeneous basis}, \ie, a basis formed by homogeneous elements.
% Let $\mc V$ be such a module of finite rank $k$, fix a homogeneous basis $\mc B = \{v_1, \ldots, v_k\}$ and let $g_i := \operatorname{deg} v_i$. We then have $\mc{V}\iso \ \D^{[g_1]}\oplus\cdots\oplus\D^{[g_k]}$, so, the graded $\mc D$-module $\mc V$ is determined by the $k$-tuple $\gamma = (g_1,\ldots, g_k)$. The tuple $\gamma$ is not unique. To capture the precise information that determines the isomorphism class of $\mc V$, we use the concept of \emph{multiset}, \ie, a set together with a map from it to the set of positive integers. If $\gamma = (g_1,\ldots, g_k)$ and $T=\supp \D$, we denote by $\Xi(\gamma)$ the multiset whose underlying set is $\{g_1 T,\ldots, g_k T\} \subseteq G/T$ and the multiplicity of $g_i T$, for $1\leq i\leq k$, is the number of entries of $\gamma$ that are congruent to $g_i$ modulo $T$.

% Using $\mc B$ to represent the linear maps by matrices in $M_k(\D) = M_k(\FF)\tensor \D$, we now construct an explicit matrix model for $\End_{\D}(\mc V)$.

% \begin{defi}\label{def:explicit-grd-assoc}
% 	Let $T \subseteq G$ be a finite subgroup, $\beta$ a nondegenerate alternating bicharacter on $T$, and $\gamma = (g_1, \ldots, g_k)$ a $k$-tuple of elements of $G$. Let $\D$ be a standard realization of a graded division algebra associated to $(T, \beta)$. Identify $M_k(\FF)\tensor \D \iso M_n(\FF)$ by means of the Kronecker product, where $n=k\sqrt{|T|}$. We will denote by $\Gamma(T, \beta, \gamma)$ the grading on $M_n(\FF)$ given by $\deg (E_{ij} \tensor d) := g_i (\deg d) g_j\inv$ for $i,j\in \{1, \ldots , k\}$ and homogeneous $d\in \D$, where $E_{ij}$ is the $(i,j)$-th matrix unit.
% \end{defi}

% If $\End(V)$, equipped with a grading, is isomorphic to $M_n(\FF)$ with $\Gamma(T, \beta, \gamma)$, we may abuse notation and also denote the grading on $\End(V)$ by $\Gamma(T,\beta,\gamma)$.
% We restate \cite[Theorem 2.27]{livromicha} (see also \cite[Theorem 2.6]{BK10}) using our notation:

% \begin{thm}\label{thm:classification-matrix}%3.5
% 	Two gradings, $\Gamma(T,\beta,\gamma)$ and $\Gamma(T',\beta',\gamma')$, on the algebra $M_n(\FF)$ are isomorphic if, and only if, $T=T'$, $\beta=\beta'$ and there is an element $g\in G$ such that $g \Xi(\gamma)=\Xi(\gamma')$.\qed
% \end{thm}

% The proof of this theorem is based on the following result (see Theorem 2.10 and Proposition 2.18 from \cite{livromicha}), which will also be needed:

% \begin{prop}\label{prop:inner-automorphism}
% 	If $\phi: \End_\D (\mc V) \rightarrow \End_\D (\mc V')$ is an isomorphism of graded algebras, then there is a homogeneous invertible 
% 	$\D$-linear map $\psi: \mc V\rightarrow \mc V'$ such that $\phi(r)=\psi \circ r \circ \psi\inv$, for all $r\in \End_\D (\mc V)$.\qed
% \end{prop}

% \end{document}