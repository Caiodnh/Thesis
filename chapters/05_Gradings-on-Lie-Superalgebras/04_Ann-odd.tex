We now proceed to the last case, the odd gradings of Type II, which will be referred to as gradings of Type II\textsubscript{Q}. 

\begin{remark}\label{prop:Q-implies-GxZZ2-grading-on-M-II}
    Similar to \cref{prop:Q-implies-GxZZ2-grading-on-M}, we have that
    every $G$-grading of Type II on the Lie superalgebra $Q(n)$ gives rise to a $G\times \ZZ_2$-grading of Type II\textsubscript{Q} on $A(n,n)$, which justifies the notation.  
    % Indeed, the associative superalgebra $Q(n+1)$ consists of the fixed points of the order $2$ automorphism $\pi$ of $M(n+1, n+1)$, and any automorphism of $Q(n+1)$ extends uniquely to an automorphism of $M(n+1, n+1)$ commuting with $\pi$ (see \cref{sec:Aut-Lie-chap}). 
\end{remark}

At the end of \cref{sec:MxM-and-QxQ-associative}, just before \cref{cor:MxMsop-odd-only-G}, we introduced a parametrization of odd gradings on $(R, \vphi)$ that make $R$ graded-simple. 
By \cref{cor:MxMsop-odd-only-G}, $(R, \vphi)$ endowed with such a grading is isomorphic to $M^{\mathrm{ex}}(T^+, \beta^+, t_p, h, \kappa, g_0)$, where $T^+ \subseteq G$ is a finite $2$-elementary subgroup, $e\neq t_p \in T^+$, $h \in G$ is such that $f \coloneqq h^2 \in T^+ \setminus \langle t_p \rangle$, $\beta^+\from T^+ \times T^+ \to \pmone$ is an alternating bicharacter such that $\rad \beta^+ = \langle t_p, f \rangle$, $g_0\in G$, and $\kappa\from G/T^+ \to \ZZ_{\geq 0}$ is a $g_0$-admissible map (see \cref{defi:odd-D-kappa-g_0-admissible}) such that $n+1 = |\kappa| \sqrt{|T^+|}/2$. 

We will use the parameters $(T^+, \beta^+, t_p, h, \kappa, g_0)$ to construct a representative of the corresponding isomorphism class of Type II\textsubscript{Q} gradings directly on the superalgebra $L$ instead of going through $\Skew(R, \vphi)$. 
Let $\tilde S$ denote the algebra $M_{n+1}(\FF)$. 
Using the Kronecker product, we can identify $S = M(n+1, n+1)$ with $M(1, 1) \tensor \tilde S$.

Let $\barr G \coloneqq G/\langle f \rangle$. 
We will, first, construct a $\barr G$-grading and a (super-)anti-automorphism on $\tilde S$, in a fashion similar to what we did for $S\even$ in \cref{ssec:grds-on-Q(n)}, and, then, we will extend the grading and the super-anti-automorphism to $S$. 
Fix a subgroup $K \subseteq T^+$ such that $T^+ = K \times (\rad \beta^+)$. 
Let $\pi\from G \to \barr G$ denote the natural homomorphism, set $\barr {T^+} \coloneqq \pi(T^+)$ and $\barr {K} \coloneqq \pi(K)$, let $\barr {\beta^+}\from \barr {T^+} \times \barr {T^+} \to \FF^\times$ be the bicharacter on $\barr {T^+}$ induced by $\beta^+$, and consider $\kappa$ as a map defined on $\barr G/\barr {T^+} \iso G/{T^+}$. 
Also, let $\chi,\chi_0 \from T^+ \to \FF^\times$ be the characters defined by $\chi(K) = 1$, $\chi(f) = -1$ and $\chi(t_p) = 1$, and $\chi_0(K) = 1$, $\chi_0(f) = 1$ and $\chi_0(t_p) = -1$ (we note that $\chi_0$ was denoted by $\chi$ in \cref{eq:equivalent-def-of-chi}). 

Since $\barr {\beta^+}\restriction_{\barr K \times \barr K}$ is nondegenerate, we have a chosen standard realization $\widetilde \D$ associated to $(\barr{K}, \barr{\beta^+}\restriction_{\barr K \times \barr K})$. 
Let $ {\widetilde\mu}\from \barr{K} \to \FF^\times$ be the map associated to the transposition on $\widetilde \D$, and let $\barr{\mu^+}\from \barr{T^+} \to \FF^\times$ be the map extending $\widetilde\mu$ defined by $\overline{\mu^+}(\bar t \bar t_p) = \overline{\mu^+}(\bar t) \coloneqq \widetilde\mu (\bar t)$, for all $\bar t\in \barr K$. 
Set $\mu^+ \coloneqq \barr {\mu^+} \circ \pi\restriction_{T^+}$ and define $\eta^+\from T^+ \to \pmone$ by
\[\label{eq:fix-eta-plus-undouble-Ann}
    \forall t\in T^+, \quad \eta^+(t) \coloneqq \mu^+(t) \chi\inv(t).
\] 
Since $\mathrm{d} \widetilde\mu = \barr{\beta^+}\restriction_{\barr K \times \barr K}$, we get $\mathrm{d} \overline{\mu^+} = \barr{\beta^+}$ and, hence, $\mathrm{d} \eta^+ = \beta^+$.  
In the proof of \cref{prop:Ann-Type-II-correspondence}, we will show that $\eta^+$ is the map associated to the superinvolution on the even part of a graded-division superalgebra as in \cref{def:std-realization-MxM-QxQ}(b). 

Now, consider the (division) $\barr G$-grading on $M(1,1)$ given by:
%
\[\label{eq:bar-G-grd-on-M(1-1)}
\begin{aligned}
	\deg \left(\begin{array}{c|c}
		 \phantom{.}1\phantom{.} & \phantom{-}0 \\
		\hline
		 \phantom{.}0\phantom{.} & \phantom{-}1 
	\end{array}\right) &= \bar e,\quad & \deg \left(\begin{array}{c|c}
		 \phantom{.}0\phantom{.} & \phantom{-}1 \\
		\hline
		 \phantom{.}1\phantom{.} & \phantom{-}0 
	\end{array}\right) &= \bar h, \\
	\deg \left(\begin{array}{c|c}
		 \phantom{.}1\phantom{.} & \phantom{-}0 \\
		\hline
		 \phantom{.}0\phantom{.} & -1 
	\end{array}\right) &= \bar t_p,\quad &
	\deg \left(\begin{array}{c|c}
		 \phantom{.}0\phantom{.} & -1           \\
		\hline
		 \phantom{.}1\phantom{.} & \phantom{-}0 
	\end{array}\right) &= \bar t_p \bar h.
\end{aligned}
\]
(Compare with \cref{ex:Pauli-2x2,ex:Pauli-2x2-super}.) 
Set $k \coloneqq |\kappa|$ and fix a set-theoretic section $\xi\from G/T^+ \to G$ of the natural homomorphism $G \to G/T^+$. 
As we did in \cref{ssec:grds-on-Q(n)}, we follow the construction after \cref{eq:fix-eta-undouble} in \cref{ssec:grds-on-A-m-n} with $\kappa_\bz \coloneqq \kappa$ and $\kappa_\bo$ being the zero map to construct an elementary $\barr G$-grading on $M_k(\FF) = M(k , 0)$. 
We identify the $\barr G$-graded superalgebra $M_{k}(\widetilde \D) = M_{k}(\FF) \tensor \widetilde \D$ with $\tilde S = M_{n+1}(\FF)$ via Kronecker product and, then, get a $\barr G$-grading on $S = M(n+1, n+1) = M(1,1) \tensor \tilde S$. 

\begin{remark}\label{rmk:change-M(1-1)-of-place}
    Note that $S = M(1,1) \tensor \tilde S = M(1,1) \tensor M_k(\FF) \tensor \widetilde \D \iso M_k(\FF) \tensor M(1,1) \tensor \widetilde\D$. 
    Hence, this $\barr G$-grading is isomorphic to $\Gamma_M(\barr {T^+}, \barr {\beta^+}, \bar t_p, \kappa)$ (see \cref{def:Gamma_M-only-G}), where the fixed character is taken to be $\chi_0$. 
\end{remark}

We will now construct a super-anti-automorphism on $S$. 
The next definition is similar to \cref{defi:blocks-of-Theta}. 
% Note that, since $\chi_0(f) = 1$, we can see it as a character on $\barr{T^+}$. 

\begin{defi}\label{defi:blocks-of-Theta-Ann}
    Let $x \in G/T^+$. 
    If $g_0x^2 = T^+$, we put $t \coloneqq g_0 \xi(x)^2 \in T^+$ and $\tilde t \coloneqq \pi(\mathrm{pr}_{K} (t) )$, where $\mathrm{pr}_{K}\from {T^+} \to K$ is the projection on $K$ corresponding to the decomposition ${T^+} = K \times (\rad \beta^+)$. 
    We define $\delta_x \coloneqq \chi_0(t)$ and $\Theta_K(x)$ to be the following $\kappa(x) \times \kappa(x)$-matrix with entries in $\widetilde \D$:
    %
    \begin{enumerate}[(i)]
        \item $I_{\kappa(x)} \tensor X_{\tilde t}$ if $\eta^+(t) = +1$;
        %
		\item  $J_{\kappa(x)} \tensor X_{\tilde t}$, where $J_{\kappa(x)} \coloneqq \begin{pmatrix}
				      0                & I_{\kappa(x)/2} \\
				      -I_{\kappa(x)/2} & 0
			      \end{pmatrix}$, if  $\eta^+(t) = -1$ (recall that, in this case, $\kappa(x)$ is even by \cref{defi:odd-D-kappa-g_0-admissible}). 
	\end{enumerate}
    %
    \noindent If $g_0 x^2 \neq T^+$, we define $\delta_x = 1$ and $\Theta_K(x)$ to be the following $2\kappa(x) \times 2\kappa(x)$-matrix:
    %
    \begin{enumerate}[(i)]
        %
        \setcounter{enumi}{2}
        %
		\item $\begin{pmatrix}
			0  &  I_{\kappa(x)} \\
			\chi(g_0 \xi(x)^2)\inv I_{\kappa(x)} & 0
		\end{pmatrix} \tensor 1_{\widetilde \D}$. 
    \end{enumerate}
\end{defi}

Let $x_1 < \cdots < x_{\ell}$ be the elements of the set $\{ x \in \supp \kappa \mid x \leq g_0\inv x\inv \}$, and set $\Theta \in S$ to be the matrix
\[\label{eq:puting-the-blocks-of-Theta-Ann}
    %
    \sbox0{$\begin{matrix}
        \Theta_K(x_1)&& \\
        & \ddots &\\
        && \Theta_K(x_{\ell})
    \end{matrix}$}
    %
    \sbox1{$\begin{matrix}
        \delta_{x_1}\Theta_K(x_1)&& \\
        & \ddots &\\
        && \delta_{x_\ell}\Theta_K(x_{\ell})
    \end{matrix}$}
    %
    \Theta \coloneqq
    \left(\begin{array}{c|c}
            \usebox{0} & 0\\
            \hline
            0 & \usebox{1}
        \end{array}\right).
\]
We, then, define the super-anti-automorphism $\theta\from S \to S$ by 
\[\label{eq:theta-with-matrix-6}
    \forall X\in M(n+1, n+1), \quad \theta(X) \coloneqq \Theta\inv\, X\sTq\, \Theta.
\]
%
%
% We define $\Theta \in S$ by 
% \[\label{eq:Theta_K-for-Ann}
%     \widetilde\Theta \coloneqq \begin{pmatrix}
%         \Theta(\bar 0, x_1)&& \\
%         & \ddots &\\
%         && \Theta(\bar 0, x_{\ell})
%     \end{pmatrix},
% \]
% where $x_1 < \ldots < x_{\ell}$ are the elements of the set $\{ x \in \supp \kappa \mid x \leq g_0\inv x\inv \}$, and $\Theta(\bar 0, x)$, for $x\in \barr G/\barr {T^+}$, is as in \cref{defi:blocks-of-Theta}.
%
Finally, we define a $G$-grading on $S^{(1)}$ by 
\[\label{eq:final-G-grd-on-Ann}
    \forall g\in G, \quad S^{(1)}_g \coloneqq \{ s \in S^{(1)}_{\barr g} \mid \theta(s) = - \chi(g) s\},
\]
and reduce it modulo the center to obtain a $G$-grading on $L$. 

In summary:

\begin{defi}\label{defi:type-II-Ann}
    Let $n \in \ZZ_{> 0}$, and denote the associative superalgebra $M(n+1, n+1)$ by $S$. 
    Let $T^+ \subseteq G$ be a $2$-elementary subgroup, let $e\neq t_p \in T^+$, let $h \in G$ be such that $f \coloneqq h^2 \in T^+ \setminus \langle t_p \rangle$, and let $\beta^+\from T^+ \times T^+ \to \pmone$ be an alternating bicharacter such that $\rad \beta^+ = \langle t_p, f \rangle$. 
    Let $\pi\from G\to \barr G \coloneqq G/\langle f \rangle$ be the natural homomorphism, fix a subgroup $K \subseteq T^+$ such that $T^+ = K \times \langle f \rangle$, set $\barr {T^+} \coloneqq \pi(T^+)$ and $\barr K \coloneqq \pi(K)$, and let $\barr {\beta^+}$ be the alternating bicharacter on $\barr {T^+}$ induced by $\beta^+$. 
    Consider the chosen standard realization $\widetilde \D$ of a matrix algebra with division grading associated to $(\barr {K}, \barr{\beta^+}\restriction_{\barr K \times \barr K})$, and define $\eta^+\from T \to \pmone$ by \cref{eq:fix-eta-plus-undouble-Ann}. 
    Then, let $g_0 \in G$ be any element and let $\kappa\from G/{T^+} \to \ZZ_{\geq 0}$ be a $g_0$-admissible map (\cref{defi:odd-D-kappa-g_0-admissible}) such that $n+1 = |\kappa| \sqrt{|T^+|}/2$. 
    Choose:
    \begin{enumerate}[(i)]
        \item a set-theoretic section $\xi\from G/T^+ \to G$ for the natural homomorphism $G \to G/T^+$;
        \label{item:choice-xi-Ann}
        %
        \item a total order $\leq$ on $G/T^+$ such that there are no elements between $x$ and $\bar g_0\inv x\inv$, for all $x\in G/T^+$; 
        \label{item:choice-leq-Ann}
    \end{enumerate}
    and construct a tuple $\bar\gamma$ realizing $\kappa$ according to $\pi \circ \xi$ and $\leq$ (\cref{defi:tuple-governed}). 
    Consider the $\barr G$-grading on $M(1,1)$ given by \cref{eq:bar-G-grd-on-M(1-1)} and the $\barr G$-grading ${\Gamma_M(\barr {K}, \barr {\beta^+}\restriction_{\barr K \times \barr K}, \kappa)}$ on $M_{n+1}(\FF)$ constructed using the choices of $\widetilde \D$ and $\barr \gamma$ above (see \cref{def:Gamma-T-beta-kappa}).  
    We, then, identify $S \coloneqq M(n+1, n+1)$ with the graded superalgebra $M(1,1) \tensor M_{n+1}(\FF)$. 
    Define $\Theta \in S$ by \cref{eq:puting-the-blocks-of-Theta-Ann} and ${\theta\from S \to S}$ by
    \cref{eq:theta-with-matrix-6}
    Finally, we define $\Gamma_A^{\mathrm{(II_Q)}}(T^+, \beta^+, t_p, h, \kappa, g_0)$ to be the $G$-grading on $L = S^{(1)}/Z(S^{(1)})$ induced from the $G$-grading $S^{(1)}$ given by \cref{eq:final-G-grd-on-Ann}.
\end{defi}

\begin{prop}\label{prop:Ann-Type-II-correspondence}
    Consider $(R, \vphi) \coloneqq M^{\mathrm{ex}}(T^+, \beta^+, t_p, h, \kappa, g_0)$, as defined before \cref{cor:MxMsop-odd-only-G}. 
    Then the graded Lie superalgebra $\Skew (R,\vphi)^{(1)}/Z(\Skew (R,\vphi)^{(1)})$ is isomorphic to the Lie superalgebra $A(n,n)$ endowed with $\Gamma_A^{\mathrm{(II_Q)}}(T^+, \beta^+, t_p, h, \kappa, g_0)$. 
\end{prop}

\begin{proof} 
    % Recall what M^ex means, including what $T$ and $t_p$ are, and how to see \kappa with the proper domain. 
    Recall that $M^{\mathrm{ex}}(T^+, \beta^+, t_p, h, \kappa, g_0) = M^{\mathrm{ex}} (T, \beta, t_p, \kappa, g_0)$ (see \cref{def:model-grd-MxM-odd-or-QxQ}), where $t_1 \coloneqq (h, \bar 1)$, $T \coloneqq T^+ \cup t_1 T^+$, $\beta\from T\times T \to \FF^\times$ is the unique alternating bicharacter extending $\beta^+$ such that $\rad \beta = \langle f \rangle$ and $\beta(t_1, \cdot) = \chi_0$, and $\kappa$ is seen as defined on $G^\#/T \iso G/ T^+$. 
    
    % Say what we are going to do
    We will now show how the choices in \cref{defi:type-II-Ann}  correspond to the choices in \cref{def:std-realization-MxM-QxQ}(b) and \cref{def:model-grd-MxM-odd-or-QxQ}. 
    
    % The choices of K and \barr\D\even given us the correct \eta^+, and fix \D as \M\tensor\C
    By \cref{lemma:barr-D-to-mc-M}, with $(T^+, \beta^+)$ playing the role of $(T, \beta)$, the choices of $K$ and $\tilde \D$  give us the same information as the choices of $K$ in item \eqref{item:K-can-be-orthogonal-to-t_1} and $\mc M$ in item \eqref{item:choose-mc-M} of \cref{def:std-realization-MxM-QxQ}(b), and the map associated to the transposition on $\mc M$ is $\mu^+\restriction_K$. 
    Let $(\D, \vphi_0)$ denote the standard realization of a graded-division superalgebra with superinvolution constructed using these choices. 
    Note that the map $\eta^+\from T\to \pmone$ defined in \cref{eq:fix-eta-plus-undouble-Ann} is such that $\mathrm{d}\eta^+ = \beta^+$, $\eta^+\restriction_{K} = \mu^+\restriction_K$, $\eta^+(f) = -1$ and $\eta^+(t_p) = 1$, so, by \cref{eq:eta+-unsharpening-MxM}, $\eta^+$ is the map associated to $\vphi_0\restriction_{\D\even}$. 
    In particular, the $g_0$-admissibility condition for $\kappa$ is the same in \cref{def:model-grd-MxM-odd-or-QxQ,defi:type-II-Ann}. 
    
    % Constructs $(\U, B)$
    In \cref{def:model-grd-MxM-odd-or-QxQ}, we have to choose a graded right $\D$-supermodule $\U$ and a $\vphi_0$-sesquilinear form $B\from \U \times \U \to \D$ such that $(\U, B)$ has inertia determined by $\kappa$. 
    To this end, set $\U \coloneqq \D^{[g_1]} \oplus \cdots \oplus \D^{[g_k]}$, where $(g_1, \ldots, g_k)$ is the $k$-tuple realizing $\kappa$ according to $\xi$ and $\leq$, and $B(u_i, u_j) = 
    \Phi_{ij}$, where $\B \coloneqq \{u_1, \ldots, u_k\}$ is the canonical $\D$-basis for $\U$ and $\Phi \in M_k(\D)$ is defined by
    \[
        \Phi \coloneqq 
        \begin{pmatrix}
            \Phi(\bar 0, x_1)&& \\
            & \ddots &\\
            && \Phi(\bar 0, x_{\ell})
        \end{pmatrix}
    \]
    (see \cref{defi:blocks-of-Phi,eq:puting-the-blocks-of-Phi-together}). 
    
    % Undoubling what we just constructed
    We will now follow \cref{ssec:undoubling} to undouble $M^{\mathrm{ex}}(T^+, \beta^+, t_p, h, \kappa, g_0)$, constructed with the choices above. 
    % Recall \D
    Recall that, by \cref{def:std-realization-MxM-QxQ}(b), $\D = {}^\alpha \mc O \tensor \mc M$, where $\mc O$ is the graded-division superalgebra of \cref{ex:superalgebra-O}, and $\alpha\from \ZZ_2 \times \ZZ_4 \to \langle t_p, t_1 \rangle$ is the group isomorphism given by $\alpha (\bar 1, \bar 0) \coloneqq t_p$ and $\alpha (\bar 0, \bar 1) \coloneqq t_1$. 
    % $\D$ is what it should be 
    Consider the central idempotent $\epsilon \coloneqq \frac{1}{2} (1+X_f) \in \D$ and the $\barr G$-graded-division superalgebra $\D\epsilon$. 
    Clearly, $\epsilon \in \mc O$.  
    It is straightforward that $\mc O \epsilon$ is isomorphic to $M(1,1)$ with grading given by \cref{eq:bar-G-grd-on-M(1-1)}, and that $\mc M \epsilon$ is isomorphic to $\widetilde \D$. 
    It follows that $\D\epsilon \iso \barr \D \coloneqq M(1,1) \tensor \widetilde \D$. 
    We will identify $\D\epsilon$ with $\barr \D$ and consider $\barr \D$ as a matrix superalgebra via Kronecker product. 
    % @2
    
    % Undoubling $\eta$
    We need to extend $\chi$ to $G^\#$. 
    Since $\chi(f) = -1$, we can do it in a way such that $\chi(t_1) = \bi$. 
    Let $\eta\from T \to \pmone$ be the map associated to $\vphi_0$ and let $\barr\mu\from \barr T \to \FF^\times$ be as defined in \cref{eq:defi-mu-undoubled}. 
    % We claim that  is the map associated to the queer supertranspose on $\barr \D$.  
    We have already shown that $\eta\restriction_{T^+} = \eta^+$ and, hence, by \cref{eq:fix-eta-plus-undouble-Ann}, we have that $\barr\mu\restriction_{\barr{T^+}} = \barr{\mu^+}$.
    By \cref{rmk:eta-for-MxM-or-QxQ}(b), we have $\barr\mu (\barr t_1) = \eta(t_1) \chi(t_1) = \bi$. 
    It follows that $\barr\mu$ is the map associated to the queer supertranspose on $\barr \D$. 
    
    % Undoubling $\Theta$
    As in the proof of \cref{prop:m-not-n-Type-II-correspondence}, let $\Lambda \in M_k(\D)$ be the diagonal matrix with entries $\Lambda_{ii} \coloneqq \chi(\deg u_i)$, consider a different graded $\D$-basis $\tilde \B = \{ \tilde u_1, \ldots, \tilde u_k \}$ of $\U$, where $\tilde u_i$ is defined as in \cref{eq:tilde-u_i-from-u_i}, and let $\tilde \Phi \in M_k(\D)$ be the matrix representing $B$ with respect to $\tilde \B$. 
    Note that the entries of $\tilde\Phi$ are in $\D\even = \mc ({}^\alpha\mc O)\even \tensor \mc M$, and that $\supp \D\even = (\rad \beta^+) \times K$. 
    Given $t \in T^+$, write $t= rs$, with $r\in \rad \beta^+$ and $s\in K$, and recall that we chose $X_t \in \D\even$ to be $X_r \tensor X_s$, where $X_r \in ({}^\alpha\mc O)\even$ and $X_s \in \mc M$. 
    Under the identification $\D\epsilon = \barr \D$, the element $X_r\epsilon$ becomes $
    \begin{pmatrix}
        1 & 0\\
        0 & \chi_0(r)
    \end{pmatrix} = 
    \begin{pmatrix}
        1 & 0\\
        0 & \chi_0(t)
    \end{pmatrix} \in M(1,1)\even$. 
    It follows that $\tilde \Phi \epsilon \in M_k(\barr \D)$ goes to $\Theta \in S = M(1,1)\tensor M_k(\widetilde \D)$, as defined in \cref{eq:puting-the-blocks-of-Theta-Ann}, if we follow the isomorphisms
    \[\label{eq:chain-of-isos-Ann}
        M_k(\barr \D) \iso
        M_k(\FF)\tensor \barr \D \iso
        M_k(\FF) \tensor M(1,1) \tensor \widetilde \D \iso
        M(1,1) \tensor M_k(\FF) \tensor \widetilde \D \iso
        M(1,1)\tensor M_k(\widetilde \D). 
    \]
    
    Therefore, all the data in the description of the undoubled model of $\Skew(R, \vphi)$ coincide with the data in \cref{defi:type-II-Ann}, concluding the proof.
\end{proof}

% --------------------


\begin{thm}\label{thm:final-A(n-n)}
    Every grading on the simple Lie superalgebra ${\mathfrak{psl}(n+1 | n+1)}$, $n\geq 2$, is isomorphic to one of 
    $\Gamma^{\mathrm{(I_M)}}_A(T, \beta, \kappa_\bz, \kappa_\bo)$, $\Gamma^{\mathrm{(I_Q)}}_A(T^+, \beta^+, h, \kappa)$, $\Gamma^{\mathrm{(II_{\mathfrak{osp}})}}_A(T, \beta, \kappa_\bz, \kappa_\bo, h_0)$, $\Gamma_A^{\mathrm{(II_P)}}(T, \beta, \kappa_\bz, h_0)$ or $\Gamma_A^{\mathrm{(II_Q)}}(T^+, \beta^+, t_p, h, \kappa, g_0)$, as in \cref{def:Type-I_M,def:Type-I_Q,def:type-II-osp,def:type-II-P,defi:type-II-Ann}. 
    Gradings belonging to different types are not isomorphic. 
    Within each type, we have:
    
    \noindent\boxed{\mathrm{Type \,\,I_M}}
        
        \noindent$\Gamma^{\mathrm{(I_M)}}_A(T, \beta, \kappa_\bz, \kappa_\bo) 
        \iso 
        \Gamma^{\mathrm{(I_M)}}_A(T', \beta', \kappa_\bz', \kappa_\bo')$ if, and only if, $T = T'$ and one of the following conditions holds:
	\begin{enumerate}[(i)]
	    \item $\beta'=\beta$ and there is $g\in G$ such that either $\kappa_{\bar 0}'=g \cdot \kappa_{\bar 0}$ and $\kappa_{\bar 1}'=g \cdot \kappa_{\bar 1}$, or $\kappa_{\bar 0}'=g \cdot \kappa_{\bar 1}$ and $\kappa_{\bar 1}'=g \cdot \kappa_{\bar 0}$; 
	    \item $\beta'=\beta\inv$ and there is $g\in G$ such that either $\kappa_{\bar 0}'=g \cdot \kappa_{\bar 0}\Star$ and $\kappa_{\bar 1}'=g \cdot \kappa_{\bar 1}\Star$, or $\kappa_{\bar 0}'=g \cdot \kappa_{\bar 1}\Star$ and $\kappa_{\bar 1}'=g \cdot \kappa_{\bar 0}\Star$. 
	\end{enumerate}
       
    \noindent\boxed{\mathrm{Type \,\,I_Q}}
    
        \noindent $\Gamma^{\mathrm{(I_Q)}}_A(T^+, \beta^+, h, \kappa)
        \iso    \Gamma^{\mathrm{(I_Q)}}_A(T'^+, \beta'^+, h', \kappa')$ if, and only if, $T'^+ = T^+$ and one of the following conditions holds:
    \begin{enumerate}[(i)]
	    \item $\beta'^+=\beta^+$, $h' \in \{ h, h t_p \}$ and there is $g\in G$ such that $\kappa' = g \cdot \kappa$;
	    \item $\beta'^+= (\beta^+)\inv$, $h' \in \{ h\inv, h\inv t_p \}$ and there is $g\in G$ such that $\kappa' = g \cdot \kappa\Star$; 
	\end{enumerate}
	where $t_p$ is the nontrivial element in $\rad \beta^+$. 
        
    \noindent\boxed{\mathrm{Type \,\,II}_{\osp}}
    
        \noindent$\Gamma^{\mathrm{(II_{\osp})}}_A (T, \beta, \kappa_\bz, \kappa_\bo, g_0) 
        \iso \Gamma^{\mathrm{(II_{\osp})}}_A (T', \beta', \kappa_\bz', \kappa_\bo', g_0')$ if, and only if, $T =T'$, $\beta = \beta'$ and there is $g \in G$ such that one of the following conditions holds:
    \begin{enumerate}[(i)]
        \item $\kappa_\bz' = g\cdot\kappa_\bz$, $\kappa_\bo' = g\cdot\kappa_\bo$ and $g_0' = g^{-2}g_0$;
        \item $\kappa_\bz' = g\cdot\kappa_\bo$, $\kappa_\bo' = g\cdot\kappa_\bz$ and $g_0' = fg^{-2}g_0$. 
    \end{enumerate}
        
    \noindent\boxed{\mathrm{Type \,\,II_P}}
    
        \noindent$\Gamma_A^{\mathrm{(II_P)}} 
        (T, \beta, \kappa_\bz, h_0)
        \iso 
        \Gamma_A^{\mathrm{(II_P)}} (T', \beta', \kappa_\bz', h_0')$ if, and only if, $T =T'$, $\beta = \beta'$ and there is $g \in G$ such that one of the following conditions holds:
    \begin{enumerate}[(i)]
        \item $\kappa_\bz' = g\cdot\kappa_\bz$ and $h_0' = g^{-2}h_0$;
        \item $\kappa_\bz' = gh_0\inv \cdot\kappa_\bz\Star$ and $h_0' = fg^{-2}h_0$. 
    \end{enumerate}
        
    \noindent\boxed{\mathrm{Type \,\,II_Q}}
    
        \noindent$\Gamma_A^{\mathrm{(II_Q)}} (T^+, \beta^+, t_p, h, \kappa, g_0) 
            \iso 
        \Gamma_A^{\mathrm{(II_Q)}} (T'^+, \beta'^+, t_p',  h', \kappa', g_0')$ are isomorphic if, and only if, $T^+ =T'^+$, $\beta^+ = \beta'^+$, $t_p = t_p'$, $h' \in h (\rad \beta^+)$ and there is $g \in G$ such that $\kappa' = g\cdot\kappa$ and
    \begin{enumerate}[(i)]
        \item $g_0' = g^{-2}g_0$ if $h' \in \{ h, f t_p h\}$;
        \item $g_0' = t_p g^{-2}g_0$ if $h' \in \{ f h, t_p h\}$.
    \end{enumerate}
\end{thm}

\begin{proof}% @3
    The result follows from \cref{thm:MxM-type-I} (Types I\textsubscript{M} and I\textsubscript{Q}), \cref{thm:MxM-even} (Types II\textsubscript{$\osp$} and II\textsubscript{P}, taking into account in the latter case that $\kappa_\bo = h_0\inv \cdot \kappa_\bz\Star$) 
    and \cref{cor:MxMsop-odd-only-G} (Type II\textsubscript{Q}). 
    Indeed, by \cref{cor:transfer-R-vphi-to-L}, the isomorphism classes of gradings on the Lie superalgebra $A(n,n)$ are in bijection with the isomorphism classes of gradings on the associative superalgebra $M(n+1, n+1) \times M(n+1, n+1)\sop$  endowed with the exchange superinvolution. 
    Gradings of Types I\textsubscript{M} and I\textsubscript{Q} are already described in terms of $M(n+1, n+1)$. 
    \Cref{prop:m-not-n-Type-II-correspondence} gives such a description for gradings of Type II\textsubscript{$\mathfrak{osp}$} and II\textsubscript{P}, and \cref{prop:Ann-Type-II-correspondence} for Type II\textsubscript{Q}. 
\end{proof}