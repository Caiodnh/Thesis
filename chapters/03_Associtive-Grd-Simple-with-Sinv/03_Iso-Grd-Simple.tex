\section{Isomorphism of graded-simple superalgebras\texorpdfstring{\\}{} with super-anti-automorphism}\label{sec:iso-vphi-abstract}

In this section, we are going to describe isomorphisms between $G$-graded superalgebras with super-anti-automorphism that are graded-simple and satisfy the \dcc on graded left ideals.
As we have seen, such superalgebras are, up to isomorphism, of the form $\End_\D (\U)$ where the super-anti-automorphism is given by the superadjunction with respect to a nondegenerate homogeneous sesquilinear form $B\from \U \times \U \to \D$.

Even though, as superalgebras, $\End_\D( \U )$ is the same as $\End_\D( \U^{[g]} )$ for every $g \in G^\#$, an extra care should be taken when considering super-anti-automorphisms.
If $g\in G^\#$ is odd, a $\vphi_0$-sesquilinear form $B$ on the $\D$-module $\U$ is not $\vphi_0$-sesquilinear if considered on the $\D$-module $\U^{[g]}$, and Equation \eqref{eq:superadjunction} does not determine the same super-anti-automorphism $\vphi$.
This motivates the following:

\begin{defi}\label{defi:shift-on-B}
	Let $\U$ be a graded right $\D$-module and $\vphi_0\from \D \to \D$ be a super-anti-automorphism.
	Given a homogeneous $\vphi_0$-sesquilinear form $B$ on $\U$ and $g\in G^\#$, we define the $\vphi_0$-sesquilinear form  $B^{[g]}$ on $\U^{[g]}$ by $B^{[g]}(u,v) \coloneqq \sign{u}{g} B(u,v)$, for all $u,v \in \U$.
\end{defi}

\begin{remark}\label{rmk:deg-B^[g]}
	Note that $\deg B^{[g]} = g^{-2} \deg B$ and, in particular, $|B^{[g]}| = |B|$.
\end{remark}

\begin{lemma}\label{lemma:B^[b]-does-the-job}
	% Let $\U$ be a graded $\D$-module, $\vphi_0$ be a super-anti-automorphism on $\D$ and $B$ be a $\vphi_0$-sesquilinear form on $\U$. 
	For every $g\in G^\#$, $B^{[g]}$ is a homogeneous $\vphi_0$-sesquilinear form on $\U^{[g]}$.
	% Further, if $B$ is nondegenerate and the pair $(\vphi_0, B)$ determines the super-anti-automorphism $\vphi$ on $R \coloneqq \End_\D(\U) = \End_\D(\U^{[g]})$, then $B^{[g]}$ is nondegenerate and $(\vphi_0, B^{[g]})$ also determines $\vphi$.
	Further, if $B$ is nondegenerate, then so is $B^{[g]}$ and the superadjunction with respect to both is the same super-anti-automorphism $\vphi$ on $\End_\D(\U) = \End_\D(\U^{[g]})$.
\end{lemma}

\begin{proof}
	Let $u, v \in \U\even \cup \U\odd$.
	To avoid confusion, we will denote $u$ and $v$ by $u^{[g]}$ and $v^{[g]}$, respectively, when regarded as elements of $\U^{[g]}$.
	For all $d\in D\even \cup \D\odd$, we have:
	\begin{align*}
		B^{[g]}(u^{[g]}, v^{[g]}d) & = \sign{g}{u} B(u, vd) = \sign{g}{u} B(u, v)d =  B^{[g]}(u^{[g]}, v^{[g]})d
		\intertext{and}
		B^{[g]}(u^{[g]}d, v^{[g]}) & = \sign{g}{ud} B(ud, v) = (-1)^{|g|( |u| + |d| )} (-1)^{|d|( |B| + |u| )} \vphi_0(d) B(u, v) \\
		                           & = (-1)^{|d| ( |B| + |u| + |g| )} \vphi_0(d) \sign{g}{u} B(u, v)                              \\
		                           & = (-1)^{|d| ( |B| + |u^{[g]}| )} \vphi_0(d) B^{[g]}(u^{[g]}, v^{[g]})d .
		\intertext{Also, for all $r\in R\even \cup R\odd$, we have:}
		B^{[g]}(ru^{[g]}, v^{[g]}) & = \sign{g}{ru} B(ru, v) = (-1)^{|g|( |r| + |u| )} (-1)^{|r||u|} B(u, \vphi(r) v)             \\
		                           & = (-1)^{|r| ( |g| + |u|)} \sign{u}{g} B(u, \vphi(r) v)                                       \\
		                           & = \sign{r}{u^{[g]}} B^{[g]}(u^{[g]}, \vphi(r) v^{[g]}).
	\end{align*}
\end{proof}

Recall the concept of a module induced by a homomorphism of algebras (Definition \ref{def:twist}).

\begin{lemma}\label{lemma:twist-on-(U,B)}
	Let $\D$ and $\D'$ be graded-division superalgebras and let $\psi_0 \from \D \to \D'$ be an isomorphism.
	If $\U'$ is a $\D'$-supermodule and $B'$ is a homogeneous $\vphi_0'$-sesquilinear form on it, then $\psi_0\inv \circ B'$ is a homogeneous $(\psi_0\inv \circ \vphi_0' \circ \psi_0)$-sesquilinear form on the $\D$-supermodule $(\U')^{\psi_0}$ of the same degree as $B'$.
	Further, if $B'$ is nondegenerate, then so is $\psi_0\inv \circ B'$, and the superadjunction with respect to both is the same super-anti-automorphism $\vphi'$ on $\End_\D((\U')^{\psi_0}) = \End_{\D'}(\U')$.
\end{lemma}

\begin{proof}
	To simplify notation, let us put $B'' \coloneqq \psi_0\inv \circ B'$ and $\vphi_0'' \coloneqq \psi_0\inv \circ \vphi_0' \circ \psi_0$.
	It is clear that $\deg B'' = \deg B$ and, hence, $|B''| = |B'|$.
	Let $u,v \in (\U')\even \cup (\U')\odd$.
	To avoid confusion, we will denote $u$ and $v$ by $u^{\psi_0}$ and $v^{\psi_0}$, respectively, when regarded as elements of $(\U')^{\psi_0}$.
	For all $d\in D\even \cup \D\odd$, we have:
	\begin{align*}
		B''(u^{\psi_0}, v^{\psi_0}d)  & = \psi_0\inv \big( B'( u, v \, \psi_0(d)) \big)
		= \psi_0\inv \big( B' ( u, v ) \psi_0(d) \big)                                                                                                   \\
		                              & =  \psi_0\inv \big( B'(u, v) \big) d
		= B''(u^{\psi_0}, v^{\psi_0}) \,d
		% \end{align*}
		\intertext{and}
		% and
		% \begin{align*}
		B'' (u^{\psi_0}d, v^{\psi_0}) & = \psi_0\inv \big( B' ( u \, \psi_0(d), v ) \big)                                                                \\
		                              & = \psi_0\inv \Big( (-1)^{(|B'| + |u|) |\psi_0(d)|} \vphi_0' \big( \psi_0(d) \big) B' ( u, v) \Big)               \\
		                              & = (-1)^{(|B'| + |u|) |d|} \psi_0\inv \big( \vphi_0' \big( \psi_0 (d) \big) \big) \psi_0\inv \big( B'(u, v) \big) \\
		                              & = (-1)^{(|B''| + |u|) |d|} \vphi_0''(d) B''(u^{\psi_0}, v^{\psi_0}).
		% \end{align*}
		\intertext{Also, for all $r\in R\even \cup R\odd$, we have:}
		% Also, for all $r\in R\even \cup R\odd$, we have:
		% \begin{align*}
		B''(ru^{\psi_0}, v^{\psi_0})  & = \psi_0\inv \big( B' ( ru, v ) \big)                                                                            \\
		                              & = \psi_0\inv \big( \sign{r}{u} B' ( u, \vphi'( r ) v ) \big)                                                     \\
		                              & = \sign{r}{u} B'' ( u, \vphi'( r ) v ).
	\end{align*}
\end{proof}

\begin{defi}\label{def:iso-(U,B)}
	Let $\U$ and $\U'$ be graded right $\D$-supermodules, and let $B$ and $B'$ be homogeneous sesquilinear forms on $\U$ and $\U'$, respectively.
	An \emph{isomorphism from $(\U, B)$ to $(\U', B')$} is an isomorphism of graded modules $\theta\from \U \to \U'$ such that $B'( \theta(u), \theta(v) ) = B(u, v)$, for all $u,v \in \U$.
\end{defi}

Note that if $(\U, B)$ and $(\U', B')$ are isomorphic, then $B$ and $B'$ have the same degree in $G^\#$ and are sesquilinear with respect to the same $\vphi_0$.

% \begin{remark}
%     odd iso and $B^{[g]}$
% \end{remark}

\begin{thm}\label{thm:iso-abstract-vphi}
	Let $R \coloneqq \End_\D(\U)$ and $R' \coloneqq \End_{\D'}(\U')$, where $\D$ and $\D'$ are graded division superalgebras, and $\U$ and $\U'$ are nonzero right graded supermodules of finite rank over $\D$ and $\D'$, respectively.
	Let $\vphi$ and $\vphi'$ be degree preserving super-anti-automorphisms on $R$ and $R'$ determined, as in Theorem \ref{thm:vphi-iff-vphi0-and-B}, by pairs $(\vphi_0, B)$ and $(\vphi_0', B')$, respectively.
	If $\psi\from (R, \vphi) \to (R', \vphi')$ is an isomorphism, then there are $g\in G^\#$, a homogeneous element $0\neq d\in \D$, an isomorphism $\psi_0\from \D \to \D'$, and an isomorphism
	\begin{equation}\label{eq:iso-B-implies-vphi}
		\psi_1 \from (\U^{[g]}, dB^{[g]}) \to ( (\U')^{\psi_0}, \psi_0\inv \circ B' )
	\end{equation}
	such that
	\begin{equation}\label{eq:iso-super-anti-auto}
		\forall r\in R, \quad \psi(r) = \psi_1 \circ r \circ \psi_1\inv.
	\end{equation}
	% $\psi(r) = \psi_1 \circ r \circ \psi_1\inv$, for all $r\in R$. 
	Conversely, for any $g$, $d$, $\psi_0$ and $\psi_1$ as above,
	% the formula $\psi(r) \coloneqq \psi_1 \circ r \circ \psi_1\inv$ 
	Equation \eqref{eq:iso-super-anti-auto}
	defines an isomorphism $\psi \from (R, \vphi) \to (R', \vphi')$.
\end{thm}

\begin{proof}
	Given an isomorphism of graded superalgebras $\psi\from R \to R'$, define
	\[
		\tilde \vphi \coloneqq \psi\inv \circ \vphi' \circ \psi.
	\]
	Then $\psi$ is an isomorphism $(R, \vphi) \to (R', \vphi')$ if, and only if, $\vphi = \tilde\vphi$.

	Since $\psi$ is an isomorphism of $G^\#$-graded algebras, we can apply Theorem \ref{thm:iso-abstract} to conclude that there are $g\in G^\#$, an isomorphism of graded superalgebras $\psi_0\from \D \to \D'$, and an isomorphism of graded modules $\psi_1\from \U^{[g]} \to (\U')^{\psi_0}$ such that $\psi(r) = \psi_1 \circ r \circ \psi_1\inv$, for all $r\in R$.

	As in the proof of Lemma \ref{lemma:twist-on-(U,B)}, consider $\vphi_0'' \coloneqq \psi_0\inv \circ \vphi_0' \circ \psi_0$ and $B'' \coloneqq \psi_0\inv \circ B'$.
	Then define $\widetilde B \from \U^{[g]} \times \U^{[g]} \to \D$ by
	\[
		\widetilde B(u, v) \coloneqq B'' \big( \psi_1(u), \psi_1(v) \big)
	\]
	for all $u, v \in \U^{[g]}$.
	We claim that $\widetilde B$ is $\vphi_0''$-sesquilinear.
	Indeed, by Lemma \ref{lemma:twist-on-(U,B)}, we have
	\begin{align*}
		\widetilde B(u, vd) & = B'' \big( \psi_1(u), \psi_1(vd) \big)
		= B''\big( \psi_1(u), \psi_1(v)d \big)                                                                                  \\
		                    & = B''\big( \psi_1(u), \psi_1(v) \big)d = \widetilde B(u, v)d
		%
		\intertext{and}
		%
		\widetilde B(ud, v) & = B'' \big( \psi_1(ud), \psi_1(v) \big) = B'' \big( \psi_1(u) d, \psi_1(v) \big)                  \\
		                    & = (-1)^{(|B''| + |\psi_1(u)|) |d|} \vphi_0'' (d) B'' \big( \psi_1(u), \psi_1(v) \big)             \\
		                    & = (-1)^{(|\widetilde B| + |u|) |d|}  \vphi_0'' (d) \widetilde B(u, v).
		% \end{align*}
		\intertext{Also, $\tilde\vphi$ is the superadjunction with respect to $\widetilde B$:}
		% Also, $\tilde\vphi$ is the superadjunction with respect to $B''$:
		% \begin{align*}
		\widetilde B(ru, v) & = B'' \big( \psi_1(ru), \psi_1(v) \big)
		= B'' \big( (\psi_1 \circ r \circ \psi_1\inv) \psi_1(u), \psi_1(v) \big)                                                \\
		                    & = B'' \big( \psi(r) \psi_1(u), \psi_1(v) \big)                                                    \\
		                    & = \sign{\psi(r)}{\psi_1(u)} B'' \big( \psi_1(u), \vphi'( \psi(r) ) \psi_1(v) \big)                \\
		% &= \sign{r}{u} \psi_0 \inv \Big( B' \big( \psi_1(u), (\vphi'\circ \psi)(r)\psi_1(v) \big)\\
		% &= \sign{r}{u} B'' \big( \psi_1(u), (\psi\circ \tilde\vphi)(r) \psi_1(v) \big)\\
		                    & = \sign{r}{u} B'' \big( \psi_1(u), \psi (\tilde\vphi(r)) \psi_1(v) \big)                          \\
		                    & = \sign{r}{u} B'' \big( \psi_1(u), (\psi_1 \circ \tilde\vphi(r) \circ \psi_1\inv) \psi_1(v) \big) \\
		                    & = \sign{r}{u} B'' \big( \psi_1(u), \psi_1 (\tilde\vphi(r) v) \big)                                \\
		                    & = \sign{r}{u} \widetilde B \big( u, \tilde\vphi(r) v\big),
	\end{align*}
	where we have used Lemma \ref{lemma:twist-on-(U,B)} in the third line and the definition of $\tilde \vphi$ on the fourth line.
	Hence, applying \cref{thm:vphi-iff-vphi0-and-B} for $\U^{[g]}$ and Lemma \ref{lemma:B^[b]-does-the-job}, we conclude that $\vphi = \tilde\vphi$ if, and only if, there is a homogeneous $0 \neq d \in \D$ such that $\widetilde B = dB^{[g]}$.
	The result follows.
\end{proof}

We can interpret Theorem \ref{thm:iso-abstract-vphi} in terms of group actions.
For that, fix a fixed graded-division superalgebra $\D$.
We will define three (left) group actions on the class of pairs $(\U, B)$, where $\U \neq 0$ is a graded $\D$-supermodule and $B$ is a nondegenerate homogeneous sesquilinear form.
Recall that, by \cref{lemma:B-determines-vphi_0}, $B$ is $\vphi_0$-sesquilinear for a unique super-anti-automorphism $\vphi_0$ of $\D$.

Let $\D^\times_{\mathrm{gr}} \coloneqq \big( \bigcup_{g \in G^\#} \D_g \big)\backslash \{ 0 \}$, the group of nonzero homogeneous elements of $\D$.
Given $d\in \D^\times_{\mathrm{gr}}$, we define
\begin{equation}\label{eq:Dx_gr-action}
	d\cdot (\U, B) \coloneqq (\U, dB).
\end{equation}
Note that $dB$ is $(\mathrm{sInt}_d \circ \vphi_0)$-sesquilinear by Theorem \ref{thm:vphi-iff-vphi0-and-B}.

Let $A \coloneqq \Aut (\D)$, the group of automorphisms of $\D$ as a graded superalgebra.
Given $\tau \in A$, we define
\begin{equation}\label{eq:Aut(D)-action}
	\tau \cdot (\U, B) \coloneqq (\U^{\tau\inv}, \tau \circ B).
\end{equation}
Note that $\tau \circ B$ is $(\tau \circ \vphi_0 \circ \tau\inv)$-sesquilinear by Lemma \ref{lemma:twist-on-(U,B)}.
% (Note that we have changed the place of the ``inverse sign'' to get a left action.)

Finally, consider the group $G^\#$.
Given $g \in G^\#$, we define
\begin{equation}\label{eq:G-action}
	g \cdot (\U, B) \coloneqq (\U^{[g]}, B^{[g]}).
\end{equation}
Note that $B^{[g]}$ is $\vphi_0$-sesquilinear by Lemma \ref{lemma:B^[b]-does-the-job}.

\begin{lemma}\label{lemma:action-on-(U,B)}
	The three actions defined above give rise to a $(\D^\times_{\mathrm{gr}} \rtimes A) \times G^\#$-action, where $A$ acts on $\D^\times_{\mathrm{gr}}$ by evaluation.
\end{lemma}

\begin{proof}
	Let $d\in \D^\times_{\mathrm{gr}}$, $\tau \in A$, $g \in G^\#$ and $u, v \in \U$.
	First note that the action of $d$ does not change $\U$, so we only have to consider its effect on $B$.
	Since
	\begin{align*}
		(\tau \circ dB)(u,v) & = \tau \big( d B(u,v) \big) = \tau (d) \tau \big( B(u,v) \big) = \big( \tau(d) (\tau \circ B) \big) (u,v),
	\end{align*}
	the $\D^\times_{\mathrm{gr}}$-action combined with the $A$-action gives us a $(\D^\times_{\mathrm{gr}} \rtimes A)$-action.
	The $G^\#$-action commutes with the $\D^\times_{\mathrm{gr}}$-action since
	\begin{align*}
		(dB)^{[g]} (u, v) = \sign{g}{u} dB(u,v) = d B^{[g]}(u,v).
	\end{align*}
	Finally, the $G^\#$-action also commutes with the $A$-action since $(\U^{\tau\inv})^{[g]} = (\U^{[g]})^{\tau\inv}$ and
	\begin{align*}
		(\tau \circ B^{[g]}) (u,v)
		 & = \tau \big( \sign{g}{u} B(u,v) \big)
		 & = \sign{g}{u} (\tau \circ B) (u,v) = (\tau \circ B)^{[g]} (u,v).
	\end{align*}
\end{proof}

% These 3 actions can be combined into a $(\D^\times_{\mathrm{gr}} \rtimes A) \times G^\#$-action, where $A$ acts on $\D^\times_{\mathrm{gr}}$ by evaluation. 
% To see that, let $d\in \D^\times_{\mathrm{gr}}$, $\psi_0 \in A$ and $g \in G^\#$. 
% The action of $d$ does not change $\U$, hence commutes in the first entry with the other actions. 
% On the other hand, $(\psi_0 \circ dB)(u,v) = \psi_0 \big( (d B(u,v) ) \big) = \psi_0 (d) \psi_0 (B(u,v)) = (\psi_0(d) \circ \psi_0(B) ) (u,v)$, for all $u,v \in \U$.

\begin{cor}\label{cor:iso-with-actions}
	Under the assumptions of Theorem \ref{thm:iso-abstract-vphi}, if $\D \not \iso \D'$, then $(R, \vphi) \not \iso (R', \vphi')$.
	Otherwise, fix an isomorphism $\psi_0\from \D \to \D'$.
	Then $(R, \vphi) \iso (R', \vphi')$ if, and only if, $\big( (\U')^{\psi_0}, \psi_0\inv \circ B' \big)$ is isomorphic to an object in the $(\D^\times_{\mathrm{gr}} \rtimes A) \times G^\#$-orbit of $(\U, B)$.
	% and lie in the same orbit of the $(\D^\times_{\mathrm{gr}} \rtimes A) \times G^\#$-action. \qed
\end{cor}

% ----- 
