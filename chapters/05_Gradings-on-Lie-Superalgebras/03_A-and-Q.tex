\section{Gradings on the Lie Superalgebras of Type \texorpdfstring{$A$ and $Q$}{A and Q}}

By the discussion in \cref{sec:Aut-Lie-chap}, we have a classification of $G$-gradings on the Lie superalgebras of series $A$ and $Q$ in terms of the model $\Skew(R, \vphi)$, where $R = S\times S\sop$ and $\vphi$ is the exchange superinvolution. 
We want to describe these gradings in terms of the original model, $S^{(-)}$, where $S$ is either $M(m,n)$ or $Q(n)$ for some positive integers $m$ and $n$. 
We use the approach similar to \cite[Appendix]{paper-adrian}. 

We will also express the parameters of the grading in terms of the group $G$ rather than $G^\# =G \times \ZZ_2$. 
 
\subsection{Undoubling}\label{ssec:undoubling}
% Reducing from \texorpdfstring{$S\times S\sop$}{SxSsop} to \texorpdfstring{$S$}{S}

Let $L$ be a finite dimensional simple Lie superalgebra in the series $A$ or $Q$, and suppose $L$ is not of type $A(1,1)$. 
By definition, we have that $L = S^{(1)}/Z(S^{(1)})$. 
On the other hand, \cref{cor:transfer-R-vphi-to-L} allows us to classify the gradings on $L$ by considering its isomorphic copy $\tilde L \coloneqq \Skew(R, \vphi)^{(1)}/Z(\Skew(R, \vphi)^{(1)})$, where $R = S\times S\sop$ and $\vphi$ is the exchange superinvolution:
the gradings (and their isomorphism classes) on $\tilde L$ are in bijection with the gradings (and their isomorphism classes) on $(R, \vphi)$. 
The goal of this subsection is to translate the classification of gradings on $L$ from $(R, \vphi)$ to $S$. 
Recall that the isomorphism $\tilde L \to L$ is induced by the projection $R \to S$. 

\begin{defi}\label{defi:types-I-and-II}
    A grading on $L$ is said to be of \emph{Type I} if it is obtained by restriction and reduction modulo the center from a (unique) grading on $S$. 
    Otherwise, the grading on $L$ is said to be of \emph{Type II}. 
\end{defi}

As recalled above, all gradings on $\tilde L \iso L$ come from gradings on $(R, \vphi)$. 
Type I and Type II gradings can be distinguished in this model as follows. 
Recall that gradings on $(R, \vphi)$ were divided in two classes: those that make $R$ graded-simple, classified in \cref{thm:MxM-even,thm:MxM-odd}, and those that do not, classified in \cref{thm:MxM-type-I,thm:QxQ-type-I}. 
We claim that these classes correspond to the Type II and Type I gradings on $\tilde L$, respectively. 
Indeed, consider a Type I grading on $L$ and the corresponding grading on $S$. 
Then $(R, \vphi)$ is naturally graded so that $S\times \{ 0 \}$ and $\{0\} \times S\sop$ are graded superideals, and the projection $R \to S$ is a homomorphism of graded superalgebras. 
This grading on $(R, \vphi)$ induces a grading on $\tilde L$ and the isomorphism of Lie superalgebras $\tilde L \to L$ preserves degrees. 
This proves that the gradings on $\tilde L$ corresponding to Type I gradings on $L$ (under the isomorphism $\tilde L \to L$) come from the gradings on $R$ that do not make it graded-simple. 
By \cref{prop:only-SxSsop-is-simple}, every grading on $(R, \vphi)$ such that $R$ is not graded-simple is of this form, so the converse follows. 

By definition, Type I gradings are described in terms of $S$, so we will now focus on Type II gradings. 

Suppose $(R, \vphi)$ is endowed with a grading $\Gamma$ making $R$ graded-simple. 
By \cref{thm:iso-(R-vphi)-with-parameters}, there is a triple $(\D, \U, B)$ with parameters $(T, \beta, p, \eta, \kappa, g_0, \delta)$ as in Definition \ref{def:E(D,U,B)} such that $(R, \vphi) \iso E(\D, \U, B)$. 
Recall that, in this case, $\rad \tilde\beta = \langle f \rangle$ for an element $f\in T$ of order $2$ such that $\eta(f) = -1$ (\cref{prop:types-of-D-via-rad-beta}). 

Set $\barr G = G/\langle f \rangle$, let  $\pi\from G\to \barr G$ be the natural homomorphism and write $\bar g \coloneqq \pi(g)$, for all $g\in G$. 
By \cref{prop:lemma-for-undoubling-and-fine-gradings}, ${}^\pi R$ is the sum of two graded-simple superideals, $S\times \{ 0 \}$ and $\{0\} \times S\sop$, so the restriction of ${}^\pi \Gamma$ to $S \iso S\times \{ 0 \}$ induces a $\barr G$-grading on $L$ of Type I. 

To recover $\Gamma$ from its coarsening ${}^\pi \Gamma$, fix a character $\chi \in \widehat {G^\#}$ such that $\chi (f) = -1$ (which exists since $\FF$ is algebraically closed) and let $\psi\from R \to R$ be the automorphism given by the action of $\chi$, \ie, $\psi(r) \coloneqq \chi(g) r$ for every $r\in R_g$. 
Clearly, $\psi$ restricts to $R_{\bar g} = R_g \oplus R_{gf}$ and acts as the multiplication by $\chi(g)$ on $R_g$ and by $\chi(gf) = - \chi(g)$ on $R_{gf}$. 
Hence:
\[\label{eq:Rg-recovered-with-chi}
    R_g = \{ r \in R_{\bar g} \mid \psi(r) = \chi(g)r \}.
\]
Let $\zeta \coloneqq (1, -1) \in Z(R)\even$. 
We have that $\vphi(\zeta) = -\zeta$ and, by \cref{prop:R-and-D-have-the-same-center-vphi}, $\zeta$ is homogeneous of degree $f$ with respect to $\Gamma$. 
Let $\theta\from R \to R$ be the super-anti-automorphism defined by $\theta \coloneqq \vphi \psi = \psi \vphi$. 
By the definition of $\psi$, $\psi(\zeta) = -\zeta$ and, hence, $\theta(\zeta) = \zeta$. 
It follows that $\theta(1,0) = \theta \left(\frac{1+\zeta}{2}\right) = (1,0)$, so  $\theta(S\times \{0\}) = \theta ((1,0)R) = S\times \{0\}$. 
Hence, $\theta$ restricts to a super-anti-automorphism on $S$. 
Then, it is straightforward to see that
\[\label{eq:psi-in-terms-of-theta}
    \psi(s_1, \overline{s_2}) = (\theta(s_2), \overline{\theta(s_1)}),
\]
for all $s_1, s_2\in S$. 
Since $\Skew(R, \vphi) = \{ (s, -s) \mid s \in S \}$, combining \cref{eq:Rg-recovered-with-chi,eq:psi-in-terms-of-theta}, we get that the Type II grading on $L$ corresponding to $\Gamma$ can be recovered from the restriction of ${}^\pi \Gamma$ to $S$ as follows: we define a $G$-grading $S^{(-)} = \bigoplus_{g\in G} S^{(-)}_g$ by setting
\[\label{eq:refiniment-on-L}
    S^{(-)}_g \coloneqq \{ s \in S_{\bar g} \mid \theta(s) = -\chi(g) s\},
\]
and induce a $G$-grading on $L = S^{(1)}/Z(S^{(1)})$. 
Hence, a Type II grading on $L$ can be described in terms of a $\barr G$-grading on $S$ and a super-anti-automorphism on $S$. 

Our next goal is to describe $(S, \theta)$, where $S$ is endowed with the restriction of ${}^\pi\Gamma$, without reference to $(R,\vphi)$. 
By \cref{prop:lemma-for-undoubling-and-fine-gradings}, $S \iso E(\barr T, \barr \beta, \barr p, \kappa)$, where $\barr T \coloneqq T/\langle f \rangle$, $\barr \beta$ and $\barr p$ are the maps induced by $\beta$ and $p$ on $\barr T$, and $\kappa\from G^\#/T \to \ZZ_{\geq 0}$ is seen as a map $\kappa\from \barr G^\#/ \barr T \to \ZZ_{\geq 0}$ via the canonical isomorphism $G^\#/T \iso \barr G^\#/ \barr T$. 

To describe $\theta\from S \to S$, let us first write $\theta\from R\to R$ in matrix terms. 
Let $\B = \{ u_1, \ldots, u_k \}$ be a $G$-graded basis of $\U$, following \cref{conv:pick-even-basis}, and use it to identify $\End_\D (\U)$ with $M_k(\D)$. 
We will also assume that $B$ is even if $\D$ is odd (\cref{conv:pick-even-form}). 
Set $g_i \coloneqq \deg u_i$. 
By \cref{prop:matrix-vphi}, we have $\vphi(X) = \Phi\inv \vphi_0(X)\stransp \Phi$, for all $X \in M_k(\D)$, where $\Phi_{ij} = B(u_i, u_j)$. 
Also, given $i, j \in \{ 1, \ldots, k \}$ and $0 \neq d \in \D_t$, $t\in T$, we have $\psi( E_{ij}(d) ) = \chi(g_i t  g_j\inv) E_{ij}(d) = \chi(g_i) \chi(t) \chi(g_j)\inv E_{ij}(d)$. 
Let $\Lambda \in M_k(\D)$ be the diagonal matrix with $\Lambda_{ii} = \chi(g_i)$ for all $i \in \{ 1, \ldots, k \}$, and let $\psi_0$ denote the action of $\chi$ on $\D$. 
Then $\psi(X) = \Lambda \psi_0(X) \Lambda\inv$, for all $X \in M_k(\D)$. 
Define $\theta_0 \coloneqq \psi_0 \vphi_0 = \vphi_0 \psi_0$. 
Then:
\[\label{eq:theta-with-matrix-1}
    \theta(X) = (\Lambda\Phi)\inv\, \theta_0(X)\stransp \, (\Lambda\Phi),
\]
for all $X \in M_k(\D)$. 

Clearly, $\theta_0$ is a super-anti-automorphism on $\D$ associated to the map $\mu \coloneqq \eta \chi$. 
Note that $\mu(f) = 1$ and, hence, we can see $\mu$ as a map $\barr \mu\from \barr T \to \FF^\times$. 
Indeed,  since $f\in \ker \tilde\beta$, we have
\begin{equation}
    \mu(tf) = \tilde\beta (t,f) \mu(t)\mu(f) = \mu(t)\mu(f) = \mu(t),
\end{equation}
for all $t\in T$, so $\barr\mu$ is well-defined for every coset $\bar t \in \barr T$. 
Recall, from \cref{cor:T+-is-elem-2-grp}, that $T^+$ is an elementary $2$-group and for every $t\in T^-$, $t^2 = f$. 
Hence, $\chi$ takes values $\pm 1$ on $T^+$ and $\pm \mathbf{i}$ on $T^-$. 
By \cref{lemma:quadratic-form-involutions}, $\bar\mu$ is a quadratic map on $\barr T$. 

Let $\epsilon \coloneqq (1,0) \in Z(R)\even$ be the identity element of $S$, and consider it also as an element of $Z(\D)\even$ using \cref{prop:R-and-D-have-the-same-center-vphi}. 
Following the proof of \cref{prop:lemma-for-undoubling-and-fine-gradings}, with $\D$ playing the role of $\mc E$ and $\U$ playing the role of $\V$, we have that $S \iso \End_{\barr \D}(\bar \U)$, where $\barr \D \coloneqq \D\epsilon$ and $\barr \U \coloneqq \U\epsilon$, and that $\B\epsilon \coloneqq \{ u_1\epsilon, \ldots, u_k\epsilon\}$ is a $\barr G^\#$-graded $\barr \D$-basis of $\barr \U$. 
Using this basis, we can identify $\End_{\barr \D}(\bar \U)$ with $M_k(\barr \D)$. 

By definition, if $Y \in M_k(\D)$ is the matrix representing an operator
$r\in R$ with respect to the basis $\B$, then $ru_j = \sum_{i = 1}^k u_i Y_{ij}$, for all $1 \leq j \leq k$. 
It follows that 
\begin{align}
    (\epsilon r)(u_j\epsilon) = r(u_j) \epsilon = \sum_i^k u_i Y_{ij} \epsilon = \sum_i^k (u_i\epsilon) Y_{ij} \epsilon, 
\end{align}
\ie, $X \in \M_k (\barr \D)$, given by $X_{ij} \coloneqq Y_{ij} \epsilon$, is the matrix representing $\epsilon r \in S$ with respect the basis $\B\epsilon$. 
From \cref{eq:theta-with-matrix-1}, we now get that
\[\label{eq:theta-with-matrix-2}
    \forall X\in M_k(\barr \D), \quad \theta(X) = \Theta\inv\, \theta_0(X)\stransp\, \Theta,
\]
where $\Theta \coloneqq \Lambda \Phi \epsilon \in M_k(\barr \D)$ or, equivalently, $\Theta_{ij} = \chi(g_i) B(u_i, u_j) \epsilon$, and $\theta_0$ is the super-anti-automorphism on $\barr \D$ associated to $\bar \mu$. 

We note that the refinement obtained in \cref{eq:refiniment-on-L}, from $\theta$ given by \cref{eq:theta-with-matrix-2}, is determined by the values of $\chi$ on the subgroup $T \subseteq G^\#$. 
To be more precise, let $\tilde \chi \in \widehat {G^\#}$ be a character such that $\chi\restriction_{T} = \tilde\chi\restriction_{T}$, let $\tilde \psi \from R\to R$ be the action by $\tilde \chi$, and set $\tilde \theta \coloneqq \tilde\psi \vphi = \vphi \tilde\psi$. 
We claim that, for every $s \in S_{\bar g}$, $\tilde\theta(s) = - \tilde\chi(g) s$ \IFF $\theta(s) = - \chi(g) s$. 

Indeed, set $\sigma \coloneqq \tilde\chi\chi\inv$, so $\tilde\chi = \sigma \chi$. 
Then $\sigma$ is a character on $G^\#$ with $\sigma(T) = 1$ and, in particular, can be seen as a character on $G^\#/\langle f \rangle$. 
Every element in $S_{\bar g}$ is a sum of elements of the form $E_{ij}(d)$, where $1 \leq i, j \leq k$ and $d\in \barr \D_{\bar t}$, such that $\bar g_i \bar t \bar g_j\inv = \bar g$. 
By \cref{eq:theta-with-matrix-2}, 
\begin{align}
    \tilde\theta(E_{ij}(d)) 
    &= \sigma(g_i) \sigma(g_j\inv) \theta (E_{ij}(d)) 
    = \sigma(\bar g_i) \sigma(\bar g_j\inv) \theta (E_{ij}(d)) \\ 
    &= \sigma(\bar t\inv \bar g) \theta(E_{ij}(d)) 
    = \sigma(\bar g) \theta(E_{ij}(d)).
\end{align} 
We conclude that $\tilde\theta(s) = \sigma(\bar g) \theta(s)$ for all $s\in S_{\bar g}$ and the claim follows. 

In the following subsections, we specialize these considerations to the cases of $A(m,n)$ and $Q(n)$. 

% ---------------------------

\subsection{Gradings on \texorpdfstring{$A(m,n)$}{A(m,n)} for \texorpdfstring{$m \neq n$}{m different than n}}\label{ssec:grds-on-A-m-n}

We now discuss the case where $S = M(m+1, n+1)$ and $m \neq n$,  
so $L = \Sl(m+1, n+1) = S^{(1)}$ is a simple Lie superalgebra of type $A(m,n)$. 
(As seen before, in this case $Z(S^{(1)}) = \{ 0 \}$.)

We first parametrize the Type I gradings. 
Since $m\neq n$, by \cref{lemma:odd-M-m=n}, every grading on $S$ is even in this case. 
In particular, $\kappa\from G^\#/T \to \ZZ_{\geq 0}$ corresponds to a pair $(\kappa_\bz, \kappa_\bo)$ (see \cref{ssec:supermodules-over-D}). 

In the next definition, we allow the possibility of $m=n$ for future reference (in \cref{ssec:grds-on-Ann}).

\begin{defi}\label{def:A-Type-I}
    Let $m, n\in \ZZ_{\geq 0}$, not both zero, $T \subseteq G$ be a finite subgroup, $\beta\from T\times T \to \FF^\times$ be a nondegenerate alternating bicharacter, and $\kappa_\bz, \kappa_\bo \from G/T \to \ZZ_{\geq 0}$ be maps with finite support such that $|\kappa_\bz| \sqrt{|T|/2} = m+1$ and $|\kappa_\bo| \sqrt{|T|/2} = n+1$. 
    We will denote by $\Gamma^{\mathrm{(I)}}_A(T, \beta, \kappa_\bz, \kappa_\bo)$ the restriction to $S^{(1)}$ of the grading $\Gamma_M(T, \beta, \kappa_\bz, \kappa_\bo)$ on $S$ (see \cref{def:Gamma-T-beta-kappa-even}). 
\end{defi}

Parametrizing Type II gradings is more involved. 
As we have seen in \cref{ssec:undoubling}, those correspond to gradings on $(R, \vphi)$ making $R$ graded-simple, where $R \coloneqq S \times S\sop$ and $\vphi$ is the exchange superinvolution. 
Again, since $m\neq n$, these gradings on $R$ are even (see \cref{cor:associative-type-II-odd-m=n}). 
Hence, by \cref{thm:MxM-even}, a Type II grading on $L$ corresponds to a grading on $(R, \vphi)$ making it isomorphic to $M^{\mathrm{ex}} (T, \beta, \kappa_\bz, \kappa_\bo, g_0)$ (\cref{def:model-grd-MxM-even}), where $T \subseteq G$ is a finite $2$-elementary subgroup, $\beta\from T\times T \to \FF^\times$ is an alternating bicharacter with $\rad \beta = \langle f \rangle$ for some $e\neq f \in T$, $g_0$ is an element in $G^\#$, and $\kappa_\bz, \kappa_\bo \from G/T \to \ZZ_{\geq 0}$ are $g_0$-admissible maps (see \cref{inertia-even-and-odd-case}) such that $|\kappa_\bz| \sqrt{|T|/2} = m+1$ and $|\kappa_\bo| \sqrt{|T|/2} = n+1$. 
Since $m\neq n$, the $g_0$-admissibility implies that $g_0 \in G$. 
Recall that the graded algebra $M^{\mathrm{ex}} (T, \beta, \kappa_\bz, \kappa_\bo, g_0)$ corresponds to $(\eta, \kappa, g_0, \delta) \in \mathbf{I}(T, \beta, p)$, where $\eta\from T \to \pmone$ is fixed, $p\from T \to \ZZ_2$ is the trivial homomorphism, and $\delta = 1$. 
The map $\eta$ was determined by fixing a standard realization $\D$ for $(T, \beta, e)$ (see \cref{def:std-realization-MxM-QxQ}), namely, $\eta$ is the map associated to the superinvolution $\vphi_{\mc C} \tensor \vphi_{\mc M}$ on $\D$. 

We will use the parameters $(T, \beta, \kappa_\bz, \kappa_\bo, g_0)$ to construct a representative for the Type II grading in terms of the ``undoubled'' model, \ie, we will construct a $\barr G$-grading on $S$, where $\barr G \coloneqq G/\langle f \rangle$, and a super-anti-automorphism $\theta\from S\to S$, as in \cref{ssec:undoubling}. 
As in that subsection, let $\pi\from G \to \barr G$ denote the natural homomorphism, set $\barr T \coloneqq T/\langle f \rangle$, let $\barr \beta$ be the (nondegenerate) bicharacter on $\barr T$ induced by $\beta$, and consider $\kappa_\bz$ and $\kappa_\bo$ as maps defined on $\barr G/\barr T \iso G/T$. 

Let $\barr \D$ be a standard realization associated to $(\barr T, \bar\beta)$ (see \cref{def:standard-realization}), and let $\bar \mu\from \barr T \to \FF^\times$ be the map associated to the transposition map. 

Choose a complement $K$ for $\langle f \rangle$ in $T$, \ie, a subgroup $K \subseteq T$ such that $T = K \times \langle f \rangle$ (it can be done, since $T$ is a elementary $2$-group). 
Let $\chi\from T \to \FF^\times$ be the character defined by $\chi(K) = 1$ and $\chi(f) = -1$, and set $\mu \coloneqq \barr \mu \circ \pi$. 
Then set 
\[\label{eq:fix-eta-undouble}
    \forall t\in T, \quad \eta(t) \coloneqq \mu(t) \chi\inv(t).
\]
Note that this definition agrees with \cref{def:std-realization-MxM-QxQ}(a) (see the proof of \cref{prop:m-not-n-Type-II-correspondence}). 

Extend $\chi$ to $G$. 
It remains to define a graded right $\barr \D$-supermodule $\barr \U$ and the matrix $\Theta$. 
The following is analogous to the construction in \cref{ssec:grds-osp}. 

Let $\xi\from G/ T \to G$ be a set-theoretic section of the natural homomorphism, and let $\leq$ be a total order on the set $G/T \iso \barr G/ \barr T$ with no elements between $x$ and $\bar g_0\inv x\inv$. 
Changing $\xi$ if necessary, we may assume that $\xi(g_0\inv x\inv) = g_0\inv \xi(x)\inv$ if $x < g_0\inv x\inv$. 
For each $i \in \ZZ_2$, set $k_i \coloneqq |\kappa_i|$, let $\gamma_i$ be the $k_i$-tuple of elements in $G$ realizing $\kappa_i$ according to $\xi$ and $\leq$ (see \cref{defi:tuple-governed}), and let $\bar \gamma_i$ be the tuple of elements in $\barr G$ consisting of the images under $\pi\from G\to \barr G$ of the entries of $\gamma_i$ (\ie, $\barr \gamma_i$ is the $k_i$-tuple realizing $\kappa$ according to $\pi \circ \xi$ and $\leq$). 
Consider on $M_{k_\bz | k_\bo}(\FF)$ the elementary grading determined by $(\bar  \gamma_\bz, \bar \gamma_\bo)$ (see \cref{defi:elementary-grd-super}). 
We identify the $\barr G$-graded superalgebra $M_{k_\bz | k_\bo}(\barr \D) = M_{k_\bz | k_\bo}(\FF) \tensor \barr\D$ with $S = M(m+1, n+1)$ via Kronecker product. 

\begin{defi}\label{defi:blocks-of-Theta}
    Let $i\in \ZZ_2$ and $x \in G/T$. 
    If $g_0x^2 = T$, we put $t \coloneqq g_0 \xi(x)^2 \in T$ and let $\bar t \in \barr T$ be its image under the natural homomorphism $T \to \barr T$. 
    We define $\Theta(i, x)$ to be the following $\kappa_i(x) \times \kappa_i(x)$-matrix with entries in $\barr \D$:
    %
    \begin{enumerate}[(i)]
        \item $I_{\kappa_i(x)} \tensor X_{\bar t}$ if $(-1)^i \eta(t) = +1$;
        %
		\item  $J_{\kappa_i(x)} \tensor X_{\bar t}$, where $J_{\kappa_i(x)} \coloneqq \begin{pmatrix}
				      0                & I_{\kappa_i(x)/2} \\
				      -I_{\kappa_i(x)/2} & 0
			      \end{pmatrix}$, if  $(-1)^i \eta(t) = -1$ (recall that, in this case, $\kappa_i(x)$ is even by \cref{inertia-even-and-odd-case}). 
	\end{enumerate}
    %
    If $g_0 x^2 \neq T$, we define $\Theta(i, x)$ to be the following $2\kappa_i(x) \times 2\kappa_i(x)$-matrix:
    %
    \begin{enumerate}[(i)]
        %
        \setcounter{enumi}{2}
        %
		\item $\begin{pmatrix}
			0  &  I_{\kappa_i(x)} \\
			(-1)^{i} \chi(g_0 \xi(x)^2)\inv I_{\kappa_i(x)} & 0
		\end{pmatrix} \tensor 1_{\barr \D}$. 
    \end{enumerate}
\end{defi}

Let $x_1 < \ldots < x_{\ell_\bz}$ be the elements of the set $\{ x \in \supp \kappa_\bz \mid x \leq g_0\inv x\inv \}$ and, similarly, let $y_1 < \ldots < y_{\ell_\bo}$ be the elements of $\{ y \in \supp \kappa_\bo \mid y \leq g_0\inv y\inv \}$. 
Then, we define 
\[\label{eq:puting-the-blocks-of-Phi-together-version-A}
    %
    \sbox0{$\begin{matrix}
        \Theta(\bar 0, x_1)&& \\
        & \ddots &\\
        && \Theta(\bar 0, x_{\ell_\bz})
    \end{matrix}$}
    %
    \sbox1{$\begin{matrix}
        \Theta(\bar 1, y_1)&& \\
        & \ddots &\\
        && \Theta(\bar 1, y_{\ell_\bo})
    \end{matrix}$}
    %
    \Theta \coloneqq
    \left(\begin{array}{c|c}
            \usebox{0} & 0\\
            \hline
            0 & \usebox{1}
        \end{array}\right).
\]
%
% @2
We define the super-anti-automorphism $\theta\from S\to S$ by \cref{eq:theta-with-matrix-2}, where $\theta_0\from \barr \D \to \barr \D$ denotes the transposition on $\barr \D$. 
Note that $\theta_0(X)\stransp \in M_{k_\bz \mid k_\bo} (\barr \D)$ becomes $X\stransp \in M(m+1,n+1)$. 
Hence, \cref{eq:theta-with-matrix-2} reduces to 
\[\label{eq:theta-with-matrix-3}
    \forall X\in M(m+1,n+1), \quad \theta(X) = \Theta\inv\, X\stransp\, \Theta.
\]

Finally, we define a $G$-grading on $L = S^{(1)}$ by 
\[
    \forall g\in G, \quad L_g \coloneqq \{ s \in S^{(1)}_{\barr g} \mid \theta(s) = - \chi(g) s\}.
\]

In the next definition, we summarize what has been done for future reference. 
We also allow the case $m=n$.

\begin{defi}\label{defi:type-II-A-m-not-n}
    Let $m,n \in \ZZ_{\geq 0}$, not both zero. 
    Let $T \subseteq G$ be a finite $2$-elementary subgroup, let $\beta\from {T\times T} \to \FF^\times$ be an alternating bicharacter with $\rad \beta = \langle f \rangle$, for some $e \neq f\in T$, and let $g_0 \in G$. 
    Set $\barr G \coloneqq G/\langle f \rangle$, $\barr T \coloneqq T/\langle f \rangle$, and let $\bar \beta$ be the nondegenerate alternating bicharacter on $\barr T$ induced by $\beta$. 
    Choose:
    \begin{enumerate}[(i)]
        \item a standard realization $\barr \D$ associated to $(\barr T, \barr \beta)$; 
        \item a subgroup $K \subseteq T$ such that $T = K \times \langle f \rangle$; 
        \item a set-theoretic section $\xi\from G/T \to G$ for the natural homomorphism $G \to G/T$;
        \item a total order $\leq$ on $G/T$ such that there are no elements between $x$ and $\bar g_0\inv x\inv$, for all $x\in G/T$. 
    \end{enumerate}
    Let $\barr \mu\from \barr T \to \FF^\times$ be the map determining the transposition on $\barr \D$ (see \cref{ssec:param-D-vphi}), and 
    let $\chi \in \widehat{T}$ be the character such that $\chi(K) = 1$ and $\chi(f) = -1$, and extend it to a character on $\widehat{G}$, also denoted by $\chi$. 
    Then define $\mu \coloneqq \bar\mu \circ \pi$, where $\pi\from G \to \barr G$ is the natural homomorphism, and define $\eta\from T \to \pmone$ by \cref{eq:fix-eta-undouble}. 
    Let $\kappa_\bz, \kappa_\bo \from G/T \to \ZZ_{\geq 0}$ be $g_0$-admissible maps (\cref{inertia-even-and-odd-case}) such that $m+1 = k_\bz \sqrt{|T|/2}$ and $n+1 = k_\bo \sqrt{|T|/2}$, where $k_i \coloneqq |\kappa_i|$, $i\in \ZZ_2$. 
    Then construct tuples $\bar\gamma_\bz$ and $\bar\gamma_\bo$ realizing $\kappa_\bz$ and $\kappa_\bo$, respectively, according to $\pi \circ \xi$ and $\leq$ (\cref{defi:tuple-governed}). 
    Consider the $\barr G$-grading $\Gamma_M(\barr T, \barr \beta, \kappa_\bz, \kappa_\bo)$ on $S \coloneqq M(m+1,n+1)$ constructed using the choices of $\barr \D$, $\barr \gamma_\bz$ and $\barr \gamma_\bo$ above (see \cref{def:Gamma-T-beta-kappa-even}), and consider its restriction to $S^{(1)}$. % = \bigoplus_{\bar g \in \barr G} S^{(1)}_{\bar g}$. 
    Define ${\Theta \in S}$ by \cref{eq:puting-the-blocks-of-Phi-together-version-A} and ${\theta\from S \to S}$ by
    \cref{eq:theta-with-matrix-3}. 
    Finally, we define $\Gamma_A^{\mathrm{(II)}}(T, \beta, \kappa_\bz, \kappa_\bo, g_0)$ to be the $G$-grading on $S^{(1)}$ given by
    \[
        S^{(1)}_{g} \coloneqq \{ s\in S^{(1)}_{\bar g} \mid \theta (s) = - \chi(g) s \},
    \]
    for all $g\in G$. 
\end{defi}

\begin{prop}\label{prop:m-not-n-Type-II-correspondence}
    Consider $(R, \vphi) \coloneqq M^{\mathrm{ex}}(T, \beta, \kappa_\bz, \kappa_\bo, g_0)$ (\cref{def:model-grd-MxM-even}). 
    Then $\Skew (R,\vphi)^{(1)}$ is isomorphic to $M(m+1, n+1)^{(1)}$ endowed with $\Gamma_A^{\mathrm{(II)}}(T, \beta, \kappa_\bz, \kappa_\bo, g_0)$. 
\end{prop}

\begin{proof}
    We will show how the choices in \cref{defi:type-II-A-m-not-n} correspond to the choices in Definitions \ref{def:std-realization-MxM-QxQ}(a) and \ref{def:model-grd-MxM-even}. 
    
    The choices in items (i) and (ii) give us a way to make the choices for a standard realization $(\D, \vphi_0)$ as in \ref{def:std-realization-MxM-QxQ}(a).
    Recall that a standard realization $\barr \D$ associated to $(\barr T, \bar \beta)$ (\cref{defi:type-II-A-m-not-n}(i)) is obtained by choosing subgroups $\barr A$ and $\barr B$ of $\barr T$ such that $\barr T = \barr A \times \barr B$ and $\barr \beta (\barr A, \barr A) = \barr \beta (\barr B, \barr B) = 1$. 
    Note that $\pi\from G \to \barr G$ restricts to an isomorphism $K \to \barr T$, and $\beta(s,t) = \barr\beta(\bar s, \bar t)$, for all $s,t \in K$. 
    Hence, the choice of the subgroups $\barr  A$ and $\barr  B$ as above is equivalent to a choice of subgroups $A, B \subseteq K$ such that $K = A\times B$ and $\beta (A, A) = \beta (B, B) = 1$. 
    In other words, our choices of $\barr \D$ and $K$ give us the same information as the choice of $\mc M$ in \cref{def:std-realization-MxM-QxQ}(a)\eqref{item:choose-mc-M}. 
    Also, $\mu (ab) = \barr \mu (ab) = \barr\beta (\bar a, \bar b) = \beta(a,b)$ for all $a\in A$ and $b\in B$, so $\mu\restriction_{K}$ is the map determining $\vphi_{\mc M}$ (see \cref{lemma:transp-std-realization}). 
    Further, since $\eta\restriction_{K} = \mu\restriction_{K}$ and $\eta\restriction_{\langle f \rangle} = \chi\inv \restriction_{\langle f \rangle}$, the map $\eta$ defined by \cref{eq:fix-eta-undouble} is the map determining to $\vphi_{\mc C} \tensor \vphi_{\mc M}$. 
    
    We can use the choices in items (iii) and (iv) to choose the pair $(\U, B)$ as in \cref{def:model-grd-MxM-even}, by following the same construction as in \cref{ssec:grds-osp}. 
    With respect with the graded basis $\B = \{ u_1, \ldots, u_k \}$ used there, $B$ is represented by the matrix $\Phi \in M_{k_\bz | k_\bo}(\D)$ as in \cref{eq:puting-the-blocks-of-Phi-together}. 
    To get the matrix $\Theta$ as in \cref{eq:puting-the-blocks-of-Phi-together-version-A}, we will consider a different graded basis $\tilde \B = \{ \tilde u_1, \ldots, \tilde u_k \}$ for $\U$, where
    \[\label{eq:tilde-u_i-from-u_i}
        \tilde u_i \coloneqq 
        \begin{cases}
            \sqrt{\chi(\deg u_i)\inv} u_i, & \text{if }(\deg u_i) T = g_0\inv (\deg u_i)\inv T;\\
            \hfill \chi(\deg u_i)\inv u_i, & \text{if }(\deg u_i) T < g_0\inv (\deg u_i)\inv T;\\
            \hfill u_i, & \text{if }(\deg u_i) T > g_0\inv (\deg u_i)\inv T.
        \end{cases}
    \]
    Let $\tilde \Phi$ be the matrix representing $B$ with respect to the graded basis $\tilde \B$. 
    Following the procedure in \cref{ssec:undoubling} with $\tilde \Phi$ playing the role of $\Phi$, we get  $\Theta$ as in \cref{eq:puting-the-blocks-of-Phi-together-version-A}. 
\end{proof}

For the next result, we will fix choices (i) and (ii), for each pair $(T, \beta)$ as in \cref{defi:type-II-A-m-not-n}. 
Recall that, for every $\kappa\Star\from G/T \to \ZZ_{\geq 0}$, we defined $\kappa\Star\from G/T \to \ZZ_{\geq 0}$ by $\kappa\Star(x) = \kappa(x)\inv$, for all $x \in G/T$ (see \cref{ssec:superdual}).

\begin{thm}\label{thm:final-m-not-n}
    Let $m,n \in \ZZ_{\geq 0}$, $m \neq n$. 
    Let $L = \Sl(m+1, n+1)$. 
    Every grading on $L$ is isomorphic to either $\Gamma_A^{\mathrm{(I)}}(T, \beta, \kappa_\bz, \kappa_\bo)$ or $\Gamma_A^{\mathrm{(II)}}(T,\beta, \kappa_\bz, \kappa_\bo, g_0)$ as in \cref{def:A-Type-I,defi:type-II-A-m-not-n}. 
    Gradings belonging to different types are not isomorphic. 
    Within each type, we have:
    
    \noindent\boxed{\mathrm{Type \,\,I}}
    
    \noindent $\Gamma_A^{\mathrm{(I)}}(T, \beta, \kappa_\bz, \kappa_\bo) \iso \Gamma_A^{\mathrm{(I)}}(T', \beta', \kappa_\bz', \kappa_\bo')$ \IFF  $T' = T$ and one of the following conditions holds:
	\begin{enumerate}[(i)]
	    \item $\beta'=\beta$ and there is $g\in G$ such that $g \cdot \kappa_{\bar 0}'=\kappa_{\bar 0}$ and $g \cdot \kappa_{\bar 1}'=\kappa_{\bar 1}$; 
	    \item $\beta'=\beta\inv$ and there is $g\in G$ such that $g \cdot \kappa_{\bar 0}'=\kappa_{\bar 0}\Star$ and $g \cdot \kappa_{\bar 1}'=\kappa_{\bar 1}\Star$.
	\end{enumerate}

    \noindent\boxed{\mathrm{Type \,\,II}}
    
    \noindent $\Gamma_A^{\mathrm{(II)}}(T,\beta, \kappa_\bz, \kappa_\bo, g_0) \iso \Gamma_A^{\mathrm{(II)}}(T,\beta, \kappa_\bz, \kappa_\bo, g_0)$ \IFF
    $T' =T$, $\beta' = \beta$ and there is $g \in G$ such that $g\cdot\kappa_\bz' = \kappa_\bz$, $g\cdot\kappa_\bo' = \kappa_\bo$ and $g_0' = g^{-2}g_0$.
\end{thm}

\begin{proof}
    The result follows from \cref{prop:m-not-n-Type-II-correspondence,thm:MxM-type-I,thm:MxM-even}. 
    Note that, since $m \neq n$, the isomorphism conditions in \cref{thm:MxM-type-I,thm:MxM-even} simplify: we cannot have $g \cdot \kappa_{\bar 1}'=\kappa_{\bar 1}$ or $g \cdot \kappa_{\bar 1}'=\kappa_{\bar 1}\Star$. 
\end{proof}

% ----------------------------------

\subsection{Gradings on \texorpdfstring{$Q(n)$}{Q(n)}}
% @3
We will now discuss the case where $S$ is the associative superalgebra $Q(n+1)$, so $L = S^{(1)}/Z(S^{(1)})$ is the Lie superalgebra $Q(n)$. 
In this subsection, we fix $\bi \in \FF$ such that $\bi^2 = -1$. 

Let us first parametrize Type I gradings. 
As seen in (\cref{ssec:classification-assc-super}), every grading on $S$ is odd. 
Nevertheless, we can use parameters in terms of the group $G$ to parametrize the gradings (see \cref{def:Gamma-T-beta-kappa-Q,cor:iso-Q}).  

\begin{defi}\label{def:Q-Type-I}
    Let $n \in \ZZ_{> 0}$, $T^+ \subseteq G$ be a finite subgroup, $\beta\from T^+ \times T^+ \to \FF^\times$ be a nondegenerate bicharacter, $h\in G$ be an element such that $h^2 = 1$, and $\kappa\from G/T^+ \to \ZZ_{\geq 0}$ be a map with finite support such that $|\kappa| \sqrt{|T^+|} = n + 1$. 
    We will denote by $\Gamma^{\mathrm{(I)}}_Q(T^+, \beta^+, h, \kappa)$ the grading on $L$ induced from the grading $\Gamma_Q (T^+, \beta^+, h, \kappa)$ (see \cref{def:Gamma-T-beta-kappa-Q}) by reduction modulo the center. 
\end{defi}

For Type II gradings on $L$,
set $R \coloneqq S\times S\sop$ and let $\vphi$ be the exchange superinvolution on it. 
We will follow the parametrization for gradings on $(R, \vphi)$ making $R$ graded-simple introduced in the end of \cref{sec:MxM-and-QxQ-associative}. 
By  \cref{cor:QxQ-reduced-to-MxM}, $(R, \vphi)$ endowed with a such grading is isomorphic to $Q^{\mathrm{ex}} (T, \beta, \kappa, g_0)$, where $T^+ \subseteq G$ is a finite $2$-elementary subgroup, $\beta^+\from T^+\times T^+ \to \FF^\times$ is an alternating bicharacter with $\rad \beta^+ = \langle f \rangle$ for some $e\neq f \in T^+$, $g_0$ is an element in $G$, and $\kappa\from G/T^+ \to \ZZ_{\geq 0}$ is a $g_0$-admissible map (see \cref{defi:odd-D-kappa-g_0-admissible}) such that $|\kappa| \sqrt{|T^+|/2} = n+1$. 

We will the parameters $(T^+, \beta^+, h, \kappa)$ to construct a representative for the Type II $G$-grading directly on the superalgebra $L$ instead of $\Skew(R, \vphi)$. 
Recall that the parametrization above for gradings on $\Skew(R, \vphi)$ was obtained by writing $R = R\even \oplus u R\even$, where $0 \neq u \in Z(R)\odd$, and using that $(R\even, \vphi\restriction_{R\even})$ is of type $M\times M\sop$.  
We will follow an analogous strategy here. 
Recall that $S = Q(n+1) = S\even \oplus u S\even$, where
$
    u = \begin{pmatrix}
            0 & I_{n+1}\\
            I_{n+1} & 0\\
        \end{pmatrix} \in Z(S)\odd
$,
that $Z(S) = \FF1 \oplus \FF u$ can be identified with the associative superalgebra $Q(1)$, and that $S\even$ can be identified with $M_{n+1}(\FF) = M(n+1, 0)$. 
With this identifications, we have $S \iso Q(1)\tensor S\even$ via Kronecker product. 

We will, first, construct a $\barr G$-grading on $S\even$, where $\barr G \coloneqq G/\langle f \rangle$, following the same steps as we did in \cref{ssec:grds-on-A-m-n}, but with $T^+$ playing the role of $T$ and $\beta^+$ playing the role of $\beta$. 
Let $\pi\from G \to \barr G$ denote the natural homomorphism, set $\barr {T^+} \coloneqq T^+/\langle f \rangle$, let $\barr {\beta^+}\from \barr {T^+} \times \barr {T^+} \to \FF^\times$ be the (nondegenerate) bicharacter on $\barr {T^+}$ induced by $\beta^+$, and consider $\kappa$ as a map defined on $\barr G/\barr {T^+} \iso G/{T^+}$.

Let $\barr \D\even$ be a standard realization associated to $(\barr{T^+}, \barr{\beta^+})$ (see \cref{def:standard-realization}), let $\barr {\mu^+}\from \barr{T^+} \to \FF^\times$ be the map associated to the transposition map, and set $\mu^+ \coloneqq \barr {\mu^+} \circ \pi$
% (Recall that this also gives us a choice of elements $0 \neq X_{\bar t} \in \barr\D\even_{\bar t}$, for all $\bar t \in \barr{T^+}$.)
% Since $\barr {\beta^+}$ is nondegenerate, we can choose a standard realization $\barr \D\even$ of a graded-division algebra associated to $(\barr{T^+}, \barr{\beta^+})$ (see \cref{def:standard-realization}). 
% Recall that this also gives us a choice of elements $0 \neq X_{\bar t} \in \barr\D\even_{\bar t}$, for all $\bar t \in \barr {T^+}$. 
% Let $\theta_0\from \barr \D\even \to \barr \D\even$ be the transposition map, let $\bar \mu\from \barr{T^+} \to \FF^\times$ be the map associated to $\theta_0$ and set $\mu \coloneqq \barr \mu \circ \pi$. 
Choose a subgroup $K \subseteq T^+$ such that $T^+ = K \times \langle f \rangle$, and let $\chi\from T^+ \to \FF^\times$ be the character defined by $\chi(K) = 1$ and $\chi(f) = -1$. 
Then set $\eta^+\from T^+ \to \pmone$ by
\[\label{eq:fix-eta-undouble-Q-v2}
    \forall t\in T^+, \quad \eta^+(t) \coloneqq \mu^+(t) \chi\inv(t).
\] 
In the proof of \cref{prop:Q-Type-II-correspondence}, we will show that $\eta^+$ is the map associated to the superinvolution on the even part of a graded-division superalgebra as in \cref{def:std-realization-MxM-QxQ}(c). 

Extend $\chi$ to $G$ and set $k \coloneqq |\kappa|$. 
Following the construction after \cref{eq:fix-eta-undouble} in \cref{ssec:grds-on-A-m-n} with $\kappa_\bz \coloneqq \kappa$ and $\kappa_\bo$ being the zero map, we get an elementary $\barr G$-grading on $M_k(\FF) = M(k , 0)$. 
We, then, identify the $\barr G$-graded superalgebra $M_{k}(\barr \D\even) = M_{k}(\FF) \tensor \barr\D\even$ with $S\even = M(n+1)$ via Kronecker product, and define 
\[\label{eq:puting-the-blocks-of-Phi-together-version-Q-v2}
    \Theta_\bz \coloneqq \begin{pmatrix}
        \Theta(\bar 0, x_1)&& \\
        & \ddots &\\
        && \Theta(\bar 0, x_{\ell})
    \end{pmatrix},
\]
where $x_1 < \ldots < x_{\ell}$ are the elements of the set $\{ x \in \supp \kappa \mid x \leq g_0\inv x\inv \}$, and $\Theta(\bar 0, x)$, for $x\in \barr G/\barr {T^+}$, is defined in \cref{defi:blocks-of-Theta}. 

% Set $k \coloneqq |\kappa|$, let $\gamma$ be the $k$-tuple of elements in $G$ realizing $\kappa$ according to $\xi$ and $\leq$ (see \cref{defi:tuple-governed}), and let $\bar \gamma$ be the tuple of elements in $\barr G$ consisting of the images under $\pi\from G\to \barr G$ of the entries of $\gamma$ (\ie, $\barr \gamma$ is the $k$-tuple realizing $\kappa$ according to $\pi \circ \xi$ and $\leq$). 
% Consider on $M_{k}(\FF)$ the elementary grading determined by $\bar \gamma$ (see \cref{defi:elementary-grd}). 

Finally, we define the super-anti-automorphism $\theta\from S\even \to S\even$ by 
\[\label{eq:theta-with-matrix-4}
    \forall X\in M_{n+1}(\FF), \quad \theta(X) \coloneqq \Theta_\bz\inv\, X\transp\, \Theta_\bz.
\]

We will now extend this $\barr G$-grading and the super-anti-automorphism $\theta$ to $S = S\even \oplus u S\even$. 

To extend the grading, we declare $\deg u = \barr{t_p}$, where $\barr{t_p} \coloneqq (\pi(h), \bar 1)$. 
More explicitly, we define $S_{\bar g} = S\even_{\bar g}$ and $S_{\barr{t_p}\bar g} = u S\even_{\bar g}$, for all $\barr g \in \barr G$. 
A straightforward computation (using that $\barr G$ is abelian, $u$ is central and $u^2 = 1$) shows that this, indeed, defines a grading on $S$. 

To extend $\theta$ to $S$, we declare $\theta (u) = \bi u$. 
Using the identification $S = Q(1)\tensor S\even$, this is the same as 
\[\label{eq:theta-with-matrix-5}
    \forall X\in Q(n+1), \quad \theta(X) = \Theta\inv\, X\sTq\, \Theta,
\]
where 
\[
    \Theta \coloneqq \left(\begin{array}{c|c}
            \Theta_\bz & 0\\
            \hline
            0 & \Theta_\bz
        \end{array}\right).
\]

Finally, we define a $G$-grading on $S^{(1)}$ by 
\[
    \forall g\in G, \quad S^{(1)}_g \coloneqq \{ s \in S^{(1)}_{\barr g} \mid \theta(s) = - \chi(g) s\},
\]
and consider the $G$-grading on $L$ by inducing it modulo the center. 

% @1

In the next definition, we summarize what have been done for future reference:

\begin{defi}\label{defi:type-II-Q}
    Let $n \in \ZZ_{> 0}$. 
    Let $T^+ \subseteq G$ be a finite $2$-elementary subgroup, let $\beta^+\from {T^+\times T^+} \to \FF^\times$ be an alternating bicharacter with $\rad \beta^+ = \langle f \rangle$, for some element $e \neq f\in T$, let $h \in G$ be an element such that $h^2=f$, and let $g_0 \in G$ be any element. 
    Set $\barr G \coloneqq G/\langle f \rangle$, $\barr {T^+} \coloneqq T/\langle f \rangle$, and let $\bar \beta^+$ be the nondegenerate alternating bicharacter on $\barr {T^+}$ induced by $\beta^+$. 
    Choose:
    \begin{enumerate}[(i)]
        \item a standard realization $\barr \D\even$ associated to $(\barr {T^+}, \barr {\beta^+})$; 
        \label{item:choice-barr-D-Q}
        %
        \item a subgroup $K \subseteq T^+$ such that $T^+ = K \times \langle f \rangle$; 
        \label{item:choice-K-Q}
        %
        \item a set-theoretic section $\xi\from G/T^+ \to G$ for the natural homomorphism $G \to G/T^+$;
        \label{item:choice-xi-Q}
        %
        \item a total order $\leq$ on $G/T^+$ such that there are no elements between $x$ and $\bar g_0\inv x\inv$, for all $x\in G/T^+$. 
        \label{item:choice-leq-Q}
    \end{enumerate}
    Set $t_1 \coloneqq (h, \bar 1)$, $T^- \coloneqq t_1 T^+ \subseteq G^\#$ and $T\coloneqq T^+ \cup T^-$. 
    Recall that, by construction of $\barr \D$, we have $T / \langle f \rangle = \supp \barr \D$.
    Let $\barr \mu\from \barr T \to \FF^\times$ be the map associated to the queer supertransposition on $\barr \D$ (see \cref{ssec:param-D-vphi}). 
    Let $\chi \in \widehat{T}$ be the character such that $\chi(K) = 1$ and $\chi(t_1) = \bi$, and extend it to a character on $\widehat{G}$, also denoted by $\chi$. 
    Then define $\mu \coloneqq \bar\mu \circ \pi$, where $\pi\from G \to \barr G$ is the natural homomorphism, and define $\eta\from T \to \pmone$ by \cref{eq:fix-eta-undouble}. 
    Let $\kappa \to \ZZ_{\geq 0}$ be a $g_0$-admissible map (\cref{defi:odd-D-kappa-g_0-admissible}) such that $n+1 = k\sqrt{|T^+|/2}$, where $k \coloneqq |\kappa|$. 
    Then construct a tuple $\bar\gamma$ realizing $\kappa$ according to $\pi \circ \xi$ and $\leq$ (\cref{defi:tuple-governed}). 
    Consider the $\barr G$-grading $\Gamma_Q(\barr {T^+}, \barr {\beta^+}, \bar h, \kappa)$ on $S \coloneqq M(m+1,n+1)$ constructed using the choices of $\barr \D$ and $\barr \gamma$ above (see \cref{def:Gamma-T-beta-kappa-Q}), and consider its restriction to $S^{(1)}$. % = \bigoplus_{\bar g \in \barr G} S^{(1)}_{\bar g}$. 
    Define ${\Theta \in S}$ by \cref{eq:puting-the-blocks-of-Phi-together-version-A} and ${\theta\from S \to S}$ by
    \cref{eq:theta-with-matrix-4}. 
    Finally, we define $\Gamma_Q^{\mathrm{(II)}}(T^+, \beta^+, h, \kappa, g_0)$ to be the $G$-grading on $L = S^{(1)}/Z(S^{(1)})$ induced from the grading $S^{(1)} = \bigoplus_{g\in G} S^{(1)}_g$, where
    \[
        S^{(1)}_{g} \coloneqq \{ s\in S^{(1)}_{\bar g} \mid \theta (s) = - \chi(g) s \},
    \]
    for all $g\in G$. 
\end{defi}

% --------------------

\begin{prop}\label{prop:Q-Type-II-correspondence}
    Consider $(R, \vphi) \coloneqq Q^{\mathrm{ex}}(T^+, \beta^+, h, \kappa, g_0)$, as defined before \cref{cor:QxQ-reduced-to-MxM}. 
    Then the grade Lie superalgebra $\Skew (R,\vphi)^{(1)}$ is isomorphic to $Q(n+1)^{(1)}$ endowed with $\Gamma_Q^{\mathrm{(II)}}(T^+, \beta^+, h, \kappa, g_0)$. 
\end{prop}

\begin{proof}
    We will show how the choices in \cref{defi:type-II-Q} correspond to the choices in Definitions \ref{def:std-realization-MxM-QxQ}(c) and \ref{def:model-grd-MxM-odd-or-QxQ}. 
    
    Let $\barr \D$ and $K \subseteq T^+$ be, respectively, the graded-division superalgebra and the subgroup chosen in items \eqref{item:choice-barr-D-Q} and \eqref{item:choice-K-Q} of \cref{defi:type-II-Q}. 
    By \cref{def:standard-realization-Q}, $\barr \D = \barr \D\even \oplus u \barr \D\even$, where $\barr \D\even$ is a standard realization of a graded-division algebra associated to $(T^+, \beta^+)$, and $u \in Z(\barr \D)$ is a element with degree $\barr t_p \coloneqq (\barr h, \bar 1) \in \barr G^\#$. 
    Following the argument in the proof of \cref{prop:m-not-n-Type-II-correspondence}, with $(T^+, \beta^+)$ playing the role of $(T, \beta)$, we have that $\barr \D\even$ corresponds to a standard realization $\mc M$ of a graded-division algebra associated to $(K, \beta^+\restriction_{K \times K})$. 
    Moreover, if $\barr \mu\from \barr T \to \FF^\times$ is the map associated to the queer supertranspose on $\barr \D$, then $\barr \mu \restriction_{\barr {T^+}}$ is the map associated to the transposition on $\barr \D\even$ and, again following the proof of \cref{prop:m-not-n-Type-II-correspondence}, we have that $(\mu = \barr \mu \circ \pi) \restriction_K$ is the map associated to the transposition on $\mc M$. 
    
    We claim that the map $\eta\from T \to \FF^\times$ defined in \cref{eq:fix-eta-undouble-Q} is the map associated to $\vphi_{\mc M} \tensor \vphi_{\mc C}$ in \cref{def:std-realization-MxM-QxQ}(c). 
    First, note that by the definition of queer supertranspose, $\barr \mu (e) = 1$ and $\barr \mu(\bar t_p) = \bi$. 
    It follows that $\mu(e)=\mu(f) = 1$ and $\mu(t_p)=\mu(t_p f) = \bi$. 
    Since $\chi$ is a homomorphism and and $\chi(t_p) = \bi$, it is straightforward that $\eta(e) = \eta(t_p) = 1$ and $\eta(f) = \eta(t_p f) = -1$, \ie, $\eta\restriction_{\langle t_p \rangle}$ is the map associated to $\vphi_{\mc C}$. 
    By definition, $\eta\restriction_K = \mu\restriction_K$ is the map associated to $\vphi_{\mc M}$, so the claim follows.  
    
    % The choices of items \eqref{item:choice-barr-D-Q} and \eqref{item:choice-K-Q} in \cref{defi:type-II-Q} give us a way to make the choices for a standard realization $(\D, \vphi_0)$ as in \ref{def:std-realization-MxM-QxQ}(c). 
    % Indeed, the only choice for standard realization of type Q $\barr \D$ associated to $(\barr {T^+}, \bar \beta^+, h)$ (\cref{def:standard-realization-Q}) is the choice of standard realization $\barr \D\even$ associated to $(\barr {T^+}, \bar \beta^+)$. 
    % Let $\barr \mu\from \barr T \to \FF^\times$ be the map associated to the queer supertransposition on $\barr \D$. 
    % It is clear that $\mu\restriction_{\barr {T^+}}$ is the map associated to the transposition on $\barr \D\even$, and that $\mu(t_1) = \bi$. 
    % By the same argument as in the proof of \cref{prop:m-not-n-Type-II-correspondence}, with $T^+$ playing the role of $T$, this corresponds to a standard realization $\mc M$ associated to $(K, \beta^+\restriction_{K \times K})$, and $\mu\restriction_{\barr {T^+}} \circ \pi$ is the map associated to the transposition on $\mc M$. 
    % It
    
    % SHOULD I BE MORE EXPLICIT HERE, OR IS IT OK TO REFER TO THAT PROOF?
    
    % Further, since $\eta\restriction_{K} = \mu\restriction_{K}$ and $\eta\restriction_{\langle f \rangle} = \chi\inv \restriction_{\langle f \rangle}$, the map $\eta$ defined by \cref{eq:fix-eta-undouble} is the map determining to $\vphi_{\mc C} \tensor \vphi_{\mc M}$. 
    
    THERE WASN'T A EXPLICIT CONSTRUCTION OF $(\U, B)$ FOR ODD $\D$, SO I DO IT HERE.
    
    In \cref{def:model-grd-MxM-odd-or-QxQ}, we have to choose a pair $(\U, B)$ that has inertia determined by $\kappa$. 
    To construct such pair, we will follow a construction similar to the one in \cref{ssec:grds-on-A-m-n}. 
    Let $(\D, \vphi_0)$ denote $(\mc M \tensor \mc C, \vphi_{\mc C} \tensor \vphi_{\mc M})$ as in \cref{def:std-realization-MxM-QxQ}(c), and let $\gamma = (g_1, \ldots, g_k)$ be the $k$-tuple realizing $\kappa$ according to $\xi$ and $\leq$ (which are chosen in items \eqref{item:choice-xi-Q} and \eqref{item:choice-leq-Q} of \cref{defi:type-II-Q}). 
    We define $\U \coloneqq \D^{[g_1]}\oplus \cdots \oplus \D^{[g_k]}$ and let $\B = \{ u_1, \ldots, u_k\}$ be its canonical graded basis. 
    Let $x_1 < \ldots < x_{\ell}$ be the elements of $\{ x \in \supp \kappa \mid x \leq g_0\inv x\inv \}$ and set 
    \[
        \begin{pmatrix}
        \Phi(\bar 0, x_1)&& \\
        & \ddots &\\
        && \Phi(\bar 0, x_{\ell})
    \end{pmatrix} \in M_k(\D)
    \]
    (see \cref{defi:blocks-of-Phi}). 
    We define $B\from \U \times \U \to \D$ to be the $\vphi_0$-sesquilinear form such that $B(u_i, u_j) \coloneqq \Phi_{ij}$. 
    It is straightforward to see that $B$ is nondegenerate and $(\U, B)$ has inertia determined by $\kappa$. 
    To get the matrix $\Theta$, we also proceed as in \cref{prop:m-not-n-Type-II-correspondence}. 
    We consider a different graded basis $\tilde B = \{ \tilde u_1, \ldots, \tilde u_k \}$, where $\tilde u_i$ is defined as in \cref{eq:tilde-u_i-from-u_i}, and let $\tilde \Phi$ be the matrix representing $B$ with respect to the graded basis $\tilde \B$. 
    Following the procedure in \cref{ssec:undoubling} with $\tilde \Phi$ playing the role of $\Phi$, we get  $\Theta$ as in \cref{eq:puting-the-blocks-of-Phi-together-version-Q}.
    %
    % We can use the choices in items \eqref{item:choice-xi-Q} and \eqref{item:choice-leq-Q} to choose the pair $(\U, B)$ as in \cref{def:model-grd-MxM-odd-or-QxQ}, by following a construction analogous to the one in \cref{ssec:grds-osp}. 
    % With respect with the graded basis $\B = \{ u_1, \ldots, u_k \}$ used there, $B$ is represented by the matrix $\Phi \in M_{k_\bz | k_\bo}(\D)$ as in \cref{eq:puting-the-blocks-of-Phi-together}. 
    % To get the matrix $\Theta$ as in \cref{eq:puting-the-blocks-of-Phi-together-version-A}, we will consider a different graded basis $\tilde B = \{ \tilde u_1, \ldots, \tilde u_k \}$ for $\U$, where
    % \[\label{eq:tilde-u_i-from-u_i}
    %     \tilde u_i \coloneqq 
    %     \begin{cases}
    %         \sqrt{\chi(\deg u_i)\inv} u_i, & \text{if }(\deg u_i) T = g_0\inv (\deg u_i)\inv T;\\
    %         \hfill \chi(\deg u_i)\inv u_i, & \text{if }(\deg u_i) T < g_0\inv (\deg u_i)\inv T;\\
    %         \hfill u_i, & \text{if }(\deg u_i) T > g_0\inv (\deg u_i)\inv T.
    %     \end{cases}
    % \]
    % Let $\tilde \Phi$ be the matrix representing $B$ with respect to the graded basis $\tilde \B$. 
    % Following the procedure in \cref{ssec:undoubling} with $\tilde \Phi$ playing the role of $\Phi$, we get  $\Theta$ as in \cref{eq:puting-the-blocks-of-Phi-together-version-A}. 
\end{proof}

% --------------------


\begin{thm}\label{thm:final-Q(n)}
    Let $n \in \ZZ_{> 0}$ and 
    let $L$ denote the Lie superalgebra $Q(n)$. 
    Every grading on $L$ is isomorphic to either $\Gamma_Q^{\mathrm{(I)}}(T^+, \beta^+, h, \kappa)$ or $\Gamma_Q^{\mathrm{(II)}}(T^+, \beta^+, h, \kappa, g_0)$ as in \cref{def:Q-Type-I,defi:type-II-Q}. 
    Gradings belonging to different types are not isomorphic. 
    Within each type, we have:
    
    \noindent\boxed{\mathrm{Type \,\,I}}
    
    \noindent $\Gamma_Q^{\mathrm{(I)}}(T^+, \beta^+, h, \kappa) \iso \Gamma_Q^{\mathrm{(I)}}(T'^+, \beta'^+, h', \kappa')$ \IFF  $T'^+ = T^+$, $h = h'$ and one of the following conditions holds:
	\begin{enumerate}[(i)]
	    \item $\beta'^+ = \beta^+$ and there is $g\in G$ such that $g \cdot \kappa'=\kappa$; 
	    \item $\beta'= (\beta^+)\inv$ and there is $g\in G$ such that $g \cdot \kappa'= \kappa\Star$.
	\end{enumerate}

    \noindent\boxed{\mathrm{Type \,\,II}}
    
    \noindent $\Gamma_A^{\mathrm{(II)}}(T^+, \beta^+, h, \kappa, g_0) \iso \Gamma_A^{\mathrm{(II)}}(T'^+, \beta'^+, h', \kappa', g_0')$ \IFF
    $T'^+ =T^+$, $\beta'^+ = \beta^+$, $h' = h$ and there is $g \in G$ such that $g\cdot\kappa' = \kappa$ and $g_0' = g^{-2}g_0$. 
\end{thm}

\begin{proof}
    By \cref{cor:transfer-R-vphi-to-L}, the isomorphism classes of gradings on the Lie superalgebra $Q(n)$ is in bijection with the isomorphism classes of gradings on associative superalgebra $Q(n+1) \times Q(n+1)\sop$  endowed with the exchange superinvolution. 
    The correspondence between the Type I gradings is clear, and for Type II gradings is established in \cref{prop:Q-Type-II-correspondence}. 
    Hence, the result follows from \cref{thm:QxQ-type-I}, and for Type II gradings follows from
    \cref{cor:QxQ-reduced-to-MxM}.
    % The classification of gradings on the Lie superalgebra $Q(n)$ is equivalent to the classification of gradings on the associative superalgebra $S \times S\sop$, endowed with exchange superinvolution, by \cref{cor:transfer-R-vphi-to-L}. 
    % The gradings on the latter 
    
    % That there are this two types, it comes from .
    
    % The isomorphism condition for Type I, follows from \cref{thm:QxQ-type-I}, but there its in in terms of different paramenters.
    
    % The isomorphism condition for Type II gradings follows from \cref{cor:QxQ-reduced-to-MxM,prop:Q-Type-II-correspondence}. 
\end{proof}


% ---------------------------

\subsection{Gradings on \texorpdfstring{$A(n,n)$}{A(n,n)}}\label{ssec:grds-on-Ann}

We now go to our final series of Lie superalgebras, $A(n,n)$. 
Let $S = M(n+1, n+1)$,  
so $L = \mathfrak{psl} (n+1, n+1) = S^{(1)}/Z(S^{(1)})$. 

Gradings on $S$ are separated into two classes: even gradings, which are similar to what we saw in \cref{ssec:grds-on-A-m-n}, and odd gradings. 
The gradings on $L$ induced by even gradings on $S$ will be said to be of \emph{Type I\textsubscript{M}}, and the gradings on $L$ induced by odd gradings on $S$ will be said to be \emph{Type I\textsubscript{Q}}. 

\begin{remark}\label{prop:Q-implies-GxZZ2-grading-on-M}
    Our notation is justified by the following fact: 
    every $G$-grading of Type I on the Lie superalgebra $Q(n)$ gives rise to a $G\times \ZZ_2$-grading of Type I\textsubscript{Q} on $A(n,n)$. 
    Indeed, the associative superalgebra $Q(n+1)$ consists of the fixed points of the order $2$ automorphism $\pi$ of $M(n+1, n+1)$, and any automorphism of $Q(n+1)$ extends uniquely to an automorphism of $M(n+1, n+1)$ commuting with $\pi$ (see \cref{sec:Aut-Lie-chap}). 
\end{remark}

\begin{defi}\label{def:Type-I_M}
    Let $n\in \ZZ_{> 0}$, $T \subseteq G$ be a finite subgroup, $\beta\from T\times T \to \FF^\times$ be a nondegenerate alternating bicharacter, and $\kappa_\bz, \kappa_\bo \from G/T \to \ZZ_{\geq 0}$ be maps with finite support such that $|\kappa_\bz| \sqrt{|T|} = |\kappa_\bo| \sqrt{|T|} = n+1$. 
    We define $\Gamma^{\mathrm{(I_M)}}_A(T, \beta, \kappa_\bz, \kappa_\bo)$ to be the grading on $L$ induced from the grading $\Gamma^{\mathrm{(I)}}_A(T, \beta, \kappa_\bz, \kappa_\bo)$ on $M(n+1, n+1)^{(1)}$ (\cref{def:A-Type-I}) by reduction modulo the center.
\end{defi}

To parametrize the gradings of Type I\textsubscript{Q}, we recall the parametrization of odd gradings on $S$ in terms of the group $G$ (\cref{sec:assc-only-G}, \cref{cor:iso-odd-M-simplified}). 
The character $\chi_0 \in \widehat{T^+}$ in the next definition plays the role of $\chi$ there. 

\begin{defi}\label{def:Type-I_Q}
    Let $T^+ \subseteq G$ be a finite subgroup, let $\beta^+\from T^+ \times T^+ \to \FF^\times$ be an alternating bicharacter with $\rad \beta^+ = \langle t_p \rangle$, where $t_p\in T^+$ is an element of order $2$, and let $\kappa \from G/T^+ \to \ZZ_{\geq 0}$ be a map with finite support such that $|\kappa| \sqrt{2|T^+|} = n+1$. 
    Choose a character $\chi_0 \in \widehat{T^+}$ such that $\chi_0 (t_p) = -1$. 
    Then let $a \in T^+$ be the unique element such that $\chi_0^2 = \beta^+(a, \cdot)$ and $\chi_0(a) = 1$ (see \cref{lemma:chi-defines-a}). 
    For each element $h \in G$ such that $h^2 = a$, we set $t_1 \coloneqq (h, \bar 1) \in G^\#$, $T^- \coloneqq t_1 T^+ \subseteq G^\#$ and $T \coloneqq T^+ \cup T^-$. 
    Let $p\from T \to \ZZ_2$ be the homomorphism with kernel $T^+$, and let $\beta\from T \times T \to \FF^\times$ be the unique alternating bicharacter such that $\beta\restriction_{T^+ \times T^+} = \beta^+$ and $\beta(t_1, t) = \chi(t)$, for all $t\in T^+$ (see \cref{lemma:existence-beta}). 
    We will denote by  $\Gamma^{\mathrm{(I_Q)}}_A(T^+, \beta^+, h, \kappa)$ the grading on $L$ induced from the grading $\Gamma_M(T, \beta, p, \kappa)$ on $S$ (see \cref{def:Gamma-T-beta-kappa-odd}) by restriction and reduction modulo the center. 
\end{defi}

Recall that an element $h \in G$ (and, hence, a grading $\Gamma^{\mathrm{(I_Q)}}_A(T^+, \beta^+, h, \kappa)$) exists \IFF $\theta(T^+_{[2]})^\perp \subseteq \barr G^{[2]}$ (see \cref{def:A_[m],defi:perp,prop:O_M-non-empty}).

The Type II gradings on $L$ will also be subdivided into different types. 
As in the case $m\neq n$ (\cref{ssec:grds-on-A-m-n}), these gradings correspond to gradings on $(R, \vphi)$ making $R$ graded-simple, where $R \coloneqq S \times S\sop$ and $\vphi$ is the exchange superinvolution. 

If the grading on $(R,\vphi)$ is even, then, by \cref{thm:MxM-even}, $R$ is isomorphic to $M^{\mathrm{ex}} (T, \beta, \kappa_\bz, \kappa_\bo, g_0)$, where $T$ is a finite elementary $2$-group, $\beta\from {T\times T} \to \FF^\times$ is an alternating bicharacter with $\rad \beta = \langle f \rangle$ for some $e \neq f \in T$, $g_0$ is an element in $G^\#$, and $\kappa_\bz, \kappa_\bo \from G/T \to \ZZ_{\geq 0}$ are $g_0$-admissible maps (see \cref{inertia-even-and-odd-case}) such that $|\kappa_\bz| \sqrt{|T|} = |\kappa_\bo| \sqrt{|T|} = n+1$ (see \cref{def:model-grd-MxM-even}). 
Write $g_0$ as $(h_0, p_0)$, with $h_0 \in G$ and $p_0 \in \ZZ_2$. 
If $p_0 = \bar 0$, we will say that the corresponding grading on $L$ is of Type II\textsubscript{osp}. 
If $p_0 = \bar 1$, we will say that the corresponding grading on $L$ is of Type II\textsubscript{P}. 

\begin{remark}
    A $G$-grading on the Lie superalgebra $\osp(m+1,n+1)$ (respectively, $P(n)$) gives rise to a $G \times \ZZ_2$-grading on $A(m,n)$ of Type II\textsubscript{osp} (respectively, II\textsubscript{P}). 
\end{remark}

\begin{defi}\label{def:type-II-osp}
    Let $n\in \ZZ_{> 0}$. 
    We define $\Gamma^{\mathrm{(II_{osp})}}_A(T, \beta, \kappa_\bz, \kappa_\bo, h_0)$ to be the grading on $L$ induced from the grading $\Gamma^{\mathrm{(II)}}_A(T, \beta, \kappa_\bz, \kappa_\bo, h_0)$ on $M(n+1, n+1)^{(1)}$ (\cref{defi:type-II-A-m-not-n}) by reduction modulo the center.
\end{defi}

To construct a model for gradings of Type II\textsubscript{P}, we follow a similar approach as for Type II\textsubscript{osp}, but it is simpler since we do not need to choose $\xi$ and $\leq$ to define the matrix $\Theta$ (in the same way, these choices were needed in \cref{ssec:grds-osp} but not in \cref{ssec:grds-P(n)}). 
Also, $\kappa_\bo$ is determined by $\kappa_\bz$ and $h_0$ (recall \cref{inertia-even-and-odd-case}).

\begin{defi}\label{def:type-II-P}
    Let $\barr G \coloneqq G/\langle f \rangle$, $T \coloneqq T/\langle f \rangle$, and let $\bar \beta$ be the nondegenerate alternating bicharacter on $\barr T$ induced by $\beta$. 
    Choose:
    \begin{enumerate}[(i)]
        \item a standard realization $\barr \D$ associated to $(\barr T, \barr \beta)$; 
        \item a subgroup $K \subseteq T$ such that $T = K \times \langle f \rangle$.
        \item a $k$-tuple $\gamma_\bz = (g_1, \ldots, g_k)$ of elements in $G$ realizing $\kappa_\bz$. 
    \end{enumerate}
    Let $\barr \mu\from \barr T \to \FF^\times$ be the map associated to the transposition on $\barr \D$, 
    let $\chi \in \widehat{T}$ be the character such that $\chi(K) = 1$ and $\chi(f) = -1$, and extend $\chi$ to a character of $G$, which will also be denoted by $\chi$. 
    Then define $\mu \coloneqq \bar\mu \circ \pi$, where $\pi\from G \to \barr G$ is the natural homomorphism, and fix $\eta\from T \to \pmone$ as in \cref{eq:fix-eta-undouble}. 
    Also, set $\gamma_\bo = (h_0\inv g_1\inv, \ldots, h_0\inv g_k\inv)$.   
    Consider the $\barr G$-grading $\Gamma_M(\barr T, \barr \beta, \kappa_\bz, \kappa_\bo)$ on $S \coloneqq M(n+1,n+1)$ using the choices above (see \cref{def:Gamma-T-beta-kappa-even}), and consider its restriction to $S^{(1)}$. % = \bigoplus_{\bar g \in \barr G} S^{(1)}_{\bar g}$. 
    Consider ${\Theta \in S}$ 
    given by
    \[\label{eq:Theta-for-Type-II-P}
        %
        \sbox0{$\begin{matrix}
            1&& \\
            & \ddots &\\
            && 1
        \end{matrix}$}
        %
        \sbox1{$\begin{matrix}
            \chi(h_0\inv g_1^{-2})&& \\
            & \ddots &\\
            && \chi(h_0\inv g_k^{-2})
        \end{matrix}$}
        %
        \Theta \coloneqq
        \left(\begin{array}{c|c}
            0 & \usebox{0}\\
            \hline
            \usebox{1} & 0
        \end{array}\right) \tensor 1_{\barr \D}.
    \]
    %
    and ${\theta\from S \to S}$ as in
    \cref{eq:theta-with-matrix-3}. 
    We define $\Gamma_A^{\mathrm{(II_P)}}(T, \beta, \kappa_\bz, h_0)$ to be the $G$-grading on $L = S^{(1)}/Z(S^{(1)})$ induced from the grading $S^{(1)} = \bigoplus_{g\in G} S^{(1)}_g$, where
    \[
        S^{(1)}_{g} \coloneqq \{ s\in S^{(1)}_{\bar g} \mid \theta(s) = - \chi(g) s \},
    \]
    for all $g\in G$. 
\end{defi}

We now proceed to the last case, the odd gradings of Type II, which will be referred to as gradings of Type II\textsubscript{Q}. 
In this case parameters $R$ are bla...
We can use $\chi$ to move it to $G$. 




\subsection{Old...}

% % -----------

% Recall that the graded algebra $M^{\mathrm{ex}} (T, \beta, \kappa_\bz, \kappa_\bo, g_0)$ corresponds to $(\eta, \kappa, g_0, \delta) \in \mathbf{I}(T, \beta, p)$, where $\eta\from T \to \pmone$ is fixed, $p\from T \to \ZZ_2$ is the trivial homomorphism, and $\delta = 1$. 
% The map $\eta$ was determined by fixing a standard realization $\D$ for $(T, \beta, e)$ (see \cref{def:std-realization-MxM-QxQ}), 
% namely, $\eta$ is the map associated to the superinvolution $\vphi_{\mc C} \tensor \vphi_{\mc M}$ on $\D$. 

% We will use the parameters $(T, \beta, \kappa_\bz, \kappa_\bo, g_0)$ to construct a representative for the Type II grading in terms of the ``undoubled'' model, \ie, we will construct a $\barr G$-grading on $S$, where $\barr G \coloneqq G/\langle f \rangle$, and a super-anti-automorphism $\theta\from S\to S$, as in \cref{ssec:undoubling}. 
% As in that subsection, let $\pi\from G \to \barr G$ denote the natural homomorphism, set $\barr T \coloneqq T/\langle f \rangle$, let $\barr \beta$ be the (nondegenerate) bicharacter on $\barr T$ induced by $\beta$, and consider $\kappa_\bz$ and $\kappa_\bo$ as maps defined on $\barr G/\barr T \iso G/T$. 

% Let $\barr \D$ be a standard realization associated to $(\barr T, \bar\beta)$ (see \cref{def:standard-realization}), let $\theta_0\from \barr \D \to \barr \D$ be the transposition map, and let $\bar \mu\from \barr T \to \FF^\times$ be the map associated to $\theta_0$. 
% (Recall that this also gives us a choice of elements $0 \neq X_{\bar t} \in \barr\D_{\bar t}$, for all $\bar t \in \barr T$.)

% Choose a complement $K$ for $\langle f \rangle$ in $T$, \ie, a subgroup $K \subseteq T$ such that $T = K \times \langle f \rangle$ (it can be done, since $T$ is a elementary $2$-group). 
% Let $\chi\from T \to \FF^\times$ be the character defined by $\chi(K) = 1$ and $\chi(f) = -1$, and set $\mu \coloneqq \barr \mu \circ \pi$. 
% Then set 
% \[\label{eq:fix-eta-undouble}
%     \forall t\in T, \quad \eta(t) \coloneqq \mu(t) \chi\inv(t).
% \]
% Note that this definition agrees with \cref{def:std-realization-MxM-QxQ} (see the proof of \cref{prop:m-not-n-Type-II-correspondence}). 

% Extend $\chi$ to $G$. 
% It remains to define a graded right $\barr \D$-supermodule $\barr \U$, and the matrix $\Theta$. 
% % Note that, once we fixed $\eta$, we can say that $(\kappa_\bz, \kappa_\bo)$ is $g_0$-balanced. 
% The following is analogous to the construction in \cref{ssec:grds-osp}. 

% Let $\xi\from G/ T \to G$ be a set-theoretic section of the natural homomorphism, and let $\leq$ be a total order on the set $G/T \iso \barr G/ \barr T$ (or just its finite subset $\supp \kappa_\bz \cup \supp \kappa_\bo$) with no elements between $x$ and $\bar g_0\inv x\inv$. 
% Changing $\xi$ if necessary, we may assume that $\xi(g_0\inv x\inv) = g_0\inv \xi(x)\inv$ if $x < g_0\inv x\inv$. 
% Set $k_\bz \coloneqq |\kappa_\bz|$ and let $\gamma_\bz$ be the $k_\bz$-tuple given by putting the elements of $\{ \xi(x) \mid x \in \supp \kappa_\bz\} \subseteq G$ following the order $\leq$ and repeating $\kappa_\bz(x)$ times each element $\xi(x)$. 
% % Clearly, $\gamma_\bz$ realizes $\kappa_\bz$. 
% Similarly, set $k_\bo \coloneqq |\kappa_\bo|$ and construct the $k_\bo$-tuple $\gamma_\bo$.
% % , which realizes $\kappa_\bo$. 
% % We then define $\barr \U\even \coloneqq \barr \D^{[\bar g_1]} \oplus \cdots \oplus \barr \D^{[\bar g_{k_\bz}]}$, $\barr \U\odd \coloneqq \barr \D^{[\bar h_1]} \oplus \cdots \oplus \barr \D^{[\bar h_{k_\bz}]}$ and $\barr \U \coloneqq \barr \U\even \oplus \barr \U\odd$. 
% Let $\bar \gamma_\bz$ and $\bar \gamma_\bo$ to be the tuples of elements of $\barr G$ consisting of the image under $\pi\from G\to \barr G$ of the entries of $\gamma_\bz$ and $\gamma_\bo$, respectively. 
% Clearly, $\barr\gamma_\bz$ and $\barr\gamma_\bo$ realize $\kappa_\bz$ and $\kappa_\bo$ (see \cref{defi:gamma-realizes-kappa}). 
% Consider on $M_{k_\bz | k_\bo}(\FF)$ the elementary grading determined by $(\bar  \gamma_\bz, \bar \gamma_\bo)$ (see \cref{defi:elementary-grd-super}). 
% We identify the $\barr G$-graded superalgebra $M_{k_\bz | k_\bo}(\barr \D) = M_{k_\bz | k_\bo}(\FF) \tensor \barr\D$ with $S = M(m+1, n+1)$ via Kronecker product. 

% % We have that $S \iso \End_{\barr \D} (\barr \U)$ as $\barr G$-graded superalgebra. 
% % Using the $\barr G^\#$-graded canonical basis of $\barr \U$, we can identify $\End_{\barr \D} (\barr \U)$ with $M_{k_\bz | k_\bo}(\barr \D)$. 
% % From now on, we will identify $S$ with $M_{k_\bz | k_\bo}(\barr \D)$. 

% \begin{defi}\label{defi:blocks-of-Theta}
%     Let $i\in \ZZ_2$ and $x \in G/T$. 
%     If $g_0x^2 = T$, we put $t \coloneqq g_0 \xi(x)^2 \in T$ and let $\bar t \in \barr T$ be its image under the canonical homomorphism $T \to \barr T$. 
%     We define $\Theta(i, x)$ to be the following $\kappa(x) \times \kappa(x)$-matrix with entries in $\D$:
%     %
%     \begin{enumerate}[(i)]
%         \item $\chi(\xi(x))\inv I_{\kappa(x)} \tensor X_{\bar t}$ if $(-1)^i \eta(t) = +1$;
%         %
% 		\item  $\chi(\xi(x))\inv J_{\kappa(x)} \tensor X_{\bar t}$, where $J_{\kappa(x)} \coloneqq \begin{pmatrix}
% 				      0                & I_{\kappa(x)/2} \\
% 				      -I_{\kappa(x)/2} & 0
% 			      \end{pmatrix}$, if  $(-1)^i \eta(t) = -1$ (recall that, in this case, $\kappa (x)$ is even by \cref{inertia-even-and-odd-case}). 
% 	\end{enumerate}
%     %
%     If $g_0 x^2 \neq T$, we define $\Phi(i, x)$ to be the following $2\kappa(x) \times 2\kappa(x)$-matrix with entries in $\barr\D$:
%     %
%     \begin{enumerate}[(i)]
%         %
%         \setcounter{enumi}{2}
%         %
% 		\item $\begin{pmatrix}
% 			0                                                  & \chi(\xi(x))\inv I_{\kappa(x)} \\
% 			(-1)^{i} \chi(\xi(x))\inv I_{\kappa(x)} & 0
% 		\end{pmatrix} \tensor 1$, where $1$ is the identity element of $\barr\D$. 
%     \end{enumerate}
% \end{defi}


% % For each $\bar g_i$, $1 \leq i \leq k_\bz$, and for each $\bar h_j$, $1 \leq j \leq k_\bo$, choose representatives $g_i \in G$ and $h_j \in G$, respectively. 
% % Extend $\chi$ to $G$.
% % , and define $\Lambda \in M_{k_\bz | k_\bo}(\FF) \subseteq M_{k_\bz | k_\bo}(\barr \D)$ to be the diagonal matrix 
% % \[\label{eq:Lambda-even-case}
% %     %
% %     \sbox0{$\begin{matrix}
% %         \chi(g_1)\inv && \\
% %         & \ddots &\\
% %         && \chi(g_{k_\bz})\inv
% %     \end{matrix}$}
% %     %
% %     \sbox1{$\begin{matrix}
% %         \chi(h_1)\inv && \\
% %         & \ddots &\\
% %         && \chi(h_{k_\bo})\inv
% %     \end{matrix}$}
% %     %
% %     \Lambda \coloneqq
% %     \left(\begin{array}{c|c}
% %             \usebox{0} & 0\\
% %             \hline
% %             0 & \usebox{1}
% %         \end{array}\right).
% % \]
% % %
% % For each $\bar t \in \barr T$, choose $0\neq X_{\bar t} \in \barr D_{\bar t}$. 

% Let $x_1 < \ldots < x_{\ell_\bz}$ be the elements of $\{ x \in \supp \kappa_\bz \mid x \leq g_0\inv x^2 \}$ and, similarly, let $y_1 < \ldots < y_{\ell_\bo}$ be the elements of $\{ y \in \supp \kappa_\bo \mid y \leq g_0\inv y^2 \}$. 
% Then, we define 
% \[\label{eq:puting-the-blocks-of-Phi-together-version-A}
%     %
%     \sbox0{$\begin{matrix}
%         \Theta(\bar 0, x_1)&& \\
%         & \ddots &\\
%         && \Theta(\bar 0, x_{\ell_\bz})
%     \end{matrix}$}
%     %
%     \sbox1{$\begin{matrix}
%         \Theta(\bar 1, y_1)&& \\
%         & \ddots &\\
%         && \Theta(\bar 1, y_{\ell_\bo})
%     \end{matrix}$}
%     %
%     \Theta \coloneqq
%     \left(\begin{array}{c|c}
%             \usebox{0} & 0\\
%             \hline
%             0 & \usebox{1}
%         \end{array}\right).
% \]
% %
% Finally, we define the super-anti-automorphism $\theta\from S\to S$ by \cref{eq:theta-with-matrix-2}. 
% Since $\theta_0$ is the transposition on $\barr \D$, $\theta_0(X)\stransp \in M_{k_\bz \mid k_\bo} (\barr \D)$ in becomes $X\stransp \in M(m+1,n+1)$. 
% Hence, \cref{eq:theta-with-matrix-2} reduces to 
% \[\label{eq:theta-with-matrix-3}
%     \forall X\in M_{k_\bz \mid k_\bo} (\barr \D), \quad \theta(X) = \Theta\inv\, X\stransp\, \Theta.
% \]

% \begin{defi}\label{defi:type-II-A-m-not-n}
%     Let $m,n \in \ZZ_{\geq 0}$, $m\neq n$. 
%     Let $T \subseteq G$ be a finite $2$-elementary subgroup, let $\beta\from {T\times T} \to \FF^\times$ be an alternating bicharacter with $\rad \beta = \langle f \rangle$, for some $f\in T$, and let $g_0 \in G$. 
%     Set $\barr G \coloneqq G/\langle f \rangle$, $T \coloneqq T/\langle f \rangle$, and let $\bar \beta$ be the nondegenerate alternating bicharacter on $\barr T$ induced by $\beta$. 
%     Choose:
%     \begin{enumerate}[(i)]
%         \item a standard realization $\barr \D$ associated to $(\barr T, \barr \beta)$; 
%         \item a subgroup $K \subseteq T$ such that $T = K \times \langle f \rangle$; 
%         \item a set-theoretic section $\xi\from G/T \to G$ for the quotient homomorphism $G \to G/T$;
%         \item a total order $\leq$ on $G/T$ such that there are no elements between $x$ and $\bar g_0\inv x\inv$, for all $x\in G/T$. 
%     \end{enumerate}
%     Let $\barr \mu\from \barr T \to \FF^\times$ be the map determining the transposition on $\barr \D$ (see \cref{ssec:param-D-vphi}), and 
%     let $\chi \in \widehat{T}$ be the character such that $\chi(K) = 1$ and $\chi(f) = -1$, and extend it to a character on $\widehat{G}$, also denoted by $\chi$. 
%     Then define $\mu \coloneqq \bar\mu \circ \pi$, where $\pi\from G \to \barr G$ is the canonical homomorphism, and fix $\eta\from T \to \pmone$ as in \cref{eq:fix-eta-undouble}. 
%     Let $\kappa_\bz, \kappa_\bo \from G/T \to \ZZ_{\geq 0}$ be $g_0$-admissible maps (\cref{inertia-even-and-odd-case}) such that $m+1 = k_\bz \sqrt{|T|/2}$ and $n+1 = k_\bo \sqrt{|T|/2}$, where $k_i \coloneqq |\kappa_i|$, $i\in \ZZ_2$. 
%     Then construct tuples $\bar\gamma_\bz$ and $\bar\gamma_\bo$ realizing $\kappa_\bz$ and $\kappa_\bo$, respectively, as did before \cref{defi:blocks-of-Theta}. 
%     Consider the $\barr G$-grading $\Gamma_M(\barr T, \barr \beta, \kappa_\bz, \kappa_\bo)$ on $S \coloneqq M(m+1,n+1)$ using the choices above (see \cref{def:Gamma-T-beta-kappa-even}), and consider its restriction $\Gamma : L = \bigoplus_{\bar g \in \barr G} L_{\bar g}$ to $L \coloneqq S^{(1)}$. 
%     Consider ${\Theta \in S}$ as in \cref{eq:puting-the-blocks-of-Phi-together-version-A} and ${\theta\from S \to S}$ as in
%     \cref{eq:theta-with-matrix-3}. 
%     We define $\Gamma_A^{\mathrm{(II)}}(T, \beta, \kappa_\bz, \kappa_\bo, g_0)$ to be the $G$-grading on $L$ given by
%     \[
%         L_{g} \coloneqq \{ s\in L_{\bar g} \mid s = - \chi(g) s \},
%     \]
%     for all $g\in G$. 
% \end{defi}

% \begin{prop}\label{prop:m-not-n-Type-II-correspondence}
%     Consider $(R, \vphi) \coloneqq M^{\mathrm{ex}}(T, \beta, \kappa_\bz, \kappa_\bo, g_0)$ (\cref{def:model-grd-MxM-even}). 
%     Then $\Skew (R,\vphi)$ is isomorphic to $M(m+1, n+1)$ endowed with $\Gamma_A^{\mathrm{(II)}}(T, \beta, \kappa_\bz, \kappa_\bo, g_0)$. 
% \end{prop}

% \begin{proof}
%     We will now show how the choices in \cref{defi:type-II-A-m-not-n} correspond to the choices in Definitions \ref{def:std-realization-MxM-QxQ}(a) and \ref{def:model-grd-MxM-even}. 
    
%     The choices in items (i) and (ii) give us a way to make to the choices in \ref{def:std-realization-MxM-QxQ}(a)
%     % First, we note that the choice of $K$ in \ref{def:std-realization-MxM-QxQ}(a)\eqref{item:K-can-be-orthogonal-to-t_1} can be the same as in \cref{defi:type-II-A-m-not-n}(ii). 
%     Recall that a standard realization $\barr \D$ associated to $(\barr T, \bar \beta)$ (choice \cref{defi:type-II-A-m-not-n}(i)) is obtained by choosing subgroups $\barr A$ and $\barr B$ of $\barr T$ such that $\barr T = \barr A \times \barr B$ and $\beta (\barr A, \barr A) = \beta (\barr B, \barr B) = 1$. 
%     Note that $\pi\from G \to \barr G$ restricts to an isomorphism $K \to \barr T$, and $\beta(s,t) = \barr\beta(\bar s, \bar t)$, for all $s,t \in K$. 
%     Hence, the choice of the subsets $\barr  A$, $\barr  B$ as above is, then, equivalent to a choice of subgroups $A,B \subseteq K$ such that $K = A\times B$ and $\beta (A, A) = \beta (B, B) = 1$. 
%     In other words, our choices of $\barr \D$ and $K$ give us the same information as the choice of $\mc M$ in \cref{def:std-realization-MxM-QxQ}(a)\eqref{item:choose-mc-M}. 
%     Also, by \cref{lemma:transp-std-realization}, $\bar\mu (ab) = \beta(a,b)$ for all $a\in \barr A$ and $b\in \barr B$, so $\mu\restriction_{K}$ is the map determining $\vphi_{\mc M})$. 
%     Further, since $\eta\restriction_{K} = \mu\restriction_{K}$ and $\eta\restriction_{\langle f \rangle} = \chi\restriction_{\langle f \rangle}$, the map $\eta$ defined \cref{eq:fix-eta-undouble} is the map determining to $\vphi_{\mc C} \tensor \vphi_{\mc M}$. 
    
%     We can use choices in items (iii) and (iv) to choose the pair $(\U, B)$ as in \cref{def:model-grd-MxM-even}, by following the exact same construction done in \cref{ssec:grds-osp}. 
    
%     With those choices being made to construct $M^{\mathrm{ex}}(T, \beta, \kappa_\bz, \kappa_\bo, g_0)$, and choosing $\chi \in \widehat{G}$ such that $\chi(K) = 1$ and $\chi(f) = -1$, it is clear that if follow the steps in \cref{ssec:undoubling}, the result follows. 
% \end{proof}

% For the next result, we will we fix choices (i) and (ii), for each pair $(T, \beta)$ as in \cref{defi:type-II-A-m-not-n}, in the ``Type II'' case. 
% Recall that, for every $\kappa\Star\from G/T \to \ZZ_{\geq 0}$, we $\kappa\Star\from G/T \to \ZZ_{\geq 0}$ by $\kappa\Star(x) = \kappa(x)\inv$, for all $x \in G/T$ (see \cref{ssec:superdual}).


\subsection{Gradings on \texorpdfstring{$A(n,n)$}{A(n,n)}}

Here we can have even and odd gradings. 
For the even ones, two kappas and define:

\begin{defi}
    \mbox{}
    
    \boxed{\mathrm{Type \,\,I}_M}
    
        even type I
        
    \boxed{\mathrm{Type \,\,II}_{osp}}
    
        even type II with even $g_0$
        
    \boxed{\mathrm{Type \,\,II}_P}
    
        even type II with odd $g_0$
\end{defi}

The reference for other types of Lie superalgebras come from ...

For odd gradings, we define: 

\begin{defi}
    \mbox{}
    
    \boxed{\mathrm{Type \,\,I}_Q}
    
        odd type I
        
    \boxed{\mathrm{Type \,\,II}_Q}
    
        odd type II
\end{defi}

The reference for other Lie superalgebras come from ....

For even gradings, we do standard realizations and $\Phi$ just like on case $m \neq n$. 
(there is a tiny difference on the isomorphism condition, though)

For odd gradings, 

\begin{defi}
    Type II gradings, standard realization in terms of $G^\#$. 
    Choose $t_p$, $t_1$ and $S$ and use it to define $\chi$. 
\end{defi}

Compare definition with case (b) in previous definition. 

Using $\bar t_1$, we can find another character, $\chi_0$, which is $\beta( t_1, \cdot)$ restricted to $T^+$. 
It is different than $\chi$ by its values on $t_p$ and $f$. 
Then we can reduce $(\bar T, \bar \beta)$ to double bars like in Chapter 1, and have a description which is only in terms of $G$. 
(actually, we first fix $\chi_0$ and then use it to choose a $t_1$ such that $t_1^2 = a$, where $a$ the element such that $\chi^2 = \beta(a, \cdot)$).

\begin{defi}
    Type II gradings, standard realization in terms of $G$. 
\end{defi}

We can then find a suitable $\Phi$ from $\kappa$ fixing stuff.

\begin{defi}
    model for the grading.
\end{defi}

\begin{thm}
    \mbox{}
    
    \boxed{\mathrm{Type \,\,I}_M}
    
        even type I
        
    \boxed{\mathrm{Type \,\,I}_Q}
    
        odd type I
        
    \boxed{\mathrm{Type \,\,II}_{osp}}
    
        even type II with even $g_0$
        
    \boxed{\mathrm{Type \,\,II}_P}
    
        even type II with odd $g_0$
        
    \boxed{\mathrm{Type \,\,II}_Q}
    
        odd type II
\end{thm}

As a final remark, we will see what all these different types of gradings mean in therms of the $\widehat{G^\#}$-action. 
(Another subsection? we need a variable $\chi$ and it does not depend on the undoubled model, it is probably easier in the doubled one.)