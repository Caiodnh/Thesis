\chapter{$G$-Gradings on Universal Algebras and $\widehat G$-actions}

In this chapter we will revisit the correspondence between $G$-gradings and $\widehat G$-actions, which can be found in many places in the literature (see, \eg, ??, ?? and ??). 
Nevertheless, our goal is to generalize it to the context of universal algebras. 
This way we can handle algebras and superalgebras, with or without (super)involutions, in a uniform manner. 

Another advantage of this approach is to have a formal result comparing gradings on univeral algebras with different signature (see Definition \ref{def:universal-algebra} and Theorem ??). 
This will be used in Chapter ?? to transfer gradings on associative superalgebras with superinvolutions to Lie superalgebras. 
The same sort of transfer has been used in other works (see \cite{livromicha} and Paper with Adrian ??), but without having the result stated properly. 

\section{Universal Algebras}

A universal algebra is a space with an arbitrary collection (possibly empty) of operations, with may also have different ``arities'' (see Definition \ref{def:universal-algebra} bellow). 
Our approach to universal algebras differs from the one in most books on the subject (see, \eg, ?? and ??) in the sense we will be working on the (monoidal) category of vector spaces instead of the (monoidal) category of sets. 
This is the approach on ?? (book used by Felipe) and it has recently been applied to gradings (and graded identities) on ??. 

\begin{notation}
    For a vector space $A$ and nonnegative integer $n$, we will denote the $n^{th}$-tensor power of $A$ by $A^{\tensor n}$, \ie,
    $A^{\tensor n} \coloneqq \underbrace{A\otimes\cdots\otimes A}_{n \text{ times}}$. 
    In the case $n = 0$, we will follow the convention that $A^{\tensor 0} \coloneqq \FF$.
\end{notation}

\begin{defi}\label{def:universal-algebra}
    A \emph{signature} $\Omega$ is a set together with a family $\{ \Omega_n \}_{n \geq 0}$ of pairwise disjoint subsets such that $\Omega = \bigcup_{n \geq 0} \Omega_n$. 
    An \emph{$n$-ary operation} on a vector space $A$ is a linear map from $A^{\tensor n}$ to $A$. 
    A \emph{$\Omega$-algebra} or a \emph{universal algebra with signature $\Omega$} is a vector space $A$ together with a operation $\omega^A$ for each $\omega \in \Omega$, such that $\omega^A$ is $n$-ary if $\omega \in \Omega_n$.
\end{defi}

We note that $0$-ary operations can be interpreted as constants in the $\Omega$-algebra $A$, since a linear map $\omega^A\from \FF \to A$ is determined by $\omega^A(1)$.

\begin{defi}
    Let $A$ and $B$ be $\Omega$-algebras. 
    A \emph{homomorphism} $\psi\from A \to B$ is a linear map such that for every $\omega \in \Omega_n$ we have that
    \[
        \psi( \omega^A (a_1 \tensor \cdots \tensor a_n) ) = \omega^B ( \psi(a_1) \tensor \cdots \tensor \psi(a_n) ),
    \]
    for all $a_1, \ldots, a_n \in A$. 
    As usual, an \emph{automorphism} of $A$ is a bijective homomorphism from $A$ to $A$ and the group of automorphisms is denoted by $\Aut(A)$.
\end{defi}

When dealing with a fixed $\Omega$-algebra, we will usually identify the signature $\Omega$ with the set of all operations.

\begin{ex}\label{ex:omega-vec-space}
    A vector space is a $\emptyset$-algebra.
\end{ex}

\begin{ex}\label{ex:omega-algebra}
    A algebra in the usual sense, with product $*$, is a $\Omega$-algebra with $\Omega = \Omega_2 = \{ * \}$. 
\end{ex}

\begin{ex}
    A algebra $A$ with unity $\mathds{1} \in A$ and product $*$ is a  $\Omega$-algebra with $\Omega = \Omega_0 \cup \Omega_2$ where $\Omega_2 = \{ * \}$ and $\Omega_0 = \{ \omega_0 \}$, where $\omega_0\from \FF \to A$ is defined by $\omega_0 (1) = \mathds 1$.
\end{ex}

% \begin{ex}
%     A superalgebra $A = A\even \oplus A\odd$, with product $*$, is a $\Omega$-algebra where $\Omega = \Omega_1 \cup \Omega_2$, with $\Omega_2 = \{ * \}$ and $\Omega_1 = \{ \pi_{\bar 0}, \pi_{\bar 1} \}$ where $\pi_{\bar 0}, \pi_{\bar 1}\from A \to A$ are the projection on the $A\even$ and $A\odd$, respectively. 
%     % Conversely, if $A$ is a $\Omega$-algebra with the signature above, it is a superalgebra if, and only if,
%     % \begin{enumerate}[(i)]
%     %     \item $\pi_{\bar 0}^2 = \pi_{\bar 0}$ and $\pi_{\bar 1}^2 = \pi_{\bar 1}$;
%     %     \item $\pi_{\bar 0} + \pi_{\bar 1} = \id$; \label{item:sum-projections}
%     %     \item $\pi_{\bar 1}\pi_{\bar 0} = \pi_{\bar 0}\pi_{\bar 1} = 0$;
%     % \end{enumerate}
% \end{ex}

\begin{ex}\label{ex:omega-graded-algebra}
    A $G$-graded algebra $A = \bigoplus_{g\in G}$, with operation $*$, is a $\Omega$-algebra with $\Omega = \Omega_1 \cup \Omega_2$, with $\Omega_2 = \{ * \}$ and $\Omega_1 = \{ \pi_g \mid g\in G\}$, where $\pi_g\from A \to A$ is the projection in the component $A_g$. 
    As a particular case, superalgebras can be viewed as $\Omega$-algebras.
\end{ex}

\begin{ex}
    A superalgebra with super-anti-automorphism $(A, \vphi)$, with product $*$, a $\Omega$-algebra with $\Omega = \Omega_1 \cup \Omega_2$ where $\Omega_2 = \{ * \}$ and $\Omega_1 = \{ \vphi, \pi_{\bar 0}, \pi_{\bar 1} \}$. 
    As a particular case, algebras with antiautomorphism can be viewed as $\Omega$-algebras.
\end{ex}

It is straightforward to check that, in each one of examples above, the usual notion of homomorphism corresponds to the notion of homomorphism as $\Omega$-algebras.

It should be noted that among these examples, only in Examples \ref{ex:omega-vec-space} and \ref{ex:omega-algebra} the correspondence is bijective. 
In the other cases, the original structures correspond to proper subclasses of the $\Omega$-algebras described, which can be axiomitized. 
We will present axioms for Example \ref{ex:omega-graded-algebra}, since gradings are not usually describe in terms of the projection maps, we will

\begin{itemize}
    \item[\done] paragraph-remark stating only exs 1 and 2 are ``precisely'', the other are proper inclusions.
    \item discuss, in a remark, the axioms of graded algebras, noting that one of them is not first-order if the group is not finite.
\end{itemize}

\section{$G$-gradings on $\Omega$-algebras and $\widehat G$-actions}

We will now consider gradings on universal algebras with arbitrary signature (which should not be confused with Example \ref{ex:omega-graded-algebra}, which realizes graded algebras as a specific type of universal algebra).

\begin{defi}
    A $G$-grading on a $\Omega$-algebra $A$ is a $G$-grading on its vector space underlying such that, if we consider the usual grading on the tensor powers $A^{\tensor n}$, all the operations $\omega^A$ for $\omega \in \Omega$ are degree preserving.
\end{defi}

It is straight forward to verify that this notion of grading corresponds to the usual notion of gradings on algebras, superalgebras, algebras with antiautomorphisms and superalgebras with super-anti-automorphisms.

We will now proceed to generalize the correspondence between $G$-gradings and $\widehat G$-actions (see ?? and ??) to $\Omega$-algebras. 
We start focusing on graded vector spaces.

\begin{defi}
    Let $V$ be a vector space. 
    Given a $G$-grading $\Gamma\from V= \bigoplus_{g\in G} V_g$ on $V$, we define a $\widehat G$-action in the following way: for $\chi \in \widehat G$ and $a_g \in V_g$, $\chi \cdot v_g = \chi(g)v_g$, and we extend the it by linearity. 
    We will denote the corresponding representation by $\eta_\Gamma\from \widehat G \to \GL(V)$.
\end{defi}

It is easy to see that this really defines an action. 

For this action to capture more information about the grading, we need some assumptions on the field $\FF$. 
From now on, let us assume $\FF$ is algebraically closed and $\Char \FF = 0$.

\begin{lemma}
    Let $V$ be a $G$-graded vector space. 
    A element $v\in V$ is homogeneous of degree $g\in G$ if, and only if, $\chi\cdot v = \chi(g)v$ for all $\chi \in \widehat G$.
\end{lemma}

\begin{proof}
    The ``only if'' direction is the definition of the action. 
    For the other direction, let $v\in V$ and write $v = \sum_{g\in G} v_g$, where $v_g \in V_g$.
\end{proof}

\begin{prop}
    Let $A$ be a $\Omega$-algebra and $\Gamma$ be a grading on its underlying vector space. 
    Then $\Gamma$ is a grading on $A$ if, and only if, $\eta_\Gamma(\widehat G) \subseteq \Aut(A)$.
\end{prop}

\begin{proof}
    First, assume $\Gamma$ is a grading on the $\Omega$-algebra $A$.
    Let $\chi \in \widehat G$ and let $\psi \coloneqq \eta_\Gamma(\chi)$. 
    We already know that $\psi$ is bijective, it only remains to prove it is a homomorphisms. 
    Let $\omega \in \Omega_n$ and let $a_1, \ldots, a_n \in A$ be homogeneous elements of degrees $g_1, \ldots, g_n \in G$, respectively.
    
    Then $a_1\tensor \cdots \tensor a_n \in A^{\tensor n}$ has degree $g_1 \cdots g_n$. Hence
    \begin{align*}
        \psi(\omega^A(a_1\tensor \cdots \tensor a_n)) &= \chi(g_1 \cdots g_n) \omega^A(a_1\tensor \cdots \tensor a_n)\\
        &=\chi(g_1) \cdots \chi(g_n) \omega^A(a_1\tensor \cdots \tensor a_n)\\
        &= \omega^A(\chi(g_1)a_1\tensor \cdots \tensor \chi(g_n)a_n)\\
        &= \omega^A(\psi(a_1)\tensor \cdots \tensor \psi(a_n)),
    \end{align*}
    so $\psi$ is a homomorphism.
\end{proof}


% \begin{remark}
%     Axioms of graded algebra?
% \end{remark}

\section{Old - Correspondence between $G$-gradings and\\ $\widehat G$-actions}\label{ssec:G-hat-action}

One of the important tools for dealing with gradings by abelian groups on (super)algebras is the well-known correspondence between  $G$-gradings and $\widehat G$-actions (see e.g. \cite[\S 1.4]{livromicha}), where $\widehat G$ is the algebraic group of characters of $G$, \ie, group homomorphisms $G \rightarrow \FF^{\times}$. The group $\widehat{G}$ acts on any $G$-graded (super)algebra $A = \bigoplus_{g\in G} A_g$ by $\chi \cdot a = \chi(g) a$ for all $a\in A_g$ (extended to arbitrary $a\in A$ by linearity). The map given by the action of a character $\chi \in \widehat{G}$ is an automorphism of $A$. If $\FF$ is algebraically closed and $\Char \FF = 0$, then $A_g = \{ a\in A \mid \chi \cdot a = \chi (g) a\}$, so the grading can be recovered from the action.

For example, if $A=A\even \oplus A\odd$ is a superalgebra, the action of the nontrivial character of $\ZZ_2$ yields the \emph{parity automorphism} $\upsilon$, which acts as the identity on $A\even$ and as the negative identity on $A\odd$. If $A$ is a $\ZZ$-graded algebra, we get a representation $\widehat \ZZ = \FF^\times \rightarrow \Aut (A)$ given by $\lambda \mapsto \upsilon_\lambda$ where $\upsilon_{\lambda} (x) = \lambda^i x$ for all $x\in A^i$, $i\in \ZZ$.

A grading on a (super)algebra over an algebraically closed field of characteristic $0$ is said to be \emph{inner} if it corresponds to an action by inner automorphisms. For example, the inner gradings on $\Sl(n)$ (also known as Type I gradings) are precisely the restrictions of gradings on the associative algebra $M_n(\FF)$.
