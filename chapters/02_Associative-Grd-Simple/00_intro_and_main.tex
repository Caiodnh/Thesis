\chapter{Graded-Simple Associative Superalgebras}\label{chap:grd-simple-assc}

In this chapter we review some results about graded-simple associative algebras and extend them to superalgebras. 
Throughout this chapter, $G$ is a fixed group. 

In \cref{sec:gradings-on-matrix-algebras} we recall the theory of graded-simple algebras satisfying the descending chain condition (d.c.c.) on graded left ideals (\cref{def:dcc}), following closely \cite[Chapter 2]{livromicha}, although with some differences in notation (see \cref{defi:gamma-realizes-kappa,rmk:kappa-as-parameter,defi:tuple-governed}) and proofs. 
In \cref{ssec:D-modules}, we consider graded-division algebras (\cref{def:graded-division-algebra}) and classify their graded modules of finite rank. 
In \cref{ssec:Grd-Wedderburn-Theory}, we reduce the classification of graded-simple algebras satisfying \dcc on graded left ideals to these objects:  \cref{thm:End-over-D} states that every such graded algebra is isomorphic to $\End_\D (\U)$, where $\D$ is a graded-division algebra and $\U$ is a graded right $\D$-module of finite rank, while \cref{thm:iso-abstract} describes isomorphisms between these endomorphism algebras. 
In \cref{ssec:grd-div-alg,ssec:param-End_D-U}, we specialize to finite dimensional algebras over an algebraically closed field. 
In the former we parametrize the graded-division algebras (\cref{prop:T-beta-determines-iso})
% , characterize their simplicity (as algebras) in terms of these parameters (\cref{cor:D-simple-iff-beta-nondeg})
and give the construction of the so called standard realizations (\cref{def:standard-realization}), which are models of graded-division algebras that are simple as algebras; in the latter, we extend the parametrization to graded-simple algebras to get a classification result (\cref{thm:iso-End_D-U-with-parameters}), and use it to recover the well-known classifications of gradings on matrix algebras (\cref{cor:grds-matrix-alg}) and of simple associative superalgebras (\cref{thm:fd-simple-SA}). 
% In the case $G = \ZZ_2$, we deduce from this theory the well-known result that simple finite dimensional associative superalgebras are either isomorphic to $M(m,n)$ or to $Q(n)$

In \cref{sec:grd-simple-salg}, we extend this theory to graded-simple superalgebras, by seeing them as $G^\#$-graded algebras, where $G^\# = G \times \ZZ_2$ (see \cref{sec:grds-and-sa}). 
The subsections follow the same pattern as in \cref{sec:gradings-on-matrix-algebras}, adapting the definitions and results to the ``super'' setting. 
In \cref{ssec:supermodules-over-D} we define graded-division superalgebras (\cref{def:grd-div-sa}), which can be even or odd (\cref{defi:even-odd-D}), and parametrize their graded supermodules of finite rank in each case. 
In \cref{ssec:wedderburn-super}, we apply \cref{thm:End-over-D} to graded-simple superalgebras satisfying \dcc on graded left ideals; these graded superalgebras can also be even or odd depending on the graded-division superalgebra involved (\cref{defi:even-odd-R}). 
In \cref{ssec:T-beta-p}, we parametrize finite dimensional graded-division superalgebras over an algebraically closed field and give standard realizations of type $Q$ (\cref{def:standard-realization-Q}), which are graded-division superalgebras that are simple as superalgebras but not as algebras. 
In \cref{ssec:classification-assc-super}, we extend this parametrization to classify even and odd graded-simple superalgebras (\cref{thm:iso-D-even,thm:iso-D-odd}), and obtain as corollaries the classification of gradings on $M(m,n)$ (\cref{cor:iso-M-even,cor:iso-M-odd}) and on $Q(n)$ (\cref{cor:iso-Q}).

The parameters used to classify even gradings in \cref{sec:grd-simple-salg} are in terms of the group $G$, but the parameters used for odd gradings are in terms of the group $G^\#$ (except in the case the underlying superalgebra is $Q(n)$). 
In \cref{sec:assc-only-G}, we present a parametrization of odd gradings in terms of $G$ (see \cref{def:odd-parameters}). 
\cref{ssec:odd-div-G-only} handles the general case (\cref{cor:classification-odd-general}), while \cref{ssec:grds-M(m-n)-only-G} handles the case where the underlying superalgebra is $M(m,n)$, giving necessary and sufficient conditions on the parameters for this to be the case (\cref{lemma:motivation-O_M,prop:O_M-non-empty}), and also simplifying the parametrization (\cref{cor:iso-odd-M-simplified}). 

We note that in \cite{BS}, a description of gradings on $M(m,n)$ and $Q(n)$ was given, but the isomorphism problem was not considered. 
A classification of gradings on $M(m,n)$ was obtained in \cite{paper-MAP} (see also \cite{Helens_thesis}). 


% In this chapter, we will review some basic results about graded associative algebras graded by a group $G$.
% First we will recall the structure of graded-simple associative algebras with a descending chain condition, following \cite{livromicha} (\cref{sec:gradings-on-matrix-algebras}), and then its consequences for finite dimensional $G$-graded graded-simple associative superalgebras, following  (\cref{sec:grd-simple-salg}). 
% For the main classification results we will assume that the ground field $\FF$ is algebraically closed and the grading group $G$ is abelian. 
% Nevertheless, many results will be presented for arbitrary $\FF$ and $G$. 
% Gradings will be classified for the matrix superalgebras $M(m,n)$ and the so called queer superalgebras $Q(n)$. 
% In the latter case, we will compare our classification with the one obtained in \cite{paper-Qn} using a different method. 

% As in \cite{paper-MAP}, we present two ways to parametrize gradings on the superalgebras above: one in terms of the group $G$ and other in terms of the auxiliary group $G^\# = G\times \ZZ_2$ (see ?? in the introduction). 
% As we will see in \cref{defi:even-odd-D,defi:even-odd-R}, we separate our gradings into two classes: even and odd. 
% The odd gradings are harder to describe in terms of $G$, and this is achieved in \cref{sec:assc-only-G}. 
% % Nevertheless, an easier approach can be used for odd gradings on $Q(n)$ (see ).

% Cite: Bahturin and Shestakov \cite{BS}. 
% It is cited in \cref{sec:grd-simple-salg}. 

% ** Sections:


\section{Graded-simple associative algebras}\label{sec:gradings-on-matrix-algebras}

In this section we will recall the classification of gradings on matrix algebras \cite{BSZ01, BZ02, BK10}. 
We will follow the exposition of \cite[Chapter 2]{livromicha} but use slightly different notation, which will be extended to superalgebras in \cref{sec:grd-simple-salg}.

Our main interest is in finite dimensional graded algebras. 
It is clear that they satisfy the following condition:

\begin{defi}
    We say that a graded algebra $R$ satisfies the \emph{descending chain condition} (or \emph{\dcc}) on graded left ideals if, for every sequence $\{I_k\}_{k\in \NN}$ of graded left ideals such that \[k \leq \ell \implies I_k \supseteq I_\ell,\] there is $n\in \NN$ such that \[n \leq \ell \implies I_n = I_\ell.\]
\end{defi}

As we will see in \cref{thm:End-over-D}, a graded-simple algebra satisfying this condition can be described, up to isomorphism, by a graded-division algebra $\D$ and a graded right $\D$-module of finite rank. 
This is the graded analog of the classical result of Wedderburn.

\subsection{Graded-division algebras and their modules}\label{ssec:D-modules}

It is easy to see that if a $G$-graded algebra $R$ has the unit element $1$, then $1 \in R_e$. 
Also, if an element $r\in R_g$ is invertible, then $r\inv \in R_{g\inv}$.

\begin{defi}
    A \emph{graded-division algebra} is a unital graded algebra $\D$ such that every nonzero homogeneous element has an inverse. 
    In this case, we may also refer to the $G$-grading on the algebra $\D$ as a \emph{division grading}.
\end{defi}

\begin{ex}\label{ex:group-algebra}
    The group algebra $\FF G$ can be regarded as a graded algebra if we declare $\FF g$ to be the homogeneous component of degree $g$, for all $g\in G$. 
    It is a graded-division algebra since $(\lambda g)\inv = \frac1\lambda g\inv$, for all $0 \neq \lambda \in \FF$.
\end{ex}

\begin{ex}\label{ex:Pauli-2x2}
    Another example of a graded-division algebra is the matrix algebra $M_2(\FF)$, $\Char \FF \neq 2$, equipped with the $\ZZ_2 \times \ZZ_2$-grading, sometimes called \emph{Pauli grading}, defined by
\begin{align*}
	\deg \begin{pmatrix}
		\phantom{.}1 & \phantom{-}0\phantom{.} \\
		\phantom{.}0 & \phantom{-}1\phantom{.}
	\end{pmatrix} = (\bar 0, \bar 0),\quad & \deg \begin{pmatrix}
		\phantom{.}0 & \phantom{-}1\phantom{.} \\
		\phantom{.}1 & \phantom{-}0\phantom{.}
	\end{pmatrix} = (\bar 0, \bar 1), \\
	\deg \begin{pmatrix}
		\phantom{.}1 & \phantom{-}0\phantom{.} \\
		\phantom{.}0 & -1\phantom{.}
	\end{pmatrix} = (\bar 1, \bar 0),\quad &
	\deg \begin{pmatrix}
		\phantom{.}0 & -1\phantom{.}           \\
		\phantom{.}1 & \phantom{-}0\phantom{.}
	\end{pmatrix} = (\bar 1, \bar 1).
\end{align*}
\end{ex}

\begin{ex}\label{ex:Pauli-ell-x-ell}
    \Cref{ex:Pauli-2x2} can be generalized to define a $\ZZ_\ell \times \ZZ_\ell$-grading on $M_\ell(\FF)$,  given that there exists a primitive $\ell^{\text{th}}$-root of unity $\xi \in \FF$. 
    Let
    \[
        A \coloneqq \begin{pmatrix}
		1 & 0 & 0 & \cdots &0\\
		0 & \xi& 0 & \cdots &0\\
		0 & 0 & \xi^2 &\cdots &0\\
		\vdots & \vdots & \vdots & \ddots  & \vdots\\
		%0 & 0 & 0 & \cdots & \xi^{\ell-2}& 0\\
		0 & 0 & 0 & \cdots & \xi^{\ell-1}\\
	\end{pmatrix}
        %
        \quad \text{and} \quad
        %
        B \coloneqq \begin{pmatrix}
		0 & 0 & 0 & \cdots & 0 & 1\\
		1 & 0& 0 & \cdots & 0 & 0\\
		0 & 1 & 0 &\cdots & 0 & 0\\
		\vdots & \vdots & \vdots & \ddots & \vdots & \vdots\\
		%0 & 0 & 0 & \cdots & 0 & 0\\
		0 & 0 & 0 & \cdots & 1 & 0\\
	\end{pmatrix}.
    \] 
    Note that $AB = \xi BA$ and $A^\ell = B^\ell = 1$. 
    One can check that setting $\deg A = (\bar 1, \bar 0)$ and $\deg B = (\bar 0, \bar 1)$ defines a grading making $M_\ell (\FF)$ a graded-division algebra.
\end{ex}

For what follows, let us fix a graded-division algebra $\D$.
Observe that $T \coloneqq \supp \D \subseteq G$ is a subgroup of $G$. 
Also, it is clear that $\D_e$ is a division algebra in the usual sense. 

Graded $\D$-modules will play an important role in this work. 
We will now recall their classification up to isomorphism. 

Consider a graded right $\D$-module $\U = \bigoplus_{g\in G} \U_g$. 
Note that a homogeneous component $0\neq \U_g$ is a $\D_e$-module but, unless $\D = \D_e$, it is not a $\D$-submodule.
% , since if $0\neq u \in \U_g$ and $0\neq d \in \D_t$, then $0\neq ud \in \U_{gt}$. 
It is easy to see that the $\D$-span of $\U_g$ is $\U_{gT} \coloneqq \bigoplus_{t\in T} \U_{gt}$. 
% This leads us to the following:

\begin{defi}
    Given a left coset $x\in G/T$, the \emph{isotypic component} $\U_x$ of a graded right $\D$-module $\U = \bigoplus_{g\in G} \U_g$ is the $\D$-submodule given by:
    \[
        \U_{x} \coloneqq \bigoplus_{g\in x} \U_{g}.
    \]
    Clearly, $\U = \bigoplus_{x\in G/T} \U_x$.
\end{defi}

% Clearly, $\U_{gT}$ is a $\D$-submodule of $\U$, for every $gT \in G/T$.

\begin{remark}
    The use of the terminology ``isotypic component'' here is consistent with its common use. 
    Recall that, in representation theory, an isotypic component is defined as the sum of all simple submodules of a given isomorphism type. 
    Since $\D$ is a graded-division algebra, the right modules ${}^{[g]}\D$ are simple $\D$-modules. 
    It is easy to see that all simple $\D$-modules are of this form.
    Indeed, if $\V$ is any graded right $\D$-module and $0\neq v \in \V_g$, then the map ${}^{[g]}\D \to v\D$ given by $d \mapsto vd$ is an isomorphism. 
    It is also easy to see when ${}^{[g]}\D$ and ${}^{[g']}\D$ are isomorphic. 
    Of course, since $\supp {}^{[g]}\D = gT$ and $\supp {}^{[g']}\D = g'T$, a necessary condition for this to happen is $g'g\inv \in T$. 
    Conversely, if $g'g\inv \in T$, then we can pick $0\neq c \in \D_{g'g}$ and define an isomorphism ${}^{[g]}\D \to {}^{[g']}\D$ by $d \mapsto cd$.
\end{remark}

\begin{lemma}\label{lemma:iso-D-modules}
    Two graded right $\D$-modules $\U$ and $\V$ are isomorphic if, and only if, the isotypic component $\U_x$ is isomorphic to $\V_x$ for all $x \in G/T$. 
\end{lemma}

\begin{proof}
    If $\psi\from \U \to \V$ is a homomorphism of graded right $\D$-modules, then it is clear that $\psi (\U_x) \subseteq \V_x$, for all $x \in G/T$, and that $\psi$ is determined by the maps $\psi\restriction_{\U_x} \from \U_x \to \V_x$, $x \in G/T$. 
    Moreover, $\psi$ is an isomorphism if, and only if, each $\psi\restriction_x$ is an isomorphism.
\end{proof}

\Cref{lemma:iso-D-modules} reduces the problem of classifying graded $\D$-modules up to isomorphism to classifying their isotypic components. 
We will now reduce the latter to (ungraded) modules over the division algebra $\D_e$. 
Note that $\D$ can be regarded as a left $\D_e$-module.

\begin{defi}\label{def:D_e-form}
    Let $\U$ be a graded $\D$-module. 
    A \emph{$\D_e$-form} of $\U$ is a graded $\D_e$-submodule $\tilde \U \subsetneq \U$ such that the map $\tilde \U \tensor_{\D_e} \D \to \U$ given by $u\tensor d \mapsto ud$ is an isomorphism of graded right $\D$-modules. 
    If this is the case, we will use this map to identify $\tilde \U \tensor_{\D_e} \D$ with $\U$. 
\end{defi}

\begin{prop}\label{prop:U_g-is-D_e-form}
    Let $\U$ be a graded right $\D$-module and let $g\in G$. 
    The homogeneous component $\U_g$ is a $\D_e$-form for the isotypic component $\U_{gT}$.
\end{prop}

\begin{proof}
    Since the $\D$-span of $\U_g$ is $\U_{gT}$, the map $\psi\from \U_g \tensor_{\D_e} \D \to \U$ given by $\psi (u\tensor d) = ud$ is surjective. 
    To see that $\psi$ is injective, pick $0 \neq X_t \in \D_t$ for every $t\in T$. 
    Let $u \in \U_g \tensor_{\D_e} \D$. 
    It is easy to see that $\{ X_t\}_{t\in T}$ is a $\D_e$-basis for $\D$, so we can write $u = \sum_{t\in T} u^t \tensor X_t$, with $u^t \in \U_g$ for all $t\in T$ and $u^t = 0$ for all but finitely many $t\in T$. 
    We then have that $\psi (u) = \sum_{t\in T} u^t X_t$ and
    hence, if $\psi (u) = 0$, we have $u^t X_t = 0$, for every $t \in T$. 
    Since $X_t$ is invertible, we conclude that $u^t = 0$ for every $t\in T$ and, therefore, $u = 0$.
\end{proof}

\begin{cor}\label{cor:iso-isotypic-components}
    Let $\U$ and $\V$ be graded right $\D$-modules and fix $g\in G$. 
    Then $\U_{gT}$ is isomorphic to $\V_{gT}$ if, and only if, $\U_g$ and $\V_g$ are isomorphic as $\D_e$-modules. \qed
\end{cor}

% \begin{proof}
%     Let $\vphi\from \U_g \to \V_g$ be an isomorphism of $\D_e$-modules. 
%     Clearly we can extend it to an isomorphism $\tilde\vphi \from \U_g\tensor_{\D_e} \D \to \V_g\tensor_{\D_e} \D$ by $ $
% \end{proof}

By \cref{prop:U_g-is-D_e-form}, a $\D_e$-basis for $\U_g$ is a $\D$-basis for $\U_{gT}$. 
Since $\D_e$ is a division algebra, we conclude that every isotypic component has a $\D$-basis consisting of elements of the same degree. 
Since $\U$ is the direct sum of its isotypic components, we conclude that $\U$ has a graded basis:

\begin{defi}
    Let $\U$ be a graded right $\D$-module. 
    A $\D$-basis $\mc B$ of $\U$ is said to be a \emph{graded basis of $\U$} if all the elements in $\mc B$ are homogeneous (of various degrees).
\end{defi}

\begin{remark}
    Alternatively, the existence of a graded basis can be proved with the same arguments as in the ungraded setting (using Zorn's Lemma).
\end{remark}

\begin{prop}\label{prop:dim-U_x}
    Let $\U$ be a graded right $\D$-module. 
    Let $\mc B$ be a graded basis and set $\B_x \coloneqq B \cap B_x$ for any $x\in G/T$. 
    Then the sets $B_x$ form a partition of $\B$ and the cardinality $|B_x|$ is independent of the choice of $\B$ and equal to $\dim_{\D_e} \U_g$ for any $g\in x$. 
\end{prop}

\begin{proof}
    Let $\B$ be any graded basis of $\U$ and set $\B_x \coloneqq \B \cap \U_x$ for all $x\in G/T$. 
    Since every element in $\B$ is homogeneous, $\B = \bigcup_{x \in G/T} \B_x$, and, hence, $\B_x$ is a graded basis for $\U_x$, for all $x\in G/T$.
    
    We claim that each $\B_x$ has cardinality $\dim_{\D_e} \U_g$, where $g$ is an arbitrary representative of the coset $x$. 
    Indeed, let $\B_x = \{ u_\lambda\}_{\lambda \in \Lambda_x}$
    For every $\lambda \in \Lambda_x$, choose $d_\lambda \in \D$ to be a nonzero element of degree $(\deg u_\lambda)\inv g \in T$. 
    Then, clearly $\B_x' = \{u_\lambda d_\lambda \}_{\lambda \in \Lambda_x}$ is also a graded basis of $\U_{gT}$, of the same cardinality as $\B_x$, but with all elements having degree $g$. 
    It is easy to see that $\B_x'$ is a $\D_e$-basis of $\U_g$ and, therefore, it has cardinality $\dim_{\D_e} \U_g$. 
\end{proof}

\begin{defi}
    We define the \emph{rank} or \emph{$\D$-dimension} of a graded right module $\U$ to be the cardinality of one (and hence any) graded basis of $\U$, and denote it by $\dim_\D (\U)$.
\end{defi}

We will now restrict ourselves to graded right $\D$-modules of finite rank.
% and study their endomorphism algebras. 

It is clear that a graded right $\D$-module $\U$ has finite rank if, and only if, all isotypic components have finite rank and only finitely many of them are nonzero. 
In view of \cref{lemma:iso-D-modules,cor:iso-isotypic-components,prop:dim-U_x}, this means that an isomorphism class of such modules is determined by the map $\kappa \from G/T \to \ZZ_{\geq 0}$ defined by $\kappa(x) \coloneqq \dim_\D (\U_x)$, which has finite support. 
In other words, $\kappa$ is a finite \emph{multiset} in $G/T$, where $\kappa(x)$ is viewed as the multiplicity of the point $x$. 
As it is usually dine for multisets, we define $|\kappa| \coloneqq \sum_{x \in G/T} \kappa(x)$. 
Clearly $|\kappa| = \dim_\D (\U)$.

Given an arbitrary map $\kappa\from G/T \to \ZZ_{\geq 0}$ with finite support, we can construct a representative $\U$ for the corresponding isomorphism class of graded $\D$-modules. 
To do so, we set $k = |\kappa|$ and choose a $k$-tuple $\gamma = (g_1, \ldots, g_k)$ such that the number of entries $g_i$ with $g_i\in x$ is equal to $\kappa (x)$ for every $x\in G/T$.
Then we put $\U \coloneqq {}^{[g_1]}\D \oplus \cdots \oplus {}^{[g_k]}\D$.

\begin{remark}
	Both parameters, the multiset and the the tuple, have been used in previous works. 
	Sometimes both are present, as in \cite{livromicha} (where the tuple is defined so that all elements belong to different left cosets and the multiplicities are recorded as recorded as a separate tuple) and \cite{paper-Qn,paper-MAP}; sometimes only the tuple, as in \cite{BK10}; sometimes only the multiset, as in \cite{paper-adrian,felipe-misha}.
	Here we follow the notation of the latter.
\end{remark}

% \begin{notation}
    Let $\U$ and $\V$ be graded right $\D$-modules. We will denote by $\Hom_\D(\U,\V)$ the vector space of all homomorphisms from $\U$ to $\V$ as $\D$-modules, not as graded $\D$-modules. 
    In other words, the elements of $\Hom_\D (\U, \V)$ do not necessarily preserve degrees. 
    As usual, we also denote $\Hom_\D (\U,\U)$ by  $\End_\D(\U)$. 
% \end{notation}

Of course, $\Hom_\D(\U,\V) \subseteq \Hom_\FF (\U, \V)$. 
If $\dim_\D (\U) < \infty$, we can say more: 

\begin{prop}\label{prop:Hom_D-is-graded}
    Let $\U$ and $\V$ be graded right $\D$-modules and suppose $\dim_\D \U < \infty$. 
    Then $\Hom_\D(\U,\V) = \Hom_\D^{\mathrm{gr}} (\U, \V)$ is a graded subspace of $\Hom_\FF^{\mathrm{gr}} (\U, \V)$ (see \cref{defi:elementary-grd-abstract}).
\end{prop}

\begin{proof}
    Let $\B = \{ u_1, \ldots, u_l \}$ be graded basis for $\U$ and $\mc C = \{v_i\}_{i\in I}$ be graded basis for $\V$, set $g_j \coloneqq \deg u_j$ and $h_i \coloneqq \deg v_i$, for $1\leq j \leq l$ and $i\in I$. 
    
    Since $\B$ is, in particular, a basis for $\D$ as a free $\D$-module, a $\D$-linear map $f\from \U \to \U$ can be defined by its values on the elements of $\B$. 
    Given $1\leq j \leq l$, $i\in I$, $t\in T$ and $0\neq d \in \D_t$, define $f_{i, j, d}\from \U \to \V$ to be the $\D$-linear map defined by $f_{i, j, d}(u_r) = \delta_{jr} v_i d$, for every $r \in \{1, \ldots, l \}$. 
    It is easy to see that $f_{i, j, d}\from \U \to \V$ is a homogeneous element in $\Hom_\FF^{\mathrm{gr}} (\U, \V)$ with
    \[\label{eq:grd-M_k(D)-nonabelian}
        \deg f_{i,j,d} = h_i t g_j\inv.
    \] 
    Since $\B$ is finite, it is clear that every map in $\Hom_\D (\U, \V)$ is a finite sum of maps of the form $f_{i, j, d}$, concluding the proof. 
\end{proof}

Under the conditions of \cref{prop:Hom_D-is-graded}, we will always consider $\Hom_\D (\U, \V)$ as a graded vector space with the grading restricted from $\Hom_\FF^{\mathrm{gr}} (\U, \V)$. 
Note that, in the case $\U = \V$, this makes $\End_\D (\U)$ a graded algebra and $\U$ a graded left $\End_\D (\U)$-module. 
% Therefore, $\U$ is $(\End_\D (\U), \D)$-bimodule.

% It is clear that the grading on $\End_\D (\U) = \Hom_\D(\U, \U)$ (defined in \cref{ssec:D-modules}, by means of \cref{prop:Hom_D-is-graded}) makes it a graded algebra
% We will follow the convention of writing $\D$-linear maps on the left and $R$-linear maps on the right. 

As usual, if both $\U$ and $\V$ have finite rank over $\D$, we can represent the elements of $\Hom_\D(\U, \V)$ as matrices with entries in $\D$. 
More precisely, given graded bases $\B = \{ u_1, \ldots, u_l \}$ and $\mc C = \{v_1, \ldots, v_k\}$ for $\U$ and $\V$, respectively, we have an isomorphism of vector spaces $\Hom_\D(\U, \V) \to M_{k\times \ell}(\D)$ given by $f\mapsto (d_{ij})$, where $f(u_j) = \sum_i v_i d_{ij}$. 
Under this isomorphism, the map $f_{i, j, d}$ defined in the proof of \cref{prop:Hom_D-is-graded}, corresponds to the matrix $E_{ij}(d) \in M_{k\times \ell}(\D)$, \ie, the matrix which has $d$ in the position $(i,j)$ and $0$ elsewhere. 

\begin{remark}\label{rmk:M_k(F)-tensor-D}
    Note that we can identify $M_{k \times \ell}(\D) = M_{k \times \ell}(\FF) \tensor_\FF \D$,     with $E_{ij}(d)$ corresponding to $E_{ij}\tensor d$. 
    In the case $G$ is abelian, \cref{eq:grd-M_k(D)-nonabelian} implies that the grading on $M_{k\times \ell}(\FF) \tensor_\FF \D$ is the one of the tensor product of graded algebras (see ??), where $M_{k\times \ell}(\FF)$ has an elementary grading (see \cref{defi:elementary-grd-matrix}).
\end{remark}

% As usual, if both $\U$ and $\V$ have finite rank over $\D$, we can represent the elements of $\Hom_\D(\U, \V)$ as matrices with entries in $\D$. 
% More precisely, given graded bases $\B = \{ u_1, \ldots, u_k \}$ and $\mc C = \{v_1, \ldots, v_k\}$ for $\U$ and $\V$, respectively, 

% Under the usual isomorphism $\End_\D(\U) \to M_k(\D)$ defined by the basis $\B$ (\ie, $f\mapsto (d_{ij})$ where $f(u_j) = \sum_i u_{ij}d_{ij}$), the map $f_{i,j,d}$ goes to the matrix $E_{ij} (d)$, \ie, the matrix which has $d$ in the position $(i,j)$ and $0$ elsewhere. 

\subsection{Graded Wedderburn theory}\label{ssec:Grd-Wedderburn-Theory}

Let $\U$ be a nonzero graded right $\D$-module of finite rank. 
In this subsection we will study the endomorphism algebra $\End_\D (\U)$. 
The main reason for this is Theorem \ref{thm:End-over-D}, below. 
Special cases of this result appeared in several works, \eg, \cite{BSZ01,MR2046303,BZ02}. 
Here we follow \cite[Theorem 2.6]{livromicha}, the converse of which also holds (see \cite[page 31]{livromicha}).

\begin{thm}\label{thm:End-over-D}
    %   Let $G$ be a group and let $R$ be a $G$-graded associative algebra. 
    % 	Then $R$ is graded-simple and satisfies the \dcc on graded left ideals if, and only if,	there is a $G$-graded division algebra $\D$ and a graded right $\D$-module $\mc{U}$ of finite rank such that $R \simeq \End_{\D} (\mc{U})$ as graded algebras. \qed
	%
	Let $G$ be a group and let $R$ be a $G$-graded graded-simple associative algebra satisfying the \dcc on graded left ideals. 
	Then a $G$-graded division algebra $\D$ and a nonzero graded right $\D$-module $\mc{U}$ of finite rank such that $R \simeq \End_{\D} (\mc{U})$ as graded algebras. \qed
\end{thm}

% with $\dim_\D (\U) = k < \infty$ and let $\B = \{u_1, \ldots, u_k\}$ be a graded basis for it, and let $g_i = \deg u_i$. 
% Since $\B$ is, in particular, a $\D$-basis of a free $\D$-module, a $\D$-linear map $f\from \U \to \U$ can be defined by its values on the elements of $\B$. 
% Hence, if we choose $u_i, u_j \in \B$ and $0\neq d \in \D_t$ for some $t\in T$, then we have a $\D$-linear map $f_{i, j, d}\from \U \to \U$ defined by $f(u_\ell) = \delta_{j\ell} u_i d$. 
% As an element of the algebra $\End^{\text{gr}}_\FF (\U)$, equipped with the elementary grading induced by $\U$ (see \cref{defi:elementary-grd-abstract}), it is clear that 
% \[\label{eq:grd-M_k(D)-nonabelian}
%     \deg f_{i,j,d} = g_i t g_j\inv.
% \] 
% It is also clear that every element of $\End_\D(\U)$ is a finite sum of maps of the form $f_{i,j,d}$, and, hence, we conclude that $\End_\D (\U)$ is a graded subalgebra of $\End_\FF^{\text{gr}} (\U)$. 
% From now on, we will consider $\End_\D (\U)$ as a graded algebra with the grading inherited from $\End_\FF^{\text{gr}} (\U)$. 
% Under the usual isomorphism $\End_\D(\U) \to M_k(\D)$ defined by the basis $\B$ (\ie, $f\mapsto (d_{ij})$ where $f(u_j) = \sum_i u_{ij}d_{ij}$), the map $f_{i,j,d}$ goes to the matrix $E_{ij} (d)$, \ie, the matrix which has $d$ in the position $(i,j)$ and $0$ elsewhere. 

% \begin{remark}
%     Note that we can identify $M_k(\D) = M_k(\FF) \tensor_\FF \D$ with $E_{ij}(d)$ corresponding to $E_{ij}\tensor \D$. 
%     In the case $G$ is abelian, \cref{eq:grd-M_k(D)-nonabelian} implies that the grading on $M_k(\FF) \tensor_\FF \D$ is the one of the tensor product of graded algebras (see ??), where $M_k(\FF)$ is given the elementary grading defined by the $k$-tuple $(g_1, \ldots, g_k)$ (see \cref{defi:elementary-grd-matrix}).
% \end{remark}

The next question is when two graded algebras $\End_\D (\U)$ and $\End_{\D'} (\U')$ are isomorphic. 
We will need the following two definitions:

\begin{defi}\label{def:inner-automorphism}
	Let $d\in \D$ be a nonzero homogeneous element.
	We define the \emph{inner automorphism} $\operatorname{Int}_d\from \D \to \D$ by $\operatorname{Int}_d (c) \coloneqq dcd\inv$, for all $c\in \D$.
\end{defi}

\begin{defi}\label{def:twist}
	Let $\psi_0\from \D' \to \D$ be a homomorphism of graded algebras and let $\U$ be a graded right $\D$-module.
	The \emph{module induced by $\psi$} is the graded right $\D'$-module $\U^{\psi_0}$ which is $\U$ as a vector space, but with $\D'$-action defined by $u\cdot d \coloneqq u\,\psi_0 (d)$, for all $u\in \U$ and $d\in \D'$.
	In the case $\psi_0\from \D \to \D$ is an automorphism, $\U^{\psi_0}$ is again a graded right $\D$-module, called the \emph{twist of $\U$ by $\psi_0$}.
\end{defi}

\begin{remark}\label{rmk:twist-does-not-change-set}
	Note that, if $\psi_0$ is surjective, a map $f\from \U^{\psi_0} \to \U^{\psi_0}$ is $\D'$-linear if, and only if, $f$ is $\D$-linear as a map $\U\to \U$. 
	In other words, $\End_{\D'} (\U^{\psi_0})$ is the same set as $\End_\D (\U)$. 
	Nevertheless, the matrix representation of $f$ can be different in each case. 
	To be more precise, note that a subset $\B \subseteq \U$ is a graded $\D'$-basis of $\U^{\psi_0}$ if, and only if, it is a graded $\D$-basis of $\U$. 
	Such a basis gives rise to isomorphisms $M_k(\D') \iso \End_{\D'} (\U^{\psi_0})$ and $M_k(\D) \iso \End_{\D} (\U)$, and it is easy to see that the matrix representing $f$ in $M_k(\D')$ is equal to the result of applying $\psi_0$ to every entry of the matrix representing $f$ in $M_k(\D)$.
\end{remark}

The next result is \cite[Theorem 2.10]{livromicha} with a slightly different notation:

\begin{thm}\label{thm:iso-abstract}
	Let $R \coloneqq \End_\D(\U)$ and $R' \coloneqq \End_{\D'}(\U')$, where $\D$ and $\D'$ are graded-division algebras, and $\U$ and $\U'$ are nonzero right graded modules of finite rank over $\D$ and $\D'$, respectively.
	Given an isomorphism $\psi\from R \to R'$, there is a triple $(g, \psi_0, \psi_1)$, where $g \in G$, $\psi_0\from {}^{[g\inv]}\D^{[g]} \to \D'$ is an isomorphism of graded algebras, $\psi_1\from \U^{[g]} \to (\U')^{\psi_0}$ is an isomorphism of graded right $\D$-modules, such that
	\begin{equation}\label{eq:def-iso-algebras}
		\forall r\in R, \quad \psi(r) = \psi_1 \circ r \circ \psi_1\inv.
	\end{equation}
	Conversely, given a triple $(g, \psi_0, \psi_1)$ as above, Equation \eqref{eq:def-iso-algebras} defines an isomorphism of graded algebras $\psi\from R \to R'$.
	Another triple $(g', \psi_0', \psi_1')$ defines the same isomorphism $\psi$ if, and only if, there are $t\in \supp \D'$ and $0 \neq d\in \D'_t$ such that $g'= gt$, $\psi_0' = \mathrm{Int}_{d\inv} \circ \psi_0$ and $\psi_1' (u) = \psi_1 (u) d$ for all $u \in \U$. \qed
\end{thm}
% \begin{defi}
%     Let $\alpha\from \D \to \D'$ be any map. 
%     Given $A\in \M_k(\D)$, we define $\alpha(A) \in M_k(\D')$ to be the matrix given by applying $\alpha$ in each entry of $A$.
% \end{defi}

For the remaining of this subsection, fix a graded-division algebra $\D$ and a nonzero graded right $\D$-module of finite rank $\U$, and set $R \coloneqq \End_{\D} (\U)$. 

First, we state here a well-known result about ungraded algebras, for future reference:

\begin{prop}\label{prop:R-simple-iff-D-simple}
	The algebra $R = \End_\D (\U)$ is simple if, and only if, the algebra $\D$ is simple. \qed
\end{prop}

% \begin{remark}\label{rmk:converse-density-thm}
%     If $\U \neq 0$ and using that it has a graded basis, it is easy to see that the $\D$-action on $\U$ gives rise to an isomorphism of graded algebras $\D \to \End_R(\U)$. 
% \end{remark}

The following result is \cite[Exercise 3 on page 60]{livromicha}.

% \begin{prop}
%     The $\D$-action on $\U$ gives rise to an isomorphism of graded algebras between $\D$ and $\End_R (\U)$. \qed
% \end{prop}

\begin{lemma}\label{lemma:converse-density-thm}
    The representation map $\rho\from \D \to \End_R (\U)$ corresponding to the right $\D$-action on $\U$
    % given by $u \rho(d) = ud$, for all $d\in \D$ and $u \in \U$, 
    is an isomorphism of graded algebras.
\end{lemma}

\begin{proof}
    By the definition of a graded module, $\rho$ is a homomorphism of graded algebras $\D \to \End_{R}^{\mathrm{gr}} (\U)$. 
    Since $\D$ is graded-simple and $\rho(1) = \id_{\U} \neq 0$, $\rho$ is injective.

    Let $\{u_1, \ldots, u_k \}$ be a graded basis of $\U$. 
    Given $u\in \U$, define $r\in R = \End_\D (\U)$ to be the $\D$-linear map such that $r u_1 = u$ and $r u_i = 0$ for $1< i \leq k$. 
    Let $f\in \End_R (\U)$ and write $u_1 f = \sum_i u_i d_i$ for $d_1, \ldots, d_k\in \D$. 
    Then
    \[
        u f = (r u_1) f = r (u_1 f) = r\big( \sum_i u_i d_i \big) = u d_1.
    \]
    Therefore $f = \rho(d_1)$, proving that the image of $\rho$ is the whole $\End_\D (\U)$.
\end{proof}
 
Finally, we will assume $G$ is abelian. 
Recall that, in this case, the center of a graded algebra is a graded subalgebra (\cref{lemma:center-is-graded}). 
% Also, note that if $d\in Z(\D)$, then the map $\rho(d)$ is not only $R$-linear, but also $\D$-linear. 

\begin{prop}\label{prop:R-and-D-have-the-same-center}
    Suppose $G$ is abelian. 
    Then the map $\iota\from Z(\D) \to Z(R)$ given by $\iota (d)(u) \coloneqq ud$ is an isomorphism of graded algebras.
\end{prop}

\begin{proof}
    First of all, $\iota$ is well-defined. 
    Indeed, $\iota(d)\from \U \to \U$ is $\D$-linear for all $d\in Z(\D)$, \ie, $\iota(d) \in R = \End_\D(\U)$. 
    Now, if $r\in R$ and $u \in \U$, then $r (\iota(d)(u)) = r(ud) = r(u) d = \iota(d) (r(u))$. 
    Hence $\iota(d) \in Z(R)$. 
    Clearly, $\deg \iota(d) = \deg d$. 
    
    To show that $\iota$ is an isomorphism, we identify $\D$ with $\End_R (\U)$ via $\rho$ as in \cref{lemma:converse-density-thm}. 
    Computations analogous to the ones above show that the map $\iota'\from Z(R) \to Z(\D)$ given by $\iota'(r)(u) \coloneqq r(u)$ is well-defined, and it is straightforward that $\iota'$ is the inverse of $\iota$. 
\end{proof}

% Recall the canonical map $Z(\D) \to Z(R)$ given by $c \mapsto r_c$.


% We will now show that we can identify $Z(\D)$ with $Z(R)$.
% For every $c\in Z(\D)$, consider $r_c\from \U \to \U$ given by $r_c(u) = uc$.
% Clearly, $r_c$ is $\D$-linear, so $r_c \in R$.
% Actually, we have $r_c\in Z(R)$.
% Indeed, for all $r\in R = \End_\D(\U)$ and all $u\in \U$, we have $r (r_c(u)) = r(uc) = r(u) c = r_c(r(u))$.

% \begin{prop}
% 	Let $R = \End_\D(\U)$. 
% 	The map $Z(\D) \to Z(R)$ given by $c \mapsto r_c$ is an isomorphism of $G$-graded algebras.
% \end{prop}

% \begin{proof}
% 	Given $r\in Z(R)$, we can define $c_r\in \End_R (\U) =\D$ by $uc_r = r(u)$ for all $u\in \U$.
% 	Computations analogous to the ones above show that $c\in Z(\D)$ (\cref{lemma:converse-density-thm}), and it is clear that the map $r\mapsto c_r$ is the inverse of the map $c \mapsto r_c$.
% 	The definition of grading on $R = \End_\D (\U)$ implies that these maps are isomorphisms of $G$-graded algebras.
% \end{proof}

\subsection{Finite dimensional graded-division algebras over an algebraically closed field}\label{ssec:grd-div-alg}

The classification of graded-division algebras over $\FF$ involves the classification of usual division algebras and certain cohomology sets of $G$ (see \cite{Guido}), which is unattainable in general. 
Fortunately, for our purposes, we only need this classification in a very special case. 
For gradings on (super)involution-simple associative (super)algebras or simple Lie (super)algebras, we may assume that $G$ is abelian (see \cref{prop:grd-simple-vphi-abelian,prop:simple-Lie-G-abelian}). 
Also, we will restrict ourselves to finite dimensional algebras over an algebraically closed field. 
Hence, for the remainder of this subsection, we will assume that $G$ is abelian and that $\FF$ is algebraically closed.

Let $\D$ be a finite dimensional graded-division algebra and let $T \coloneqq \supp \D$. 
We then have that $T$ is a finite subgroup of $G$.
For every $t\in T$, let us fix $0 \neq X_t\in D_t$. 
Then it is easy to see that $\D_t = \D_e X_t$. 
Since $\D_e$ is a finite dimensional division algebra and $\FF$ is algebraically closed, $\D_e = \FF$. 
It follows that $\dim_\FF \D_t = 1$ for all $t\in T$.

\begin{defi}\label{def:bicharacter}
    A map $b\from T\times T \to \FF^\times$ is said to be a \emph{bicharacter} if, for every $t\in T$, both maps $b(t, \cdot)\from T \to \FF^\times$ and $b(\cdot, t)\from T \to \FF^\times$ are characters. 
    We say that a bicharacter $b$ is 
    \begin{itemize}
        \item \emph{symmetric} if $b(t,s) = b(s,t)$ for all $t,s\in T$;
        \item \emph{skew-symmetric} if $b(t,s) = b(s,t)\inv$ for all $t,s\in T$;
        \item \emph{alternating} if $b(t,t) = 1$ for all $t\in T$.
    \end{itemize}
    Clearly, every alternating bicharacter is skew-symmetric. 
    The \emph{radical} of a (skew-)symmetric bicharacter is the subgroup of $T$ defined by
    \[
        \rad b \coloneqq \{ t \in T \mid b(t, T) =1 \}.
    \]
    If $\rad b = \{e\}$, we say that $b$ is \emph{nondegenerate}.
\end{defi}

Since $T$ is abelian and every homogeneous component is one-dimensional, for any $t, s \in T$, there exists a nonzero scalar $\beta(t,s)$ such that 
\[\label{eq:beta}
    X_t X_s = \beta(t, s) X_s X_t.
\]
Note that $\beta(t,s)$ does not depend on the choice of $X_t$ and $X_s$. 
It is easy to see that the map $\beta\from T\times T \to \FF^\times$ is an alternating bicharacter and that $\rad \beta$ is the support of $Z(\D)$. 
The following is a consequence of a well-known result in group cohomology (see \cite[Section 2.2]{EK_d4}), but we give a different proof here for completeness (see \cite[Section 4]{BZ18}). 

\begin{prop}\label{prop:T-beta-determines-iso}
    The pair $(T, \beta)$ determines the isomorphism class of the graded-division algebra $\D$.
\end{prop}

\begin{proof}
    Write $T = \langle t_1 \rangle \times \cdots \times \langle t_k \rangle$ and let $n_i$ denote the order of $t_i$, for all $1\leq i \leq k$. 
    Since $\FF$ is algebraically closed, scaling $X_{t_i}$ if necessary, we can assume  $X_{t_i}^{n_i} = 1$. 
    
    Let $\mc F$ be the free associative algebra generated by the symbols $Y_{t_1}, \ldots, Y_{t_k}$. 
    We can make $\mc F$ a $T$-graded algebra by assigning $\deg Y_{t_i} \coloneqq t_i$. 
    % Why?
    Let $\mathfrak{D}$ denote the quotient of $\mc F$ by the ideal generated by 
    \[\label{eq:relations-D}
        Y_{t_i}^{n_i} - 1 \text{ and }Y_{t_i}Y_{t_j} - \beta(t_i, t_j) Y_{t_j}Y_{t_i},
    \] 
    for all $1\leq i,j \leq k$. 
    Since the relators are homogeneous, $\mathfrak{D}$ is also a $T$-graded algebra. 
    Note that $\mathfrak{D}$ depends only on $T$, $\beta$ and the choice of the elements $t_1, \ldots, t_k \in T$.  
    Also, it is clear from the relators that $\mathfrak{D}$ is spanned by the elements of the form $Y_{t_1}^{m_1}Y_{t_2}^{m_2} \cdots Y_{t_k}^{m_k}$, where $0\leq m_i \leq n_i - 1$. 
    In particular $\dim \mathfrak{D} \leq |T| = \dim \D$.
    
    Clearly, there is a unique surjective algebra homomorphism $\psi\from \mathfrak{D} \to \D$ such that $\psi(Y_{t_i}) = X_{t_i}$, which is degree preserving. 
    We then must have that $\dim \mathfrak{D} \geq \dim \D$, so $\dim \mathfrak{D} = \dim \D$ and, therefore, $\psi$ is an isomorphism of $T$-graded algebras. 
    %
    % It follows that $\dim \mathfrak{D} = \dim \D$. 
    % From this fact we have that the elements of the form $Y_{t_1}^{m_1}Y_{t_2}^{m_2} \cdots Y_{t_k}^{m_k}$ form a basis of $\mathfrak{D}$, and is easy to see that declaring $\deg Y_{t_1}^{m_1}Y_{t_2}^{m_2}\cdots Y_{t_k}^{m_k} \coloneqq t_1^{m_1}t_2^{m_2}\cdots t_k^{m_k}$ defines a grading on $\mathfrak{D}$. 
    % It also follows that $\psi$ is a bijection, and with the grading we just defined, it is isomorphism of graded algebras. 
\end{proof}

There remains the question of existence of a graded-division algebra $\D$ for a given $(T,\beta)$. 
This, again, follows from cohomology. 
We give another proof for completeness. 

\begin{lemma}\label{lemma:colour-tensor-product}
    Let $T$ be a finite abelian group and let $\beta\from T\times T \to \FF^\times$ be an alternating bicharacter. 
    Suppose that $T = A\times B$ for subgroups $A, B \subseteq T$ and that there are graded-division algebras $\mc A$ and $\mc B$ associated to $(A, \beta\restriction_{A\times A})$ and $(B, \beta\restriction_{B\times B})$, respectively. 
    Then there is a graded-division algebra associated to $(T, \beta)$. 
    Further, if $\beta(A,B) = 1$, then this graded-division algebra is isomorphic to $\mc A \tensor \mc B$ (with its usual product).
\end{lemma}

\begin{proof}
    Choose elements $0 \neq X_a \in \mc A_a$ and $0 \neq X_b \in \mc B_b$, for all $a\in A$ and $b\in B$. 
    On the graded vector space $\mc A \tensor \mc B$, define a product by \[(X_a \tensor X_b) (X_{a'} \tensor X_{b'}) \coloneqq \beta(b, a') (X_{a} X_{a'}) \tensor (X_{b}X_{b'}),\] for all $a,a' \in A$ and $b, b' \in B$ (this construction is called \emph{colour tensor product} in \cite[page 88]{MR1192546}, and is a special case of the concept of \emph{twisted tensor product} in \cite{twisted-tensor}). 
    Clearly, this makes $\mc A \tensor \mc B$ a graded algebra, and it is associative (see \cite{MR1192546}) %:
%    \begin{align}
  %      \big( (X_{a} \tensor X_{b}) (X_{a'} \tensor X_{b'}) \big) (X_{a''} \tensor & X_{b''}) 
  %      = \beta(b, a') \big( (X_{a} X_{a'}) \tensor (X_{b}X_{b'}) \big) (X_{a''} \tensor X_{b''})\\
   %     &= \beta(b, a') \beta(bb', a'') (X_{a} X_{a'} X_{a''}) \tensor (X_{b}X_{b'}X_{b''}) \\
    %    &= \beta(b, a') \beta(b, a'') \beta(b', a'') (X_{a} X_{a'} X_{a''}) \tensor (X_{b}X_{b'}X_{b''}) \\
     %   &= \beta(b, a'a'') \beta(b', a'') (X_{a} X_{a'} X_{a''}) \tensor (X_{b}X_{b'}X_{b''}) \\
        %&= \beta(b', a'') (X_a \tensor X_b) \big( (X_{a'} X_{a''}) \tensor (X_{b'} X_{b''}) \big) \\
        %&= (X_a \tensor X_b) \big( (X_{a'} \tensor X_{b'}) \tensor (X_{a''} X_{b''}) \big),
    %\end{align}
    %for all $a,a', a'' \in A$ and $b, b', b'' \in B$. 
    with identity element $1_A \tensor 1_B$. Since the homogeneous elements in $A\tensor B$ are scalar multiples of $X_a \tensor X_b$, $a\in A$, $b\in B$, we see that it is a graded-division algebra, and that it is associated to $(T, \beta)$. 
\end{proof}

\begin{prop}
    For every pair $(T, \beta)$, there is a graded-division algebra associated to it. 
\end{prop}

\begin{proof}
    We write $T = \langle t_1 \rangle \times \cdots \times \langle t_k \rangle$ and proceed by induction on $k$. 
    If $k = 1$, then $\beta$ must be trivial: $\beta(t_1^i,t_1^j) = \beta(t_1, t_1)^{ij} = 1$, for all $i, j \in \ZZ$. 
    Hence, the group algebra $\FF T$ is a graded-division algebra associated to $(T, \beta)$. 
    The induction step follows from the case $k=1$ and \cref{lemma:colour-tensor-product}.
\end{proof}

If we suppose that $\beta$ is nondegenerate, than we can construct a graded-division algebra associated to $(T,\beta)$ using matrices. 
For that, we follow \cite[Remark 18]{EK15} (see also \cite[Remark 2.16]{livromicha}).

First of all, we can decompose the group $T$ as $A\times B$, where the restrictions of $\beta$ to each of the subgroups $A$ and $B$ are trivial (see \cite[page 36]{livromicha}) and, hence, $A$ and $B$ are in duality by $\beta$, \ie, the map $A \to \widehat B$ given by $a \mapsto \beta(a, \cdot)$ is an isomorphism of groups (note that, in particular, $|T|$ is a perfect square). 

Let $V$ be the vector space with basis $\{e_b\}_{b\in B}$ (\ie, $V$ is the vector space underlying the group algebra $\FF B$). 
For each $a\in A$, define $X_a\in \End(V)$ by
\[
    \forall b' \in B, \quad X_a (e_{b'}) \coloneqq \beta(a, b')e_{b'},
\]
and, for each $b\in B$, define $X_b\in \End(V)$ by
\[
    \forall b' \in B, \quad X_b (e_{b'}) \coloneqq e_{bb'}.
\]
Finally, we define $X_{ab} \coloneqq X_a X_b$, for all $a\in A$ and $b\in B$. 

\begin{prop}
    The operators $X_t$, $t\in T$ form a basis of the algebra $\End(V)$ and define a division grading associated to $(T, \beta)$. 
\end{prop}

\begin{proof}
    Clearly, the operators $X_a$ and $X_b$ (and, hence, $X_{ab})$ are invertible, for all $a\in A$ and $b\in B$. 
    It is easy to see that $X_a X_{a'} = X_{aa'}$ and $X_{b} X_{b'} = X_{bb'}$, for all $a, a' \in A$ and $b, b'\in B$. 
    Also, $X_a ( X_b(e_{b'}) ) = \beta(a, bb') e_{bb'} = \beta(a, b) \beta(a,b') e_{bb'}$ and $X_b ( X_a(e_{b'}) ) = \beta(a, b') e_{bb'}$, so $X_a X_b = \beta(a,b) X_b X_a$. 
    It follows that, for all $t,s \in T$, $X_t X_s = \beta(t,s) X_s X_t$.

    It remains to show that the operators $X_t$, $t\in T$, form a basis of $\End(V)$. 
    Since $|T| = |A||B| = |B|^2$ and $\dim V = |B|$, it suffices to prove linear independence. 
    Supposes $\sum_{a,b} \lambda_{ab} X_{ab} = 0$, for some $\lambda_{ab} \in \FF$. 
    Then for each $b' \in B$, we have that
    \[
        \sum_{a,b} \lambda_{ab} X_{ab} (e_{b'}) = \lambda_{ab} \beta(a, b) \beta(a,b') e_{bb'} = 0.
    \]
    Since $\{ e_{bb'} \}$ is linearly independent, we have
    \[
        \sum_a \lambda_{ab} \beta(a, b) \beta(a,b') = 0,
    \]
    for all $b,b' \in B$. 
    It follows that
    \[
        \sum_a \lambda_{ab} \beta(a, b) \beta(a, \cdot) = 0,
    \]
    for all $b\in B$. 
    Since $\beta$ is nondegenerate, the maps $\beta(a, \cdot) \in \widehat B$ are all distinct characters in, and since distinct characters are linearly independent, we have
    $
        \lambda_{ab} \beta(a, b) = 0,
    $
    for all $a\in A$ and $b\in B$. 
    Therefore, $\lambda_{ab} = 0$, as desired. 
\end{proof}

\begin{defi}\label{def:standard-realization}
	We will refer to these matrix models of $\mc D$ as its \emph{standard realizations}.
\end{defi}

Note that \cref{ex:Pauli-ell-x-ell} is a standard realization for $(T, \beta)$ where $T = \ZZ_\ell \times \ZZ_\ell$ and $\beta ((i, j), (i', j')) = \xi^{ij' - i'j}$, for all $i, i', j, j' \in \ZZ_\ell$.

\begin{cor}\label{cor:D-simple-iff-beta-nondeg}
    The graded-division algebra $\D$ is simple as an (ungraded) algebra if, and only if, $\beta$ is nondegenerate. 
\end{cor}

\begin{proof}
    If $\beta$ is nondegenerate, then the construction of a standard realization above shows that $(T, \beta)$ has a model with a simple algebra and, hence, $\D$ must be simple by \cref{prop:T-beta-determines-iso}. 
    
    Conversely, if $\D$ is simple as an algebra, then, since $\FF$ is algebraically closed, $\D$ must be central (\ie, $Z(\D) = \FF$), so $\rad \beta = \{e\}$.
\end{proof}

We finish with a well-known result for future reference.

\begin{lemma}\label{lemma:Aut(D)-widehat-T}
    An $\FF$-linear map $\psi_0\from \D \to \D$ is an automorphism of the graded algebra $\D$ if, and only if, there is $\chi \in \widehat T$ such that $\psi_0( X_t ) = \chi(t) X_t$ for all $t\in T$. 
\end{lemma}

\begin{proof}
    Any invertible degree-preserving linear map $\psi\from \D \to \D$ is determined by a map $\chi\from T \to \FF^\times$ such that $\psi(X_t) = \chi(t) X_t$, for all $t\in T$. 
    It is easy to see that $\psi$ is an automorphism if, and only if, $\chi$ is a group homomorphism, \ie, $\chi \in \widehat T$. 
\end{proof}

% \begin{remark}
%     Recall the $\widehat T$-action defined by the $T$-grading on $\D$ (see \cref{sec:g-hat-action}) and let $\rho\from \widehat T \to \Aut (\D)$ be its corresponding representation map. 
%     Then \cref{lemma:Aut(D)-widehat-T} implies that that $\rho$ is an isomorphism of groups.
% \end{remark}

\subsection{Finite dimensional graded-simple algebras over an algebraically closed field}\label{ssec:param-End_D-U}

We continue assuming that $\FF$ is algebraically closed and that $G$ is abelian.

If $R$ is a graded-simple algebra then, by \cref{thm:End-over-D}, $R \iso \End_\D (\U)$, where $\D$ is a graded-division algebra and $\U$ is graded right $\D$-module of finite rank, say, $k$. 
Since $\End_\D (\U) \iso M_k(\D)$, we have that $R$ is finite dimensional if, and only if, $\D$ is finite dimensional if, and only if, $\D$ is associated to a pair $(T,\beta)$ as in the previous subsection.

\begin{defi}\label{def:E(D,U)}
    Let $\D$ be a finite dimensional graded-division algebra over an algebraically closed field $\FF$ and let $\U$ be a graded right $\D$-module of finite rank. 
	If $\D$ is associated to $(T, \beta)$ and $\U$ is associated to $\kappa\from G/T \to \ZZ_{\geq 0}$ (see Subsection \ref{ssec:D-modules}), we say that $(T, \beta, \kappa)$ are the \emph{parameters} of the pair $(\D, \U)$.
\end{defi}

It is easy to see that, since $G$ is abelian, $\U^{[g]}$ is associated to $g \cdot \kappa$, where the $G$-action on functions $G/T \to \ZZ_{\geq 0}$ is defined as usual: $(g\cdot \kappa) (x) \coloneqq \kappa(g\inv x)$ for all $x\in G/T$.

If $\psi_0\from \D \to \D'$ is an isomorphism of graded algebras and $\U'$ is a graded right $\D'$-module associated to $\kappa'$, it is clear that $\dim_{\D'} \U_x' = \dim_\D (\U_x')^{\psi_0}$, for all $x \in G/T$, and, hence, the graded $\D$-module $(\U')^{\psi_0}$ is also associated to $\kappa'$. 

Thus, \cref{thm:iso-abstract} becomes the following:

\begin{thm}\label{thm:iso-End_D-U-with-parameters}
	Let $(\D, \U)$ and $(\D', \U')$ be pairs as in Definition \ref{def:E(D,U)}, and let $(T, \beta, \kappa)$ and $(T', \beta', \kappa')$ be their parameters. 
	Then $\End_\D (\U) \iso \End_{\D'} (\U')$ if, and only if, $T = T'$, $\beta = \beta'$, and $\kappa'$ belongs to the $G$-orbit of $\kappa$. \qed
% 	and there is $g\in G$ such that $\kappa' = g \cdot \kappa$. 
\end{thm}

It should be noted that, combining \cref{prop:R-simple-iff-D-simple,cor:D-simple-iff-beta-nondeg}, we have that $\End_\D (\U)$ is simple as an (ungraded) algebra if, and only if, $\beta$ is nondegenerate. 
If this is the case, \cref{thm:iso-End_D-U-with-parameters} gives us the classification of abelian group gradings on matrix algebras up to isomorphism:

\begin{defi}\label{def:Gamma-T-beta-kappa}
    Let $n > 0$ be a natural number. 
    Given a finite subgroup $T \subseteq G$, a nondegenerate bicharacter $\beta\from T\times T \to \FF^\times$ and a map $\kappa\from G/T \to \ZZ_{\geq 0}$ with finite support such that $|\kappa| \sqrt{|T|} = n$, consider
    \begin{enumerate}[(i)]
        \item a standard realization $\D$ (see \cref{def:standard-realization}) of a matrix algebra with a division grading associated to $(T,\beta)$;
        \item the elementary grading (see \cref{defi:elementary-grd}) on $M_{k}(\FF)$ defined by a $k$-tuple $\gamma = (g_1, \ldots, g_{k})$ of elements of $G$, $k \coloneqq |\kappa|$, such that the number of entries $g_i$ with $g_i\in x$ is equal to $\kappa (x)$ for every $x\in G/T$. 
    \end{enumerate}
    We define $\Gamma (T, \beta, \kappa)$ to be the grading on $M_n(\FF)$ given by identifying $M_n(\FF)$ with the graded algebra $M_k(\FF) \tensor \D$ via Kronecker product, \ie,
    \[
        \deg \left( E_{ij} \tensor d \right) = g_ig_j\inv t,
    \] 
    for all $1\leq i, j \leq k$, $t\in T$ and $0 \neq d \in \D_t$.
\end{defi}

Note that we are abusing notation in \cref{def:Gamma-T-beta-kappa}. 
The grading $\Gamma (T, \beta, \kappa)$ actually depends on the choices of the standard realization $\D$ and of the $k$-tuple $\gamma$. 
Nevertheless, its isomorphism class depends only on $(T, \beta, \kappa)$. 

\begin{cor}[{\cite[Theorem 2.6]{BK10},\cite[Theorem 2.27]{livromicha}}]
    Every $G$-grading on $M_n(\FF)$ is isomorphic to $\Gamma (T, \beta, \kappa)$ as in \cref{def:Gamma-T-beta-kappa}. 
    Two such gradings $\Gamma (T, \beta, \kappa)$ and $\Gamma (T', \beta', \kappa')$ are isomorphic if, and only if, $T = T'$, $\beta = \beta'$ and there is $g\in G$ such that  $g \cdot \kappa = \kappa'$. \qed
\end{cor}

As an application of \cref{thm:End-over-D,thm:iso-End_D-U-with-parameters}, we can obtain a classification of finite dimensional simple superalgebras over an algebraically closed field. 
For a classification over an arbitrary field, we refer the reader to \cite{racine}. 

\begin{thm}\label{thm:fd-simple-SA}
    Let $R$ be a finite dimensional simple superalgebra over an algebraically closed field $\FF$. 
    Then $R \iso M(m,n)$ or $R \iso Q(n)$, for some $m,n \in \ZZ_{\geq 0}$. 
    Moreover,
    \begin{enumerate}[(i)]
        \item $M(m,n) \not\iso Q(n')$;
        \item $M(m,n) \iso M(m', n')$ if, and only if, either $m=m'$ and $n=n'$, or $m=n'$ and $n=m'$;
        \item $Q(n) \iso Q(n')$ if, and only if, $n = n'$.
    \end{enumerate}
\end{thm}

\begin{proof}
    Since $R$ is simple as a $\ZZ_2$-graded algebra, so, by \cref{thm:End-over-D}, $R \iso \End_\D (\U)$. 
    Let $(T, \beta, \kappa)$ be the parameters of $(\D, \U)$. 
    Since $T\subseteq \ZZ_2$, we either have $T = \{ \bar 0\}$ or $T = \ZZ_2$. 
    
    If $T = \{\bar 0\}$, then $\D = \FF$ and $R \iso \End_\FF (\U)$. 
    The isomorphism class of $\U$ is determined by the map $\kappa\from \ZZ_2/\{\bar 0\} = \ZZ_2 \to \ZZ_{\geq 0}$ defined by $\kappa(i) = \dim_\FF (\U^i)$ for all $i\in \ZZ_2$. 
    In other words, the isomorphism class of $\U$ is determined by the numbers $m \coloneqq \kappa(\bar 0)$ and $n \coloneqq \kappa(\bar 1)$. 
    By choosing bases for $\U\even$ and $\U\odd$, we get $R \iso M(m,n)$. 
    By \cref{thm:iso-End_D-U-with-parameters}, we conclude that $M(m,n) \iso M(m',n')$ if, and only if, either $m=m'$ and $n=n'$ (for $g = \bar 0$), or $m=n'$ and $n=m'$ (for $g = \bar 1)$. 
    
    If $T = \ZZ_2$, we first note that the only alternating bicharacter $\beta\from \ZZ_2\times \ZZ_2 \to \FF^\times$ is the trivial one. 
    Hence, $\D \iso \FF\ZZ_2 \iso Q(1)$ and $R \iso \End_{Q(1)}(\U)$. 
    The isomorphism class of $\U$ is determined by a map $\kappa\from \ZZ_2/\ZZ_2 \to \ZZ_{\geq 0}$, \ie, by the single number $n \coloneqq \kappa(\ZZ_2) = \dim_{Q(1)} (\U)$. 
    
    To conclude the proof, we will show that $\End_{Q(1)}(\U) \iso Q(n)$.  
    By \cref{prop:U_g-is-D_e-form}, we can take a graded basis $\B$ for $\U$ with all elements having degree $\bar 0$.
    We, then, can write $\End_{Q(1)}(\U) \iso M_n(Q(1)) \iso \M_n(\FF) \tensor Q(1)$ as graded algebras (\cref{rmk:M_k(F)-tensor-D}), where the grading on $\M_n(\FF)$ is trivial. 
    Finally, using the Kronecker product, $M_n(\FF) \tensor Q(1) \iso Q(1)\tensor M_n(\FF) \iso Q(n)$. 
\end{proof}

\begin{defi}
    Let $R$ be a finite dimensional simple associative superalgebra. 
    If $R \iso M(m,n)$, for some $m,n \geq 0$, we say that $R$ is of type $M$. 
    If $R \iso Q(n)$, for some $n \geq 0$, we say that $R$ is of type $Q$. 
\end{defi}

\section{Graded-simple associative superalgebras}\label{sec:grd-simple-salg}

We will now adapt the results of the previous section to graded-simple superalgebras. 
For any abelian group $G$, we obtain a classification of $G$-graded graded-simple superalgebra over an algebraically closed field of any characteristic, following the approach used in \cite{paper-MAP} fom $M(m,n)$. 
A different (and more complicated) approach was used in \cite{BS} to obtain a description of $G$-gradings (with some restrictions on characteristic), but the isomorphism problem was not solved there. 

Let $G$ be a group. 
Recall that a $G$-graded superalgebra $R$ can be seen as a $G^\# \coloneqq G\times \ZZ_2$-graded algebra (see \cref{rmk:G-sharp}) by defining $R_{(g,i)} = R_g \cap R^i$, for all $g\in G$ and $i \in \ZZ_2$. 
We identify $G$ with $G \times \{ \bar 0 \} \subseteq G^\#$. 
Clearly, $R$ is graded-simple as a $G$-graded superalgebra if, and only if, it is graded-simple as a $G^\#$-graded algebra. 
This allows us to easily transfer the results of previous section to gradings on superalgebras, but at the cost of working in the group $G^\#$ instead of $G$. 

\begin{remark}
	If the canonical $\ZZ_2$-grading is a coarsening of the $G$-grading by means of a homomorphism $p\colon G\rightarrow \ZZ_2$ (referred to as the \emph{parity homomorphism}), then we could work with the pair $(G, p)$ instead of the group $G^\#$. 
	Note that, in this case we would have another isomorphic copy of $G$ in $G^\#$, namely, the image of the embedding $g\mapsto (g, p (g))$, which contains the support of the $G^\#$-grading. 
    % 	In this case, we do not need $G^\#$ and can work with the original $G$-grading.
\end{remark}

\subsection{Graded-division superalgebras and their supermodules}\label{ssec:supermodules-over-D}

\begin{defi}
    A $G$-graded superalgebra $\D$ is said to be a \emph{graded-division superalgebra} if every nonzero element which is homogeneous with respect to both the $G$-grading and the canonical $\ZZ_2$-grading is invertible. 
    In this case, we may also refer to the $G$-grading on the superalgebra $\D$ as a \emph{division grading}.
\end{defi}

In other words, the graded-division superalgebras with respect the $G$-grading are precisely the graded-division algebras with respect the $G^\#$-grading. 
Recall that $T \coloneqq \supp \D$ is a subgroup of $G^\#$, so we can see $\D$ as a $T$-graded algebra and the canonical $\ZZ_2$-grading is the coarsening by the group homomorphism $p\from T \subseteq G^\# \to \ZZ_2$ given by $p (g, i) = i$ for all $(g,i)\in G^\#$. 
We will denote the kernel of $p$ by $T^+$, \ie, $T^+ \coloneqq T\cap (G\times \{ \bar 0 \}) = \supp \D\even$. 
Similarly, we define $T^- \coloneqq T\cap (G\times \{ \bar 1 \}) =  \supp \D\odd$. 

\begin{notation}
    For graded-division superalgebras we will use subscripts to refer to the $T$-grading, \ie, $\D_t$ refers to the homogeneous component $\D_g^i$ where $t = (g, i) \in T \subseteq G^\#$. 
\end{notation}

\begin{ex}\label{ex:Q(1)-as-grd-div-SA}
    Let $\langle u \rangle$ be a cyclic group of order $2$ and let $G = \langle h \rangle$ where $h$ has order at most $2$.  
    The group algebra $\D \coloneqq \FF\langle u \rangle$, with canonical $\ZZ_2$-grading given by $\D\even \coloneqq \FF 1$ and $\D\odd = \FF u$, becomes a $G$-graded graded-division superalgebra if we declare the $G$-degree of $u$ to be $h$. 
    In this case $G^\# = \langle h \rangle \times \ZZ_2$ and $T = \langle (h, \bar 1) \rangle \iso \ZZ_2$ 
    (compare with \cref{ex:group-algebra}). 
    Note that $\D \iso Q(1)$ as a superalgebra.
\end{ex}

\begin{ex}\label{ex:Pauli-2x2-super}
    \Cref{ex:Pauli-2x2} can be seem as a $\ZZ_2$-grading division grading on $M(1,1)$, $\Char \FF \neq 2$. 
    In this case, $G^\# = \ZZ_2 \times \ZZ_2$.
\end{ex}

Let $\U = \U\even \oplus \U\odd$ be a graded right $\D$-supermodule of finite rank. 
The isomorphism class of the graded right $\D$-supermodule $\U$ is, as in \cref{ssec:D-modules}, determined by the map $\kappa\from G^\#/T \to \ZZ_{\geq 0}$ given by $\kappa (x) = \dim_\D \U_x$ for all $x\in G^\#/T$. 
We also have a description only in terms of the group $G$, but for that we separate the graded-division superalgebras in two classes:

\begin{defi}\label{defi:even-odd-D}
    % Let $\D$ be a graded-division superalgebra. 
    If $\D = \D\even$ (\ie, $\D\odd = 0$) we say that $\D$ is an \emph{even} graded-division superalgebra, and if $\D \neq \D\even$ (\ie, $\D\odd \neq 0$) we say that $\D$ is an \emph{odd} graded-division superalgebra. 
\end{defi}

Note that if $\D$ is odd and finite dimensional, then $\dim_\FF \D\even = \dim_\FF \D\odd$. 

\begin{lemma}\label{lemma:odd-M-m=n}
    If $\D$ is odd and $\D \iso M(m,n)$ as a superalgebra, then $m = n$. 
\end{lemma}

\begin{proof}
    We have that $\dim M(m,n)\even = m^2 + n^2$ and $\dim M(m,n)\odd = 2mn$. 
    Hence $\dim M(m,n)\even = \dim M(m,n)\odd$ if, and only if, $m=n$. 
\end{proof}

Assume that $\D$ is an even graded-division superalgebra. 
Then both $\U\even$ and $\U\odd$ are graded $\D$-submodules, hence we can describe their isomorphism classes, respectively, by maps $\kappa_\bz, \kappa_\bo\from G/T \to \ZZ_{\geq 0}$ given by $\kappa_i (x) = \dim \U^i_x$ for all $i\in \ZZ_2$ and $x\in G/T$. 
Note that, in this case, $G^\#/T = G^\#/T^+$ is the disjoint union of $G/T$ and $(e, \bar 1) \cdot G/T$ and, clearly, $\kappa ((g,i)T) = \kappa_i (gT)$, for all $i\in \ZZ_2$ and $g\in G$. 

Now assume that $\D$ is odd. 
In this case, unless $\U = 0$, the graded subspaces $\U\even$ and $\U\odd$ are not $\D$-submodules. 
But we can follow a different approach.

\begin{defi}
    A graded basis of a graded supermodule $\U$ is said to be an \emph{even basis} if it consists only of even elements.
\end{defi}

Let $\B = \{u_\lambda\}_{\lambda \in \Lambda}$ be any basis for $\U$, and let $0 \neq d_1\in \D\odd$. 
For every $\lambda \in \Lambda$, if $u_\lambda$ is an odd element, let us replace it by $u_\lambda d_1$. 
The resulting set is, clearly, a even basis of $\U$. 

\begin{convention}\label{conv:pick-even-basis}
    If $\D$ is an odd graded-division superalgebra, we choose the graded basis $\B$ to be an even basis.
\end{convention}

It follows that the canonical $\ZZ_2$-grading on $\U$ is determined only by $\D$. 
To see that, take any even basis of $\U$ and let $\tilde \U$ be its $\D_e$-span. 
Clearly, $\tilde \U \subseteq \U\even$ and it is a  $\D_e$-form of $\U$, \ie, we can identify $\U = \tilde\U \tensor_{\D_e} \D$ and, hence, $\U\even = \tilde\U \tensor_{\D_e} \D\even$ and  $\U\odd = \tilde\U \tensor_{\D_e} \D\odd$. 

This can also be seen from the point of view of the map $\kappa$. 
Note that, every coset $x \in G^\#/T$ has an even representative. 
This entails that the map $\iota\from G/T^+ \to G^\#/T$ given by $\iota ( gT^+) = (g, \bar 0) T$ is a bijection, so we can work with $\kappa \circ \iota$, which we will, by abuse of notation, also denote by $\kappa$. 

It should be noted that, even though $\U\even$ and $\U\odd$ are not $\D$-submodules, they are $\D\even$-submodules. 
Clearly, the map $\kappa\from G/T^+ \to \ZZ_{\geq 0}$ above is the map associated to both $\U\even$ and $\U\odd$, and a even graded $\D$-basis for $\U$ is a graded $\D\even$-basis for $\U\even$. 

\begin{remark}\label{rmk:R-even-identificatios}
    Letn $R \coloneqq \End_{\D}(\U)$. 
    If $\D$ is even, then we can identify $R\even$ with $\End_{\D}(\U\even) \oplus \End_{\D} (\U\odd)$. 
    If $\D$ is odd, then we can identify $R\even$ with $\End_{\D\even}(\U\even)$. 
\end{remark}

\subsection{Graded Wedderburn theory for superalgebras}\label{ssec:wedderburn-super}

Let $R$ be a $G$-graded superalgebra. 
Note that the graded (left) superideals of $R$ are precisely the graded (left) ideals of $R$ as an $G^\#$-graded superalgebra. 
Hence, \cref{thm:End-over-D,thm:iso-abstract} translate to graded superalgebras by simply attaching the ``super'' prefix wherever it is needed. 

% \begin{thm}\label{thm:super-End-over-D}
% 	Let $G$ be a group and let $R = R\even \oplus R\odd$ be a $G$-graded associative algebra. 
% 	Then $R$ is graded-simple and satisfies the \dcc on graded left superideals if, and only if,	there is a graded-division superalgebra $\D=\D\even \oplus \D\odd$ and a graded right $\D$-supermodule $\U = \U\even \oplus \U\odd$ of finite rank such that $R \simeq \End_{\D} (\mc{U})$ as graded algebras. \qed
% \end{thm}

% \begin{thm}\label{thm:iso-abstract}
% 	Let $R \coloneqq \End_\D(\U)$ and $R' \coloneqq \End_{\D'}(\U')$, where $\D$ and $\D'$ are graded-division superalgebras, and $\U$ and $\U'$ are nonzero right graded supermodules of finite rank over $\D$ and $\D'$, respectively.
% 	Given an isomorphism $\psi\from R \to R'$, there is a triple $(g, \psi_0, \psi_1)$, where $g \in G$, $\psi_0\from {}^{[g\inv]}\D^{[g]} \to \D'$ is an isomorphism of graded superalgebras, $\psi_1\from \U^{[g]} \to (\U')^{\psi_0}$ is an isomorphism of graded right $\D$-supermodules, such that
% 	\begin{equation}\label{eq:def-iso-algebras}
% 		\forall r\in R, \quad \psi(r) = \psi_1 \circ r \circ \psi_1\inv.
% 	\end{equation}
% 	Conversely, given a triple $(g, \psi_0, \psi_1)$ as above, Equation \eqref{eq:def-iso-algebras} defines an isomorphism of graded superalgebras $\psi\from R \to R'$.
% 	Another triple $(g', \psi_0', \psi_1')$ defines the same isomorphism $\psi$ if, and only if, there are $t\in \supp \D'$ and $0 \neq d\in \D'_t$ such that $g'= gt$, $\psi_0' = \mathrm{Int}_{d\inv} \circ \psi_0$ and $\psi_1' (u) = \psi_1 (u) d$ for all $u \in \U$. \qed
% \end{thm}

Let $R$ be graded-simple superalgebra satisfying the \dcc on graded left superideals and write, as in \cref{thm:End-over-D}, $R\iso \End_\D (\U)$ where $\D$ is a graded-division superalgebra and $\U$ is a graded right $\D$-supermodule of finite rank. 
Since the property of being an even or an odd a graded-division superalgebra is invariant under isomorphism, by \cref{thm:iso-abstract}, we can extend \cref{defi:even-odd-D} to $R$:

\begin{defi}\label{defi:even-odd-R}
    If $\D$ is an even graded-division superalgebra, we say that the grading on $R$ is \emph{even}; if $\D$ is an odd graded-division superalgebra, we say that the grading on $R$ is \emph{odd}. 
\end{defi}

For the remaining part of this subsection, fix a graded-division algebra $\D$ and a nonzero graded right $\D$-module of finite rank $\U$, and set $R \coloneqq \End_{\D} (\U)$. 

By the discussion on \cref{ssec:supermodules-over-D}, if the grading on $R$ is even, all the information related to the canonical $\ZZ_2$-grading on $R$ is encoded in the graded right $\D$-supermodule $\mc U=\mc U\even \oplus \mc U\odd$. 
More explicitly, 
\[\label{eq:almost-ZZ-grading}
    R\even = \End_\D(\U\even) \oplus \End_\D(\U\odd) \AND R\odd = \Hom_\D(\U\even, \U\odd) \oplus \Hom_\D(\U\odd, \U\even).
\]
This can also be seen by fixing a graded basis $\{u_1, \ldots, u_{k_\bz}\}$ of $\U\even$ and a graded basis $\{u_{k_\bz + 1}, \ldots, u_{k_\bz + k_\bo}\}$ of $\U\odd$. 
Then $R$ can be identified with $M_{k_\bz \mid k_\bo}(\D)$ as a graded superalgebra. 
If $G$ is an abelian group, following \cref{rmk:M_k(F)-tensor-D}, we can identify $R$ with the tensor product of $G^\#$-graded algebras $M_{k_\bz \mid k_\bo}(\FF)\tensor \D$. 
So, $R\even = M_{k_\bz \mid k_\bo}(\FF)\even \tensor \D$ and $R\odd = M_{k_\bz \mid k_\bo}(\FF)\odd \tensor \D$. 

\begin{remark}\label{rmk:even-ZZ-grading}
    Note that, if the $G$-grading on $R$ is even, we can refine the $G^\#$-grading to a $G\times \ZZ$-grading by defining $R\inv \coloneqq \Hom_{\D} (\U\odd, \U\even)$, $R^0 \coloneqq R\even = \End_\D(\U\even) \oplus \End_\D(\U\odd)$ and $R^1 \coloneqq \Hom_{\D} (\U\even, \U\odd)$. 
\end{remark}

If the grading on $R$ is odd, then, again by the discussion on \cref{ssec:supermodules-over-D}, all the information about the canonical $\ZZ_2$-grading is encoded in $\D$. 
To see it more explicitly, let us follow \cref{conv:pick-even-basis} and recall the maps $f_{i, j, d} \in R = \End_\D (\U)$ defined in the beginning of \cref{ssec:Grd-Wedderburn-Theory}. 
Those maps generate $R$ and, by \cref{eq:grd-M_k(D)-nonabelian}, their parity depend only on $d \in \D$. 

If $G$ is an abelian group, again, this can also be seen from the identification $R = M_k(\FF)\tensor \D$ in \cref{rmk:M_k(F)-tensor-D}. 
The grading on $M_k(\FF)$ is elementary grading defined by a tuple of elements in $G = G \times \{ \bar 0 \}$ and, hence, $R\even = M_k(\D) \tensor \D\even$ and $R\odd = M_k(\D) \tensor \D\odd$. 

\begin{remark}\label{rmk:M(D)=M(FF)-tensor-D}
	For superalgebras, tensor product is usually defined with a different multiplication on the superspace $M_k(\FF)\tensor \D$, given by \[(r_1 \tensor s_1) (r_2 \tensor s_2) = \sign{r_2}{s_1} (r_1 r_2) \tensor (s_1 s_2).\]
	If we are following Convention \ref{conv:pick-even-basis}, either $M_k(\FF)$ or $\D$ have trivial canonical $\ZZ_2$-grading, hence the tensor product of superalgebras coincides with that of algebras.
\end{remark}

Note that we already have a description of even gradings only in terms of the group $G$. 
For odd gradings, though, we still need to describe $\D$ only in terms of the group $G$, which is a harder task, and we are going to do that only for the case $R$ is a finite dimensional superalgebra over an algebraically closed field (see \cref{ssec:T-beta-p,sec:assc-only-G}).
 
\begin{prop}\label{prop:simple-R-D-super}
	The superalgebra $R = \End_\D (\U)$ is simple if, and only if, the superalgebra $\D$ is simple.
\end{prop}

\begin{proof}
    Pick a graded basis for $\U$ following \cref{conv:pick-even-basis} and use it to identify $R$ with $M_k(\D) = M_k(\FF) \tensor \D$, as in \cref{rmk:M(D)=M(FF)-tensor-D}.

	It is well known that the ideals of $M_k(\D)$ are precisely the sets of the form $M_k(I)$ for $I$ an ideal of $\D$.
	We will prove an analog of this for superideals.

	If $I$ is a superideal, $M_k(I) = M_k(\FF) \tensor I$ is also a superideal since it is spanned by a set of $\ZZ_2$-homogeneous elements, namely, the elements of the form $E_{ij}\tensor d$ where $1 \leq i,j \leq k$ and $d \in I\even \cup I\odd$.
	Conversely, if $J = M_k(I)$ is a superideal, then we can write $I = \{ d\in  \D \mid E_{11}\tensor d \in J\}$.
	For every $d\in I$, write $d = d_{\bar 0} + d_{\bar 1}$, where $d_\alpha \in \D^\alpha$, $\alpha \in \ZZ_2$.
	Since the $\ZZ_2$-homogeneous components of $E_{11}\tensor d$ are $E_{11}\tensor d_{\bar 0}$ and $E_{11}\tensor d_{\bar 1}$ and they belong to $J$, we have $d_{\bar 0}, d_{\bar 1} \in I$.
\end{proof}

We also could easily adapt the correspondence between the centers of $\End_\D (\U)$ and $\D$ (\cref{prop:R-and-D-have-the-same-center}) to a correspondence between supercenters. 
It turns out, though, that the correspondence between centers is more useful even in the case of superalgebras. 
For example, if $\Char \FF \neq 2$, then $sZ(M(m,n)) = sZ(Q(n)) = \FF$ while $Z(M(m,n)) \neq Z(Q(n))$, so we can use the centers to distinguish the types of superalgebra (see \cref{prop:types-of-SA-via-center}). 

% -----------------
\subsection{Finite dimensional graded-division superalgebras over an algebraically closed field}\label{ssec:T-beta-p}

We will now focus on the case where $G$ is abelian and $\FF$ is algebraically closed. 
For many of the results in this section we will also assume that $\Char \FF \neq 2$, but this hypothesis is not necessary for the classification results in \cref{ssec:classification-assc-super}.

Let $\D$ be a finite dimensional graded-division superalgebra and consider the pair $(T, \beta)$ associated to it, where $T \subseteq G^\#$ is a finite subgroup of $G^\#$ and $\beta$ is an alternating bicharacter on $T$. 
As we did in \cref{ssec:supermodules-over-D}, let $p\from T \to \ZZ_2$ be the restriction to $T$ of the projection on the second component of $G^\# = G\times \ZZ_2 \to \ZZ_2$, and set $T^+ = \ker p = T \cap (G \times \{ \bar 0 \})$ and $T^- = T \cap (G \times \{ \bar 1 \})$. 
We will also denote by $\beta^+$ the restriction of $\beta$ to $T^+ \times T^+$. 
As done in \cref{ssec:grd-div-alg}, for every $t\in T$ we fix $0 \neq X_t\in D_t$.

It is useful to consider another bicharacter on $T$. 
We define $\tilde\beta\from T\times T \to \FF^\times$ by
\[\label{eq:tilde-beta-def}
    \tilde\beta(t,s) \coloneqq (-1)^{p(t)p(s)}\beta(t,s),
\]
for all $t, s\in T$. 
In other words, we have that
\[\label{eq:tilde-beta}
    X_t X_s = (-1)^{p(t)p(s)} \tilde\beta(t,s) X_s X_t
\]
(compare with \cref{eq:beta}). 
Clearly, the support of $sZ(\D)$, the supercenter of $\D$, is $\rad \tilde\beta$. 

The following result is the first step in the reduction of Lie colour algebras to Lie superalgebras, due to M. Scheunert (see \cite{MR529734}). 

\begin{prop}\label{prop:skew-bicharacter-grd-SA}
    Suppose $\Char \FF \neq 2$. 
    Let $T$ be an abelian group and $\tilde\beta$ be a skew-symmetric bicharacter on $T$. 
    Then there is a alternating bicharacter $\beta$ on $T$ and a group homomorphism $p\from T \to \ZZ_2$ such that $\tilde\beta(t,s) = (-1)^{p(t)p(s)}\beta(t,s)$ for all $t,s\in T$.
\end{prop}

\begin{proof}
    Since $\tilde\beta$ is skew-symmetric, $\tilde\beta(t,t) = \tilde\beta(t,t)\inv$, so $\tilde\beta(t,t)\in \pmone$. 
    Define $h\from T \to \pmone$ by $h(t) \coloneqq \tilde\beta(t,t)$, for all $t\in T$. 
    We have that 
    \[
        h(ts) = \tilde\beta(ts,ts) = \tilde\beta(t,t)\tilde\beta(t,s)\tilde\beta(s,t)\tilde\beta(s,s) = h(t)h(s),
    \] 
    for all $t,s \in T$, so $h$ is a group homomorphism. 
    We define $p\from T \to \ZZ_2$ as the unique group homomorphism such that $h(t) = (-1)^{p(t)}$ for all $t\in T$. 
    
    Finally, we define $\beta\from T\times T \to \FF^\times$ by $\beta(t,s) \coloneqq (-1)^{p(t)p(s)}\tilde\beta(t,s)$ for all $t,s\in T$. 
    Clearly, $\beta$ is a bicharacter and $\tilde\beta(t,s) = (-1)^{p(t)p(s)}\beta(t,s)$. 
    It remains to show that $\beta$ is alternating: 
    $\beta(t,t) = (-1)^{p(t)p(t)}\tilde\beta(t,t) = (-1)^{p(t)}\tilde\beta(t,t) = h(t) \tilde\beta(t,t) = 1$, for all $t\in T$.
\end{proof}

\Cref{prop:skew-bicharacter-grd-SA} tells us that a pair $(T, \tilde\beta)$, where $\tilde\beta$ is a skew-symmetric bicharacter on $T$, carries the same information as a triple $(T, \beta, p)$, where $\beta$ is an alternating bicharacter and $p\from T\to \ZZ_2$ is a group homomorphism. 
Throughout this work we decided to use the triples $(T, \beta, p)$ to parametrize finite dimensional graded-division superalgebras over an algebraically closed field, since we refer to the bicharacter $\beta$ and the homomorphism $p$ frequently and also because this parametrization is valid even in the case $\Char \FF = 2$. 
We we will say that the graded-division superalgebra $\D$ is \emph{associated} to triple $(T, \beta, p)$ if the graded-division algebra $\D$ is associated to $(T, \beta)$ and $p$ is its parity homomorphism.

\begin{remark}
    If $\D$ is considered as a $G^\#$-graded algebra, the parameter $p$ is redundant. 
    Nevertheless, it is convenient to keep it since there are situations where we may want to regard $\D$ as a $T$-graded algebra or as a $G$-graded superalgebra.
\end{remark}

\begin{lemma}\label{lemma:rad-tilde-beta}
	Suppose $\Char \FF \neq 2$. 
	Then $\rad \tilde\beta = (\rad \beta)\cap T^+$ and, therefore, $sZ(\D) = Z(\D)\cap \D\even$.
\end{lemma}

\begin{proof}
	Let $t\in T^+$.
	In this case $\tilde\beta (t, \cdot) = \beta (t, \cdot)$, so
	$\tilde\beta (t, T) = 1$ if, and only if, $\beta (t, T) = 1$.
	Hence $(\rad \tilde\beta)\cap T^+ = (\rad \beta)\cap T^+$.

	Now let $t \in T^-$.
	In this case $\tilde \beta (t,t) = (-1)^{|t|} = -1$, so $t \not\in \rad \tilde\beta$.
	Therefore $\rad \tilde\beta = (\rad \tilde\beta)\cap T^+$, concluding the proof.
\end{proof}

Note that if $\Char \FF = 2$, then $\tilde\beta = \beta$. 
Hence, regardless of the characteristic, we have that if $\beta$ is nondegenerate, then so is $\tilde\beta$. 
The converse, however, is not true. 
In view of \cref{lemma:rad-tilde-beta}, if $\Char \FF \neq 2$, the next result characterizes the case where $\tilde\beta$ is nondegenerate but $\beta$ is degenerate. 

\begin{lemma}\label{lemma:beta-deg-beta-tilde-nondeg}
    Suppose $\beta$ is degenerate. 
    Then $\beta^+$ is nondegenerate if, and only if, $(\rad \beta) \cap T^+ = \{e\}$. 
    If this is the case, then $\rad \beta = \langle t_1 \rangle$ for an element $t_1 \in T^-$ of order $2$. 
\end{lemma}

\begin{proof}
    By definition of radical, it is clear that $(\rad \beta) \cap T^+ \subseteq \rad \beta^+$, hence if $\beta^+$ is nondegenerate, then $(\rad \beta) \cap T^+ = \{e\}$. 
    
    Conversely, suppose $(\rad \beta) \cap T^+ = \{e\}$. 
    Since $\rad\beta \neq \{e\}$, there is $t_1 \in (\rad \beta) \cap T^-$ and, hence, $T = (\rad\beta) \times T^+$. 
    We conclude that $\rad \beta = \langle t_1 \rangle$ and that $t_1$ has order $2$. 
    To show that $\beta^+$ is nondegenerate, let $t_0 \in \rad \beta^+$. 
    Then $\beta(t_0, T^+) = 1$ and, clearly, $\beta(t_0, \rad \beta) = 1$. 
    It follows that $t_0 \in \rad \beta$ and, since $t_0 \in T^+$, $t_0 = \{e\}$, concluding the proof. 
\end{proof}

% \begin{lemma}\label{lemma:beta-deg-beta-tilde-nondeg}
%     Suppose $\tilde\beta$ is nondegenerate but $\beta$ is degenerate. 
%     Then:
%     \begin{enumerate}[(i)]
%         \item $\rad \beta = \langle t_1 \rangle$, where  $t_1 \in T^-$ is a order two element; \label{item:rad-beta=t_1}
%         \item $T = (\rad \beta) \times T^+$; \label{item:T+tensor-t_1}
%         \item the restriction of $\beta$ to $T^+ \times T^+$ is nondegenerate. \label{item:beta+nondeg}
%     \end{enumerate}
% \end{lemma}

In \cref{def:standard-realization} we introduced the standard realizations for finite dimensional graded-division algebras that are simple as algebras. 
We are now going to extend this definition for finite dimensional graded-division superalgebras that are simple as superalgebras, \ie, to include superalgebras of type $Q$. 

If $\D \iso Q(n)$, then $\D\even \iso M(n)$ is a graded-division algebra that is simple as an algebra. 
Clearly, $\D\even$ is associated to the pair $(T^+, \beta^+)$, so by \cref{cor:D-simple-iff-beta-nondeg,lemma:beta-deg-beta-tilde-nondeg}, we have that $\rad \beta = \langle t_1 \rangle$ for an element $t_1\in T^-$ of order $2$, hence $t_1 = (h, \bar 1) \in G^\#$ for an element $h \in G$ of order at most $2$. 

Conversely, given a finite subgroup $T^+ \subseteq G$, a nondegenerate alternating bicharacter $\beta^+\from T^+ \times T^+ \to \FF^\times$ and an element $h\in G$ of order at most $2$, let $\D\even$ be a standard realization (see \cref{def:standard-realization}) of a matrix algebra with a division grading associated to $(T^+, \beta^+)$, set $t_1 \coloneqq (h, \bar 1)$ and $T \coloneqq \langle t_1 \rangle \times T^+$, and define $\beta\from T \times T \to \FF^\times$ by $\beta(s t_1^i, t t_1^j) \coloneqq \beta^+(s,t)$. 
It is clear that $\beta$ is an alternating bicharacter and that $\rad \beta = \langle t_1 \rangle$. 
If we consider $Q(1) = \FF \langle u \rangle$ as in \cref{ex:Q(1)-as-grd-div-SA}, with $h$ the $G$-degree of $u$, then $Q(1)$ is the graded-division superalgebra associated to $(\langle t_1 \rangle, \beta\restriction_{ \langle t_1 \rangle \times \langle t_1 \rangle })$. 
By \cref{lemma:colour-tensor-product}, $\D \coloneqq Q(1)\tensor \D\even$ is the graded $G^\#$-graded graded-division algebra associated to $(T, \beta)$, and, via Kronecker product, it is easy to see that $\D$ is a superalgebra of type $Q$. 

\begin{defi}\label{def:standard-realization-Q}
    The $G$-graded superalgebra $\D \coloneqq Q(1) \tensor \D\even = \D\even \oplus u\D\even$, where we declare the $G$-degree of $u$ to be $h$, will be referred to as a \emph{standard realization} of type $Q$ superalgebra with division grading associated to $(T^+, \beta^+, h)$.
\end{defi}

% \begin{cor}\label{prop:Q-T-beta-t_1}
%     The graded-division superalgebra $\D$ is of type $Q$ if, and only if, 
%     $\D$ is odd and $\D\even$ is simple as an algebra. 
% \end{cor}

% \begin{proof}
%     If $\D \iso Q(n)$, then
%     $Q(n)\odd \neq 0$ and $Q(n)\even \iso M_n(\FF)$. 
    
%     Conversely, if $\D$ is odd and $\D\even$ is simple, then $T^- \neq \emptyset$ and $\beta^+$ is nondegenerate. 
%     Hence, $|T^+|$ is a perfect square. 
%     We then have that $|T| = 2 |T^+|$ is not a perfect square, so $\beta$ must be degenerate. 
%     By definition of radical, it is clear that $(\rad \beta) \cap T^+ \subseteq \rad \beta^+$, hence $(\rad \beta) \cap T^+ = \{e\}$. 
%     It follows that $\rad \beta$ is a complement for $T^+$ in $T$, so $\rad \beta = \langle t_1 \rangle$ where $t_1 \in T^-$ is an element of order $2$. 
%     Since $T = (\rad \beta) \times T^+$,
%     \cref{lemma:colour-tensor-product} tells us that, as a $G^\#$-graded algebra, $\D \iso \mc A \tensor \mc B$, where $\mc A$ is the graded-division algebra associated to $(\langle t_1 \rangle, \beta\restriction_{\langle t_1 \rangle \times \langle t_1 \rangle} )$ and $\mc B$ is the graded-division algebra associated to $(T^+, \beta^+)$. 
%     By \cref{ex:Q(1)-as-grd-div-SA}, we have $\mc A \iso Q(1)$ as a superalgebra. 
%     By \cref{cor:D-simple-iff-beta-nondeg}, $\mc B$ is a $G^\#$-graded matrix algebra, say $M_n(\FF)$. 
%     Hence, as a superalgebra, $\D \iso \mc A \tensor \B \iso Q(1) \tensor M_n(\FF) \iso Q(n)$, where the second isomorphism is given by the Kronecker product. 
% \end{proof}

% Under the conditions of \cref{prop:Q-T-beta-t_1}, we can write $t_1 = (h, \bar 1)$, where $h\in G$ is an element of order at most $2$. 

% Conversely, given a finite subgroup $T^+ \subseteq G$, a nondegenerate alternating bicharacter $\beta^+\from T^+ \times T^+ \to \FF^\times$ and an element $h\in G$ of order at most $2$, let $\D\even$ be a standard realization (see \cref{def:standard-realization}) of a matrix algebra with a division grading associated to $(T^+, \beta^+)$, set $t_1 \coloneqq (h, \bar 1)$ and $T \coloneqq \langle t_1 \rangle \times T^+$, and define $\beta\from T \times T \to \FF^\times$ by $\beta(s t_1^i, t t_1^j) \coloneqq \beta^+(s,t)$. 
% It is clear that $\beta$ is an alternating bicharacter and that $\rad \beta = \langle t_1 \rangle$. 

\begin{cor}\label{cor:tilde-beta-nondeg}
    The graded-division superalgebra $\D$ is simple as a superalgebra if, and only if, $(\rad \beta) \cap T^+ = \{e\}$. 
    More precisely, if this is the case, then
    \begin{enumerate}[(i)]
        \item $\D$ is a superalgebra of type $M$ if, and only if, $\beta$ is nondegenerate;
        \item $\D$ is a superalgebra of type $Q$ if, and only if, if $\beta$ is degenerate.
    \end{enumerate}
\end{cor}

\begin{proof}
    If $\D$ is simple as a superalgebra, then by \cref{thm:fd-simple-SA}, $\D \iso M(m,n)$ or $\D \iso Q(n)$ as superalgebra. 
    In both cases, $sZ(\D) = \FF$, so $\tilde\beta$ is nondegenerate.

    Conversely, suppose $\tilde\beta$ is nondegenerate. 
    If $\beta$ is nondegenerate, then $\D$ is simple as an algebra, by \cref{cor:D-simple-iff-beta-nondeg}.  
    If $\beta$ is degenerate, then combining  \cref{lemma:beta-deg-beta-tilde-nondeg} with \cref{lemma:colour-tensor-product}, we get that, as a $G^\#$-graded algebra, $\D \iso \mc A \tensor \mc B$, where $\mc A$ is the graded-division algebra associated to $(\rad \beta, \beta\restriction_{(\rad \beta)\times (\rad \beta)})$ and $\mc B$ is the graded-division algebra associated to $(T^+, \beta\restriction_{T^+ \times T^+})$. 
    
    By \cref{lemma:beta-deg-beta-tilde-nondeg}, it is easy to see that $\mc A \iso Q(1)$, where we declare the elements of $Q(1)\odd$ to have degree $t_1$. 
    By \cref{lemma:beta-deg-beta-tilde-nondeg} and \cref{cor:D-simple-iff-beta-nondeg}, $\mc B$ is a $G^\#$-graded matrix algebra, say $M_n(\FF)$, with $\supp \mc B = T^+$. 
    Hence, as a superalgebra, $\D \iso \mc A \tensor \B \iso Q(1) \tensor M_n(\FF) \iso Q(n)$, where the second isomorphism is given by the Kronecker product. 
\end{proof}

\begin{remark}\label{rmk:D-simple-iff-tilde-beta-nondeg}
    Clearly, if $\Char \FF \neq 2$, \cref{lemma:rad-tilde-beta,cor:tilde-beta-nondeg} implies that $\D$ is simple as a superalgebra if, and only if, $\tilde\beta $ is nondegenerate (compare with \cref{cor:D-simple-iff-beta-nondeg}). 
\end{remark}

Recall that, any nonzero $G^\#$-homogeneous element $d\in \D$ gives rise to the inner automorphism $\operatorname{Int}_d\from \D \to \D$ defined by $\operatorname{Int}_d (c) \coloneqq dcd\inv$, for all $c\in \D$ (Definition \ref{def:inner-automorphism}). 
We will now generalize this to graded superalgebras:

\begin{defi}\label{def:superinner}
	Let $d\in \D$ be a nonzero $G^\#$-homogeneous element.
	We define the \emph{superinner automorphism} $\operatorname{sInt}_d\from \D \to \D$ by $\operatorname{sInt}_d (c) \coloneqq \sign{c}{d} dcd\inv$, for all $c\in \D$.
\end{defi}

% The next result is a generalization of \cite[??]{livromicha}:

\begin{prop}\label{prop:all-central-automorphisms-of-D-are-superinner}  
    Let $\psi_0\from \D \to \D$ be an automorphism that restricts to the identity on $Z(\D) \cap \D\even$. 
    Then $\psi_0 = \operatorname{sInt}_{X_t}$ for some $t\in T$. 
\end{prop}

\begin{proof}
    By \cref{lemma:Aut(D)-widehat-T}, there is a character $\chi\in \widehat{T}$ such that $\psi_0(X_t) = \chi(t) X_t$, for all $t\in T$. 
    Since $\psi_0$ is the identity on $sZ(\D)$, we have that $\chi(t) = 1$ for all $t \in \rad \tilde\beta = (\rad\beta) \cap T^+ = \rad \tilde\beta$. 
    
    Set $\barr T \coloneqq T/\rad \tilde\beta$ and let $\pi \from T \to \barr T$ denote the quotient homomorphism. 
    It is clear that $b\from \barr T \times \barr T \to \FF^\times$ given by $b( \pi(s), \pi(t) ) \coloneqq \tilde\beta (s,t)$ is well-defined and that $b$ is a nondegenerate skew-symmetric bicharacter on $\barr T$. 
    Therefore the map $\barr T \to \widehat{\barr T}$ given by $t \mapsto b(t, \cdot)$ is a group isomorphism.
    
    Since $\chi$ is trivial on $\rad \tilde\beta$, there is $\barr \chi \in \widehat{\barr T}$ such that $\chi = \barr \chi \circ \pi$ and, since $b$ is nondegenerate, there is an element $t\in T$ such that $b(\pi(t), \cdot ) = \barr \chi$. 
    An straightforward computation shows that $\psi_0 = \operatorname{sInt}_{X_t}$.
\end{proof}

\begin{defi}\label{def:parity-element}
    We say that $t_p \in T$ is \emph{a parity element} if $\tilde\beta(t_p, t) = (-1)^{p(t)}$ for all $t\in T$. 
\end{defi}

Clearly, $t_p$ is a parity element if, and only if, $\operatorname{sInt}_{X_{t_p}} = \nu$, where $\nu\from \D \to \D$ is the parity automorphism $\nu(X_t) = (-1)^{p(t)} X_t$ for every $t\in T$. 
%
For \cref{ex:Q(1)-as-grd-div-SA}, the parity automorphism is given by $\operatorname{sInt}_u$, 
% where
% \[
%     u coloneqq \begin{pmatrix}
%             0 & 1\\
%             1 & 0
%         \end{pmatrix},
% \]
hence $\bar 1 = \deg u \in T^-$ is a parity element. 
For \cref{ex:Pauli-2x2-super}, the parity automorphism is given by $\operatorname{sInt}_d = \operatorname{Int}_d$ where
\[
    d \coloneqq \begin{pmatrix}
            1 & 0\\
            0 & -1
        \end{pmatrix},
\]
hence $\deg d = (\bar 1, \bar 0) \in T^+$ is a parity element. 

\begin{cor}\label{cor:existence-parity-element}
    There always exists a parity element $t_p \in T$, and 
    the set $T_p$ of all parity elements in $T$ is the coset $t_p (\rad \tilde\beta)$. 
    % In particular, the parity element is unique if, and only if, $\tilde\beta$ is nondegenerate. 
\end{cor}

\begin{proof}
    Existence follows from \cref{prop:all-central-automorphisms-of-D-are-superinner}, since the parity automorphism is trivial on $Z(\D) \cap \D\even$. 
    The second assertion is clear from the definition. 
\end{proof}

We can describe the set $T_p = t_p (\rad \tilde\beta)$ differently: 

\begin{lemma}\label{lemma:set-parity-elements}
    Suppose $\Char \FF \neq 2$. 
    If $T^- =\emptyset$ or $\Char \FF =2$, then the set of all parity elements is $\rad \tilde\beta = \rad \beta = \rad \beta^+$. 
    Otherwise, $T_p \cap T^+ = (\rad \beta^+) \smallsetminus (\rad\beta)$ and $T_p \cap T^- = (\rad \beta) \cap T^-$. 
\end{lemma}

\begin{proof}
    If $T^- =\emptyset$, then $(-1)^{p(t)} = 1$ for all $t\in T$, so $T_p = \rad \tilde\beta$ and $\tilde\beta = \beta = \beta^+$. 
    We will now suppose $T^- \neq\emptyset$.
    
    If $t_p$ is an even parity element, then clearly $t_p \in \rad\beta^+$ but $t_p \not\in \rad \beta$. 
    Conversely, if $t_p \in \rad \beta^+ \setminus \rad\beta$, let $t_1\in T^-$. 
    Since $\beta(t_p, T^+) = 1$, we have $\beta(t_p, T^-) = \beta(t_p, t_1 T^+) = \beta(t_p, t_1) \neq 1$. 
    Since $t_1^2 \in T^+$, $\beta(t_p, t_1)^2 = \beta(t_p, t_1^2) = 1$ and, hence, $\beta(t_p, t_1) = - 1$, proving that $t_p$ is a parity element. 
    
    Finally, an odd element $t_p$ is a parity element if, and only if, $\tilde\beta(t_p, t) = (-1)^{p(t)} = (-1)^{p(t_p) p(t)}$ for all $t\in T$ which, by the definition of $\tilde\beta$, is equivalent to $\beta(t_p, t) = 1$ for all $t\in T$. 
    The result follows. 
\end{proof} 

Note that, if $\Char \FF \neq 2$, then, by \cref{lemma:rad-tilde-beta}, all elements in $T_p$ have the same parity, \ie, either $T_p \cap T^+ = \emptyset$ or $T_p \cap T^- = \emptyset$. 

\begin{cor}\label{cor:radical-with-parity}
    Let $t_p\in T$ be a parity element. 
    If $t_p\in T^+$, then $\rad \beta = \rad \tilde\beta$ and $\rad \beta^+ = (\rad \tilde\beta) \cup  t_p(\rad \tilde\beta)$. 
    If $t_p\in T^-$, then $\rad \beta^+ = \rad \tilde\beta$ and $\rad \beta = (\rad \tilde\beta) \cup  t_p(\rad \tilde\beta)$. \qed
\end{cor}



% -----------------
\subsection{Finite dimensional graded-simple superalgebras over an algebraically closed field}\label{ssec:classification-assc-super}

We are now going to specialize the main results of \cref{ssec:param-End_D-U} to the superalgebra case, so we continue assuming that $\FF$ is algebraically closed and that $G$ is abelian. 
Recall from \cref{ssec:supermodules-over-D} that our classification a graded $\D$-modules of finite rank depend on $\D$ being even or odd.

\begin{defi}\label{def:E(D,U)-super}
    Let $\D$ be a finite dimensional graded-division superalgebra over an algebraically closed field $\FF$, associated to the triple $(T, \beta, p)$, and let $\U$ be a graded right $\D$-supermodule of finite rank. 
    We define the \emph{parameters} of the pair $(\D, \U)$ as:
    \begin{enumerate}[(i)]
        \item the quadruple $(T, \beta, \kappa_\bz, \kappa_\bo)$, if $\D$ is even and $\U$ is associated to the maps $\kappa_\bz, \kappa_\bo \from G/T \to \ZZ_{\geq 0}$;
        \item the quadruple $(T, \beta, p, \kappa)$, if $\D$ is odd and $\U$ is associated to the map $\kappa\from G/T^+ \to \ZZ_{\geq 0}$.
    \end{enumerate}
\end{defi}

Recall that for every map $\kappa\from G/T\to \ZZ_{\geq 0}$ with finite support, we define $|\kappa| \coloneqq \sum_{x \in G/T} \kappa(x)$. 
We, then, have that  \cref{thm:iso-End_D-U-with-parameters} translates the following result in the ``super'' case:

\begin{thm}\label{thm:iso-D-even}
	Let $(\D, \U)$ and $(\D', \U')$ be pairs as in Definition \ref{def:E(D,U)-super}, with both $\D$ and $\D'$ even. 
	Let $(T, \beta, \kappa_\bz, \kappa_\bo)$ and $(T', \beta', \kappa_\bz', \kappa_\bo')$ be the parameters of $(\D, \U)$ and $(\D', \U')$, respectively. 
	Then $\End_\D (\U) \iso \End_{\D'} (\U')$ if, and only if, $T=T'$, $\beta=\beta'$, $p = p'$ and there is $g\in G$ such that either $g \cdot \kappa_{\bar 0}=\kappa_{\bar 0}'$ and $g \cdot \kappa_{\bar 1}=\kappa_{\bar 1}'$, or $g \cdot \kappa_{\bar 0}=\kappa_{\bar 1}'$ and $g \cdot \kappa_{\bar 1}=\kappa_{\bar 0}'$. \qed
\end{thm}

\begin{thm}\label{thm:iso-D-odd}
    Let $(\D, \U)$ and $(\D', \U')$ be pairs as in Definition \ref{def:E(D,U)-super}, with both $\D$ and $\D'$ odd. 
    Let $(T, \beta, p, \kappa)$ and $(T', \beta', p', \kappa')$ be the parameters of $(\D, \U)$ and $(\D', \U')$, respectively. 
	Then $\End_\D (\U) \iso \End_{\D'} (\U')$ if, and only if, $T=T'$, $\beta=\beta'$, $p = p'$ and there is $g\in G$ such that $\kappa' = g \cdot \kappa$. \qed
\end{thm}

Gradings on simple associative superalgebras will be needed in \cref{chap:Lie} and are interesting in their own right. 
\Cref{prop:simple-R-D-super,rmk:D-simple-iff-tilde-beta-nondeg} together imply that $\End_\D (\U)$ is simple as a superalgebra if, and only if, $(\rad \beta) \cap T^+ = \{e\}$, which, in the case $\Char \FF \neq 2$, is equivalent to $\tilde\beta$ being nondegenerate (\cref{lemma:rad-tilde-beta}). 
If this is the case,  \cref{thm:iso-D-even,thm:iso-D-odd} give a classification of abelian group gradings on finite dimensional simple superalgebras, as follows. 
We note that, in the same way it as with  \cref{def:Gamma-T-beta-kappa}, there is an abuse of notation in \cref{def:Gamma-T-beta-kappa-even,def:Gamma-T-beta-kappa-odd,def:Gamma-T-beta-kappa-Q}. 

% (compare with \cite{paper-Qn}, \cite{paper-MAP} and \cite{Helens_thesis}). 

% ------------------------------------

We start with even gradings on superalgebras of type $M$:

\begin{defi}\label{def:Gamma-T-beta-kappa-even}
    Let $m, n\in \ZZ_{\geq 0}$, not both zero. 
    Given a finite subgroup $T \subseteq G$, a nondegenerate bicharacter $\beta\from T\times T \to \FF^\times$ and maps $\kappa_\bz, \kappa_\bo \from G/T \to \ZZ_{\geq 0}$ with finite support such that $|\kappa_\bz| \sqrt{|T|} = m$ and $|\kappa_\bo| \sqrt{|T|} = n$, consider:
    \begin{enumerate}[(i)]
        \item a standard realization $\D$ (see \cref{def:standard-realization}) of a matrix algebra with a division grading associated to $(T,\beta)$;
        \item the elementary grading (see \cref{defi:elementary-grd-super}) on $M(k_\bz, k_\bo)$ defined by $(\gamma_\bz, \gamma_\bo)$, where $\gamma_\bz$ and $\gamma_\bo$ are a $k_\bz$-tuple and a $k_\bo$-tuple of elements of $G$ realizing $\kappa_\bz$ and $\kappa_\bo$, respectively, where $k_\bz \coloneqq |\kappa_\bz|$ and $k_\bo \coloneqq |\kappa_\bo|$  (see \cref{defi:gamma-realizes-kappa}).
    \end{enumerate}
    %
    We define $\Gamma_M (T, \beta, \kappa_\bz, \kappa_\bo)$ to be the even grading on $M(m,n)$ given by identifying $M(m,n)$ with the graded superalgebra $M(k_\bz, k_\bo) \tensor \D$ via Kronecker product, \ie,
    \[
        \deg \left( E_{ij} \tensor d \right) = g_ig_j\inv t,
    \] 
    for all $1\leq i, j \leq k_\bz + k_\bo$, $t\in T$ and $0 \neq d \in \D_t$. 
    We will denote the superalgebra $M(m,n)$ endowed with $\Gamma_M (T, \beta, \kappa_\bz, \kappa_\bo)$ by $M(T, \beta, \kappa_\bz, \kappa_\bo)$. 
\end{defi}

\begin{cor}\label{cor:iso-M-even}
    Every even $G$-grading on $M(m,n)$ is isomorphic to $\Gamma_M (T, \beta, \kappa_\bz, \kappa_\bo)$ as in \cref{def:Gamma-T-beta-kappa-even}. 
    Two such gradings $\Gamma_M (T, \beta, \kappa_\bz, \kappa_\bo)$ and $\Gamma_M (T', \beta', \kappa_\bz', \kappa_\bo')$ are isomorphic if, and only if, $T = T'$, $\beta = \beta'$ and there is $g\in G$ such that either $g \cdot \kappa_{\bar 0}=\kappa_{\bar 0}'$ and $g \cdot \kappa_{\bar 1}=\kappa_{\bar 1}'$, or $g \cdot \kappa_{\bar 0}=\kappa_{\bar 1}'$ and $g \cdot \kappa_{\bar 1}=\kappa_{\bar 0}'$.  \qed
\end{cor}

Note that if $m \neq n$, 
then $k_\bz \neq k_\bo$ and, hence, only the case $g \cdot \kappa_{\bar 0}=\kappa_{\bar 0}'$ and $g \cdot \kappa_{\bar 1}=\kappa_{\bar 1}'$ is possible.

% -----

Now let us consider the odd gradings on $M(m,n)$. 
Note that in this case, by \cref{lemma:odd-M-m=n}, we have that $m = n$. 

% Our result will be in terms of the group $G^\#$.

\begin{defi}\label{def:Gamma-T-beta-kappa-odd}
    Let $n > 0$ be a natural number. 
    Given a finite subgroup $T \subseteq G^\#$, $T\not\subseteq G$, a nondegenerate bicharacter $\beta\from T\times T \to \FF^\times$ and a map $\kappa\from G/T^+ \to \ZZ_{\geq 0}$ with finite support such that $|\kappa| \sqrt{|T|} = n$, consider 
    \begin{enumerate}[(i)]
        % \item the group homomorphism $p\from T \to \ZZ_2$ given by the projection on the second component of $G^\# = G\times \ZZ_2$;
        \item a standard realization $\D$ (see \cref{def:standard-realization}) of a matrix algebra with a division grading associated to $(T,\beta)$, viewed as a $G$-graded superalgebra with canonical $\ZZ_2$-grading given by the projection on the second entry $p\from T \subseteq G^\# \to \ZZ_2$;
        \item the elementary grading (see \cref{defi:elementary-grd}) on $M_{k}(\FF)$ defined by a $k$-tuple $\gamma$ of elements of $G$ realizing $\kappa$, where $k \coloneqq |\kappa|$ (see \cref{defi:gamma-realizes-kappa}). 
    \end{enumerate}
    %
    We define $\Gamma_M (T, \beta, p, \kappa)$ to be the odd grading on $M(n,n)$ given by identifying $M(n,n)$ with the graded superalgebra $M_k(\FF) \tensor \D$ via Kronecker product.  
    We will denote the superalgebra $M(n,n)$ endowed with $\Gamma_M (T, \beta, p, \kappa)$ by $M (T, \beta, p, \kappa)$. 
\end{defi}

\begin{cor}\label{cor:iso-M-odd}
    Every odd $G$-grading on $M_n(\FF)$ is isomorphic to $\Gamma_M (T, \beta, p, \kappa)$ as in \cref{def:Gamma-T-beta-kappa}. 
    Two such gradings $\Gamma_M (T, \beta, p, \kappa)$ and $\Gamma_M (T', \beta', p, \kappa')$ are isomorphic if, and only if, $T = T'$, $\beta = \beta'$, $p = p'$ and there is a $g\in G$ such that $g\cdot \kappa = \kappa'$. \qed
\end{cor}

% ----------------------------------

Finally, we classify the gradings on the superalgebra $Q(n)$. 
Note that we only have odd gradings in this case. 


\begin{defi}\label{def:Gamma-T-beta-kappa-Q}
    Let $n > 0$ be a natural number. 
    Given a finite subgroup $T^+ \subseteq G$, a nondegenerate bicharacter $\beta\from T^+ \times T^+ \to \FF^\times$, an element $h\in G$ such that $h^2 = 1$ and a map $\kappa\from G/T^+ \to \ZZ_{\geq 0}$ with finite support such that $|\kappa| \sqrt{|T^+|} = n$, consider 
    \begin{enumerate}[(i)]
        \item a standard realization $\D$ (see \cref{def:standard-realization-Q}) of a superalgebra of type $Q$ with a division grading associated to $(T^+, \beta^+, h)$;
        \item the elementary grading (see \cref{defi:elementary-grd}) on $M_{k}(\FF)$ defined by a $k$-tuple $\gamma$ of elements of $G$  of elements of $G$ realizing $\kappa$, where $k \coloneqq |\kappa|$ (see \cref{defi:gamma-realizes-kappa}).  
    \end{enumerate}
    %
    We define $\Gamma_Q (T^+, \beta^+, h, \kappa)$ to be the grading on $Q(n)$ given by identifying $Q(n)$ with the graded superalgebra $M_k(\FF) \tensor \D$ via Kronecker product. 
    We will denote the superalgebra $Q(n)$ endowed with $\Gamma_Q (T^+, \beta^+, h, \kappa)$ by $Q (T^+, \beta^+, h, \kappa)$. 
\end{defi}

\begin{cor}\label{cor:iso-Q}
    Every $G$-grading on $Q(n)$ is isomorphic to $\Gamma_Q (T^+, \beta^+, h, \kappa)$ as in \cref{def:Gamma-T-beta-kappa}. 
    Two such gradings $\Gamma_Q (T^+, \beta^+, h, \kappa)$ and $\Gamma_Q (T'^+, \beta'^+, h', \kappa')$ are isomorphic if, and only if, $T^+ = T'^+$, $\beta^+ = \beta'^+$, $h = h'$ and there is a $g\in G$ such that $g\cdot \kappa = \kappa'$. \qed
\end{cor}

\section{Odd gradings on simple associative superalgebras only in terms of \texorpdfstring{$G$}{G}}\label{sec:assc-only-G}

Here $\FF$ is algebraically closed and $\D$ is finite dimensional.

\subsection{Odd graded-simple superalgebras in terms of $G$}\label{ssec:odd-div-G-only}

Let $\D$ be an odd finite dimensional graded-division superalgebra associated to $(T, \beta, p)$. 
Clearly, $\D\even$ is a graded-division algebra associated to $(T^+, \beta^+)$. 
Given any $t_1 \in T^-$, then $t_1 = (h, \bar 1)$ for some $h\in G$ such that $h^2\in T^+$ and the map $\chi \from T^+ \to \FF^\times$ given by $\chi(t) \coloneqq \beta(t_1, t)$, for all $t\in T^+$, is a character in $\widehat{T^+}$ such that $\chi(h^2) = 1$ and $\chi^2 = \beta^+ (h^2, \cdot)$. 

\begin{defi}\label{def:O(T+-beta+)}
    We say that the odd graded-division superalgebra $\D$ is a \emph{associated} to the quadruple $(T^+, \beta^+, h,\chi)$. 
    If $\U$ is a graded $\D$-supermodule of finite rank associated to the map $\kappa\from G/T^+ \to \FF^\times$, we say that $(T^+, \beta^+, h, \chi, \kappa)$ are $G$-parameters of the pair $(\D, \U)$. 
\end{defi}

Clearly, a quadruple $(T^+, \beta^+, h,\chi)$ associated to $\D$ has enough information to recover $T$ and $\beta$, since $T^- = (h, \bar 1) T^+$ and
\[\label{eq:beta-from-h-chi}
    \forall s,t\in T^+,\, \forall i,j \in \ZZ, \quad \beta(s\, t_1^i,t \, t_1^j) = \beta^+(s,t)\, \chi (s)^{-j}\, \chi (t)^i,
\]
and, therefore, determines the isomorphism class of the graded-division superalgebra $\D$ (see \cref{ssec:T-beta-p}). 

\begin{defi}\label{def:odd-parameters}
    Let $T^+$ be any finite subgroup of $G$ and let $\beta^+$ be an alternating bicharacter on $T^+$. 
    A pair $(h,\chi) \in G \times \widehat{T^+}$ is said to be \emph{$(T^+, \beta^+)$-admissible} if 
    $h^2 \in T^+$, $\chi(h^2) = 1$ and $\chi^2 = \beta^+(h^2, \cdot)$. 
    We will denote the set of all $(T^+, \beta^+)$-admissible pairs by $\mathbf {O} (T^+, \beta^+)$. 
\end{defi}

For each $(T^+, \beta^+)$-admissible pair $(h, \chi)$ we can construct a triple $(T, \beta, p)$ corresponding to a odd graded-division superalgebra. 
First, we set $t_1 \coloneqq (h, \bar 1)$ and define $T \coloneqq T^+ \cup t_1 T^-$. 
Note that $T$ is a subgroup of $G^\#$ since $h^2 \in T^+$. 
We then take $\beta\from T\times T \to \FF^\times$ as in \cref{lemma:existence-beta}, below:

\begin{lemma}\label{lemma:existence-beta}
    There is a unique alternating bicharacter $\beta$ on $T$ such that $\beta\restriction_{T^+ \times T^+} = \beta^+$ and $\beta(t_1, t) = \chi(t)$, for all $t\in T^+$.
\end{lemma}

\begin{proof}
    Clearly, a bicharacter $\beta$ satisfies the conditions above if, and only if, \cref{eq:beta-from-h-chi} is satisfied. 
    It follows that there is at most one such bicharacter. 
    
    % One could also use \cref{eq:beta-from-h-chi} as a definition for $\beta$, but it is not clear if it is well defined. 
    
    Let $\widetilde T$ be the direct product of $T^+$ and the infinite cyclic group generated by a new symbol $\tau$, and let $\phi \from \widetilde T \to T$ be the unique group homomorphism such that $\phi(t) = t$ for all $t\in T^+$ and $\phi(\tau) = t_1$. 
    It is easy to see that $\ker \phi = \langle h^2 \tau^{-2} \rangle$ and that $\phi$ is surjective. 
    
    Let us define $b\from \widetilde T\times \widetilde T \rightarrow \FF^\times$ by $b(s\tau^i,t\tau^j) \coloneqq \beta^+(s,t)\, \chi (s)^{-j}\, \chi (t)^i$, for all $s,t\in T^+$ and $i,j \in \ZZ$. 
    Clearly, $b$ is an alternating bicharacter. 
    We claim that $h^2 \tau^{-2} \in \rad b$. 
    Indeed, for all $t \in T^+$ and $j \in \ZZ$,
    \[%begin{align}
        b(h^2 \tau^{-2},t\tau^j) = \beta^+(h^2,t)\, \chi (h^2)^{-j}\, \chi (t)^{-2} = \chi (t)^{2}\, \chi (h^2)^{-j}\, \chi (t)^{-2} = \chi (h^2)^{-j} = 1.
    \]%end{align}
    
    It follows that $\ker \phi \subseteq \rad b$ and, hence, $b$ induces an alternating bicharacter $\beta\from T\times T \to \FF$ such that $b = \beta \circ (\phi \times \phi)$. 
    It is clear that $\beta$ satisfies \cref{eq:beta-from-h-chi}, concluding the proof.
\end{proof}

Together with \cref{lemma:colour-tensor-product}, we have:

\begin{prop}\label{prop:D-defined-by-quadruple}
    Let $T^+$ and $\beta^+$ be as in \cref{def:odd-parameters} and let $(h, \chi)$ be a $(T^+, \beta^+)$-admissible pair. 
    Then there exist an odd graded-division superalgebra $\D$ associated to the quadruple $(T^+, \beta^+, h,\chi)$. \qed
\end{prop}

We now have a parametrization of odd graded-division superalgebras only in terms of $G$. 
It remains to see when different quadruples determine isomorphic graded-division superalgebras. 

\begin{defi}\label{def:T^+-action}
    Let $T^+$ and $\beta^+$ be as in \cref{def:odd-parameters}, we define an $T^+$-action on $\mathbf{O} (T^+, \beta^+)$ by
    \[
        t \cdot (h, \chi) = (th, \beta^+(t, \cdot) \chi),
    \]
    for all $t\in T^+$ and $(h, \chi) \in \mathbf{O} (T^+, \beta^+)$.
\end{defi}

If $t\in T^+$ and $(h, \chi) \in \mathbf {O} (T^+, \beta^+)$, then $(th)^2 = t^2h^2\in T^+$ and $\beta(th^2, \cdot) = \beta(t, \cdot) \chi$, hence, indeed, $t \cdot (h, \chi) \in \mathbf {O} (T^+, \beta^+)$. 
It is straightforward that the axioms of action are satisfied. 

\begin{thm}\label{thm:iso-odd-D-only-G}
    Let $\D$ and $\D'$ be finite dimensional odd graded-division superalgebras, associated to quadruples $(T^+, \beta^+, h, \chi)$ and $(T'^+, \beta'^+, h', \chi')$, respectively. 
    Then $\D \iso \D'$ if, and only if, $T^+ = T'^+$, $\beta^+ = \beta'^+$, and $(h', \chi')$ is in the same $T^+$-orbit of $(h, \chi)$ in $\mathbf {O} (T^+, \beta^+)$. 
\end{thm}

\begin{proof}
    Let $(T, \beta, p)$ and $(T', \beta', p')$ be the triples associated to $\D$ and $\D'$, respectively. 
    By other theorem, $\D \iso \D'$ if, and only if, $T= T'$ and $\beta = \beta'$. 
    
    By definition, $T= T'$ if, and only if, $T^+ = T'^+$ and $(h, \bar 1)T^+ = (h', \bar 1)T^+$. 
    Note that the latter condition is equivalent to 
    % Hence $T= T'$ if, and only if, $T^+ = T'^+$ and 
    $h' = th$ for some $t\in T^+$. 
    
    Now suppose $T= T'$ and let $t\in T^+$ be such that $h' = th$. 
    Clearly, $\beta = \beta'$ if, and only if, 
    $\beta^+ = \beta'^+$ and $\beta((h', \bar 1), \cdot) = \beta'((h', \bar 1), \cdot)$. 
    By definition of $\beta'$, the latter condition is equivalent to $\chi' = \beta((h', \bar 1), \cdot)$. 
    Since $\beta((h', \bar 1), \cdot) = \beta ((th, \bar), \cdot) = \beta^+ (t, \cdot) \beta ((h, \bar 1), \cdot) = \chi$, we have the desired result. 
\end{proof}

\begin{cor}
    Iso for any odd grading.
\end{cor}

For specific cases, this parametrization can be simplified using the following lemma about group actions: 

\begin{lemma}\label{lemma:lemma-on-actions}
    Let $H$ be a group, let $X$ and $Y$ be $H$-sets, and let $\pi\from X \to Y$ be an $H$-equivariant map. 
    For every $y \in Y$, the $H$-action on $X$ restricts to a $\operatorname{Stab}_H (y)$-action on $\pi\inv(y)$ and, if the $H$-action on $Y$ is transitive, the inclusion map $\pi\inv(y) \hookrightarrow X$ induces a bijection between $\operatorname{Stab}_H (y)$-orbits in $\pi\inv(y)$ and $H$-orbits in $X$.  
\end{lemma}

\begin{proof}
    The first assertion is straightforward: if $x\in \pi\inv(y)$ and $g \in \operatorname{Stab}_H (y)$, then $g \cdot x \in \pi\inv(y)$. 
    
    Now suppose that the $H$-action on $Y$ is transitive. 
    It is clearly that the $\operatorname{Stab}_H (y)$-orbit of any $x\in \pi\in(y)$ is contained in the $H$-orbit of $x$. Conversely, given $x\in X$, by transitivity, there is $g\in H$ such that $g \cdot \pi(x) = y$ and, hence, $g\cdot x \in \pi\inv (y)$. 
    We claim that $\operatorname{Stab}_H (y)$-orbit of $g\cdot x$ is the only contained in the $H$-orbit of $x$. 
    % We claim the $\operatorname{Stab}_H (y)$-orbit of $g\cdot x$ is the only $\operatorname{Stab}_H (y)$-orbit contained in the $H$-orbit of $x$. 
    Indeed, if for some $g'\in H$ we have $g'\cdot x \in \pi\inv(y)$, then $g'\cdot x = (g'g\inv)\cdot (g\cdot x)$ and, hence, $y = \pi(g'\cdot x) = (g'g\inv)\cdot \pi(g\cdot x) = (g'g\inv) \cdot y$, so $g'g\inv \in \operatorname{Stab}_H (y)$.
\end{proof}

As an application of this lemma, we will revisit the classification of graded-division superalgebras of type $Q$. 

fix $(T^+, \beta^+)$ with $\beta^+$ nondegenerate. 
Consider the action of $T^+$ on $\widehat{T^+}$ given by $t\cdot \chi = \beta^+(t,\cdot) \chi$ and let $\pi\from \mathbf {O} (T^+, \beta^+) \to \widehat{T^+}$ be the projection on the second entry, which clearly a $\widehat{T^+}$-equivariant map. 

Since $\beta^+$ is nondegenerate, the $T^+$-action on $\widehat{T^+}$ is transitive. 
Indeed, for every $\chi \in \widehat{T^+}$, there is $t \in T^+$ such that $\chi = \beta(t, \cdot)$, so $\chi = t \cdot 1$, where $1\in \widehat{T^+}$ is the trivial character. 

The nondegeneracy of $\beta$ also implies that $\operatorname{Stab}_{T^+} (1) = \{e \}$. 
By \cref{lemma:lemma-on-actions}, it follows that the $T^+$-orbits in $\mathbf {O} (T^+, \beta^+)$ are in bijection with points in $\pi\inv (1)$. 
By the definition of $\mathbf {O} (T^+, \beta^+)$, $(h, 1) \in G \times \widehat{T^+}$ if, and only if, $h^2 \in T^+$ and $h^2 \in \ker \beta^+$, hence, if, and only if, $h^2 = e$. 

Our theorem reduces to ... and $h = h'$. 
Note that in this case $(h, \bar 1) \in \ker \beta$, so in view of \cref{lemma:beta-deg-beta-tilde-nondeg}, $\ker \beta = \langle (h, \bar 1) \rangle$ and $\D$ is isomorphic to the standard realization of ... (see \cref{def:standard-realization-Q}).

\subsection{Odd graded-division superalgebras of type \texorpdfstring{$M$}{M} in terms of \texorpdfstring{$G$}{M}}

Let $\D$ be an odd finite dimensional graded-division superalgebra associated to a triple $(T, \beta, p)$. 
By \cref{prop:char-does-not-divide-T}, below, if $\Char \FF = 2$, then $\D$ cannot be of type $M$. 
Note that, since $\D$ is odd, $|T| = |T^+| + |T^-| = 2|T^+|$.

\begin{prop}\label{prop:char-does-not-divide-T}
    Suppose $\Char \FF = p > 0$. 
    If $p \mid |T|$, then $\beta$ is degenerate.
\end{prop}

\begin{proof}
    Let $t\in T$ be an element of order $p$. 
    For every $s\in T$, we have $1 = \beta(e, s) = \beta(t^p, s) = \beta(t, s)^p$, hence, since $\Char \FF = p$, $\beta(t, s) = 1$, \ie, $t\in \rad \beta$. 
\end{proof}
 
For the remaining of this subsection, we will assume that $\Char \FF \neq 2$.

Suppose that $\D$ is of type $M$, \ie, that $\beta$ is nondegenerate. 
Propositions ?? establishes

We will now investigate necessary conditions on the pair $(T^+, \beta^+)$ for $\beta$ be nondegenerate. 
% We need to see what are the characteristics on the parameters of the general case that ensures $M(n,n)$.

% First, necessary conditions.

\begin{prop}\label{prop:parity-element}
    There is an element $t_0 \in T^+$ of order $2$ such that $\rad \beta^+ = \langle t_0 \rangle$.
\end{prop}

\begin{proof}
    Consider $\chi \in \widehat{T}$ given by $\chi(t) = (-1)^{p(t)}$. 
    Since $\beta$ is nondegenerate, there is $t_0\in T$ of order $2$ such that $\chi = \beta(t_0, \cdot)$. 
    Since $\beta(t_0, t_0) = 1 = (-1)^{p(t_0)}$, $t_0$ is even. 
    Hence $t_0 \in \rad \beta^+$. 
    
    If $0 \neq t \in \rad \beta^+$, using again that $\beta$ is nondegenerate, there is $t_1 \in T^-$ such that $\beta(t, t_1) \neq 1$. 
    Since $t_1^2 \in T^+$, $\beta(t_0, t_1)^2 = \beta(t, t_1^2) = 1$. 
    It follows that $\beta(t, t_1) = -1$, and since $T^- = t_1 T^+$, $\beta(t, s) = -1$ for all $s\in T^-$. 
    Hence $\beta(t, s) = (-1)^{p(s)}$ and, therefore, $t = t_0$.
\end{proof}

\begin{remark}\label{rmk:t_0-is-parity}
    Note that $t_0$ is the parity element (see ??).
\end{remark}

% \begin{prop}
%     Let $\Gamma$ be the $T$-grading on $\D\iso M(n,n)$, let $H$ be an abelian group and let $\alpha\from T \to H$ be a group homomorphism. 
%     The grading ${}^\alpha\Gamma$ (see def) is even grading if, and only if, $t_0 \in \ker \alpha$. 
% \end{prop}

% \begin{proof}
%     Fix an element $0\neq d_0\in \D$  of degree $t_0$. 
%     Since $t_0 \in \rad \beta$, $d_0 \in Z(\D\even)$, and 
%     since $d_0^2\in \D_e = \FF$, and $\FF$ is algebraically closed, we may rescale $d_0$ so that $d_0^2=1$. 
%     Then $\epsilon := \frac{1}{2}(1+d_0)$ is a central idempotent of $\D\even$. 
%     % = M(n,n)\even \iso M(n) \oplus M(n)$. 
%     Of course, $1-\epsilon$ is another central idempotent of $\D\even$, and $\D\even = \epsilon \D\even \oplus (1-\epsilon)\D\even$ (direct sum of ideals).
    
%     Let $0\neq d_1\in\D\odd$ be a $T$-homogeneous element. 
%     Then $d_1\epsilon d_1\inv = \frac{1}{2}(1-d_0)=1-\epsilon$, which is another central idempotent of $\D\even$ and must have the same rank as $\epsilon$. 
%     Hence, $\D\even \iso \epsilon\D\even \oplus (1-\epsilon)\D\even$ (direct sum of ideals) and, consequently, $E\even \iso \End(\tilde U)\tensor \D\even = \End(\tilde U)\tensor \epsilon\D\even \oplus \End(\tilde U)\tensor (1-\epsilon)\D\even$, where the two summands have the same dimension. Therefore, odd gradings exist only if $m=n$. 
% Also note that we have 
% \begin{equation}\label{eq:D1eps}
% \D\odd \epsilon = (1-\epsilon) \D\odd.
% \end{equation}
% \end{proof}

% -- Existence, necessary condition --

Not every pair $(T^+, \beta^+)$ with $|\rad \beta^+| = 2$ is restriction of $(T, \beta, p)$ with $\beta$ nondegenerate (see \cref{ex:T+-but-no-T}). 
In vague terms, this is because $(T^+, \beta^+)$ carries some information restricting which elements can be squares of an element $t_1 \in T^-$ (see \cref{prop:square-subgroup}). 

\begin{defi}\label{defi:perp}
    Let $H$ be an abelian group and $b$ be a symmetric or skew-symmetric bicharacter on $H$. 
    For every subgroup $A \subseteq H$, we define the \emph{orthogonal complement to $A$ with respect to $b$} to be the subgroup
    \[
        A^\perp \coloneqq \{ h\in H \mid b(h, A) = 1 \}.
    \]
\end{defi}

\begin{lemma}\label{lemma:perp-perp}
    Let $H$ and $b$ be as in \cref{defi:perp}, and suppose that $H$ is finite and $b$ is nondegenerate. 
    Then for every subgroup $A \subseteq H$, we have that $|A^\perp| = [H : A]$ and $(A^\perp)^\perp = A$.
\end{lemma}

\begin{proof}
    It is easy to see that the map $A^\perp \to \widehat{\left( \frac{H}{A}\right)}$ given by $t \mapsto \beta(t, \cdot)$ is an isomorphism of groups. 
    Also, since $\beta$ is nondegenerate, by \cref{prop:char-does-not-divide-T},  $\Char \FF$ does not divide $[H : A]$. 
    In this case, it is well-known that $\widehat{\left( \frac{H}{A}\right)} \iso \frac{H}{A}$, proving the first assertion.

    Since $b$ is symmetric or skew-symmetric, it is clear that $A \subseteq (A^\perp)^\perp$. 
    Using the first assertion, we have that $|A| = |(A^\perp)^\perp|$ and, therefore, $A = (A^\perp)^\perp$.
\end{proof}

\begin{defi}
    Let $A$ be an abelian group and $m \in \ZZ$. 
    We define $A_{[m]} \coloneqq \{a^m \mid a\in A\}$ and $A_{[m]} \coloneqq \{a\in A \mid a^m = e \}$.
\end{defi}

Note that $(T^+)^{[2]} \subseteq T^{[2]} \subseteq T^+$, but we can have $(T^+)^{[2]} \neq T^{[2]}$. 

\begin{lemma}\label{lemma:small-perp}
    Let $H$ and $b$ be as in \cref{defi:perp}, and suppose that $H$ is finite and $b$ is nondegenerate.
    Then, for every $m\in \ZZ$, $(H_{[m]})^\perp = H^{[m]}$.
\end{lemma}

\begin{proof}
    By \cref{lemma:perp-perp}, it suffices to prove that $(H^{[m]})^\perp = H_{[m]}$:
    %
	\begin{align}
		(H^{[m]})^\perp & = \{ h \in H \mid b (h, g^m) = 1 \text{ for all } g\in H \}\\ 
		& = \{ h \in H \mid b (h^m, g) = 1 \text{ for all } g\in H \}\\ 
		& = \{ h \in H \mid h^m = e \} \numberthis\label{4th-line-again} \\ 
		&= H_{[m]},
	\end{align}
	%
	where we are using that $b$ is nondegenerate in the line \eqref{4th-line-again}. 
\end{proof}

We now set $\barr G \coloneqq G/\langle t_0 \rangle$ and let $\theta\from G\rightarrow \barr G$ be the natural homomorphism. 
Also set $\barr {T^+} \coloneqq \theta(T^+)$ and let $\bar\beta^+$ be the bicharacter on $\barr T^+$ induced by $\beta^+$. 
Note that $\bar\beta^+$ is nondegenerate.

The next result is similar to \cref{lemma:small-perp}, but does not follow from it.

\begin{prop}\label{prop:square-subgroup}
	Consider $\theta(T^{[2]})$ and $\theta(T^+_{[2]})$ as subgroups of $\barr {T^+}$. 
	We have that $\theta(T^{[2]})$ is the orthogonal complement of $\theta(T^+_{[2]})$ with respect to $\barr{\beta^+}$, \ie, $ \theta(T^+_{[2]})^\perp = \theta(T^{[2]})$. 
	In particular, $\theta(T^{[2]})^\perp \subseteq \barr G^{[2]}$.
\end{prop}

\begin{proof}
	By \cref{lemma:perp-perp}, it suffices to prove that $\theta(T^{[2]})^\perp = \theta(T^+_{[2]})$:
	%
	\begin{align}
		\theta(T^{[2]})^\perp & = \{ \theta(t) \mid t\in T^+ \AND \barr{\beta^+} (\theta (t), \theta (s^2)) = 1 \text{ for all }s\in T\}\\ 
		& = \{\theta(t) \mid t\in T^+ \AND \beta^+ (t, s^2) = 1 \text{ for all }s\in T\}\\ 
		& = \{ \theta(t) \mid t\in T^+ \AND \beta (t^2, s) =1 \text{ for all }s\in T\}\\ 
		& = \{ \theta(t) \mid t\in T^+ \AND t^2=e \} \numberthis\label{4th-line} \\ 
		&= \theta(T^+_{[2]}),
	\end{align}
	%
	where we are using that $\beta$ is nondegenerate in the line \eqref{4th-line}. 
\end{proof}

\begin{example}\label{ex:T+-but-no-T}
    Take $G \coloneqq T^+ \coloneqq \ZZ_2 \times \ZZ_4$ and set $\beta^+((i, j),(i', j')) \coloneqq (-1)^{ij' - i'j}$, for all $i,j, i',j' \in \ZZ$. 
    It is easy to see that $\rad \beta^+ = \langle (\bar 0, \bar 2) \rangle$. 
    Clearly,
    $T^+_{[2]} = \langle (\bar 1, \bar 0) , (\bar 0, \bar 2) \rangle$ and, hence, $\theta(T^+_{[2]}) = \langle \theta(\bar 1, \bar 0) \rangle$. 
    Since $\theta(T^+_{[2]})$ is cyclic, $\theta(\bar 1, \bar 0) \in \theta(T^+_{[2]})^\perp$. 
    But $\theta(\bar 1, \bar 0)$ is not a square. 
\end{example}

% ------------------------------------

We are now in position to restrict the results of the previous section to our specific case. 
Let $T^+$ be a finite subgroup of $G$ and $\beta^+$ be an alternating bicharacter on $T^+$ such that $\rad \beta^+ = \langle t_0 \rangle$, where $t_0\in T^+$ is an element of order $2$. 
Also, consider $\barr G$, $\theta$, $\barr {T^+}$ and $\bar{\beta^+}$ as defined before. 

\begin{lemma}\label{lemma:motivation-O_M}
    Let $(h, \chi) \in \mathbf{O}(T^+, \beta^+)$ and let $\D$ be a graded-division superalgebra associated to $(T^+, \beta^+, h, \chi)$. 
    Then $\D$ is of type $M$ if, and only if, $\chi(t_0) = -1$.
\end{lemma}

\begin{proof}
    Let $(T, \beta, p)$ be the triple constructed from $(T^+, \beta^+, h, \chi)$ and let $t_1 = (h, \bar 1)$. 
    
    If $\beta$ is nondegenerate, then $\chi(t_0) = \beta(t_1, t_0) = -1$ by \cref{rmk:t_0-is-parity}. 
    
    Conversely, assume $\chi(t_0) = -1$. 
    Then $\beta(t_0, T^-) = \beta(t_0, t_1T^+) = \beta (t_0, t_1) = \chi(t_0)\inv = -1$. 
    We conclude that $(\rad \beta)\cap T^- = \emptyset$ and that $t_0 \not\in \rad\beta$. 
    Also, it is clear that $(\rad \beta)\cap T^+ \subseteq \rad \beta^+ = \langle t_0 \rangle$, hence $(\rad \beta)\cap T^+ = \{e\}$ and, therefore, $\rad \beta = \{e\}$.
\end{proof}

\begin{defi}\label{def:O_M}
    We define $\mathbf{O_M}(T^+, \beta^+) \coloneqq \{ (h, \chi) \in \mathbf{O}(T^+, \beta^+) \mid \chi(t_0) = -1 \}$.
\end{defi}

% Combining \cref{thm:iso-odd-D-only-G,lemma:motivation-O_M}, we have:

% \begin{cor}\label{iso-odd-M-no-existence}
%     Let $\D$ and $\D'$ be finite dimensional odd graded-division superalgebras, associated to quadruples $(T^+, \beta^+, h, \chi)$ and $(T'^+, \beta'^+, h', \chi')$, respectively. 
%     Then $\D \iso \D'$ if, and only if, $T^+ = T'^+$, $\beta^+ = \beta'^+$, and $(h', \chi')$ is in the same $T^+$-orbit of $(h, \chi)$ in $\mathbf {O} (T^+, \beta^+)$.  \qed
% \end{cor}

We note that the $T^+$-action on $\mathbf{O}(T^+, \beta^+)$ restricts to $\mathbf{O_M}(T^+, \beta^+)$ since $t_0\in \rad \beta^+$. 
In view of \cref{lemma:lemma-on-actions}, we can simplify the description of the orbits of this action. 

Set $Y \coloneqq \{\chi \in \widehat{T^+} \mid \chi (t_0)\}$ with $T^+$-action defined by $t\cdot \chi = \beta^+(t, \cdot) \chi$. 
Again, this action is well-defined since $t_0 \in \rad \beta^+$. 
We also let $\pi\from \mathbf{O_M}(T^+, \beta^+) \to Y$ be the $T^+$-equivariant map given by $(h, \chi) \mapsto \chi$. 

\begin{lemma}
    We have that $Y \neq \emptyset$ and that the $T^+$-action on $Y$ is transitive.
\end{lemma}

\begin{proof}
    To see that $Y$ is no empty, note that $\beta^+$ is a nondegenerate bicharacter on $\barr T^+$ and that $|T^+| = 2 |\barr T^+|$. 
    By \cref{prop:char-does-not-divide-T}, $\Char \FF$ does not divide $|\barr T^+|$ and, since $\Char \FF \neq 2$, $\Char \FF$ does not divide $|T^+|$. 
    In this case, it is well-known that every character on a subgroup of $|T^+|$ can be extended to a character of $T^+$. 
    Clearly, $\chi(t_0) = -1$ defines a character on $\langle t_0 \rangle$, therefore $Y \neq \emptyset$. 
    
    To see that the action is transitive, let $\chi, \chi' \in Y$. 
    We have $(\chi' \chi\inv) (t_0) = 1$, hence $(\chi' \chi\inv)$ induces a character on $\barr T^+ = T^+/ \langle t_0 \rangle$. 
    By nondegeneracy of $\barr \beta^+$, there is $t\in T^+$ such that $(\chi' \chi\inv) = \beta(\bar t, \cdot)$ as a character of $\barr T^+$. 
    It follows that $(\chi' \chi\inv) = \beta(t, \cdot)$ as a character of $T^+$, so $\chi' = \beta(t, \cdot)\chi$. 
    We conclude that $\chi' = t \cdot \chi$, as desired. 
\end{proof}

Recall that in the case of $\beta^+$ nondegenerate, in \cref{ssec:odd-div-G-only}, we were able to fix the character to be the trivial one. 
In the present case, we can also fix the character $\chi \in Y$, but there is no canonical choice for it. 

\begin{lemma}\label{lemma:chi-defines-a}
    Given $\chi \in Y$, there is a unique element $a \in T^+$ such that $\chi^2 = \beta^+(a, \cdot)$ and $\chi(a) = 1$. 
    Further, a element $h\in T^+$ is such that $(h, \chi) \in \mathbf{O_M}(T^+, \beta^+)$ if, and only if, $h^2 = a$. 
\end{lemma}

\begin{proof}
    Since $\chi^2(t_0) = 1$, $\chi^2$ can be seem as a character on $\barr {T^+}$. 
    It follows that there is a unique element $\barr a\in T^+$ such that $\chi^2(\barr t) = \barr\beta(\barr a, \barr t)$ for all $\barr t\in \barr {T^+}$. 
    We have that $\theta\inv(\barr a) = \{ a, tt_0 \}$ for some $a\in T^+$. 
    Since $\chi^2(a) = \chi^2(\barr a) = 1$, $\chi (a) = \pm 1$. 
    Relabeling if necessary, we can assume $\chi (a) = 1$ and $\chi(at_0) = -1$. 
    This shows existence and uniqueness.
    
    The ``further'' part follows directly from \cref{def:O(T+-beta+),def:O_M}.
\end{proof}

\begin{defi}
    Given $\chi \in \widehat{T^+}$ such that $\chi(t_0) = -1$, let $a\in T^+$ as in \cref{lemma:chi-defines-a}. 
    We define $\mathbf{O_M}(T^+, \beta^+, \chi) \coloneqq \{ h\in G \mid h^2 = a \}$.
\end{defi}

It is clear that $\mathbf{O_M}(T^+, \beta^+, \chi)$ can be identified with $\pi\inv (\chi)$. 
Hence, by \cref{lemma:lemma-on-actions,iso-odd-M-no-existence}, we have:

\begin{cor}\label{cor:iso-odd-M-simplified}
    Iso simplified. \qed
\end{cor}

% -- Existence, sufficiency --

As stated before, the pair $(T^+, \beta^+)$ might not correspond to any triple $(T, \beta, p)$ with $\beta$ nondegenerate, \ie, the set $\mathbf{O_M}(T^+, \beta^+)$ might be empty.

\begin{prop}
    % Let $R \coloneqq \frac{T^+_{[2]}}{\langle t_0 \rangle}$. 
    The set $\mathbf{O_M}(T^+, \beta^+)$ is non-empty if, and only if, $\theta(T^+_{[2]})^\perp \subseteq \barr G^{[2]}$.
\end{prop}

\begin{proof}
    The ``only if'' part follows from \cref{prop:square-subgroup}. 
    
    For the ``if'' part, let $\chi \in Y$ and $a$ as in \cref{lemma:chi-defines-a}. 
    We will prove that $\mathbf{O_M}(T^+, \beta^+) \neq \empty$, \ie, that $a \in \barr G^{[2]}$. 
    
    First, we claim that $\barr a \in \theta(T^+_{[2]})^\perp$. 
	Indeed, if $b \in T^+_{[2]}$, then 
	\[
	    \barr\beta^+( \theta(a) , \theta(b) ) = \chi^2 (b) = \chi (b^2) = \chi (e) = 1.
	\]    
	By our assumption, we conclude that $\theta(a) \in G^{[2]}$. 
	Let $h\in G$ such that $\theta(h)^2 = \theta(a)$. 
	Then, either $a = h^2$ or $a = h^2t_0$. 
% 	and, in particular, $h^2 \in T^+$. 
% 	If $a=h^2$, the claim is established, so let us suppose $a=h^2t_0$. 

	If $t_0 \in G^{[2]}$, then both $h^2$ and $h^2 t_0$ are in $G^{[2]}$, and we are done. 
	Otherwise, we have that $\theta(T^+_{[2]}) = \theta(T^+)_{[2]}$. 
	Hence, using \cref{lemma:small-perp}, we have $a \in \theta(T^+_{[2]})^\perp = (\theta(T^+)_{[2]})^\perp = \theta(T^+)^{[2]}$, so we can suppose $h\in T^+$. 
	In this case, 
	\[
	    \chi(h^2) = \chi^2(h) = \beta^+ (a, h)
	     = \barr{\beta^+} (\theta(a), \theta(h)) = \barr{\beta^+} (\theta(h^2), \theta(h)) = 1.
	\] 
	By the condition that $\chi(a) = 1$, we conclude that $a = h^2$. 
\end{proof}

% -------------------




% \Cref{cor:iso-odd-M-simplified} does not gives us the full picture. 
It can be the case (see \cref{ex:T+-but-no-T}) that $\mathbf{O_M}(T^+, \beta^+)$ is empty, \ie, there might be no graded-division superalgebra $\D$ of type $M$ such that $\D\even$ is associated to the pair $(T^+, \beta^+)$.

\begin{defi}
	For every abelian group $H$, we define $H^{[2]} \coloneqq \{h^2 \mid h\in H\}$ and $H_{[2]} \coloneqq \{h\in H \mid h^2 = e \}$.
\end{defi}

If there is $(T, \beta, p)$, then it is clear that $(T^+)^{[2]} \subseteq T^{[2]} \subseteq T^+$, but we can have $(T^+)^{[2]} \neq T^{[2]}$. 

\begin{defi} 
    For any subgroup $A\subseteq \barr{T^+}$, we define its \emph{orthogonal complement} to be
    \[
        A^\perp \coloneqq \{t\in T\mid \bar\beta^+ (t, A) =1\}.
    \]
\end{defi}

\begin{lemma}
    For every subgroup $A \subseteq \barr T^+$, $|A^\perp| = [\barr T^+ : A]$ and $(A^\perp)^\perp = A$.
\end{lemma}

\begin{proof}
    It is easy to see that the map $A^\perp \to \widehat{\left( \frac{\barr T^+}{A}\right)}$ given by $t \mapsto \beta(t, \cdot)$ is an isomorphism of groups. 
    Also, since $\beta$ is nondegenerate, by \cref{prop:char-does-not-divide-T},  $\Char \FF$ does not divide the order of $\frac{\barr T^+}{A}$. 
    In this case, it is well-known that $\widehat{\left( \frac{\barr T^+}{A}\right)} \iso \frac{\barr T^+}{A}$, proving the first assertion.

    Since $\bar\beta^+$ is skew-symmetric, it is clear that $A \subseteq (A^\perp)^\perp$. 
    Using the first assertion, we have that $|A| = |(A^\perp)^\perp|$ and, therefore, $A = (A^\perp)^\perp$.
\end{proof}


\begin{prop}\label{prop:small-perp}
    Consider the subgroups of $\barr T^+$ defined by $A \coloneqq \theta(T^+)_{[2]}$ and $B \coloneqq \theta(T^+)^{[2]}$. 
    Then $A^\perp = B$.
\end{prop}

\begin{proof}
    Let us first prove that $B^\perp = A$:
    %
	\begin{align}
		B^\perp & = \{ \theta(t) \mid t\in T^+ \AND \barr\beta^+ (\theta (t), \theta (s^2)) = 1 \text{ for all } s\in T^+ \}\\ 
		& = \{\theta(t) \mid t\in T^+ \AND \beta^+ (t, s^2) = 1 \text{ for all } s\in T^+\}\\ 
		& = \{ \theta(t) \mid t\in T^+ \AND \beta^+ (t^2, s) = 1 \text{ for all } s\in T^+\}\\ 
		& = \{ \theta(t) \mid t\in T^+ \AND \theta(t)^2=e \} \numberthis\label{4th-line-again} \\ 
		&= A,
	\end{align}
	%
	where we are using that $\beta^+$ is nondegenerate in the fourth line \eqref{4th-line}. 
	By \cref{lemma:perp-perp}, $A^\perp = (B^\perp)^\perp = B$, concluding the proof.
\end{proof}

\begin{prop}\label{prop:square-subgroup}
    Suppose the pair $(T^+, \beta^+)$ extends to a triple $(T, \beta, p)$ with $\beta$ nondegenerate. 
	Consider the subgroups $S \coloneqq \theta(T^{[2]})$ and $R \coloneqq \theta(T^+_{[2]})$ of $\barr T^+$. 
	Then $R^\perp = S$ and, in particular, $\barr R^\perp \subseteq \barr G^{[2]}$.
\end{prop}

\begin{proof}
	Let us first prove that $S^\perp = R$:
	%
	\begin{align}
		S^\perp & = \{ \theta(t) \mid t\in T^+ \AND \barr\beta^+ (\theta (t), \theta (s^2)) = 1 \text{ for all }s\in T\}\\ 
		& = \{\theta(t) \mid t\in T^+ \AND \beta^+ (t, s^2) = 1 \text{ for all }s\in T\}\\ 
		& = \{ \theta(t) \mid t\in T^+ \AND \beta (t^2, s) =1 \text{ for all }s\in T\}\\ 
		& = \{ \theta(t) \mid t\in T^+ \AND t^2=e \} \numberthis\label{4th-line} \\ 
		&= R,
	\end{align}
	%
	where we are using that $\beta$ is nondegenerate in the fourth line \eqref{4th-line}. 
	By \cref{lemma:perp-perp}, $R^\perp = (S^\perp)^\perp = S$, concluding the proof.
\end{proof}


\begin{prop}
    Let $R \coloneqq \frac{T^+_{[2]}}{\langle t_0 \rangle}$. 
    The set $\mathbf{O_M}(T^+, \beta^+)$ is non-empty if, and only if, $R^\perp \subseteq \barr G^{[2]}$.
\end{prop}

\begin{proof}
    The only if part follows from \cref{prop:square-subgroup,lemma:motivation-O_M}. 
    
    For the ``if'' part, let $\chi \in Y$ and $a$ as in \cref{lemma:chi-defines-a}. 
    We claim that $\barr a \in R^\perp$. 
	Indeed, if $b \in T^+_{[2]}$, then $\barr\beta^+( \theta(a) , \theta(b) ) = \chi^2 (b) = \chi (b^2) = \chi (e) = 1$. 
	By our assumption, we conclude that $\theta(a) \in \barr G^{[2]}$. 
	
	We claim that, actually, $a\in G^{[2]}$. 
	Indeed, let $h\in G$ such that $\theta(h)^2 = \theta(a)$. 
	Then, either $a = h^2$ or $a = h^2t_0$ and, in particular, $h^2 \in T^+$. 
% 	If $a=h^2$, the claim is established, so let us suppose $a=h^2t_0$. 
	If $t_0 \in G^{[2]}$, then both $h^2$ and $h^2 t_0$ are in $G^{[2]}$, and the claim is established. 
	Otherwise, $\theta(T^+_{[2]}) = \theta(T^+)_{[2]}$ and, by \cref{prop:small-perp}, .
	
	the restriction of $\theta$ to $G^{[2]}$ is injective. 
	Clearly, $\theta(a) \in $
	Hence $a$
	
	\cref{prop:small-perp}
	
	Hence $R^\perp = (\barr T^+_{[2]})^\perp = \barr T^{[2]} = \theta ((T^+)^{[2]})$. 
	Thus, in this case, we can assume $h\in T^+$. 
	Then $\chi(h^2) = \chi^2(h) = \barr \beta (\barr a, \barr u) = \barr \beta (\barr h^2, \barr h) =1$, hence $h^2 = a$. 
%	
% 	Finally, we set $t_1=(h,\barr 1) \in G^\#$. 
% 	Since $\chi^2(t_0)=1$, we can consider $\chi^2$ as a character of the group $\barr T = \frac{T^+}{\langle t_0 \rangle}$, hence there is $a\in T^+$ such that $\chi^2(\barr t) = \barr\beta(\barr a, \barr t)$ for all $\barr t\in \barr T$. 
\end{proof}






% % ---------
% \section{Odd gradings on simple associative superalgebras only in terms of \texorpdfstring{$G$}{G}}\label{sec:assc-supersimple-only-G}

% In this section we will present a classification, up to isomorphism, only in terms of $G$ of odd gradings on finite dimensional simple superalgebras over an algebraically closed field. 

% Note that there are no odd gradings if $\Char \FF =2$. 
% Indeed, in this case, there is no nondegenerate bicharacter on $T$ because the characteristic of the field divides $|T|=2|T^+|$. 
% From now on, we suppose $\Char \FF \neq 2$.

% As stated in \cref{ssec:wedderburn-super}, the only problem is to describe the odd graded-division superalgebras only in terms of $G$. 
% For what follows, let $\D$ be a finite dimensional odd graded-division superalgebra, associated to the triple $(T, \beta, p)$. 
% We will denote the restriction of $\beta$ to $T^+ \times T^+$. 

% In general terms, our strategy is to describe the isomorphism class of $\D$ in terms of $\D\even$ and and some additional data, \ie, describe $(T, \beta, p)$ in terms of $(T^+, \beta^+)$ and some additional data. 

% \subsection{Odd gradings on \texorpdfstring{$M(n,n)$}{M(n,n)} only in terms of \texorpdfstring{$G$}{G}}\label{ssec:second-odd}

% Assume that we have $(T, \beta, p)$ with $\beta$ nondegenerate. 

% Since $\beta$ is nondegenerate, we have an isomorphism $T\rightarrow \widehat T$ given by $t\mapsto \beta(t,\cdot)$. For a subgroup $A\subseteq T$, we denote by $A^\perp$ its orthogonal complement in $T$ with respect to $\beta$, i.e., $A^\perp = \{t\in T\mid \beta(t, A) =1\}$. In view of the above isomorphism, $|A^\perp| = [T:A]$.

% In particular, we have $(T^+)^\perp = \langle t_0 \rangle$ where $t_0$ is an element of order 2. It follows that $\beta(t_0, t) = 1$ if $t\in T^+$ and $\beta(t_0, t) = -1$ if $t\in T^-$. For this reason, we call $t_0$ the \emph{parity element} of the odd grading $\Gamma$. Note that $\rad \beta^+ = T^+\cap (T^+)^\perp = \langle t_0 \rangle$.

% Fix an element $0\neq d_0\in \D$  of degree $t_0$. By the definition of $\beta$, $d_0$ commutes with all elements of $\D\even$ and anticommutes with all elements of $\D\odd$. Since $d_0^2\in \D_e = \FF$, and $\FF$ is algebraically closed, we may rescale $d_0$ so that $d_0^2=1$. Then $\epsilon := \frac{1}{2}(1+d_0)$ is a central idempotent of $\D\even$. Take a homogeneous element $0\neq d_1\in\D\odd$. Then $d_1\epsilon d_1\inv = \frac{1}{2}(1-d_0)=1-\epsilon$, which is another central idempotent of $\D\even$ and must have the same rank as $\epsilon$. 
% Hence, $\D\even\iso \epsilon\D\even\oplus (1-\epsilon)\D\even$ (direct sum of ideals) and, consequently, $E\even \iso \End(\tilde U)\tensor \D\even = \End(\tilde U)\tensor \epsilon\D\even \oplus \End(\tilde U)\tensor (1-\epsilon)\D\even$, where the two summands have the same dimension. Therefore, odd gradings exist only if $m=n$. 
% Also note that we have 
% \begin{equation}\label{eq:D1eps}
% \D\odd \epsilon = (1-\epsilon) \D\odd.
% \end{equation}

% We are now going construct an even grading by coarsening a given odd grading. The reverse of this construction will be used in Subsection \ref{ssec:second-odd}.

% Let $H$ be a group and suppose we have an even grading $\Gamma'$ on $M(n,n)$ that is the coarsening of $\Gamma$ induced by a group homomorphism $\alpha: G\rightarrow H$. Since $\Gamma'$ is even, then the idempotent $\id_{\tilde U}\tensor\epsilon$ must be homogeneous with respect to $\Gamma'$. This means that $\alpha(t_0)=e$, so $\alpha$ factors through $\barr G := G/\langle t_0 \rangle$. This motivates the following definition:

% \begin{defi}
% 	Let $\Gamma$ be an odd $G$-grading on $M(n,n)$ with parity element $t_0$. 
% 	The \emph{finest even coarsening of $\Gamma$} is the $\barr G$-grading ${}^\theta \Gamma$, where $\barr G := G/\langle t_0 \rangle$ and $\theta: G \to \barr G$ is the natural homomorphism.
% \end{defi}

% % Thm: this is the finest even coarsening
% \begin{thm}
% 	Let $\Gamma = \Gamma(T, \beta, \gamma)$ be an odd grading on $M(n,n)$ with parity element $t_0$. Then its finest even coarsening is isomorphic to $\barr \Gamma = \Gamma(\barr T, \barr \beta, \barr \gamma, \barr u\barr \gamma)$, where $\barr T= \frac{T^+}{\langle t_0 \rangle}$, $\barr\beta$ is the nondegenerate bicharacter on $\barr T$ induced by $\beta^+$, $\barr\gamma$ is the tuple whose entries are the images of the entries of $\gamma$ under $\theta$, and $u \in G$ is 
% 	any element such that $(u, \barr 1) \in T^-$.
% \end{thm}

% \begin{proof}
% 	% D now decomposes as left module
% 	Let us focus our attention on the $G$-graded division algebra $\D$. We now consider it as a $\barr G$-graded algebra, which has a decomposition $\D=\D\epsilon \oplus \D(1-\epsilon)$ as a graded left module over itself.

% 	% Claim 1: D\epsilon is simple
% 	\setcounter{claim}{0}
% 	\begin{claim}
% 		The $\D$-module $\D\epsilon$ is simple as a graded module.
% 	\end{claim}

% 	% Proof of Claim 1
% 	To see this, consider a nontrivial graded submodule $V\subseteq \D\epsilon$ and take a homogeneous element $0\neq v\in V$. Then we can write $v=d\epsilon$ where $d$ is a $\barr G$-homogeneous element of $\D$, so $d = d' + \lambda d' d_0$ where $d'$ is a $G$-homogeneous element and $\lambda\in \FF$. Hence, $v = d'\epsilon + \lambda d'd_0\epsilon = (1+\lambda)d'\epsilon$, where we have used $d_0\epsilon=\epsilon$. Clearly, $(1+\lambda)d'\neq 0$, so it has an inverse in $\D$. We conclude that $\epsilon\in V$, hence $V=\D\epsilon$.\qedclaim 

% 	Let $\barr \D := \epsilon \D \epsilon \iso \End_{\D}(\D\epsilon)$, where we are using the convention of writing endomorphisms of a left module on the right. By Claim 1 and the graded analog of Schur's Lemma (see \eg \cite[Lemma 2.4]{livromicha}), $\barr \D$ is a $\barr G$-graded division algebra.

% 	% Claim 2: parameters of \barr D
% 	\begin{claim}
% 		The support of $\barr \D$ is $\barr T= \frac{T^+}{\langle t_0 \rangle}$ and the bicharacter $\barr \beta: \barr T\times \barr T\rightarrow \FF^\times$ is induced by $\beta^+: T^+\times T^+ \rightarrow \FF^\times$.
% 	\end{claim}

% 	% Proof of Claim 2
% 	We have $\barr \D = \epsilon \D\even \epsilon + \epsilon \D\odd \epsilon$ and $\epsilon \D\odd \epsilon = 0$ by Equation \eqref{eq:D1eps}, so $\supp \barr \D \subseteq \barr T$. On the other hand, for every $0\neq d\in \D\even$ with $G$-degree $t\in T^+$, we have that $\epsilon d\epsilon = d\epsilon = \frac{1}{2}(d+dd_0)\neq 0$, since the component of degree $t$ is different from zero. Hence $\supp \barr \D = \barr T$. Since $\epsilon$ is central in $\D\even$, we obtain $\barr\beta (\barr t,\barr s) = \beta (s, t) = \beta^+ (s, t)$ for all $t, s\in T^+$.\qedclaim 

% 	% a basis for D\epsilon
% 	We now consider $\D\epsilon$ as a graded right $\barr \D$-module. Then we have the decomposition $\D\epsilon = \epsilon \D\epsilon \oplus (1-\epsilon) \D\epsilon$. The set $\{\epsilon\}$ is clearly a basis of $\epsilon \D\epsilon$. To find a basis for $(1-\epsilon)\D\epsilon$, fix any $G$-homogeneous $0\neq d_1\in \D\odd$  with $\deg d_1 = t_1\in T^-$. Then we have $(1-\epsilon)\D\epsilon = (1-\epsilon)\D\even \epsilon + (1-\epsilon)\D\odd \epsilon = (1-\epsilon)\D\odd \epsilon = \D\odd \epsilon$ by Equation \eqref{eq:D1eps}. Since $d_1$ is invertible, $\{d_1\epsilon\}$ is a basis for $(1-\epsilon) \D\epsilon$. We conclude that $\{\epsilon, d_1\epsilon\}$ is a basis for $\D\epsilon$.

% 	% All parameters for D
% 	Using the graded analog of the Density Theorem (see e.g. \cite[Theorem 2.5]{livromicha}), we have $\D\iso \End_{\barr \D}(\D\epsilon)\iso \End(\FF\epsilon\oplus \FF d_1\epsilon)\tensor \barr\D$. Hence,
% 	%
% 	\[
% 	\begin{split}
% 	\End_\D(\mc U)&\iso\End (\tilde U) \tensor \D \iso \End (\tilde U) \tensor \End(\FF\epsilon \oplus \FF d_1\epsilon) \tensor \barr\D \\
% 	&\iso \End(\tilde U\tensor \epsilon \oplus \tilde U\tensor d_1\epsilon) \tensor \barr\D
% 	\end{split}
% 	\]
% 	%
% 	as $\barr G$-graded algebras. The result follows.
% \end{proof}

% In the next section, we will show how to recover $\Gamma$ from $\barr\Gamma$ and some extra data. The following definition and result will be used there.

% \begin{defi}
% 	For every abelian group $A$ we put $A^{[2]} = \{a^2 \mid a\in A\}$ and $A_{[2]} = \{a\in A \mid a^2 = e \}$.
% \end{defi}

% Note that $T^{[2]}\subseteq T^+$, but $T^{[2]}$ can be larger than $(T^+)^{[2]}$ since it also includes the squares of elements of $T^-$. Also, the subgroup $\barr S = \{\barr t \in \barr T \mid t \in T^{[2]}\}$ of $\barr T$ can be larger than $\barr T^{[2]}$, but we will show that, surprisingly, it does not depend on $T^-$.

% \begin{lemma}
% 	Let $\theta: T^+\rightarrow \barr T=\frac{T^+}{\langle t_0 \rangle}$ be the natural homomorphism. 
% 	Consider the subgroups $\barr S = \theta(T^{[2]})$ and $\barr R=\theta(T^+_{[2]})$ of $\barr T$. 
% 	Then $\barr S$ is the orthogonal complement of $\barr R$ with respect to the nondegenerate bicharacter $\barr\beta$.
% \end{lemma}

% \begin{proof}
% 	We claim that $\barr S^\perp = \barr R$. Indeed,
% 	%
% 	\[
% 		\begin{split}
% 			\barr S^\perp & = \{ \theta(t) \mid t\in T^+ \AND \barr\beta(\theta (t), \theta (s^2)) =1 \text{ for all }s\in T\}\\ & = \{\theta(t) \mid t\in T^+ \AND \beta (t, s^2) =1 \text{ for all }s\in T\}\\ & = \{ \theta(t) \mid t\in T^+ \AND \beta (t^2, s) =1 \text{ for all }s\in T\}\\ & = \{ \theta(t) \mid t\in T^+ \AND t^2=e \}\\ & = \barr R\,.
% 		\end{split}
% 	\]
% 	Since $\bar{\beta}$ is nondegenerate, it follows that $\barr S = \barr R^\perp$.
% \end{proof}

% Our second description of an odd grading consists of its finest even coarsening and the data necessary to recover the odd grading from this coarsening. 
% % All parameters will be obtained in terms of $G$ rather than its extension $G^\#=G\times \ZZ_2$.

% Let $t_0\in G$ be an arbitrarily fixed element of order 2 and set $\barr G = \frac{G}{\langle t_0 \rangle}$. Let $\barr T \subseteq \barr G$ be a finite subgroup and let $\barr \beta: \barr T \times \barr T \rightarrow \FF^\times$ be a nondegenerate alternating bicharacter. 
% We define $T^+\subseteq G$ to be the inverse image of $\barr T$ under the natural homomorphism $\theta: G\rightarrow \barr G$. Note that $\barr \beta$ gives rise to a bicharacter $\beta^+$ on $T^+$ whose radical is generated by the element $t_0$. 
% We wish to define $T^-\subseteq G\times \{\barr 1\}$ so that $T=T^+\cup T^-$ is a subgroup of $G^\#$ and $\beta^+$ extends to a nondegenerate alternating bicharacter on $T$.

% From Lemma \ref{lemma:square-subgroup}, we have a necessary condition for the existence of such $T^-$, namely, for $\barr R=\frac{T^+_{[2]}}{\langle t_0 \rangle}$, we need $\barr R' \subseteq \barr G^{[2]}$ (indeed, $\barr S$ is a subgroup of $\overline {G^{[2]}} = \barr G^{[2]}$). 
% We will now prove that this condition is also sufficient.

% \begin{prop}\label{prop:square-subgroup-converse}
% 	If $\left( \frac{T^+_{[2]} }{\langle t_0 \rangle}\right)^\perp \subseteq \barr G^{[2]}$, then there exists an element $t_1\in G\times \{\barr 1\} \subseteq G^\#$ such that $T= T^+ \cup t_1\, T^+$ is a subgroup of $G^\#$ and  $\beta^+$ extends to a nondegenerate alternating bicharacter $\beta:T\times T\rightarrow \FF^\times$.
% % \end{prop}

% \begin{proof}
% 	Let $\chi\in \widehat {T^+}$ be such that $\chi(t_0) = -1$. Since $\chi^2(t_0)=1$, we can consider $\chi^2$ as a character of the group $\barr T = \frac{T^+}{\langle t_0 \rangle}$, hence there is $a\in T^+$ such that $\chi^2(\barr t) = \barr\beta(\barr a, \barr t)$ for all $\barr t\in \barr T$. Note that $\chi (a) = \pm 1$ and hence, changing $a$ to $a t_0$ if necessary, we may assume $\chi (a) = 1$.
	
% 	\bigskip 

% 	\textit{(i) Existence of $t_1$}:
	
% 	\medskip 

% 	As before, let $\barr R = \frac{T^+_{[2]}}{\langle t_0 \rangle}$. Then $\barr a \in \barr R^\perp$. Indeed, if $b\in T^+_{[2]}$, then $\barr\beta(\barr a,\barr b) = \chi^2 (\barr b) = \chi (b^2) = \chi (e) =1$. By our assumption, we conclude that $\barr a\in \barr G^{[2]}$. We are going to prove that, actually, $a\in G^{[2]}$. Pick $u\in G$ such that $\barr u^2 = \barr a$. Then, either $a=u^2$ or $a=u^2t_0$. If $t_0 = c^2$ for some $c\in G$, then replacing $u$ by $uc$ if necessary, we can make $u^2 = a$. Otherwise, $t_0$ has no square root in $T^+$, which implies that $\barr R=\barr T_{[2]}$. Hence $\barr R^\perp = (\barr T_{[2]})^\perp = \barr T^{[2]} = \theta ((T^+)^{[2]})$. Thus, in this case, we can assume $u\in T^+$. Then $\chi(u^2) = \chi^2(u) = \barr \beta (\barr a, \barr u) = \barr \beta (\barr u^2, \barr u) =1$, hence $u^2 = a$. Finally, we set $t_1=(u,\barr 1) \in G^\#$.

% 	\bigskip 

% 	\textit{(ii) Existence of $\beta$}:
	
% 	\medskip 

% 	We wish to extend $\beta^+$ to $T=T^+ \cup t_1\, T^+$ by setting $\beta(t_1, t) = \chi (t)$ for all $t\in T^+$. It is clear that there is at most one alternating bicharacter on $T$ with this property that extends $\beta^+$. To show that it exists and is nondegenerate, we will first introduce an auxiliary group $\widetilde T$ and a bicharacter $\tilde\beta$.

% 	Let $\widetilde T$ be the direct product of $T^+$ and the infinite cyclic group generated by a new symbol $\tau$. We define $\tilde\beta:\widetilde T\times \widetilde T \rightarrow \FF^\times$ by $ \tilde\beta(s\tau^i,t\tau^j) = \beta^+(s,t)\, \chi (s)^{-j}\, \chi (t)^i$, where $s,t\in T^+$. It is clear that $\tilde\beta$ is an alternating bicharacter.

% 	\begin{claim*}
% 	$\langle a\tau^{-2} \rangle = \rad \tilde \beta\,$.
% 	\end{claim*}

% 	Let $t\in T^+$ and $\ell\in \ZZ$. Then
% 	\[
% 	\tilde \beta (a\tau^{-2},t\tau^\ell) =
% 	\beta^+(a, t)\,\, \chi(t)^{-2} \, \chi(a)^{-\ell} = 
% 	\barr\beta(\barr a, \barr t)\,\, \chi(t)^{-2} = \chi(t)^2 \, \chi(t)^{-2} = 1,
%     \]
%     hence, $\langle a\tau^{-2} \rangle \subseteq \rad \tilde \beta$.

%     Conversely, if $s\tau^k \in \rad \tilde\beta$, then, $1 = \tilde \beta (s\tau^k, t_0) = \beta^+(s,t_0)\, \chi(t_0)^k = (-1)^k$, hence $k$ is even. From the previous paragraph, we know that $a\tau^{-2} \in \rad \tilde\beta$, hence $a^\frac{k}{2} \tau^{-k} \in \rad \tilde\beta$ and $s a^\frac{k}{2} = (s \tau^k) (a^\frac{k}{2} \tau^{-k}) \in \rad \tilde\beta$. Since $s a^\frac{k}{2} \in T^+$, we get $s a^\frac{k}{2} \in \rad \beta^+ = \{ e, t_0 \}$. But, if $sa^\frac{k}{2} = t_0$, we have $1 = \tilde\beta (sa^\frac{k}{2}, \tau) = \tilde\beta (t_0, \tau) = \chi(t_0)\inv = -1$, a contradiction. It follows that $sa^\frac{k}{2} = e$ and, hence, $s\tau^k = a^{-\frac{k}{2}}\tau^k = (a\tau^{-2})^{\frac{k}{2}}$, concluding the proof of the claim.
%     \qedclaim 

%     We have a homomorphism $\vphi:\widetilde T\rightarrow T$ that is the identity on $T^+$ and sends $\tau$ to $t_1$. Clearly, $\ker \vphi = \langle a\tau^{-2} \rangle$. By the above claim, $\tilde\beta$ induces a nondegenerate alternating bicharacter on $\frac{\widetilde T}{\langle a\tau^{-2} \rangle}$, which can be transferred via $\vphi$ to a nondegenerate alternating bicharacter on $T$ that extends $\beta^+$.
% %
% \end{proof}

% Now fix $\chi\in \widehat {T^+}$ with $\chi(t_0)=-1$ and let $a$ be the unique element of $T^+$ such that $\chi(a)=1$ and $\chi^2(\barr t) = \barr\beta (\barr a, \barr t)$ for all $t\in T^+$. Suppose that the condition of Proposition \ref{prop:square-subgroup-converse} is satisfied. 
% Then part (i) of the proof shows that there exists $u\in G$ such that $u^2=a$. Moreover, part (ii) shows that there exists an extension of $\beta^+$ to 
% a nondegenerate alternating bicharacter $\beta$ on $T=T^+\cup t_1T^+$, where $t_1=(u,\bar 1)$, such that $\beta(t_1,t)=\chi(t)$ for all $t\in T^+$.
% Clearly, such an extension is unique. We will denote it by $\beta_u$ and its domain by $T_u$.

% \begin{prop}\label{prop:roots-of-a}
% For every $T\subseteq G^\#$ such that $T\subsetneq G$ and $T\cap G=T^+$ and for every extension of $\beta^+$ to a nondegenerate alternating bicharacter $\beta$
% on $T$, there exists $u\in G$ such that $u^2=a$, $T=T_u$ and $\beta=\beta_u$.
% Moreover, $\beta_u=\beta_{\tilde{u}}$ if, and only if, 
% $u^{-1} \tilde{u} \in \langle t_0 \rangle$.
% \end{prop}

% \begin{proof}

% We have $T=T^+ \cup T^-$ where $T^-\subseteq G\times \{\barr 1\}$ is a coset of $T^+$.
% We can extend $\chi$ to a character of $T$, which we still denote by $\chi$, and, since $\beta$ is nondegenerate, 
% there is $t_1\in T$ such that $\beta(t_1, t) = \chi(t)$ for all $t\in T$. We have $t_1\in T^-$ since $\beta(t_1,t_0)=\chi(t_0)=-1$, so $t_1=(u,\bar 1)$, 
% for some $u\in G$, and hence $T=T_u$. We claim that $t_1^2 = a = (a, \bar 0)$ and, hence, $u^2 = a$. Indeed, $\chi(t_1^2) = \beta(t_1,t_1^2)=1$ and, for every $t\in T^+$,
% \[
%  	\chi^2(\barr t) = \chi(t)^2 = \beta (t_1, t)^2 = \beta (t_1^2, t) = \barr\beta (\,\overline {(t_1^2)},\, \barr t)\,,
% \]
% so $t_1^2$ satisfies the definition of the element $a$. This completes the proof of the first assertion.

% Now suppose $\beta_u=\beta_{\tilde{u}}$, so in particular $t_1\,T^+=\tilde{t}_1\,T^+$ where $t_1 = (u, \barr 1)$ and $\tilde{t}_1 = (\tilde u, \barr 1)$.
% Then there is $r\in T^+$ such that $\tilde{t}_1 = t_1\,r$. Also, for every $t\in T^+$,
% \[
% \chi(t) = \beta_{\tilde{u}}(\tilde{t}_1,t) = \beta_u (t_1\,r, t)
% 		= \beta_u(t_1, t)\,\beta_u(r,t) = \chi(t) \beta^+(r, t)
% \]
% and, hence, $\beta^+(r, t)=1$ for all $t\in T^+$. This means that $r = u\inv \tilde{u} \in \langle t_0 \rangle$.

% Conversely, if $\tilde u = u r$ for some $r\in \langle t_0 \rangle$, then $t_1\, T^+ = \tilde t_1\, T^+$. Also, for all $t\in T^+$,
% \[
% \beta_u(t_1, t) = \chi(t) = \beta_{\tilde{u}}(\tilde{t}_1, t) = \beta_{\tilde{u}}(t_1r, t) = 
% \beta_{\tilde{u}}(t_1, t)\, \beta^+(r, t) = \beta_{\tilde{u}}(t_1, t).
% \]
% It follows that $\beta_u=\beta_{\tilde{u}}$.
% \end{proof}

% Note that, keeping the character $\chi \in \widehat {T^+}$ with $\chi(t_0) = -1$ fixed, we have a surjective map from the square roots of $a$ to all possible pairs $(T,\beta)$. If we had started with a different character above, we would have obtained a different surjective map. Hence, for parametrization purposes, $\chi$ (and, hence, $a$) will be fixed.

% We are now in a position to give a classification of odd graded-division superalgebras gradings in terms of $G$ only. 
% We already have the following parameters: an element $t_0\in G$ of order $2$ and a pair $(\barr T, \barr\beta)$. 
% For any such $t_0$ and $\barr T$, we fix a character $\chi\in \widehat {T^+}$ satisfying $\chi(t_0) = -1$. 
% The next parameter is an element $u\in G$ such that $u^2 = a$, where $a$ is the unique element of $T^+$ such that $\chi(a)=1$ and $\chi^2(\barr t) = \barr\beta (\barr a, \barr t)$ for all $t\in T^+$. 
% % Finally, let $\gamma = (g_1, \ldots, g_k)$ be a $k$-tuple of elements of $G$. With these data, we construct the grading $\Gamma (t_0, \barr T, \barr \beta, u, \gamma)$ as follows:

% \begin{defi}\label{def:odd-grd-on-Mmn-2}
% 	Let $\D$ be a standard realization of the $G^\#$-graded division algebra with parameters $(T_u,\beta_u)$. 
% 	As a superalgebra, $\D$ is isomorphic to $M(n,n)$. 
% 	We define $\Gamma (t_0, \barr T, \barr \beta, u)$ as the corresponding $G$-grading on $M(n,n)$.
% \end{defi}

% % Theorem \ref{thm:first-odd-iso} together with Proposition \ref{prop:roots-of-a} give the following result:

% \begin{thm}\label{thm:2nd-odd-iso}
% 	Every odd division $G$-grading on the superalgebra $M(n,n)$ is isomorphic to some $\Gamma (t_0, \barr T, \barr \beta, u)$ as in Definition \ref{def:odd-grd-on-Mmn-2}.
% 	Two odd gradings, $\Gamma (t_0, \barr T, \barr \beta, u)$ and $\Gamma (t_0', \barr T', \barr \beta', u')$, 
% 	are isomorphic if, and only if, $t_0=t_0'$, $\barr T = \barr T'$, $\barr\beta = \barr\beta'$ and $u^{-1} u' \in \langle t_0 \rangle$. \qed
% \end{thm}

% \subsection{Gradings on \texorpdfstring{$Q(n)$}{Q(n)} only in terms of \texorpdfstring{$G$}{G}}

% % In this subsection we will not only find a description of gradings on $Q(n)$ only in terms of $G$, but we

% Suppose $\D \iso Q(n)$ as a superalagebra. 
% Recall that, by \cref{prop:tilde-beta-nondeg}, we have that $\tilde\beta$ is nondegenerate but $\beta$ is degenerate, \ie, we are under the conditions of \cref{lemma:beta-deg-beta-tilde-nondeg}. 

% We will denote by $u \coloneqq
%     \begin{pmatrix}
%         0 & I\\
%         I & 0
%     \end{pmatrix}
%     \in Q(n)$. 
% Note that $Z(Q(n))\even = \FF 1$ and $Z(Q(n))\odd = \FF u$. 

% % Let $\Gamma$ be a division on $Q(n)$ and let $(T, \beta, p)$ be the triple associated to the graded-division superalgebra $\D \coloneqq (Q(n), \Gamma)$. 
% % Note that, by \cref{prop:tilde-beta-nondeg}, we are under the conditions of \cref{lemma:beta-deg-beta-tilde-nondeg}. 

% % Clearly, $(T^+, \beta^+)$ defines a division-grading on the algebra $Q(n)\even \iso M_n(\FF)$. 
% By \cref{lemma:beta-deg-beta-tilde-nondeg}\eqref{item:rad-beta=t_1}, we have that $\rad \beta = \langle t_1 \rangle$ for a order two element $t_1 \in T^-$. 
% Then $t_1 = (h, \bar 1)$ for some element $h\in G$ of order at most two. 
% Since $\rad \beta = \supp Z(Q(n))$, we must have that $\deg u = t_1$. 

% \begin{defi}
%     If $\D \iso Q(n)$ as a superalgebra, we define the $G$-parameters of $\D$ to be the triple $(T^+, \beta^+, h)$.
% \end{defi}

% Note that the triple $(T^+, \beta^+, h)$ has the same information as the pair $(\Gamma\even, h)$, where $\Gamma\even$ denote the grading on $\D\even$.

% Conversely, given a division-grading $\Gamma\even$ on the algebra $Q(n)\even$ and an order two element $h \in G$, let $(T^+, \beta^+)$ be the pair corresponding to the graded-division algebra $(Q(n)\even, \Gamma\even)$ and define $t_1 \coloneqq (h, \bar 1)$. 
% We can then define $T \coloneqq T^+ \times \langle t_1 \rangle$, and define $\beta\from T\times T \to \FF^\times$ by $\beta(s t_1^i, t t_1^j) \coloneqq \beta^+(s, t)$. 
% It is easy to see that $\beta$ is a alternating bicharacter on $T$ and $\rad \beta = \langle t_1 \rangle$, and that $(T, \beta, p)$ is the triple associated to the graded-division superalgebra $Q(1)$ where the grading on $Q(n)\even$ is extended to $Q(n)$ by declaring $\deg u \coloneqq t_1$.  

% % We conclude the following:

% % \begin{thm}\label{thm:grd-div-Q-only-G}
    
% % \end{thm}

% \begin{remark}
%     The description of gradings on $Q(n)$ we found here
%     % in \cref{thm:grd-div-Q-only-G} 
%     is essentially the same one as in \cite[Theorem 5.5]{paper-Qn}.
% \end{remark}

% \begin{itemize}
%     \item Add that every $G$-grading on $Q(n)$ is restriction of a $G\times \ZZ_2$-grading on $M(n,n)$.
% \end{itemize}

% % For every associative superalgebra $R = R\even \oplus R\odd$, it is easy to see that $R\odd$ is an $(R\even, R\even)$-bimodule. 
% % In the case of $R \iso Q(n)$, then it is clear that $R\even \iso M(n)$ as an algebra and $R\odd \iso M(n)$ as a $(M(n), M(n))$-bimodule. 


% % \begin{itemize}
% %     \item We have two parametrizations, one here and one in \cite{paper-Qn}.
% %     \item The one here is as follows:
% %     \begin{itemize}
% %         \item Every grading is odd, since it contains $Q(1)$ in the center, and hence in the center of $\D$.
% %         \item Hence we describe it by $(T, \beta, \kappa)$, where $T\subseteq G^\#$ but $T\not\subseteq G$. 
% %         \item Also, $\rad \beta$ is generated by an odd element of order $2$, say $t_0$.
% %         \item $t_0$ is the parity element according to $\tilde\beta$.
% %     \end{itemize}
% %     \item The old parametrization consisted of a grading on the even part and an element of order $2$ (in $G$) to shift it to the odd part.
% %     \item The old parametrization is only in terms of $G$.
% %     \item the connection must be, top down:
% %     \begin{itemize}
% %         \item The grading on the even part is $(T^+, \beta^+, \kappa)$.
% %         \item Note that $\beta^+$ is nondegenerate on $T^+$.
% %         \item If $t_0 = (h, \bar 1)$, then the shift on the odd part is by $h$, which clearly has order $2$.
% %     \end{itemize}
% %     \item the connection must be, bottom up:
% %     \begin{itemize}
% %         \item Define $t_0$ as $(h, \bar 1)$
% %         \item Define $T$ as $T^+ \cup t_0 T^+$
% %         \item Extend $\beta^+$ to $\beta$ by defining $t_0$ to be in the radical
% %     \end{itemize}
% %     \item we can see it not only on the level of parameters but also on the level of $\End_\D (\U)$:
% %     \item top down:
% %     \begin{itemize}
% %         \item Note that $\End_\D (\U) = \End_{\D\even} (\U\even) \oplus d_0 \End_{\D\even} (\U\even)$, where $d_0 \in Z(\D) \cap \D\odd$. 
% %         Why is that?
% %         \item Well, I guess it starts we getting an even basis for $\U$ and its corresponding $\FF$-form $\widetilde U$. 
% %         \item Then $\End_\D (\U) \equiv \End_\FF (\tilde U) \tensor \D$, hence the even part is $\End_\FF (\tilde U) \tensor \D\even$. 
% %         \item Since $\D\odd = d_0 \D\even$, we get that the odd part is $\End_\FF (\tilde U) \tensor \D\odd = d_0 \End_\FF (\tilde U) \tensor \D\even$
% %         \item The apparent problem here is that the choice of the shift on the odd part doesn't matter.
% %         \item Modulo $T^+$ they are clearly the same, but we don't need $T^+$ in the parametrization
% %         \item Oh, it should be the definition of the product in the down model! $(d_0 r) r' = r (d_0 r')$ and its version for only odd elements. So $d_0 \in Z(R)$.
% %         % \item Also, we could see that $\U\even = \tilde U \tensor \D\even$ and $\U\odd = \tilde U \tensor \D\odd$.
% %     \end{itemize}
% %     \item bottom up:
% %     \begin{itemize}
% %         \item The grading is determined be $\D\even$ and $\U\even$
% %         \item Let $R\even = \End_{\D\even} (\U\even)$ 
% %         \item define $t_0 = (h, \bar 1)$ and introduce a symbol $d_0$ of degree $t_0$, such that $d_0^2 = 1$ and it commutes with every element. This defines $R$.
% %         \item one should check the parameters for $R$. What is the new $\D$?
% %         \item By uniqueness of $\D$ up to iso, we may be able to use the top down part to prove it is what is asked. But we then have to show an isomorphism. Luckily, this is obvious by the definition of the product. 
% %         \item Wait, we have to define $\D$ as $\D\even \oplus d_0\D\odd$, and check it is a graded-division algebra (obvious).
% %     \end{itemize}
% % \end{itemize}
