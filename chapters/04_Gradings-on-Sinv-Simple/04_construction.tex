\section[Gradings on superinvolution-simple superalgebras of types  \texorpdfstring{$M\times M\sop$ and $Q\times Q\sop$}{MxMsop and QxQsop}]{Gradings on superinvolution-simple \\ superalgebras of types $M\times M\sop$ and $Q\times Q\sop$}\label{sec:model-O}

For this section we assume $\FF$ is algebraically closed and $\Char \FF \neq 2$. 

Let $R \coloneqq S \times S\sop$ where $S$ is a finite dimensional simple superalgebra, and let $\vphi$ denote the exchange superinvolution. 
By \cref{lemma:SxSsop-simple}, $(R, \vphi)$ is a superinvolution-simple superalgebra. 
If we endow $(R, \vphi)$ with a grading, then it becomes, clearly, graded-superinvolution-simple, but not necessarily graded-simple. 

In the case $R$ is not graded-simple, the situation is easier: since the nonzero proper superideals are $S \times \{ 0 \}$ and $\{ 0 \} \times S\sop$, at lest one of them must be a graded superideal, and since $\vphi$ sends graded subsuperspace onto graded subspaces, $S \times \{ 0 \}$ is a graded superideal if, and only if, $\{ 0 \} \times S\sop$ also is. 
Therefore, by \cref{cor:SxSsop-with-dcc,thm:iso-D-even-ExEsop,thm:iso-D-odd-ExEsop}, in \cref{grd-sinv-simple}, and \cref{cor:iso-M-even,cor:iso-M-odd,cor:iso-Q}, in \cref{sec:grd-simple-salg}, we have following classification results:

\begin{thm}
    Let $(R, \vphi)$ be a superinvolution-simple superalgebra of type $M\times M\sop$ endowed with a $G$-grading and suppose it is not graded-simple. 
    % Then there is a finite subgroup $T \subseteq G^\#$ such that: 
    Then $(R, \vphi)$ is either isomorphic to $M(T, \beta, \kappa_\bz, \kappa_\bo) \times M(T, \beta, \kappa_\bz, \kappa_\bo)\sop$ (see \cref{def:Gamma-T-beta-kappa-even}), or to $M (T, \beta, p, \kappa) \times M (T, \beta, p, \kappa)\sop$ (see \cref{def:Gamma-T-beta-kappa-odd}), where we consider the exchange superinvolution in both cases. 
    
    Moreover, $M(T, \beta, \kappa_\bz, \kappa_\bo) \times M(T, \beta, \kappa_\bz, \kappa_\bo)\sop \iso M(T', \beta', \kappa_\bz', \kappa_\bo') \times M(T', \beta', \kappa_\bz', \kappa_\bo')\sop$ if, and only if, $T = T'$ and one of the following conditions holds:
	\begin{enumerate}[(i)]
	    \item $\beta'=\beta$ and there is $g\in G$ such that such that either $g \cdot \kappa_{\bar 0}'=\kappa_{\bar 0}$ and $g \cdot \kappa_{\bar 1}'=\kappa_{\bar 1}$, or $g \cdot \kappa_{\bar 0}'=\kappa_{\bar 1}$ and $g \cdot \kappa_{\bar 1}'=\kappa_{\bar 0}$; 
	    \item $\beta'=\beta\inv$ and there is $g\in G$ such tha either $g \cdot \kappa_{\bar 0}'=\kappa_{\bar 0}\Star$ and $g \cdot \kappa_{\bar 1}'=\kappa_{\bar 1}\Star$, or $g \cdot \kappa_{\bar 0}'=\kappa_{\bar 1}\Star$ and $g \cdot \kappa_{\bar 1}'=\kappa_{\bar 0}\Star$;
	\end{enumerate}
    and $M (T, \beta, p, \kappa) \times M (T, \beta, p, \kappa)\sop \iso M (T', \beta', p', \kappa') \times M (T', \beta', p', \kappa')\sop$ if, and only if, $T = T'$, $p=p'$ and one of the following conditions holds:
    \begin{enumerate}[(i)]
        \setcounter{enumi}{2}
	    \item $\beta'=\beta$ and there is $g\in G$ such that $\kappa' = g \cdot \kappa$;
	    \item $\beta'=\beta\inv$ and there is $g\in G$ such that $\kappa' = g \cdot \kappa\Star$. \qed
	\end{enumerate}
\end{thm}

\begin{thm}
    Let $(R, \vphi)$ be a superinvolution-simple superalgebra of type $Q\times Q\sop$ endowed with a $G$-grading and suppose it is not graded-simple. 
    Then $(R, \vphi)$ is to $Q (T^+, \beta^+, h, \kappa) \times Q (T^+, \beta^+, h, \kappa)\sop$ (see \cref{def:Gamma-T-beta-kappa-Q}), where we consider the exchange superinvolution. 
    
    Moreover, $Q (T^+, \beta^+, h, \kappa) \times Q (T^+, \beta^+, h, \kappa)\sop \iso Q (T^+, \beta^+, h, \kappa) \times Q (T^+, \beta^+, h, \kappa)\sop$ if, and only if, $T^+ = T'^+$, $h = h'$ and one of the following conditions holds: 
    \begin{enumerate}[(i)]
	    \item $\beta'^+ = \beta^+$ and there is $g\in G$ such that $\kappa' = g \cdot \kappa$;
	    \item $\beta'^+ = (\beta^+)\inv$ and there is $g\in G$ such that $\kappa' = g \cdot \kappa\Star$. \qed
	\end{enumerate}
\end{thm}

\subsection{Center and superinvolution}

\begin{enumerate}
    \item Lemma of centers, general
    \item remark about convention
    \item Lemma of centers, types
    \item supercenters are the same, anyway
    \item the element $f$, unique class
    \item prop: tensor product
\end{enumerate}


For the remainder of the section, we will consider, as before, $R \coloneqq \End_\D(\U)$, where $\D$ is a graded division superalgebra and $\U$ is a nonzero graded right $\D$-module of finite rank.
Also, let $\vphi$ be a super-anti-automorphism on $R$ and let $(\vphi_0, B)$ be pair determining $\vphi$ as in Theorem \ref{thm:vphi-iff-vphi0-and-B} and following Convention \ref{conv:pick-even-form}.

We will now show that we can identify $(Z(\D), \vphi_0)$ with $(Z(R), \vphi)$.
For every $c\in Z(\D)$, consider $r_c\from \U \to \U$ given by $r_c(u) = uc$.
Clearly, $r_c$ is $\D$-linear, so $r_c \in R$.
Actually, we have $r_c\in Z(R)$.
Indeed, for all $r\in R = \End_\D(\U)$ and all $u\in \U$, we have $r (r_c(u)) = r(uc) = r(u) c = r_c(r(u))$.

\begin{prop}%\label{prop:R-and-D-have-the-same-center}
	The map $Z(\D) \to Z(R)$ given by $c \mapsto r_c$ is an isomorphism of $G$-graded superalgebras.
	Moreover, $\vphi (r_c) = (-1)^{|B||c|} r_{\vphi_0(c)}$. 
\end{prop}

\begin{proof}
	Given $r\in Z(R)$, we can define $c_r\in \End_R (\U) =\D$ by $uc_r = r(u)$ for all $u\in \U$.
	Computations analogous to the ones above show that $c\in Z(\D)$, and it is clear that the map $r\mapsto c_r$ is the inverse of the map $c \mapsto r_c$.
	The definition of grading on $R = \End_\D (\U)$ implies that these maps are isomorphisms of $G$-graded superalgebras.

	For the ``moreover'' part, fix $c\in Z(\D)\even \cup Z(\D)\odd$ and let $u, v \in \U\even \cup \U\odd$.
	On the one hand,
	\begin{align*}
		B(uc,v) = B(r_c u), v) = (-1)^{|r_c||u|} B(u, \vphi(r_c) v) = (-1)^{|c||u|} B(u, \vphi(r_c) v).
	\end{align*}
	On the other hand,
	\begin{align*}
		B(uc, v) = (-1)^{(|B| + |u|) |c|} \vphi_0(c) B(u, v) & = (-1)^{(|B| + |u|) |c|} B(u, v) \vphi_0(c)     \\
		                                                     & = (-1)^{(|B| + |u|) |c|} B(u, v \vphi_0(c) )    \\
		                                                     & = (-1)^{(|B| + |u|) |c|} B(u, r_{\vphi_0(c)}v).
	\end{align*}
	%
	Since we are following Convention \ref{conv:pick-even-form}, either $\D$ is even, and hence $|c| = \bar 0$, or
	$|B| = \bar 0$.
	In any case, $|B||c| = \bar 0$, so we have that \[(-1)^{|c||u|} B(u, \vphi(r_c) v) = (-1)^{|c||u|} B(u, r_{\vphi_0(c)}v),\] and, hence, $B(u, (\vphi(r_c) - \vphi_0(c)) v) = 0$ for all $u, v \in \U$.
	Since $B$ is nondegenerate, the results follows.
\end{proof}

\begin{prop}\label{prop:vphi-R-simple-D-simple}
	The superalgebra $R$ is $\vphi$-simple if, and only if, the superalgebra $\D$ is $\vphi_0$-simple.
\end{prop}

\begin{proof}
	Pick a homogeneous $\D$-basis for $\U$ following Convention \ref{conv:pick-even-basis} and use it to identify $R$ with $M_k(\D) = M_k(\FF) \tensor \D$.
	By \cref{conv:pick-even-form}, we may assume that $B$ is even and then,
	by Proposition \ref{prop:matrix-vphi}, for every $X \in M_k(\D)$, we have
	$\vphi(X) = \Phi\inv \vphi_0(X\stransp) \Phi$, where $\Phi \in M_k(\D)$ is the matrix representing $B$. 

	It is well known that the ideals of $M_k(\D)$ are precisely the sets of the form $M_k(I)$ for $I$ an ideal of $\D$.
	We will prove an analog of this, first, for superideals and, then, for $\vphi$-invariant superideals.

	If $I$ is a superideal, $M_k(I) = M_k(\FF) \tensor I$ is also a superideal since it is spanned by a set of $\ZZ_2$-homogeneous elements, namely, the elements of the form $E_{ij}\tensor d$ where $1 \leq i,j \leq k$ and $d \in I\even \cup I\odd$.
	Conversely, if $J = M_k(I)$ is a superideal, then we can write $I = \{ d\in  \D \mid E_{11}\tensor d \in J\}$.
	For every $d\in I$, write $d = d_{\bar 0} + d_{\bar 1}$, where $d_\alpha \in \D^\alpha$, $\alpha \in \ZZ_2$.
	Since the $\ZZ_2$-homogeneous components of $E_{11}\tensor d$ are $E_{11}\tensor d_{\bar 0}$ and $E_{11}\tensor d_{\bar 1}$ and they belong to $J$, we have $d_{\bar 0}, d_{\bar 1} \in I$.

	Now we are going to show that $M_k(I)$ is $\vphi$-invariant if, and only if, $I$ is $\vphi_0$-invariant.
	Suppose $I$ is $\vphi_0$-invariant.
	Then if $X \in M_k(I)$, it is clear that $\vphi_0 (X\stransp)$ is also in $M_k(I)$.
	But then $\vphi(X) = \Phi\inv \vphi_0(X\stransp) \Phi \in M_k(I)$ since $M_k(I)$ is an ideal.
	Conversely, suppose $M_k(I)$ is $\vphi$-invariant.
	Let $d \in I$ and consider $X = E_{11} \tensor d \in M_k(I)$.
	Then $E_{11} \tensor \vphi_0(d) = \vphi_0(X\stransp) = \Phi\, \vphi(X)\, \Phi\inv \in M_k(I)$, which shows that $\vphi_0(d) \in I$.
\end{proof}

\begin{cor}\label{cor:D-has-the-same-type}
	Suppose $\FF$ is an algebraically closed field and $\Char \FF \neq 2$.
	Assume that $\vphi$ is a superinvolution and that $R$ is $\vphi$-simple.
	Then $(R, \vphi)$ is of the same type as $(\D, \vphi_0)$. \qed
\end{cor}


Recall that we can distinguish among the different types of superinvolution-simple superalgebras by their center (Proposition \ref{prop:types-of-SA-via-center}).
In the case our superalgebra has a division grading, this information is captured by $\rad \beta$:

\begin{lemma}\label{lemma:types-of-D-via-rad-beta}
	Let $(D, \vphi_0)$ be a graded-division superalgebra with superinvolution associated to $(T, \beta, p, \eta)$.
	\begin{enumerate}[(i)]
		\item If $(D, \vphi_0)$ is of type $M$, then $\rad \beta = (\rad \beta) \cap T^+ = \{e\}$;
		\item If $(D, \vphi_0)$ is of type $M\times M\sop$, then $\rad \beta = (\rad \beta) \cap T^+ = \langle f \rangle$, where $f$ has order $2$ and $\eta(f) = -1$;
		\item If $(D, \vphi_0)$ is of type $Q\times Q\sop$, then $\rad \beta = \langle t_1 \rangle$, where $t_1 \in T^-$ has order $4$ and $\eta(t_1) = 1$, and $(\rad \beta) \cap T^+ = \langle f \rangle$, where $f = t_1^2$ and, hence, $\eta(f) = -1$. \qed
	\end{enumerate}
\end{lemma}

\begin{thm}
	Let $\D$ be a finite dimensional superinvolution-simple superalgebra endowed with a division grading.
	Then $\D \iso \D' \otimes \D''$ as graded superalgebras with superinvolution, where $\D'$ is of type $M$ (and, hence, even) and
	\begin{enumerate}[(i)]
		\item if $\D$ is of type $M\times M\sop$, then $\D''$ is weakly isomorphic to $\mc O$, as in Example \ref{ex:superalgebra-O};
		\item if $\D$ is of type $Q\times Q\sop$, then $\D''$ is weakly isomorphic to $\FF\ZZ_4$, as in Example \ref{ex:FZ4-revisited}.
	\end{enumerate}
\end{thm}

\begin{proof}
	Let $\D$ be associated to $(T, \beta, p, \eta)$.
	By Lemma \ref{lemma:types-of-D-via-rad-beta}, we can write $\rad \beta \cap T^+ = \{ e, f\}$, with $\eta(e) = 1$ and $\eta(f) = -1$.

	Consider $\overline{T} = \frac{T}{\rad \beta}$ with its induced bicharacter $\bar \beta$.
	Since $\bar\beta$ takes values in $\{\pm 1\}$ and it is nondegenerate, by Prop ?? we have that $\overline{T}$ is an elementary $2$-group.
	%We claim that $T^+$ is also an elementary $2$-group. 
	Let $t\in T$.
	We have that $\bar t^2 = \bar e \in \overline{T}$, so, since $t^2$ is even, $t^2 \in \{ e, f \}$.
	If $t$ is even, then $\eta(t^2) = \eta(t)^2 = 1$, hence $t^2 = e$.
	If $t$ is odd, then, $\eta(t^2) = -\eta(t)^2 = -1$, hence $t^2 = f$.
	Note that, in particular, $T^+$ is an elementary $2$-group.

	If $\D \iso Q(n)$, let $t_1$ be an odd element in $\rad \beta$ and complete $\{f\}$ to a $\ZZ_2$-basis $\{f, t_2, \ldots, t_\ell \}$ of $T^+$.
	Then $T$ is generated by $\{t_1, t_2, \ldots, t_\ell \}$ and this generating set give us an isomorphism $T \iso \ZZ_4 \times (\ZZ_2)^{\ell - 1}$.
	Restricting $\beta$ and $\eta$ to $\ZZ_4$ and $(\ZZ_2)^{\ell - 1}$ we find $\D'$ and $\D''$, respectively.

	If $\D \iso M(m,n) \times M(m,n)\sop$, them there are no odd elements in $\rad \beta$ so, in particular, $p$ induces an homomorphism in $\overline{T}$.
	Since $\bar \beta$ is nondegenerate, this homomorphism corresponds to a element $\bar {t_0} \in \overline T$ such that $\bar \beta (\bar {t_0}, \bar t) = (-1)^{p(t)}$.
	We choose $t_0 \in T$ to be a preimage of $\bar {t_0}$ and choose $t_1$ to be any element in $T^-$.
	Note that $t_0 \in T^+$ since $1 = \beta(\bar {t_0}, \bar {t_0}) = (-1)^{p(t_0)}$.
	Again, since $T^+$ is an elementary $2$-group, complete $\{t_0, f\}$ to a $\ZZ_2$-basis $\{ t_0, f, t_2, \ldots, t_\ell \}$ of $T^+$.
	Hence $\{ t_0, t_1, t_2, \ldots, t_\ell \}$ is a generating set for $T$ that gives us an isomorphism $T \iso \ZZ_2 \times \ZZ_4 \times (\ZZ_2)^{\ell - 1}$.
	If, for some $i$, $2 \leq i \leq \ell$, we have $\beta(t_1, t_i)= -1$, replace $t_i$ for $t_it_0$ and so we have $\beta(t_1, t_it_0) = \beta(t_1, t_i) \beta(t_1, t_0) = +1$.
	We can, then, assume that the sets $\{ t_0, t_1 \}$ and $\{ t_2, \ldots, t_\ell \}$ are orthogonal with respect to $\beta$.
	Restricting $\beta$ and $\eta$ to $\ZZ_2 \times \ZZ_4$ and to $(\ZZ_2)^{\ell - 1}$ we find $\D'$ and $\D''$, respectively.
\end{proof}