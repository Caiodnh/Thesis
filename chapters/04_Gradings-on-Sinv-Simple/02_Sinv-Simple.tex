
\section[Finite dimensional superinvolution-simple superalgebras]{Finite dimensional superinvolution-simple \\ superalgebras}\label{sec:sinv-simple}

As an application of our results of \cref{sec:classification-grd-simple-with-sinv,grd-sinv-simple}, we can obtain the known classification of superinvolution-simple superalgebras over an algebraically closed field $\FF$ with $\Char \FF \neq 2$. 

We start with a general observation: 

\begin{lemma}
	Let $(R, \vphi)$ be a superalgebra with super-anti-automorphism.
	Then $Z(R)$ is $\vphi$-invariant.
\end{lemma}

\begin{proof}
	By \cref{lemma:center-is-graded}, with $G = \ZZ_2$, we have $Z(R) = Z(R)\even \oplus Z(R)\odd$, so it is sufficient to show that if $c \in Z(R)\even \cup Z(R)\odd$, then $\vphi(c) \in Z(R)$. 
	Let $r \in R\even \cup R\odd$.
	Since $c\vphi\inv (r) = \vphi\inv (r)c$, we can apply $\vphi$ on both sides and get $\sign{c}{r} r \vphi(c) = \sign{c}{r} \vphi(c) r$ and, hence, $r \vphi(c) = \vphi(c) r$.
\end{proof}

The following result was achieved by similar methods in \cite[Theorem 8.1]{Sh98} and \cite[Theorem 28]{MR2407903}. 

\begin{cor}\label{cor:Q-no-sinv-center}
	If $\Char \FF \neq 2$, the associative superalgebra $Q(n)$ does not admit a superinvolution.
\end{cor}

\begin{proof}
	The center of $Q(n)$ is isomorphic to $\FF1 \oplus \FF u$, where $u$ is an odd element with $u^2 = 1$.
	Let $\vphi$ be a super-anti-automorphism on $Q(n)$.
	Since $u$ is odd and central, $\vphi(u)$ is odd and central.
	Hence there is $\lambda \in \FF$ such that $\vphi(u) = \lambda u$.
	Using that $u^2 = 1$, we have $1 = \vphi(1) = \vphi(u^2) = - \vphi(u)^2 = - \lambda^2$.
	But then $\vphi^2 (u) = \lambda^2 u = -u \neq u$, hence $\vphi^2 \neq \id$.
\end{proof}

\begin{remark}
    \Cref{cor:Q-no-sinv-center} could also be deduced from \cref{lemma:sinv-type-M} (assuming $\FF$ is algebraically closed), which shows that a graded-division superalgebra $\D$ that is simple as a superalgebra and has nontrivial odd component does not admit a superinvolution (in the case $\D$ is of type $M$, this result appeared in \cite[Theorem 3]{BTT}, where it is erroneously stated that $\D$ does not admit a super-anti-automorphism). 
\end{remark}

It follows from \cref{thm:fd-simple-SA} and \cref{prop:only-SxSsop-is-simple} that, over an algebraically closed field $\FF$ with $\Char \FF \neq 2$, every finite dimensional superinvolution-simple superalgebra is of one of the following 3 types:

\begin{defi}\label{defi:types-sinv-simple}
    Let $(R, \vphi)$ be a finite dimensional superinvolution-simple superalgebra. 
    \begin{enumerate}[(i)]
        \item If $R$ is of type $M$, we say that $(R, \vphi)$ is of type $M$;
        \item If $(R, \vphi) \iso M(m,n) \times M(m,n)\sop$, for some $m,n \geq 0$, we say that $(R, \vphi)$ is of type $M\times M\sop$;
        \item If $R \iso Q(n) \times Q(n)\sop$, for some $n \geq 0$, we say that $(R, \vphi)$ is of type $Q \times Q\sop$.
    \end{enumerate}
\end{defi}

We can distinguish these types using the center:

\begin{prop}\label{prop:types-of-SA-via-center}
	Let $(R, \vphi)$ be a superalgebra with superinvolution.
	\begin{enumerate}[(i)]
		\item If $(R, \vphi)$ is of type $M$, then $(Z(R), \vphi) \iso (\FF, \id)$;\label{item:F-id}
		\item If $(R, \vphi)$ is of type $M\times M\sop$, then $(Z(R), \vphi)$ is isomorphic to the superalgebra with superinvolution in Example \ref{ex:FxF-iso-FZ2};\label{item:FZ2-exchg}
		\item If $(R, \vphi)$ is of type $Q\times Q\sop$, then $(Z(R), \vphi)$ is isomorphic to the superalgebra with superinvolution in Example \ref{ex:FZ2xFZ2sop-iso-FZ4}.\label{item:FZ4-exchg}
	\end{enumerate}
\end{prop}

\cref{prop:types-of-SA-via-center,item:F-id}

\begin{proof}
	Item \eqref{item:F-id} follows from the fact that $Z(M_{m+n}(\FF)) \iso \FF$.
	It is easy to check that $Z(S\times S\sop) = Z(S) \times Z(S)\sop$ for any superalgebra $S$, so items \eqref{item:FZ2-exchg} and \eqref{item:FZ4-exchg} follow from $Z(M_{m+n}(\FF)) \iso \FF$ and $Z(Q(n)) \iso Q(1)$.
\end{proof}

The classification of the superalgebras with superinvolution of types $M\times M\sop$ and $Q\times Q\sop$ is easier. 
The following result is valid over any field.

\begin{prop}\label{prop:classify-MxM-QxQ}
    Let $m,m',n,n' \geq 0$. % such that $m \geq n$ and $m' \geq n'$. 
    Then 
    \begin{enumerate}[(i)]
        \item $M(m,n) \times M(m,n)\sop \iso M(m',n') \times M(m',n')\sop$ if, and only if, either $m = m'$ and $n = n'$, or $m = n'$ and $n = m'$; 
        %
        \item $Q(n) \times Q(n)\sop \iso Q(n') \times Q(n')\sop$ if, and only if, $n = n'$.
    \end{enumerate}
\end{prop}

\begin{proof}
    Let $S_1$ and $S_2$ be simple superalgebras. 
    From \cref{lemma:iso-SxSsop} with trivial $G$, $S_1\times S_2\sop \iso S_1\times S_2\sop$ if, and only if, $S_1 \iso S_2$ or $S_1 \iso S_2\sop$. 
    In our case, we claim that  $S_1 \iso S_2$ or $S_1 \iso S_2\sop$ if, and only if, $S_1 \iso S_2$. 
    
    Indeed, if $S_1$ and $S_2$ are of type $M$, then $S_1 \iso S_2\sop$ implies $S_1 \iso S_2$ since $S_2\sop \iso S_2$ via supertransposition. 
    If $S_1$ and $S_2$ are of type $Q$, then $S_1 \iso S_2\sop$ implies $S_1 \iso S_2$ by dimension count. 
    
    The isomorphism condition follows from \cref{thm:fd-simple-SA}.
\end{proof}

It remains to classify the superinvolution-simple superalgebras of type $M$. 
It should be noted that $M(m,n)$ does not admit a superinvolution for all values of $m$ and $n$. 
Also, $M(m,n)$ endowed with different superinvolutions may lead to non-isomorphic superalgebras with superinvolution. 

\begin{defi}\label{defi:M(m-n-p_0)}
    Let $m,n \in \ZZ_{\geq 0}$, not both zero, and let $p_0 \in \ZZ_2$. 
    If $p_0 = \bar 0$ and $n$ is even, set 
    \[
        \sbox0{$\begin{matrix} 0& I_{n/2}\\ -I_{n/2}& 0\end{matrix}$}
        \Phi \coloneqq \left(\begin{array}{c|c}
                    I_m & 0\\
                    \hline
                    0 & \usebox{0}
                \end{array}\right).
    \]
    If $p_0 = \bar 1$ and $m = n$, let
    \[
        \Phi \coloneqq \left(\begin{array}{c|c}
                    0 & I_n\\
                    \hline 
                    I_n & 0
                \end{array}\right).
    \]
    It is straightforward to see that $\vphi(X) \coloneqq \Phi\inv X\stransp \Phi$ defines a superinvolution on $M(m,n)$. 
    We will denote the superalgebra with superinvolution $(M(m,n), \vphi)$ by $M^*(m,n, p_0)$. 
\end{defi}

We note that the superalgebras with superinvolution $M^*(m,n,p_0)$ are the ones used in the Introduction to define orthosymplectic Lie superalgebras (series $B$, $C$ and $D$) and the periplectic superalgebra (series $P$).

\begin{prop}\label{prop:iso-M-with-vphi}
    Every superalgebra with superinvolution of type $M$ is isomorphic to $M^*(m,n, p_0)$ for some $m,n \geq 0$ and $p_0 \in \ZZ_2$ as in \cref{defi:M(m-n-p_0)}. 
    Moreover, $M^*(m,n, p_0) \iso M^*(m',n', p_0')$ if, and only if $m = m'$, $n = n'$ and $p_0 = p_0'$.
\end{prop}

\begin{proof}
    This is precisely the case of graded-simple superalgebras considered in \cref{sec:classification-grd-simple-with-sinv}, but with trivial $G$ (and, hence, with $G^\# = \ZZ_2$). 
    By \cref{subsec:param-(R-phi)}, these graded superalgebras are parametrized by $(T, \beta, p, \eta, \kappa, g_0, \delta)$ where $(T, \beta, p)$ are associated to a graded-division superalgebra (see \cref{ssec:param-D-vphi}), and $(\eta, \kappa, g_0, \delta) \in \mathbf{I}(\D)$ (see \cref{defi:X(D)}). 
    
    Superalgebras of type $M$ are of the form $\End_\FF (U)$ for a finite dimensional superspace $U$, so we have $\D = \FF$ and, hence, $T$ is the trivial group and $\beta$, $p$ and $\eta$ are trivial maps. 
    Since we have only one possible $\eta$, we have only one equivalence class of $\eta$ and, therefore, our parametrization reduces to elements of $\mathbf{I}(\FF)_\eta^+$ (see \cref{prop:after-fixing-delta}), \ie, we choose $\delta = 1$ and the only parameters left are $\kappa\from G^\#/T = \ZZ_2 \to \ZZ_{\geq 0}$ and $p_0 \coloneqq g_0 \in G^\# = \ZZ_2$. 
    
    Then, the isomorphism classes are in bijection with the orbits by $\mc G$-action (see \cref{eq:mathcal-G}) on $\mathbf{I}(\FF)_\eta^+$. 
    In the present case, it is clear that $\mc G$ is trivial, hence the isomorphism classes are in bijection with points in $\mathbf{I}(\FF)_\eta^+$. 
    Let us describe these points and find a representative for each isomorphism class. 
    
    The map $\kappa\from G^\#/T \iso \ZZ_2 \to \ZZ_{\geq 0}$ is determined by the numbers $m \coloneqq \kappa(\bar 0)$ and $n \coloneqq \kappa(\bar 1)$, so our parametrization reduces to triples $(m,n,p_0)$ satisfying some conditions that come from \cref{defi:X(D)}. 
    The only conditions that are not automatically satisfied are \eqref{item:kappa-duality} and \eqref{item:kappa-parity}. 
    Condition \eqref{item:kappa-duality} is tautological if $p_0 = \bar 0$, and equivalent to $m=n$ if $p_0 = \bar 1$. 
    Condition \eqref{item:kappa-parity} simplifies to the following: if $p_0 = \bar 0$ and $g = \bar 1$, then $n = \kappa (g)$ is even. 
    In other words, conditions \eqref{item:kappa-duality} and \eqref{item:kappa-parity} become equivalent to $n$ being even if $p_0 = \bar 0$, and $m = n$ if $p_0= \bar 1$. 
    
    A representative for each point in $\mathbf{I}(\FF)_\eta^+$ can be found by using the matrices in \cref{prop:self-dual-components,prop:pair-of-dual-components}. 
\end{proof}