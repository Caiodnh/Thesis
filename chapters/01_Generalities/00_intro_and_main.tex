\chapter{Generalities on gradings}\label{sec:generalities}

% ** Intro to Chapter



The purpose of this chapter is to introduce the basic notions and constructions involving gradings and also to fix notation and terminology.  
% We will consider $G$ to be fixed group. 

Let $G$ be a group. 
In \cref{sec:graded-bimodules}, we introduce the concepts of homogeneous linear maps, elementary $G$-gradings on matrix (super)algebras, $G$-graded modules, tensor product of $G$-graded vector spaces and (super)algebras, and supercenter of an associative superalgebra.
In \cref{sec:Omega-algebras}, we define universal algebra, so we can handle gradings on different structures (such as algebras, superalgebras and superalgebras with superinvolution) in a uniform manner. 
In \cref{sec:g-hat-action}, under the assumptions that $G$ is abelian, $\FF$ is algebraically closed and $\Char \FF = 0$, we present the duality between $G$-gradings and $\widehat G$-actions for universal algebras. 
This allows us to ``transfer'' $G$-gradings between universal algebras with different signatures (see  \cref{def:universal-algebra,thm:transfer-of-gradings}); we will use this in \cref{chap:Lie} to get a classification of gradings on classical Lie superalgebras from a classification of gradings on associative superinvolution-simple superalgebras. 
% In that section, we will assume that $G$ is abelian, $\FF$ is algebraically closed and $\Char \FF = 0$.
% In \cref{sec:g-hat-action}, we present the duality between $G$-gradings on universal algebras and $\widehat G$-actions by automorphisms (in the case $\FF$ is algebraically closed and $\Char \FF = 0$); the main result (\cref{thm:transfer-of-gradings}) will be used in \cref{chap:Lie} to transfer gradings between universal algebras with different signatures (see \cref{def:universal-algebra}): associative superalgebras with superinvolution and Lie superalgebras. 
Finally, \cref{ssec:universal_group} is devoted to concepts related to refinement, coarsening and fine gradings, where we cannot keep the grading group fixed. 
We introduce set gradings on universal algebras and define equivalence of gradings and universal grading group in this context. 
We warn the reader that some terms appearing in \cref{ssec:universal_group} are not used consistently in the literature (see discussion in \cite[Section 2.7]{GS}); here we follow \cite{livromicha}. 

% The notion of grading is introduced in a very general way, not only for (super)algebras, but also for vector (super)spaces, (super)modules (\cref{sec:graded-bimodules}) and universal algebras (\cref{sec:Omega-algebras}). 
% In \cref{sec:g-hat-action} we discuss the correspondence between gradings and actions, which gives us a tool to transfer gradings between different (universal) algebras. 
% Finally, in \cref{ssec:universal_group}, we discuss the different ways to classify gradings (up to isomorphism or up to equivalence), as well as the concepts of coarsening and refinement of a grading.  


% --------------------






% ** Sections: 

\section{Gradings on vector spaces and (bi)modules}\label{sec:graded-bimodules}
% --------------------------------------------------

% Def: Graded vector space + Def: Shift by g
Let $G$ be a group. By a \emph{$G$-grading} on a vector space $V$ we mean simply a vector space decomposition $\Gamma:\,V = \bigoplus_{g \in G} V_g$ where the summands are labeled by elements of $G$. If $\Gamma$ is fixed, $V$ is referred to as a {\em $G$-graded vector space}. A subspace $W \subseteq V$ is said to be \emph{graded} if $W = \bigoplus_{g \in G} (W \cap V_g)$. We will refer to $\ZZ_2$-graded vector spaces as \emph{superspaces} and their graded subspaces as \emph{subsuperspaces}.

An element $v$ in a graded vector space $V = \bigoplus_{g \in G} V_g$ is said to be \emph{homogeneous} if $v\in V_g$ for some $g\in G$.
If $0\ne v\in V_g$, we will say that $g$ is the \emph{degree} of $v$ and write $\deg v = g$.
In reference to the canonical $\ZZ_2$-grading of a superspace, we will instead speak of the \emph{parity} of $v$ and write $|v| = g$.
Every time we write $\deg v$ or $|v|$, it should be understood that $v$ is a nonzero homogeneous element.

% Def: Grading on tensor product
\begin{defi}
	Given two $G$-graded vector spaces, $V=\bigoplus_{g\in G} V_g$ and $W=\bigoplus_{g\in G} W_g$, we define their tensor product to be the vector space $V\otimes W$
	together with the $G$-grading given by $(V \otimes W)_g = \bigoplus_{ab=g} V_{a} \otimes W_{b}$.
\end{defi}

The concept of grading on a vector space is connected to gradings on algebras by means of the following:

% Def: Homogeneous Map
\begin{defi}
	If $V=\bigoplus_{g\in G} V_{g}$ and $W=\bigoplus_{g\in G} W_{g}$ are two graded vector spaces and $T: V\rightarrow W$ is a linear map, we say that $T$ is \emph{homogeneous of degree $t$}, for some $t\in G$, if $T(V_g)\subseteq W_{tg}$ for all $g\in G$.
\end{defi}

% Def: Homogeneous Transformations
If $S: U\rightarrow V$ and $T: V\rightarrow W$ are homogeneous linear maps of degrees $s$ and $t$, respectively,
then the composition $T\circ S$ is homogeneous of degree $ts$.
% Def: Space of Homogeneous Transformations
We define the {\em space of graded linear transformations} from $V$ to $W$ to be:
%
\[ \Hom^{\text{gr}} (V,W) = \bigoplus_{g\in G} \Hom (V,W)_{g}\]
%
% Prop: End(V) is a graded algebra
where $\Hom (V,W)_{g}$ denotes the set of all linear maps from $V$ to $W$ that are homogeneous of degree $g$.
If we assume $V$ to be finite-dimensional then we have $\Hom(V,W)=\Hom^{\gr}(V,W)$ and, in particular, $\End (V) = \bigoplus_{g\in G} \End (V)_g$ is a graded algebra.

\begin{defi}\label{defi:elementary-grd}
    Elementary grading, on $\End(V)$ and on $M_n(\FF)$.
\end{defi}

% Prop: V is a graded module
We also note that $V$ becomes a graded module over $\End(V)$ in the following sense:

% Def: graded module
\begin{defi}\label{def:graded-module}
	Let $A$ be a $G$-graded algebra (associative or Lie) and let $V$ be a left module over $A$ that is also a $G$-graded vector space. 
	We say that $V$ is a \emph{graded left $A$-module} if $A_g \cdot V_h \subseteq V_{gh}$, for all $g$,$h\in G$. 
	The concepts of \emph{graded right $A$-module} and \emph{graded bimodule} are defined similarly.
\end{defi}

% Def: supermodule
Under the conditions of \cref{def:graded-module}, if $A$ is a superalgebra and $V$ is a superspace, then we call $V$ a \emph{$A$-supermodule}. 
A \emph{graded $A$-supermodule} is a $G^\#$-graded $A$-module.

% L1 is a graded module over L0
If we have a $G$-grading on a Lie superalgebra $L=L\even \oplus L\odd$ then, in particular, we have a grading on the Lie algebra $L\even$ and a grading on the space $L\odd$ that makes it a graded $L\even$-module. If we have a $G$-grading on an associative superalgebra $C=C\even \oplus C\odd$, then $C\odd$ becomes a graded bimodule over $C\even$.

If $ \Gamma$ is a $G$-grading on a vector space $V$ and $g\in G$, we denote by $\Gamma^{[g]} $ the grading given by relabeling the component
$V_h$ as $V_{hg}$, for all $h \in G$. This is called the \emph{(right) shift of the grading $\Gamma$ by $g$}.
We denote the graded space $(V, \,  \Gamma^{[g]})$ by $V^{[g]}$.

% Lemma: Shifts on (bi)modules
From now on, we assume that $G$ is abelian.
If $V$ is a graded module over a graded algebra (or a graded bimodule over a pair of graded algebras), then $V^{[g]}$ is also a graded (bi)module.
We will make use of the following partial converse (see e.g. \cite[Proposition 3.5]{paper-Qn}):

\begin{lemma}\label{lemma:simplebimodule}
	Let $A$ and $B$ be $G$-graded algebras and let $V$ be a finite-dimensional (ungraded) simple $A$-module or $(A,B)$-bimodule.  If $\Gamma$ and $\Gamma'$ are  two $G$-gradings that make $V$ a graded (bi)module, then $\Gamma'$ is a  shift of $\Gamma$.\qed
\end{lemma}

Certain shifts of grading may be applied to graded $\ZZ$- or $\ZZ_2$-superalgebras. In the case of a $\ZZ$-superalgebra $L=L^{-1}\oplus L^{0}\oplus L^{1}$, we have the following:

% Lemma: opposite directions
\begin{lemma}\label{lemma:opposite-directions}
	Let $L=L^{-1}\oplus L^0\oplus L^1$ be a $\ZZ$-superalgebra such that $L^1\, L^{-1}\neq 0$. If we shift the grading on $L^1$ by $g\in G$ and the grading on $L^{-1}$ by $g' \in G$, then we have a grading on $L$ if and only if $g' = g^{-1}$. \qed
\end{lemma}

We will describe this situation as \emph{shift in opposite directions}.

We finish this section with some usual concepts of algebra.

\begin{defi}\label{defi:center}
	Let $R$ be an algebra.
	The \emph{center} of $R$ is the set
	\[
		Z(R) = \{c\in R \mid cr = rc \text{ for all } r\in R \}.
	\]
\end{defi}

\begin{lemma}\label{lemma:center-is-graded}
	Suppose $G$ be an abelian group and let $R$ be a $G$-graded algebra.
	Then the center $Z(R)$ is $G$-graded subalgebra of $R$.
\end{lemma}

\begin{proof}
	Let $c \in Z(R)$ and write $c = \sum_{g \in G} c_g$, where $c_g \in R_g$ for all $g \in G$.
	For every homogeneous $r \in R$, we have that
	%
	\begin{align*}
		\big(\sum_{g\in G} c_g\big)r = r \big(\sum_{g\in G} c_g\big).
	\end{align*}
	%
	Comparing the components of degree $gh = hg$, where $h = \deg r$, we conclude that $rc_g = c_g r$ for all $g \in G$.
	By linearity, $r c_g = c_g r$ for all $r\in R$, hence $c_g \in Z(R)$.
\end{proof}

Note that for $G=\ZZ_2$, Lemma \ref{lemma:center-is-graded} implies that the center of a superalgebra is a subsuperalgebra. 


\begin{defi}\label{defi:supercenter}
	Let $R$ be a superalgebra.
	The \emph{supercenter} of $R$ is the subsuperalgebra $sZ(R) = sZ(R)\even \oplus sZ(R)\odd$, where
	\[
		sZ(R)^i = \{c\in R^i \mid cr = (-1)^{i|r|} rc \text{ for all } r\in R\even \cup R\odd \}.
	\]
\end{defi}

The proof of \cref{lemma:center-is-graded} can be easily adapted to show that, if $G$ is abelian, then the supercenter of a graded superalgebra is a graded subsuperalgebra.


\section{Gradings on universal algebras}\label{sec:Omega-algebras}
% \texorpdfstring{$\Omega$}{Omega}-algebras

In the \hyperref[chap:intro]{Introduction}, we defined different graded structures: vector spaces, algebras, superspaces, superalgebras and superalgebras endowed with a super-anti-automorphism. 
The language of \emph{universal algebra} allows us to consider all these in a uniform fashion. 
It is particularly convenient to formulate results like \cref{thm:transfer-of-gradings} , which allows us to transfer gradings between different structures: in \cref{chap:Lie}, we will use it to transfer our classification of $G$-gradings on superinvolution-simple superalgebras (achieved in \cref{chap:grds-sinv-simple}) to classical Lie superalgebras. 
We will also use this language in \cref{ssec:universal_group} to give unified definitions for fine gradings and universal group (\cref{defi:fine-grading,defi:universal-group}), which is usually done \emph{ad hoc} for different cases. 

We note that universal algebras differs are usually defined in the category of sets (see, \eg, \cite{Cohn_universal}), but we will work in the category of vector spaces over $\FF$. 
This approach was used in \cite{Razmyslov} and has recently been applied to gradings (and graded identities) in \cite{MR3886336}. 

\begin{defi}
    An \emph{$n$-ary operation} on a vector space $V$ is 
    % a multilinear map $V^n \to V$ or, equivalently, 
    a linear map $V^{\tensor n} \to V$, where $V^{\tensor n} \coloneqq \underbrace{V\otimes\cdots\otimes V}_{n \text{ times}}$. 
\end{defi}

In other words, an $n$-ary operation is a multilinear map $V^n \to V$. 
In the case $n = 0$, we will follow the convention that $V^{\tensor 0} \coloneqq \FF$. 
In particular, a $0$-ary operation is determined by its value on $1 \in \FF$ and, hence, $0$-ary operations are constants in $V$. 

\begin{defi}\label{def:universal-algebra}
	A \emph{signature} $\Omega$ is a set with a partition $\Omega = \bigcup_{n \geq 0} \Omega_n$.
	%
	An \emph{$\Omega$-algebra} or a \emph{universal algebra with signature $\Omega$} is a vector space $A$ endowed with $n$-ary operations $\omega^A$ for each $\omega \in \Omega_n$, for all $n \geq 0$. 
	%
	A \emph{homomorphism between $\Omega$-algebras} $A$ and $B$ is a linear map such that for every $\omega \in \Omega_n$ we have
	\[
		\forall a_1, \ldots, a_n \in A, \quad 
		\psi( \omega^A (a_1 \tensor \cdots \tensor a_n) ) = \omega^B ( \psi(a_1) \tensor \cdots \tensor \psi(a_n) ). 
	\]
\end{defi}

\begin{notation}
    When dealing with a fixed $\Omega$-algebra $A$, we will usually drop the superscript $A$ in the operations $\omega$, \ie, we will identify the signature $\Omega$ with its corresponding set of operations on $A$. 
\end{notation}

\begin{ex}\label{ex:omega-vec-space}
	A vector space is an $\Omega$-algebra with $\Omega = \emptyset$.
\end{ex}

\newcommand{\multiplication}{\cdot}

\begin{ex}\label{ex:omega-algebra}
	An algebra $A$ in the usual sense, with a bilinear product $\multiplication : A\tensor A \to A$, is an $\Omega$-algebra with $\Omega = \Omega_2 = \{ \multiplication \}$. 
\end{ex}

\begin{ex}\label{ex:Omega-unity-inv}
    A unital algebra $A$ with product $\multiplication$ and unity element $1_A \in A$ is an $\Omega$-algebra with $\Omega = \Omega_0 \cup \Omega_2$ where $\Omega_0 = \{ \omega_0 \}$, with $\omega_0\from \FF \to A$ determined by $\omega_0 (1) \coloneqq 1_A$, and $\Omega_2 = \{ \multiplication \}$. 
    An algebra $A$ with product $\multiplication$ and involution $\vphi\from A \to A$ is an $\Omega$-algebra with $\Omega = \Omega_1 \cup \Omega_2$, where $\Omega_1 = \{ \vphi \}$ and $\Omega_2 = \{ \multiplication \}$. 
    A unital algebra $A$ with involution $\vphi$ is an $\Omega$-algebra with $\Omega = \Omega_0 \cup \Omega_1 \cup \Omega_2$, where $\Omega_0 = \{ \omega_0 \}$, $\Omega_1 = \{ \vphi \}$ and $\Omega_2 = \{ \multiplication \}$. 
\end{ex}


\begin{ex}\label{ex:omega-superspace}
	A superspace $V = V\even \oplus V\odd$ can be seen as an $\Omega$-algebra by taking $\Omega = \Omega_1 = \{ \pi_\bz, \pi_\bo \}$, where $\pi_\bz, \pi_\bo \from V \to V$ are the projections onto the components $V\even$ and $V\odd$, respectively. 
	A universal algebra $V$ with this signature is a superspace \IFF for all $x \in V$, we have:
	%
	\begin{enumerate}[(i)]
		\item $\pi_{\bar 0}(x) + \pi_{\bar 1}(x) = x$; \label{item:sum-projections}
% 		\item $(\pi_{\bar 0} \circ \pi_{\bar 1}) (v)
% 		= (\pi_{\bar 1} \circ \pi_{\bar 0}) (v)
% 		= 0$.
		\item $\pi_{\bar 0} ( \pi_{\bar 1} (x) )
		= \pi_{\bar 1} ( \pi_{\bar 0} (x) )
		= 0$. \label{item:orthogonal-projections}
	\end{enumerate}
\end{ex}


\begin{ex}\label{ex:omega-alg-SA}
	A superalgebra $A = A\even \oplus A\odd$ is an $\Omega$-algebra with $\Omega = \Omega_1 \cup \Omega_2$, where $\Omega_2 = \{ \multiplication \}$, $\Omega_1 = \{ \pi_{\bar 0}, \pi_{\bar 1} \}$.
	An algebra $A$ with this signature is a superalgebra if, and only if, we have identities \eqref{item:sum-projections} and \eqref{item:orthogonal-projections} as above and, for all $x,y\in A$ and $i, j\in \ZZ_2$,
	\begin{enumerate}[(i)]
        \setcounter{enumi}{2}
		\item $\pi_i^{} (x) \multiplication \pi_j^{} (y) = \pi_{i+j}^{}( \pi_i^{} (x) \multiplication \pi_j^{} (y) )$.
	\end{enumerate}
\end{ex}

\begin{ex}\label{ex:omega-alg-SA-sinv}
    Similarly to \cref{ex:Omega-unity-inv}, a superalgebra $A = A\even \oplus A\odd$ with superinvolution $\vphi\from A \to A$ is an  $\Omega$-algebra with $\Omega = \Omega_1 \cup \Omega_2$ where $\Omega_1 = \{\pi_{\bar 0}, \pi_{\bar 1}, \vphi \}$ and  $\Omega_2 = \{ \multiplication \}$.
\end{ex}

\begin{remark}
    The signatures in \cref{ex:omega-superspace,ex:omega-alg-SA} can be generalized for $G$-graded spaces/algebras, and, if $G$ is finite, the axioms can be stated as identities, which allows one to define these objects as varieties of algebras (see \cite[Section 2]{MR3886336}), but this is not the approach we are going to follow. 
\end{remark}

\begin{defi}\label{def:grds-on-Omega-algebras}
	A \emph{$G$-grading on an $\Omega$-algebra $A$} is a $G$-grading on its underlying vector space such that, for all $\omega \in \Omega$, the operation $\omega^A\from A^{\tensor n} \to A$ is degree preserving if we consider $G$-gradings on tensor products $A^{\tensor n}$ induced by the $G$-grading on $A$ (see \cref{defi:tensorProduct}). 
	If $\Gamma$ is fixed, we say that $A$ is a \emph{$G$-graded $\Omega$-algebra}. 
	A \emph{homomorphism of $G$-graded $\Omega$-algebras} is a degree-preserving homomorphism of the underlying $\Omega$-algebras. 
	Given two $G$-gradings $\Gamma$ and $\Delta$ on a fixed $\Omega$-algebra $A$, we say that $\Gamma$ is \emph{isomorphic} to $\Delta$ if $(A, \Gamma) \iso (A, \Delta)$.  
\end{defi}

\phantomsection\label{lemma:1-has-deg-e}

It is straightforward to check that, for each of the \cref{ex:omega-vec-space,ex:omega-algebra,ex:Omega-unity-inv,ex:omega-superspace,ex:omega-alg-SA,ex:omega-alg-SA-sinv}, the notions of homomorphism and $G$-grading as $\Omega$-algebras coincide with the notions of homomorphism and $G$-grading we had before. 
Note that \cref{def:grds-on-Omega-algebras} entails that, in a unital algebra, the unity element is homogeneous of degree $e$, but this is automatic according to \cref{def:grading}: 
if we write $1 = \sum_{g\in G} a_g$, with $a_g \in A_g$, then, for any $h\in G$, $a_h = a_h 1 = \sum_{g\in G} a_h a_g \in A_h$ and, since $a_h a_g \in A_{hg}$, the only nonzero element in this sum $a_h a_e$. 
It follows that $a_h = a_h a_e$ and, hence, $1 = \sum_{g\in G} a_g = \sum_{g\in G} a_g a_e = 1 a_e = a_e \in A_e$. 

% , note that, by it is easy to verify that $1_A$ automatically belongs to the homogeneous component $A_e$.

\section{\texorpdfstring{$G$}{G}-gradings and \texorpdfstring{$\widehat{G}$}{G-hat}-actions}\label{sec:g-hat-action}

In this section, $G$ will be assumed to be an \emph{abelian} group. 
% In this section, 
We will introduce an important tool in the theory of gradings: the duality between $G$-gradings and $\widehat G$-actions. 
Here $\widehat G$ denotes the \emph{group of characters} of $G$, \ie, $\widehat G$ is the group whose elements are the group homomorphisms $\chi\from G \to \FF^\times$, with point-wise multiplication of maps. 
We will also assume that $\FF$ is algebraically closed and $\Char \FF = 0$, since this is the only case we need in this work. 
The reader interested in arbritary fields can refer to \cite{livromicha}. 

% the main advantage of this approach is to have formal statement \cref{thm:transfer-of-gradings}, in the following subsection, which is a result that has been used informally in the literature for specific cases (see, \eg,  \cite[Remark 1.40]{livromicha}). 

\begin{defi}
    Given a $G$-grading $\Gamma : V = \bigoplus_{g\in G} V_g$ on a vector space $V$, we define a $\widehat G$-action by
    \[
        \forall \chi \in \widehat G, \, g\in G, \, v_g \in V_g, \quad \chi \cdot v_g \coloneqq \chi(g) v_g. 
    \]
    The corresponding representation map will be denotes by $\eta_\Gamma \from \widehat G \to \GL (V)$. 
\end{defi}

With our assumptions on $G$ and $\FF$, it is well known that $\widehat G$ separate points, \ie, given distinct elements $g, g'\in G$, there is a character $\chi \in \widehat G$ such that $\chi(g) \neq \chi(g')$. 
Hence, we have 
\[
    \forall g\in G, \quad V_g = \{ v\in V \mid \forall\, \chi \in \widehat G, \, \chi \cdot v = \chi(g)v \},
\]
Thus, we can recover the $G$-grading $\Gamma$ from its corresponding $\widehat G$-action $\eta_\Gamma$. 

% As usual, an \emph{automorphism} of $A$ is a bijective homomorphism from $A$ to itself, and the group of automorphisms is denoted by $\Aut(A)$.

\begin{prop}\label{prop:g-hat-Aut-A}
	Let $A$ be an $\Omega$-algebra and $\Gamma$ be a $G$-grading on its underlying vector space.
	Then $\Gamma$ is a $G$-grading on $A$ 
% 	(see \cref{def:grds-on-Omega-algebras}) 
	if, and only if, $\eta_\Gamma(\widehat G) \subseteq \Aut(A)$.
\end{prop}

\begin{proof}
    Let $\omega \in \Omega_n$ and let $a_1, \ldots, a_n \in A$ be homogeneous elements of degrees $g_1, \ldots, g_n \in G$, respectively. 
    Note that $a_1\tensor \cdots \tensor a_n \in A^{\tensor n}$ has degree $g_1 \cdots g_n$. 
    
% 	First, let us assume that $\Gamma$ is a grading on the $\Omega$-algebra $A$. 

    The element $\omega^A (a_1\tensor \cdots \tensor a_n)$ has degree $g_1 \cdots g_n$ \IFF 
    \[\label{eq:aut-g-hat}
        \forall \chi \in \widehat G, \quad
        \chi \cdot \omega^A (a_1\tensor \cdots \tensor a_n) 
		= \chi(g_1 \cdots g_n) \, \omega^A(a_1\tensor \cdots \tensor a_n).
    \]
	Since we have
	\begin{align*}
		\chi(g_1 \cdots g_n) \, \omega^A(a_1\tensor \cdots \tensor a_n) 
		& =\chi(g_1) \cdots \chi(g_n) \, \omega^A(a_1\tensor \cdots \tensor a_n) \\
		& = \omega^A \big(\chi(g_1)a_1\tensor \cdots \tensor \chi(g_n)a_n \big) \\
		& = \omega^A(\chi \cdot a_1 \tensor \cdots \tensor  \chi \cdot a_n),
	\end{align*}
%	
	\cref{eq:aut-g-hat} holds for all homogeneous $a_1, \ldots, a_n \in A$ \IFF $\eta_\Gamma (\chi)$ is an automorphism for every $\chi \in \widehat G$.  
\end{proof}

% ---

The following is straightforward:

\begin{prop}\label{prop:iso-g-hat-action}
    Two $G$-gradings $\Gamma$ and $\Delta$ on an $\Omega$-algebra $A$ are isomorphic \IFF there is an automorphism $\psi \in \Aut(A)$ such that $\eta_\Delta(\chi) = \psi \circ \eta_\Gamma(\chi) \circ \psi\inv$, for all $\chi \in \widehat G$. \qed
\end{prop}

% \begin{proof}
%     Let $\chi \in \widehat G$ be a fixed character and set $\vphi_\Gamma = \eta_\Gamma(\chi)$ and $\vphi_\Delta = \eta_\Delta(\chi)$. 
%     The gradings $\Gamma$ and $\Delta$ are isomorphic \IFF there is $\psi\from (A, \Gamma) \to (A, \Delta)$. 
    
%     $\vphi_\Gamma (v_g) = \chi(g) v_g$
    
%     $\vphi_\Delta ( \psi(v_g) ) = \chi(g) \psi(v_g)$
    
%     Hence, $\vphi_\Delta = \psi \circ \vphi_\Gamma \circ \psi\inv$.
% \end{proof}

% ---

% \input{chapters/01_Generalities/10_excerpt}

If $G$ is a finite abelian group, it is well known (see, \eg, \cite[\S 1.2]{FultonAndHarris}) that every action by $\widehat G$ on a finite dimensional vector space $V$ is \emph{diagonalizable}, \ie, $V$ can be written as a direct sum of subspaces in which each $\chi \in \widehat G$ acts as a nonzero scalar $\lambda_\chi$. 
It follows that the map $\chi \mapsto \lambda_\chi$ is a character of $\widehat G$ and, by duality, there is a unique $g \in G$ such that $\lambda_\chi = \chi(g)$, for every $\chi \in \widehat G$. 
In summary, every $\widehat G$-action corresponds to a $G$-grading.

This can be extended to the case of finitely generated $G$ by considering actions of \emph{algebraic groups} (for a background on algebraic groups, we refer to \cite{MR1064110}, \cite{Arzhantsev-notes} or \cite[Appendix A]{livromicha}). 
Both $\widehat G$ and $\GL (V)$ have natural structures of algebraic groups  (assuming $\dim V < \infty$) and the representation $\eta_\Gamma \from \widehat G \to \GL (V)$ is a homomorphism of algebraic groups. 
The algebraic group $\widehat G$ is a \emph{quasitorus}, \ie, $\widehat G \iso (\FF^\times)^n \times G_f$ where $G_f$ is a finite abelian group. 
Every algebraic representation of a quasitorus $H$ on a finite dimensional vector space $V$ is diagonalizable (see, \eg, \cite[Chapter 3, \S 2, Theorem 3]{MR1064110} or \cite[Theorem 1.6.13]{Arzhantsev-notes}), \ie, $V$ can be written as a direct sum of subspaces in which each $h \in H$ acts as a nonzero scalar $\lambda_h$. 
It can be shown that the map $h \mapsto \lambda_h$ is an \emph{algebraic} character, \ie, a homomorphism of algebraic groups $H \to \FF^\times$. 
The duality can be extended to this case: for every algebraic character $\lambda\from \widehat G \to \FF^\times$, there is a unique element $g\in G$ such that $\lambda_\chi = \chi(g)$. 
We then have the following:

\begin{prop}\label{prop:g-hat-1-to-1}
    Let $V$ be a finite dimensional vector space and assume $G$ is finitely generated. 
    Then the mapping $\Gamma \mapsto \eta_\Gamma$ is a one-to-one correspondence between $G$-gradings on $V$ and homomorphisms of algebraic groups $\widehat G \to \GL(V)$. \qed
\end{prop}

% ---

For an $\Omega$-algebra $A$, it is easy to see that $\Aut (A)$ is a (Zariski) closed subgroup of $\GL (A)$, hence an algebraic group.
Therefore, \cref{prop:g-hat-Aut-A,prop:g-hat-1-to-1} imply that the following is well defined:

\begin{defi}
    Let $A$ and $B$ be finite dimensional universal algebras, not necessarily with the same signature, and assume $G$ is finitely generated.  
    Given a homomorphism of algebraic groups $\theta\from \Aut(A) \to \Aut(B)$ and a $G$-grading $\Gamma$ on $A$, we define $\theta(\Gamma)$ to be the $G$-grading on $B$ corresponding to the homomorphism $\theta \circ \eta_\Gamma\from \widehat G \to \Aut(B)$. 
\end{defi}

Using \cref{prop:iso-g-hat-action}, we get that if $\Gamma$ and $\Delta$ are isomorphic $G$-gradings on $A$, then $\theta(\Gamma)$ and $\theta(\Delta)$ are isomorphic $G$-gradings on $B$. 
Finally, we note that we can drop the hypothesis that $G$ is finitely generated: every $G$-grading $\Gamma$ on $A$ can be seen as a grading by the subgroup generated by $\supp \Gamma$, which can be used to define $\theta (\Gamma)$. 
% because, when working with a finite number of finite dimensional graded spaces, we can replace $G$ by its subgroup generated by the supports. 
We summarize these considerations in the following: 

\begin{thm}\label{thm:transfer-of-gradings}
    Suppose $\FF$ is an algebraically closed field of characteristic $0$. 
    Let $G$ be an abelian group and let $A$ and $B$ be finite dimensional universal algebras, not necessarily with the same signature. 
    If there is a isomorphism of algebraic groups $\Aut(A) \to \Aut(B)$, then there is a bijection between the $G$-gradings on $A$ and the $G$-gradings on $B$ preserving the isomorphism classes. \qed
\end{thm}

We note that this sort of transfer, between algebras with different signatures, has been used in other works, but without having the result stated formally (see, \eg,  \cite[Remark 1.40]{livromicha}).

% ---


\section{Refinement, coarsening and equivalence}\label{ssec:universal_group}
% ---

In this section, we will introduce some concepts that do not involve a fixed grading group. 
For these, it is useful to have a ``group free'' notion of grading:

\begin{defi}\label{defi:set-grading}
    A \emph{set grading $\Gamma$ on a vector space $V$} is a vector space decomposition indexed by elements of a set $S$, \ie, 
    $\Gamma : V = \bigoplus_{s\in S} V_s$. 
    % \[ 
    %     \Gamma : V = \bigoplus_{s\in S} V_s.
    % \]
    If $V$ is a superspace, we further impose that each component $V_s$ is a subsuperspace. 
    When endowed with a fixed set grading $\Gamma$, we say that $V$ is a \emph{set graded vector space}. 
\end{defi}

\begin{defi}\label{defi:ref-coars}
    Let $\Gamma : V = \bigoplus_{s\in S} V_s$ and $\Delta : V = \bigoplus_{t \in T} V_{t}$ be set gradings on a vector space $V$. 
    We say that $\Gamma$ is a \emph{refinement} of $\Delta$, or that $\Delta$ is a \emph{coarsening} of $\Gamma$, if for every $s \in S$ there is $t \in T$ such that $V_s \subseteq V_t$. 
    If, for some $s \in S$, this inclusion is strict, we say that the refinement/coarsening is \emph{proper}. 
\end{defi}


As in \cref{support}, we define the \emph{support} of a set grading $\Gamma : V = \bigoplus_{s\in S} V_s$ to be the set $\supp \Gamma \coloneqq \{ s \in S \mid V_s \neq 0 \}$. 
Note that we can always replace $S$ by $\supp \Gamma$. 
If $\Delta$ is a coarsening of $\Gamma$ as above and $s\in \supp \Gamma$, then there is a unique element $t \in T$ such that $V_s \subseteq V_t$. 
This motivates the following:

\begin{defi}\label{coars-induced}
    Let $V$ be a vector space and let $\Gamma : V = \bigoplus_{s\in S} V_s$ be set grading. 
    Given a set $T$ and a map $\alpha\from S \to T$, the \emph{coarsening of $\Gamma$ induced by $\alpha$} is the set grading 
    \[
        {}^{\alpha}\Gamma : V = \bigoplus_{t\in T} V_t,
    \]
    where 
    \[
        V_t \coloneqq \bigoplus_{s \in \alpha\inv (t)} V_s.
    \]
\end{defi}

Before defining set gradings on $\Omega$-algebras, we need the following:

\begin{defi}\label{defi:graded-map}
    Let $V = \bigoplus_{s\in S} V_s$ and $W = \bigoplus_{t\in T} W_t$ be set graded vector spaces. 
    %
    A linear map $f\from V \to W$ is said to be  \emph{graded} if for any $s \in S$, there is $t \in T$ such that $f(V_s) \subseteq W_t$. 
\end{defi}

Note that, by definition, a grading $\Gamma$ is a refinement of a grading $\Delta$ on a vector space $V$ \IFF the identity map seen as $(V, \Gamma) \to (V, \Delta)$ is a graded map. 

\begin{defi}
    Let $V = \bigoplus_{s\in S} V_s$ and $W = \bigoplus_{t\in T} W_t$ be set graded vector spaces. 
    The \emph{tensor product} of $V$ and $W$ is the the vector space $V \tensor W$ endowed with the grading
    \[
        \Gamma : V \tensor W = \bigoplus_{(s,t) \in S \times T} V_s \tensor W_t. 
    \]
\end{defi}

% \begin{remark}
    We note that the $G$-grading on the tensor product of two  vector spaces (\cref{defi:tensorProduct}) is the coarsening of the set grading in \cref{defi:graded-map} induced by the map $\alpha\from G\times G \to G$ given by $\alpha (g,h) \coloneqq gh$, for all $g,h \in G$. 
% \end{remark}

\begin{defi}
    A \emph{set grading on an $\Omega$-algebra $A$} is a set grading $\Gamma : A = \bigoplus_{s\in S} A_s$ on the underlying vector space of $A$ such that $\omega^A\from A^{\tensor n} \to A$ is a graded linear map for all $\omega \in \Omega$. 
\end{defi}

In particular, if $A$ is an algebra in the usual sense, a set grading on $A$ is a vector space decomposition $\Gamma : A = \bigoplus s_{s\in S} A_s$ such that, for any $s_1,s_2\in S$ there exists $s_3\in S$ such that $A_{s_1} A_{s_2} \subseteq A_{s_3}$.

\begin{defi}\label{defi:equivalence}
    Let $A$ and $B$ be $\Omega$-algebras endowed, respectively, with set gradings ${\Gamma : A = \bigoplus_{s \in S} A_{s}}$ and ${\Delta = \bigoplus_{t \in T} A_{t}}$. 
    An \emph{equivalence} $\psi\from A \to B$ is an isomorphism of $\Omega$-algebras such that both $\psi$ and $\psi\inv$ are graded maps. 
    If $A = B$ and there is an equivalence $\psi\from (A, \Gamma) \to (A, \Delta)$, we say that $\Gamma$ and $\Delta$ are \emph{equivalent gradings}. 
\end{defi}

Note that an equivalence $\psi\from A \to B$ determines a bijection $\alpha \from \supp \Gamma \to \supp \Delta$ by $\vphi(A_s) = B_{\alpha(s)}$. 

We will now bring groups and group gradings back to the picture:

\begin{defi}
    We say that a set grading $\Gamma$ on an $\Omega$-algebra $A$ can be \emph{realized as an (abelian) group grading} if there is an (abelian) group $G$ and a injective map $\alpha\from \supp \Gamma \to G$ such that ${}^{\alpha}\Gamma$ is a $G$-grading on $A$. 
\end{defi}

\begin{defi}\label{defi:fine-grading}
    Let $A$ be an $\Omega$-algebra and let $\Gamma$ be a set grading on $A$. 
    We say that $\Gamma$ is a \emph{fine (abelian) group grading} if it can be realized as an (abelian) group grading but no proper refinement of $\Gamma$ can. 
\end{defi}

When a grading can be realized as an (abelian) group grading, it can be done in many different ways. 
But there is a special realization that has a universal property:

\begin{defi}\label{defi:universal-group}
    Let $A$ be an $\Omega$-algebra and let $\Gamma$ be a set grading on $A$. 
    A \emph{universal (abelian) group} of $\Gamma$ is a group $G$ together with a map $\iota\from \supp\Gamma \to G$ such that ${}^{\iota}\Gamma$ is a $G$-grading and, for every (abelian) group $G'$ and map $\iota'\from \supp \Gamma \to G'$ such that ${}^{\iota'}\Gamma$ is a $G'$-grading, there is a unique group homomorphism $\alpha\from G \to G'$ such that $\iota' = \alpha \circ \iota$, \ie, the following diagram commutes:
    %
	\begin{center}
		\begin{tikzcd}
            S \arrow[to = G, "\iota"] \arrow[to = H, "\iota'"]
            && |[alias = G]| G \arrow[to = H, dashed, "\alpha"]\\
            &&\\
            && |[alias = H]| G'
        \end{tikzcd}
	\end{center}
\end{defi}

Clearly, $\Gamma$ can be realized as an (abelian) group grading \IFF the map $\iota$ above is injective. 
Also, we can construct a universal (abelian) group using generators and relations: we take $\supp \Gamma$ as the set of generators and, for each $n\geq 0$ and $\omega \in \Omega_n$, we consider relations $s_1 \cdots s_n = s_{n+1}$ for all $s_1, \ldots, s_n, s_{n+1} \in \supp \Gamma$ such that $0 \neq \omega^A (A_{s_1} \tensor \cdots \tensor A_{s_n}) \subseteq A_{s_{n+1}}$. 

\begin{remark}\label{rmk:coars-grp-induced}
    Let $A$ be an $\Omega$-algebra, $G$ and $G'$ be groups and $\alpha\from G \to G'$ be a group homomorphism. 
    If $\Gamma : A = \bigoplus_{g\in G} A_g$ is a $G$-grading, then it is easy to see that ${}^{\alpha} \Gamma$ is a $G'$-grading. 
    We note that, by \cref{defi:universal-group}, if $G$ is the universal group of $\Gamma$, then every $G'$-grading that is a coarsening of $\Gamma$ is obtained this way.  
\end{remark}

By means of the duality between gradings and actions outlined in \cref{sec:g-hat-action}, the fine abelian group gradings on a finite dimensional algebra $A$ over an algebraically closed field of characteristic $0$ correspond to maximal quasitori in the algebraic group $\Aut(A)$. 
Moreover, the group of algebraic characters of a maximal quasitorus is the universal abelian group of the corresponding grading.  

We conclude this chapter with some comments about the two types of classification mentioned in the \hyperref[intro-equiv]{Introduction}: fine gradings up to equivalence and $G$-gradings up to isomorphism. 
%
Any group grading on a finite dimensional algebra $A$ is a coarsening of a fine group grading. 
So, if we have a classification of fine group gradings on $A$ up to equivalence and know their universal groups, we can obtain any $G$-grading as $^{\alpha}\Gamma$ for some fine grading $\Gamma$ and homomorphism $\alpha$ from the universal group of $\Gamma$ to $G$ (see \cref{defi:universal-group}). 
However, $\Gamma$ and $\alpha$ are not unique and in practice it is difficult to determine when two such induced $G$-gradings $^{\alpha}\Gamma$ and $^{\beta}\Delta$ are isomorphic. 

On the other hand, if we know all $G$-gradings on $A$, for any $G$, we can try to determine which of them are fine and compute their universal groups. 
This was done for simple Lie superalgebras of series $Q$, $P$ and $B$ in \cite{paper-Qn,paper-MAP,Helens_thesis}. 

% ---

% The following result appears to be ``folklore''. We include a proof for completeness.


% \begin{lemma}\label{lemma:universal-grp}
% 	Let $\mathcal{F}=\{\Gamma_i\}_{i\in I}$, be a family of pairwise nonequivalent fine (abelian) group gradings on a $\Omega$-algebra $A$, where $\Gamma_i$ is a $G_i$-grading and $G_i$ is generated by $\supp \Gamma_i$. 
% 	Suppose that $\mathcal{F}$ has the following property:
% 	for any grading $\Gamma$ on $A$ by an (abelian) group $H$, there exists $i\in I$ and a homomorphism $\alpha:G_i\to H$ such that $\Gamma$
% 	is isomorphic to ${}^\alpha\Gamma_i$. Then
% 	%
% 	\begin{enumerate}[(i)]
% 		\item every fine (abelian) group grading on $A$ is equivalent to a unique $\Gamma_i$; \label{item:all-fine}
% 		\item for all $i\in I$, $G_i$ is the universal (abelian) group of $\Gamma_i$. \label{item:Gi-is-univeral}
% 	\end{enumerate}
% \end{lemma}

% \begin{proof}
% 	Let $\Gamma$ be a fine grading on $A$, realized over its universal group $H$. 
% 	Then there is $i\in I$ and $\alpha: G_i \to H$ such that ${}^\alpha \Gamma_i \iso \Gamma$. 
% 	Writing $\Gamma_i: A = \bigoplus_{g\in G_i} A_g$ and $\Gamma: A = \bigoplus_{h\in H} B_h$, we then have $\vphi \in \Aut(A)$ such that
% 	\[
% 		\vphi\,\big( \bigoplus_{g\in \alpha\inv (h)} A_g \big) = B_h
% 	\]
% 	for all $h\in H$. 
% 	Since $\Gamma$ is fine, we must have $B_h \neq 0$ if, and only if, there is a unique $g\in G_i$ such that $\alpha(g) = h$, $A_g\neq 0$ and $\vphi(A_g) = B_h$. 
% 	Equivalently, $\alpha$ restricts to a bijection $\supp(\Gamma_i) \to \supp(\Gamma)$ and $\vphi(A_g) = B_{\alpha(g)}$ for all $g \in S_i:= \supp (\Gamma_i)$. This proves assertion $(i)$.

% 	Let $G$ be the universal group of $\Gamma_i$. 
% 	It follows that, for all $s_1, s_2, s_3 \in S_i$,
% 	%
% 	\begin{equation*} \label{eq:relations-unvrsl-grp}
% 		\begin{split}
% 			& s_1s_2 = s_3 \text{ is a defining relation of } G \\
% 			\iff & 0 \neq A_{s_1} A_{s_2} \subseteq A_{s_3}\\
% 			\iff & 0 \neq B_{\alpha(s_1)} B_{\alpha(s_2)} \subseteq B_{\alpha (s_3)}\\
% 			\iff & \alpha(s_1)\alpha(s_2) = \alpha(s_3) \text{ is a defining relation of } H.
% 		\end{split}
% 	\end{equation*}
% 	%
% 	Therefore, the bijection $\alpha\restriction_{S_i}$ extends uniquely to an isomorphism $\widetilde{\alpha}: G\rightarrow H$.

% 	By the universal property of $G$, there is a unique homomorphism $\sigma: G\to G_i$ that restricts to the identity on $S_i$. Hence, the following diagram commutes:
% 	%
% 	\begin{center}
% 		\begin{tikzcd}
% 			G \arrow[to=Gi, "\sigma"] \arrow[to = H, "\widetilde{\alpha}"]&&\\
% 			&& |[alias=H]|H\\
% 			|[alias=Gi]|G_i \arrow[to=H, "\alpha"]&&
% 		\end{tikzcd}
% 	\end{center}
% 	%

% 	Since $\widetilde{\alpha}$ is an isomorphism, $\sigma$ must be injective. But $\sigma$ is also surjective since $S_i$ generates $G_i$. 
% 	Hence $G_i$ is isomorphic to $G$. Since $\Gamma$ was an arbitrary fine grading, for each given $j\in I$, we can take $\Gamma = \Gamma_j$ (hence, $i=j$ and $H=G$). This concludes the proof of $(ii)$.
% \end{proof}

% \begin{defi}[\cite{PZ}]
% 	Let $\Gamma$ be a grading on an algebra $A$. We define $\Aut(\Gamma)$ as the group of all self-equivalences of $\Gamma$, i.e., automorphisms of $A$ that
% 	permute the components of $\Gamma$. Let $\operatorname{Stab}(\Gamma)$ be the subgroup of $\Aut(\Gamma)$ consisting of the automorphisms that fix
% 	each component of $\Gamma$. Clearly, $\operatorname{Stab}(\Gamma)$ is a normal subgroup of $\Aut(\Gamma)$, so we can define the \emph{Weil group} of
% 	$\Gamma$ by $\operatorname W (\Gamma) := \frac{\Aut(\Gamma)}{\operatorname{Stab}(\Gamma)}$. The group $\operatorname W (\Gamma)$ can be seen as a subgroup
% 	of the permutation group of the support and also as a subgroup of the automorphism group of the universal group of $\Gamma$.
% \end{defi}



