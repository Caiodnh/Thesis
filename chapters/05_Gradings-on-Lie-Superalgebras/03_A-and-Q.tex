\section{Gradings on the Lie Superalgebras of Type \texorpdfstring{$A$ and $Q$}{A and Q}}

By the discussion \cref{sec:Aut-Lie-chap}, we have a classification of $G$-gradings on the Lie superalgebras of series $A$ and $Q$ in terms of the model $\Skew(R, \vphi)$, where $R = S\times S\sop$ and $\vphi$ is the exchange superinvolution. 
We want to describe these gradings in terms of the original model, $S^{(-)}$, where $S$ is either $M(m,n)$ or $Q(n)$ for some positive integers $m$ and $n$. 
We use the approach similar to \cite[Appendix]{paper-adrian}

We will also express the parameters in terms of the group $G$ rather than $G^\# =G \times \ZZ_2$. 
 
\subsection{Undoubling}\label{ssec:undoubling}
% Reducing from \texorpdfstring{$S\times S\sop$}{SxSsop} to \texorpdfstring{$S$}{S}

Let $L$ be a finite dimensional simple Lie superalgebra in the series $A$ or $Q$, and suppose $L$ is not of type $A(1,1)$. 
By definition, we have that $L = S^{(1)}/Z(S^{(1)})$. 
On the other hand, \cref{cor:transfer-R-vphi-to-L} allows us to classify the gradings on $L$ by considering its isomorphic copy $\tilde L \coloneqq \Skew(R, \vphi)^{(1)}/Z(\Skew(R, \vphi)^{(1)})$, where $R = S\times S\sop$ and $\vphi$ is the exchange superinvolution:
the gradings on $\tilde L$ (and their isomorphism classes) are in bijection with the gradings (and their isomorphism classes) on $(R, \vphi)$. 
The goal of this subsection is to translate the classification of gradings on $L$ from $(R, \vphi)$ to $S$. 
Recall that the isomorphism $\tilde L \to L$ is induced by the projection $R \to S$. 

\begin{defi}\label{defi:types-I-and-II}
    A grading on $L$ is said to be of \emph{Type I} if it is obtained by restriction and reduction modulo the center from a (unique) grading on $S$. 
    Otherwise, the grading on $L$ is said to be of \emph{Type II}. 
\end{defi}

As recalled above, all gradings on $\tilde L \iso L$ come from gradings on $(R, \vphi)$. 
Type I and Type II gradings can be distinguished in this model as follows. 
Recall that gradings on $(R, \vphi)$ were divided in two classes: those that make $R$ graded-simple, classified in \cref{thm:MxM-even,thm:MxM-odd}, and those that do not, classified in \cref{thm:MxM-type-II,thm:QxQ-type-II}. 
We claim that these classes correspond to the Type II and Type I gradings on $\tilde L$, respectively. 
Indeed, consider a Type I grading on $L$ and the corresponding grading on $S$. 
Then $(R, \vphi)$ is naturally graded so that $S\times \{ 0 \}$ and $\{0\} \times S\sop$ are graded superideals, and the projection $R \to S$ is a homomorphism of graded superalgebras. 
This grading on $(R, \vphi)$ induces to a grading on $\tilde L$ and the isomorphism of Lie superalgebras $\tilde L \to L$ preserves degrees. 
This proves that the gradings on $\tilde L$ corresponding to Type I gradings on $L$ (under the isomorphism $\tilde L \to L$) come from the gradings on $R$ that do not make it graded-simple. 
By \cref{prop:only-SxSsop-is-simple}, every grading on $(R, \vphi)$ such that $R$ is not graded-simple is of this form, so the converse follows. 

By definition, Type I gradings are described in terms of $S$, so we will now focus on Type II gradings. 

Suppose $(R, \vphi)$ is endowed with a grading $\Gamma$ making $R$ graded-simple. 
By \cref{thm:iso-(R-vphi)-with-parameters}, there is a triple $(\D, \U, B)$ with parameters $(T, \beta, p, \eta, \kappa, g_0, \delta)$ as in Definition \ref{def:E(D,U,B)} such that $(R, \vphi) \iso E(\D, \U, B)$. 
Recall that, in this case, $\rad \tilde\beta = \langle f \rangle$ for an element $f\in T$ of order $2$ such that $\eta(f) = -1$ (\cref{prop:types-of-D-via-rad-beta}). 

Set $\barr G = G/\langle f \rangle$, let  $\pi\from G\to \barr G$ be the natural homomorphism and write $\bar g \coloneqq \pi(g)$, for all $g\in G$. 
By \cref{prop:lemma-for-undoubling-and-fine-gradings}, ${}^\pi R$ is the sum of two graded-simple superideals, $S\times \{ 0 \}$ and $\{0\} \times S\sop$, so the restriction of ${}^\pi \Gamma$ to $S \iso S\times \{ 0 \}$ induces a $\barr G$-grading on $L$ of Type I. 

To recover $\Gamma$ from its coarsening ${}^\pi \Gamma$, fix a character $\chi \in \widehat G$ such that $\chi (f) = -1$ (which exists since $\FF$ is algebraically closed) and let $\psi\from R \to R$ be the automorphism given by the action of $\chi$, \ie, $\psi(r) \coloneqq \chi(g) r$ for every $r\in R_g$. 
Clearly, we have that,
\[\label{eq:Rg-recovered-with-chi}
    R_g = \{ r \in R_{\bar g} \mid \psi(r) = \chi(g)r \}.
\]
Let $\zeta \coloneqq (1, -1) \in Z(R)\even$. 
We have that $\vphi(\zeta) = -\zeta$ and, by \cref{prop:R-and-D-have-the-same-center-vphi}, $\zeta$ is homogeneous of degree $f$ with respect to $\Gamma$. 
Let $\theta\from R \to R$ be the super-anti-automorphism defined by $\theta \coloneqq \vphi \psi = \psi \vphi$. 
By the definition of $\psi$, $\psi(\zeta) = -\zeta$ and, hence, $\theta(\zeta) = \zeta$. 
It follows that $\theta(1,0) = \theta \left(\frac{1+\zeta}{2}\right) = (1,0)$, so  $\theta(S\times \{0\}) = \theta ((1,0)R) = S\times \{0\}$. 
Hence, $\theta$ restricts to a super-anti-automorphism on $S$. 
Then, it is straightforward to see that
\[\label{eq:psi-in-terms-of-theta}
    \psi(s_1, \overline{s_2}) = (\theta(s_2), \overline{\theta(s_1)}),
\]
for all $s_1, s_2\in S$. 
Since $\Skew(R, \vphi) = \{ (s, -s) \mid s \in S \}$, combining \cref{eq:Rg-recovered-with-chi,eq:psi-in-terms-of-theta}, we get that the Type II grading on $L$ corresponding to $\Gamma$ can be recovered from the Type I grading corresponding to ${}^\pi \Gamma$ as follows:
\[\label{eq:refiniment-on-L}
    L_g = \{ s \in L_{\bar g} \mid \theta(s) = -\chi(g) s\}.
\]
Hence, a Type II grading on $L$ can be described in terms of a Type I grading and a super-anti-automorphism on $S$. 

Our next goal is to describe $(S, \theta)$, where $S$ is endowed with the restriction of ${}^\pi\Gamma$, without reference to $(R,\vphi)$. 
By \cref{prop:lemma-for-undoubling-and-fine-gradings}, $S \iso E(\barr T, \barr \beta, \barr p, \kappa)$, where $\barr T \coloneqq T/\langle f \rangle$, $\barr \beta$ and $\barr p$ are the maps induced by $\beta$ and $p$ on $\barr T$, and $\kappa\from G^\#/T \to \ZZ_{\geq 0}$ is seen as a map $\kappa\from \barr G^\#/ \barr T \to \ZZ_{\geq 0}$ via the canonical isomorphism $G^\#/T \iso \barr G^\#/ \barr T$. 

To describe $\theta\from S \to S$, let us first write $\theta\from R\to R$ in matrix terms. 
Let $\B = \{ u_1, \ldots, u_k \}$ be a $G$-graded basis of $\U$, following \cref{conv:pick-even-basis}, and use it to identify $\End_\D (\U)$ with $M_k(\D)$. 
We will also assume that $B$ is even if $\D$ is odd (\cref{conv:pick-even-form}). 
Set $g_i \coloneqq \deg u_i$. 
By \cref{prop:matrix-vphi}, we have $\vphi(X) = \Phi\inv \vphi_0(X) \Phi$, for all $X \in M_k(\D)$, where $\Phi_{ij} = B(u_i, u_j)$. 
Also, given $i, j \in \{ 1, \ldots, k \}$ and $0 \neq d \in \D_t$, $t\in T$, we have $\psi( E_{ij}(d) ) = \chi(g_i\inv t  g_j) E_{ij}(d) = \chi(g_i)\inv \chi(t) \chi(g_j) E_{ij}(d)$. 
Let $\Lambda \in M_k(\D)$ be the diagonal matrix with $\Lambda_{ii} = \chi(g_i)$ for all $i \in \{ 1, \ldots, k \}$, and let $\psi_0$ denote the action of $\chi$ on $\D$. 
Then $\psi(X) = \Lambda\inv \psi_0(X) \Lambda$, for all $X \in M_k(\D)$. 
Define $\theta_0 \coloneqq \psi_0 \vphi_0 = \vphi_0 \psi_0$. 
Then:
\[\label{eq:theta-with-matrix-1}
    \theta(X) = (\Lambda\Phi)\inv \theta_0(X)\, \Lambda\Phi,
\]
for all $X \in M_k(\D)$. 

Clearly, $\theta_0$ is a super-anti-automorphism on $\D$ associated to the map $\mu \coloneqq \eta \chi$. 
Note that $\mu(f) = 1$ and, hence, we can see $\mu$ as a map $\barr \mu\from \barr T \to \FF^\times$. 
Indeed,  since $f\in \ker \tilde\beta$, we have
\begin{equation}
    \mu(tf) = \tilde\beta (t,f) \mu(t)\mu(f) = \mu(t)\mu(f) = \mu(t),
\end{equation}
for all $t\in T$, so $\mu$ is well-defined for every coset $\bar t \in \barr T$. 
Recall, from \cref{cor:T+-is-elem-2-grp}, that $T^+$ is a elementary $2$-group and for every $t\in T^-$, $t^2 = f$. 
Hence $\chi$ takes values $\pm 1$ on $T^+$ and $\pm \mathbf{i}$ on $T^-$. 
By \cref{lemma:quadratic-form-involutions}, $\mu$ is a quadratic map on $\barr T$. 

Let $\epsilon \coloneqq (1,0) \in Z(R)\even$ be the identity element of $S$, and consider it also as an element of $Z(\D)\even$ using \cref{prop:R-and-D-have-the-same-center-vphi}. 
Following the proof of \cref{prop:lemma-for-undoubling-and-fine-gradings}, with $\D$ playing the role of $\mc E$ and $\U$ playing the role of $\V$, we have that $S \iso \End_{\barr \D}(\bar \U)$, where $\barr \D \coloneqq \D\epsilon$ and $\barr \U \coloneqq \U\epsilon$, and that $\B\epsilon \coloneqq \{ u_1\epsilon, \ldots, u_k\epsilon\}$ is a $\barr G^\#$-graded $\barr \D$-basis of $\barr \U$. 
Using this basis, we can identify $\End_{\barr \D}(\bar \U)$ with $M_k(\barr \D)$, and \cref{eq:theta-with-matrix-1} implies that 
\[\label{eq:theta-with-matrix-2}
    \theta(X) = (\Lambda\barr\Phi)\inv \theta_0(X)\, \Lambda\barr\Phi,
\]
for all $X \in M_k(\barr \D)$, where $\barr \Phi \in M_k(\barr \D)$ is defined by $\barr \Phi_{ij} \coloneqq B(u_i, u_j) \epsilon$. 

We note that the refinement obtained in \cref{eq:refiniment-on-L} is determined by the values of $\chi$ on the subgroup $T \subseteq G^\#$. 
To be more precise, let $\tilde \chi \in \widehat {G^\#}$ be a character such that $\chi\restriction_{T} = \tilde\chi\restriction_{T}$, let $\tilde \psi \from R\to R$ be the action by $\tilde \chi$, and set $\tilde \theta \coloneqq \tilde\psi \vphi = \vphi \tilde\psi$. 
We claim that, for every $s \in S_{\bar g}$, $\tilde\theta(s) = - \tilde\chi(g) s$ \IFF $\theta(s) = - \chi(g) s$. 

Indeed, set $\sigma \coloneqq \tilde\chi\chi\inv$, so $\tilde\chi = \sigma \chi$. 
Then $\sigma$ is a character on $G^\#$ with $\sigma(T) = 1$ and, in particular, can be seen as a character on $G^\#/\langle f \rangle$. 
We identify $S$ with $M_k(\barr \D)$. 
Then every element in $S_{\bar g}$ is a linear combination of elements of the form $E_{ij}(d)$, where $1 \leq i, j \leq k$ and $d\in \barr \D_{\bar t}$, such that $\bar g_i \bar t \bar g_j\inv = \bar g$. 
By \cref{eq:theta-with-matrix-2}, $\tilde\theta(E_{ij}(d)) = \sigma\inv(g_i) \sigma\inv(g_j\inv) \theta (E_{ij}(d)) = \sigma\inv(\bar g_i) \sigma(\bar g_j\inv) \theta (E_{ij}(d)) = \sigma\inv(\bar t\inv \bar g) \theta(E_{ij}(d)) = \sigma(\bar g) \theta(E_{ij}(d))$. 
We conclude that $\tilde\theta(E_{ij}(d)) = \sigma(g)\chi(g) E_{ij}(d)$ \IFF $\theta(E_{ij}(d)) = \chi(g) E_{ij}(d)$, as desired. 
$\sigma \coloneqq \chi \tilde\chi\inv$ and let $\tilde\Lambda \in M_k(\D)$ be the matrix . Then, by \cref{eq:theta-with-matrix-2}, 

% \begin{remark}
%     By \cref{lemma:tildeT-finally}, we could have chosen any quadratic form $\mu$ and there would be a $\chi \in \widehat T$ such that $\mu = \eta\chi$, which we can extend to a character on $G$. 
% \end{remark}

% The condition that $(\eta, \kappa, g_0, \delta) \in \mathbf{I}(T, \beta, p)$ (\cref{defi:X(D)}) can be restated in terms of $(\mu, \kappa, g_0, \delta)$: 

% \begin{defi}\label{def:chi-admissible}
%     Given $\mu\from \barr T \to \FF^\times$,
% 	$\kappa\from \barr G^\#/\barr T \to \ZZ_{\geq 0}$, $g_0 \in G^\#$, and $\delta \in \pmone$, we say that the quadruple $(\mu, \kappa, g_0, \delta)$ is \emph{$\chi$-admissible} if:
% 	\begin{enumerate}[(i)]
% 		\item $\mu$ is a quadratic form with polarization  
% 		$\mathrm{d}\mu (\bar s, \bar t) = \sign{\bar s}{\bar t} \bar\beta (\bar s, \bar t)$; 
% 		%
% 		\item $\kappa$ has finite support; 
% 		%
% 		\item $\kappa(x) = \kappa(\barr{g_0}\inv x\inv)$ for all $x \in \barr G^\#/\barr T$;
% 		%
% 		\item for any $x\in \barr G^\#/\barr T$, if $\barr{g_0} x^2 = \barr T$ and $(-1)^{|\bar g|} \chi\inv (g_0) \chi^{-2} (\bar g) \mu (\barr{g_0} \bar g^2)\delta =-1$, for some (and, hence, any) $\bar g\in x$, then $\kappa (x)$ is even.
% 	\end{enumerate}
% \end{defi}

% But we can choose a basis $\B$ where $B(u_i, u_j)$ depending only on the parameters, removing the reference to $(\U, B)$ from the picture. 

% Clearly, the existence of a good basis for $\mc U$ implies the existence of a good basis for $\bar \U$. 

% \begin{defi}\label{def:type-II-G-sharp}
%     We denote the Type II grading on $L$ corresponding to $R$ by $\Gamma^{\mathrm{(II)}}(\barr T, \bar \beta, \bar p, \mu, \kappa, g_0, \delta)$. 
% \end{defi}

% We conclude with the following theorem:

% \begin{thm}\label{thm:type-II-G-sharp}
%     Let $L$ be as above. 
%     Every Type II grading on $L$ is isomorphic to some $\Gamma^{\mathrm{(II)}}(\barr T, \bar \beta, \bar p, \mu, \kappa, g_0, \delta)$. 
%     Further, $\Gamma^{\mathrm{(II)}}(\barr T, \bar \beta, \bar p, \mu, \kappa, g_0, \delta) \iso \Gamma^{\mathrm{(II)}}(\barr T', \bar \beta', \bar p', \mu', \kappa', g_0', \delta')$ if, and only if, $\barr T = \barr T'$, $\bar \beta = \bar \beta'$, $\bar p = \bar p'$ and orbits, action?
% \end{thm}

% \newpage


% --------------

% We will now describe a formula for the super-anti-automorphism $\theta\from S \to S$. 

% Let $\B = \{ u_1, \ldots, u_k \}$ be a $G$-graded basis of $\U$, following \cref{conv:pick-even-form}, and use it to identify $\End_\D (\U)$ with $M_k(\U)$. 
% Let $g_i \coloneqq \deg u_i$. 

% By xx, we have that $\vphi(X) = \Phi\inv \vphi_0(X) \Phi$, for all $X \in M_k(\D)$, where $\Phi_{ij} = B(u_i, u_j)$. 

% Also, given $i, j \in \{ 1, \ldots, k \}$ and $0 \neq d \in \D_t$, $t\in T$, we have that $\psi( E_{ij}(d) ) = \chi(g_i t \inv g_j) E_{ij}(d) = \chi(g_i) \chi(t \inv) \chi(g_j) E_{ij}(d)$. 
% Let $\Lambda \in M_k(\D)$ be the diagonal matrix with $\Lambda_{ii} = \chi(g_i)$ for all $i \in \{ 1, \ldots, k \}$, and let $\psi_0$ denote the action of $\chi$ on $\D$. 
% Then $\psi(X) = \Lambda\inv \psi_0(X) \Lambda$, for all $X \in M_k(\D)$. 
% Define $\theta_0 \coloneqq \psi_0 \vphi_0 = \vphi_0 \psi_0$. 
% Then:
% \[\label{eq:theta-with-matrix-3}
%     \theta(X) = (\Lambda\Phi)\inv \theta_0(X)\, \Lambda\Phi,
% \]
% for all $X \in M_k(\D)$. 

% We will now translate \cref{eq:theta-with-matrix-1} to $S$.

% Clearly, $\theta_0$ is a super-anti-automorphism on $\D$ associated to the map $\mu \coloneqq \eta \chi$. 
% Note that $\mu(f) = 1$ and, hence, since $f\in \ker \tilde\beta$, we can see $\mu$ as a map $\barr \mu\from \barr T \to \FF^\times$. 
% Indeed,
% \begin{equation}
%     \mu(tf) = \tilde\beta (t,f) \mu(t)\mu(f) = \mu(t)\mu(f) = \mu(t),
% \end{equation}
% for all $t\in T$, so $\mu$ is well-defined for every equivalence class $\bar t \in \barr T$. 



% By the proof of \cref{prop:lemma-for-undoubling-and-fine-gradings},

% It is ea

% % --------------
% Need: the matrix representation of $\theta$. 







To describe it with parameters, let $\barr T$, bla, bla. 
By \cref{prop:lemma-for-undoubling-and-fine-gradings}, the type I grading has parameters ...

Let $\mu \coloneqq \chi \eta$. 
It restricts to $\barr T$ by polarization formula and $\mu(f) = +1$. 
Also, corresponds to a super-anti-auto on $\overline{\D} \coloneqq \epsilon \D$. 
Compare with \cref{prop:lemma-for-undoubling-and-fine-gradings} changing roles of things. 
The parameters of $\Gamma$ are blah, which can be recovered from $(..., \mu, \kappa, g_0, \delta)$. 
Note the condition that $(\eta, \kappa, g_0, \delta) \in \mathbf{I}(T, \beta, p)$ can be restated in terms of $\chi$. 



We will follow the notation of \cref{lemma:undoubling-D,}


In this case, the grading corresponds to a grading making $(R, \vphi) \iso E(\D, \U, B)$, with parameters $(T, \beta, p, \eta, \kappa, g_0, \delta)$ as in Def. ??. 
Also, by ??, $\ker \tilde\beta = \langle f \rangle$ with $\eta(f) = -1$. 
% In this case, $\zeta \coloneqq (1, -1) \in R$ is a homogeneous element with $\deg \zeta = f$. 

We will now focus on the Type II gradings. 
Suppose $(R, \vphi)$ has a grading making it graded-simple. 
As we saw before, in this case the element $\zeta \coloneqq (1, -1) \in R$ is homogeneous and $f \coloneqq \deg \zeta \in G$ has order $2$. 
Fix $\chi \in \widehat G$ such that $\chi (f) = -1$ and let $\psi\from R \to R$ be the automorphism given by the action of $\chi$, \ie, $\psi(r) = \chi(g) r$ for every $r\in R_g$. 
If we consider ${}^\pi R$, where $\pi\from G\to \overline{G} \coloneqq G/\langle f \rangle$ is the quotient map, then $\deg \zeta = \bar e$ and, hence, ${}^\pi R$ is not graded-simple. 
In particular, the corresponding $\overline{G}$-grading on $L$ is a Type I grading. 

We can recover the $G$-grading on $R$ using $\psi$. 
Since $\chi(f) = -1$, we have that $R_g = \{ r \in R_{\bar g} \mid \psi(r) = \chi(g)r \}$. 

Define $\epsilon \coloneqq (1,0) \frac{1+\zeta}{2}$ and $\epsilon' \coloneqq (0, 1) = \frac{1-\zeta}{2}$. 
Clearly, $\epsilon$ and $\epsilon'$ are central idempotents,  $\vphi(\epsilon) = \epsilon'$ and $S\times \{ 0 \} = \epsilon R$ and $\{ 0 \} \times S\sop = \epsilon' R$. 
Since both $\epsilon$ and $\epsilon'$ are $\overline{G}$-homogeneous, the grading ${}^\pi R$ is not graded-simple. 

Also, $\psi(\epsilon) = \epsilon'$, so $\psi(S\times \{ 0 \}) = \{ 0 \} \times S\sop$. 
Note that $\psi \vphi = \vphi \psi$ is a super-anti-automorphism of $R$. 
Then $\theta \coloneqq (\psi \vphi)\restriction_{S\times \{ 0 \}}$ is a super-anti-automorphism of $S = S\times \{0 \}$. 
It is straightforward that $\psi(s_1, \bar s_2) = ( \theta(s_2), \overline{\theta(s_1)})$. 
In particular, $\psi( s, - \bar s) = ( -\theta(s), \overline{\theta(s)})$. 
Hence, eq xx reduces to 
\[
    L_g = \{ s \in L_{\bar g} \mid \theta(s) = - \chi(g) s\}.
\]

In other words, given $f$ and with $\chi$ is fixed, the type II grading on $L$ is described by a type I grading and an super-anti-automorphism $\theta\from S \to S$. 
Conversely, with this data, we can define $\psi$ by the formula above and get a $G$-grading on $(R, \vphi)$. 
It is graded-simple since both $S \times \{ 0 \}$ and $\{ 0 \} \times S\sop$ are not $G$-graded superideals. 
To see that, note that $r $

\begin{lemma}
    Let $f \in G$ be an order $2$ element and let $\chi\in \widehat G$ such that $\chi(f) = -1$. 
    Set $\barr G \coloneqq G/ \langle f \rangle$. 
    Let $(S, \theta)$ be $\barr G$-graded superalgebra with superinvolution such that $S$ is graded-simple and $\theta^2(s) = \chi^2(\bar g) s$ for all $s\in S_{\bar g}$ (note that $\chi^2$ is well-defined in $\barr G$). 
    Then the grading on $R \coloneqq S \times S\sop$ can be refined by to a $G$-grading by setting 
    \[
        R_g = \{ (s, \chi(g) \theta\inv(s)) \mid s \in R_{\bar g}\},
    \]
    which makes $R$ graded-simple.
\end{lemma}

\begin{proof}
    afjks
\end{proof}

\begin{defi}
    Grading $\Gamma(f, S, \theta)$. 
    We fix $\chi$. 
\end{defi}

\begin{thm}
    The isomorphism, which must follows from the other iso thm. 
    But not with so little parameters...
\end{thm}

The isomorphism condition on $(R, \vphi)$ implies that $f = f'$ and $(S, \theta)$ is isomorphic to $(S', \theta')$. 
Conversely, from this we have...

It only remains to see how to describe the $\overline{G}$-grading on $S$ and the $\theta$. 
Let $\Lambda$ be the matrix...

\cref{prop:lemma-for-undoubling-and-fine-gradings}.

% ---------------------------------

\newpage


As we have seen, we can classify the gradings on the Lie superalgebras in the series $A$ and $Q$ by considering an different model than the one in their definitions. 
The goal of this subsection is to describe these gradings in terms of the original model. 
For uniformity, we will use the parameters in terms of the group $G^\#$. 

Let $S$ be a finite dimensional simple associative superalgebra and let $\vphi$ denote the exchange superinvolution on $R \coloneqq S\times S\sop$. 
We will put $\tilde L \coloneqq S^{(1)}/Z(S^{(1)})$ and $L_2 \coloneqq \Skew(R, \vphi)^{(1)}/Z(\Skew(R, \vphi)^{(1)})$. 

\begin{defi}
    
\end{defi}


Let $\Gamma$ be a grading on $L$ and let $\tilde \Gamma$ be the corresponding grading on $(R, \vphi)$. 
    If $\Gamma$ corresponds to a grading on $(R, \vphi)$ that does not make it graded simple, then
    We say that $\Gamma$ is of \emph{Type I} if $\Gamma$ correspond to a grading on 
    A grading on $L_2$ is said to be of \emph{Type I} if the

Let $\zeta = (1, -1) \in R = S\times S\sop$. 
For type I gradings, $\zeta$ is homogeneous of degree $e$. 
By identification, it is clear that $\zeta$ is homogeneous and $\deg \zeta = f$. 

A Type I grading on $L_2$ can be easily seen as a grading on $\tilde L$: the superideal $S\times \{ 0 \} \subseteq S\times S\sop$ is graded, and the grading on $\tilde L$ is given by the restriction and reduction modulo the center of the grading on $S$. 

\begin{defi}
    Parameters of a type I grading. 
\end{defi}

Paragraph describing the grading in terms of $\End_\D (\U)$. 
Remark that the isomorphism condition (see future thm) still comes from the second model.

We will now consider the type II gradings. 
Those are of the form $E(\D, \U, B)$. 
Let $T$, $\beta$, $f$. 
Fix $\chi\in \widehat G$ such that $\chi(f) = -1$. 
% Extend $\chi$ to $G^\#$ by $\chi (g, p) = \chi(g)$. 




\subsection{Gradings on \texorpdfstring{$A(m,n)$}{A(m,n)} for \texorpdfstring{$m \neq n$}{m different than n}}



We define Type I and Type II gradings. 

For each $(T, \beta)$ with $T =T^+$ and $\rad \beta \langle f \rangle$, as in \cref{def:model-grd-MxM-odd-or-QxQ}(a), fix $\chi\from G \to \FF^\times$ with $\chi(f) = -1$. 
Note that this exists since $f\in G$ has order $2$. 

For each graded superalgebra $Q^{\mathrm{ex}}(T, \beta, \kappa, g_0)$, the action of $\chi$ determines an automorphism $\psi$, \ie, $\psi(r_g) \coloneqq \chi(g) r_g$. 
The composition $\tilde\vphi \coloneqq \vphi\psi$ is a super-anti-automorphism, and $\tilde\vphi (\zeta) = \zeta$. 
It follows that


\begin{thm}
    \boxed{\mathrm{Type \,\,I}}
    
    \boxed{\mathrm{Type \,\,II}}
\end{thm}

\subsection{Gradings on \texorpdfstring{$A(n,n)$}{A(n,n)}}

For odd Type I, we have descriptions only in terms of $G$ using bla. The classification becomes:

\begin{thm}
    \boxed{\mathrm{Type \,\,I}_M}
    
    \boxed{\mathrm{Type \,\,I}_Q}
    
    \boxed{\mathrm{Type \,\,II}_{osp}}
    
    \boxed{\mathrm{Type \,\,II}_P}
    
    \boxed{\mathrm{Type \,\,II}_Q}
\end{thm}