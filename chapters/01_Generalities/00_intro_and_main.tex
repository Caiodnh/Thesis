\chapter{Generalities on gradings}\label{sec:generalities}

% ** Intro to Chapter



The purpose of this chapter is to introduce the basic notions and constructions involving gradings and also to fix notation and terminology.  
% We will consider $G$ to be fixed group. 

Let $G$ be a group. 
In \cref{sec:graded-bimodules}, we introduce the concepts of homogeneous linear maps, elementary $G$-gradings on matrix (super)algebras, $G$-graded modules, tensor product of $G$-graded vector spaces and (super)algebras, and supercenter of an associative superalgebra.
In \cref{sec:Omega-algebras}, we define universal algebra, so we can handle gradings on different structures (such as algebras, superalgebras and superalgebras with superinvolution) in a uniform manner. 
In \cref{sec:g-hat-action}, under the assumptions that $G$ is abelian, $\FF$ is algebraically closed and $\Char \FF = 0$, we present the duality between $G$-gradings and $\widehat G$-actions for universal algebras. 
This allows us to ``transfer'' $G$-gradings between universal algebras with different signatures (see  \cref{def:universal-algebra,thm:transfer-of-gradings}); we will use this in \cref{chap:Lie} to get a classification of gradings on classical Lie superalgebras from a classification of gradings on associative superinvolution-simple superalgebras. 
% In that section, we will assume that $G$ is abelian, $\FF$ is algebraically closed and $\Char \FF = 0$.
% In \cref{sec:g-hat-action}, we present the duality between $G$-gradings on universal algebras and $\widehat G$-actions by automorphisms (in the case $\FF$ is algebraically closed and $\Char \FF = 0$); the main result (\cref{thm:transfer-of-gradings}) will be used in \cref{chap:Lie} to transfer gradings between universal algebras with different signatures (see \cref{def:universal-algebra}): associative superalgebras with superinvolution and Lie superalgebras. 
Finally, \cref{ssec:universal_group} is devoted to concepts related to refinement, coarsening and fine gradings, where we cannot keep the grading group fixed. 
We introduce set gradings on universal algebras and define equivalence of gradings and universal grading group in this context. 
We warn the reader that some terms appearing in \cref{ssec:universal_group} are not used consistently in the literature (see discussion in \cite[Section 2.7]{GS}); here we follow \cite{livromicha}. 

% The notion of grading is introduced in a very general way, not only for (super)algebras, but also for vector (super)spaces, (super)modules (\cref{sec:graded-bimodules}) and universal algebras (\cref{sec:Omega-algebras}). 
% In \cref{sec:g-hat-action} we discuss the correspondence between gradings and actions, which gives us a tool to transfer gradings between different (universal) algebras. 
% Finally, in \cref{ssec:universal_group}, we discuss the different ways to classify gradings (up to isomorphism or up to equivalence), as well as the concepts of coarsening and refinement of a grading.  


% --------------------






% ** Sections: 

\section{Basic concepts}\label{sec:graded-bimodules}
% ------------------

In this section, we will expand on concepts discussed in the \hyperref[chap:intro]{Introduction} and present some basic constructions. 

\phantomsection\label{support}
Let $V$ be a vector space. 
The \emph{support of a $G$-grading} $\Gamma: V = \bigoplus_{g\in G} V_g$ is the subset of $G$ given by
\[
    \supp \Gamma \coloneqq \{ g \in G \mid V_g \neq 0 \}.
\]
If $\Gamma$ is fixed, then we may refer to $\supp \Gamma$ as the \emph{support of $V$} and denote it by $\supp V$. 

\begin{defi}\label{defi:elementary-grd-abstract}
    Let $V = \bigoplus_{g\in G} V_g$ and $W = \bigoplus_{g\in G} W_g$ be $G$-graded vector spaces. 
    A linear map $T \from V \to W$ is said to be \emph{homogeneous of degree $g$} if 
    \[
    \forall h \in G, \quad T(V_h) \subseteq W_{gh} .
    \]
    The subspace of $\Hom(V,W)$ consisting of all linear maps of degree $g$ is denoted by $\Hom (V, W)_g$, and we define the graded vector space $\Hom^{\textrm{gr}} (V, W)$ by 
    \[
        \Hom^{\textrm{gr}} (V, W) \coloneqq \bigoplus_{g \in G} \Hom (V, W)_g. 
    \]
    % The grading on $\Hom^{\textrm{gr}} (V, W)$ is called the \emph{elementary grading induced by $V$ and $W$}. 
    In the case $V=W$, we denote $\Hom (V, W)_g$ by $\End (V)_g$ and $\Hom^{\textrm{gr}} (V, W)$ by $\End^{\textrm{gr}} (V)$. 
\end{defi}

If $V$ is finite dimensional, it is easy to see that $\Hom^{\textrm{gr}} (V, W) = \Hom (V, W)$. 
Also, if $U$ is another graded vector space and $S\from U \to V$ and $T\from V \to W$ are homogeneous linear maps of degrees $h$ and $g$, respectively, then $T \circ S$ is a homogeneous map of degree $gh$. 
In particular, $\End^{\textrm{gr}} (V)$ is a $G$-graded algebra (compare with \cref{eq:elementary-1st}). 

We emphasize that by a homomorphism of graded vector spaces we mean a degree preserving linear map, \ie, an element of $Hom (V, W)_e$. 

The following is an easy result that will be used in \cref{chap:grds-sinv-simple}:

\begin{lemma}\label{lemma:eigenvector-homogeneous}
	Let $V$ be a $G$-graded vector space. 
	If $T \from V \to V$ is a degree preserving map, then its eigenspaces are graded subspaces of $V$.
\end{lemma}

\begin{proof}
	Let $v\in V$ be an eigenvector with eigenvalue $\lambda \in \FF$ and write $v = \sum_{g\in G} v_g$, where $v_g \in V_g$. 
	On the one hand, $T(v) = \lambda v = \sum_{g\in G} \lambda v$. 
	On the other hand, $T(v) = \sum_{g\in G} T(v_g)$. 
	Since the sum of $V_g$, $g \in G$, is direct, we must have that $T(v_g) = \lambda v_g$ for all $g\in G$, and the result follows. 
\end{proof}

As we saw above, if $V$ is a finite dimensional vector space, then any $G$-grading on $V$ gives rise to a grading on the associative algebra $\End(V)$; gradings of this form are called \emph{elementary}.
An elementary grading can be described in matrix terms. 
Given a $n$-tuple $\gamma = (g_1, \ldots, g_n)$ of elements in $G$, we can define a $G$-grading on $\FF^n$ by setting $\deg e_i = g_i$, where $\{ e_1, \ldots, e_n\}$ is the canonical basis of $\FF^n$. 
Clearly, any finite dimensional $G$-graded vector space is isomorphic to $\FF^n$ endowed with such a grading. 
The grading on $M_n (\FF)$ induced from $\FF^n$ is the following: 

\begin{defi}\label{defi:elementary-grd-matrix}
    The \emph{elementary grading defined by an $n$-tuple $\gamma = (g_1, \ldots, g_n) \in G^n$} on $M_{n}(\FF)$ is the $G$-grading determined by 
    \[
        E_{ij} = g_i g_j\inv, \quad 1 \leq i, j \leq n,
    \]
    where $E_{ij}$ denotes the matrix with $1$ in the $(i,j)$-entry and $0$ in every other entry. 
\end{defi}

Recall that the superspace $V = \FF^{m|n}$ is defined by setting $V\even = \FF^m$ and $V\odd = \FF^n$ (see \cref{def:grd-superspace-canonical}). 
Given an $m$-tuple $\gamma_\bz$ and an $n$-tuple $\gamma_\bo$ of elements in $G$, we can consider the gradings defined as above on $\FF^m$ and $\FF^n$, and, hence, a grading on $\FF^{m|n}$. 
The corresponding grading on $M(m,n)$ is the following:

\begin{defi}\label{defi:elementary-grd-super}
    Let $\gamma_\bz$ be an $m$-tuple and $\gamma_\bo$ be an $n$-tuple of elements of $G$. 
    The \emph{elementary grading defined by $\gamma_\bz$ and $\gamma_\bo$} on the superalgebra $M(m,n)$ is the elementary grading on its underlying algebra $M_{m+n}(\FF)$ defined by the concatenation of $\gamma_\bz$ and $\gamma_\bo$.  
\end{defi}

It is useful to consider gradings not only on vector (super)spaces and (super)algebras, but also on modules:

\begin{defi}\label{defi:grdModule}
    Let $R = \bigoplus_{g\in G} R_g$ be a graded associative algebra and let $V = \bigoplus_{g\in G} V_g$ be a graded vector space. 
    If $V$ has a structure of left $R$-module, we say that $V$ is a \emph{graded left module} if 
    \[
        \forall g,h \in G, \quad R_g \cdot V_h \subseteq V_{gh},
    \]
    \ie, if the image of the representation $\rho\from R \to \End(V)$ is in $\End^{\textrm{gr}}(V)$ and $\rho$ is degree-preserving. 
    One can define graded right modules and graded bimodules analogously. 
\end{defi}

In the case $R$ is a superalgebra and $V$ is a superspace with their canonical $\ZZ_2$-gradings, we say that $V$ is a \emph{left $R$-supermodule}. 
A \emph{$G$-graded left $R$-supermodule} is a $G^\#$-graded $R$-module (recall the definition of $G^\#$ in \cref{defi:G-sharp}).

\begin{ex}
    Any graded vector space $V$ is a graded left $\End^{\textrm{gr}}(V)$-module, with the natural action. 
    Also $\Hom^{\textrm{gr}}(V, W)$ is a graded $(\End^{\textrm{gr}}(W),\End^{\textrm{gr}}(V))$-bimodule, with the action given by map composition. 
\end{ex}

% \begin{ex}
%     Given a graded Lie superalgebra $L = L\even \oplus L\odd$, we can see $L\odd$ as a module over the graded Lie algebra $L\even$. 
% \end{ex}

%  --- shift


\begin{defi}
    Let $\Gamma: V = \bigoplus_{h\in G} V_h$ be a grading on a vector space $V$. 
    For an element $g\in G$, the \emph{right shift of $\Gamma$ by $g$}, denoted by $\Gamma^{[g]}$, is the grading obtained by replacing every index $h \in G$ by $hg$, \ie, $\Gamma^{[g]} : V = \bigoplus_{h\in G} V_h'$ where $V_{h}' \coloneqq V_{h g\inv}$, for all $h\in G$. 
    Analogously, the \emph{left shift of $\Gamma$ by $g$} is defined to be ${}^{[g]}\Gamma : V = \bigoplus_{h\in G} V_h''$ where $V_{h}'' \coloneqq V_{g\inv h}$, for all $h\in G$. 
    If $\Gamma$ is fixed, we define the \emph{right (respectively, left) shift of $V$} to be $V$ endowed with $\Gamma^{[g]}$ (respectively, ${}^{[g]}\Gamma$) and denote it by $V^{[g]}$ (respectively, ${}^{[g]}V$).
\end{defi}

If $V$ is a graded left module over a graded algebra $R$, then $V^{[g]}$ is a graded left $R$-module. 
Furthermore, ${}^{[g]} R ^{[g\inv]}$ is a graded algebra and ${}^{[g]}V$ is a graded left module over ${}^{[g]} R ^{[g\inv]}$. 
Of course, similar statements hold for graded right modules. 

% THE FOLLOWING IS NOT USED, REMOVE?

% We will make use of the following partial converse (see e.g. \cite[Proposition 3.5]{paper-Qn}):

% \begin{lemma}\label{lemma:simplebimodule}
% 	Let $R$ be a $G$-graded algebra and let $V$ be a finite-dimensional (ungraded) simple $R$-module. 
% 	If $\Gamma$ and $\Gamma'$ are  two $G$-gradings that make $V$ a graded module, then there is $g\in G$ such that $\Gamma' = \Gamma^{[g]}$. \qed
% \end{lemma}

% ---- Tensor

We are now going to define tensor product of graded spaces:

% Def: Grading on tensor product
\begin{defi}\label{defi:tensorProduct}
    Let $V=\bigoplus_{g\in G} V_g$ and $W=\bigoplus_{g\in G} W_g$ be $G$-graded vector spaces. 
    The \emph{tensor product} of $V$ and $W$ is the vector space $V \tensor W$ endowed with the $G$-grading $\Gamma : V \tensor W = \bigoplus_{g\in G} (V \tensor W)_g$ where
    \[
        \forall g\in G, \quad (V \otimes W)_g = \bigoplus_{h\in G} V_{h} \otimes W_{h\inv g}.
    \]
\end{defi}

\phantomsection\label{defi:tensor-algebras}

If $G$ is an abelian group and $R$ and $S$ are $G$-graded algebras, then it follows that $R \tensor S$ is a graded algebra with the usual multiplication: $(r \tensor s) (r' \tensor s') = rr' \tensor ss'$. 
Note that, in this case, the notion of graded bimodule can be reduced to the notion of left module, as in the ungraded case. 

% The external tensor product of is always a graded algebra. 

For superalgebras, though, following the rule of signs mentioned in \cref{sec:Grassmann}, one often considers a different multiplication on the tensor product:

\begin{defi}\label{defi:tensorSuperalgebras}
    Let $R = R\even \oplus R\odd$ and $S = S\even \oplus S\odd$ be superalgebras. 
    We define the superalgebra $R\, \underline\tensor \, S$ to be the $\ZZ_2$-graded tensor product $R \tensor S$ endowed with the multiplication determined by
    \[
        \forall r, r' \in R\even \cup R\odd, \, s,s' \in S\even \cup S\odd, \quad 
        (r \tensor s) (r' \tensor s')
        = \sign{s}{r'} \, rr' \tensor ss'.
    \]
\end{defi}

If one of the superalgebras has trivial canonical $\ZZ_2$-grading, \ie, if $R = R\even$ or $S = S\even$, then the tensor product of superalgebras coincides with the tensor product of algebras. 
We will encounter this situation in \cref{chap:grd-simple-assc} (see \cref{rmk:M(D)=M(FF)-tensor-D}). 

% -- Center:

% We finish this section with an important concept for the associative algebra:

\begin{defi}\label{defi:center}
	Let $R$ be an associative superalgebra.
	The \emph{center} of $R$ is the set
	\[
		Z(R) = \{c\in R \mid cr = rc \text{ for all } r\in R \},
	\]
	\ie, the center of $R$ seen as an algebra, and the \emph{supercenter} of $R$ is the set $sZ(R) \coloneqq sZ(R)\even \oplus sZ(R)\odd$, where
	\begin{align*}
		sZ(R)^i = \{c\in R^i \mid cr = (-1)^{i |r|} rc \text{ for all } r\in R\even\cup R\odd \},\quad i \in \ZZ_2.
	\end{align*}
\end{defi}

\begin{ex}
    Let $V$ be a vector space and consider the Grassmann superalgebra $\mc G (V)$ (\cref{def:Grassmann-algebra}). 
    It is easy to see that $Z(G (V)) = G (V)\even$ while $sZ(G (V)) = G (V)$. 
\end{ex}

\begin{ex}
    Let $n > 0$ and consider the associative superalgebra $Q(n)$. 
    One can see that $Z(G (V))$ is the subspace spanned by $I$ and $u$, where
    \[
        u \coloneqq
        \left(\begin{array}{c|c}
            0 & I\\
            \hline
            I & 0
        \end{array}\right),
    \]
    so $Z(G (V)) \iso Q(1)$. 
    On the other hand, $sZ(Q(n))$ is the subspace generated by $I$, \ie, $sZ(Q(n)) = Z(Q(n)) \cap Q(n)\even \iso \FF$.
\end{ex}

\begin{lemma}\label{lemma:center-is-graded}
	Let $G$ be an abelian group and let $R$ be an associative $G$-graded superalgebra.
	Then the center $Z(R)$ and the supercenter $sZ(R)$ are $G$-graded subsuperalgebras of $R$.
\end{lemma}

\begin{proof}
	We consider $R$ as a $G^\#$-graded algebra.
	Let $c \in Z(R)$ and write $c = \sum_{g \in G^\#} c_g$, where $c_g \in R_g$ for all $g \in G^\#$.
	For every homogeneous $r \in R$, we have
	%
	\begin{align*}
		\big(\sum_{g\in G^\#} c_g\big)r = r \big(\sum_{g\in G^\#} c_g\big).
	\end{align*}
	%
	Comparing the components of degree $gh = hg$, where $h = \deg r$, we conclude that $rc_g = c_g r$ for all $g \in G^\#$.
	By linearity, $r c_g = c_g r$ for all $r\in R$, hence $c_g \in Z(R)$.

	The same argument works to show that $sZ(R)\even$ is graded and, with straightforward modifications, to show that $sZ(R)\odd$ is graded.
\end{proof}

Note that, in particular (for trivial $G$), the center of a superalgebra is a subsuperalgebra. 

% -----


\section{\texorpdfstring{$\Omega$}{Omega}-algebras}\label{sec:Omega-algebras}

A \emph{general} or \emph{universal algebra} is a set with an arbitrary collection (possibly empty) of operations, which may have different ``arities'' (see Definition \ref{def:universal-algebra} below).
This is a very general concept (see, \eg, ?? and ??), which includes all classical objects in algebra (groups, rings, etc.), but here we will be interested in the linear case: the objects will be vector spaces over a field $\FF$ and the operations will be assumed multilinear (following ??, book by Razmylov).
%Our approach to universal algebras differs from the one in most books on the subject (see, \eg, ?? and ??) in the sense we will be working on the (monoidal) category of vector spaces instead of the (monoidal) category of sets. 
% This is the approach on ?? (Razmylov: book used by Felipe) and it has recently been applied to gradings (and graded identities) on ??.
This will give us a framework to deal with algebras, superalgebras, superalgebras with superinvolution, etc., in a uniform manner.

\begin{notation}
	For a vector space $A$ and a nonnegative integer $n$, we will denote the $n^{th}$-tensor power of $A$ by $A^{\tensor n}$, \ie,
	$A^{\tensor n} \coloneqq \underbrace{A\otimes\cdots\otimes A}_{n \text{ times}}$.
	In the case $n = 0$, we will follow the convention that $A^{\tensor 0} \coloneqq \FF$.
\end{notation}

\begin{defi}\label{def:universal-algebra}
	A \emph{signature} $\Omega$ is a set together with a family $\{ \Omega_n \}_{n \geq 0}$ of pairwise disjoint subsets such that $\Omega = \bigcup_{n \geq 0} \Omega_n$.
	An \emph{$n$-ary operation} on a vector space $A$ is a multilinear map $A^n \to A$ or, equivalently, a linear map $A^{\tensor n} \to A$.
	An \emph{$\Omega$-algebra} or a \emph{(general) algebra with signature $\Omega$} is a vector space $A$ together with operations $\omega^A$, one for each $\omega \in \Omega$, such that $\omega^A$ is $n$-ary if $\omega \in \Omega_n$.
\end{defi}

We note that $0$-ary operations can be interpreted as constants in the $\Omega$-algebra $A$, since a linear map $\omega^A\from \FF \to A$ is determined by $\omega^A(1)$.

\begin{defi}
	Let $A$ and $B$ be $\Omega$-algebras.
	A \emph{homomorphism} $\psi\from A \to B$ is a linear map such that for every $\omega \in \Omega_n$ we have
	\[
		\psi( \omega^A (a_1 \tensor \cdots \tensor a_n) ) = \omega^B ( \psi(a_1) \tensor \cdots \tensor \psi(a_n) ),
	\]
	for all $a_1, \ldots, a_n \in A$.
	As usual, an \emph{automorphism} of $A$ is a bijective homomorphism from $A$ to itself, and the group of automorphisms is denoted by $\Aut(A)$.
\end{defi}

When dealing with a fixed $\Omega$-algebra $A$, we will usually identify the signature $\Omega$ with the corresponding set of operations on $A$.

\begin{ex}\label{ex:omega-vec-space}
	A vector space is an $\Omega$-algebra with $\Omega = \emptyset$.
\end{ex}

\begin{ex}\label{ex:omega-algebra}
	An algebra in the usual sense, with product $*$, is an $\Omega$-algebra with $\Omega = \Omega_2 = \{ * \}$.
\end{ex}

\begin{ex}
	An algebra $A$ with unity $\mathds{1} \in A$ is an  $\Omega$-algebra with $\Omega = \Omega_0 \cup \Omega_2$ where $\Omega_2 = \{ * \}$, $\Omega_0 = \{ \omega_0 \}$, and $\omega_0\from \FF \to A$ is defined by $\omega_0 (1) = \mathds 1$.
\end{ex}

\begin{ex}\label{ex:omega-alg-SA}
	A superalgebra $A = A\even \oplus A\odd$ is an $\Omega$-algebra with $\Omega = \Omega_1 \cup \Omega_2$, where $\Omega_2 = \{ * \}$, $\Omega_1 = \{ \pi_{\bar 0}, \pi_{\bar 1} \}$, and $\pi_{\bar 0}, \pi_{\bar 1}\from A \to A$ are the projection on the $A\even$ and $A\odd$, respectively.
	More precisely, an algebra $A$ with such signature is a superalgebra if, and only if,
	\begin{enumerate}[(i)]
		\item $\pi_{\bar 0}^{} + \pi_{\bar 1}^{} = \id$; \label{item:sum-projections}
		\item $\pi_{\bar 0}^{}\pi_{\bar 1}^{} = \pi_{\bar 1}^{}\pi_{\bar 0}^{} = 0$
		      %\item $\pi_{\bar 0}^2 = \pi_{\bar 0}^{}$ and $\pi_{\bar 1}^2 = \pi_{\bar 1}^{}$;
		\item For every $x,y \in A$ and $i, j\in \ZZ_2$, we have that $\pi_{i+j}^{}( \pi_i^{} (x)*\pi_j^{} (y) ) = \pi_i^{} (x)*\pi_j^{} (y)$.
	\end{enumerate}
\end{ex}

% \begin{ex}\label{ex:omega-graded-algebra}
%     A $G$-graded algebra $A = \bigoplus_{g\in G}$, with operation $*$, is a $\Omega$-algebra with $\Omega = \Omega_1 \cup \Omega_2$, with $\Omega_2 = \{ * \}$ and $\Omega_1 = \{ \pi_g \mid g\in G\}$, where $\pi_g\from A \to A$ is the projection in the component $A_g$. 
%     As a particular case, superalgebras can be viewed as $\Omega$-algebras.
% \end{ex}

\begin{ex}
	A superalgebra with super-anti-automorphism $(A, \vphi)$ an $\Omega$-algebra with $\Omega = \Omega_1 \cup \Omega_2$ where $\Omega_2 = \{ * \}$ and $\Omega_1 = \{ \vphi, \pi_{\bar 0}, \pi_{\bar 1} \}$.
	In a similar fashion, algebras with anti-automorphism can be viewed as $\Omega$-algebras.
\end{ex}

It is straightforward to check that, in each of the examples above, the usual notion of homomorphism coincides with the notion of homomorphism as $\Omega$-algebras.

\begin{remark}
	Example \ref{ex:omega-alg-SA} could be generalized to encompass all $G$-graded algebras for a fixed group $G$.
	Such generalization has been used in the study of graded identities (see ??), but we will not follow this approach.
\end{remark}

Finally, we can define gradings on $\Omega$-algebras:

\begin{defi}\label{def:grds-on-Omega-algebras}
	A \emph{$G$-grading} on an $\Omega$-algebra $A$ is a $G$-grading on its underlying vector space such that, if we consider the usual grading on the tensor powers $A^{\tensor n}$, all the operations $\omega^A$ for $\omega \in \Omega$ are degree preserving.
\end{defi}

It is easy to verify that this notion of grading coincides with to the usual notion for algebras, superalgebras, superalgebras with superinvolution, etc.

% It should be noted that among these examples, only in Examples \ref{ex:omega-vec-space} and \ref{ex:omega-algebra} the correspondence is bijective. 
% In the other cases, the original structures correspond to proper subclasses of the $\Omega$-algebras described, which can be axiomitized. 
% We will present axioms for Example \ref{ex:omega-graded-algebra}, since gradings are not usually describe in terms of the projection maps, we will

% \begin{itemize}
%     \item[\done] paragraph-remark stating only exs 1 and 2 are ``precisely'', the other are proper inclusions.
%     \item discuss, in a remark, the axioms of graded algebras, noting that one of them is not first-order if the group is not finite.
% \end{itemize}

\begin{thm}\label{thm:transfer-of-gradings}
    transfer
\end{thm}


\section{\texorpdfstring{$G$}{G}-gradings on and \texorpdfstring{$\widehat{G}$}{G-hat}-actions}\label{sec:g-hat-action}

In this section we will review the correspondence between $G$-gradings and $\widehat G$-actions, which has been used extensively in the study of gradings.
It can be found in many places in the literature (see, \eg, ??, ?? and ??), but our goal in this section is to set it up in the context of general algebras.
% The main purpose of this approach is to transfer gradings between general algebras with different signatures (Theorem ??).% This will allow us to handle algebras and superalgebras, with or without (super)involutions, in a uniform manner. 

The main advantage of this approach is to have a formal result comparing gradings on general algebras with different signature (Theorem ??).
This will be used in Chapter ?? to transfer gradings between Lie superalgebras and associative superalgebras with superinvolutions.
The same sort of transfer has been used in other works (see \cite{livromicha} and Paper with Adrian ??), but without having the result stated formally.

% If $\FF$ is algebraically closed of characteristic $0$ and $G$ is finitely generated abelian group, there is a well known correspondence between $G$-gradings and algebraic (rational) $\widehat G$-actions on vector spaces. 

The following is well known:

\begin{thm}\label{thm:g-hat-correspondence}
	Suppose $\FF$ is algebraically closed of characteristic $0$ and let $G$ be a finitely generated abelian group.
	Then there is a bijective correspondence between $G$-gradings and algebraic $\widehat G$-actions on a vector space $V$, definied as follows:
	\begin{enumerate}[(i)]
		\item Given a grading $\Gamma: V = \bigoplus_{g\in G} V_g$, the corresponding action is defined by $\chi\cdot v = \chi(g) v$ for all $v\in V_g$ and all $g\in G$ and extended to all $V$ by linearity;
		\item Given an algebraic action of $\widehat G$ on $V$, the corresponding grading is defined by declaring $v\in V$ homogeneous of degree $g\in G$ if, and only if, $\chi\cdot v = \chi(g) v$ for all $\chi \in \widehat G$. \label{item:action-to-grading} \qed
	\end{enumerate}
\end{thm}

\begin{notation}
	Given a grading $\Gamma$ on $V$, we will denote the corresponding representation of the algebraic group $\widehat G$ on $V$ by $\eta_\Gamma\from \widehat G \to \GL(V)$.
\end{notation}

\begin{remark}\label{rmk:G-hat-preserves-degree}
    It must be noted that the maps $\eta_\Gamma(\chi)\from V \to V$ are degree preserving with respect to $\Gamma$ or any refinement of $\Gamma$, since every nonzero homogeneous element of $V$ is an eigenvector. 
\end{remark}

% \begin{remark}
	The correspondence of Theorem \ref{thm:g-hat-correspondence} can be generalized to arbitrary fields and arbitrary abelian groups with the use of group schemes instead of algebraic groups, but this generality is not needed for the current work.
% \end{remark}

We will now see how this correspondence applies to $\Omega$-algebras:

% We will now consider gradings on universal algebras with arbitrary signature (which should not be confused with Example \ref{ex:omega-graded-algebra}, which realizes graded algebras as a specific type of universal algebra).

% \begin{defi}
%     A $G$-grading on a $\Omega$-algebra $A$ is a $G$-grading on its vector space underlying such that, if we consider the usual grading on the tensor powers $A^{\tensor n}$, all the operations $\omega^A$ for $\omega \in \Omega$ are degree preserving.
% \end{defi}

% It is straight forward to verify that this notion of grading corresponds to the usual notion of gradings on algebras, superalgebras, algebras with antiautomorphisms and superalgebras with super-anti-automorphisms.

% We will now proceed to generalize the correspondence between $G$-gradings and $\widehat G$-actions (see ?? and ??) to $\Omega$-algebras. 
% We start focusing on graded vector spaces.

% \begin{defi}
%     Let $V$ be a vector space. 
%     Given a $G$-grading $\Gamma\from V= \bigoplus_{g\in G} V_g$ on $V$, we define a $\widehat G$-action in the following way: for $\chi \in \widehat G$ and $a_g \in V_g$, $\chi \cdot v_g = \chi(g)v_g$, and we extend the it by linearity. 
%     We will denote the corresponding representation by $\eta_\Gamma\from \widehat G \to \GL(V)$.
% \end{defi}

% It is easy to see that this really defines an action. 

% For this action to capture more information about the grading, we need some assumptions on the field $\FF$. 

% \begin{lemma}
%     Let $V$ be a $G$-graded vector space. 
%     A element $v\in V$ is homogeneous of degree $g\in G$ if, and only if, $\chi\cdot v = \chi(g)v$ for all $\chi \in \widehat G$.
% \end{lemma}

% \begin{proof}
%     The ``only if'' direction is the definition of the action. 
%     For the other direction, let $v\in V$ and write $v = \sum_{g\in G} v_g$, where $v_g \in V_g$.
% \end{proof}

\begin{prop}\label{prop:g-hat-Aut-A}
	Assume $\FF$ is algebraically closed and $\Char \FF = 0$.
	Let $A$ be a $\Omega$-algebra and $\Gamma$ be a $G$-grading on its underlying vector space.
	Then $\Gamma$ is a $G$-grading on $A$ if, and only if, $\eta_\Gamma(\widehat G) \subseteq \Aut(A)$.
\end{prop}

\begin{proof}
	First, assume $\Gamma$ is a grading on the $\Omega$-algebra $A$.
	Let $\chi \in \widehat G$ and let $\psi \coloneqq \eta_\Gamma(\chi)$.
	We already know that $\psi$ is bijective, it only remains to prove it is a homomorphism.
	Let $\omega \in \Omega_n$ and let $a_1, \ldots, a_n \in A$ be homogeneous elements of degrees $g_1, \ldots, g_n \in G$, respectively.

	Then $a_1\tensor \cdots \tensor a_n \in A^{\tensor n}$ has degree $g_1 \cdots g_n$. Hence
	\begin{align*}
		\psi(\omega^A(a_1\tensor \cdots \tensor a_n)) & = \chi(g_1 \cdots g_n) \omega^A(a_1\tensor \cdots \tensor a_n)      \\
		                                              & =\chi(g_1) \cdots \chi(g_n) \omega^A(a_1\tensor \cdots \tensor a_n) \\
		                                              & = \omega^A(\chi(g_1)a_1\tensor \cdots \tensor \chi(g_n)a_n)         \\
		                                              & = \omega^A(\psi(a_1)\tensor \cdots \tensor \psi(a_n)),
	\end{align*}
	so $\psi$ is a homomorphism.

	Conversely, let, again, $\omega \in \Omega_n$ and $a_1, \ldots, a_n \in A$ be homogeneous elements of degrees $g_1, \ldots, g_n \in G$.
	Since $\psi$ is an automorphism, we have:
	\begin{align*}
		\psi(\omega^A(a_1\tensor \cdots \tensor a_n)) & = \omega^A(\psi(a_1)\tensor \cdots \tensor \psi(a_n))               \\
		                                              & = \omega^A(\chi(g_1)a_1\tensor \cdots \tensor \chi(g_n)a_n)         \\
		                                              & =\chi(g_1) \cdots \chi(g_n) \omega^A(a_1\tensor \cdots \tensor a_n) \\
		                                              & =\chi(g_1 \cdots g_n) \omega^A(a_1\tensor \cdots \tensor a_n),
	\end{align*}
	hence, by item \eqref{item:action-to-grading} in Theorem \ref{thm:g-hat-correspondence}, $\omega^A(a_1\tensor \cdots \tensor a_n)$ is homogeneous of degree $g_1 \cdots g_n$, \ie, $\omega^A$ preserves degrees.
\end{proof}

\begin{remark}
    Note that, under the conditions of \cref{prop:g-hat-Aut-A}, all elements of $\eta_\Gamma(\widehat G) \subseteq \Aut(A)$ are automorphisms of 
\end{remark}

Now let $A$ be an $\Omega$-algebra and $B$ be an $\Omega'$-algebra.
If there is a group homomorphism $\theta\from \Aut(A) \to \Aut(B)$, then, as consequence of Proposition \ref{prop:g-hat-Aut-A}, we can use it to transfer $G$-gradings on $A$ to $G$-gradings on $B$, even if $\Omega \neq \Omega'$.
Explicitly, for a $G$-grading $\Gamma$ on $A$, we consider the group homomorphism $\eta_\Gamma\from \widehat G \to \Aut (A)$ and then take the composition $\theta \circ \eta_\Gamma\from \widehat G \to \Aut(B)$, which in turn corresponds to a $G$-grading on $B$.
We will denote the grading on $B$ by $\theta(\Gamma)$.

% --------------------------------------------------




\section{Refinement, coarsening and equivalence}\label{ssec:universal_group}
% ---

In this section, we will introduce some concepts that do not involve a fixed grading group. 
For these, it is useful to have a ``group free'' notion of grading:

\begin{defi}\label{defi:set-grading}
    A \emph{set grading $\Gamma$ on a vector space $V$} is a vector space decomposition indexed by elements of a set $S$, \ie, 
    $\Gamma : V = \bigoplus_{s\in S} V_s$. 
    % \[ 
    %     \Gamma : V = \bigoplus_{s\in S} V_s.
    % \]
    If $V$ is a superspace, we further impose that each component $V_s$ is a subsuperspace. 
    When endowed with a fixed set grading $\Gamma$, we say that $V$ is a \emph{set graded vector space}. 
\end{defi}

\begin{defi}\label{defi:ref-coars}
    Let $\Gamma : V = \bigoplus_{s\in S} V_s$ and $\Delta : V = \bigoplus_{t \in T} V_{t}$ be set gradings on a vector space $V$. 
    We say that $\Gamma$ is a \emph{refinement} of $\Delta$, or that $\Delta$ is a \emph{coarsening} of $\Gamma$, if for every $s \in S$ there is $t \in T$ such that $V_s \subseteq V_t$. 
    If, for some $s \in S$, this inclusion is strict, we say that the refinement/coarsening is \emph{proper}. 
\end{defi}


As in \cref{support}, we define the \emph{support} of a set grading $\Gamma : V = \bigoplus_{s\in S} V_s$ to be the set $\supp \Gamma \coloneqq \{ s \in S \mid V_s \neq 0 \}$. 
Note that we can always replace $S$ by $\supp \Gamma$. 
If $\Delta$ is a coarsening of $\Gamma$ as above and $s\in \supp \Gamma$, then there is a unique element $t \in T$ such that $V_s \subseteq V_t$. 
This motivates the following:

\begin{defi}\label{coars-induced}
    Let $V$ be a vector space and let $\Gamma : V = \bigoplus_{s\in S} V_s$ be set grading. 
    Given a set $T$ and a map $\alpha\from S \to T$, the \emph{coarsening of $\Gamma$ induced by $\alpha$} is the set grading 
    \[
        {}^{\alpha}\Gamma : V = \bigoplus_{t\in T} V_t,
    \]
    where 
    \[
        V_t \coloneqq \bigoplus_{s \in \alpha\inv (t)} V_s.
    \]
\end{defi}

Before defining set gradings on $\Omega$-algebras, we need the following:

\begin{defi}\label{defi:graded-map}
    Let $V = \bigoplus_{s\in S} V_s$ and $W = \bigoplus_{t\in T} W_t$ be set graded vector spaces. 
    %
    A linear map $f\from V \to W$ is said to be  \emph{graded} if for any $s \in S$, there is $t \in T$ such that $f(V_s) \subseteq W_t$. 
\end{defi}

Note that, by definition, a grading $\Gamma$ is a refinement of a grading $\Delta$ on a vector space $V$ \IFF the identity map seen as $(V, \Gamma) \to (V, \Delta)$ is a graded map. 

\begin{defi}
    Let $V = \bigoplus_{s\in S} V_s$ and $W = \bigoplus_{t\in T} W_t$ be set graded vector spaces. 
    The \emph{tensor product} of $V$ and $W$ is the the vector space $V \tensor W$ endowed with the grading
    \[
        \Gamma : V \tensor W = \bigoplus_{(s,t) \in S \times T} V_s \tensor W_t. 
    \]
\end{defi}

% \begin{remark}
    We note that the $G$-grading on the tensor product of two  vector spaces (\cref{defi:tensorProduct}) is the coarsening of the set grading in \cref{defi:graded-map} induced by the map $\alpha\from G\times G \to G$ given by $\alpha (g,h) \coloneqq gh$, for all $g,h \in G$. 
% \end{remark}

\begin{defi}
    A \emph{set grading on an $\Omega$-algebra $A$} is a set grading $\Gamma : A = \bigoplus_{s\in S} A_s$ on the underlying vector space of $A$ such that $\omega^A\from A^{\tensor n} \to A$ is a graded linear map for all $\omega \in \Omega$. 
\end{defi}

In particular, if $A$ is an algebra in the usual sense, a set grading on $A$ is a vector space decomposition $\Gamma : A = \bigoplus s_{s\in S} A_s$ such that, for any $s_1,s_2\in S$ there exists $s_3\in S$ such that $A_{s_1} A_{s_2} \subseteq A_{s_3}$.

\begin{defi}\label{defi:equivalence}
    Let $A$ and $B$ be $\Omega$-algebras endowed, respectively, with set gradings ${\Gamma : A = \bigoplus_{s \in S} A_{s}}$ and ${\Delta = \bigoplus_{t \in T} A_{t}}$. 
    An \emph{equivalence} $\psi\from A \to B$ is an isomorphism of $\Omega$-algebras such that both $\psi$ and $\psi\inv$ are graded maps. 
    If $A = B$ and there is an equivalence $\psi\from (A, \Gamma) \to (A, \Delta)$, we say that $\Gamma$ and $\Delta$ are \emph{equivalent gradings}. 
\end{defi}

Note that an equivalence $\psi\from A \to B$ determines a bijection $\alpha \from \supp \Gamma \to \supp \Delta$ by $\vphi(A_s) = B_{\alpha(s)}$. 

We will now bring groups and group gradings back to the picture:

\begin{defi}
    We say that a set grading $\Gamma$ on an $\Omega$-algebra $A$ can be \emph{realized as an (abelian) group grading} if there is an (abelian) group $G$ and a injective map $\alpha\from \supp \Gamma \to G$ such that ${}^{\alpha}\Gamma$ is a $G$-grading on $A$. 
\end{defi}

\begin{defi}\label{defi:fine-grading}
    Let $A$ be an $\Omega$-algebra and let $\Gamma$ be a set grading on $A$. 
    We say that $\Gamma$ is a \emph{fine (abelian) group grading} if it can be realized as an (abelian) group grading but no proper refinement of $\Gamma$ can. 
\end{defi}

When a grading can be realized as an (abelian) group grading, it can be done in many different ways. 
But there is a special realization that has a universal property:

\begin{defi}\label{defi:universal-group}
    Let $A$ be an $\Omega$-algebra and let $\Gamma$ be a set grading on $A$. 
    A \emph{universal (abelian) group} of $\Gamma$ is a group $G$ together with a map $\iota\from \supp\Gamma \to G$ such that ${}^{\iota}\Gamma$ is a $G$-grading and, for every (abelian) group $G'$ and map $\iota'\from \supp \Gamma \to G'$ such that ${}^{\iota'}\Gamma$ is a $G'$-grading, there is a unique group homomorphism $\alpha\from G \to G'$ such that $\iota' = \alpha \circ \iota$, \ie, the following diagram commutes:
    %
	\begin{center}
		\begin{tikzcd}
            S \arrow[to = G, "\iota"] \arrow[to = H, "\iota'"]
            && |[alias = G]| G \arrow[to = H, dashed, "\alpha"]\\
            &&\\
            && |[alias = H]| G'
        \end{tikzcd}
	\end{center}
\end{defi}

Clearly, $\Gamma$ can be realized as an (abelian) group grading \IFF the map $\iota$ above is injective. 
Also, we can construct a universal (abelian) group using generators and relations: we take $\supp \Gamma$ as the set of generators and, for each $n\geq 0$ and $\omega \in \Omega_n$, we consider relations $s_1 \cdots s_n = s_{n+1}$ for all $s_1, \ldots, s_n, s_{n+1} \in \supp \Gamma$ such that $0 \neq \omega^A (A_{s_1} \tensor \cdots \tensor A_{s_n}) \subseteq A_{s_{n+1}}$. 

\begin{remark}\label{rmk:coars-grp-induced}
    Let $A$ be an $\Omega$-algebra, $G$ and $G'$ be groups and $\alpha\from G \to G'$ be a group homomorphism. 
    If $\Gamma : A = \bigoplus_{g\in G} A_g$ is a $G$-grading, then it is easy to see that ${}^{\alpha} \Gamma$ is a $G'$-grading. 
    We note that, by \cref{defi:universal-group}, if $G$ is the universal group of $\Gamma$, then every $G'$-grading that is a coarsening of $\Gamma$ is obtained this way.  
\end{remark}

By means of the duality between gradings and actions outlined in \cref{sec:g-hat-action}, the fine abelian group gradings on a finite dimensional algebra $A$ over an algebraically closed field of characteristic $0$ correspond to maximal quasitori in the algebraic group $\Aut(A)$. 
Moreover, the group of algebraic characters of a maximal quasitorus is the universal abelian group of the corresponding grading.  

We conclude this chapter with some comments about the two types of classification mentioned in the \hyperref[intro-equiv]{Introduction}: fine gradings up to equivalence and $G$-gradings up to isomorphism. 
%
Any group grading on a finite dimensional algebra $A$ is a coarsening of a fine group grading. 
So, if we have a classification of fine group gradings on $A$ up to equivalence and know their universal groups, we can obtain any $G$-grading as $^{\alpha}\Gamma$ for some fine grading $\Gamma$ and homomorphism $\alpha$ from the universal group of $\Gamma$ to $G$ (see \cref{defi:universal-group}). 
However, $\Gamma$ and $\alpha$ are not unique and in practice it is difficult to determine when two such induced $G$-gradings $^{\alpha}\Gamma$ and $^{\beta}\Delta$ are isomorphic. 

On the other hand, if we know all $G$-gradings on $A$, for any $G$, we can try to determine which of them are fine and compute their universal groups. 
This was done for simple Lie superalgebras of series $Q$, $P$ and $B$ in \cite{paper-Qn,paper-MAP,Helens_thesis}. 

% ---

% The following result appears to be ``folklore''. We include a proof for completeness.


% \begin{lemma}\label{lemma:universal-grp}
% 	Let $\mathcal{F}=\{\Gamma_i\}_{i\in I}$, be a family of pairwise nonequivalent fine (abelian) group gradings on a $\Omega$-algebra $A$, where $\Gamma_i$ is a $G_i$-grading and $G_i$ is generated by $\supp \Gamma_i$. 
% 	Suppose that $\mathcal{F}$ has the following property:
% 	for any grading $\Gamma$ on $A$ by an (abelian) group $H$, there exists $i\in I$ and a homomorphism $\alpha:G_i\to H$ such that $\Gamma$
% 	is isomorphic to ${}^\alpha\Gamma_i$. Then
% 	%
% 	\begin{enumerate}[(i)]
% 		\item every fine (abelian) group grading on $A$ is equivalent to a unique $\Gamma_i$; \label{item:all-fine}
% 		\item for all $i\in I$, $G_i$ is the universal (abelian) group of $\Gamma_i$. \label{item:Gi-is-univeral}
% 	\end{enumerate}
% \end{lemma}

% \begin{proof}
% 	Let $\Gamma$ be a fine grading on $A$, realized over its universal group $H$. 
% 	Then there is $i\in I$ and $\alpha: G_i \to H$ such that ${}^\alpha \Gamma_i \iso \Gamma$. 
% 	Writing $\Gamma_i: A = \bigoplus_{g\in G_i} A_g$ and $\Gamma: A = \bigoplus_{h\in H} B_h$, we then have $\vphi \in \Aut(A)$ such that
% 	\[
% 		\vphi\,\big( \bigoplus_{g\in \alpha\inv (h)} A_g \big) = B_h
% 	\]
% 	for all $h\in H$. 
% 	Since $\Gamma$ is fine, we must have $B_h \neq 0$ if, and only if, there is a unique $g\in G_i$ such that $\alpha(g) = h$, $A_g\neq 0$ and $\vphi(A_g) = B_h$. 
% 	Equivalently, $\alpha$ restricts to a bijection $\supp(\Gamma_i) \to \supp(\Gamma)$ and $\vphi(A_g) = B_{\alpha(g)}$ for all $g \in S_i:= \supp (\Gamma_i)$. This proves assertion $(i)$.

% 	Let $G$ be the universal group of $\Gamma_i$. 
% 	It follows that, for all $s_1, s_2, s_3 \in S_i$,
% 	%
% 	\begin{equation*} \label{eq:relations-unvrsl-grp}
% 		\begin{split}
% 			& s_1s_2 = s_3 \text{ is a defining relation of } G \\
% 			\iff & 0 \neq A_{s_1} A_{s_2} \subseteq A_{s_3}\\
% 			\iff & 0 \neq B_{\alpha(s_1)} B_{\alpha(s_2)} \subseteq B_{\alpha (s_3)}\\
% 			\iff & \alpha(s_1)\alpha(s_2) = \alpha(s_3) \text{ is a defining relation of } H.
% 		\end{split}
% 	\end{equation*}
% 	%
% 	Therefore, the bijection $\alpha\restriction_{S_i}$ extends uniquely to an isomorphism $\widetilde{\alpha}: G\rightarrow H$.

% 	By the universal property of $G$, there is a unique homomorphism $\sigma: G\to G_i$ that restricts to the identity on $S_i$. Hence, the following diagram commutes:
% 	%
% 	\begin{center}
% 		\begin{tikzcd}
% 			G \arrow[to=Gi, "\sigma"] \arrow[to = H, "\widetilde{\alpha}"]&&\\
% 			&& |[alias=H]|H\\
% 			|[alias=Gi]|G_i \arrow[to=H, "\alpha"]&&
% 		\end{tikzcd}
% 	\end{center}
% 	%

% 	Since $\widetilde{\alpha}$ is an isomorphism, $\sigma$ must be injective. But $\sigma$ is also surjective since $S_i$ generates $G_i$. 
% 	Hence $G_i$ is isomorphic to $G$. Since $\Gamma$ was an arbitrary fine grading, for each given $j\in I$, we can take $\Gamma = \Gamma_j$ (hence, $i=j$ and $H=G$). This concludes the proof of $(ii)$.
% \end{proof}

% \begin{defi}[\cite{PZ}]
% 	Let $\Gamma$ be a grading on an algebra $A$. We define $\Aut(\Gamma)$ as the group of all self-equivalences of $\Gamma$, i.e., automorphisms of $A$ that
% 	permute the components of $\Gamma$. Let $\operatorname{Stab}(\Gamma)$ be the subgroup of $\Aut(\Gamma)$ consisting of the automorphisms that fix
% 	each component of $\Gamma$. Clearly, $\operatorname{Stab}(\Gamma)$ is a normal subgroup of $\Aut(\Gamma)$, so we can define the \emph{Weil group} of
% 	$\Gamma$ by $\operatorname W (\Gamma) := \frac{\Aut(\Gamma)}{\operatorname{Stab}(\Gamma)}$. The group $\operatorname W (\Gamma)$ can be seen as a subgroup
% 	of the permutation group of the support and also as a subgroup of the automorphism group of the universal group of $\Gamma$.
% \end{defi}



