\section{Application: Superinvolutions on Simple Superalgebras}

\begin{itemize}
    \item We will classify first the graded-simple superalgebras with superinvolution (case $G$ trivial), and then the gradings on them.
    \item Note that, from lemma bla, for general $G$, we have that simple as superalgebra implies type $M$ and even gradings.
    \item In particular, for trivial $G$, the superalgebra must be of the form $\End_\FF (U)$ for a superspace $U$.
    \item Lemm
    \item Prop: G trivial, $G^\# = \ZZ_2$.
    \item even grading, $T = e$ so only one $\eta$.
    \item The action is only a $G^\# = \ZZ_2$ action
    \item fix $\delta = 1$, as above, and then we have only a $\mc G = e$ action.
    \item Good, no orbits, only points
    \item Conditions of admissibility:
    \item i and ii are trivial now
    \item define $k(0) = m$ and $k(1) = n$
    \item iii becomes k(i) = k(g0i)
    \item in the case $g_0 = \bar 1$, so $m = n$. In the other case, trivial
    \item condition (iv): for any $i \in \ZZ_2$, if $g_0 + 2i = \bar 0$, \ie, $g_0 = 0$ and $\mu_i = (-1)^{i} = -1$, \ie, $i = \bar 1$, then $k(i) = m$ is even.
    \item the construction we have there implies we have $\End_\FF (\FF^m \oplus \FF^n)$ with the matrices of $\vphi$ given by what they have to be.
    \item this is what we have in the introduction.
\end{itemize}

Nice, let us repeat the same proof for general $G$ to get what we have to get

\begin{itemize}
    \item still, $T$ is even.
    \item also, elementary $2$-group
    \item What are the possible $\eta$? given one $\vphi_0$, any other is composition with a automorphism
    \item hence, by lemma, we get $\chi \eta$, since each automorphism correspond to $\chi$
    \item By nondegeneracy, one class at most!
    \item but is there one?
    \item yes, we can refer to paper and standard realization, and get the transpose
    \item in other words, $\eta(ab) = \beta(a,b)$
\end{itemize}

Fine, so only one class of $\eta$, fix $\eta$ as transpose and let's go. 
Then make $\delta = 1$. 

What is the $I^+$? 

\begin{itemize}
    \item Once we fixed $\delta$, we cannot act by $g$ odd... $\mc G$ is precisely $G$.
    \item again, conditions of admissibility...
    \item Ugly, isn't it?
    \item 
\end{itemize}

Calm down. 
From $\End_\D(\U)$ and $B$, how we find the underlying superalgebra?

\begin{itemize}
    \item Take $\D$ standard realization with transposition
    \item From $\U$ to $U$... Take $U = \U \tensor \FF B$?
    \item Nope
    \item Take the basis with the matrices as in propositions 4.40 and 3.42
    \item Take $B_x$ as defined there.
    \item from this many $\FF$-forms, we get a big $\FF$-form
    \item 
\end{itemize}

Other approach:

\begin{itemize}
    \item Ok, from kappas we get $m$ and $n$, as in corollary in chap 2.
    \item We need only to see what is the parity of $g_0$. 
    It must be the same! 
    But how to argue that?
    \item Well, we need a standard realization any way...
    \item The problem here is that we can't work only on the superalgebra level as before, we need to go to the superspace level.
    \item ok, back to define $U_x = V_x \tensor \FF B$
\end{itemize}