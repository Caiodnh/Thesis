\chapter{Scrap Version --- Gradings on Superinvolution-simple Associative Superalgebras}

In this chapter we first see what are the graded-superinvolution-simple superalgebras and then apply it to classify the superinvolution-simple superalgebra and the gradings on them. 

Let $G$ be an abelian group. 

\section{Graded-superinvolution-simple superalgebras}

In \cite{racine}, finite dimensional superinvolution-simple superalgebras are classified over any field $\FF$ with $\Char \FF \neq 2$. 
In this section we adapt some of the results there to classify the finite dimensional graded-superinvolution-simple superalgebras in the case $\FF$ is algebraically closed. 
Nevertheless, $\FF$ con be any field for the results for some of the results below. 

% The next definition and lemma give us a class of examples of graded-superinvolution-simple superalgebras that are not simple as graded superalgebras.
% In fact, Proposition \ref{prop:only-SxSsop-is-simple} will tell us that these are the only examples. 

\begin{defi}\label{def:SxSsop}
	Let $S$ be a $G$-graded superalgebra and consider the $G$-graded superalgebra $S \times S\sop$ with the homogeneous component of degree $g\in G$ being $S_g \times S\sop_g$. 
	We define the \emph{exchange superinvolution} on $S \times S\sop$ to be the map $\vphi\from S \times S\sop \to S \times S\sop$ given by $\varphi (s_1, \bar s_2) = (s_2, \bar s_1)$ (recall from \cref{ssec:superdual} that $\bar s$ denotes the element $s \in S$ seen as an element of $S\sop$). 
\end{defi}

We will now give two examples where $G$ is trivial, but that will play a major role in \cref{sec:the-nonsimple-case}.

\begin{ex}\label{ex:FxF-iso-FZ2}
	The simplest possible example is to take $S = \FF$, with superalgebra structure. 
	If $\Char \FF \neq 2$, then $S\times S\sop = \FF [\zeta]$ where $\zeta = (1, -1)$ and the exchange superinvolution is given by $\vphi(1) = 1$ and $\vphi(\zeta) = -\zeta$.
	Note that $\FF [\zeta] \iso \FF\ZZ_2$ with the trivial superalgebra structure. 
\end{ex}

\begin{ex}\label{ex:FZ2xFZ2sop-iso-FZ4}
	Consider $S = Q(1)$, so $S\even = \FF 1$ and $S\odd = \FF u$ where $u^2 =1$.
	Note that $S$ is isomorphic to $\FF\ZZ_2$, but this time with the superalgebra structure given by its natural $\ZZ_2$-grading. 
	% we are the group algebra $S = \FF \langle u \rangle$, where $u$ has order $2$, as a superalgebra by declaring  (in other words, $S \iso Q(1)$).
	% We claim that $S\times S\sop$ is isomorphic to $\FF\langle \omega \rangle$, where $\omega$ has order $4$.
	If $\Char \FF \neq 2$, we claim that $R \coloneqq S\times S\sop$ is isomorphic to $F\ZZ_4$.
	Indeed, the element $\omega \coloneqq (u, \bar u) \in S\times S\sop$ has order $4$ and generates $S\times S\sop$: $\omega^2 = (1, - \bar 1)$, $\omega^3 = (u, - \bar u)$ and $\omega^4 = (1, 1)$.
	Hence $R\even = \FF1 \oplus \FF \omega^2$ and $R\odd = \FF \omega \oplus \FF \omega^3$.
	Also, the exchange superinvolution on $R$ is given by $\vphi(1) = 1$, $\vphi(\omega) = \omega$, $\vphi(\omega^2) = -\omega^2$ and $\vphi(\omega^3) = -\omega^3$.
\end{ex}

When those graded superalgebras with superinvolution are graded-superinvolution-simple and, in this case, when they are isomorphic.

\begin{lemma}\label{lemma:SxSsop-simple}
	Let $S$ be a graded superalgebra and consider the exchange superinvolution on $S \times S\sop$. 
	Then $S \times S\sop$ is graded-superinvolution-simple if, and only if, $S$ is graded-simple. 
\end{lemma}

\begin{proof}
    Suppose $S \times S\sop$ is graded-superinvolution-simple and let $I \subseteq S$ be a graded ideal. 
    We have that $I \times I\sop$ is a superinvolution-invariant graded superideal in $S \times S\sop$, and hence either $I \times I\sop = 0$ or $I \times I\sop = S \times S\sop$. 
    In the first case $I = 0$ and in the second $I = S$, hence $S$ is graded-simple. 
    
    Conversely, suppose $S$ is graded-simple. 
    It is clear that $S\sop$ is also graded-simple and, hence, by a standard argument, the graded superideals of $S \times S\sop$ are $0$, $\{ 0 \} \times S\sop$, $S \times \{ 0 \}$ and $S\times S\sop$. 
    Among those, only $0$ and $S\times S\sop$ are superinvolution-invariant, concluding the proof. 
\end{proof}

% \begin{lemma}\label{lemma:ideals-in-SxSsop}
% 	Let $S$ be a graded unital superalgebra and let $\vphi$ be the exchange superinvolution on $R \coloneqq S \times S\sop$. 
% 	The $\vphi$-invariant graded superideals of $(R, \vphi)$ are precisely the subsets of the form $I \times I\sop$ where $I$ is a superideal of $S$. 
% \end{lemma}

% \begin{proof}
% 	Let $I$ be a graded superideal of $S$ and consider $J \coloneqq I \times I\sop$.
% 	Clearly, $J$ is a $\vphi$-invariant graded subsuperspace of $S \times S\sop$.
% 	Suppose $x, y \in I\even \cup I\odd$ and let $r = (s_1, \bar s_2) \in R\even \cup R\odd$.
% 	Then $r\, (x, \bar y) = (s_1 x, \sign{s_2}{y}\, \overline{y s_2}) \in J$ and $(x, \bar y)\, r = (x s_1, \sign{s_2}{y}\, \overline{s_2 y}) \in J$, so $J$ is, indeed, a superideal.

% 	Now let $J$ be any graded $\vphi$-invariant superideal of $S\times S\sop$ and let $I \coloneqq (1,0)\, J$, which we can regard as a subspace $S \iso S \times \{ 0 \}$.
% 	First, note that $I$ is a graded superideal of $S$. 
% 	Since $J$ is a $\vphi$-invariant graded superideal and $I \subseteq J$, we have that $I + \vphi(I) = I \times I\sop \subseteq J$.
% 	If $(x, \bar y) \in J$, then, on the one hand, $x \in I$, and, on the other hand, $\vphi(x, \bar y) = (y, \bar x) \in J$, so $y \in I$.
% 	Therefore $(x, \bar y)\in I \times I\sop$, concluding the proof.
% \end{proof}

% \begin{cor}\label{cor:SxSsop-simple-iff-S-simple}
%     Under the conditions of \cref{lemma:ideals-in-SxSsop}, $S\times S\sop$ is graded-superinvolution-simple if, and only if, $S$ is graded-simple. \qed
% \end{cor}

\begin{lemma}\label{cor:iso-SxSsop}
    Let $S_1$ and $S_2$ be graded-simple superalgebras. 
    Then $S_1\times S_1\sop \iso S_2\times S_2\sop$ as graded superalgebras with superinvolution if, and only if, $S_1 \iso S_2$ or $S_1 \iso S_2\sop$ as graded superalgebras.
\end{lemma}

\begin{proof}
    Let $\psi\from S_1\times S_1\sop \to S_2\times S_2\sop$ be an isomorphism of graded superalgebras with superinvolution. 
    Since, the only nonzero proper grade superideals of $S_i \times S_i\sop$ are $\{ 0 \} \times S_i\sop$ and $S_i\times S_i\sop$, $i = 1,2$, we have that $\psi(S_1\times \{ 0 \}) = S_1\times \{ 0 \}$ or $\psi(S_2\times \{ 0 \}) = \{ 0 \} \times S_2\sop$.  
    
    For the converse, we can suppose $S_1 \iso S_2$ by relabeling $S_2$ with $S_2\sop$ if necessary.
    If $\zeta \from S_1 \to S_2$ is an isomorphism of graded superalgebras, it is clear that $\psi\from S_1\times S_1\sop \to S_2\times S_2\sop$ given by $\psi (s, \overline{s'}) \coloneqq (\zeta(s), \overline{\zeta(s')})$, for all $s,s' \in S_1$, is an isomorphism of graded superalgebras with superinvolution.
\end{proof}

\begin{prop}\label{prop:only-SxSsop-is-simple}
	Let $(R, \vphi)$ be a 
	graded superalgebra with superinvolution. 
	Then $(R, \vphi)$ is 
	graded-superinvolution-simple if, and only if, either $R$ is a graded-simple or $(R, \vphi)$ is isomorphic to $S\times S\sop$ with the exchange superinvolution, for some graded-simple superalgebra $S$.
\end{prop}

\begin{proof}
	Suppose $(R, \vphi)$ is 
	graded-superinvolution-simple but $R$ is not graded-simple. 
	Let $0 \neq I \subsetneq R$ be a graded superideal.
	Note that $\vphi(I)$ is also a graded superideal, hence $I \cap \vphi(I)$ and $I + \vphi (I)$ are $\vphi$-invariant graded superideals. 
	Since $I \cap \vphi(I) \subseteq I \neq R$, we have $I \cap \vphi(I) = 0$, so we can write $I + \vphi (I) = I \oplus \vphi (I)$. 
	Since $0 \neq I \subseteq I \oplus \vphi (I)$, we conclude that $R = I \oplus \vphi (I)$.
	Clearly, this implies that $(R, \vphi)$ is isomorphic to $I \times I\sop$ with exchange superinvolution. 
	By \cref{lemma:SxSsop-simple}, $I$ must be simple as a graded superalgebra. 
	
	The converse is obvious if $R$ is graded-simple, and follows from \cref{lemma:SxSsop-simple} in the other case.
\end{proof}
 
Sometimes we have a more convenient model for $S\sop$ then $S\sop$ instead of $S\sop$ itself. 
More precisely, suppose $\psi\from S \to S'$ is super-anti-isomorphism of graded superalgebras. 
In this case, it is clear that $S\times S\sop$ with the exchange superinvolution is isomorphic to $S\times S'$ endowed with the superinvolution given by $(s_1, s_2) \mapsto (\psi\inv (s_2), \psi (s_1))$. 

\begin{defi}\label{defi:superdual-exchange}
    Let $\D$ be a graded-division superalgebra, let $\U$ be a graded right $\D$-supermodule of finite rank and set $S \coloneqq \End_\D (\U)$. 
    Recall that $\U\Star \coloneqq \Hom_\D(\U, \D)$ (\cref{def:superdual-supermodule}) is a graded right $\D\sop$-module, and that the map $\psi\from \End_\D (\U) \to \End_{\D\sop} (\U\Star)$ given by $\psi(L) = L\Star$ (\cref{defi:superdual-map}) is a super-anti-isomorphism (\cref{prop:dual-super-anti-iso}). 
    Hence $S\times S\sop$ with exchange superinvolution is isomorphic to $\End_\D (\U) \times \End_{\D\sop} (\U\Star)$ with superivolution $\vphi$ given by $\vphi(L_1, L_2) = (\psi\inv(L_2), \psi(L_1))$. 
\end{defi}


\begin{cor}\label{cor:SxSsop-with-dcc}
    Let $(R, \vphi)$ be a graded-superalgebra with superinvolution and suppose $R$ satisfies the \dcc{} on graded left superideals. 
    Then $(R, \vphi)$ is graded-superinvolution-simple if, and only if, $R$ is graded-simple or $(R, \vphi)$ is isomorphic to a graded-superalgebra with superinvolution as in \cref{ex:superdual-exchange}. 
\end{cor}

\begin{proof}
    If $R$ is graded-simple, there is nothing to prove. 
    Suppose $R$ is not graded-simple. 
    If $(R, \vphi)$ is graded-superinvolution-simple, then $R \iso S\times S\sop$ for some graded-simple superalgebra $S$. 
    It is clear that $S$ also satisfies the \dcc{} on graded left superideals, so by \cref{thm:End-over-D}, $S \iso \End_\D (\U)$, and hence, $(R, \vphi)$
    is isomorphic to a graded-superalgebra with superinvolution as in \cref{ex:superdual-exchange}. 
    The converse follows from \cref{lemma:ideals-in-SxSsop} and the fact $\End_\D (\U)$ is unital. 
\end{proof}

Let us now assume $\FF$ is algebraically closed and $\Char \FF \neq 2$. 
We are in a position to classify up to isomorphism the finite dimensional graded-superinvolution-simple  superalgebras over $\FF$. 
We have two disjoint classes of those: the ones that are graded-simple and the ones as in \cref{ex:superdual-exchange}. 
The former ones were classified in Theorem \ref{thm:iso-(R,vphi)-with-parameters}. 

\begin{defi}
    Let $\D$ be a finite dimensional graded-division superalgebra over an algebraically closed field $\FF$, $\Char \FF \neq 2$, and let $\U\neq 0$ be a graded right $\D$-module of finite rank. 
	We define $E\times E\sop(\D, \U)$ to be the (finite dimensional) graded superalgebra $\End_\D(\U) \times \End_{\D\sop}(\U\Star)$ endowed with the superinvolution given by $(L_1, )$ where $\psi(L) = L\Star$.
	
	with superinvolution defined in \cref{ex:superdual-exchange}.
	If $\D$ is associated to $(T, \beta, p)$ (see \cref{ssec:grd-div-alg}) and $\U$ is associated to $\kappa\from $ $(\eta, \kappa, g_0, \delta) \in \mathbf{I}(T, \beta, p)$ (see Definitions \ref{def:parameter-of-(U,B)} and \ref{defi:X(D)}), then we say that $(T, \beta, p, \eta, \kappa, g_0, \delta)$ are the parameters of the triple $(\D, \U, B)$.
\end{defi}

\begin{cor}
    Let $\D$
\end{cor}

\begin{defi}
    Under the conditions of \cref{ex:superdual-exchange}, suppose $\D$ is finite dimensional. 
    Let $(T, \beta, \kappa)$ be the parameters of the pair $(\D, \U)$ seen as $G^\#$-graded (Definition \ref{def:E(D,U)}) and let $p\from T \to \ZZ_2$ be the parity map, as usual. 
    We will say that $(T, \beta, p, \kappa)$ are t
\end{defi}

The classification of the later is an easy consequence of 
\cref{cor:iso-SxSsop} and the description of parameters of $\D\sop$ and $\U\Star$ in 


\begin{cor}\label{cor:iso-graded-simple-double}
    Let $(R, \vphi)$ be a finite dimensional graded-superinvolution-simple superalgebra. 
    Then $R$ is not graded-simple if, and only if, there exists a finite dimensional graded-division superalgebra 
    
\end{cor}



    Let $\D$ be a finite dimensional graded-division algebra over an algebraically closed field $\FF$ and let $\U$ be a graded right $\D$-module of finite rank. 
	If $\D$ is associated to $(T, \beta)$ and $\U$ is associated to $\kappa\from G/T \to \ZZ_{\geq 0}$ (see Subsection \ref{ssec:D-modules}), we say that $(T, \beta, \kappa)$ are the \emph{parameters} of the pair $(\D, \U)$.

\begin{prop}
    Let $(\D, \U)$ be as in Definition \ref{def:E(D,U)}, 
\end{prop}


Let $S$ be a finite dimensional graded-simple superalgebra. 
By \cref{thm:End-over-D}, we have $S \iso \End_\D (\U)$ as $G^\#$-graded algebras, \ie, as $G$-graded superalgebras. 





It should be noted that, if $S$ admits a super-anti-isomorphism $\psi\from S \to S$, then $S\times S\sop$ with the exchange superinvolution is isomorphic to $S\times S$ with the superinvolution given by $(s_1, s_2) \mapsto (\psi\inv (s_2), \psi (s_1))$.

\begin{defi}\label{def:E(D,U)-exchange}
     
	If $\D$ is associated to $(T, \beta)$ and $\U$ is associated to $\kappa\from G/T \to \ZZ_{\geq 0}$ (see Subsection \ref{ssec:D-modules}), we say that $(T, \beta, \kappa)$ are the \emph{parameters} of the pair $(\D, \U)$.
\end{defi}

Let $S$ be a graded-simple superalgebra.  
If $R$ is graded-simple, the job was done in \cref{sec:classification-grd-simple-with-sinv}.



If $R \iso S\times S\sop$, it is actually easier. 
We parametrize by $S$, which was done in \cref{ssec:classification-assc-super}. 
But the iso thm is different from there: follows from \cref{cor:iso-SxSsop},  \cref{ssec:superdual} and unlabeled remark above. 

As a corollary, we classify the finite dimensional superinvolution simple-superalgebras. 

\begin{lemma}
    If $R$ simple as superalgebra, admits superinvolution and $\Char \FF\neq 2$, then it is of type $M$.
\end{lemma}

\begin{proof}
    Follows from \cref{prop:R-simple-iff-D-simple,cor:sinv-implies-type-M}, in the case $G$ is trivial. 
\end{proof}

\begin{cor}
    xx
\end{cor}

\begin{proof}
    
\end{proof}

\begin{itemize}
    \item Translate, proposition for $G$ trivial.
    \item Case of simple as superalgebra, apply result on odd with order $4$.
    \item As a paragraph, for clarity, hence it is $\End_\FF (U)$, and by result, comes from for $\langle \, , \, \rangle$ on $U$.
    \item Proposition, as corollary of Chapter 3.
\end{itemize}

Then... double case

Recall that, over an algebraically closed field, every finite dimensional simple associative superalgebra is either isomorphic to $M(m,n)$ or to $Q(n)$ (prop ??).
Hence, Proposition \ref{prop:only-SxSsop-is-simple} tells us that if $(R, \vphi)$ is a  finite dimensional superinvolution-simple superalgebra, then $R$ is isomorphic to either $M(m,n)$, $Q(n)$, $M(m,n) \times M(m,n)\sop$ or $Q(n) \times Q(n)\sop$.
However, it is well known that the associative superalgebra $Q(n)$ does not admit a superinvolution (see, for example, ?? and ?? or Corollary \ref{cor:Q-no-spuerinv-center}, below),
% (it is a well-known fact, see ?? and ??, and it follows as corollary of the theoryand we have a proof for it later this section, , and it also follows from ??). 
hence, our superalgebra $R$ can only belong to 3 different families of superalgebras.
We will say $(R,\vphi)$ is of \emph{type} $M$, $M\times M\sop$ or $Q\times Q\sop$ according to which family it belongs.
And, as we are going to see, we can use the center of $R$ to distinguish among them.

\begin{lemma}
	Let $(R, \vphi)$ be a superalgebra with super-anti-automorphism.
	Then $Z(R)$ is $\vphi$-invariant.
\end{lemma}

\begin{proof}
	By \cref{lemma:center-is-graded}, with $G = \ZZ_2$, we have $Z(R) = Z(R)\even \oplus Z(R)\odd$, so it is sufficient to show that if $c \in Z(R)\even \cup Z(R)\odd$, then $\vphi(c) \in Z(R)$. 
	Let $r \in R\even \cup R\odd$.
	Since $c\vphi\inv (r) = \vphi\inv (r)c$, we can apply $\vphi$ on both sides and get $\sign{c}{r} r \vphi(c) = \sign{c}{r} \vphi(c) r$ and, hence, $r \vphi(c) = \vphi(c) r$.
\end{proof}

\begin{cor}\label{cor:Q-no-spuerinv-center}
	If $\Char \FF \neq 2$, the associative superalgebra $Q(n)$ does not admit a superinvolution.
\end{cor}

\begin{proof}
	The center of $Q(n)$ is isomorphic to $\FF1 \oplus \FF u$, where $u$ is an odd element with $u^2 = 1$.
	Let $\vphi$ be a super-anti-automorphism on $Q(n)$.
	Since $u$ is odd and central, $\vphi(u)$ is odd and central.
	Hence there is $\lambda \in \FF$ such that $\vphi(u) = \lambda u$.
	Using that $u^2 = 1$, we have $1 = \vphi(1) = \vphi(u^2) = - \vphi(u)^2 = - \lambda^2$.
	But then $\vphi^2 (u) = \lambda^2 u = -u \neq u$, hence $\vphi^2 \neq \id$.
\end{proof}

% \begin{defi}
%     Let $R$ be a superalgebra and let $\vphi\from R\to R$ be a superinvolution. 
%     We say that $\vphi$ is of the \emph{first kind} if it fixes all the elements of $Z(R)$. 
%     Otherwise, we say that $\vphi$ is of the \emph{second kind}.
% \end{defi}

\begin{prop}\label{prop:types-of-SA-via-center}
	Let $(R, \vphi)$ be a superalgebra with superinvolution.
	\begin{enumerate}[(i)]
		\item If $(R, \vphi)$ is of type $M$, then $(Z(R), \vphi) \iso (\FF, \id)$;\label{item:F-id}
		\item If $(R, \vphi)$ is of type $M\times M\sop$, then $(Z(R), \vphi)$ is isomorphic to the superalgebra with superinvolution in Example \ref{ex:FxF-iso-FZ2};\label{item:FZ2-exchg}
		\item If $(R, \vphi)$ is of type $Q\times Q\sop$, then $(Z(R), \vphi)$ is isomorphic to the superalgebra with superinvolution in Example \ref{ex:FZ2xFZ2sop-iso-FZ4}.\label{item:FZ4-exchg}
	\end{enumerate}
\end{prop}

\begin{proof}
	Item \eqref{item:F-id} follows from the fact that $Z(M_{m+n}(\FF)) \iso \FF$.
	It is easy to check that $Z(S\times S\sop) = Z(S) \times Z(S)\sop$ for any superalgebra $S$, so items \eqref{item:FZ2-exchg} and \eqref{item:FZ4-exchg} follow from $Z(M_{m+n}(\FF)) \iso \FF$ and $Z(Q(n)) \iso Q(1)$.
\end{proof}

\section{Gradings on superinvolution-simple superalgebras}

For the remainder of the section, we will consider, as before, $R \coloneqq \End_\D(\U)$, where $\D$ is a graded division superalgebra and $\U$ is a nonzero graded right $\D$-module of finite rank.
Also, let $\vphi$ be a super-anti-automorphism on $R$ and let $(\vphi_0, B)$ be pair determining $\vphi$ as in Theorem \ref{thm:vphi-iff-vphi0-and-B} and following Convention \ref{conv:pick-even-form}.

We will now show that we can identify $(Z(\D), \vphi_0)$ with $(Z(R), \vphi)$.
For every $c\in Z(\D)$, consider $r_c\from \U \to \U$ given by $r_c(u) = uc$.
Clearly, $r_c$ is $\D$-linear, so $r_c \in R$.
Actually, we have $r_c\in Z(R)$.
Indeed, for all $r\in R = \End_\D(\U)$ and all $u\in \U$, we have $r (r_c(u)) = r(uc) = r(u) c = r_c(r(u))$.

\begin{prop}%\label{prop:R-and-D-have-the-same-center}
	The map $Z(\D) \to Z(R)$ given by $c \mapsto r_c$ is an isomorphism of $G$-graded superalgebras.
	Moreover, $\vphi (r_c) = r_{\vphi_0(c)}$.
\end{prop}

\begin{proof}
	Given $r\in Z(R)$, we can define $c_r\in \End_R (\U) =\D$ by $uc_r = r(u)$ for all $u\in \U$.
	Computations analogous to the ones above show that $c\in Z(\D)$, and it is clear that the map $r\mapsto c_r$ is the inverse of the map $c \mapsto r_c$.
	The definition of grading on $R = \End_\D (\U)$ implies that these maps are isomorphisms of $G$-graded superalgebras.

	For the ``moreover'' part, fix $c\in Z(\D)\even \cup Z(\D)\odd$ and let $u, v \in \U\even \cup \U\odd$.
	On the one hand,
	\begin{align*}
		B(uc,v) = B(r_c u), v) = (-1)^{|r_c||u|} B(u, \vphi(r_c) v) = (-1)^{|c||u|} B(u, \vphi(r_c) v).
	\end{align*}
	On the other hand,
	\begin{align*}
		B(uc, v) = (-1)^{(|B| + |u|) |c|} \vphi_0(c) B(u, v) & = (-1)^{(|B| + |u|) |c|} B(u, v) \vphi_0(c)     \\
		                                                     & = (-1)^{(|B| + |u|) |c|} B(u, v \vphi_0(c) )    \\
		                                                     & = (-1)^{(|B| + |u|) |c|} B(u, r_{\vphi_0(c)}v).
	\end{align*}
	%
	Since we are following Convention \ref{conv:pick-even-form}, either $\D$ is even, and hence $|c| = \bar 0$, or
	$|B| = \bar 0$.
	In any case, $|B||c| = \bar 0$, so we have that \[(-1)^{|c||u|} B(u, \vphi(r_c) v) = (-1)^{|c||u|} B(u, r_{\vphi_0(c)}v),\] and, hence, $B(u, (\vphi(r_c) - \vphi_0(c)) v) = 0$ for all $u, v \in \U$.
	Since $B$ is nondegenerate, the results follows.
\end{proof}

\begin{prop}\label{prop:vphi-R-simple-D-simple}
	The superalgebra $R$ is $\vphi$-simple if, and only if, the superalgebra $\D$ is $\vphi_0$-simple.
\end{prop}

\begin{proof}
	Pick a homogeneous $\D$-basis for $\U$ following Convention \ref{conv:pick-even-basis} and use it to identify $R$ with $M_k(\D) = M_k(\FF) \tensor \D$.
	By \cref{conv:pick-even-form}, we may assume that $B$ is even and then,
	by Proposition \ref{prop:matrix-vphi}, for every $X \in M_k(\D)$, we have
	$\vphi(X) = \Phi\inv \vphi_0(X\stransp) \Phi$, where $\Phi \in M_k(\D)$ is the matrix representing $B$. 

	It is well known that the ideals of $M_k(\D)$ are precisely the sets of the form $M_k(I)$ for $I$ an ideal of $\D$.
	We will prove an analog of this, first, for superideals and, then, for $\vphi$-invariant superideals.

	If $I$ is a superideal, $M_k(I) = M_k(\FF) \tensor I$ is also a superideal since it is spanned by a set of $\ZZ_2$-homogeneous elements, namely, the elements of the form $E_{ij}\tensor d$ where $1 \leq i,j \leq k$ and $d \in I\even \cup I\odd$.
	Conversely, if $J = M_k(I)$ is a superideal, then we can write $I = \{ d\in  \D \mid E_{11}\tensor d \in J\}$.
	For every $d\in I$, write $d = d_{\bar 0} + d_{\bar 1}$, where $d_\alpha \in \D^\alpha$, $\alpha \in \ZZ_2$.
	Since the $\ZZ_2$-homogeneous components of $E_{11}\tensor d$ are $E_{11}\tensor d_{\bar 0}$ and $E_{11}\tensor d_{\bar 1}$ and they belong to $J$, we have $d_{\bar 0}, d_{\bar 1} \in I$.

	Now we are going to show that $M_k(I)$ is $\vphi$-invariant if, and only if, $I$ is $\vphi_0$-invariant.
	Suppose $I$ is $\vphi_0$-invariant.
	Then if $X \in M_k(I)$, it is clear that $\vphi_0 (X\stransp)$ is also in $M_k(I)$.
	But then $\vphi(X) = \Phi\inv \vphi_0(X\stransp) \Phi \in M_k(I)$ since $M_k(I)$ is an ideal.
	Conversely, suppose $M_k(I)$ is $\vphi$-invariant.
	Let $d \in I$ and consider $X = E_{11} \tensor d \in M_k(I)$.
	Then $E_{11} \tensor \vphi_0(d) = \vphi_0(X\stransp) = \Phi\, \vphi(X)\, \Phi\inv \in M_k(I)$, which shows that $\vphi_0(d) \in I$.
\end{proof}

\begin{cor}\label{cor:D-has-the-same-type}
	Suppose $\FF$ is an algebraically closed field and $\Char \FF \neq 2$.
	Assume that $\vphi$ is a superinvolution and that $R$ is $\vphi$-simple.
	Then $(R, \vphi)$ is of the same type as $(\D, \vphi_0)$. \qed
\end{cor}

% \begin{cor}\label{cor:D-has-the-same-type}
%     Suppose $\FF$ is an algebraically closed field and $\Char \FF \neq 2$. 
%     Assume that $\vphi$ is a superinvolution and that $R$ is $\vphi$-simple.
%     %
%      \begin{enumerate}[(i)]
%         \item If $R\iso M(m,n)$, then there are $m', n'\geq 0$ such that $\D \iso M(m', n')$;
%         \item If $R\iso M(m,n)\times M(m,n)\sop$ with exchange superinvolution, then there are $m', n'\geq 0$ such that $(\D, \vphi_0)$ is isomorphic to $M(m', n')\times M(m', n')\sop$ with exchange superinvolution;
%         \item If $R\iso Q(n)\times Q(n)\sop$ with exchange superinvolution, then there is $n' \geq 0$ such that $(\D, \vphi_0)$ is isomorphic to $Q(n')\times Q(n')\sop$ with exchange superinvolution.\qed
%     \end{enumerate}
% \end{cor}

We will obtain more precise information about $\D$ in the next section.

% As a consequence of this last Corollary, we can only put a $G$-grading on $M(m,n)\times $Em particular, em MxM precisamos de um elemento par de ordem 2 e em QxQ precisamos de um elemento impar de ordem 4 (comparar com resultado no paper sobre Q).

% \begin{remark}\label{rmk:we-only-need-B'-nonzero}
%     Note that in the proof of the ``moreover'' part of Theorem \ref{thm:vphi-iff-vphi0-and-B}, we do not need to assume $B'$ is nondegenerate, we only use that it is nonzero.
% \end{remark}

% \section{Graded division superalgebras with super-anti-automorphism}
% % -----------------------------------------

% We are now going is to investigate the finite dimensional graded division superalgebras that admit a superinvolution. 
% Throughout this section, we will assume that $\FF$ is an algebraically closed field with $\Char \FF \neq 2$, and we will fix a primitive fourth root of unity $i \in \FF$.
% %In this case, because of Theorem \ref{thm:vphi-involution-iff-delta-pm-1}, we are primarily interested in the case the graded division superalgebra admits a superinvolution.
% Our final goal is to classify the division gradings on finite dimensional superinvolution-simple associative superalgebras.

% Recall that a graded division superalgebra is the same as a graded division algebra if we consider the $G^\#$-grading. 
% In particular, the isomorphism class of a finite dimensional graded division superalgebra $\D$ is determined by a pair $(T, \beta)$ where $T \coloneqq \supp \D \subseteq G^\#$ is a finite abelian group and $\beta\from T\times T \to \FF^\times$ is an alternating bicharacter {\tt (see ??)}. 
% \marginpar{\tt (for the ``see'' part: [EK] $D_4$ and [BK] gradings on classical \\lie algebras)}
% Instead of writing subscripts for the $G$-grading and superscripts for the canonical $\ZZ_2$-grading, it will be convenient to write $\D = \bigoplus_{t\in T} \D_t$ and recover the parity via the map $p\from T \to \ZZ_2$ which is the restriction of the projection $G^\# = G\times \ZZ_2 \to \ZZ_2$.

% Since each component $\D_t$ of $\D$ is one-dimensional, an invertible degree-preserving map $\vphi_0\from \D \to \D$ is completely determined by a map $\eta\from T \to \FF^\times$.

% \begin{prop}\label{prop:superpolarization}
%     Let $\vphi_0\from \D \to \D$ be the invertible degree-preserving map determined by $\eta\from T \to \FF^\times$  such that $\vphi_0(X_t) = \eta(t) X_t$ for all $t\in T$ and $X_t\in \D_t$. 
%     Then $\vphi_0$ is a super-anti-automorphism if, and only if,
%     %
%     \begin{equation}\label{eq:superpolarization}
%         \forall a,b\in T, \quad (-1)^{p(a) p(b)} \beta(a,b) =  \eta(ab) \eta(a)\inv \eta(b)\inv.
%     \end{equation}
%     %
%     Moreover, $\D$ admits a super-anti-automorphism if, and only if, $\beta$ only takes values $\pm 1$.
% \end{prop}

% \begin{proof}
%     For all $a,b \in T$, let $X_a \in \D_a$ and $X_b\in \D_b$. Then:
%     %
%     \begin{alignat*}{2}
%         &&\vphi_0(X_a X_b) &= (-1)^{p(a) p(b)} \vphi_0(X_b) \vphi_0(X_a)\\
%         \iff&&\,\, \eta(ab)X_a X_b &= (-1)^{p(a) p(b)} \eta(a) \eta(b) X_b X_a\\
%         \iff&&\, \eta(ab)X_a X_b &= (-1)^{p(a) p(b)} \eta(a) \eta(b) \beta(b,a) X_a X_b\\
%         \iff&& \eta(ab) &= (-1)^{p(a) p(b)} \eta(a) \eta(b) \beta(b,a)
%         \\
%         \iff&& (-1)^{p(a) p(b)} \beta(b, a) &=  \eta(ab) \eta(a)\inv \eta(b)\inv.
%     \end{alignat*}
%     The right-hand side of this last equation does not change if $a$ and $b$ are switched, hence $\beta(b,a) = \beta(a,b)$. 

%     Since $\beta$ is alternating, $\beta(b, a) = \beta (a, b)\inv$, therefore $\beta(a,b)^2 = 1$, which proves one direction of the ``moreover'' part.
%     %from where we conclude that $\beta(a,b) = \pm 1$. 
%     The converse follows from the fact that the isomorphism class of $\D\sop$ is determined by $(T, \beta\inv)$, so if $\beta$ takes only values in $\{ \pm 1 \}$, there must be an isomorphism from $\D$ to $\D\sop$, which can be seen as a super-anti-isomorphism on $\D$.
% \end{proof}

% With this we can translate our task to the level of abelian groups. 
% Instead of considering the graded superalgebra with super-anti-automorphism $(\D, \vphi_0)$, we can focus, instead, on the data $(T, \beta, p, \eta)$, where $T$ is a finite abelian group, $\beta$ is an alternating bicharacter on $T$, $p\from T \to \ZZ_2$ is a group homomorphism and $\eta\from T \to \FF^\times$ satisfies Equation \eqref{eq:superpolarization}. 
% Also, the condition of $\vphi_0$ being a superinvolution clearly corresponds to $\eta(t) \in \{ \pm 1 \}$ for all $t\in T$.


\section{Application: Superinvolutions on Simple Superalgebras}\label{ssec:simple-sa-but-with-vphi}

\begin{itemize}
    \item We will classify first the graded-simple superalgebras with superinvolution (case $G$ trivial), and then the gradings on them.
    \item Note that, from lemma bla, for general $G$, we have that simple as superalgebra implies type $M$ and even gradings.
    \item In particular, for trivial $G$, the superalgebra must be of the form $\End_\FF (U)$ for a superspace $U$.
    \item Lemm
    \item Prop: G trivial, $G^\# = \ZZ_2$.
    \item even grading, $T = e$ so only one $\eta$.
    \item The action is only a $G^\# = \ZZ_2$ action
    \item fix $\delta = 1$, as above, and then we have only a $\mc G = e$ action.
    \item Good, no orbits, only points
    \item Conditions of admissibility:
    \item i and ii are trivial now
    \item define $k(0) = m$ and $k(1) = n$
    \item iii becomes k(i) = k(g0i)
    \item in the case $g_0 = \bar 1$, so $m = n$. In the other case, trivial
    \item condition (iv): for any $i \in \ZZ_2$, if $g_0 + 2i = \bar 0$, \ie, $g_0 = 0$ and $\mu_i = (-1)^{i} = -1$, \ie, $i = \bar 1$, then $k(i) = m$ is even.
    \item the construction we have there implies we have $\End_\FF (\FF^m \oplus \FF^n)$ with the matrices of $\vphi$ given by what they have to be.
    \item this is what we have in the introduction.
\end{itemize}

Nice, let us repeat the same proof for general $G$ to get what we have to get

\begin{itemize}
    \item still, $T$ is even.
    \item also, elementary $2$-group
    \item What are the possible $\eta$? given one $\vphi_0$, any other is composition with a automorphism
    \item hence, by lemma, we get $\chi \eta$, since each automorphism correspond to $\chi$
    \item By nondegeneracy, one class at most!
    \item but is there one?
    \item yes, we can refer to paper and standard realization, and get the transpose
    \item in other words, $\eta(ab) = \beta(a,b)$
\end{itemize}

Fine, so only one class of $\eta$, fix $\eta$ as transpose and let's go. 
Then make $\delta = 1$. 

What is the $I^+$? 

\begin{itemize}
    \item Once we fixed $\delta$, we cannot act by $g$ odd... $\mc G$ is precisely $G$.
    \item again, conditions of admissibility...
    \item Ugly, isn't it?
    \item 
\end{itemize}

Calm down. 
From $\End_\D(\U)$ and $B$, how we find the underlying superalgebra?

\begin{itemize}
    \item Take $\D$ standard realization with transposition
    \item From $\U$ to $U$... Take $U = \U \tensor \FF B$?
    \item Nope
    \item Take the basis with the matrices as in propositions 4.40 and 3.42
    \item Take $B_x$ as defined there.
    \item from this many $\FF$-forms, we get a big $\FF$-form
    \item 
\end{itemize}

Other approach:

\begin{itemize}
    \item Ok, from kappas we get $m$ and $n$, as in corollary in chap 2.
    \item We need only to see what is the parity of $g_0$. 
    It must be the same! 
    But how to argue that?
    \item Well, we need a standard realization any way...
    \item The problem here is that we can't work only on the superalgebra level as before, we need to go to the superspace level.
    \item ok, back to define $U_x = V_x \tensor \FF B$
\end{itemize}