\chapter*{Introduction}

\section{Basic Concepts}

\subsection{Gradings and superalgebras} Throughout this paper we will be always talking about group gradings and superalgebras, so we start defining these. For our purposes, all algebras and vector spaces are over a fixed algebraically closed field $\mathbb{F}$ such that $\operatorname{char} \mathbb{F}=0$.

% defi: grading
\begin{defi}\label{def:grading}
	Let $G$ be a group. A $G$-\emph{grading} on an algebra $A$ is a vector space decomposition indexed by the elements of $G$
	\[\Gamma : A= \bigoplus_{g \in G} A_g,\]
	such that $ab \in A_{gh}$, for all $a \in A_{g}$ and $b \in A_{h}$. We say that the elements elements of $A_g$ are \emph{homogeneous of degree $g$}.
\end{defi}

% defi: graded algebra and superalgebra
It is usual to fix a grading on an algebra, considering the pair $(A, \Gamma)$ as one object, called \emph{$G$-graded algebra}. A \emph{superalgebra} is simply a $\Zmod2$-graded algebra; however, the concept of superalgebra of a certain variety (e.g., associative, Lie, Jordan) is less straightforward, as we will see at Subsection \ref{subsec:super-analog}.

% superscripts and names even, odd
To distinguish the canonical $\mathbb{Z}_2$-grading of a superalgebra from any other $G$-grading that it may carry, we will use superscripts to label the components: $A=A\even \oplus A\odd$. Also, the homogeneous elements of degree $\bar0$ are called \emph{even} and the homogeneous elements of degree $\bar1$ are called \emph{odd}.

\subsection{Superalgebra of a variety}\label{subsec:super-analog}

We will show now a general way to define the ``super'' analog of a class of algebras. We start defining what is the Grassman superalgebra of vector space:

\begin{defi}\label{def:Grassmann-algebra}
	Let $V$ be a vector space. The Grassmann algebra of $V$ is
	%
	\[\mc G(V) = \sum_{k=0}^{+\infty} \Exterior^k V\]
	%
	with product given by the ``$\wedge$'' operation.

	It has a natural $\Zmod2$-grading, given by $\mc G(V)\even = \sum_{i=0}^{+\infty}\Exterior^{2i} V$ and $\mc G(V)\odd = \sum_{i=0}^{+\infty}\Exterior^{2i+1} V$, so we can consider it as a superalgebra.
\end{defi}

For the scope of this subsection, $V$ will be a fixed vector space with a infinite enumerable basis and will denote its Grassmann algebra simply by $\mc G$. If $A=A\even\oplus A\odd$ is a superalgebra we define its \emph{Grassmann envelope} by $A\even\tensor \mc G\even + A\odd\tensor \mc G\odd$, considered as an algebra.

Let $\mc R$ be a class of algebras (\eg, associative, Lie, Jordan). We define the class of $R$-superalgebras, denoted by $\mc R_S$ the class of superalgebras $A$ such that its Grassmann envelope is in $R$ and call its elements $\mc R$-superalgebras. In the case of $R$ being the class of Lie algebras we call the elements of $R_S$ Lie superalgebras.

% Lie superalgebras

\subsection{Lie Superalgebras}
Following the procedure of last subsection, one can see that a \emph{Lie superalgebra} is not (in general) a $\Zmod2$-graded Lie algebra. The notion can be axiomatized as follows (with product denoted by brackets)

\begin{defi}
	A \emph{Lie superalgebra} is a superalgebra $L=L\even\oplus L\odd$ that satisfies, for all $a \in L^{\alpha}$ and $b \in L^{\beta}$, $\,\alpha,\beta\in\Zmod2$, the following two axioms\footnote{However, in characteristics  2 and 3 there are further requirements, see \cite[Subsection 1.2]{MR1192546} }:

	\begin{enumerate}[(i)]
		\item \makebox[6.8cm][l]{$[ a, b ] = - (-1)^{\alpha \beta} [b, a]$}(super anticommutativity)
		\item \makebox[6.8cm][l]{$[a,[b,c]] = [[a,b],c] + (-1)^{\alpha \beta} [b, [a,c]]$}(super Jacoby identity)
	\end{enumerate}
\end{defi}

A important remark is that, by definition, the even component $L\even$ is a Lie algebra and the odd component $L\odd$ is an $L\even$-module. This does not encode the Lie superalgebra structure entirely, but it is very helpful.

\subsection{Gradings on superalgebras}

% granding on superalgebra
By a \emph{$G$-grading on a superalgebra $A = A\even\oplus A\odd$} we mean a $G$-grading $\Gamma : A= \bigoplus_{g \in G} A_g$ that is also compatible with its canonical $\Zmod2$-grading, \ie, $A_g=\left(A_g\cap A\even\right)\oplus\left(A_g\cap A\odd\right)$ for every $g\in G$.

In particular, a $G$-grading on a Lie superalgebra $L=L\even\oplus L\odd$ can be restricted to its even and odd parts, in the sense that $L\even = \bigoplus_{g\in G} L_g\cap L\even$ and $L\odd = \bigoplus_{g\in G} L_g\cap L\odd$. This makes a $L\even$ a graded Lie algebra and makes $L\odd$ a graded module:

\begin{defi}
	Let $G$ be a group, $L = \bigoplus_{g\in G} L_g$ be a $G$-graded Lie algebra and $V$ is a module over $L$. A $G$-grading on $V$ is a vector space decomposition $V=\bigoplus_{g\in G} V_g$ such that $L_a\cdot V_b \subseteq V_{ab}$ for all $a,b\in G$.
\end{defi}

\section{Motivation}

Gradings on Lie algebras first appeared in W.~ Killing’s 1888 paper \cite{MR1510529}, where the root space decomposition of a complex semisimple Lie algebra $L$ was introduced. This is a grading by $\ZZ^n$, where $n$ is the rank of $L$, and it turned out to be of fundamental importance in the theory of Lie algebras. Gradings by the cyclic group $\ZZ_2$ appeared in the work of E.~Cartan on real semisimple Lie algebras \cite{Cartan-1914}. The interest in gradings increased in the 1960's, in connection with the works of J.~Tits, I.L.~Kantor, and M.~Koecher \cite{Tit62,Kan64,Koe67}. V.~Kac classified gradings by cyclic groups on complex semisimple Lie algebras and used them in the theory of symmetric spaces in differential geometry \cite{Kac68} and later for the construction of the so called twisted loop algebras, which are fundamental for the theory of the famous affine Kac--Moody Lie algebras \cite{Kac90}. Gradings remain important in the theory of Lie algebras and their representations ever since.

According to \cite{Freund:1986ws}, Lie superalgebras' first appearances were related to cohomology (see  \cite{FN56}, \cite{Gen63}, \cite{Gen64} and \cite{MM65}). About the same time, they were independently introduced in physics, in connection with the so called supersymmetries (see \cite{GN64}, \cite{Miy68} and \cite{Mic69}). They become a mainstream topic in theoretical physics with the development of string theory in the 1970's.

In one of the earliest papers on gradings on Lie superalgebras (\cite{MR974693}), J.~Van der Jeugt wrote:

\begin{quote}\emph{Gradings for Lie (super)algebras are important because they give rise to preferred bases of the algebra which admit `additive quantum numbers'.}\end{quote}

Another motivation is that the classification problem for Lie colored superalgebras can be reduced to the classification of the gradings on Lie superalgebras, see \cite{MR2497949}.

\section{Simple Lie superalgebras}

An \emph{ideal} $I$ of a Lie superalgebra $L=L\even\oplus L\odd$ is an ideal of the underlying algebra that is compatible with the canonical $\Zmod2$-grading in the sense that $I=\left(I\cap A\even \right) \oplus \left(I\cap A\odd \right)$. The superalgebra $L$ is said to be \emph{simple} if it has no proper nonzero ideals and $[L,L]\neq 0$.

The simple finite dimensional Lie superalgebras (that are not Lie algebras) were classified by V. G. Kac, see \cite{artigokac}, \cite{livrosuperalgebra}. They are divided in two big classes:

\begin{defi}
	Let $L=L\even \oplus L\odd$ be a simple Lie superalgebra.
	\begin{enumerate}[(i)]
		\item We say that $L$ is \emph{classical} if $L\odd$ is a semisimple module over $L\even$.
		\item We say that $L$ is of \emph{Cartan type} if $L\odd$ has a largest proper submodule, \ie, a proper submodule that contains all proper submodules.
	\end{enumerate}

\end{defi}

Every simple finite dimensional Lie superalgebra (that is not a Lie algebra) are either classical or of Cartan type. The classical ones are, in their turn,  divided as follows:

\begin{itemize}
	\item 4 series, $A(m,n)$, $B(m,n)$, $C(n)$, $D(m,n)$, which are generalizations of the corresponding series of simple Lie algebras;

	\item 2 series, $P(n)$ and $Q(n)$, which are called the \emph{strange Lie superalgebras};

	\item 3 exceptional cases: $F(4)$, $G(3)$ and the family $D(2,1,\alpha)$, $\alpha \in \mathbb{F}\setminus \{0,1\}$.
\end{itemize}

The Lie superalgebras of Cartan type are divided in 4 series:
%
\begin{itemize}
	\item the \emph{Witt algebras} $W(n)$;
	\item the \emph{special algebras} $S(n)$ and their deformations, $\tilde S(n)$;
	\item the \emph{Hamiltonian algebras} $H(n)$.
\end{itemize}

This algebras are the analogous of the corresponding simple infinite dimensional Lie algebras of Cartan type, as well as the simple restricted Lie algebras of Cartan type.

It is important to say that there are restrictions to the parameters above. We are going to explicit them together with the definitions of these Lie superalgebras.

\subsection{The classical Lie superalgebras}\label{sec:defi-classical-SA}

Through all this subsection $m, n$ are always non-negative integers and $U$ is a finite dimensional vector space with fixed decomposition $U=U\even\oplus U\odd$.

Before we define the classical Lie superalgebras we need to define an important non-simple Lie superalgebra that is related to most of them: the \emph{general linear Lie Superalgebra} $\gl(U)$. Its underlying set is
%
\[\End(U)=\left(\begin{matrix}
			\End(U\even)       & \Hom(U\odd,U\even) \\
			\Hom(U\even,U\odd) & \End(U\odd)        \\
		\end{matrix}
	\right).\]
%
We define the $\Zmod2$-grading on $\gl(m|n)$ by declaring
\[\gl(m|n)\even = \left(\begin{matrix}
			\End(U\even) & 0           \\
			0            & \End(U\odd) \\
		\end{matrix}
	\right)\] and
\[\gl(m|n)\odd = \left(\begin{matrix}
			0                  & \Hom(U\odd,U\even) \\
			\Hom(U\even,U\odd) & 0                  \\
		\end{matrix}
	\right).\]
%
We define the product of $a\in\gl(U)^{\alpha}$ and $b\in\gl(U)^{\beta}$, for all $\alpha,\beta\in\Zmod2$, by
\[ [a,b] = ab - (-1)^{\alpha\beta}ba\]
and extend it to the whole algebra by linearity.

If $U\even = \FF^m$ and $U\odd = \FF^n$ we denote $\gl(U)$ by $\gl(m|n)$.

\subsubsection{The series $A(m,n)$} The \emph{special linear Lie superalgebra}, denoted by $\Sl (m|n)$, is the derived algebra of $\gl(m|n)$. Following the matrix notation we had above, we can write:
%
\[ \Sl(m,n) =\left\{
	\left(\begin{matrix}
			a & b \\
			c & d \\
		\end{matrix}\right) \mid \tr a = \tr d\right\}.
\]
%
If $m\neq n$ then $\Sl(m|n)$ is already a simple Lie superalgebra. However, if $m=n$, the identity map $1\in \Sl(n|n)$ is a central element and hence $\Sl(n|n)$ is not simple, but the quotient $\Sl(n|n)/ \FF 1$ is.

For $m$,$n\geq 0$, the simple Lie superalgebra $A(m,n)$ is defined to be $\Sl(m+1|n+1)$ if $m\neq n$, and to be $\Sl(n+1|n+1)/ \FF 1_U$ if $m=n$.

We have that $A(m,n)\iso A(n,m)$, so we also ask $m\geq n$ to avoid repetitions.

\subsubsection{The series $B(m,n)$, $C(n)$ and $D(m,n)$}

Let $B:U\times U \rightarrow \FF$ be a bilinear form on $U$. We say that $B$ is \emph{supersymmetric} if, for all $\alpha,\beta\in\Zmod2$, $x\in U^{\alpha}$ and $y\in U^{\beta}$ we have that
\[B(x,y) = (-1)^{\alpha\beta}B(y,x).\]

Also, we say that $B$ is \emph{even} if $F(V^0,V^1) = F(V^1,V^0)=0$ and \emph{odd} if $F(V^0,V^0) = F(V^1,V^1)=0$.

If $B$ is a nondegenerate, supersymmetric and even bilinear form on $V$, we define

\begin{itemize}
	\item $\osp(m|n)\even = \{T\in \End (V)^{0} \mid F(T(x),y) = - F(x,T   (y))\,\, \forall x,y\in V\}$;
	\item $\osp(m|n)\odd = \{T\in \End (V)^{1} \mid F(T(x),y) = -(-1)^{\alpha} F(x,T(y))
		      \,\, \forall x\in V^{\alpha}, \alpha\in\mathbb{Z}_2 \text{ and } \forall y\in V\}$.
\end{itemize}

We then define

\begin{itemize}
	\item $B(m,n) = \osp(2m+1,2n)$, for $n\geq 0$ and $m\geq 1$;
	\item $C(n) = \osp(2, 2(n-1))$, for $n\geq 2$;
	\item $D(m,n) = \osp(2m,2n)$, for $m\geq 2$ and $n\geq 1$.
\end{itemize}
But we have that $C(2)\iso A(1,0)$, so, to avoid repetition, we ask for $n\geq 3$ in the $C(n)$ case.

\subsubsection{The series $P(n)$}
From now on, we suppose that $B$ is non-degenerate, supersymmetric and odd. In this case, it is clear that $U\odd$ is isomorphic to $(U\even)^*$ by means of the map $u \mapsto B(u, \cdot)$. %Again supposing $U$ to be finite dimensional, let us put $n + 1 = \Dim (U\even) = \Dim (U\odd)$.

\subsubsection{The series $Q(n)$}

This series, called \emph{queer}, is different in that we have to start from an associative superalgebra that is simple as a superalgebra but not as an algebra. Namely, let $A=R\times R$, with component-wise product, where $R=M_{n+1}$, $n\ge 1$. Then $(x,y)\mapsto(y,x)$ is an automorphism of $A$ of order $2$ and hence its eigenspace decomposition is a $\mathbb{Z}_2$-grading on $A$, with $A^\bz=\{(x,x)\;|\;x\in R\}$ and $A^\bo=\{(x,-x)\;|\;x\in R\}$. Note that $A^\bz$ is isomorphic to $R$ as an algebra and $A^\bo=uA^\bz$ where $u=(1,-1)$, so we may write $A=R\oplus uR$ where $u$ is odd, commutes with the elements of $R$ and satisfies $u^2=1$. This latter definition works even if $\mathrm{char}\,\mathbb{F}=2$. The associative superalgebra $A$ can be identified with a subalgebra of $M(n+1,n+1)$ as follows:
\[
	\left\{
	\left[
		\begin{matrix}
			a & b \\
			b & a \\
		\end{matrix}
		\right] \in M(n+1,n+1) \; \Big| \;
	a,b \in M_{n+1} \right\}\stackrel{\sim}{\to}A,\quad
	\left[\begin{matrix}
			a & b \\
			b & a \\
		\end{matrix}\right]\mapsto a+ub.
\]
Let $\widetilde{Q}(n)$ be the derived superalgebra of $A^{(-)}$. Then
\[
	\widetilde{Q} (n) = \left\{
	\left[
		\begin{matrix}
			a & b \\
			b & a \\
		\end{matrix}
		\right] \in M(n+1,n+1) \; \Big| \;
	a,b \in M_{n+1},\,\tr(b)=0 \right\}.
\]
Set $Q(n)$ to be the quotient of $\widetilde{Q}(n)$ by its center, which is spanned by the identity matrix:
\[
	Q(n)=\frac{\widetilde{Q} (n)}{\mathbb{F}1}.
\]

The Lie superalgebra $Q(n)$ is simple for $n\geq 3$.

The \emph{periplectic Lie superalgebra} $\mathfrak{p}(U)$ is defined as $\mathfrak{p}(U)\even \oplus \mathfrak{p}(U)\odd$ where \[\begin{aligned}
		\mathfrak{p}(U)^{i} = \{T\in \gl(U)^i\mid&\,\, B(T(x),y) = - (-1)^{i\alpha}B(x,T(y))\\ &\text{for all}\,\,\alpha\in\Zmod2,\,\, x\in U^{\alpha} \AND y\in U\}\end{aligned}\] for all $i\in\Zmod2$. The superalgebra $\mathfrak{p}(U)$ is not simple, but its derived superalgebra is.

The Lie superalgebra $P(n)$ is defined to be $[\mathfrak{p}(U),\mathfrak{p}(U)]$ in the case \\ $\dim (U\even) =$ $\Dim (U\odd) = n+1$.

The isomorphism class of $P(n)$ does not depend on the choice of the bilinear form $B$. %Since $U\even$ and $U\odd$ are in duality by $B$, we see that, as vector space with bilinear form, $U$ is isomorphic to $V = V\even\oplus V\odd$ where $V\even=U\even$ and $V\odd=(U\even)^*$ and the bilinear form $B':V\times V\rightarrow \FF$ is given by
$U\odd$ can be identified with $(U\even)^*$ and with this identification we can write $U=V\oplus V^*$, $\operatorname{dim} V = n+1$, and $B$ is given by
\[B(v_1+v^*_1,v_2 + v_2^*) = \langle v_1^*, v_2\rangle + \langle v_2^*, v_1\rangle\] for all $v_1, v_2\in V$ and $v_1^*, v_2^*\in V^*$,%
%\smallskip
%\noindent for all $u_1, u_2\in U\even$ and $u_1^*, u_2^*\in U\odd = (U\even)^*$,
%
%In this new setting it is easy to see that
%
\[\mathfrak{p}(U) = \left\{\left(\begin{matrix}
			a & b         \\
			c & -a\transp \\
		\end{matrix}
	\right) \Big| \, b=b\transp \AND c=-c\transp\right\}\]
% 
%\smallskip
and
% 
%\smallskip
\[P(n) = \left\{\left(\begin{matrix}
			a & b         \\
			c & -a\transp \\
		\end{matrix}
	\right)\Big| \tr a = 0,\, b=b\transp \AND c=-c\transp\right\}.\]

We have that $P(n)$ is simple if $n\geq 3$.

\subsubsection{Exceptional Lie Superalgebras} Since defining them here would be a long detour we refer to see \cite{artigokac}.

\subsection{Lie superalgebras of Cartan type}

\subsubsection{The series $W(n)$}
Consider the Grassmann algebra $\mc G(V)$ of a vector space $V$ of dimension $n$. We define a \emph{superderivation} of degree $\gamma\in\Zmod2$ is as a linear map $D\colon \mc{G} \mapsto \mc G$ such that $D (a\wedge b)= D (a)\wedge b + (-1)^{\alpha\gamma} a\wedge D (b)$ for all $a \in \mc G^{\alpha}$ and $b\in \mc G.$

We define $W(n)$ to be the \emph{algebra of superderivations} of $\mc G$, ie, $W(n)$ is a subsuperalgebra of $\gl $ direct sum of the space of even derivations and the space of odd derivations inside $\End(\mc G(V)\even\oplus\mc G(V)\odd)$.

From now on we fix a basis $\{e_1,\ldots,e_n\}$ for $V$. For $i\in \{1,\ldots ,n\}$ a derivation \[\partial_i \colon \mc G(V) \rightarrow \mc G(V)\] of degree $\bar 1$ is uniquely determined by \[\partial_i (e_j) =  \delta_{ij},\] for all $j\in \{1,\ldots ,n\}.$

Every element of $W(n)$ is of the form $\sum_{i=1}^{n} f_i \partial_i$ where $f_i \in \mc G$ for all $i\in\{1,\, ...\, ,n\}$.

The Lie superalgebra $W(n)$ is simple for all $n\geq 2$, but $W(2)\iso A(1,0)\iso C(2)$, so it is not even of Cartan type, hence we impose $n\geq 3$ in our list of simple Lie superalgebras.

\subsubsection{The series $S(n)$}

We define $S(n)$ as the subsuperalgebra of $W(n)$ given by

\[
	S(n) = \left\{ \sum_{i=1}^{n} f_i \partial_i \in W(n) \mid
	\sum_{i=1}^{n} \partial_i (f_i) =0
	\right\}.
\]

If $n\geq 3$, then $S(n)$ is simple. However, $S(3)\iso P(2)$), hence we only consider the case $n\geq 4$ in our list.

\subsubsection{The series $\tilde S(n)$}

Let $n$ be even and fix $\omega = 1 - e_1\wedge e_2\wedge \, ...\, \wedge e_n$. We then define $\tilde S(n)$ by

\[
	\tilde S(n) = \left\{ \sum_{i=1}^{n} f_i \partial_i \in W(n) \mid
	\sum_{i=1}^{n} \partial_i (\omega f_i) =0
	\right\}.
\]

It is good to notice that if we change the $\omega$ above by any other element in $W(n)$ the result would be isomorphic to $S(n)$ or $\tilde S(n)$.
We have that $\tilde S(n)$ is simple for $n\geq 4$.

\subsubsection{The series $H(n)$}

We now define another multiplication on $\mc G(V)$. For every $\alpha \in \Zmod2$, all  $f\in \mc G(V)^{\alpha}$ and all $g\in \mc G$, we define the \emph{Poisson bracket} by

\[
	\{f,g\}=(-1)^{|f|} \sum_{i=1}^{n} \partial_i(f)\wedge \partial_i(g)
\]

and we extend it to the whole superalgebra by linearity. The Poisson bracket makes $\mc G$ into a Lie superalgebra which  we will denote by $\tilde{H} (n)$. This superalgebra has a center, generated by $1$. Moding out the center is not enough to make it simple: The Hamiltonian superalgebra $H(n)$ is definied to be the derived superalgebra of $\frac{\tilde{H}(n)}{\FF 1}$, which is simple for $n\geq 4$.

The algebra $H(n)$ can be seen as a subalgebra of $W(n)$. We define an embedding as follows: for every $\alpha\in\Zmod2$ and every $f\in W(n)^{\alpha}$ we send $f$ to $(-1)^{\alpha}\sum_{i=1}^n \partial_i(f)\partial_i$ and then extend it by linearity.

Though the superalgebra $H(n)$ is simple for $n\geq 4$, we have $H(4)\iso A(1,1)$, so we have to impose $n\geq 5$ in our list.

% sec:What's already known
\section{Overview of known results}

The classification of gradings on classical Lie algebras (over an algebracally closed field of characteristic different than 2) is essencially complete (see \cite{livromicha}, \cite{EK_d4}). Over characteristic $p$ we also have a classification of the gradings on the restricted Cartan type Lie algebras in the Witt and special series.  There is also a classification of graded simple modules over semisimple graded Lie algebras \cite{EK15}.

In the realm of Lie superalgebras, the  $\mathbb{Z}$-gradings on classical Lie superalgebras is known, see \cite{kacZ}. In \cite{serganova} gradings by finite cyclic groups are considered and the corresponding twisted loop superalgebras were classified. More recently, the fine gradings on the exceptional Lie superalgebras were also classified, see \cite{artigoelduque}.

It should also be mentioned that gradings by arbitrary groups on simple and $*$-simple associative superalgebras were studied in \cite{BS} and \cite{MR2535573}, but classification up to isomorphism remains open.
