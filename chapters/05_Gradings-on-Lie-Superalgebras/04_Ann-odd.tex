We now proceed to the last case, the odd gradings of Type II, which will be referred to as gradings of Type II\textsubscript{Q}. 
By \cref{thm:MxM-odd}, $(R, \vphi)$ endowed with such a grading is isomorphic to $M^{\mathrm{ex}} (T, \beta, t_p, \kappa, g_0)$ as in \cref{def:model-grd-MxM-odd-or-QxQ}. 
As we did in \cref{ssec:grds-on-Q(n)}, we will in terms of $G$ as 
As we did in \cref{ssec:grds-on-Q(n)}, we will use parameters in terms of $G$. 
 

% Need more details on the data? It won't be used directly. 

, where $T = T^+ \cup T^-\subseteq G$ is a finite subgroup of $G$ with $T^+$ being a elementary $2$-group and $T^-$ is , $\beta^+\from T^+\times T^+ \to \FF^\times$ is an alternating bicharacter with $\rad \beta^+ = \langle f \rangle$ for some $e\neq f \in T^+$, $h\in G$ is an element in $G$ such that $h^2 = f$, $g_0\in G$, and $\kappa\from G/T^+ \to \ZZ_{\geq 0}$ is a $g_0$-admissible map (see \cref{defi:odd-D-kappa-g_0-admissible}) such that $n+1 = |\kappa| \sqrt{|T^+|/2}$. 

In this case parameters $R$ are bla...
We can use $\chi$ to move it to $G$. 


For Type II gradings on $L$,
set $R \coloneqq S\times S\sop$ and let $\vphi$ be the exchange superinvolution on it. 
We will follow the parametrization for gradings on $(R, \vphi)$ making $R$ graded-simple introduced in the end of \cref{sec:MxM-and-QxQ-associative}. 
By  \cref{cor:QxQ-reduced-to-MxM}, $(R, \vphi)$ endowed with such a grading is isomorphic to $Q^{\mathrm{ex}} (T^+, \beta^+, h, \kappa, g_0)$, where $T^+ \subseteq G$ is a finite $2$-elementary subgroup, $\beta^+\from T^+\times T^+ \to \FF^\times$ is an alternating bicharacter with $\rad \beta^+ = \langle f \rangle$ for some $e\neq f \in T^+$, $h\in G$ is an element in $G$ such that $h^2 = f$, $g_0\in G$, and $\kappa\from G/T^+ \to \ZZ_{\geq 0}$ is a $g_0$-admissible map (see \cref{defi:odd-D-kappa-g_0-admissible}) such that $n+1 = |\kappa| \sqrt{|T^+|/2}$. 

We will use the parameters $(T^+, \beta^+, h, \kappa, g_0)$ to construct a representative of the corresponding isomorphism class of Type II gradings directly on the superalgebra $L$ instead of going through $\Skew(R, \vphi)$. 
Recall that the parametrization above for gradings on $\Skew(R, \vphi)$ was obtained by writing $R = R\even \oplus u R\even$, where $0 \neq u \in Z(R)\odd$, and using that $(R\even, \vphi\restriction_{R\even})$ is of type $M\times M\sop$.  
We will follow an analogous strategy here. 
Recall that $S = Q(n+1) = S\even \oplus u S\even$, where
$ 
    u \coloneqq
    \left(\begin{array}{c|c}
        0 & I_{n+1}\\
        \hline
        I_{n+1} & 0\\
    \end{array}\right)
    \in Z(S)\odd
$,
that $Z(S) = \FF1 \oplus \FF u$ can be identified with the associative superalgebra $Q(1)$, that $S\even$ can be identified with $M_{n+1}(\FF) = M(n+1, 0)$ and that $S$ can be identified with $Q(1)\tensor S\even$ via Kronecker product. 
% With these identifications, we have $S \iso Q(1)\tensor S\even$ via Kronecker product. 

We will, first, construct a $\barr G$-grading on $S\even$, where $\barr G \coloneqq G/\langle f \rangle$, following the same steps as we did in \cref{ssec:grds-on-A-m-n}, but with $T^+$ playing the role of $T$ and $\beta^+$ playing the role of $\beta$. 
Let $\pi\from G \to \barr G$ denote the natural homomorphism, set $\barr {T^+} \coloneqq T^+/\langle f \rangle$, let $\barr {\beta^+}\from \barr {T^+} \times \barr {T^+} \to \FF^\times$ be the (nondegenerate) bicharacter on $\barr {T^+}$ induced by $\beta^+$, and consider $\kappa$ as a map defined on $\barr G/\barr {T^+} \iso G/{T^+}$.

Let $\barr \D\even$ be the chosen standard realization associated to $(\barr{T^+}, \barr{\beta^+})$, let $\barr {\mu^+}\from \barr{T^+} \to \FF^\times$ be the map associated to the transposition on $\barr \D\even$, and set $\mu^+ \coloneqq \barr {\mu^+} \circ \pi$. 
Fix a subgroup $K \subseteq T^+$ such that $T^+ = K \times \langle f \rangle$, and let $\chi\from T^+ \to \FF^\times$ be the character defined by $\chi(K) = 1$ and $\chi(f) = -1$. 
Then define $\eta^+\from T^+ \to \pmone$ by
\[\label{eq:fix-eta-plus-undouble-Ann}
    \forall t\in T^+, \quad \eta^+(t) \coloneqq \mu^+(t) \chi\inv(t).
\] 
Since $\mathrm{d} \barr{\mu^+} = \barr{\beta^+}$, we get $\mathrm{d} \eta^+ = \beta^+$. 
In the proof of \cref{prop:Ann-Type-II-correspondence}, we will show that $\eta^+$ is the map associated to the superinvolution on the even part of a graded-division superalgebra as in \cref{def:std-realization-MxM-QxQ}(c). 

Extend $\chi$ to $G$ and set $k \coloneqq |\kappa|$. 
Following the construction after \cref{eq:fix-eta-undouble} in \cref{ssec:grds-on-A-m-n} with $\kappa_\bz \coloneqq \kappa$ and $\kappa_\bo$ being the zero map, we get an elementary $\barr G$-grading on $M_k(\FF) = M(k , 0)$. 
We identify the $\barr G$-graded superalgebra $M_{k}(\barr \D\even) = M_{k}(\FF) \tensor \barr\D\even$ with $S\even = M_{n+1}(\FF)$ via Kronecker product, and define 
\[\label{eq:Theta_bz-for-Ann}
    \Theta_\bz \coloneqq \begin{pmatrix}
        \Theta(\bar 0, x_1)&& \\
        & \ddots &\\
        && \Theta(\bar 0, x_{\ell})
    \end{pmatrix},
\]
where $x_1 < \ldots < x_{\ell}$ are the elements of the set $\{ x \in \supp \kappa \mid x \leq g_0\inv x\inv \}$, and $\Theta(\bar 0, x)$, for $x\in \barr G/\barr {T^+}$, is as in \cref{defi:blocks-of-Theta}. 
We, then, define the (super-)anti-automorphism $\theta\from S\even \to S\even$ by 
\[\label{eq:theta-with-matrix-4}
    \forall X\in M_{n+1}(\FF), \quad \theta(X) \coloneqq \Theta_\bz\inv\, X\transp\, \Theta_\bz.
\]

We extend the $\barr G$-grading on $S\even$ to $S = S\even \oplus u S\even$ by declaring $\deg u = \barr h$, \ie, we define $S\odd_{\barr{h}\bar g} = u 
S\even_{\bar g}$, for all $\barr g \in \barr G$. 
In terms of the identification $S = Q(1) \tensor S\even$, this corresponds to the usual tensor product of graded superalgebras. 

\begin{remark}\label{rmk:change-M(1-1)-of-place}
    Note that $S = Q(1) \tensor S\even = Q(1) \tensor M_k(\FF) \tensor \barr \D\even \iso M_k(\FF) \tensor Q(1) \tensor \barr\D\even$. 
    Hence, this $\barr G$-grading is isomorphic to $\Gamma_Q(\barr {T^+}, \barr {\beta^+}, \bar h, \kappa)$ (see \cref{def:Gamma-T-beta-kappa-Q}), by choosing the standard realization of type Q (\cref{def:standard-realization-Q}) to be $\barr \D \coloneqq Q(1)\tensor \barr\D\even = \barr\D\even \oplus u \barr\D\even$.
\end{remark}

We also extend the (super-)anti-automorphism $\theta$ to $S$, by declaring $\theta (u) = \bi u$. 
In terms of the identification $S = Q(1)\tensor S\even$, this corresponds to  
\[\label{eq:theta-with-matrix-6}
    \forall X\in Q(n+1), \quad \theta(X) \coloneqq \Theta\inv\, X\sTq\, \Theta,
\]
where 
\[\label{eq:Theta-for-Ann}
    \Theta \coloneqq \left(\begin{array}{c|c}
            \Theta_\bz & 0\\
            \hline
            0 & \Theta_\bz
        \end{array}\right).
\]
Note that, differently from \cref{eq:theta-with-matrix-3}, we are using the queer supertranspose (\cref{def:queer-stp}) here.

Finally, we define a $G$-grading on $S^{(1)}$ by 
\[\label{eq:final-G-grd-on-Ann}
    \forall g\in G, \quad S^{(1)}_g \coloneqq \{ s \in S^{(1)}_{\barr g} \mid \theta(s) = - \chi(g) s\},
\]
and consider the $G$-grading on $L$ by reducing it modulo the center. 

In the next definition, we summarize what have been done for future reference:

\begin{defi}\label{defi:type-II-Ann}
    Let $n \in \ZZ_{> 0}$ and let $S$ denote the associative superalgebra $Q(n+1)$. 
    Let $T^+ \subseteq G$ be a finite $2$-elementary subgroup, let $\beta^+\from {T^+\times T^+} \to \FF^\times$ be an alternating bicharacter with $\rad \beta^+ = \langle f \rangle$, for some element $e \neq f\in T$ and let $h \in G$ be an element such that $h^2=f$. 
    Set $\barr {T^+} \coloneqq T/\langle f \rangle$, and let $\barr {\beta^+}$ be the (nondegenerate) alternating bicharacter on $\barr {T^+}$ induced by $\beta^+$. 
    Consider the chosen standard realization $\barr \D\even$ of a matrix algebra with division grading associated to $(\barr {T^+}, \barr{\beta^+})$, fix a subgroup $K \subseteq T^+$ such that $T^+ = K \times \langle f \rangle$, and define $\eta^+\from T \to \pmone$ by \cref{eq:fix-eta-plus-undouble-Ann}. 
    Then, let $g_0 \in G$ be any element and let $\kappa\from G/{T^+} \to \ZZ_{\geq 0}$ be a $g_0$-admissible map (\cref{defi:odd-D-kappa-g_0-admissible}) such that $n+1 = |\kappa| \sqrt{|T^+|/2}$. 
    Choose:
    \begin{enumerate}[(i)]
        \item a set-theoretic section $\xi\from G/T^+ \to G$ for the natural homomorphism $G \to G/T^+$;
        \label{item:choice-xi-Ann}
        %
        \item a total order $\leq$ on $G/T^+$ such that there are no elements between $x$ and $\bar g_0\inv x\inv$, for all $x\in G/T^+$; 
        \label{item:choice-leq-Ann}
    \end{enumerate}
    and construct a tuple $\bar\gamma$ realizing $\kappa$ according to $\pi \circ \xi$ and $\leq$ (\cref{defi:tuple-governed}), where $\pi\from G\to \barr G$ is the natural homomorphism. 
    Consider the $\barr G$-grading $\Gamma_M(\barr {T^+}, \barr {\beta^+}, \kappa)$ on $S\even \iso M_{n+1}(\FF)$ constructed using the choices of $\barr \D\even$ and $\barr \gamma$ above (see \cref{def:Gamma-T-beta-kappa}), and extend it to $S$ by declaring $\deg u \coloneqq \barr h$. 
    Define $\Theta_\bz \in M_{n+1}(\FF)$ by \cref{eq:Theta_bz-for-Ann}, $\Theta \in S$ by \cref{eq:Theta-for-Ann} and ${\theta\from S \to S}$ by
    \cref{eq:theta-with-matrix-6}
    Finally, we define $\Gamma_Q^{\mathrm{(II)}}(T^+, \beta^+, h, \kappa, g_0)$ to be the $G$-grading on $L = S^{(1)}/Z(S^{(1)})$ induced from the $G$-grading $S^{(1)}$ given by \cref{eq:final-G-grd-on-Ann}.
\end{defi}

% -------------------------------

\begin{remark}
    We note that in \cite[Theorem 5.1]{paper-Qn} it is described how to extend a $G$-grading $\Gamma$ on $S\even$ to $S$, without using a full description of $\Gamma$. 
    The construction above corresponds to the case where $\Gamma$ is of Type II. 
\end{remark}

% \begin{lemma}
%     The $\barr G$-grading on $S$ constructed above is isomorphic to $\Gamma_Q(\barr {T^+}, \barr {\beta^+}, \bar h, \kappa)$ (see \cref{def:Gamma-T-beta-kappa-Q}). 
% \end{lemma}

% \begin{proof}
%     The $\barr G$-grading on $S$ is given by $S = Q(1) \tensor S\even = Q(1) \tensor $
    
%     Since $u$ is central. 
    
%     by using the choices (i) $\barr \D \coloneqq \barr \D\even \oplus u\barr\D\even$ and (ii) $\barr \gamma$ as above
% \end{proof}

% --------------------

% A BIT OF OVER-EXPLANATION:

% The following is an easy result that will be also used in \cref{ssec:grds-on-Ann}:

% \begin{lemma}
%     Let $\D$ be an odd graded-division superalgebra, $\U$ be graded right $\D$-supermodule and $B\from \U \times \U \to \D$ be a homogeneous nondegenerate sesquilinear form with $\deg B \in G$. 
%     Set $R = \End_\D(\U)$ and, as in \cref{rmk:R-even-identificatios}, identify $R\even$ with $\End_{\D\even} (\U\even)$. 
%     If $\vphi\from R \to R$ is the superadjunction with respect to $B$ (\cref{def:superadjunction}), then $\vphi\restriction_{R\even}\from R\even \to R\even$ is the superadjunction with respect to $B\restriction_{\U\even \times \U\even} \from \U\even \times \U\even \to \D\even$. \qed
% \end{lemma}

\begin{prop}\label{prop:Ann-Type-II-correspondence}
    Consider $(R, \vphi) \coloneqq Q^{\mathrm{ex}}(T^+, \beta^+, h, \kappa, g_0)$, as defined before \cref{cor:QxQ-reduced-to-MxM}. 
    Then the graded Lie superalgebra $\Skew (R,\vphi)^{(1)}/Z(\Skew (R,\vphi)^{(1)})$ is isomorphic to the Lie superalgebra $Q(n)$ endowed with $\Gamma_Q^{\mathrm{(II)}}(T^+, \beta^+, h, \kappa, g_0)$. 
\end{prop}

\begin{proof}
    % Recall what Q^ex means, including what $T$ and $t_p$ are, and how to see \kappa with the proper domain. 
    First recall that, by definition, $Q^{\mathrm{ex}}(T^+, \beta^+, h, \kappa, g_0) = Q^{\mathrm{ex}} (T, \beta, t_p, \kappa, g_0)$ (see \cref{def:model-grd-MxM-odd-or-QxQ}), where $t_p \coloneqq (h, \bar 1)$, $T \coloneqq T^+ \cup t_p T^+$, $\beta\from T\times T \to \FF^\times$ is the unique alternating bicharacter extending $\beta^+$ such that $\rad \beta = \langle t_p \rangle$, and $\kappa$ is seen as defined on $G^\#/T \iso G/ T^+$. 
    
    % Say what we are going to do
    We will now show how the choices in \cref{defi:type-II-Ann} correspond to the choices in \cref{def:std-realization-MxM-QxQ}(c) and \cref{def:model-grd-MxM-odd-or-QxQ}. 
    % (and, hence, also in \cref{def:std-realization-MxM-QxQ}(c)).
    
    % The choices of K and \barr\D\even given us the correct \eta^+, and fix \D as \M\tensor\C
    By lemma \cref{lemma:barr-D-to-mc-M}, with $(T^+, \beta^+)$ playing the role of $(T, \beta)$, the choices of $\barr \D\even$ and $K$ give us the same information as the choice of $\mc M$ in \cref{def:std-realization-MxM-QxQ}\eqref{item:choose-mc-M}, and the map associated to the transposition on $\mc M$ is $\mu^+\restriction_K$. 
    Let $(\D, \vphi_0)$ denote $(\mc M \tensor \mc C, \vphi_{\mc C} \tensor \vphi_{\mc M})$ as in \cref{def:std-realization-MxM-QxQ}(c), and let
    $\eta\from T \to \pmone$ be the map associated to $\vphi_0$.  
    By definition, $\eta\restriction_{K} = \mu^+\restriction_K$, and it is clear that $\eta\restriction_{\langle f \rangle} = \chi\inv\restriction_{\langle f \rangle}$. 
    Since $\chi\inv\restriction_K$ and $\mu\restriction_{\langle f \rangle}$ are trivial, it follows that $\eta\restriction_{T^+} = \eta^+$. 
    In particular, the condition that $\kappa$ is $g_0$-admissible is the same in both \cref{def:model-grd-MxM-odd-or-QxQ,defi:type-II-Ann}.
    
    % Constructs $(\U, B)$
    In \cref{def:model-grd-MxM-odd-or-QxQ}, we have to choose a graded right $\D$-supermodule $\U$ and a $\vphi_0$-sesquilinear form $B\from \U \times \U \to \D$ such that $(\U, B)$ has inertia determined by $\kappa$. 
    We will first construct a graded right $\D\even$-supermodule $\U\even$ and a $(\vphi_0\restriction_{\D\even})$-sesquilinear form $B\even\from \U \times \U \to \D$ such that $(\U\even, B\even)$ has inertia determined by $(\kappa_\bz, \kappa_\bo)$, where we take $\kappa_\bz \coloneqq \kappa$  (seen as a map defined on $G/T^+$) and $\kappa_\bo$ to be the zero map. 
    % (we note that, by \cref{rmk:R-even-identificatios}, the superadjunction with respect to $B\even$ would give us a). 
    To that end, it suffices to follow the same construction as in \cref{ssec:grds-osp}. 
    In other words, define $\D\even \coloneqq (\D\even)^{[g_1]}\oplus \cdots \oplus (\D\even)^{[g_k]}$, where $(g_1, \ldots, g_k)$ is the $k$-tuple of elements in $G$ realizing $\kappa$ according to $\xi$ and $\leq$, and define $B$ by $B(u_i, u_j) \coloneqq \Phi_{ij}$, where $\B \coloneqq \{ u_1, \ldots, u_k \}$ is the canonical graded $\D\even$-basis of $\U\even$ and $\Phi \in M_k(\D\even)$ is the matrix defined by \cref{eq:puting-the-blocks-of-Phi-together}. 
    We then define $\U\odd \coloneqq \U\even \tensor_{\D\even} \D$. 
    Note that $\B$ is an even graded $\D$-basis of $\U$, and let $B\from \U \times \U \to \D$ be the unique $\vphi_0$-sesquilinear extension of $B\even$, which is also represented by $\Phi$, seen as a matrix in $M_k(\D)$.  
    
    % Undoubling what we just constructed
    We will now follow \cref{ssec:undoubling} to undouble $Q^{\mathrm{ex}}(T^+, \beta^+, h, \kappa, g_0)$, constructed with the choices above. 
    % $\D$ is what it should be 
    Let $\epsilon \coloneqq \frac{1+\zeta}{2}\in \D$ be a central idempotent, where $\deg \zeta = f$, and consider the $\barr G$-graded-division superalgebra $\D\epsilon$. 
    By \cref{prop:lemma-for-undoubling-and-fine-gradings}, $\D\epsilon$ has parameters $(\barr T, \barr \beta, \barr p)$. 
    Also, $\rad \barr\beta = \langle \barr t_p \rangle$, so $\D\epsilon$ is isomorphic to a standard realization of type Q associated to $(\barr {T^+}, \barr{\beta^+}, \barr h)$ (see discussion preceding \cref{def:standard-realization-Q}), \ie, it is isomorphic to $\barr \D \coloneqq Q(1) \tensor \barr \D\even$ as in \cref{rmk:change-M(1-1)-of-place}. 
    From now on, we will identify $\D\epsilon$ with $\barr \D$. 
    
    We need to extend $\chi$ to $G^\#$. 
    Since $\chi(f) = -1$, we can do it in a way such that $\chi(t_p) = \bi$. 
    We will proceed to prove that the map $\barr\mu$ defined in \cref{eq:defi-mu-undoubled} coincides with the map $\barr \mu$ we defined in this subsection, \ie, the map associated to the queer supertranspose on $\barr \D$. 
    We already showed that $\eta\restriction_{T^+} = \eta^+$, hence, by \cref{eq:fix-eta-plus-undouble-Ann}, we have that $\barr\mu\restriction_{T^+}$, indeed, is the transposition on $\barr \D\even$. 
    Also, from \cref{def:std-realization-MxM-QxQ}(c), $\eta(t_p) = 1$, so $\barr\mu (\barr t_p) = \eta(t_p) \chi(t_p) = \bi $, as desired. 
    
    % @1 
    Let $\Lambda \in M_k(\D)$ be as in \cref{ssec:undoubling}. 
    % We will now describe the matrix $\Lambda\Phi\epsilon$. 
    As in the proof of \cref{prop:m-not-n-Type-II-correspondence}, we consider a different graded $\D$-basis $\tilde B = \{ \tilde u_1, \ldots, \tilde u_k \}$ of $\U$, where $\tilde u_i$ is defined as in \cref{eq:tilde-u_i-from-u_i}, and let $\tilde \Phi \in M_k(\D)$ be the matrix representing $B$ with respect to $\tilde \B$. 
    Clearly, all the entries of $\tilde\Phi$ are in $\barr \D\even$. 
    We will denote $\tilde\Phi$ by $\tilde\Phi\even$ when seen as a matrix in $M_k(\D\even)$. 
    In the same way $\tilde\Phi \epsilon$ can be seen as a matrix in $M_k(\barr \D)$, $\tilde\Phi\even \epsilon$ can be seen as a matrix in $M_k(\barr \D\even)$. 
    We identify $M_k(\barr \D\even) \iso M_k(\FF) \tensor \barr \D\even$ with $S\even = M_{n+1}(\FF)$ via Kronecker product. 
    By construction, this gives us the restriction of $\Gamma_Q^{\mathrm{(II)}}(T^+, \beta^+, h, \kappa, g_0)$ on $S\even$, and it is clear that it sends $\Lambda \tilde\Phi\even$ to $\Theta\even$. 
    Also, following the isomorphisms
    \[
        M_k(\barr \D) \iso M_k(\FF) \tensor \barr \D \iso M_k(\FF) \tensor Q(1) \tensor \barr \D\even \iso Q(1) \tensor M_k(\FF) \tensor \barr \D\even \iso Q(1) \tensor S\even.
    \]
    we get $\Gamma_Q^{\mathrm{(II)}}(T^+, \beta^+, h, \kappa, g_0)$ on $S = Q(1) \tensor S\even$, and $\Lambda\tilde\Phi\epsilon$ is sent to $I \tensor \Theta\even = \Theta$. 
    Therefore, all the data in the description of the undoubled model of $\Skew(R, \vphi)$ coincide with with the data in \cref{defi:type-II-Ann}, concluding the proof.
\end{proof}

% --------------------
