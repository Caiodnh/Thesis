
\section{Gradings on universal algebras}\label{sec:Omega-algebras}
% \texorpdfstring{$\Omega$}{Omega}-algebras

In the \hyperref[chap:intro]{Introduction}, we defined different graded structures: vector spaces, algebras, superspaces, superalgebras and superalgebras endowed with a super-anti-automorphism. 
The language of \emph{universal algebra} allows us to consider all these in a uniform fashion. 
It is particularly convenient to formulate results like \cref{thm:transfer-of-gradings} , which allows us to transfer gradings between different structures: in \cref{chap:Lie}, we will use it to transfer our classification of $G$-gradings on superinvolution-simple superalgebras (achieved in \cref{chap:grds-sinv-simple}) to classical Lie superalgebras. 
We will also use this language in \cref{ssec:universal_group} to give unified definitions for fine gradings and universal group (\cref{defi:fine-grading,defi:universal-group}), which is usually done \emph{ad hoc} for different cases. 

We note that universal algebras differs are usually defined in the category of sets (see, \eg, \cite{Cohn_universal}), but we will work in the category of vector spaces over $\FF$. 
This approach was used in \cite{Razmyslov} and has recently been applied to gradings (and graded identities) in \cite{MR3886336}. 

\begin{defi}
    An \emph{$n$-ary operation} on a vector space $V$ is 
    % a multilinear map $V^n \to V$ or, equivalently, 
    a linear map $V^{\tensor n} \to V$, where $V^{\tensor n} \coloneqq \underbrace{V\otimes\cdots\otimes V}_{n \text{ times}}$. 
\end{defi}

In other words, an $n$-ary operation is a multilinear map $V^n \to V$. 
In the case $n = 0$, we will follow the convention that $V^{\tensor 0} \coloneqq \FF$. 
In particular, a $0$-ary operation is determined by its value on $1 \in \FF$ and, hence, $0$-ary operations are constants in $V$. 

\begin{defi}\label{def:universal-algebra}
	A \emph{signature} $\Omega$ is a set with a partition $\Omega = \bigcup_{n \geq 0} \Omega_n$.
	%
	An \emph{$\Omega$-algebra} or a \emph{universal algebra with signature $\Omega$} is a vector space $A$ endowed with $n$-ary operations $\omega^A$ for each $\omega \in \Omega_n$, for all $n \geq 0$. 
	%
	A \emph{homomorphism between $\Omega$-algebras} $A$ and $B$ is a linear map such that for every $\omega \in \Omega_n$ we have
	\[
		\forall a_1, \ldots, a_n \in A, \quad 
		\psi( \omega^A (a_1 \tensor \cdots \tensor a_n) ) = \omega^B ( \psi(a_1) \tensor \cdots \tensor \psi(a_n) ). 
	\]
\end{defi}

\begin{notation}
    When dealing with a fixed $\Omega$-algebra $A$, we will usually drop the superscript $A$ in the operations $\omega$, \ie, we will identify the signature $\Omega$ with its corresponding set of operations on $A$. 
\end{notation}

\begin{ex}\label{ex:omega-vec-space}
	A vector space is an $\Omega$-algebra with $\Omega = \emptyset$.
\end{ex}

\newcommand{\multiplication}{\cdot}

\begin{ex}\label{ex:omega-algebra}
	An algebra $A$ in the usual sense, with a bilinear product $\multiplication : A\tensor A \to A$, is an $\Omega$-algebra with $\Omega = \Omega_2 = \{ \multiplication \}$. 
\end{ex}

\begin{ex}\label{ex:Omega-unity-inv}
    A unital algebra $A$ with product $\multiplication$ and unity element $1_A \in A$ is an $\Omega$-algebra with $\Omega = \Omega_0 \cup \Omega_2$ where $\Omega_0 = \{ \omega_0 \}$, with $\omega_0\from \FF \to A$ determined by $\omega_0 (1) \coloneqq 1_A$, and $\Omega_2 = \{ \multiplication \}$. 
    An algebra $A$ with product $\multiplication$ and involution $\vphi\from A \to A$ is an $\Omega$-algebra with $\Omega = \Omega_1 \cup \Omega_2$, where $\Omega_1 = \{ \vphi \}$ and $\Omega_2 = \{ \multiplication \}$. 
    A unital algebra $A$ with involution $\vphi$ is an $\Omega$-algebra with $\Omega = \Omega_0 \cup \Omega_1 \cup \Omega_2$, where $\Omega_0 = \{ \omega_0 \}$, $\Omega_1 = \{ \vphi \}$ and $\Omega_2 = \{ \multiplication \}$. 
\end{ex}


\begin{ex}\label{ex:omega-superspace}
	A superspace $V = V\even \oplus V\odd$ can be seen as an $\Omega$-algebra by taking $\Omega = \Omega_1 = \{ \pi_\bz, \pi_\bo \}$, where $\pi_\bz, \pi_\bo \from V \to V$ are the projections onto the components $V\even$ and $V\odd$, respectively. 
	A universal algebra $V$ with this signature is a superspace \IFF for all $x \in V$, we have:
	%
	\begin{enumerate}[(i)]
		\item $\pi_{\bar 0}(x) + \pi_{\bar 1}(x) = x$; \label{item:sum-projections}
% 		\item $(\pi_{\bar 0} \circ \pi_{\bar 1}) (v)
% 		= (\pi_{\bar 1} \circ \pi_{\bar 0}) (v)
% 		= 0$.
		\item $\pi_{\bar 0} ( \pi_{\bar 1} (x) )
		= \pi_{\bar 1} ( \pi_{\bar 0} (x) )
		= 0$. \label{item:orthogonal-projections}
	\end{enumerate}
\end{ex}


\begin{ex}\label{ex:omega-alg-SA}
	A superalgebra $A = A\even \oplus A\odd$ is an $\Omega$-algebra with $\Omega = \Omega_1 \cup \Omega_2$, where $\Omega_2 = \{ \multiplication \}$, $\Omega_1 = \{ \pi_{\bar 0}, \pi_{\bar 1} \}$.
	An algebra $A$ with this signature is a superalgebra if, and only if, we have identities \eqref{item:sum-projections} and \eqref{item:orthogonal-projections} as above and, for all $x,y\in A$ and $i, j\in \ZZ_2$,
	\begin{enumerate}[(i)]
        \setcounter{enumi}{2}
		\item $\pi_i^{} (x) \multiplication \pi_j^{} (y) = \pi_{i+j}^{}( \pi_i^{} (x) \multiplication \pi_j^{} (y) )$.
	\end{enumerate}
\end{ex}

\begin{ex}\label{ex:omega-alg-SA-sinv}
    Similarly to \cref{ex:Omega-unity-inv}, a superalgebra $A = A\even \oplus A\odd$ with superinvolution $\vphi\from A \to A$ is an  $\Omega$-algebra with $\Omega = \Omega_1 \cup \Omega_2$ where $\Omega_1 = \{\pi_{\bar 0}, \pi_{\bar 1}, \vphi \}$ and  $\Omega_2 = \{ \multiplication \}$.
\end{ex}

\begin{remark}
    The signatures in \cref{ex:omega-superspace,ex:omega-alg-SA} can be generalized for $G$-graded spaces/algebras, and, if $G$ is finite, the axioms can be stated as identities, which allows one to define these objects as varieties of algebras (see \cite[Section 2]{MR3886336}), but this is not the approach we are going to follow. 
\end{remark}

\begin{defi}\label{def:grds-on-Omega-algebras}
	A \emph{$G$-grading on an $\Omega$-algebra $A$} is a $G$-grading on its underlying vector space such that, for all $\omega \in \Omega$, the operation $\omega^A\from A^{\tensor n} \to A$ is degree preserving if we consider $G$-gradings on tensor products $A^{\tensor n}$ induced by the $G$-grading on $A$ (see \cref{defi:tensorProduct}). 
	If $\Gamma$ is fixed, we say that $A$ is a \emph{$G$-graded $\Omega$-algebra}. 
	A \emph{homomorphism of $G$-graded $\Omega$-algebras} is a degree-preserving homomorphism of the underlying $\Omega$-algebras. 
	Given two $G$-gradings $\Gamma$ and $\Delta$ on a fixed $\Omega$-algebra $A$, we say that $\Gamma$ is \emph{isomorphic} to $\Delta$ if $(A, \Gamma) \iso (A, \Delta)$.  
\end{defi}

\phantomsection\label{lemma:1-has-deg-e}

It is straightforward to check that, for each of the \cref{ex:omega-vec-space,ex:omega-algebra,ex:Omega-unity-inv,ex:omega-superspace,ex:omega-alg-SA,ex:omega-alg-SA-sinv}, the notions of homomorphism and $G$-grading as $\Omega$-algebras coincide with the notions of homomorphism and $G$-grading we had before. 
Note that \cref{def:grds-on-Omega-algebras} entails that, in a unital algebra, the unity element is homogeneous of degree $e$, but this is automatic according to \cref{def:grading}: 
if we write $1 = \sum_{g\in G} a_g$, with $a_g \in A_g$, then, for any $h\in G$, $a_h = a_h 1 = \sum_{g\in G} a_h a_g \in A_h$ and, since $a_h a_g \in A_{hg}$, the only nonzero element in this sum $a_h a_e$. 
It follows that $a_h = a_h a_e$ and, hence, $1 = \sum_{g\in G} a_g = \sum_{g\in G} a_g a_e = 1 a_e = a_e \in A_e$. 

% , note that, by it is easy to verify that $1_A$ automatically belongs to the homogeneous component $A_e$.